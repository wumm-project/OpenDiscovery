\documentclass[11pt,a4paper]{article}
\usepackage{a4wide,amsmath}
\usepackage[utf8]{inputenc}
\usepackage[ngerman]{babel}
\usepackage{tikz}
\usetikzlibrary{positioning,shapes.geometric}

\begin{document}
\begin{tikzpicture}[thick]
\tikzset{ op/.style={ rectangle, fill=gray!10, draw=black } }

\node[draw] at (3,6)
(A1) {\parbox{3cm}{\small\centering Gesetz der Erhöhung der Idealität}};

\node[draw] at (0,3)
(A2) {\parbox{3cm}{\small\centering Gesetz der Erhöhung der Vollständigkeit des Systems}}; 

\node[draw] at (3,3)
(A3) {\parbox{3cm}{\small\centering Gesetz der Erhöhung der Koordination}};

\node[draw] at (6,3)
(A4) {\parbox{3cm}{\small\centering  Das Gesetz der Erhöhung des Energieleitvermögens}};

\node[draw] at (9,3)
(A5) {\parbox{3cm}{\small\centering  Das Gesetz der ungleichen Entwicklung von Systemteilen}};

\node[draw] at (12,3)
(A6) {\parbox{3cm}{\small\centering  Die Entwicklung technischer Systeme längs einer S-Kurve}};

\node[draw] at (0,0)
(A7) {\parbox{3cm}{\small\centering  Das Gesetz der Verdrängung des Menschen}};

\node[draw] at (3,0)
(A8) {\parbox{3cm}{\small\centering  Das Gesetz der zunehmenden Steuerbarkeit}};

\node[draw] at (6,0)
(A9) {\parbox{3cm}{\small\centering  Das Gesetz der Dynamisierung}};

\node[draw] at (9,0)
(A10) {\parbox{3cm}{\small\centering  Das Gesetz des Übergangs zum Obersystem}};

\node[draw] at (12,0)
(A11) {\parbox{3cm}{\small\centering  Das Gesetz des Zusammenfallens}};

\draw[->] (A1) to (A2);
\path[->] (A1) to (A3);
\end{tikzpicture}

\end{document}
