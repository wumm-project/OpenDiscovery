\documentclass[11pt,a4paper]{article}
\usepackage{od,amsmath}
\usepackage[utf8]{inputenc}
\usepackage[ngerman]{babel}
\usepackage{tikz}
\usetikzlibrary{positioning,shapes.geometric}

\begin{document}
\begin{tikzpicture}[thick]
\tikzset{ op/.style={ rectangle, fill=gray!10, draw=black } }
\node[op] at (3,6)
  (A1) {\parbox{5cm}{\small\centering Gesetz der Erhöhung der Idealität}};
\node[op] at (0,3)
  (A2) {\parbox{5cm}{\small\centering Gesetz der Erhöhung der Vollständigkeit
    des Systems}}; 
\node[op] at (5,2)
(A3) {\parbox{5cm}{\small\centering Gesetz der Erhöhung der Koordination}};
\path[->] (A1) edge (A2);
\path[->] (A1) edge (A3);
\end{tikzpicture}

\end{document}
