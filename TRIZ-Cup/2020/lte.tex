\documentclass[11pt,a4paper]{article}
\usepackage{a4wide,amsmath}
\usepackage[utf8]{inputenc}
\usepackage[ngerman]{babel}
\usepackage{tikz}
%\usetikzlibrary{positioning}
%\usetikzlibrary{shapes.geometric}
\usetikzlibrary{shapes.misc}

\newcommand{\law}[2]{\parbox{#1cm}{\small\centering #2}}

\begin{document}
\tikz{
  
\node[draw] at (6,10) [rounded rectangle]
(A1) {\law{2}{Gesetz der Erhöhung der Idealität}};

\node[draw] at (1,4) [rounded rectangle]
(A2) {\law{2}{Gesetz der Erhöhung der Vollständigkeit des Systems}}; 

\node[draw] at (0,7) [rounded rectangle]
(A3) {\law{2}{Gesetz der Erhöhung der Koordination}};

\node[draw] at (4,7) [rounded rectangle]
(A4) {\law{2}{Das Gesetz der Erhöhung des Energieleitvermögens}};

\node[draw] at (5,4) [rounded rectangle]
(A5) {\law{2}{Das Gesetz der ungleichen Entwicklung von Systemteilen}};

\node[draw] at (5,1) [rounded rectangle]
(A6) {\law{2}{Die Entwicklung technischer Systeme längs einer S-Kurve}};

\node[draw] at (2,1) [rounded rectangle]
(A7) {\law{2}{Das Gesetz der Verdrängung des Menschen}};

\node[draw] at (8,7) [rounded rectangle]
(A8) {\law{2}{Das Gesetz der zunehmenden Steuerbarkeit}};

\node[draw] at (12,7) [rounded rectangle]
(A9) {\law{2}{Das Gesetz der Dynamisierung}};

\node[draw] at (9,4) [rounded rectangle]
(A10) {\law{2}{Das Gesetz des Übergangs zum Obersystem}};

\node[draw] at (9,1) [rounded rectangle]
(A11) {\law{2}{Das Gesetz des Zusammenfallens}};

\draw[->] (A1) |- (A2);
\draw[->] (A1) -- (A3);
\draw[->] (A1) -| (A4);
\draw[->] (A1) to (A5);
\draw[->] (A1) to (A8);
\draw[->] (A1) to (A9);
\draw[->] (A1) to (A10);
\draw[->] (A2) to (A7);
\draw[->] (A5) to (A6);
\draw[->] (A10) to (A11);

}

\end{document}
