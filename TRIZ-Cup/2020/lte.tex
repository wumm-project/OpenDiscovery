\documentclass[11pt,a4paper]{article}
\usepackage{a4wide,amsmath}
\usepackage[utf8]{inputenc}
\usepackage[ngerman]{babel}
\usepackage{tikz}
\usetikzlibrary{shapes.misc}
\usetikzlibrary{arrows.meta}

\newcommand{\law}[2]{\parbox{#1cm}{\small\centering #2}}

\begin{document}
\tikz[>={Triangle[length=3pt 9, width=3pt 3]}] {
  
\node[draw] at (5,10) [rounded rectangle]
(A1) {\law{4}{Gesetz der Erhöhung der Idealität}};

\node[draw] at (0,3.5) [rounded rectangle]
(A2) {\law{5}{Gesetz der Erhöhung der Vollständigkeit des Systems}}; 

\node[draw] at (0,8.5) [rounded rectangle]
(A3) {\law{4}{Gesetz der Erhöhung der Koordination}};

\node[draw] at (2,7) [rounded rectangle]
(A4) {\law{4}{Gesetz der Erhöhung des Energieleitvermögens}};

\node[draw] at (5,4.8) [rounded rectangle]
(A5) {\law{5}{Gesetz der ungleichen Entwicklung von Systemteilen}};

\node[draw] at (5,2.2) [rounded rectangle]
(A6) {\law{5}{Entwicklung technischer Systeme längs einer S-Kurve}};

\node[draw] at (0,1) [rounded rectangle]
(A7) {\law{4}{Gesetz der Verdrängung des Menschen}};

\node[draw] at (8,7) [rounded rectangle]
(A8) {\law{4}{Gesetz der zunehmenden Steuerbarkeit}};

\node[draw] at (10,8.5) [rounded rectangle]
(A9) {\law{4}{Gesetz der Dynamisierung}};

\node[draw] at (10,3.5) [rounded rectangle]
(A10) {\law{4}{Gesetz des Übergangs zum Obersystem}};

\node[draw] at (10,1) [rounded rectangle]
(A11) {\law{4}{Gesetz des Zusammenfallens}};

\draw[->] (A1) -| (A3);
\draw[->] (A1) -- (A4);
\draw[->] (A1) -- (A8);
\draw[->] (A1) -| (A9);
\draw[->] (A1) -- (5,6) -| (A2);
\draw[->] (5,6) -- (A5);
\draw[->] (5,6) -| (A10);
\draw[->] (A2) -- (A7);
\draw[->] (A5) -- (A6);
\draw[->] (A10) -- (A11);

}

\end{document}
