\documentclass[11pt,a4paper]{article}
\usepackage{od,array}
\usepackage[utf8]{inputenc}
\usepackage[main=german,russian]{babel}

\pagestyle{headings}
\title{Kurze Analyse des Problems der „Black Box einer Zivilisation“}
\author{M.S. Rubin}
\date{2020}

\begin{document}
\maketitle

Die Aufgabe der „Black Box einer Zivilisation“ wurde als Generator neuer
Forschungsthemen vorgeschlagen, als Quelle für würdige Ziele eines Forschers.

G.S. Altschuller und I.M. Vertkin \cite{Altshuller1991} unterscheiden drei
Themenebenen:
\begin{itemize}[noitemsep]
\item eng-technische, eng-wissenschaftliche, eng-künstlerische Themen, zum
  Beispiel die Erfindung einer Rakete;
\item allgemeine technische, allgemeine wissenschaftliche, allgemeine
  künstlerische Themen, zum Beispiel die Entwicklung der Raumfahrt; 
\item Themen, die die gesamte Zivilisation betreffen (sozio-technisch,
  sozial-wissenschaftlich, sozial-künstlerisch usw.), zum Beispiel die
  Gestaltung einer kesmischen Zivilisation oder einer Zivilisation mit
  Fernsehen.
\end{itemize}

Aus Sicht der Forschung ist es am effektivsten, mit Themen der dritten Stufe
zu arbeiten, auf der Ebene allgemein-menschlicher Probleme. Hier besteht der
größte Spielraum für die Entwicklung von Themen und die geringste
Wahrscheinlichkeit, bei kleinen Themen „hängen zu bleiben“ oder sich in einer
akuten Konfliktsituation mit anderen Forschungen zu befinden. Als Beispiel
möchte ich wenigstens ein solches Problem auf allgemein-menschlicher Ebene
präsentieren. Es geht um die Aufgabe der Schaffung einer „Black Box“ für den
Fall einer weltweiten Katastrophe. (\cite[S. 166-168]{Altshuller1991})

Nur sehr wenige TRIZ-Spezialisten haben an dem vorgeschlagenen Problem
gearbeitet, und es gibt wenige Veröffentlichungen zu diesem Thema. Die Aufgabe
ist schwierig und wir haben beschlossen, einige Ansätze dazu anzubieten. Dies
ist nur unsere Ansicht, und wir bestehen nicht nicht darauf, dass dies die
einzige Annäherung an das Problem der „Black Box der Zivilisation“ ist.

Wir möchten gleich offensichtlich schwache Lösungen aussortieren. Zum
Beispiel: Ein globales Systeme zum Sammeln von Informationen über jedem
Bewohner der Erde und diese auf einigen Satellit der Erde oder Venus zu
speichern. Trotz der offensichtlichen Unerfüllbarkeit, der Nichtidealität
solcher Ideen werden sie immer wieder vorgetragen. Das Forschungsthema wird
dann interessant und stark, wenn es darum geht, Widersprüche zu lösen, die im
Konzept der „Black Box“ der Zivilisation enthalten sind (zum Beispiel sollten
sehr viele Informationen gesammelt werden, um alle Aspekte der Zivilisation zu
reflektieren, und zugleich sollte wenig Information gesammelt werden, weil es
einfacher ist, diese zu speichern und für die Nachwelt zu bewahren; oder der
Speicherort sollte in der Nähe zu sein, um das Einspielen von Informationen zu
erleichtern, und er sollte zugleich weit entfernt sein, damit die
Informationen im Katastrophenfall sicherer erhalten bleibt) und einem idealen
Endresultat maximal nahe kommen (zum Beispiel, indem die katastrophale Kraft,
welche die Zivilisation zerstört, \emph{von sich aus} die „Black Box“ für die
Nachwelt aufbewahrt).

Während der TRIZ-Konferenz von Schülern und Studenten „IKARiade-2001“ wurde
eine morphologische Tabelle zur Analyse des Problems vorgeschlagen:

\begin{center}\small
  \begin{tabular}{|*{6}{>{\raggedright}p{1.9cm}|}c|}\hline
1. Welches Objekt & 1.1 Familie & 1.2. Haus & 1.3. Stadt & 1.4 Land &
1.5. Welt & \ldots \\\hline 
2. Wem, wohin & 2.1. anderen Menschen & 2.2. anderem Staat & 2.3. Nachfahren
in 50 (100, 1000) Jahren & 2.4. Außerirdische Zivilisationen & 2.5. Unbewohnte
Planeten& \ldots \\\hline
3. Ziel & 3.1. Information erhalten & 3.2. Fehler vermeiden &
3.3. Zivilisation erhalten & 3.4. Zivilisation neu auferstehen & 3.5. Grenzen
der Zivilisation ausdehnen& \ldots \\\hline
4. Ressourcen & 4.1. Ressourcen des zu erhaltenden Systems & 4.2. Ressourcen
der vernichtenden Kraft & 4.3. Ressourcen der heutigen Umwelt &
4.4. Ressourcen der Umwelt, in welche die „Black Box“ fällt & \ldots &
\\\hline  
5. Gründe der Katastrophe & 5.1 Natürliche & 5.2. Soziale Konflikte &
5.3. Natürliche Kataklysmen & 5.4 Zwischen- menschliche Konflikte &
5.5. Terror und Verbrechen & \ldots \\\hline
  \end{tabular}
\end{center}
Diese Tabelle kann verwendet werden, um eine Vielzahl von Teilproblemen zu
formulieren, zum Beispiel kann die Variante (1.3-2.3-3.2-4.1-5.1) wie folgt
formuliert werden: Wie kann die Informationen über das moderne Petrosawodsk
für die Generation der Menschen des 22. Jahrhunderts aufbewahrt werden, um zu
vermeiden, dass sich die begangenen Fehler durch die Kräfte der Stadt selbst
wiederholen, wenn diese sich infolge natürlicher Evolution verändert. Die
Ebene der formulierten Aufgaben und deren Schärfe kann nach eigenem Geschmack
verändert werden.

Es gibt eine Vielzahl von Bereichen für weitere Forschungsarbeiten zum Thema
„Black Box der Zivilisation“. Es ist beispielsweise möglich, die Achsen der
Morphologie zu präzisieren, deren Anzahl und Inhalt. Sie können eine Auswahl
spezifischer Aufgaben treffen, die unter dem allgemeinen Thema der „Black Box
der Zivilisation“ gesammelt wurden. Die Analyse jeder dieser Teilaufgaben kann
durchgeführt werden: Widersprüche formulieren, Ideales Endresultat
formulieren, Ressourcen analysieren.  Man kann eine Hierarchie von Aufgaben
für das Thema „Black Box der Zivilisation“ erstellen. Man kann Informationen
über die häufigsten Ursachen für den Untergang von Zivilisationen
sammeln. Sehr fruchtbar kann eine Analyse von Methoden zur Wiederherstellung
von Informationen sien, die in der Archäologie und bei Historikern verfügbar
sind.  Möglichkeiten für Forschungsarbeiten gibt es wirklich sehr viel.

Sehr effektiv kann die Sammlung verschiedener Kartotheken sein (siehe dazu den
Aufsatz \cite{RubinXX} von M. Rubin zum Thema „Black Box der Zivilisation“).

Sie können zum Beispiel mit einer Kartothek zum Thema „Black Boxes“ mit
verschiedenen Flugzeugtypen, Ozeanschiffen oder anderen Systemen beginnen. Es
ist spannend, die Art der aufgezeichneten Informationen herauszufinden. Die
Hauptprobleme und Widersprüche von „Black Boxes“ bestehen darin, eine Prognose
ihrer Entwicklung zu geben. Es kann sich zum Beispiel herausstellen, dass die
„Black Box der Zivilisation“ eine logische Fortsetzung der Entwicklung
konventioneller „Black Boxes“ ist. Nebenbei können Sie neue Ideen und Lösungen
in diesem Bereich anbieten.

Eine sehr wichtige Frage: Was genau soll man über die untergegangene
Zivilisation an Nachkommen oder Außerirdische weitergeben? Um dies zu
beantworten, ist möglicherweise eine andere Kartei erforderlich -- über die
\emph{Schlüsselartefakte} (Denkmäler), die dazu beitragen, die Geheimnisse der
untergegangenen Zivilisation zu entschlüsseln.

Ich gebe nur ein Beispiel. Für das Studium der untergegangenen
Maya-Zivilisation war es sehr wichtig, die Schrift dieses Volkes zu
entziffern.  Dies gelang hauptsächlich an Hand von zwei
Dokumenten: 
\begin{itemize}[noitemsep]
\item Dem Manuskript „Berichte über Angelegenheiten in Yucatan“ aus der Zeit
  der spanischen Eroberung der Maya-Indianer, wo insbesondere das abgekürzte
  Maya-Alphabet mit seiner \emph{Transkription ins gesprochene Spanische} sehr
  hilfreich war;
\item Aufzeichnungen über alte Mythen (Bücher Chilam Balam) aus dem
  16. Jahrhundert in \emph{lateinischen Buchstaben}, welche die Entwicklung
  der Maya-Sprache seit Beginn unserer Ära reflektieren.
\end{itemize}
Dazu muss man natürlich die moderne (lebendige) Maya-Sprachen hinzuzufügen,
die bis heute erhalten sind, sowie Informationen über die spanische Sprache
des 16. Jahrhunderts.

Es ist spannend zu bemerken, dass die „Black Box“ der Maya-Zivilisation in
Form eines Bi-Systems überliefert ist.  Das sind zuerst die Städte der alten
Maya mit ihren Tempeln, Inschriften, Bildern, Zeichnungen sowie der Schlüssel
zu ihrer Erklärung -- das Alphabet, das gerade erwähnt wurde.

Im schwierigsten Fall (alles Leben verschwindet von der Erde) wird eine
lebendige Sprache natürlich nicht erhalten. Dies geschah zum Beispiel mit den
sumerischen Texten: es gelang, diese vollständig zu entziffern, aber niemand
weiß, wie diese Sprache klang. Es ist erforderlich ein Objekt oder eine
Handlung (die in verschiedene Situationen eingebettet sind) mit dessen
schriftlicher oder tonaler (redender) Abbildung zusammenzuführen. So etwa wie
in TV-Comics zum Lernen einer Fremdsprache.

Das Beispiel der „Black Box“ der alten Maya-Zivilisation führt auf eine
interessante Idee. Das Manuskript „Bericht über die Angelegenheiten in
Yucatan“ ist tatsächlich das wichtigste, vollständigste und genaueste Dokument
über eine antike Zivilisation. Es wurde vom spanischen Bischof Diego de Landa
zusammengestellt. Gerade unter seiner Führung wurde fast alles schriftliche
und anderes Maya-Kulturerbe sorgfältigst vernichtet (für diese Aktivität
erhielt de Landa den Titel eines Bischofs). De Landas Manuskript ist eine Art
Bericht an die Behörden über seine „Arbeit“ zur Ausrottung der Häresie. Daraus
ergibt sich eine interessante Hypothese: der Grund, der zur Katastrophe führte
(zum Beispiel die von Diego de Landa vertretenen Spanier), erzeugt
\emph{selbst} eine solche „Black Box“ (ebena das berühmte Bischofsmanuskript).
Das zweite Schlüsseldokument zur Entschlüsselung der Maya-Texte -- die
lateinisch geschriebenen „Bücher von Chilam Balam“ -- haben eine ähnliche
Herkunft wie das Manuskript von de Landa. Die Spanier verboten den Maya, ihre
einheimische Schrift zu verwenden, und jene begannen, die alten Texte in der
erlaubten lateinischen Schriftform niederzuschreiben. Dieselbe Situation: Die
Spanier zwangen die Maya durch ihre Verbote \emph{selbst} dazu, Dokumente über
ihre Kultur zu erstellen, die für die Nachkommen verständlich sind. Ein
weiteres Beispiel, an das wir bereits erinnert haben: Der Ausbruch des Vesuvs
hat Pompeji zerstört, hat aber für uns diese Stadt auch erhalten, indem die
sie mit einer dicken Ascheschicht bedeckt wurde.

Die Kraft, die zu einer Katastrophe führt, schafft also \emph{selbst} eine
„Black Box“ zum Zeitpunkt der Katastrophe. Was ist das -- eine Hypothese, eine
ideale Variante der Lösung des Problems oder eine Gesetzmäßigkeit der Bildung
von „Black Boxes von Zivilisationen? Bei der Beantwortung dieser Frage hilft
eine Kartothek von \emph{Schlüsselfunden} zum Enträtseln der Kultur verlorener
Zivilisationen: mit Hilfe welcher Artefakte ist es gelungen, Informationen
über das Leben der Menschen zu entschlüsseln. In diesem Zusammenhang kann man
sich auch an die vielfältigen Beispiele von Schlüsselfunden ausgestorbener
Tiere erinnern: Mammuts, Neandertaler, Dinosaurier.

Es ist zum Beispiel der Fund eines sehr gut erhaltenen Dinosauriers aus der
Gattung Allosaurier bekannt.  Dieser Fund ist für die Wissenschaft so wichtig
geworden, dass die Überreste dieses Dinosauriers einen Namen bekamen - Big Al.
Nicht nur das Skelett ist gut erhalten, sondern auch ein Abdruck des
Herzen. Den Wissenschaftlern gelang es, die Besonderheiten des Verhaltens von
Allosauriern mit großer Präzision wiederherzustellen, ihr Lebensraum,
Bewegungsgeschwindigkeit, Gewohnheiten. Man hat es sogar geschafft die
Biographie vom Big Al zu rekonstruieren: In welchem Alter, welche Verletzungen
er erlitten hat, in welchen Situationen und auf welche Weise diese
Verletzungen und Krankheiten entstanden sein könnten, woran dieser Dinosaurier
gestorben ist, der vor 45 Millionen Jahren lebte. Der bereits alte und kranke
Big Al ging zum Fluss während eines trockenen Sommers. Es gab keinen Regen und
es kam immer noch kein Wasser. Erschöpft wartete er auf Wasser im Bett des
ausgetrockneten Flusses. Bald kam das Wasser, aber Big Al war bereits tot.
Die Schlickabdeckung des Körper und bewahrte ihn für uns für viele Millionen
von Jahren. Der Fluss, der den Dinosaurier getötet hat, hat \emph{selbst} die
Informationen über ihn konserviert.

Es ist möglich, dass es mit nur einer einzigen die Zerstörung einer
Zivilisation verursachenden Kraft nicht gelingen wird, einen absolut
zuverlässigen Mechanismus der Schaffung einer „Black Box“ um zum richtigen
Zeitpunkt zu finden. Daraus ergibt sich ein neues Problem: Wie kann man es
einrichten, das die „Black Box“ auch in einem normalen Leben ohne Katastrophen
nützlich und notwendig ist?  Dies würde das ganze Projekt der Schaffung einer
„Black Box der Zivilisation“ viel billiger machen, würde die
Wahrscheinlichkeit seiner Umsetzung erhöhen und die Zuverlässigkeit seines
Betriebs in der Periode vor einer Katastrophe gewährleisten.

Wie jedes Superproblem bietet das Black-Box-Projekt eine ganze Reihe von
Aufgaben, Problemen und Themen für die Forschung. Ich stelle noch eine davon
vor, aber zunächst die Karte „Wer rettet sich in einer Katastrophe“ aus meiner
Kartothek:
\begin{quote}  
  Mit dem Eintritt in das Atomzeitalter stehen die Supermächte vor einem
  ständigen Dilemma -- wessen Überleben ist im Falle eines großen Konflikts in
  erster Linie zu sichern: der Bevölkerung des Landes oder der höchsten
  Machtschichten? Ganz am Anfang seiner achtjährigen Amtszeit schob Präsident
  Reagan die Vorurteile des Kalten Krieges mit seiner Orientierung auf den
  Zivilschutz beiseite, und erklärte, dass ein Atomkrieg gewonnen werden kann,
  aber nur, wenn das Überleben der höchsten zivilen und militärischen Führer
  gesichert werden kann. Für Vertreter wichtiger Behörden ist ein
  Evakuierungsverfahren vorgesehen, deren Überleben als notwendig erachtet
  wird, um „die Kontinuität der Staatsmacht für insgesamt mehr als 1000
  Menschen zu gewährleisten“. (Komsomolskaya Pravda 26.08.1989).
\end{quote}

Ähnliche Programme gibt es nicht nur in den USA, sondern auch in anderen
Ländern. Als Ergebnis der Umsetzung solcher Projekte ist durchaus eine
Situation möglich, in der nach der Katastrophe auf der Erde nur „die höchsten
Ebenen der Macht“ erhalten bleiben. Sind das die Leute, die bleiben sollten,
um intelligentes Leben auf der Erde fortzusetzen?  Sie wissen, wie man einen
Staat führt, den es nicht gibt. Werden diese Menschen selbst überleben können,
ohne ihre Untergebenen?  Wenn eine moderne Zivilisation in der Lage ist, das
Überleben nur eines sehr kleinen Teils der Menschheit zu sichern, welche
Menschen sollten dabei überleben?  Welcher soziale Mechanismus soll diese
Auswahl garantieren? Was sollen diese Leute während der Zeit erzwungener
Untätigkeit tun, bis wieder akzeptable Bedingungen auf der Erde hergestellt
sind?  Können all diese Fragen beantwortet werden, ohne das Problem der „Black
Box der Zivilisation“ zu lösen.

\begin{thebibliography}{xxx}
\bibitem{Altshuller1991} G.S. Altschuller, I.M. Vertkin (1991). Wie man ein
  Ketzer wird.  Karelia, 1991, p.  166-168).
\bibitem{RubinXX} M. Rubin. \foreignlanguage{russian}{“Личные картотеки –
  фундамент творчества”, статья опубликована на сайте ОТСМ-ТРИЗ-технологий}
  \url{http://www.triz.minsk.by/e/221001.htm} (Persönliche Kartotheken als
  Grundlage der Kreativität)
\end{thebibliography}

\end{document}
