\documentclass[11pt,a4paper]{article}
\usepackage{od}
\usepackage[utf8]{inputenc}
\usepackage[russian]{babel}

\pagestyle{headings}
\title{Kurze Analyse des Problems der „Black Box“ der Zivilisation}
\author{M.S. Rubin}
\date{2020}

\begin{document}
\maketitle

Die Aufgabe der „Black Box“ der Zivilisation wurde als Generator neuer
Forschungsthemen vorgeschlagen, als Quelle für würdige Ziele eines Forschers.

G.S. Altschuller und I.M. Vertkin \cite{Altshuller1991} unterscheiden drei
Themenebenen:
\begin{itemize}
\item eng-technische, eng-wissenschaftliche, eng-künstlerische Themen, zum
  Beispiel die Erfindung einer Rakete;
\item allgemeine technische, allgemeine wissenschaftliche, allgemeine
  künstlerische Themen, zum Beispiel die Entwicklung der Raumfahrt; 
\item Themen, die die gesamte Zivilisation betreffen (sozio-technisch,
  sozial-wissenschaftlich, sozial-künstlerisch usw.), zum Beispiel die
  Gestaltung einer kesmischen Zivilisation oder einer Zivilisation mit
  Fernsehen.
\end{itemize}

Aus Sicht der Forschung ist es am effektivsten, mit Themen der dritten Stufe
zu arbeiten, auf der Ebene allgemein-menschlicher Probleme. Hier besteht der
größte Spielraum für die Entwicklung von Themen und die geringste
Wahrscheinlichkeit, bei kleinen Themen „hängen zu bleiben“ oder sich in einer
akuten Konfliktsituation mit anderen Forschungen zu befinden. Als Beispiel
möchte ich wenigstens ein solches Problem auf allgemein-menschlicher Ebene
präsentieren. Es geht um die Aufgabe der Schaffung einer „Black Box“ für den
Fall einer weltweiten Katastrophe. (\cite[S. 166-168]{Altshuller1991})

Nur sehr wenige TRIZ-Spezialisten haben an dem vorgeschlagenen Problem
gearbeitet, und es gibt wenige Veröffentlichungen zu diesem Thema. Die Aufgabe
ist schwierig und wir haben beschlossen, einige Ansätze dazu anzubieten. Dies
ist nur unsere Ansicht, und wir bestehen nicht nicht darauf, dass dies die
einzige Annäherung an das Problem der „Black Box der Zivilisation“ ist.

Wir möchten gleich offensichtlich schwache Lösungen aussortieren. Zum
Beispiel: Ein globales Systeme zum Sammeln von Informationen über jedem
Bewohner der Erde und diese auf einigen Satellit der Erde oder Venus zu
speichern. Trotz der offensichtlichen Unerfüllbarkeit, der Nichtidealität
solcher Ideen werden sie immer wieder vorgetragen. Das Forschungsthema wird
dann interessant und stark, wenn es darum geht, Widersprüche zu lösen, die im
Konzept der „Black Box“ der Zivilisation enthalten sind (zum Beispiel sollten
sehr viele Informationen gesammelt werden, um alle Aspekte der Zivilisation zu
reflektieren, und zugleich sollte wenig Information gesammelt werden, weil es
einfacher ist, diese zu speichern und für die Nachwelt zu bewahren; oder der
Speicherort sollte in der Nähe zu sein, um das Einspielen von Informationen zu
erleichtern, und er sollte zugleich weit entfernt sein, damit die
Informationen im Katastrophenfall sicherer erhalten bleibt) und einem idealen
Endresultat maximal nahe kommen (zum Beispiel, indem die katastrophale Kraft,
welche die Zivilisation zerstört, \emph{von sich aus} die „Black Box“ für die
Nachwelt aufbewahrt).

Während der TRIZ-Konferenz von Schülern und Studenten „IKARiade-2001“ wurde
eine morphologische Tabelle zur Analyse des Problems vorgeschlagen:

\begin{center}\small
  \begin{tabular}{|*{6}{>{\raggedright}p{1.9cm}|}c|}\hline
1. Welches Objekt & 1.1 Familie & 1.2. Haus & 1.3. Stadt & 1.4 Land &
1.5. Welt & \ldots \\\hline 
2. Wem, wohin & 2.1. anderen Menschen & 2.2. anderem Staat & 2.3. Nachfahren
in 50 (100, 1000) Jahren & 2.4. Außerirdische Zivilisationen & 2.5. Unbewohnte
Planeten& \ldots \\\hline
3. Ziel & 3.1. Information erhalten & 3.2. Fehler vermeiden &
3.3. Zivilisation erhalten & 3.4. Zivilisation neu auferstehen & 3.5. Grenzen
der Zivilisation ausdehnen& \ldots \\\hline
4. Ressourcen & 4.1. Ressourcen des zu erhaltenden Systems & 4.2. Ressourcen
der vernichtenden Kraft & 4.3. Ressourcen der heutigen Umwelt &
4.4. Ressourcen der Umwelt, in welche die „Black Box“ fällt & \ldots &
\\\hline  
5. Gründe der Katastrophe & 5.1 Natürliche & 5.2. Soziale Konflikte &
5.3. Natürliche Kataklysmen & 5.4 Zwischenmenschliche Konflikte & 5.5. Terror
und Verbrechen & \ldots \\\hline
  \end{tabular}
\end{center}
Diese Tabelle kann verwendet werden, um eine Vielzahl von Teilproblemen zu
formulieren, zum Beispiel kann die Variante (1.3-2.3-3.2-4.1-5.1) wie folgt
formuliert werden: Wie kann die Informationen über das moderne Petrosawodsk
für die Generation der Menschen des 22. Jahrhunderts aufbewahrt werden, um zu
vermeiden, dass sich die begangenen Fehler durch die Kräfte der Stadt selbst
wiederholen, wenn diese sich infolge natürlicher Evolution verändert. Die
Ebene der formulierten Aufgaben und deren Schärfe kann nach eigenem Geschmack
verändert werden.

Es gibt eine Vielzahl von Bereichen für weitere Forschungsarbeiten zum Thema
„Black Box der Zivilisation“. Es ist beispielsweise möglich, die Achsen der
Morphologie zu präzisieren, deren Anzahl und Inhalt. Sie können eine Auswahl
spezifischer Aufgaben treffen, die unter dem allgemeinen Thema der „Black Box
der Zivilisation“ gesammelt wurden. Die Analyse jeder dieser Teilaufgaben kann
durchgeführt werden: Widersprüche formulieren, Ideales Endresultat
formulieren, Ressourcen analysieren.  Man kann eine Hierarchie von Aufgaben
für das Thema „Black Box der Zivilisation“ erstellen. Man kann Informationen
über die häufigsten Ursachen für den Untergang von Zivilisationen
sammeln. Sehr fruchtbar kann eine Analyse von Methoden zur Wiederherstellung
von Informationen sien, die in der Archäologie und bei Historikern verfügbar
sind.  Möglichkeiten für Forschungsarbeiten gibt es wirklich sehr viel.

Sehr effektiv kann die Sammlung verschiedener Kartotheken sein (siehe dazu den
Aufsatz \cite{RubinXX} von M. Rubin zum Thema „Black Box der Zivilisation“).

Sie können zum Beispiel mit einer Kartothek zum Thema „Black Boxes“ mit
verschiedenen Flugzeugtypen, Ozeanschiffen oder anderen Systemen beginnen. Es
ist spannend, die Art der aufgezeichneten Informationen herauszufinden. Die
Hauptprobleme und Widersprüche von „Black Boxes“ bestehen darin, eine Prognose
ihrer Entwicklung zu geben. Es kann sich zum Beispiel herausstellen, dass die
„Black Box der Zivilisation“ eine logische Fortsetzung der Entwicklung
konventioneller „Black Boxes“ ist. Nebenbei können Sie neue Ideen und Lösungen
in diesem Bereich anbieten.

Eine sehr wichtige Frage: Was genau soll man über die untergegangene
Zivilisation an Nachkommen oder Außerirdische weitergeben? Um dies zu
beantworten, ist möglicherweise eine andere Kartei erforderlich -- über die
\emph{Schlüsselartefakte} (Denkmäler), die dazu beitragen, die Geheimnisse der
untergegangenen Zivilisation zu entschlüsseln.

Ich gebe nur ein Beispiel. Für das Studium der untergegangenen
Maya-Zivilisation war es sehr wichtig, die Schrift dieses Volkes zu
entziffern.  Dies gelang hauptsächlich an Hand von zwei
Dokumenten: 
\begin{itemize}
\item Dem Manuskript „Berichte über Angelegenheiten in Yucatan“ aus der Zeit
  der spanischen Eroberung der Maya-Indianer, wo insbesondere das abgekürzte
  Maya-Alphabet mit seiner \emph{Transkription ins gesprochene Spanische} sehr
  hilfreich war;
\item Aufzeichnungen über alte Mythen (Bücher Chilam Balam) aus dem
  16. Jahrhundert in \emph{lateinischen Buchstaben}, welche die Entwicklung
  der Maya-Sprache seit Beginn unserer Ära reflektieren.
\end{itemize}
Dazu muss man natürlich die moderne (lebendige) Maya-Sprachen hinzuzufügen,
die bis heute erhalten sind, sowie Informationen über die spanische Sprache
des 16. Jahrhunderts.

Es ist spannend zu bemerken, dass die „Black Box“ der Maya-Zivilisation in
Form eines Bi-Systems überliefert ist.  Das sind zuerst die Städte der alten
Maya mit ihren Tempeln, Inschriften, Bildern, Zeichnungen sowie der Schlüssel
zu ihrer Erklärung -- das Alphabet, das gerade erwähnt wurde.

Im schwierigsten Fall (alles Leben verschwindet von der Erde) ist eine
lebendige Sprache natürlich nicht gespeichert. Dies geschah zum Beispiel
mit den sumerischen Texten: Es gelang ihnen vollständig Ich kann entziffern,
aber niemand weiß, wie diese Sprache klang. Kombination erforderlich ein
Objekt oder eine Handlung (in verschiedenen Situationen gezeigt) mit seiner
Schrift und seinem Ton kovy (Rede) Bild. So etwas wie TV-Comics zum Lernen
Fremdsprachen.  Das Beispiel der "Black Box" der alten Maya-Zivilisation legt
eine interessante Idee nahe. Unternehmen ist, dass das Manuskript "Bericht
über die Angelegenheiten in Yucatan das wichtigste, vollständigste und
genaueste ist Kein Dokument über die antike Zivilisation - wurde vom
spanischen Bischof Diego de zusammengestellt Landa. Es war fast unter seiner
Führung alles Schreiben und anderes Maya-Kulturerbe (für diese Aktivität de
Landa und erhielt den Titel eines Bischofs). De Landas Manuskript ist eine Art
Bericht an die Behörden über die "Arbeit" zur Ausrottung der Häresie. Eine
interessante Hypothese ergibt sich: der Grund was zu einer Katastrophe führte
(zum Beispiel die von Diego de Landa vertretenen Spanier), die sich selbst
geschaffen haben Es gibt eine "Black Box" (berühmtes Bischofsmanuskript). Das
zweite Dokument ist der Schlüssel zur Lösung Maya-Texte - "Die lateinisch
geschriebenen Bücher von Chilam Balam - haben so etwas gleiche Herkunft wie
das Manuskript von de Landa. Die Spanier verbannten die Maya aus um ihre
einheimische Schrift genannt zu werden, und sie begannen, alte Texte in einem
erlaubten niederzuschreiben Lateinisches Alphabet. Trotzdem: Die Spanier
zwangen die Maya durch ihre Verbote selbst dazu um ein Dokument über Ihre
Kultur zu erstellen, das für die Nachwelt verständlich ist. Ein weiteres
Beispiel darüber Seite 4 4 Wir haben uns bereits daran erinnert: Der Ausbruch
des Vesuvs hat Pompeji zerstört, aber auch erhalten für uns diese Stadt, die
sie mit einer dicken Ascheschicht bedeckt.  Als die Kraft, die zu einer
Katastrophe führt, schafft ITSELF zum Zeitpunkt der Strophen. Was ist das -
eine Hypothese, eine ideale Lösung für ein Problem oder ein Muster die Bildung
von "Black Boxes" von Zivilisationen? Eine Kartei hilft bei der Beantwortung
dieser Frage.  ka WICHTIGE ERGEBNISSE, um die Kultur verlorener Zivilisationen
zu enträtseln: mit Hilfe die es geschafft haben, Informationen über das Leben
der Menschen zu entschlüsseln. In dieser Hinsicht können Sie verwenden Faden
und über zahlreiche Beispiele von Schlüsselfunden antiker Tiere: Mammuts,
Neandertaler, Dinosaurier.  Es ist zum Beispiel der Fund eines sehr gut
erhaltenen Dinosauriers aus der Gattung al- bekannt.  Verlierer. Dieser Fund
ist für die Wissenschaft so wichtig geworden, dass die Überreste dieses
Dinosauriers beim Namen genannt - Big Al. Nicht nur das Skelett ist gut
erhalten, sondern auch der Abdruck Herzen. Den Wissenschaftlern gelang es, die
Verhaltensmerkmale mit absoluter Genauigkeit wiederherzustellen Allosaurier,
ihr Lebensraum, Bewegungsgeschwindigkeit, Gewohnheiten. Ich habe es sogar
geschafft Biographie von Big Al: In welchem ​​Alter, welche Verletzungen er
erlitten hat, auf welche Weise Diese Verletzungen und Krankheiten könnten
jedoch entstehen, von denen dieser Dinosaurier lebte Vor 45 Millionen
Jahren. Big Al war schon alt und krank und ging währenddessen zum Fluss
trockener Sommer. Es gab keinen Regen und es kam immer noch kein
Wasser. Erschöpft wartete er Wasser im Bett eines ausgetrockneten
Flusses. Bald kam das Wasser, aber Big Al war bereits tot. Schlickabdeckung
Körper und bewahrte es für uns für viele Millionen von Jahren. Der Fluss, der
den Dinosaurier getötet hat, SAMA hielt auch Informationen über ihn.  Es ist
möglich, dass mit nur einer einzigen Kraft die Zerstörung der Zivilisation
nicht gelingen wird Finden Sie einen absolut zuverlässigen Mechanismus, um zum
richtigen Zeitpunkt eine "Black Box" zu erstellen. Zu- Wenn ein neues Problem
auftritt: Wie macht man die "Black Box" nützlich und notwendig?  in einem
normalen Leben ohne Katastrophen. Dies würde das ganze Projekt machen Die
Schaffung einer "Black Box" der Zivilisation ist viel billiger, würde die
Wahrscheinlichkeit ihrer erhöhen Implementierung würde die Zuverlässigkeit
seines Betriebs in einer Katastrophe gewährleisten Zeitraum.  Wie jedes
Superproblem bietet das Black Box-Projekt eine ganze Reihe von Aufgaben und
Problemen und Themen für die Forschung. Ich werde noch eine davon
präsentieren, aber zuerst - eine Karte von mir Aktenschränke: "Wer wird in
einer Katastrophe gerettet."  "Seit dem Eintritt in das Atomzeitalter stehen
die Supermächte vor einem ständigen Dilemma - deren Überleben im Falle eines
großen Konflikts in erster Linie zu sichern ist: die Bevölkerung des Landes
oder die höchsten Machtschichten? Ganz am Anfang seines achtjährigen
Lebensjahres Während seiner Amtszeit warf Präsident Reagan die Vorurteile des
Kalten Krieges beiseite sein Fokus auf Zivilschutz, erklärte, dass ein
Atomkrieg gewonnen werden kann, aber nur wenn das Überleben der höchsten
Zivilisten und Militär Führer. Das Evakuierungsverfahren ist für Vertreter
höherer Behörden vorgesehen Machtstufen, deren Überleben als notwendig
erachtet wird, um "die Kontinuität zu gewährleisten" Durchbruch Staatsmacht
für insgesamt mehr als 1000 Menschen. "(" Komsomolskaya Prav- ja 26.08.1989).
Ähnliche Programme sind nicht nur in den USA, sondern auch in anderen Ländern
verfügbar. Als Ergebnis Bei der Umsetzung solcher Projekte ist durchaus eine
Situation möglich, in der nach dem Seite 5 fünf Strophen auf der Erde werden
nur "die höchsten Ebenen der Macht" bleiben. Sind das Leute, die sollte
bleiben und intelligentes Leben auf der Erde fortsetzen?  Sie wissen, wie man
den Staat führt Geschenke, die nicht sein werden. Werden diese Menschen ohne
ihre Untergebenen alleine überleben können?  Wenn die moderne Zivilisation in
der Lage ist, das Überleben nur eines sehr kleinen zu sichern Der größte Teil
der Menschheit, welche Art von Menschen sollte dies überleben? Was für ein
soziales Der Mechanismus sollte diese Auswahl bieten? Was sollen die Leute
während dir tun?  bedürftige "Inhaftierung, bis akzeptable Bedingungen auf der
Erde wiederhergestellt sind?  all diese Fragen zu beantworten, ohne das
Problem der "Black Box der Zivilisation" zu lösen.

\begin{thebibliography}{xxx}
\bibitem{Altshuller1991} G.S. Altschuller, I.M. Vertkin (1991). Wie man ein
  Ketzer wird.  Karelia, 1991, p.  166-168).
\bibitem{RubinXX} M. Rubin. “Личные картотеки – фундамент творчества”, статья
  опубликована на сайте ОТСМ-ТРИЗ-технологий
  \url{http://www.triz.minsk.by/e/221001.htm} (Persönliche Kartotheken als
  Grundlage der Kreativität)
\end{thebibliography}

\end{document}

Очень эффективным можно считать сбор различных картотек (см. статью М. Рубина
“Личные картотеки – фундамент творчества”, статья опубликована на сайте
ОТСМ-ТРИЗ-технологий  \url{http://www.triz.minsk.by/e/221001.htm} ) по теме
“черного ящика цивилизации”.

Можно, например, начать с картотеки "черных ящиков" - с самолетов разного
типа, океанских лайнеров или других систем. Любопытно выяснить характер
фиксируемой информации, основные проблемы и противоречия "черных ящиков",
сделать прогноз их развития. Может оказаться, например, что "Черный ящик
цивилизации" - логическое продолжение развития обычных "черных
ящиков". Попутно можно предложить новые идеи и решения в этой области.

Очень важный вопрос: что именно передавать потомкам или пришельцам о погибшей
цивилизации. Для ответа на него может оказаться другая картотека - о КЛЮЧЕВЫХ
НАХОДКАХ (памятников), помощью которых удалось раскрыть тайны погибших
цивилизаций.
\medskip

Приведу только один пример. Для изучения погибшей цивилизации майя очень важно было расшифровать письменность этого народа. Сделать это удалось, главным образом, с помощью двух документов:
\begin{itemize}
\item рукописи "Сообщение о делах в Юкатане" времен завоевания испанцами
  индейцев майя, где, в частности, приводился сокращенный алфавит майя с его
  "ОЗВУЧИВАНИЕМ" ИСПАНСКИМИ буквами;
\item записям древних мифов (книги Чилам Балам), сделанных в XVI веке
  ЛАТИНСКИМИ БУКВАМИ и отражающих язык майя времен начала нашей эры.
\end{itemize}
К этому, безусловно, необходимо добавить дошедшие до нас современные (живые)
языки майя и испанцев, а также сведения об испанском языке ХYI века.

Любопытно отметить, что "черные ящик" цивилизации майя дошел до нас в виде
бисистемы: в первую очередь это сами города древних майя с храмами, надписями
на них, картинками, рисунками, а также ключ к их объяснению - алфавит, о
котором мы только что упоминали.
\medskip

При самом трудном варианте (с Земли исчезает всякая жизнь) живой язык,
естественно, не сохранится. Так, например, произошло с шумерскими текстами: их
удалось полностью расшифровать, но никто не знает, как звучал этот
язык. Необходимо совмещение предмета или действия (показанных в различных
ситуациях) с его письменным и звуковым (речевым) изображением. Что-то вроде
телевизионных комиксов для изучения иностранных языков.

Пример "черного ящика" цивилизации древних майя наводит на интересную
мысль. Дело в том, что рукопись "Сообщение о делах в Юкатане" - основной,
наиболее полный и точный документ о древней цивилизации - была составлена
епископом-испанцем Диего де Ланда. Именно под его руководством тщательнейшим
образом было уничтожено почти все письменно и другое культурное наследие майя
(за эту свою деятельность де Ланда и получил звание епископа). Рукопись де
Ланды - своеобразный отчет перед начальством о проделанной "работе" по
истреблению ереси. Возникает любопытная гипотеза: причина, приводящая к
катастрофе (например, испанцы в лице Диего де Ланды) САМА создает "черный
ящик" (знаменитую рукопись епископа). Второй документ-ключ к разгадке текстов
майя - "Книги Чилам Балам", написанные на латинице - имеют примерно такой же
характер происхождения, как и рукопись де Ланды. Испанцы запрещали майя
пользоваться родной письменностью, и те стали записывать древние тексты на
дозволенной латинице. Все та же ситуация: испанцы, своими запретами, САМИ
вынудили майя создать документ о своей культуре, понятный для потомков. Еще
один пример, о котором мы уже вспоминали: извержение Везувия уничтожило
Помпеи, но оно же и сохранило для нас этот город, засыпав его толстым слоем
пепла.
\medskip

Итак, сила, приводящая к катастрофе, САМА создает "черный ящик" в момент
катастрофы. Что это - гипотеза, идеальный вариант решения проблемы или
закономерность образования "черных ящиков" цивилизаций? Ответить на этот
вопрос поможет картотека КЛЮЧЕВЫХ НАХОДОК к разгадке культуры погибших
цивилизаций: с помощью чего удалось расшифровать информацию о жизни людей. В
связи с этим можно вспомнить и о многочисленных примерах ключевых находок
древних животных: мамонтов, неандертальцев, динозавров.

Общеизвестна, например, находка очень хорошо сохранившегося динозавра из рода
аллозавров. Эта находка стала настолько важной для науки, что останки этого
динозавра назвали по имени – Большой Ал. Хорошо сохранился не только скелет,
но отпечаток сердца. Ученым удалось восстановить с абсолютной точностью
особенности поведения аллозавров, среду их обитания, скорость передвижения,
повадки. Удалось даже сделать жизнеописание Большого Ала: в каком возрасте,
какие травмы он получил, в каких ситуациях эти травмы и болезни могли
возникнуть, от чего умер этот динозавр, живший 45 миллионов лет назад. Уже
старый и больной Большой Ал отправился к реке во время засушливого
лета. Дождей не было, и вода все не приходила. Обессиленный, он ждал воду в
русле высохшей реки. Вскоре вода пришла, но Большой Ал уже умер. Ил обволок
тело и сохранил его для нас на многие миллионы лет. Река, убившая динозавра,
САМА же сохранила информацию о нем.
\medskip

Возможно, что с помощью одной только разрушающей цивилизацию силы не удастся
найти абсолютно надежный механизм создания в нужный момент "черного
ящика". Тогда возникает новая задача: как сделать, чтобы "черный ящик" был
полезен и необходим при нормальном течении жизни без катастроф. Это позволило
бы сделать весь проект создания "черного ящика" цивилизации гораздо дешевле,
повысило бы вероятность его реализации, обеспечило бы надежность его
функционирования в до катастрофический период.

Как и всякая сверхпроблема, проект "Черный ящик" дает целый спектр задач,
проблем и тем для исследования. Представлю еще одну из них, но прежде -
карточка из моей картотеки:

"Кто спасется в катастрофе".

"С момента вступления в ядерную эпоху перед сверхдержавами постоянно стоит
дилемма - чье выживание обеспечить в первую очередь в случае крупномасштабного
конфликта: населения страны или высших эшелонов власти? В самом начале своего
восьмилетнего пребывания у власти президент Рейган, отбросив предрассудки
"холодной войны" с ее ориентацией на гражданскую оборону, заявил, что в
ядерной войне можно победить, но лишь в том случае, если будет обеспечено
выживание высших гражданских и  военных руководителей. Процедура эвакуации
предусмотрена для представителей высших эшелонов власти, выживание которых
считается необходимым для "обеспечения непрерывности государственной власти" -
всего для более 1000 человек".  ("Комсомольская правда", 26.08.1989).

Аналогичные программы имеются не только в США, но и в других странах. В
результате реализации подобных проектов вполне возможна ситуация, при которой
после катастрофы на Земле останутся только "высшие эшелоны власти". Те ли это
люди, которые должны остаться и продолжить разумную жизнь на Земле? Они умеют
руководить государствами, которых не будет. Смогут ли эти люди выжить сами,
без своих подчиненных? Если современная цивилизация способна обеспечить
выживание только очень небольшой части человечества, то какие люди должны при
этом выжить? Какой социальный механизм должен обеспечить этот отбор? Чем
должны быть заняты люди во время вынужденной "отсидки", пока не восстановились
приемлемые условия на Земле? Удастся ли ответить на все эти вопросы без
решения задачи о “черном ящике цивилизации”.


\end{document}
