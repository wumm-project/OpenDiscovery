\documentclass[11pt,a4paper]{article}
\usepackage{od,wrapfig,array}
\usepackage[utf8]{inputenc}
\usepackage[russian]{babel}

\title{РЕГИСТР КАК СПОСОБ ПОЗНАНИЯ}
\author{Амнуэль П.}
\date{2000}

\begin{document}
\maketitle


\url{https://www.altshuller.ru/rtv/sf-registern.asp}

ЗАКЛЮЧЕНИЕ

Не думаю, что даже внимательный и усидчивый читатель прочитал Регистр от
первой до последней страницы. Не для такого "сквозного" чтения этот труд
предназначен. О целях создания Регистра я уже писал в Предисловии. Теперь,
когда читатель представил себе объем проделанной Г.С.Альтшуллером работы,
поговорим о результатах.

Во-первых, создана методика конструирования НФИ - а если шире: методика РТВ,
развития творческого воображения.

Во-вторых, анализ идей и ситуаций, собранных в Регистре, позволил
сконструировать шкалу оценки НФИ "Фантазия-2".

В-третьих, на основе Регистра создан Патентный фонд фантастики (ПФФ).

Рассмотрим эти результаты более детально.

* * *

Г.Альтов создал так называемую ЭТАЖНУЮ СХЕМУ конструирования НФИ. Суть заключается в следующем.

Выберем объект, развитие которого мы хотим спрогнозировать. Например, космический скафандр. И спросим себя: для какой цели он существует? Скафандр необходим, чтобы оградить человека от влияния космоса: от вакуума, жесткого излучения... Итак, мы выбрали объект и цель. Первый этаж схемы - ИСПОЛЬЗОВАНИЕ ОДНОГО ОБЪЕКТА (в нашем случае - скафандра). Это, конечно, давно не фантастика: достаточно вспомнить А.Леонова, Н.Армстронга. Но заметьте: это не фантастика сейчас, а лет сто назад рассказ о том, как человек надел скафандр и вышел в космос, был точным предвидением!

Этаж второй - используется МНОГО скафандров. Например, люди расселяются в космосе, создаются "эфирные города", описанные К.Э.Циолковским. Но что такое "много"? Пятьсот? Или пятьсот тысяч? А.Беляев в "Звезде КЭЦ" писал о космическом городе, где живут сотни человек. В "Туманности Андромеды" И.Ефремова в космосе обитают миллионы. А если человек победит природу на Земле и вынужден будет переселиться в космос, то каждый из нас будет обладателем персонального скафандра. Или даже десятка - скафандр для работы, для прогулки, для посещения заповедника на Земле... Кстати, такой роман еще не написан, вполне прогностическая идея ждет автора. Возможны варианты: очень много скафандров, небольшое число скафандров... Скажем, наступят времена, когда выпуск скафандров будет количественно ограничен, производство скафандров свернется, когда их полное число достигнет, скажем, пятисот (или пятисот тысяч). Фантастическое допущение создает сюжетные коллизии (скафандр - редкость, за обладание им идет жестокая борьба) и позволяет на этом воображаемом полигоне проверить те или иные тенденции реальной космонавтики, но позволяет выявить и нечто новое в характере героев.

А перед нами третий этаж: достижение ТОЙ ЖЕ цели, но БЕЗ ИСПОЛЬЗОВАНИЯ объекта (в данном случае - скафандра). Человек защищен от влияния космоса, однако, скафандра на нем нет. Если на первых двух этажах число объектов возрастало, то теперь произошел качественный скачок (вот, что труднее всего дается ученым-футурологам, вот где фантаст выходит вперед!). Нужно придумать качественно новую ситуацию, предсказать изобретение или открытие будущего. Третий этаж для объекта "скафандр" - киборгизация человека, создание разумных существ, соединяющих в себе лучшие качества человека и машины. Те части человеческого тела, которые, будучи искусственными, станут функционировать лучше данных нам природой, в будущем непременно будут заменены. В космосе не нужно дышать, и у будущих космических путешественников "ампутируют" легкие, заменив их более простым устройством, способным накачивать в кровь кислород.

Фантасты первыми разглядели такую возможность в эволюции человека. Один из прообразов литературных киборгов (см. Регистр!) появился в 1911 году в рассказе Д.Ингленда "Человек со стеклянным сердцем". Киборг, управляющий космическим кораблем, описан Г.Каттнером в рассказе "Маскировка". Человек, работающий без скафандра в условиях космоса или чужой планеты, - тема таких прекрасных произведений, как "Город" К.Саймака (1944 год), "Зовите меня Джо" П.Андерсона (1957 год), "Далекая Радуга" А. и Б.Стругацких (1964 год) и др.

Поднимемся еще выше - на четвертый этаж. Ситуация, когда вовсе ОТПАДАЕТ НЕОБХОДИМОСТЬ в достижении поставленной цели. В нашем примере это ситуация, когда не нужно защищать человека от космоса, потому что космос для человека безвреден. То есть, в космосе есть воздух, чтобы дышать. Откуда? Перечитайте повесть Г.Альтова "Третье тысячелетие" (1974 год). Идея такая: нужно распылить Юпитер, превратить его вещество в пыль, газ. Вокруг Солнца образуется газовый диск, внутри которого проходит и орбита Земли. Нет больше пустоты пространства! От Земли к Луне и Марсу можно летать на реактивных самолетах и даже на... воздушных шарах. В космосе между планетами сгущаются облака, гремят грозы... А как вам нравится космическая радуга, протянувшаяся на десятки миллионов километров семицветной дугой - от Венеры к поясу астероидов?

Разумеется, рассмотренные идеи третьего и четвертого этажей - вовсе не единственно возможные для объекта "скафандр". Каждый автор волен придумывать свой вариант ответа на вопрос, поставленный этажной схемой. На каждом этаже рассмотренной схемы можно разместить очень многие идеи научно-фантастических произведений.

Этажное конструирование хорошо тем, что идеи распределены всего по четырем классам-этажам. Есть и недостаток - метод хорошо "работает", если выбран неодушевленный объект. Желательно - объект искусственный. Тогда не возникают трудности при формулировании цели, которой предстоит достичь, используя данный объект. Но попробуйте придумать идею третьего этажа для объекта "человек". Придется сначала ответить на "простой" вопрос: какова цель существования человека? В чем смысл жизни?..

* * *

В начале семидесятых, использовав классификацию идей из Регистра и приемы ТРИЗ, П.Амнуэль и Р.Леонидов разработали еще одну методологию конструирования НФИ: конструирование по приемам. Анализ НФИ показал, что каждая идея может быть получена как результат изменения некоей реалистической идеи (явления, объекта) с помощью того или иного ПРИЕМА. Был сформирован список приемов РТВ, лишь частично совпадавший с аналогичными приемами ТРИЗ.

Расскажу лишь о нескольких приемах, чтобы читателю стало ясно, о чем идет речь.

...В бухте появился страшный хищник, способный лодку с людьми превратить в плоский блин. И что странно: никто этого монстра никогда не видел. Так начинается рассказ советского фантаста Севера Гансовского "Хозяин бухты". Оказалось, что в воде жили миллиарды микроорганизмов, которые в минуту опасности объединялись в единое существо, способное переломить хребет акуле. Опасность исчезает, и существо тут же распадается на миллиарды составляющих. Вот и попробуй побороться с таким чудовищем!

Гансовский использовал прием ОБЪЕДИНЕНИЯ.

Очень популярный в фантастике прием - СДЕЛАТЬ НАОБОРОТ. Фантастические идеи, полученные с помощью этого приема, любопытны и парадоксальны. Вспомним рассказ Уильяма Тэнна "Срок авансом". Как известно, если кто-то кого-то убьет, то получит большой срок заключения, если не "вышку". Это в наши дни. А в мире будущего все наоборот. Некто является в суд, заявляет, что намерен убить своего врага и получает за это срок. Отсидев (за хорошее поведение - половину срока), некто получает полное право отыскать этого врага и убить его. Согласитесь, нетривиальная идея, отличная работа воображения, а сколько психологических коллизий! Ведь герой рассказа вовсе не объявляет заранее, кого из своих знакомых он намерен "пришить", вернувшись из заключения. Десятки людей, с которыми он был так или иначе связан, теряют покой - кто из них?..

Среди приемов развития воображения прием НАОБОРОТ стоит особняком. Причина простая: с ног на голову можно ведь ставить не только вещи, явления или ситуации, но и приемы развития воображения. Например, вместо приема объединения получим ДРОБЛЕНИЕ. Вспомним лемовскую идею передачи людей на расстояние. Сначала профессора Тарантогу РАЗДРОБИЛИ на отдельные атомы, а потом, в другом уже месте, эти атомы ОБЪЕДИНИЛИ в милого профессора.

Анализ Регистра показал, что у фантастов есть приемы, которыми изобретатели не пользуются и которых нет в списке приемов ТРИЗ - слишком уж они сильны. Например: если какое-то свойство предмета или явления кажется вам совершенно неизменным, - измените его. Это прием ИЗМЕНИТЬ НЕИЗМЕНЯЕМОЕ.

Пример: астроинженерная деятельность. Переделка небесных тел: астероидов, планет и даже звезд и галактик.

К области астроинженерной деятельности ("изменить неизменяемое") относится, например, переделка климата планет - прежде всего Марса и Венеры. В 1961 году Карл Саган предложил распылить в атмосфере Венеры простейшие водоросли, которые перерабатывают углекислый газ в кислород. Аналогичным образом было предложено (автор проекта М.Д.Нусинов) изменить и климат Марса.

Но ведь на самом деле обе эти идеи пришли из фантастики! Еще в тридцатых годах герои романа Олафа Степлдона "Последние и первые люди" начали создавать на Венере кислородную атмосферу. Впоследствии к этой задаче обращались герои "Большого дождя" П.Андерсона, "Плеска звездных морей" Е.Войскунского и И.Лукодьянова и др.

Другая идея, полученная с помощью этого приема: управление тяготением. Ученые и сейчас считают, что это невозможно. Но разве это мешает фантастам создавать интересные произведения? Герберт Уэллс в романе "Первые люди на Луне" изобрел вещество "кейворит", которым можно отгородиться от поля тяжести.

Есть в фантастике еще управление разбеганием галактик ("Порт Каменных Бурь" Г.Альтова), управление процессами зарождения жизни на планетах ("Великая сушь" В.Рыбакова), изменение мировых постоянных - скорости света и постоянной Планка ("Все законы Вселенной", "Крутизна", "Бомба замедленного действия" П.Амнуэля).

Вот еще прием, используемый фантастами: ВЫНЕСЕНИЕ. В ТРИЗ его называют приемом "отделения функции от объекта". Заключается прием в следующем: нужно отделить от объекта одно из его главных свойств. Или наоборот - приписать данному объекту свойство совершенно другого объекта.

Космические корабли должны иметь двигатели (ведь это транспортное средство) и создавать условия для жизнедеятельности экипажа (в сущности, выполнять функции огромных скафандров). А теперь отделим от космического корабля эти два его основных качества. Отделив от корабля свойство создавать условия для жизни экипажа, мы получим всего лишь корабль-автомат, которым управляет экипаж, находящийся на Земле. И долгое время никому в голову не приходило, что от космического корабля (или от простого корабля, или от подводной лодки) можно отделить такую неотъемлемую часть, как двигатель.

Прием "отделить функцию от объекта" не говорит, что функция исчезает вовсе. Просто ее перемещают в другое место: корабль летит в космосе, а его двигатель стоит на Земле. В 1896 году французские фантасты Жак Ле Фор и Антуан де Графиньи опубликовали повесть "Вокруг Солнца". Русский физик П.Н.Лебедев лишь два года спустя после выхода этой повести начал свои опыты, которые еще через несколько лет привели к открытию давления света на твердые тела. А герои повести "Вокруг Солнца" поставили на Земле огромный прожектор, направили его луч на корму космического корабля, и давление света заставило корабль улететь в космос. Полет на острие светового луча в 1896 году был для науки таким же нонсенсом, как сейчас - вечный двигатель...

В середине пятидесятых годов, когда первые счетно-вычислительные машины выполняли в секунду каких-то две-три тысячи операций, а в СССР кибернетика числилась еще в продажных девках империализма, А.Азимов опубликовал рассказ "Все грехи мира" о суперкомпьютере, в который стекается информация обо всем, что происходит на планете. Информация о людях - в том числе. Взял фантаст "обычный" компьютер, использовал прием УВЕЛИЧЕНИЯ.

В фантастике сотни интереснейших идей, связанных с будущим кибернетики. Многие сбылись. Многие сбудутся. Румынский писатель Раду Нор (рассказ "Живой свет", 1959 год) писал о думающей машине размером с молекулу (прием УМЕНЬШЕНИЯ). Станислав Лем в романе "Непобедимый" - о цивилизации микророботов (УМЕНЬШЕНИЕ). Это - очередное поколение компьютеров, проблема, над которой ученые задумались всерьез в начале ХХI века.

Еще один прием - УСКОРЕНИЕ (и соответственно - ЗАМЕДЛЕНИЕ): выбрать объект или процесс и ускорить его действие до такой степени, чтобы возникло новое качество. Герой рассказа Герберта Уэллса "Новейший ускоритель" выпивает некий препарат, и все процессы жизнедеятельности в организме ускоряются во много раз. Он все делает так быстро, что окружающий мир для него как бы застывает. Люди, подобно черепахам или улиткам, медленно-медленно переставляют ноги - герой рассказа успевает пройти целый квартал, прежде чем кто-нибудь другой делает шаг. Скоро он замечает, что на нем начинает тлеть одежда - он (на самом-то деле!) двигается так быстро, что от сопротивления воздуха нагревается до высокой температуры! Ему кажется, что он медленно прикасается пальцем к металлу - на самом деле это происходит так быстро, что палец может сломаться...

Сочетание приемов УСКОРЕНИЯ и ИЗМЕНЕНИЯ НЕИЗМЕНЯЕМОГО: если звездолетам всегда суждено (как утверждает теория относительности) двигаться медленнее света, то прием ускорения требует увеличить скорость света. Идею ускорения света в импульсном режиме предлагает герой рассказа Г.Альтова "Полигон ’Звездная река’" (1960 год).

Прием УНИВЕРСАЛИЗАЦИИ (ОБОБЩЕНИЯ) позволяет сделать идею изменения скорости света еще более фантастической - речь идет об изменении всех законов природы ("Все законы Вселенной" П.Амнуэля, 1968).

Какой должна быть следующая идея, полученная с помощью этого приема? Разве не ясно: очередное обобщение: не только человек может менять законы природы, но и другие цивилизации тоже этому научились. В 1971 году С.Лем опубликовал эссе "Новая космогония", в котором сделал фантастическое "открытие": известные нам законы природы, утверждал фантаст, являются результатом совместной деятельности внеземных цивилизаций! Польский фантаст воспользовался приемом "СДЕЛАТЬ ИСКУССТВЕННЫМ". Наука полагает, что законы природы - естественное свойство материи? Сделаем их искусственными. И другой важный принцип фантазирования соблюден - мы говорили, что изменять прежде всего нужно то, что, казалось бы, никаким изменениям не поддается? Законы природы - из этой категории "неизменяемых" объектов.

* * *

Исследуя НФИ, собранные в Регистре, Г.Альтов показал, что многие фантастические идеи можно получить, применив метод, используемый также в изобретательстве. Это МОРФОЛОГИЧЕСКИЙ АНАЛИЗ. Цель метода - систематический обзор и анализ всех мыслимых вариантов данного явления или объекта. Как и при работе с этажной схемой, выбираем объект, который хотим исследовать. Затем составляем список всех мыслимых характеристик выбранного объекта (для простоты можно выбрать одну-две главные характеристики). После этого для каждой характеристики перечисляем все мыслимые варианты. В результате получаем таблицу, на одной оси которой выписаны все параметры объекта, а на другой - все варианты и значения этих параметров. Отбираем те клетки таблицы, в которых заключены самые невероятные сочетания параметров.

Выберем в качестве примера автомобиль. Список характеристик: двигатель, движитель, кабина, горючее, опора, система управления, дорога... Вот варианты этих характеристик:

1. Двигатель: а) внутреннего сгорания, б) внешнего сгорания, в) электрический, г) магнитогидродинамический, д) реактивный, е) паровой, ж) турбовинтовой, з) газотурбинный, и) атомный, к) термоядерный, л) плазменный...

2. Движитель: а) колесо, б) гусеницы, в) ноги, г) винт, д) струя...

3. Расположение двигателя по отношению к кабине: а) впереди, б) сверху, в) сзади, г) снизу, д) сбоку, е) вне объекта...

4. Источник энергии: а) горение топлива, б) электрическая батарея, в) распад атомных ядер, г) ядерный синтез, д) химические процессы, не связанные с горением, е) ветер, ё) Солнце...

5. Расположение источника энергии: а) в автомобиле, б) вне автомобиля...

6. Опора: а) движитель, б) пол кабины, в) полозья, г) воздушная подушка, д) паровая подушка, е) магнитная подушка...

7. Управление: а) ручное, б) автоматическое, в) полуавтоматическое, г) дистанционное, д) биотоковое...

8. Дороги: а) с твердым покрытием, б) грунтовые, в) жидкие, г) отсутствие дорог...

Конечно, этот список весьма неполон, и вы можете заполнить "морфологический ящик" более "плотно". Обычному автомобилю в нем соответствует следующее сочетание: 1а-2а-3а-4а-5а-6а-7а-8а... Уже в этом "малом морфологическом ящике" содержатся около 100 тысяч возможных (а также и технически невероятных) комбинаций-вариантов автомобиля. Здесь наверняка есть автомобили близкого и далеко будущего (найдите их!), и автомобили, которые никогда не будут сконструированы. Вот, например, одно из необычных сочетаний: 1д-2в-3е-4е-5б-6г-7д-8г. Двигатель реактивный и расположен вне автомобиля, например, на заправочной станции; работает на солнечной энергии; передвигается при помощи ног; управление биотоковое; автомобиль может двигаться совсем без дорог, способен, например, взбираться на горные кручи...

* * *

Объединив конструирование НФИ по приемам и морфологический анализ, Г.Альтов создал еще один метод РТВ: МЕТОД ФАНТОГРАММ. Фантограмма представляет собой трехмерную таблицу - двумерный морфологический ящик дополняется осью ИЗМЕНЕНИЯ ПО ПРИЕМАМ.

Фантограмма потенциально содержит намного больше идей, нежели способен дать морфологический анализ, поскольку каждая из идей, полученных морфологическим методом, многократно изменяется, приобретая фантастические качества.

Рассмотрим для примера клетку морфологического ящика, находящуюся на пересечении линий "непрерывное оптическое излучение" и "планета в иной звездной системе". Намеренно выбрана довольно тривиальная начальная идея - планета светит отраженным светом, и на фоне звезды это излучение неразличимо. Что ж, достроим фантограмму - обратимся к приемам. Прием увеличения требует усилить оптическое излучение планеты, сделать его более мощным, чем полное излучение звезды. Если наша цель - посылка сообщения, достаточно, чтобы излучение планеты было столь мощным лишь в течение короткого времени (прием КВАНТОВАНИЯ). Откуда берется энергия излучения? Либо изнутри (ИСПОЛЬЗОВАНИЕ СВОЙСТВ ОБЪЕКТА + УВЕЛИЧЕНИЕ), либо снаружи (ИСПОЛЬЗОВАНИЕ СВОЙСТВ СРЕДЫ). Единственным достаточно мощным источником энергии является звезда, около которой обращается наша гипотетическая планета.

Итак, первая из идей такова. Каким-то образом планета накапливает энергию, получаемую от звезды, и через некоторое время выделяет эту энергию в виде оптического импульса, который может быть, в частности, модулирован с целью посылки сообщения. Напомню, что речь идет не о передатчике на поверхности планеты (это другая клетка фантограммы), а об использовании свойств самой планеты. Каким образом планета может накапливать энергию светила? Либо в почве, либо в атмосфере.

Рассмотрим накопление энергии в атмосфере. Энергия в атмосфере планеты может быть накоплена, в частности, за счет ионизации с последующим использованием энергии рекомбинации (в этом случае нужно еще изобрести способ удержать от рекомбинации газ атмосферы в течение долгого времени). Накопление энергии в атмосфере может происходить за счет возбужденных атомов: атомы в атмосфере не ионизируются, но долгое время находятся в возбужденном состоянии (на физическом языке это называется инверсной заселенностью энергетических уровней).

В последнем случае речь идет о создании, в сущности, сверхмощного газового лазера с накачкой от излучения центральной звезды. Для этого атмосфера планеты должна иметь специфические химический состав и плотность.

Кстати, излучение лазерного типа в атмосферах планет (например, Марса) уже наблюдалось. Используя этот факт вместе с приемом увеличения, можно получить идею о планете-лазере непосредственно, не прибегая к методу фантограмм. В фантастике, однако, идея межзвездной связи появилась на десять лет раньше, чем был обнаружен реальный астрономический аналог (рассказ П.Амнуэля "Летящий Орел", 1969 год).

Метод фантограмм - очень эффективное "оружие" в создании фантастических идей, в том числе и прогностического характера.

* * *

Анализ Регистра позволил Г.Альтову доказать на множестве примеров, что в фантастике эффективно "работают" методы, заимствованные из ТРИЗ.

Таков, например, метод ФОКАЛЬНЫХ ОБЪЕКТОВ, являющийся по существу модифицированным приемом ВЫНЕСЕНИЯ (ВНЕСЕНИЯ). Выбираем некий объект, называем его фокальным, и на этот объект, как в фокус собирающей линзы, проецируем свойства нескольких других объектов или явлений, подобранных произвольным образом.

Выберем для иллюстрации фокальный объект: подводная лодка. Случайные объекты: эрозия, кенгуру, компас.

Свойство компаса - стрелка всегда показывает на север. Перенос: подводная лодка способна двигаться только вдоль магнитных силовых линий или вдоль других избранных и неизменных направлений, например, по глубинным течениям. Безмоторное движение под водой совершается медленно, но зато это дешевый способ - в будущем такие своеобразные подводные "парусники" можно будет использовать для транспортировки грузов или для туризма.

Кенгуру - передвигается скачками, носит детенышей в сумке на животе. Пусть и наша подводная лодка передвигается скачками. Порт расположен на дне, куда пассажиров доставляют в лифте. Лодка совершает прыжок, отталкиваясь от дна,- до следующего порта.

Эрозия - процесс разрушения почвы. Пусть подводная лодка также разрушает воду во время движения (например, превращает в пар, как в "Тайне двух океанов" Г.Адамова, или разлагает комплексы молекул на составные части, как в рассказе В.Журавлевой "Снежный мост над пропастью").

Результат использования метода фокальных объектов: имеем подводную лодку, которая начинает движение, отталкиваясь от дна, как кенгуру, для того, чтобы набрать начальную скорость. При этом она попадает в подводное течение, где разворачивает "парус" и плывет, разлагая перед собой воду с целью уменьшения лобового сопротивления...

Аналогом метода фокальных объектов является метод АССОЦИАЦИЙ, при использовании которого свойствами обмениваются не отдельные объекты, а целые классы объектов или явлений.

Пример. Выберем классы объектов: животные и элементарные частицы. Свойства частиц - масса, заряд, импульс, момент вращения, четность и т.д. Частицы обладают и специфически квантовыми особенностями - например, для них справедлив туннельный эффект. Припишем животным свойство проникать сквозь силовые барьеры, например, проходить сквозь стены, но - не всегда, ведь и для частиц существует лишь не равная нулю вероятность такого перехода. Кроме того, животные намагничены и заряжены. Обмениваются друг с другом сигналами в виде вариаций магнитного поля или индуцированием на шкуре своего партнера электрических зарядов в определенном порядке...

Еще несколько методов РТВ, заимствованных из ТРИЗ и объясняющих появление немалого числа НФИ - МЕТОД ЗОЛОТОЙ РЫБКИ и ОПЕРАТОР РВС, а также ММЧ (МЕТОД МАЛЕНЬКИХ ЧЕЛОВЕЧКОВ). Последний метод есть по сути комбинация приемов УМЕНЬШЕНИЯ и РАЗДЕЛЕНИЯ. Описание этих методов читатель может найти в книгах Г.С.Альтшуллера "Творчество как точная наука", "Найти идею" и других трудах автора ТРИЗ.

* * *

И наконец, еще одна задача, которая также была решена с помощью Регистра - создание Патентного фонда фантастики (ПФФ). Собрав и классифицировав идеи Жюля Верна, Герберта Уэллса и Александра Беляева, Г.Альтов доказал еще в шестидесятых годах, что писатели-фантасты способны придумывать вполне патентоспособные идеи. Речь идет не только об идеях-изобретениях, но и об идеях-открытиях. Впоследствии, анализируя тысячи НФИ из Регистра, Вл.Гаков и П.Амнуэль начали собирать Патентный фонд фантастики (первые публикации ПФФ - в журнале "Изобретатель и рационализатор", 1981 год).

Предлагаю вниманию читателей несколько описаний фантастических патентов, отобранных из большого числа вполне патентоспособных НФИ.

* * *

ДЕСЯТЬ НФИ-ИЗОБРЕТЕНИЙ

Патентное описание 1
СПОСОБ ПОДЪЕМА ЗАТОНУВШИХ КОРАБЛЕЙ
Автор изобретения Г.Гуревич. Приоритет - 1951 год, повесть "Иней на пальмах".
Способ подъема затонувших кораблей, отличающийся тем, что, с целью увеличения быстроты подъема и величины поднимаемых судов, замораживают окружающую корабль воду, причем образуется айсберг, всплывающий вместе с вмороженным в него кораблем на поверхность.

Патентное описание 2
КОРАБЛЬ
Автор изобретения В.Журавлева. Приоритет - 1959 год, рассказ "Летящая черепаха".
Корабль для перемещения по воде, отличающийся тем, что, с целью увеличения грузоподъемности и удешевления производства, судно изготовляют из тяжелых сплавов и нагружают так, что оно становится тяжелее воды, причем при неподвижном положении и при малой скорости движения судно поддерживается на плаву с помощью надувных поплавков, которые при большой скорости движения судна убирают.

Патентное описание 3
СПОСОБ ДЛИТЕЛЬНОГО ПРЕБЫВАНИЯ ПОД ВОДОЙ
Автор изобретения В.Журавлева. Приоритет - 1959 год, рассказ "За двадцать минут до смерти".
Способ длительного пребывания под водой, отличающийся тем, что, с целью увеличения длительности пребывания, удешевления и упрощения аппаратуры, в организм водолаза вводят химические вещества, содержащие кислород и разлагающиеся со временем.

Патентное описание 4
СКАФАНДР
Автор изобретения В.Журавлева. Приоритет - 1960 год, рассказ "Человек, создавший Атлантиду".
Скафандр для погружения в воду, снабженный системой жизнеобеспечения, отличающийся тем, что, с целью увеличения глубины погружения и уменьшения опасности "кессонной болезни", скафандр изготовляют двухслойным, причем внутренний металлизированный слой заряжается электричеством, что заменяет внутреннее давление.

Патентное описание 5
СПОСОБ ПЕРЕДВИЖЕНИЯ ПО ПОВЕРХНОСТИ ВОДЫ
Автор изобретения В.Шефнер. Приоритет - 1963 год, повесть "Скромный гений".
Способ передвижения по поверхности воды (или иной жидкости), отличающийся тем, что, с целью упрощения передвижения, движение осуществляют с помощью обуви, покрытой составом, усиливающим поверхностное натяжение жидкости.

Патентное описание 6
СПОСОБ ПЕРЕДВИЖЕНИЯ В КОСМИЧЕСКОМ ПРОСТРАНСТВЕ
Автор изобретения Б.Красногорский. Приоритет - 1913 год, повесть "По волнам эфира".
Способ передвижения в космическом пространстве за счет давления света, отличающийся тем, что, с целью удешевления аппарата и уменьшения его массы, движение осуществляют давлением солнечных лучей на парус с большой рабочей поверхностью.

Патентное описание 7
СПОСОБ ПЕРЕДВИЖЕНИЯ В КОСМИЧЕСКОМ ПРОСТРАНСТВЕ
Автор изобретения Г.Альтов. Приоритет - 1966 год, рассказ "Ослик и аксиома".
Способ передвижения в космическом пространстве за счет давления света, отличающийся тем, что, с целью увеличения эффективности, движение осуществляют давлением света мощного лазера, установленного на Земле.

Патентное описание 8
КОСМИЧЕСКИЙ КОРАБЛЬ
Автор изобретения А.Богданов (Малиновский). Приоритет - 1908 год, роман "Красная звезда".
Применение энергии, высвобождаемой при распаде атомов, для движения космического корабля.

Патентное описание 9
СПОСОБ ДОБЫЧИ ПОЛЕЗНЫХ ИСКОПАЕМЫХ
Автор изобретения А.Беляев. Приоритет - 1936 год, роман "Звезда КЭЦ".
Способ добычи полезных ископаемых, например, золота и других ценных металлов, отличающийся тем, что, с целью максимального использования космических запасов руд, ископаемые добывают из астероидов и метеоров либо непосредственно в поясе астероидов, либо после того, как астероид (метеор) транспортируют на орбиту искусственного спутника Земли.

Патентное описание 10
СПОСОБ ПОЛУЧЕНИЯ ПРЕСНОЙ ВОДЫ
Патент Х. Автор изобретения А.Азимов. Приоритет - 1940 год, рассказ "Путь марсиан".
Способ получения пресной воды, отличающийся тем, что, с целью коренного решения проблемы водоснабжения, добычу воды производят из ледяных астероидов, метеоров и других небесных тел (расположенных, например, в кольцах Сатурна), причем объект может быть предварительно транспортирован в удобное для дальнейшей обработки место.

* * *

ЧЕТЫРЕ НФИ-ОТКРЫТИЯ

Патентное описание 1
ВСЕЛЕННАЯ ВНУТРИ АТОМА
Автор открытия Р.Кеннеди. Приоритет - 1912 год, роман "Тривселенная".
Впервые обнаружено физическое явление, заключающееся в том, что каждый атом представляет собой замкнутую вселенную со всеми свойствами той единственной Вселенной, которая открывается нам в мире звезд и галактик.

Как литературное произведение роман Р.Кеннеди не выдержал испытания временем, но его фантастическая идея живет. Тесная связь Вселенной и микрокосмоса проявляется в фантастике и таким образом: исследователь, воздействуя на микромир, тем самым меняет мегаструктуру Вселенной. Бомбардируя элементарные частицы, мы меняем свойства квазаров в нашем же мире...

Правомерность идеи далеко не очевидна, но ясно стремление фантастов создать своего рода "единую теорию мироздания", связывающую все структурные уровни материального мира. Такие модели описаны в рассказах В.Тивиса "Четвертое измерение" (1961 год), М.Емцева и Е.Парнова "Уравнение с Бледного Нептуна" (1964 год).

Есть аналогичные идеи и в науке. Академик М.А.Марков писал о том, что может существовать мир, находящийся, можно сказать, на грани исчезновения для внешнего наблюдателя. Воспринимается он как элементарная частица с массой в миллионную долю грамма. Такой объект (фридмон) может заключать в себе целую Метагалактику.

Патентное описание 2
ГИПЕРПРОСТРАНСТВО
Автор открытия Д.Кэмпбелл. Приоритет - 1934 год, роман "Ловушка".
Впервые обнаружено физическое явление, заключающееся в том, что существует измерение пространства (гиперпространство), передвигаясь в котором можно мгновенно преодолевать любые расстояния в пространстве трех измерений.

Теория относительности запрещает движение материальных тел быстрее света. Физики с этим постулатом примирились, фантасты - нет. После того, как на страницах фантастических произведений к звездам отправились первые экспедиции, авторы начали искать способ обойти постулат Эйнштейна. Первым это сделал Д.Кэмпбелл, снабдивший пространство еще одним измерением, в котором скорость света - вовсе не предел. Впоследствии фантасты писали о над-, под-, нуль- и прочих пространствах, ничем не отличавшихся от гиперпространства Кэмпбелла. Принцип один: использование для движения неких, пока неизвестных, измерений пространства.

В научной литературе последних десяти лет уже не редки работы, описывающие космос как структуру многомерную. Количество измерений пространства, вводимых авторами (не фантастами!), достигает десяти и более. Физическое четырехмерное пространство-время является как бы проекцией, доступной нашим органам чувств и приборам. Вопрос о том, является ли это многомерие лишь математической абстракцией, пока открыт. Не исключено, однако, что идея гиперпространства станет реальностью науки.

Для фантастики же открытие Д.Кэмпбелла сыграло положительную роль, и не только потому, что позволило героям фантастических произведений летать от звезды к звезде, как из Москвы в Париж. Исподволь воздействуя на читателя, литература приучает его к мысли о гораздо большей сложности мироздания, чем это предполагает обыденное сознание.

Патентное описание 3
УВЕЛИЧЕНИЕ СКОРОСТИ СВЕТА
Автор открытия Г.Альтов. Приоритет - 1960 год, рассказ "Полигон ’Звездная река’".
Впервые обнаружено физическое явление, заключающееся в том, что при определенном (например, импульсном) характере излучения света скорость его распространения может быть больше, чем 300 тысяч км/сек.

Патентное описание 4
ПРОИСХОЖДЕНИЕ ЗАКОНОВ ПРИРОДЫ
Автор открытия С.Лем. Приоритет - 1971 год, эссе "Новая космогония".
Впервые показано, что известные законы природы являются результатом совместной деятельности цивилизаций.

Разумеется, реальных доказательств искусственного происхождения законов природы не обнаружено, но фантастическое открытие С.Лема не противоречит и логике науки, нарушая разве что известный принцип "бритвы Оккама" - не умножать сущностей сверх необходимого. Фантаст логически последовательно создает ситуацию, настолько парадоксальную, что читатель не может не задуматься. У идей подобного класса сильна обратная связь с читателем - не только положительная, но и (чаще!) отрицательная, призывающая читателя активно возражать автору. Модели мира, подобные той, что создана С.Лемом, заставляют воображение активно работать.

* * *

С помощью Регистра решена была также задача оценки НФИ - сконструирована шкала "Фантазия-2". Не буду подробно останавливаться на описании шкалы. Отмечу лишь, что использование шкалы "Фантазия-2" совместно с методами РТВ позволяет улучшать уже существующие в фантастике идеи, а также конструировать идеи, принципиально новые.

К сожалению, работа над Регистром была практически прекращена в начале
восьмидесятых годов, а сам Регистр оставался недоступен для авторов-фантастов
и исследователей жанра. Остается надеяться, что публикация Регистра - через
два десятилетия после того, как этот гигантский труд был завершен - подвигнет
исследователей РТВ и НФ на создание более совершенной и полной версии,
отвечающей современным требованиям. В настоящее время существует (но, к
сожалению, больше года не пополняется) интернетовский Реестр фантастических
идей, создаваемый в память Г.С.Альтшуллера группой энтузиастов - Д.Гавриловым,
Р.Мухамеджановым, С.Дорофеевым и др. Авторы Реестра лишь слышали о
существовании альтовского Регистра и потому структура Реестра существенно
отличается от структуры Регистра. Хотелось бы надеяться, что этот или иной
авторский коллектив, объединив новые и прежние разработки, продолжит дело,
начатое Г.С.Альтшуллером около сорока лет назад.


\end{document}
