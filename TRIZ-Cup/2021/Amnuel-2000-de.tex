\documentclass[11pt,a4paper]{article}
\usepackage{od}
\usepackage[utf8]{inputenc}
\usepackage[main=german,russian]{babel}

\title{TRIZ und wissenschaftliche Fantasy-Literatur}
\author{P. Amnuel, Auswahl und Übersetzung Hans-Gert Gräbe}
\date{2000}

\begin{document}
\maketitle

\begin{quote}
  Der hier übersetzte Text ist ein Auszug aus dem Text
  \emph{\foreignlanguage{russian}{Регистр как Способ Познания}} von Pavel
  Amnuel aus dem Jahr 2000, wobei die Teile zur morphologischen Tabelle sowie
  die Parallelen zu Patentschriften nicht mit übernommen wurden. Der
  Schwerpunkt liegt auf der Darstellung von Altschullers \emph{Etagenschema}
  sowie weiteren Parallelen zwischen der Sujetplanung von wissenschaftlichen
  Fantasy-Erzählungen und der TRIZ.

  Quelle des Originals: \url{https://www.altshuller.ru/rtv/sf-registern.asp}
\end{quote}

\section*{Fazit}
Ich glaube nicht, dass selbst ein aufmerksamer und fleißiger Leser das
Register (Altschullers \emph{Register wissenschaftlich-fantastischer Ideen}
(WFI-Register)) von Anfang bis zum Ende durchgelesen hat. Diese Arbeit ist
nicht für ein solches „Lesen in einem Stück“ gedacht. Über die Ziele, die mit
der Erstellung des registers verfolgt wurden, habe ich bereits im Vorwort
geschrieben.  Nun, da der Leser versteht, welches Arbeitsvolumen
G.S. Altschuller absolviert hat, lassen Sie uns über die Ergebnisse sprechen.

Erstens wurde eine Methodik zum Entwerfen von WFI erstellt -- und umfassender:
es wurde eine Methodik der RTV, der \emph{Entwicklung schöpferischen
Vorstellungsvermögens}, entwickelt.

Zweitens ermöglichte uns die Analyse der im Register gesammelten Ideen und
Situationen die Konstruktion einer Bewertungsskala
\emph{Fantasy-2}\footnote{\url{https://www.altshuller.ru/rtv/rtv7.asp}} für
WFI.

Drittens wurde auf der Grundlage des Registers der Fantastik-Patentfundus
geschaffen.

Betrachten wir diese Ergebnisse genauer.

\begin{center}  * * * \end{center}

G. Altov hat das sogenannte \emph{Etagenschema} für die Konstruktion von WFI
erstellt. Dessen Kern ist wie folgt.

Wählen wir ein Objekt aus, dessen Entwicklung wir vorhersagen möchten. Zum
Beispiel ein Raumanzug. Und fragen wir uns: Zu welchem Zweck existiert er?
Ein Raumanzug ist erforderlich, um eine Person vor dem Einfluss des Weltraums
schützen: vor Vakuum, harter Strahlung ... Also, wir haben ein Objekt und
einen Zweck gewählt. Die erste Etage des Schemas ist die \emph{Verwendung des
  einzelnen Objekts} (in unserem Fall -- eines Raumanzugs). Dies ist natürlich
lange keine Fantastik mehr: Es reicht aus, sich an A. Leonow oder N. Armstrong
zu erinnern. Aber beachten Sie: Dies ist heute keine Fantastik mehr, aber vor
hundert Jahren war eine Geschichte darüber, wie ein Mensch einen Raumanzug
anzog und in den Kosmos ging, eine präzise Voraussicht!

In der zweiten Etage werden \emph{viele} Raumanzüge verwendet. Zum Beispiel
lassen sich Menschen im Kosmos nieder, schaffen „Städte im Äther“ wie sie vom
K.E. Ziolkovsky beschrieben werden. Aber was heißt „viel“?  Fünfhundert? Oder
fünfhunderttausend?  A. Belyaev schrieb im Buch „Der KEZ-Stern"
(\foreignlanguage{russian}{Звезда КЭЦ}) über eine Weltraumstadt, wo Hunderte
von Menschen leben. Im Buch „Andromedanebel“ lebt I. Efremov leben im Weltraum
Millionen von Menschen. Und wenn der Mensch die Natur auf der Erde besiegt und
gezwungen ist, in den Kosmos umzusiedeln, dann wird jeder von uns der Besitzer
eines persönlichen Raumanzugs sein. Oder sogar eines Dutzends -- ein Raumanzug
für die Arbeit, einer für den Spaziergang, einer für den Besuch eines
Naturschutzgebiets auf der Erde ... Übrigens, ein solcher Roman wurde noch
nicht geschrieben, eine völlig vorausschauende Idee wartet auf ihren Autor.
Möglich sind Varianten: sehr viele Raumanzüge, eine kleine Anzahl von
Raumanzügen ... Nehmen wir an, es kommt eine Zeit, wenn die Produktion von
Raumanzügen quantitativ begrenzt ist, die Produktion von Raumanzügen
eingestellt wird, wenn ihre Gesamtzahl beispielsweise fünfhundert (oder
fünfhunderttausend) erreicht.  Die fantastische Annahme führt zu
Handlungskollisionen (ein Raumanzug ist eine Rarität, um den Besitz werden
heftige Kämpfe geführt), was es Ihnen erlaubt, auf diesem imaginierten
Testgelände die eine oder andere Tendenz der echten Kosmonautik zu prüfen,
aber es erlaubt auch, etwas Neues in der Figur des Helden zu enthüllen.

Vor uns liegt die dritte Etage: das \emph{gleiche} Ziel erreichen, aber ohne
\emph{das Objekt zu benutzen} (in in diesem Fall -- den Raumanzug). Eine
Person ist vor dem Einfluss des Raumes geschützt, jedoch ohne Raumanzug. Wenn
auf den ersten beiden Etagen die Anzahl der Objekte zugenommen hat, gibt es
jetzt einen qualitativen Sprung (das ist das Schwierigste für alle
Wissenschaftler-Futuristen, hier geht der Fantasy-Autor voran!). Sie müssen
sich eine qualitativ neue Situation ausdenken, eine Erfindung vorhersagen oder
eine zukünftige Entdeckung. Die dritte Etage für das Objekt „Raumanzug“ ist
die Cyborgisation des Menschen, die Schaffung intelligenter Wesen, welche die
besten Eigenschaften von Menschen und Maschinen verbinden. Die Teile des
menschlichen Körpers, die künstlich sind, funktionieren besser als die, die
uns von Natur aus gegeben wurden, und werden in Zukunft ständig ersetzt.  Im
Kosmos kann man nicht atmen und zukünftigen Raumfahrern werden die Lungen
„amputiert“ und durch ein einfacheres Gerät ersetzt, das Sauerstoff in das
Blut pumpen kann.

Fantasyautoren waren die ersten, die eine solche Möglichkeit in der
menschlichen Evolution ausmachten. Einer der Vorbilder literarischer Cyborgs
(siehe das Register!) erschien 1911 in der Geschichte von D. England „Der Mann
mit dem Glasherz“. Ein Cyborg, der ein Raumschiff steuert, wird in G. Kuttners
Geschichte „Maskerade“ beschrieben. Ein Mann, ohne Raumanzug im freien
Weltraum oder auf einem fremden Planeten arbeitet, ist das Thema so
wundervoller Werke wie „Die Stadt“ von C. Simak (1944), „Nenne mich Joe“ von
P. Anderson (1957), „Der ferne Regenbogen“ von A. und B. Strugatski (1964)
usw.

Gehen wir noch höher -- auf die vierte Etage. Die Situation, in der die
\emph{Notwendigkeit entfällt}, das gesetzte Ziel überhaupt noch zu
erreichen. In unserem Beispiel ist dies eine Situation, in der es nicht mehr
erforderlich ist, Menschen vor dem Kosmos schützen, weil der Kosmos für den
Menschen nicht mehr schädlich ist. Das heißt, im Kosmos gibt es Luft zum
atmen.  Woher? Lesen Sie die Novelle „Das dritte Jahrtausend“ (1974) von
G. Altov noch einmal. Die Idee ist folgende: Man muss den Jupiter zerstäuben,
seinen Stoff in Staub und Gas umwandeln. Um die Sonne herum bildet sich eine
Gaskugel, in der auch die Umlaufbahn verläuft Erde. Keine Leere mehr! Von der
Erde zum Mond und zum Mars können Sie mit Düsenflugzeugen und sogar mit
... Ballons fliegen. Im Raum zwischen den Planeten ballen sich Wolken
zusammen, entladen sich Gewitter ... Wie gefällt Ihnen ein kosmischer
Regenbogen, ein sich Dutzende Millionen Kilometer ausdehnender siebenfarbigen
Bogen -- von der Venus bis zum Asteroidengürtel?

Natürlich sind die vorgestellten Ideen der dritten und vierten Etage
keineswegs die einzigen möglich für das Objekt „Raumanzug“. Jeder Autor kann
seine eigene Version einer Antwort auf die Frage der entsprechenden Etage
erstellen. Auf jeder der Etagen des betrachteten Schemas lassen sich viele
SciFi-Ideen platzieren.

Die Konstruktion des Etagenschemas ist auch deshalb gut, weil Ideen in nur
vier Klassen von Etagen verteilt sind.  Es gibt auch einen Nachteil -- die
Methode „funktioniert“ gut, wenn unbelebte Objekte gewählt werden -- möglichst
ein künstliches Objekt. Dann gibt es keine Schwierigkeiten bei der
Formulierung des Ziels, das mit diesem Objekt erreicht werden soll. Versuchen
Sie aber einmal, sich eine Idee für die dritte Etage für ein menschliches
Objekt einfallen zu lassen. Wir müssen dazu zuerst eine „einfache“ Frage
beantworten: Was ist der Zweck der menschlichen Existenz? Was ist der Sinn des
Lebens? ..

\begin{center} * * * \end{center}

Anfang der siebziger Jahre haben P. Amnuel und R. Leonidov unter Verwendung
der Klassifizierung von Ideen aus dem Register und von
TRIZ-Techniken\footnote{\url{https://www.altshuller.ru/triz/technique1.asp}},
eine andere Methode zur Konstruktion von WFI entwickelt: Konstrutionen nach
den TRIZ-Prinzipien. Die Analyse von WFI zeigte, dass jede solche Idee als
Ergebnis der Änderung einer bestimmten realistischen Idee (Erscheinung,
Objekt) mit Hilfe von diesem oder jenem \emph{TRIZ-Prinzip} erhalten werden
kann. Es wurde eine Liste von RTV-Prinzipen erstellt, die sich nur teilweise
mit den analogen TRIZ-Prinzipien überlappte.

Ich werde nur ein paar der RTV-Prinzipien vorstellen, um dem Leser klar zu
machen, worum es geht.  

... In der Bucht erschien ein schreckliches Raubtier, das ein Boot mit
Menschen in eine flache Plinse verwandeln konnte. Und was seltsam daran war:
Niemand hat dieses Monster je gesehen. So beginnt die Geschichte des
sowjetischen Science-Fiction-Autors Sever Gansovsky „Der Herr der Bucht“.  Es
stellte sich heraus, dass in der Bucht Milliarden von Mikroorganismen lebten,
die sich im Moment der Gefahr zu einem einzigen Wesen vereinigten, das in der
Lage, den Rücken eines Hais zu brechen. Verschwindet die Gefahr, so
verschwindet auch die Kreatur -- sie löst sich sofort auf in Milliarden von
Komponenten.  Versuche, ein solches Monster zu bekämpfen!

Gansovsky benutzte das TRIZ-Prinzip der \emph{Vereinigung}.

Ein sehr beliebtes TRIZ-Prinzip in der Science-Fiction ist das \emph{Ausführen
  des Gegenteils}. Fantastische Ideen, die mit dieser TRIZ-Technik erhalten
werden, sind spannend und paradox. Erinnern wir uns an die Geschichte von
William Tenn „Die Welt der Zukunft“ („Time in Advance“). Wie Sie wissen,
erhält jemand, der eine Person tötet, eine lange Haftstrafe, wenn er nicht gar
zum Tode verurteilt wird.  So wenigstens in unseren Tagen.  In der Welt der
Zukunft ist es umgekehrt. Jemand erscheint vor Gericht, erklärt, dass er
seinen Feind töten will und erhält dafür eine Haftstrafe. Nach deren Verbüßung
(für gute Führung -- die Hälfte der Frist) hat er das Recht, diesen Feind zu
finden und ihn zu töten. Stimmen Sie zu, eine nicht triviale Idee, eine
großartige Arbeit der Vorstellung, und wie viele psychologische Kollisionen!
Immerhin muss der Held der Geschichte nicht im Voraus bekannt geben, welchen
seiner Bekannten er nach seiner Rückkehr aus der Haft „meucheln“ will.
Dutzende von Menschen, mit denen er auf die eine oder andere Weise zusammen
war, verlieren ihre Ruhe -- wer von ihnen?..

Unter den Methoden zur Entwicklung der Vorstellungskraft steht das
TRIZ-Prinzip des \emph{Gegenteils} besonders da. Der Grund ist einfach:
Schließlich können Sie nicht nur Dinge, Phänomene oder Situationen auf den
Kopf stellen, sondern auch Methoden zur Entwicklung der Vorstellungskraft.
Statt zum Beispiel des TRIZ-Prinzips der \emph{Vereinigung} erhalten wir das
TRIZ-Prinzip der \emph{Zerlegung}. Erinnern wir uns an Lems Idee, Menschen
über eine Distanz zu transferieren. Zuerst wurde Professor Tarantoga in
einzelne Atome \emph{zerlegt} und dann an einer anderen Stelle diese Atome
wieder zum liebenswürdigen Professor \emph{zusammengesetzt}.

Die Analyse des Registers ergab, dass Science-Fiction-Autoren Techniken haben,
die von Erfindern nicht verwendet werden und die nicht in der Liste der
TRIZ-Techniken enthalten sind -- sie sind zu stark. Zum Beispiel: wenn Ihnen
eine Eigenschaft eines Objekts oder Phänomens völlig unveränderbar erscheint
-- verändern Sie diese. Dies ist die Technik, \emph{das Unveränderliche zu
  ändern}.

Beispiel: Astro-Engineering-Aktivitäten. Veränderung von Himmelskörpern, von
Asteroiden, Planeten und sogar Sternen und Galaxien.

Das Gebiet des Astroengineering („ändere das Unveränderliche“) umfasst zum
Beispiel Umbau des Klimas von Planeten -- vor allem von Mars und Venus. Im
Jahr 1961 hat Carl Sagan vorgeschlagen, in der Atmosphäre der Venus einfachste
Algen zu versprühen, die das Kohlendioxid in Sauerstoff umwandeln. Ebenso
wurde vorgeschlagen (der Autor des Projekts -- M.D. Nusinov), das Klima auf
dem Mars zu verändern.

In Wirklichkeit kamen diese beiden Ideen aus der Fantastik! Bereits in den
dreißiger Jahren, begannen die Helden des Roman „Die letzten und die ersten
Menschen“ von Olaf Stapledon, auf der Venus eine Sauerstoffatmosphäre
herzustellen. Später wendeten sich die Helden von P. Andersons Buch „The Big
Rain“ („Rebellion auf der Venus“) diesem Problem zu, wie auch in „Das
Plätschern der Sternenmeere“ (\foreignlanguage{russian}{Плеск звездных морей})
von E. Voiskunsky und I. Lukodyanov usw.

Eine andere Idee, die nach diesem TRIZ-Prinzip konstruiert ist:
Schwerkraftsteuerung. Wissenschaftler denken heute, dass dies unmöglich ist.
Aber hindert dies Science-Fiction-Autoren daran, dazu interessante Werke zu
schaffen? Herbert G. Wells hat im Roman „Die ersten Menschen auf dem Mond“ das
„Cavorit“ erfunden, womit man sich vom Schwerkraftfeld abschirmen kann.

In der Fantastik gibt es auch die Steuerung der Ausdehung von Galaxien („Der
Hafen der steinernen Stürme“ von G. Altov), Management der Prozesse der
Entstehung von Leben auf Planeten („Die große Trockenheit“
\foreignlanguage{russian}{«Великая сушь»} von V.  Rybakov), die Veränderung
der Weltkonstanten -- der Lichtgeschwindigkeit und der Planckschen Konstante
(„Alle Gesetze des Universums“, „Gelassenheit“, „Die Zeitbombe“
\foreignlanguage{russian}{«Все законы Вселенной», «Крутизна», «Бомба
  замедленного действия»} von P. Amnuel).

Hier noch eine Technik, die von Science-Fiction-Autoren verwendet wird: das
\emph{Herausnehmen}. In der TRIZ wird es als Technik der „Trennung einer
Funktion von Objekt“ bezeichnet.  Die Technik arbeitet wie folgt: Es ist vom
Objekt eine seiner Haupteigenschaften abzutrennen. Oder umgekehrt -- diesem
Objekt wird eine Eigenschaft eines völlig anderen Objekts zugewiesen.

Raumschiffe müssen Antriebe haben (schließlich ist dies ein Transportmittel)
und es gilt, Bedingungen für das Leben der Besatzung zu schaffen (im
Wesentlichen die Funktionen eines riesigen Raumanzugs). Und jetzt trennen wir
diese beiden Hauptfunktionen vom Raumschiff.  Wenn wir die Eigenschaft vom
Raumschiff trennen, dass Bedingungen für das Leben der Besatzung zu schaffen
sind, erhalten wir nur ein automatisches Schiff, das von der Besatzung auf der
Erde aus der Ferne gesteuert wird. Lange kam niemandem der Gedanke, dass man
von einem Raumschiff (oder von einem einfaches Schiff oder von einem U-Boot)
auch einen solchen integralen Bestandteil wie den Antrieb abtrennen kann.

Die Technik „Funktion vom Objekt entkoppeln“ besagt nicht, dass die Funktion
vollständig verschwindet. Sie wird einfach an einen anderen Ort gebracht: Das
Schiff fliegt im Weltraum, und sein Motor befindet sich auf der Erde.  1896
veröffentlichten die französischen Science-Fiction-Autoren Jacques Le Fort und
Antoine de Graffigny (?\footnote{\foreignlanguage{russian}{французские
    фантасты Жак Ле Фор и Антуан де Графиньи опубликовали повесть «Вокруг
    Солнца»}}) die Geschichte „Um die Sonne“. Der russische Physiker
P.N. Lebedev hat nur zwei Jahre nach der Veröffentlichung dieser Geschichte
seine Experimente begonnen, die nach einigen weiteren Jahren zur Entdeckung
des Lichtdrucks auf Festkörper führten. Und die Helden der Geschichte „Um die
Sonne“ haben auf der Erde einen riesigen Projektor aufgestellt, seinen Strahl
auf das Heck des Raumfahrzeugs gerichtet und der Lichtdruck ließ das Schiff in
den Kosmos fliegen. Fliegen auf der Spitze eines Lichtstrahls war 1896 für die
Wissenschaft der gleiche Unsinn wie heute ein perpetuum mobile ...

Mitte der fünfziger Jahre, als die ersten Rechenmaschinen pro Sekunde etwa
zwei- bis dreitausend Operationen ausführten, und in der UdSSR die Kybernetik
die korrupte Dirne des Imperialismus war, veröffentlichte I. Azimov die
Geschichte „Alle Sünden der Welt“, in der ein Supercomputer Informationen über
alles sammelt, was auf dem Planeten passiert.  Informationen über Personen
eingeschlossen. Der Science-Fiction-Autor nahm einen „normalen“ Computer und
wendete die Technik der \emph{Vergrößerung} an.

In der Science Fiction gibt es Hunderte interessanter Ideen zur Zukunft der
Kybernetik. Viele werden wahr werden. Der rumänische Schriftsteller Radu Nor
(„Das lebendige Licht“, 1959) schrieb über eine denkende Maschine von der
Größe eines Moleküls (Technik der \emph{Verkleinerung}). Stanislav Lem setzt
im Roman „Der Unbesiegbare“ eine Zivilisation von Mikrorobotern ein (Technik
der \emph{Verkleinerung}). Das ist die nächste Generation von Computern, ein
Problem, über das Wissenschaftler seit Anfang des 21. Jahrhunderts ernsthaft
nachdenken.

Eine andere Technik ist die \emph{Beschleunigung} (und dazu die
\emph{Verzögerung}): Ein Objekt oder ein Prozess werden ausgewählt und ihre
Aktivität so weit beschleunigt, dass eine neue Qualität entsteht. Der Held der
Erzählung „The New Accelerator“ von Herbert G. Wells trinkt ein bestimmtes
Präparat, und alle Prozesse der Lebenstätigkeit im Körper beschleunigen sich
um ein Vielfaches.  Das macht alles so schnell, dass die Welt um ihn herum
eingefroren zu sein scheint. Menschen bewegen sich langsam wie Schildkröten
oder Schnecken, langsam bewegen sie ihre Beine -- der Held der Geschichte
schafft es, einen ganzen Block weit zu gehen, bevor jemand einen Schritt
macht. Bald bemerkt er, dass seine Kleidung zu schwelen beginnt -- er bewegt
sich (ja tatsächlich!) so schnell, dass der Luftwiderstand den Körper auf hohe
Temperaturen aufheizt! Es scheint ihm, dass er mit seinen Finger langsam eine
Metallplatte berührt -- in der Tat passiert es aber so schnell, dass der Finger
brechen kann ...

Eine Kombination der Techniken \emph{Beschleunigung} und \emph{Ändern des
  Unveränderlichen}: Wenn Raumschiffe immer langsamer als mit
Lichtgeschwindigkeit fliegen werden (wie die Relativitätstheorie behauptet),
dann erfordert die Technik \emph{Beschleunigung} die Erhöhung der
Lichtgeschwindigkeit.  Die Idee, Licht im gepulsten Modus zu beschleunigen,
wird vom Helden von G. Altovs Geschichte „Das Polygon am Sternenfluss“
(\foreignlanguage{russian}{«Полигон 'Звездная река'»} -- 1960) vorgeschlagen.

Die Technik der \emph{Universalisierung} (\emph{Generalisierung}) ermöglicht
es, die Idee der Änderung der Lichtgeschwindigkeit noch fantastischer zu
entwickeln -- es geht um die Änderung aller Naturgesetze („Alle Gesetze des
Universums“, P. Amnuel, 1968).

Was sollte die nächste Idee sein, die mit dieser Technik produziert werden
sollte? Ist es etwa nicht klar: eine andere Verallgemeinerung: nicht nur der
Mensch kann die Naturgesetze ändern, sondern auch andere Zivilisationen haben
dies schon gelernt. 1971 veröffentlichte S. Lem den Aufsatz „Die neue
Kosmogonie“, in der er eine fantastische „Entdeckung“ machte: die uns
bekannten Naturgesetze, argumentierte der Science-Fiction-Autor, sind das
Ergebnis gemeinsamer Aktivitäten außerirdischer Zivilisationen! Der polnische
Science-Fiction-Autor verwendete die Technik \emph{„etwas künstlich machen“}.
Die Wissenschaft geht davon aus, dass die Naturgesetze eine natürliche
Eigenschaft der Materie sind? Machen wir sie künstlich. Und ein weiteres
wichtiges Prinzip des Fantasierens wird beachtet -- wir sagten, dass sich dies
vor allem das ändern soll, das sich anscheinend nicht ändern lässt?  Die
Gesetze Natur sind aus dieser Kategorie „unveränderlicher Objekte.  
\end{document}
