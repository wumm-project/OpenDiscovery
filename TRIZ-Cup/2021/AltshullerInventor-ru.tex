\documentclass[11pt,a4paper]{article}
\usepackage{od}
\usepackage[utf8]{inputenc}
\usepackage[russian]{babel}

\title{Вездеход на марсе}
\author{Г. Альтов}
\date{1989}

\begin{document}
\maketitle

\begin{quote}
  Выдержка из книги\\
  Г. Альтов.  И тут появился изобретатель.  Москва «Детская литература», 1989
\end{quote}

\section*{Задача 8. Вездеход на марсе}

В одном фантастическом рассказе описана экспедиция на Марс. Космический
корабль опустился в долину с очень неровной поверхностью: всюду холмы, ямы,
камни. Космонавты быстро снарядили вездеход — колесный, с большими надувными
шинами. Но на первом же крутом склоне вездеход опрокинулся набок.

И тут... Нет, к сожалению, в рассказе изобретатель не появился.

А как вы думаете: что бы он предложил?

Учтите, у космонавтов не было возможности переделывать вездеход.

Эту задачу тоже напечатали в «Пионерской правде». В большинстве писем был
такой ответ: «Под днищем вездехода подвесить груз. Центр тяжести станет ниже,
машина будет устойчивее». Не спешите выдвигать свою идею, давайте сначала
оценим чужие предложения. У нас теперь есть критерий для оценки:
преодолевается техническое противоречие или нет?

Груз, подвешенный под днищем машины, повысит ее устойчивость. Но одновременно
ухудшится проходимость: груз будет цепляться за выступы в почве, за камни.
Техническое противоречие!

Были и другие предложения:
\begin{itemize}[noitemsep]
\item выпустить воздух из шин, чтобы они просели наполовину;
\item снабдить вездеход дополнительной парой боковых колес;
\item экипажу высовываться из окон и дверей и держать равновесие, как это
  делают мотогонщики...
\end{itemize}
Нетрудно заметить, что в каждом из этих предложений выигрыш связан с
проигрышем. Просевшие наполовину шины резко замедлят движение вездехода.
Дополнительные колеса — серьезное усложнение конструкции, а мастерских на
Марсе нет. Заставлять космонавтов выполнять опасные акробатические трюки —
недопустимый риск...  Избежать противоречий так трудно, что автор одного из
писем признался: «Ничего не могу придумать. Пусть космонавты идут пешком...»

Представьте себе моряка, который не знает, что рифы и скалы надо обходить.
Примерно так выглядит изобретатель, не учитывающий, что нужно обязательно
устранять технические противоречия. Помните задачу об измерении давления газа
внутри лампы\footnote{Однажды директор электролампового завода собрал
  инженеров и показал пачку писем.  — Жалуются потребители, недовольны нашими
  лампами, — грустно сказал директор. — Надо повысить качество продукции.  Я
  думаю, все дело в том, что давление газа внутри готовой лампы иногда больше
  нормы, иногда меньше... Кто скажет, как измерить это давление?  — Очень
  просто, — поднялся один из инженеров. —Берем лампу, разбиваем и...  —
  Разбиваем?! — воскликнул директор.  — Можно для контроля разбивать одну
  лампу из ста, — не сдавался инженер.  — Проверять хотелось бы каждую лампу,
  — вздохнул директор. — Думайте, товарищи инженеры!  И тут появился
  изобретатель.  — Задача для школьников,— сказал он. — Откройте-ка
  учебник...  И он объяснил, в каком учебнике можно прочитать почти готовый
  ответ на эту задачу.}?  Идея разбивать лампы была запатентована, но
изобретения фактически не получилось: противоречие не было устранено. Чем
больше ламп мы разобьем, тем точнее будет проверка...  и тем больше получится
брака, лома!

Прежде чем сказать: «Я решил изобретательскую задачу!» — обязательно спросите
себя: «Какое противоречие я устранил?» Подвесить груз к вездеходу нетрудно, но
подвешивать надо как можно ниже, а чем ниже расположен груз, тем чаще он будет
задевать за камни и выступы. Попытка повысить устойчивость машины, не применяя
изобретательской хитрости, приводит к ухудшению проходимости машины: вездеход
перестает быть вездеходом...

Используем теперь такую хитрость: пусть груз будет расположен очень низко, у
самого грунта, но не снаружи, а внутри вездехода. Спрячем груз в...  колеса!
Поместим туда металлические шарики или круглые камни — пусть перекатываются...
Такое изобретение запатентовано в Японии для повышения устойчивости
автопогрузчиков, тягачей, автокранов. Запомните этот прием, он называется
«матрешка»: для экономии места можно расположить один предмет внутри другого.
Задача и ответ — два берега реки.  Попытка сразу угадать ответ — все равно что
попытка перепрыгнуть с берега на берег. Технические противоречия, приемы
образуют мост. Теория решения изобретательских задач — это, в сущности, наука
о том, как возводить незримые мосты, по которым мысль приходит к новым идеям.

Впрочем, противоречия и приемы правильнее сравнить с опорами моста.
Перепрыгнуть с опоры на опору тоже не так просто: нужна догадка, чтобы перейти
от задачи к противоречию и от противоречия к приему. Кроме опор необходимы
балки, соединяющие опоры, — вот тогда получится хороший мост, по которому
можно спокойно и уверенно, шаг за шагом, перейти от задачи к ответу.

О таком мосте мы еще поговорим.  Пока важно одно: изобретателю необходимо
находить и преодолевать технические противоречия. С этой простой идеи и
начинается теория решения изобретательских задач.

\end{document}
