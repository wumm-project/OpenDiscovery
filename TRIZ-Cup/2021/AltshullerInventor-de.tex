\documentclass[11pt,a4paper]{article}
\usepackage{od}
\usepackage[utf8]{inputenc}
\usepackage[main=german,russian]{babel}

\title{Der Marsrover}
\author{G. Altov}
\date{1989}

\begin{document}
\maketitle

\begin{quote}
  Auszug aus dem Buch\\ \foreignlanguage{russian}{Г. Альтов.  И тут появился
    изобретатель.}  (G. Altov: Und da erschien der Erfinder),
  \foreignlanguage{russian}{Москва «Детская литература»}, 1989.

  Übersetzt von Hans-Gert Gräbe, Leipzig
\end{quote}

\section*{Aufgabe 8. Der Marsrover}

In einer fantastischen Geschichte wird eine Expedition zum Mars beschrieben.
Das Raumschiff landete in ein Tal mit einer sehr unebenen Oberfläche: überall
Hügel, Gruben, Felsen. Die Kosmonauten luden schnell den Geländewagen ab --
auf Rädern, mit großen aufblasbaren Reifen. Doch am ersten Steilhang kippte
der Geländewagen zur Seite.

Und dann... Nein, leider ist der Erfinder in der Geschichte nicht aufgetaucht.

Was würde er deiner Meinung nach anbieten?

Beachte, dass die Kosmonauten keine Möglichkeit hatten, das Geländefahrzeug
umzubauen.

Diese Aufgabe wurde auch in der „Pionerskaya Prawda“\footnote{Dies war eine
  sowjetische Kinderzeitschrift} abgedruckt. In den meisten der Briefe lautete
die Antwort: „Hänge ein Gewicht unter den Boden des Geländewagens. Der
Schwerpunkt liegt tiefer liegen, das Auto wird stabiler“. Habe es nicht eilig
mit einer eigenen Idee, lass uns zunächst fremde Ideen bewerten. TRIZ liefert
ein Kriterium für die Bewertung: Wird ein technischer Widerspruch überwunden
oder nicht?

Eine unter dem Boden der Maschine aufgehängtes Gewicht erhöht deren
Stabilität.  Aber zur gleichen Zeit verschlechtert sich die Geländegängigkeit.
Das Gewicht wird an den Vorsprüngen im Boden und den Steinen hängen bleiben.
Technischer Widerspruch!

Es gab auch weitere Vorschläge:
\begin{itemize}[noitemsep]
\item Luft aus den Reifen lassen, damit das Fahrzeug um die Hälfte der Reifen
  einsinkt;
\item den Geländewagen mit einem zusätzlichen Paar Seitenräder ausstatten;
\item die Besatzung lehnt sich aus den Fenstern und Türen, um das
  Gleichgewicht zu halten, wie das Motorrad-Rennfahrer machen...
\end{itemize}
Es ist nicht schwer zu erkennen, dass bei jedem dieser Vorschläge Gewinne mit
Verlusten erkauft werden. Schlappe Reifen verlangsamen die Fahrt des
Geländewagens deutlich.  Zusätz\-liche Räder sind eine seriöse
Verkomplizierung der Konstruktion, und Werkstätten gibt es auf dem Mars nicht.
Und die Kosmonauten gefährliche akrobatische Tricks vollführen zu lassen ist
ein inakzeptables Risiko...  Widersprüche zu vermeiden ist so schwierig, dass
der Autor einer der Briefe gestand: „Mir fällt nichts ein. Lass die
Kosmonauten zu Fuß gehen...“

Stelle dir einen Seemann vor, der nicht weiß, dass Riffe und Felsen umrundet
werden müssen.  In einer ähnlichen Situation ist ein Erfinder, der nicht
bedenkt, dass man zur Lösung unbedingt technische Widersprüche lösen muss.
Denke an die Aufgabe der Gasdruckmessung in einer Lampe\footnote{Einmal rief
  der Direktors des Glühlampenwerks seine Ingenieure zusammen und zeigte ein
  Bündel von Briefen.  -- „Verbraucher beschweren sich, sind unzufrieden mit
  unseren Lampen“, sagte der Direktor traurig. -- „Wir müssen die
  Produktqualität verbessern.  Ich glaube, der Grund ist, dass der Gasdruck im
  Inneren der fertigen Lampe manchmal höher und manchmal niedriger ist...  Wer
  hat einen Vorschlag, wie dieser Druck zu messen ist?“ -- „Sehr einfach“,
  schlug einer der Ingenieure vor. -- „Wir nehmen die Lampe, zerschlagen sie
  und...“ — „Zerschlagen?!“ - rief der Direktor aus.  -- „Sie können eine von
  Hundert zur Kontrolle zerschlagen, gab der Ingenieur nicht auf“.  -- „Ich
  möchte jede Lampe überprüfen“, -- seufzte der Direktor. -- „Denken Sie nach,
  liebe Ingenieurskollegen!“ Da kam ein Erfinder des Wegs.  -- „Eine Aufgabe
  für Schulkinder“, sagte er. -- Öffnen Sie das Lehrbuch...  Und er erklärte,
  was für ein Lehrbuch man lesen muss, um eine fast fertige Antwort auf diese
  Aufgabe zu erhalten. (Auszug aus dem Buch, S. 7)}?  Die Idee, Lampen zu
zerschlagen, wurde patentiert, aber die Erfindungen hat eigentlich nicht
funktioniert: Der Widerspruch wurde nicht beseitigt. Je mehr Lampen wir
zerschlagen, desto genauer wird der Test...  und desto mehr Ausschuss und
Schrott wird produziert!

Bevor man sagen kann: „Ich habe ein erfinderisches Problem gelöst!“ müssen Sie
sich unbedingt selbst fragen: „Welchen Widerspruch habe ich beseitigt?“ Es ist
nicht schwer, ein Gewicht an einen Geländewagen zu hängen, und man muss es so
niedrig wie möglich aufhängen.  Aber je niedriger das Gewicht hängt, desto
häufiger wird es auf Felsen und Steine treffen. Der Versuch, die Stabilität
der Maschine ohne erfinderische Tricks zu erhöhen, führt zur Verschlechterung
der Geländegängigkeit des Wagens: Der Geländewagen hört auf, ein Geländewagen
zu sein...

Lasst uns einen solchen Trick anwenden: Das Gewicht muss sehr niedrig hängen,
am besten am Boden selbst, aber nicht \emph{außerhalb}, sondern
\emph{innerhalb} des Geländewagens. Wir stecken die Gewichte in die... Räder!
Legen wir Metallkugeln oder runde Steine hinein -- sie kullern in den Reifen
herum...  Diese Erfindung wurde in Japan patentiert, um die Stabilität zu von
Gabelstaplern, Traktoren und Autokränen zu erhöhen.  Das ist eine der
TRIZ-Techniken, sie heißt „Matrjoschka“ (oder „Schachtelpuppe“): Um Platz zu
sparen, kann man ein Objekt in ein anderes platzieren.  Die Aufgabe und die
Antwort sind die beiden Ufer des Flusses.  Der Versuch, die Antwort sofort zu
erraten, ist wie der Versuch, von Ufer zu Ufer zu springen. Technische
Widersprüche und Prinzipien bilden eine Brücke. Die Theorie der Lösung
erfinderischer Probleme ist im Wesentlichen die Wissenschaft darüber, wie man
unsichtbare Brücken baut, auf denen Gedanken zu neuen Ideen kommen.

Richtiger ist es jedoch, die Widersprüche und Prinzipien mit den Stützen der
Brücke zu vergleichen.  Es ist auch nicht so einfach, von einer Stütze zur
nächsten zu springen: Man braucht eine Ahnung, um von der Aufgabe zum
Widerspruch und vom Widerspruch zur Lösung zu gelangen. Abgesehen von den
Stützen sind Balken notwendig, die die Stützen verbinden, dann erhält man eine
gute Brücke, über die man sicher und souverän, Schritt für Schritt gehen kann,
von der Aufgabe zur Antwort.

Über diese Brücke werden wir noch sprechen.  Bisher ist eines wichtig: der
Erfinder muss technische Widersprüche finden und überwinden. Mit dieser
einfachen Idee und beginnt TRIZ -- die Theorie der Lösung erfinderischer
Probleme.
\enlargethispage{1em}

\end{document}
