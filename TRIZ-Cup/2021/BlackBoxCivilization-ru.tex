\documentclass[11pt,a4paper]{article}
\usepackage{od,wrapfig,array}
\usepackage[utf8]{inputenc}
\usepackage[russian]{babel}

\pagestyle{headings}
\title{Краткий анализ задачи о “черном ящике” цивилизации}
\setcounter{tocdepth}{2} 
\author{Рубин М.С.}
\date{2020}

\begin{document}
\maketitle

Задача о “черном ящике” цивилизации была предложена как генератор новых
исследовательских тем, источник Достойной Цели для исследователей.  

“Г.С. Альтшуллер и И.М. Верткин выделяют три яруса тем:
\begin{itemize}[noitemsep]
\item темы узкотехнические, узко-научные, узко-художественные, например,
  изобретение ракет;
\item темы общетехнические, общенаучные, общехудожественные, например,
  развитие космонавтики;
\item темы, затрагивающие цивилизацию в целом (социально-технические,
  социально-научные, социально-художественные и пр.), например, проектирование
  космической цивилизации или цивилизации с телевидением.
\end{itemize}
С исследовательской точки зрения эффективнее всего работать с темами третьего
яруса - на уровне общечеловеческих проблем. Здесь самый большой простор для
развития тем, наименьшая вероятность "застрять" на мелкотемье или оказаться в
ситуации острой конкуренции с другими исследованиями. В качестве примера мне
хотелось представить хотя бы одну проблему общечеловеческого уровня. Речь
пойдет о проблеме создания "черного ящика" на случай всемирной катастрофы.
("Как стать еретиком", Карелия, 1991, стр. 166-168).

Очень не многие из специалистов по ТРИЗ работали над предложенной задачей, еще
меньше публикаций на эту тему. Задача сложная и мы решили предложить некоторые
подходы к ней. Это лишь наш взгляд, и мы не претендуем на то, что это
единственный подход к задаче о “черном ящике” цивилизации.

Нам сразу хотелось бы отсечь очевидно слабые решения. Например: создание
глобальных систем сбора информации от каждого жителя Земли и размещение ее на
каком-нибудь спутнике Земли или Венеры. Не смотря, на очевидность
невыполнимости, не идеальности, подобные идеи все равно
высказываются. Исследовательская тема будет тогда интересной и сильной, когда
она будет направлена на разрешение противоречий, заложенных в задаче о “черном
ящике” цивилизации (например, информации должно быть очень много, чтобы
отразить все стороны цивилизации, и должно быть мало, чтобы ее легче было
сохранить и донести до потомков; или место хранения информации должно быть
близко, чтобы легче было размещать информацию, и должно быть далеко, чтобы
информация надежнее сохранялась при катастрофе), и максимально приближаться к
идеальному конечному результату (например, катастрофическая сила, разрушающая
цивилизацию САМА сохраняет ее для потомков).

Во время конференции школьников и студентов по ТРИЗ “ИКАРиада-2001” была
предложена морфологическая таблица для анализа задачи:

\begin{center}\small
  \begin{tabular}{|*{6}{>{\raggedright}p{1.9cm}|}c|}\hline
1. Какой объект & 1.1 Семья & 1.2. Дом & 1.3. Город & 1.4 Страна & 1.5. Весь
мир & \ldots \\\hline
2. Кому, куда & 2.1. Другим людям & 2.2. Другому государству & 2.3. Потомкам
через 50 (100, 1000) лет & 2.4. Иноземным цивилизациям & 2.5. Необитаемой
планете & \ldots \\\hline
3. Цель & 3.1. Сохранить информацию & 3.2. Избежать ошибок & 3.3. Сохранить
цивилизацию & 3.4. Возродить цивилизацию & 3.5. Расширить границы цивилизации
& \ldots \\\hline
4. Ресурсы & 4.1. Ресурсы сохраняемой системы & 4.2. Ресурсы уничтожающей силы
& 4.3. Ресурсы современной внешней среды & 4.4. Ресурсы среды, в которую
попадает “Черный ящик” & \ldots & \\\hline 
5. Причина катастрофы & 5.1 Естественные & 5.2. Социальные конфликты &
5.3. Природные катаклизмы & 5.4 Межличностные конфликты & 5.5. Террор и
преступления & \ldots \\\hline
  \end{tabular}
\end{center}
Эта таблица может быть использована для формулировки множества подзадач,
например, вариант (1.3-2.3-3.2-4.1-5.1) может быть сформулирован следующим
образом: как сохранить информацию о современном Петрозаводске для поколения
людей 22-го века в целях избегания повторения допущенных ошибок силами самого
города, если он будет изменяться в результате естественной эволюции. Уровень
формулированных задач и их остроту можно изменять по собственному вкусу.

Имеется очень широкий выбор направлений ведения дальнейших исследовательских
работ по теме “черного ящика цивилизации”. Можно, например, уточнить оси
морфологического ящика: их количество и содержание. Можно сделать подборку
конкретных задач, собранных в рамках общей темы “черного ящика
цивилизации”. Можно провести анализ каждой из подзадач: сформулировать
противоречия, ИКР, проанализировать ресурсы. Можно выстроить иерархию задач
темы “черного ящика цивилизации”. Можно собрать информацию о наиболее частых
причинах гибели цивилизаций. Очень плодотворным может стать анализ методов
восстановления информации, которые имеются в археологии и у
историков. Возможностей для ведения исследовательской работы, действительно
очень много.

Очень эффективным можно считать сбор различных картотек (см. статью М. Рубина
“Личные картотеки – фундамент творчества”, статья опубликована на сайте
ОТСМ-ТРИЗ-технологий  \url{http://www.triz.minsk.by/e/221001.htm} ) по теме
“черного ящика цивилизации”.

Можно, например, начать с картотеки "черных ящиков" - с самолетов разного
типа, океанских лайнеров или других систем. Любопытно выяснить характер
фиксируемой информации, основные проблемы и противоречия "черных ящиков",
сделать прогноз их развития. Может оказаться, например, что "Черный ящик
цивилизации" - логическое продолжение развития обычных "черных
ящиков". Попутно можно предложить новые идеи и решения в этой области.

Очень важный вопрос: что именно передавать потомкам или пришельцам о погибшей
цивилизации. Для ответа на него может оказаться другая картотека - о КЛЮЧЕВЫХ
НАХОДКАХ (памятников), помощью которых удалось раскрыть тайны погибших
цивилизаций.
\medskip

Приведу только один пример. Для изучения погибшей цивилизации майя очень важно было расшифровать письменность этого народа. Сделать это удалось, главным образом, с помощью двух документов:
\begin{itemize}
\item рукописи "Сообщение о делах в Юкатане" времен завоевания испанцами
  индейцев майя, где, в частности, приводился сокращенный алфавит майя с его
  "ОЗВУЧИВАНИЕМ" ИСПАНСКИМИ буквами;
\item записям древних мифов (книги Чилам Балам), сделанных в XVI веке
  ЛАТИНСКИМИ БУКВАМИ и отражающих язык майя времен начала нашей эры.
\end{itemize}
К этому, безусловно, необходимо добавить дошедшие до нас современные (живые)
языки майя и испанцев, а также сведения об испанском языке ХYI века.

Любопытно отметить, что "черные ящик" цивилизации майя дошел до нас в виде
бисистемы: в первую очередь это сами города древних майя с храмами, надписями
на них, картинками, рисунками, а также ключ к их объяснению - алфавит, о
котором мы только что упоминали.
\medskip

При самом трудном варианте (с Земли исчезает всякая жизнь) живой язык,
естественно, не сохранится. Так, например, произошло с шумерскими текстами: их
удалось полностью расшифровать, но никто не знает, как звучал этот
язык. Необходимо совмещение предмета или действия (показанных в различных
ситуациях) с его письменным и звуковым (речевым) изображением. Что-то вроде
телевизионных комиксов для изучения иностранных языков.

Пример "черного ящика" цивилизации древних майя наводит на интересную
мысль. Дело в том, что рукопись "Сообщение о делах в Юкатане" - основной,
наиболее полный и точный документ о древней цивилизации - была составлена
епископом-испанцем Диего де Ланда. Именно под его руководством тщательнейшим
образом было уничтожено почти все письменно и другое культурное наследие майя
(за эту свою деятельность де Ланда и получил звание епископа). Рукопись де
Ланды - своеобразный отчет перед начальством о проделанной "работе" по
истреблению ереси. Возникает любопытная гипотеза: причина, приводящая к
катастрофе (например, испанцы в лице Диего де Ланды) САМА создает "черный
ящик" (знаменитую рукопись епископа). Второй документ-ключ к разгадке текстов
майя - "Книги Чилам Балам", написанные на латинице - имеют примерно такой же
характер происхождения, как и рукопись де Ланды. Испанцы запрещали майя
пользоваться родной письменностью, и те стали записывать древние тексты на
дозволенной латинице. Все та же ситуация: испанцы, своими запретами, САМИ
вынудили майя создать документ о своей культуре, понятный для потомков. Еще
один пример, о котором мы уже вспоминали: извержение Везувия уничтожило
Помпеи, но оно же и сохранило для нас этот город, засыпав его толстым слоем
пепла.
\medskip

Итак, сила, приводящая к катастрофе, САМА создает "черный ящик" в момент
катастрофы. Что это - гипотеза, идеальный вариант решения проблемы или
закономерность образования "черных ящиков" цивилизаций? Ответить на этот
вопрос поможет картотека КЛЮЧЕВЫХ НАХОДОК к разгадке культуры погибших
цивилизаций: с помощью чего удалось расшифровать информацию о жизни людей. В
связи с этим можно вспомнить и о многочисленных примерах ключевых находок
древних животных: мамонтов, неандертальцев, динозавров.

Общеизвестна, например, находка очень хорошо сохранившегося динозавра из рода
аллозавров. Эта находка стала настолько важной для науки, что останки этого
динозавра назвали по имени – Большой Ал. Хорошо сохранился не только скелет,
но отпечаток сердца. Ученым удалось восстановить с абсолютной точностью
особенности поведения аллозавров, среду их обитания, скорость передвижения,
повадки. Удалось даже сделать жизнеописание Большого Ала: в каком возрасте,
какие травмы он получил, в каких ситуациях эти травмы и болезни могли
возникнуть, от чего умер этот динозавр, живший 45 миллионов лет назад. Уже
старый и больной Большой Ал отправился к реке во время засушливого
лета. Дождей не было, и вода все не приходила. Обессиленный, он ждал воду в
русле высохшей реки. Вскоре вода пришла, но Большой Ал уже умер. Ил обволок
тело и сохранил его для нас на многие миллионы лет. Река, убившая динозавра,
САМА же сохранила информацию о нем.
\medskip

Возможно, что с помощью одной только разрушающей цивилизацию силы не удастся
найти абсолютно надежный механизм создания в нужный момент "черного
ящика". Тогда возникает новая задача: как сделать, чтобы "черный ящик" был
полезен и необходим при нормальном течении жизни без катастроф. Это позволило
бы сделать весь проект создания "черного ящика" цивилизации гораздо дешевле,
повысило бы вероятность его реализации, обеспечило бы надежность его
функционирования в до катастрофический период.

Как и всякая сверхпроблема, проект "Черный ящик" дает целый спектр задач,
проблем и тем для исследования. Представлю еще одну из них, но прежде -
карточка из моей картотеки:

"Кто спасется в катастрофе".

"С момента вступления в ядерную эпоху перед сверхдержавами постоянно стоит
дилемма - чье выживание обеспечить в первую очередь в случае крупномасштабного
конфликта: населения страны или высших эшелонов власти? В самом начале своего
восьмилетнего пребывания у власти президент Рейган, отбросив предрассудки
"холодной войны" с ее ориентацией на гражданскую оборону, заявил, что в
ядерной войне можно победить, но лишь в том случае, если будет обеспечено
выживание высших гражданских и  военных руководителей. Процедура эвакуации
предусмотрена для представителей высших эшелонов власти, выживание которых
считается необходимым для "обеспечения непрерывности государственной власти" -
всего для более 1000 человек".  ("Комсомольская правда", 26.08.1989).

Аналогичные программы имеются не только в США, но и в других странах. В
результате реализации подобных проектов вполне возможна ситуация, при которой
после катастрофы на Земле останутся только "высшие эшелоны власти". Те ли это
люди, которые должны остаться и продолжить разумную жизнь на Земле? Они умеют
руководить государствами, которых не будет. Смогут ли эти люди выжить сами,
без своих подчиненных? Если современная цивилизация способна обеспечить
выживание только очень небольшой части человечества, то какие люди должны при
этом выжить? Какой социальный механизм должен обеспечить этот отбор? Чем
должны быть заняты люди во время вынужденной "отсидки", пока не восстановились
приемлемые условия на Земле? Удастся ли ответить на все эти вопросы без
решения задачи о “черном ящике цивилизации”.


\end{document}
