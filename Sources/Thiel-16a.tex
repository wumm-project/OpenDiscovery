\documentclass[12pt,a4paper]{article}
\usepackage{od}
\usepackage[utf8]{inputenc}

\title{Hegel, Altschuller, TRIZ\\[1em]\large Zehn Anmerkungen}

\author{Rainer Thiel, Storkow}
\date{Veröffentlicht am 25.09.2016 in LIFIS-Online.\\[4pt] DOI:
  \url{10.14625/thiel_20160925}} 

\begin{document}
\maketitle

\begin{abstract}\noindent
  Die enge Beziehung zwischen TRIZ und der Hegelschen Dialektik wurde schon in
  der DDR im Rahmen der Erfinderschulbewegung bemerkt.  Der Autor als einer
  der wichtigen Promotoren dieser Entwicklungen sowie Augen- und Zeitzeuge
  präsentiert in diesem Aufsatz einige Anmerkungen zu diesen Zusammenhängen.   
\end{abstract}

\paragraph{1)} 
Vermutlich hat der hochgebildete Altschuller schon in frühen Lebensjahren von
Hegel gewusst, denn Hegel war auch von den russischen Philosophen Mitte des
19. Jahrhunderts hoch geachtet worden, besonders von Alexander Iwanowitsch
Herzen (1812 – 1870). Von Herzen stammen die Worte: „Die Philosophie Hegels ist
die Algebra der Revolution, sie übt eine ungemein befreiende Wirkung auf die
Menschen aus …“ Zwei Sammelbände mit Schriften von Herzen wurden 1946 vom
philosophischen Institut der Akademie der Wissenschaften der UdSSR besorgt,
daraus wurde 1949 ein Band „Ausgewählte Werke“ in deutscher Übersetzung
abgeleitet. Darin konnte ich 30 Bezugnahmen auf Hegel notieren. So könnte das
auch von Altschuller getan worden sein. Aber den Namen „Hegel“ habe ich nur ein
einziges Mal bei Altschuller gefunden („Flügel für Ikarus“, S.~14) und auch
dort nur als Wiedergabe eines Hegel-Zitats durch einen Mathematiker. Doch Hegel
und Altschuller sind vom gleichen Geiste.

Vielleicht war Altschuller durch die landesweit verbreitete Schrift Stalins
über dialektischen und historischen Materialismus auf „Hegel“ gestoßen, doch
Hegel wurde von Stalin sehr abfällig behandelt wegen Hegels
nicht-materialistischer Philosophie. Und Stalin hat in einer
populär-philosophischen Lehrschrift die Dialektik mechanizistisch verballhornt.
Wie auch andere Geistes-Arbeiter und vor allem wie Schriftsteller der
Arbeiterbewegung wird Altschuller berührt gewesen sein davon, dass durch Stalin
das Muster „Widerspruch“ behandelt wurde, ohne dass Stalin das Muster
„Widerspruch“ in einer Weise behandelt hat, die vor Hegel und auch vor Marx
hätte bestehen können. Bezugnahmen auf sowjetische Philosophen habe ich bei
Altschuller auch nicht gefunden. Sie hätten ihm auch nicht geholfen.
Desgleichen fand ich bei Altschuller auch keine Bezugnahme auf Lenin, der 1915
im Schweizer Exil Hegel gründlich gelesen und in Annotationen begeistert
kommentiert hatte. („Philosophische Hefte“, LW Band 38) Paradox ist: Manche
Philosophen wissen von Lenins Annotationen, aber achten sie nicht. Altschuller
schien nichts gewusst zu haben von Lenins Hegel-Annotationen und dachte so, als
hätte er sie in groben Zügen gekannt.

\paragraph{2)} 
Wird über Dialektik gesprochen, dann treten in den Vordergrund einige Muster,
die bei Hegel eine Rolle gespielt haben: „Widerspruch“, „Werden“,
„Entwicklung“, „die Bewegung in den Dingen selbst“ und einige andere. Dem Hegel
werden in der populärphilosophischen Literatur auch Muster zugeschrieben wie
z.\,B.\ „Negation der Negation“, „These-Antithese-Synthese“, „Aufheben in dem
Doppelsinn von Überwinden und Aufbewahren“. Das ist verständlich. Man könnte
solche Muster als Annäherungen an die Dialektik bewerten, und wir haben sie als
geistesgeschichtliche Fakten zu respektieren. Sie können propädeutische Dienste
zum Verständnis Hegels leisten. Hegel selbst hat sie gelegentlich benutzt. Doch
er hat Vorbehalte, sogar Aversionen gegenüber solchen Mustern – von ihm auch
„Definitionen“ genannt – zum Ausdruck gebracht, z.\,B.\ in „Wissenschaft der
Logik“, notiert von mir nach der Ausgabe von 1934/1948 des Felix-Meiner
Verlages, Zweiter Teil, S.~253, 314, 329\,ff., 497\,ff.  Oft wird im
alltäglichen Sprachgebrauch das Wortpaar „Analyse – Synthese“ gebraucht. Hegels
Werk ist wie ein Geflecht von Analysen und Synthesen. Altschuller hat mit
seinem ARIZ systematisch zu Analysen und Synthesen angeregt und
technisch-ökonomische Entwicklungen demonstrativ ausgeführt. Doch ich fand bei
Altschuller keine Nennung sonstiger Muster der Dialektik.

\paragraph{3)} 
Fakt ist aber die Dialektik selbst, vermittelt in den Werken von Hegel und
Marx. Das ist mindestens ebenso zu respektieren wie die populären Annäherungen.
Altschuller – zumindest der junge Altschuller bis 1969/73 – hat in Werke von
Marx hineingeschaut und wichtige Bewandtnisse bemerkt, die zumindest in den
Bänden MEW 13, 15, 20 und 23 zu finden sind. Nur beim Muster „Widerspruch“, das
für Altschuller zentral ist wie für Hegel und Marx, entging Altschuller nicht
einer Unachtsamkeit: Zuweilen sprach er von „\emph{Beseitigung} eines
Widerspruchs“.  Aber so etwas wie „Beseitigen“ ist nicht typisch für den
Dialektiker Hegel, der stets zusammenhängende Entwicklungsprozesse ausgeführt
oder nachvollzogen hat.  Gerade dadurch ist sein Riesenwerk entstanden. Deshalb
ist von mir 1984 beim Übersetzen Altschullers Wort „Beseitigung“
stillschweigend durch „Überwindung“ oder auch „Lösung“ ersetzt worden. Und
überwiegend wird von Altschuller auch so gedacht.

\paragraph{4)} 
Bei wiederholtem Lesen Hegels fiel mir auf, besonders an seinem Werk
„Wissenschaft der Logik“, Erste Ausgabe 1812, Zweite Ausgabe 1831: Hegels
zwingende Logik, absolute Strenge der Gedankenführung, wohlverstandene
Konkretheit und unaufhörliches Weitertreiben der Gedanken. (Darin ist Hegel
manchem Mathematiker seelenverwandt, z.\,B.\ bei der Erweiterung von
Zahlen-Bereichen, um Gleichungen lösbar zu machen.) Hegel bewährte sich auch in
der kritischen Auseinandersetzung mit seinen berühmtesten, von ihm durchaus
hochgeachteten Vorgängern Platon, Kant und Fichte. Dazu gehört unter anderem
Hegels Kritik der mechanizistischen Raum-Auffassung von Newton und Kant. Das
wurde von Friedrich Engels gewürdigt. Von Hegels Newton- und Kant-Kritik hat
Einstein nichts gewusst, doch sie darf als philosophische Grundlage der
Theorien von Einstein verstanden werden\footnote{Siehe Rainer Thiel:
  Diplom-Arbeit 1956 „Philosophische Probleme der Speziellen
  Relativitätstheorie“, veröffentlicht unter dem Titel „Newton, Marx und
  Einstein“ in „Aufbau“ 1957, Nr.~5 und 6.}.
\enlargethispage{2em}


Noch viel bedeutsamer ist Hegels Kritik der Kategorien-Lehre von Kant, welche
u.\,a.\ den Kategorien „Quantität und Qualität“ gewidmet ist. Hegel analysiert
ausführlich die Verhältnisse von linearen Wandlungen und Wandlungen via
mathematischer „Potenzenverhältnisse“. Hegels Dialektik – das Verhältnis von
„Quantität“ und „Qualität“ betreffend – muss daher als philosophische Basis
heutiger Erkenntnisse gelten, die mit dem Stichwort „Nichtlinearität“
gekennzeichnet werden. Obendrein sind sie fundamental für jegliche
Revolutionstheorie in der Gesellschaft\footnote{Siehe dazu Rainer Thiel: „Die
  Allmählichkeit der Revolution – Blick in sieben Wissenschaften“, LIT-Verlag
  2000.  Ders.: „Allmähliche Revolution. Tabu der Linken. Zwei Arten Abstand
  vom Volk: Auf Wunder warten, und 'Gebt eure Stimme bei uns ab'“,
  Kai-Homilius-Verlag 2009.}.  Altschuller/Seljuzki haben in ihrem Buch „Flügel
für Ikarus“ 1980/83 ein Kapitel mit der Ironie überschrieben „Wenn die
Schornsteine in den Himmel wachsen“. Das entspricht einem fundamentalen Muster
in Hegels Philosophie der Nichtlinearität.

In Deutschland glauben Naturwissenschaftler, der wichtigste Philosoph sei Kant,
und Hegel sei verschwommen. Das ist verrückt, das haben sie sich einreden
lassen.  Hegel textet mit kurzen, klaren, wohlgeformten sprachlichen Sätzen, im
Unterschied zu Kant mit seinen Schachtelsätzen. Hegel beginnt stets mit der
Eigenschaft eines Objekts, die gemeinhin als primär oder als weithin anerkannt
gilt. Und diese Eigenschaft entwickelt Hegel, bis das primär ins Auge gefasste
Objekt (Semantem) in ein anderes Objekt (Semantem) oder in seinen Gegensatz
umschlägt. Auf diese Weise entwickelt Hegel die Vielfalt der geistigen Welt,
ohne Zuflucht suchen zu müssen in sprachlichen Sätzen mit dreißig Kommata.

Von Hegel als Philosoph und Weltbürger wird Dialektik entdeckt und
nachvollziehbar aufgeschrieben. Nichts von wegen Verschwommenheit oder
Wischiwaschi. Dialektik als exakte Wissenschaft, gerade durch peinlichste
Beachtung der Logik und deren Gebrauch. Dabei nutzt Hegel, dass vielen Wörtern
der Sprache – auch der Philosophensprache – \emph{mehrere Begriffe} zugeordnet
werden können. Meist geschieht das unbewusst vom Sprecher oder Schreiber, auch
unbemerkt vom Publikum. Das produziert Missverständnisse und Zerwürfnisse. Aber
Hegel deckt es auf. Hegel wechselt die Begriffe, die sich unter einer Worthülse
verbergen: Nicht unter der Hand, ohne sie zu bemerken, im Gegenteil: Hegel
nutzt die Mehrdeutigkeit, um Begriffe zu \emph{entwickeln} und diese
Entwicklung bis zum Umschlagen in einen Gegensatz oder in einen anderen Begriff
voranzutreiben.  Also \emph{Strenge und Vorantreiben\/}! Analog hat es auch
Altschuller versucht, auf seine Weise, allein schon durch Auseinandersetzung
mit unklaren Aufgabenstellungen. In einigen meiner eigenen kurzen Texte habe
ich begonnen zu zeigen, wie Hegel mit den philosophischen Abstraktionsprodukten
„Sein“ und „Nichts“ umgegangen ist. Ich hoffe darauf, in diesem Sinne auch
längere Texte zur Dialektik verfertigen zu können. Da werde ich auch an
Altschuller denken.

Ebenso wie Hegel denkt und schreibt Karl Marx. Vor allem ist Marx mit Hegel
verbunden im Verständnis von Hegels Einleitung in die Geschichte der
Philosophie: „Was wir sind, sind wir zugleich geschichtlich. … Der Besitz an
selbstbewußter Vernünftigkeit … ist nicht unmittelbar entstanden und nur aus
dem Boden der Gegenwart gewachsen, sondern es ist dies wesentlich in ihm, eine
Erbschaft und näher das Resultat der Arbeit, und zwar der Arbeit aller
vorhergegangenen Generationen des Menschengeschlechts. …“ (Hegel: Vorlesungen
über die Geschichte der Philosophie, Band I, Ausgabe 1940/1944, S.~12) Gerade
das wird von Hegel anhand der menschlichen Erkenntnis- und Geistesgeschichte
ausgeführt. Marx hingegen hat vor allem die Geschichte der Arbeit als Prozess
der materiellen Produktion im Auge: „Meine dialektische Methode ist der
Grundlage nach von der Hegelschen nicht nur verschieden, sondern ihr direktes
Gegenteil. Für Hegel ist der Denkprozess … der Demiurg des Wirklichen. … Bei
mir ist umgekehrt das Ideelle nichts andres als das im Menschenkopf umgesetzte
und übersetzte Materielle.“ Marx würdigte dennoch die Entdeckung der
geschichtlichen Rolle der Erkenntnis- und Geistesarbeit durch Hegel und fügt
hinzu: „Ich bekannte mich daher offen als Schüler jenes großen Denkers. …“ (MEW
23, S.~27).

\paragraph{5)} 
Ist nun genug gesagt zum Thema „Hegel und Altschuller“? Gibt es eine
Denk-Ebene, das Thema erneut aufzugreifen? Hätte sich auch Altschuller als
Schüler jenes großen Denkers empfinden können? Ist intellektuelle
Schwerstarbeit, die Hegel geleistet hat, ohne dessen – sagen wir mal – „Seele“
zu verstehen? Einer Seele, welcher Antrieb und Motivation entsprang, gerade
jene Schwerstarbeit zu leisten, deren Ausführung wir soeben zu begreifen
versuchten? Um solche Schwerstarbeit zu leisten, muss man auch Lust und Liebe
haben, und die Lust muss verschwistert sein mit dem Gefühl der Verantwortung
für die menschliche Gesellschaft, für die Bedürfnisse der Menschen. Ich sage:
Neugier und Vortrieb, Liebe und die Beziehung auf die ganze Menschengattung –
in diesem Sinne Revolution\footnote{Ich habe versucht, dieses Wesentlichste als
  Titel eines Buches anzudeuten mit den Worten: „Neugier, Liebe,
  Revolution“.}. Ich wage zu behaupten: Das war die Motivation, die Lust und
die gefühlte Verantwortung des Philosophen Hegel und des Genrich Saulowitsch
Altschuller. Das trieb Hegel und Altschuller. Es trieb sie nicht zur Errichtung
babylonischer Bauwerke aus Lehm und Ziegeln, zu Kathedralen und Kaufhäusern,
wohl aber zur Entwicklung menschlicher Geistesarbeit, soweit sie nicht
kriegerischen oder ausbeuterischen Zwecken oder einfach nur dem
wirtschaftlichen Wachstum dient.

Hegel und Altschuller seien paradigmatisch noch einmal gekennzeichnet. Zuerst:
In Hegels „Wissenschaft der Logik“, erster Teil, entwickelt Hegel ein Kapitel
unter der Überschrift „Das Fürsichsein“, nach 100 Seiten Entwicklung seiner
Begriffs-Dialektik. Nun im Kapitel „Das Fürsichsein“, das seinerseits
Begriffs-Dialektik auf 30 Druck-Seiten vorführt – darunter auch das
Begriffspaar „Attraktion und Repulsion“, findet man Worte des Humanisten Hegel,
welche die kapitalistische Geisteswelt kennzeichnen, z.\,B.\ die folgenden
Worte: „Das Eins und das Leere ist das Fürsichsein, das höchste qualitative
Insichsein, zur völligen Äußerlichkeit herabgesunken. … Weil es die Negation
alles Anderssein ist. … Für dessen absolute Sprödigkeit bleibt also alle
Bestimmung, Mannigfaltigkeit, Verknüpfung schlechthin äußerliche Beziehung. …
An den Atomen, dem Prinzip der höchsten Äußerlichkeit und damit der höchsten
Begrifflosigkeit, leidet die Physik in den Molekules, Partikeln ebenso als die
Staatswissenschaft, die von dem absoluten Willen der Individuen ausgeht.“
(S.~156\,ff.) An anderer Stelle schreibt Hegel: „Man muss, wenn von Freiheit
gesprochen wird, wohl achtgeben, ob es nicht eigentlich Privatinteressen sind,
von denen gesprochen wird.“ Also die Ausschließlichkeit von Privatinteressen
als Begriffslosigkeit und absolutes Fürsichsein (vgl.\ LW 38, S.~303). Solche
Worte Hegels – vor 200 Jahren geschrieben – erschüttern mich. Sie passen
übrigens auch auf den atomistisch beschränkten Geist jener bildungslosen,
staatsbürgerlich verführten, \emph{nur für sich seienden Personen}, die heute
über die Ausländer schimpfen.

Was hat nun Hegels Kapitel über das Fürsichsein mit Altschuller zu tun? Ich
meine, Altschuller nahm Leute aufs Korn, die sich schwer tun, das Fürsichsein
der Individuen ihrer Fachwelt zu transzendieren.

Nun hat Hegel aber das Fürsichsein begrifflich überwunden mit seiner
begrifflichen Entwicklungsdialektik, im letzten Kapitel seines 900-Seiten-Werks
„Wissenschaft der Logik“. Dort lassen sich die Worte herausgreifen: „Die
höchste zugeschärfteste Spitze ist die \emph{reine Persönlichkeit}, die allein
durch die absolute Dialektik, die ihre Natur ist, ebenso \emph{alles in sich
  befasst und hält}, weil sie sich zum Freiesten macht. …“ (Zweiter Teil,
S.~502). Also \emph{Mensch, der Natur und Mensch in sich befasst}. So hat das
Marx gesagt, ich habe es 1998 in Erinnerung gerufen, dem Mainstream entgegen,
unbeachtet von der PDS. Und Hegel schrieb auf der letzten Seite seiner
„Wissenschaft der Logik“ – da lag das Fürsichsein siebenhundert Seiten zurück:
„Die reine Idee, in welcher die Bestimmtheit oder Realität des Begriffes selbst
zum Begriffe erhoben ist, ist … absolute Befreiung.“ (Zweiter Teil, S.~505).

Was hat nun \emph{das} wieder mit Altschuller zu tun? Ich meine dreierlei.
Erstens: Hegel hat die Menschheit als Ganzes im Blick gehabt, unabhängig von
Hautfarbe weiß, schwarz, gelb, unabängig von gebildet oder ungebildet, Lehrer
oder Schüler.  Das ist ja auch die Philosophie von Marx, die es zu entdecken
gilt.  Dazu hat sich Altschuller nicht explizit geäußert. Doch es scheint ihm
selbstverständlich gewesen zu sein. Zweitens: Hegel hat als höchstes Ideal der
Geschichte die Befreiung des Menschen vermittels seiner Denkarbeit gesehen.
Auch das hat Altschuller nicht ausdrücklich gesagt, doch er hat der Denkarbeit
der Menschen – nicht nur der Techniker und Wirtschaftler – Wege des
vorantreibenden Denkens gebahnt. Und drittens: Oberflächlich betrachtet hatte
Hegel nur die Geschichte der Geistigkeit im Auge gehabt. Mit Blick auf den
philosophischen Idealismus Hegels hatte aber schon Lenin provoziert: „Ein
kluger Idealismus steht dem klugen Materialismus näher als dummer
Materialismus“ (LW 38, S.~263). Der Materialismus Stalins und vieler seiner
Nachfolger war durchsetzt von dummem Materialismus. Altschuller hat sich davon
nicht beeindrucken lassen, obwohl er als junger Mann von Stalins Schergen schon
mal ein Jahr lang im Gulag festgehalten wurde. Altschuller ist seinem Lande und
seinen russischen Mitbürgern stets treu geblieben.

Der junge Altschuller hat in der menschlichen Erkenntnisgeschichte ein neues
Kapitel eingeleitet, ein Kapitel, das den Problemen der materiellen Produktion
und deren Friedlichkeit entsprang. Dieses neue Kapitel kann zugleich auch der
geistigen Produktion in ihrer gesamtmenschlichen Dimension dienen, falls wir
Geistesarbeitern die interdisziplinäre Dimension der Erkenntnisse Altschullers
nahe bringen. Also gebührt Altschullern ein Platz auf dem Olymp, in der Nähe
Hegels.

\paragraph{6)} 
Nun zu Altschullers Mantra TRIZ. Ich gebe zu, dass sich das Label „TRIZ“ in der
Öffentlichkeit besser macht als der Name „Altschuller“. Es kommt aber nicht
darauf an, „Schöpfertum als exakte Wissenschaft“ („Twortschestwo kak totschnaja
nauka“) oder als TRIZ (Teoria reschenija isobretatelskich sadatsch – Theorie
der Lösung von Erfindungsaufgaben) darzubieten. Als „Theorie“ hatte Hegel sein
eigenes kolossales Werk nicht bezeichnet. Und wenn es darauf ankommt, zum
Problemlösen, also zur Kreativität zu befähigen, also die meist verborgenen
menschlichen Potenzen freizusetzen – auch dann scheint mir eine Theorie nicht
passend zu sein. Deshalb habe ich einer Altschuller-Übersetzung den Buchtitel
gegeben „Erfinden – Wege zur Lösung technischer Probleme“.

Hegel hätte auch beim Vorwort Altschullers zu dessen TRIZ-Buch die Stirn
gerunzelt. Altschuller hatte betont, es gälte die Kreativität „lenkbar“ zu
machen. Hegel dagegen hatte demonstriert, wie Logik und Gründlichkeit in der
aktiven Auseinandersetzung mit Vorgefundenem zur Weiterentwicklung führt. Das
war Hegels Art und nicht die Lenkbarkeit.

\paragraph{7)} 
War Altschuller übermütig geworden, als er ab 1979 sein Werk als „Theorie“
betitelte? Nämlich als „Theorie der Lösung von Erfindungsaufgaben“, also als
TRIZ? Vielleicht weil dieser Titel für manche Kunden besonders attraktiv ist?
Hätte sich Hegel im 19.~Jahrhundert von solchen Ambitionen leiten lassen? Von
Altschuller 1969/73 war ich hellauf begeistert. Das war der erste Akt meiner
Altschuller-Rezeption. Und auch 2016 finde ich bei erneutem Lesen: Altschullers
Denk-Art verträgt sich weitgehend mit Hegels Dialektik: Altschuller kolportiert
nicht plötzlich oder zufällig entstehende Probleme und Problemlösungen,
Altschuller bietet dem Leser \emph{Entwicklungen:} Entstehung, Entwicklung und
Lösung von dialektischen Widersprüchen, oftmals auch gegenseitige Kompensation
von Widerspruchs-Komponenten, allmähliches Umschlagen von quantitativen
Wandlungen in qualitative. Mehrmals konstatiert Altschuller die Insuffienz von
Patent-Formulierungen für das Verständnis von Widerspruchslösungen! Das alles
finde ich wieder bei erneuter Lektüre von Altschuller 1969/73 und Altschuller
1979/84.

Der zweite Akt meiner Altschuller-Rezeption nach 1974 begann mit Enttäuschung
über Altschullers Liste der ca. 40 Lösungs-Prinzipe. Auch heute noch halte ich
die meisten dieser Prinzipe für trivial, auch nicht anregend für gebildete
Techniker, höchstens für geeignet, nach Kreation gewichtiger Ideen noch einmal
hindurchzuscrollen, ob man etwas vergessen hat. An dieser meiner Bewertung der
35 oder 40 oder 50 Prinzipe halte ich fest. Ich meine auch, dass Altschuller
mit seinem WEPOL und mit seinen Standards nicht wesentlich darüber
hinausgegangen ist. Mich ärgert, dass sich Altschuller abfällig über den
Psychologen Carl Duncker geäußert, aber das großartige Pendel-Beispiel bei
Duncker nicht bemerkt hat. Dieses Paradigma der Spaltung eines unliebsamen
Phänomens hätte Hegel begeistert. Es könnte zu einem der Dialektik-Muster
avancieren in einer \emph{Propädeutik} zur Dialektik. Rindfleisch/Thiel haben
es 1988 in ihrem Lehrbrief zu ProHEAL beschrieben und
kommentiert\footnote{Siehe Hans-Jochen Rindfleisch, Rainer Thiel:
  Erfindungsmethodische Grundlagen.  Lehrmaterial zur
  Erfinderschule. Lehrbriefe 1 und 2, Kammer der Technik, Berlin 1988 und 1989.
  Abschnitte 1.5 und 1.9.}. Doch vorerst begann ich, für die Edition des neuen
Altschuller-Buches von 1979 – also Altschullers TRIZ-Buch – einen Verlag zu
gewinnen. Meine Frau übersetzte und ich redigierte. Da war es am wichtigsten,
die Worte und Sätze Altschullers genau wiederzugeben und meine eigenen
Hypothesen diszipliniert außen vor zu lassen. So entstand für mich ein
Zwiespalt in meiner Altschuller-Rezeption. Erst bei erneutem Studium des
TRIZ-Buches bemerkte ich, dass Altschuller die trivialen Prinzipe nicht nur
wiederholt, sondern auch hochgradig relativiert und transzendiert hat.
Altschuller hatte nämlich begonnen, triviale und nicht-triviale
Erfindungsaufgaben zu unterscheiden: Auf fünf Ebenen. Die Kriterien des
Unterscheidens bleiben leider etwas verschwommen. Doch von Altschuller wurden
sehr kritisch beleuchtet a) die übliche Art der Kennzeichnung technischer
Hypothesen in juristischen Patentschriften. Das hatte Altschuller auch schon
1969 getan, und dennoch hatte er den Millionenmassen von Patentschriften
mitunter kritiklos vertraut. Und b) wurde von Altschuller jetzt auch die Liste
seiner eigenen 40 Prinzipe kritisch kommentiert. Darüber hinausgehende
Lösungsprozesse wurden beispielhaft vorgeführt.

In diesem Sinne a) und b) scheint mir Altschullers TRIZ-Buch nicht in sich
konsistent zu sein, nicht ausgegoren. Hegel hätte die Liste der Prinzipe in
eine Anmerkung zur Dialektik verbannt, in eine Anmerkung zur Problem- und
Widerspruchs-Erkennung und Lösung, geeignet zum Hindurch-Scrollen \emph{nach}
dialektischer Erkennungs- und Lösungs-Arbeit, ob man etwas vergessen hat. Und
das Patentrecht hätte Hegel vor dem Hintergrund seiner Dialektik des
kapitalistischen „Fürsichseins“ ebenfalls in eine Anmerkung verbannt, um zu
unterstreichen: In der Dialektik geht es nicht um Neuigkeit und schon gar nicht
um Sicherung von Marktrechten. Es geht ums Erkennen und Lösen von Problemen,
von Entwicklungswidersprüchen. Auf dem Markt wird den Leuten zugerufen: „Kauft,
kauft, wir bieten Euch Neues.“ Stattdessen haben \emph{wir} die Kennzeichnung
von Erkenntnissen in ihrer Eigenschaft als weltweite Neuigkeit in
Patentschriften entschieden zu relativieren. Es geht nicht um Neuigkeit,
sondern um Lösung menschlicher und gesellschaftlicher Probleme\footnote{Im
  Patentrecht der DDR wurde „Erfinderische Leistung“ gefordert, „Neuigkeit“ wie
  bei einem Markenzeichen reichte nicht.}. Journalisten und Markt-Händler
können sich leisten, von Neuigkeit und Innovation zu sprechen. Aber wir haben
die Worte „Innovation“ einerseits und andererseits die Erkennung und Lösung von
Problemen, von Widersprüchen deutlich zu unterscheiden und zu bewerten. Mir
schwant, das hätte auch Hegel getan, mit aller begrifflichen Exaktheit. Das sei
bei aller Hochachtung vor Altschuller angemerkt. Weiterentwicklungen seines
Werkes nach 1993 kenne ich nicht. Der ARIZ von Altschuller scheint mir sehr
sinnvoll zu sein, auch kompatibel mit Hegel. Doch das Kürzel „ARIZ“ ist nicht
so knallig wie das Kürzel „TRIZ“.

\paragraph{8)} 
Das alles habe ich erzählt, um die Frage aufzuwerfen, wie Fachleute für die
Lösung von technisch-ökonomisch-ökologischen Problemen motiviert werden können
und in welchem Grade das auch in der Hoch- und Fachschulausbildung geschehen
müsste. Altschuller berichtet 1979/84 (Seite 148), dass in den Lehrgängen und
Erfinderschulen seiner sehr starken kollektiven Bewegung in der Sowjetunion bis
zu 240 Stunden Direktstudium veranschlagt werden. Einerseits ist erfreulich,
was Altschullers Kollegen und deren Aspiranten an Arbeit aufbringen. Aber
andererseits ist das in der fachlichen Ausbildung von Technikern an Hoch- und
Fachhochschulen nicht in dem nötigen Umfang möglich. Daraus leite ich die Frage
ab, ob es Wege gibt, \emph{alle} Fachleute und auch \emph{alle} angehenden
Fachleute in ihrer Ausbildung zu befähigen und zu motivieren,
technisch-ökonomisch-ökologische Widersprüche aus ihrer jeweiligen Situation
heraus zu lösen – selbstständig oder im Team. Ich werfe also die Frage auf, ob
ProHEAL aus Berlin – das Programm zum Herausarbeiten von Problemen und Lösungen
– dazu besser geeignet ist als TRIZ. Damit ist die Frage aufgeworfen, ob alle –
ich betone \emph{alle} – Techniker motiviert werden können, ihr
\emph{Fachwissen} für Erkennung und Lösung von Widersprüchen zu nutzen. Von mir
wird damit die Hypothese aufgestellt: Das ist notwendig und auch möglich mit
ProHEAL, in dessen Zentrum die sogenannte ABER-Matrix steht. Dazu braucht es
Erprobungen, viel Zeit und längere Fristen. Das haben bisher die Moderatoren –
die Trainer von Workshops wie den Erfinderschulen in der DDR und in aller Welt
– noch nicht gehabt. 

Quasi unterirdisch hat auch Altschuller in seinem TRIZ-Buch 1979/84 den Weg von
1969/73 umgekehrt: Fachleute erarbeiten zuerst echte Problemlösungen, und
anschließend suchen sie – also rückwirkend –, einige seiner 40 Prinzipe darin
zu finden. Auch in Altschuller 1969/73 findet man deutliche Berufung auf die
Rolle von Fachkenntnissen. Deshalb scheint mir, es kommt vor allem darauf an,
Fachleute zur Problemlösung zu motivieren. Der Text von Altschullers TRIZ-Buch
könnte wesentlich verkürzt und dadurch wesentlich handlicher ausgeführt werden.
Mein Freund Dietmar Zobel sieht das ähnlich. Zobel ist habilitierter
Naturwissenschaftler und Verdienter Erfinder, und er war jahrzehntelang Leiter
eines bedeutenden Produktions-Bereiches in der Industrie mit vielen
Hochschul-Absolventen. Ich verweise auf sein Buch „Kreatives Arbeiten“,
Expert-Verlag 2007, die Kapitel 3 und 4. Diese Kapitel, ca.\ 150 Druckseiten,
halte ich für ausreichend, um Fachleute zu Problemerkennung und Problemlösung
zu motivieren. Diese Kapitel sind auch schriftstellerisch hervorragend
gestaltet, sie können den Fachmann auch spät abends begeistern. Das ist es, was
wir wollen.

Geprüft werden könnte auch, ob das ProHEAL aus Berlin einen direkteren,
wirksameren Zugang zu Dialektik und Kreativität, zu Problemlösungen bietet als
TRIZ, zugleich auch mit viel weniger Druckseiten: Etwa ProHEAL mit seinem
Zentrum, der sogenannten ABER-Matrix. Ingenieure werden angeregt, ihr
\emph{Fachwissen} zu aktivieren, um \emph{die Entwicklung} zu analysieren. Und
das mit sehr kurzem Text: 20 Druckseiten plus 50 Seiten Zugaben. Auch ProHEAL
wurzelt im Werk von Altschuller. Prinzipe und Standards wie bei Altschuller
sind aber nur erwähnt zum Hindurch-Scrollen. Darin ist ProHEAL der Hegelschen
Dialektik näher. ProHEAL wendet sich an alle Ingenieure. Probleme gibt es
überall. Problemerkennung und -lösung stehen im Vordergrund. Erfinden ist
Mittel zum Zweck. Das gilt auch für die Leiter von Produktionsprozessen: Sie
empfinden viele Probleme und wünschen sie hinweg. Deshalb wünschte ich mir in
der DDR, dass Verdiente Erfinder zur Rechten der Chefs zu sitzen kommen.

Dietmar Zobel ist einer der seltenen Idealfälle, wo ein Produktionsleiter,
zugleich auch hochqualifizierter Fachmann, selbst zum Erfinder, sogar zum
Verdienten Erfinder geworden ist. Deshalb hat mein Freund Dietmar das ProHEAL
nicht selber benötigt. Leider hat er die Schriftfassung des ProHEAL nicht
annähernd zutreffend beschrieben. (Das betrifft vor allem die zentrale
ABER-Matrix und die Kürze des HEAL-Programms.) Das ändert nichts an unsrer
Freundschaft. 

\paragraph{9)} 
Hier stutze ich abermals. Hegel konnte sich die Entwicklung der menschlichen
Erkennt"|nis- und Begriffsentwicklung dank seiner überragenden Kenntnisse und
Fähigkeiten selber vor Augen führen. Doch der Ingenieur ist nur „fachlich“
ausgebildet, auf die Lösung von \emph{Aufgaben} orientiert und nicht für die
Lösung von Problemen, schon gar nicht für den Umgang mit dialektischen
Entwicklungs-Widersprüchen. Hierzu einige Impressionen.

\textbf{9.1} Der Nobelpreis-Träger Wilhelm Ostwald überspitzt in einem Text
1928: „Wie macht man den Fachmann unschädlich?“ Altschuller unterscheidet
zwischen „Neuerern“ und „konservativem“ Fachmann. (1969/73) Das kommt der Lage
viel näher. Altschuller schätzt mehrmals ausdrücklich den Fachmann
(u.a.\ 1969/73 S.~200, 236\,ff., 274\,ff., 291–294; 1979/84 S.~58, 146\,ff.).
Das ist dem Hegel kongenial.

\textbf{9.2} Ein Verdienter Erfinder der DDR, Ing. Karl Speicher, geb. 1925,
parteilos, hoch angesehen in seinem VEB und bei Leningrader Turbinen-Bauern,
hatte in den sechziger Jahren begonnen, dem Minister für Hoch- und
Fachschulwesen der DDR immer wieder vorzuschlagen: Ergänzung der fachlichen
Ausbildung durch fakultative Studien-Zirkel, wo Verdiente Erfinder den
Studenten vorführen, wie man Probleme erfinderisch lösen kann. Fünfundzwanzig
Jahre lang hat Karl Speicher vergeblich gerungen, bis die DDR untergegangen
war. Ich weiß, wie der Parteilose Karl Speicher darunter gelitten hat.

\textbf{9.3} In der DDR war die fachliche Ausbildung von Ingenieuren
hervorragend. Deshalb waren Ingenieure der DDR im Westen unsres Vaterlandes
hoch begehrt. Die fachliche Ausbildung betreffend hatte auch ich etliche
Impressionen, als Mitarbeiter des Ministeriums für Wissenschaft und Technik
sowie als Gruppenleiter im Institut für Hochschulbildung und als Dozent in der
Ausbildung von Patent-Ingenieuren:

\textbf{9.3.1} Vor Dutzenden von angehenden Patent-Ingenieuren fragte ich, was
vor hundert Jahren geschehen war, als sich bemerkbar machte, dass die Länge der
Uhren-Pendel nautischer Instrumente den ozeanischen Temperatur-Änderungen
folgte statt konstant zu bleiben. Ich sagte deutlich „vor hundert Jahren“ und
gab 5 Minuten Zeit zum Überlegen. Von ca. 120 diplomierten Ingenieuren –
Geburtsjahrgänge um 1950 – besann sich ein einziger, mal etwas von der
Pendelaufgabe des Psychologen Carl Duncker gehört zu haben. Alle anderen hatten
der Mehrheitsmeinung begeistert zugestimmt: Die nautischen Instrumente in einer
Thermo-Kajüte verpacken und deren Temperatur mit elektrischen Instrumenten
konstant halten. Und das vor hundert Jahren!

\textbf{9.3.2} Ca.\ zweihundert dieser Ingenieure hatten für ihre Abschluss-
oder Jahres-Arbeit ein Thema gewählt, das ich angeboten hatte. Ich hatte
gebeten: Bitte keine Zitate aus der Partei-Presse, bitte befassen Sie sich mit
Problemen der Ingenieure in Ihrem Betrieb. Doch in den Texten kam nichts von
Problemen zur Sprache. Nur sieben von ca.\ 200 Texte-Schreibern hatten sich
einigermaßen bemüht.

\textbf{9.3.3} Um in einem Berlin-bekannten Kombinat Probleme für eine
Erfinderschule zu gewinnen, sprach ich dort mit zwei Abteilungsleitern. Deren
Antwort war: „Wir haben keine Probleme, höchstens wenn die Putzfrau mal krank
ist. Dann nehmen wir selber den Besen.“ Trotzdem fand ich heraus, dass sie mit
einer Kunststoff-Folie ein Problem hatten. Da bat ich um ein Muster und zeigte
es einem Trainer-Kollegen unserer Erfinderschulen. Dieser Kollege fand zu Hause
die Lösung in einer halben Stunde.

\textbf{9.4} Im Institut für Hochschulbildung sollte ich mit meiner Gruppe
Grundlagen für einen Beschluss des Polit-Büros zur Ausbildung von Ingenieuren,
Naturwissenschaftlern und Ökonomen ausarbeiten. Ich selber ging sofort in
medias res. Doch meine Mitarbeiter – darunter drei ehemals hochrangige
Funktionsträger des Hochschulwesens – wollten das anders: Zwanzig Seiten
Einleitung zu unseren großen Erfolgen. Mein Institutsdirektor hatte mir kurz
zuvor bescheinigt: Thiel arbeitet wie ein Besessener. Aber jetzt eskalierte der
Streit. Der Wissenschaftliche Rat des Instituts wurde von mir angerufen, doch
alle Ratsmitglieder schwiegen, schwiegen, schwiegen. Auch der damalige
Partei-Sekretär hatte geschwiegen. Da wurde ich aus dem Institut
rausgeschmissen. Fünf Jahre später sagte mir der damalige Partei-Sekretär unter
vier Augen: „Rainer, Du warst der Einzige von uns, der es richtig gemacht hat.“
Noch ein Jahr später hatte ich ein Gespräch mit zwei abteilungsleitenden
Professoren im Zentral-Institut für sozialistische Wirtschaftsführung in
Berlin-Rahnsdorf. Von mir auf die Probleme angesprochen meinten sie: Ich solle
noch zehn Jahre warten, dann habe sich voll ausgewirkt – na was denn? Der
Politbüro-Beschluss zur Hochschulausbildung, dessentwegen ich gefeuert worden
war.

\textbf{9.5} Schon vor dem Gespräch in Rahnsdorf hatte ich zwei Mal eine
Denkschrift verfertigt, mitsamt empirischen Nachweisen in den Anlagen: für den
Chefideologen der SED, Prof. Kurt Hager. An ca. 80 Personen hatte ich Kopien
versandt. Aus dem Stab von Kurt Hager empfing ich die Info, mein Material sei
dem Minister für Hoch- und Fachschulwesen übergeben worden, und dieses werde
Mitgliedern seines Wissenschaftlichen Beirats zur Auswertung übergeben. Doch
der Beirat blieb untätig. Zwei oder drei Jahre später schrieb mir der
zuständige Abteilungsleiter aus Hagers Stab, er werde jetzt Leben in die Sache
bringen. Doch es war zu spät. Die DDR ging unter.

\textbf{9.6} Heutzutage soll die Hochschulausbildung durch den Bologna-Prozess
rationalisiert werden. In der Presse las ich gerade – ich zitiere nur die
schärfsten Worte: Es gab eine Zeit, in der die Freie Universität Berlin noch
keine in die Zukunft verlängerte Schulbank war und keine Dressuranstalt, die
sicherstellte dass der Trottelnachwuchs für die Konzerne nicht ausbleibt. Es
sei den „Arschgeigen“ gelungen, aus dem, was mal war, eine „Brutstätte des
Einverstandenseins“ zu machen. Am Ende wurde Goethe zitiert: „Ich finde nicht
die Spur von einem Geist, und alles ist Dressur.“

\paragraph{10)} 
Was kann aus alledem gefolgert werden für das Verhältnis von Altschuller,
TRIZ und Hegel? Altschuller hat wie Hegel ein neues Kapitel der menschlichen
Geistesgeschichte eröffnet. Der Techniker Genrich Saulowitsch Altschuller hat
sich auch in seinen wichtigsten schriftstellerischen Äußerungen, also auch im
Detail, als Dialektiker im Sinne Hegels erwiesen. Doch leider kann man sich in
TRIZ wegen der vielen Prinzipe und Standards auch verhaspeln. Eine solche
Gefahr besteht bei ProHEAL nicht, und dem Ingenieur wird zugetraut, sein
„fachliches“ Wissen – und sei es im Team – auch ohne zusätzliche Prinzipe und
Standards zum Problemlösen zu aktivieren.

Hegel hatte gefragt: „Womit muß der Anfang der Wissenschaft gemacht werden?“
Altschuller hatte das 1969/73 auf seine Weise getan. Mit seinem Buch 1979/84
hat er es bekräftigt. Wege zur Kreativität hat Altschuller gebahnt, auch unterm
Mantra „TRIZ“. Am Anfang war seine Tat. Wie würde sich Altschuller zweihundert
Jahre nach Hegels Vision von der Befreiung äußern? Wie wollen wir uns von den
Gefahren befreien, die unsere kosmische Heimat bedrohen?
\ccnotice
\end{document}
