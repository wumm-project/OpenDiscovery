\documentclass[11pt,a4paper]{article}
\usepackage{a4wide,url,enumitem}
\usepackage[utf8]{inputenc}
\usepackage[main=russian]{babel}

\parindent0pt
\parskip3pt

\title{Законы и закономерности развития систем} 

\author{Владимир Петров}

\date{2020}

\begin{document}
\maketitle

\begin{quote}
  Законы и закономерности развития систем. Книга 1. Бесплатный
  фрагмент\footnote{\url{https://ridero.ru/books/zakony_i_zakonomernosti_razvitiya_sistem/freeText}} 
\end{quote}
\section*{Введение}

Монография «Законы и закономерности развития систем» уникальна, так как
является самым полным описанием законов и закономерностей развития систем. Она
состоит из четырех книг. Монография представляет собой усовершенствование
книги «Законы развития систем». За это время автор изменил некоторые свои
взгляды на законы и закономерности. Кроме того, автор сделал монографию более
удобной для чтения, разделив ее на отдельные книги.

\textbf{Первая книга} — введение в монографию. Она описывает основные понятия
и определения, структуру законов и закономерностей развития систем, каждый
закон и закономерность, предназначение законов и закономерностей и методику
прогнозирования развития систем.

Эта книга как бы взгляд с птичьего полета на всю систему этих законов и
закономерностей. По этой книге вы сможете познакомиться не только с общей
структурой законов и закономерностей, но и с каждым из них. Однако в этой
книге не будут приведены примеры. Они будут приведены в последующих книгах. В
этой книге примеры приводятся только на понятия и определения, которые даются
в первой главе. Книга 1 — это своего рода реферат монографии.

\textbf{Вторая книга} описывает всеобщие законы развития систем (законы
диалектик, закономерность S-образного развития и закономерности развития
потребностей и изменения функций.

\textbf{Третья книга} посвящена законам и закономерностям построения и
эволюции систем.  Четвертая книга описывает закономерность изменения степени
управляемости и динамичности и прогнозирование развития систем. Практически
это вторая часть третей книги. Кроме того, имеются приложения.

Монография предназначена для широкого круга читателей, интересующихся или
занимающихся инновациями. В первую очередь она предназначена научным
работникам, инженерам и изобретателям, решающим творческие задачи. Она может
быть полезна преподавателям университетов, аспирантам и студентам, изучающим
теорию решения изобретательских задач (ТРИЗ), инженерное творчество, системный
подход и инновационный процесс, а также руководителям предприятий и
бизнесменам. 

Особый интерес книга может представлять для патентных поверенных.

\section*{Предисловие}

\begin{flushright}
  \begin{minipage}{.8\textwidth}\it
    Технические системы развиваются закономерно. Закономерности эти познаваемы,
    их можно использовать для сознательного совершенствования старых и создания
    новых технических систем, превратив процесс решения изобретательских задач в
    точную науку развития технических систем. Здесь и проходит граница между
    методами активизации перебора вариантов и современной теорией решения
    изобретательских задач (ТРИЗ). 

    Надо знать и использовать законы развития технических систем.
    \begin{flushright}
      Г. С. Альтшуллер
    \end{flushright}
  \end{minipage}
\end{flushright}
Законы развития технических систем представляют собой фундамент теории решения
изобретательских задач (ТРИЗ).

Первую систему законов развития технических систем предложил автор ТРИЗ
Г. С. Альтшуллер. В дальнейшем эту систему законов совершенствовали на только
Альтшуллер, но и его коллеги. История законов развития технических систем
описана в книге «История развития законов: ТРИЗ».

Данная монография — это усовершенствование книги «Законы развития систем».
Книга получала достаточно широкое распространение, из-за ее уникальности. Это
единственное самое полное изложение законов и закономерностей развития
систем. С такой подробностью законы еще не были изложены ни в одной книге.

Однако за время выхода этой книги автор изменил некоторые свои взгляды на
законы и закономерности. Кроме того, автор сделал новую монографию более
удобной для чтения, разделив ее на отдельные книги.

В целом законы и закономерности используются для:
\begin{itemize}[noitemsep]
  \item поиска инновационных решений;
  \item развития сильного (изобретательского, талантливого) мышления;
  \item прогнозирования развития систем.
\end{itemize}
\textbf{Книга 1} является введением в монографию. Она описывает основные
понятия и определения, структуру законов и закономерностей развития систем,
каждый закон и закономерность, предназначение законов и закономерностей и
методику прогнозирования развития систем.

Эта книга как бы взгляд с птичьего полета на всю систему этих законов и
закономерностей. По этой книге вы сможете познакомиться не только с общей
структурой законов и закономерностей, но и с каждым из них. Однако в этой
книге не будут приведены примеры. Они будут приведены в последующих книгах. В
этой книге примеры приводятся только на понятия и определения, которые даются
в первой главе и примеры к тенденции изменения полей, описанной в приложении.

Первая книга своего рода реферат всей монографии. Она может быть так же
использована как справочное пособие.

В остальных книгах будут детально описан каждый из законов и закономерностей,
методики и алгоритмы их применения для различных целей. Кроме того, каждый
закон и закономерность будут проиллюстрированы многочисленными примерами,
задачами и графическим материалом.

\textbf{Вторая книга} описывает всеобщие законы развития систем (законы
диалектик, закономерность S-образного развития и закономерности развития
потребностей и изменения функций.

\textbf{Третья книга} посвящена законам и закономерностям построения и
эволюции систем.

\textbf{Четвертая книга} описывает закономерность изменения степени
управляемости и динамичности и прогнозирование развития систем. Практически
это вторая часть третей книги. Кроме того, имеются приложения.

Нумерация глав в книгах будет сплошная.

Монография предназначена научным работникам, инженерам и изобретателям,
решающим творческие задачи. Она может быть полезна преподавателям
университетов, аспирантам и студентам, изучающим теорию решения
изобретательских задач (ТРИЗ), инженерное творчество, системный подход и
инновационный процесс.

\section*{Благодарности}

Я премного благодарен моему учителю, коллеге и другу Генриху Альтшуллеру,
прежде всего, за то, что он создал основу теории развития технических систем —
законы их развития, за то, что имел счастье общаться и обсуждать с ним разные
аспекты ТРИЗ и жизни и, в частности, некоторые материалы данной книги.

Очень многим я обязан Эсфирь Злотиной — моей жене и соратнику по ТРИЗ. Долгие
годы мы с ней совместно разрабатывали различные материалы по ТРИЗ, в том числе
обсуждали первоначальные материалы этой работы.

Хотелось бы выразить искреннюю благодарность своему другу и коллеге Борису
Голдовскому (Россия) за ценные советы и замечания, высказанные при составлении
книги, которые способствовали изменению моего мнения по некоторым аспектам,
описанных в этой книге.

\section*{Введение}

\textbf{Основа ТРИЗ} — законы развития технических систем. Они представляют
взаимосвязанную структуру законов, закономерностей и тенденций развития
техники.

Прежде чем рассматривать законы развития технических систем, ответим на часто
встречающиеся возражения: «Законов развития техники не может быть. Технику
развивают люди по своему желанию, это случайный процесс».

Безусловно, технику развивает человек.

Первые «изобретения» делал древний человек, используя природу. Для охоты ему
не хватало сил, и он прибегнул к помощи дубины, для обработки шкур он начал
применять острый камень и т. п. Так он начал удовлетворять свои первые
потребности. Эти «инструменты» ломались или не совсем удовлетворяли его, и он
совершенствовал их, а старые больше не использовались… Таким образом, даже в
те далекие времена действительность диктовала, какую технику следовало
оставить, а какой умереть. В дальнейшем эти условия становились все более
жесткими.

Жизнь технической системы зависит от очень многих факторов: среды, в которой
она работает, ее эргономических, экологических, экономических и прочих
характеристик.  На следующем этапе исправляют недостатки неудачной системы.
Кроме того, потребности человека постоянно растут. Для их удовлетворения
разрабатываются новые технические системы, которые конкурируют друг с другом.

Выживают только системы с наилучшими характеристиками. Таким образом,
осуществляется «естественный отбор» — процесс эволюции технических систем.
Этот процесс подобен естественному отбору в природе. Если проанализировать
историю развития конкретных систем, можно получить закономерности их развития,
а, обобщив закономерности — получить законы. Именно такую работу проделал
Генрих Альтшуллер, исследовав сотни тысяч патентов.

Подобный процесс свойственен и для других искусственных систем.

Из трех миров человеческого творчества — \textbf{науки, техники, искусства} —
наука первой лишилась ореола личностной исключительности. Она изучает
\textbf{объективные закономерности}, и путь ее развития предопределен.

В отличие от исследователей (людей науки), многие люди, развивающие технику
(изобретатели) даже не подозревают о существовании каких-либо закономерностей
в ее развитии.

Между тем, смысл творчества в науке и технике очень близок: \textbf{цель
  науки} — добыча знаний о свойствах материи, \textbf{цель техники} —
использование этих свойств для удовлетворения потребностей человека и
общества.

\section*{Глава 1. Понятия и определения}

\subsection*{1.1. Закон}

Приведем некоторые определения.

\textbf{Закон}, необходимое, существенное, устойчивое, повторяющееся отношение
между явлениями. Закон выражает связь между предметами, составными элементами
данного предмета, между свойствами вещей, а также между свойствами внутри
вещи.

Но не всякая связь есть закон. Связь может быть необходимой и случайной.
Закон — это \textbf{необходимая связь}. Он выражает существенную связь между
сосуществующими в пространстве вещами. Это закон функционирования.

Законы существуют \emph{объективно}, независимо от сознания людей.

\textbf{Закономерность}, обусловленность объективными законами; существование
и развитие соответственно законам.

В. П. Тугаринов дает следующее определение закона: 
\begin{quote}
  «Закон есть такая взаимосвязь между существенными свойствами или ступенями
  развития явлений объективного мира, которая имеет всеобщий и необходимый
  характер и проявляется в относительной устойчивости и повторяемости этой
  связи».

  «Понятие «закон» служит для обозначения существенной и необходимой, общей
  или всеобщей связи между предметами, явлениями, системами их сторонами или
  другими составляющими в процессе существования и развития. Эти связи и
  отношения объективны. Законы науки являются их отражением в человеческом
  сознании.

  Понятие «закономерность» отличается от закона по своему содержанию и
  принятому употреблению. Довольно часто, говоря о закономерности того или
  иного явления, подчеркивают тем самым только то обстоятельство, что данный
  процесс или данное явление не случайно, а подчинено действию определенного
  закона или совокупности законов. Последнее особенно характерно для
  закономерности, которая по своему содержанию шире закона и обозначает также
  совокупное действие ряда законов и его итоговый результат.

  Различие между законами и закономерностями, не исключающие, а
  подразумевающие частичное совпадение содержания этих понятий».
\end{quote}
История возникновения и формирования понятия закона подробно описана
Л. А. Друяновым. Кроме того, он выделяет две черты, присущие закону, а
описывает четыре (иерархия этих черт и выделение текста выполнены автором
статьи):
\begin{itemize}
\item \textbf{Существенная связь}. «Объективный закон… — это существенная
  связь явлений (или же сторон одного и того же явления). Объективный закон
  относится не к отдельному объекту, а к совокупности объектов, составляющих
  определенный класс, вид, множество, определяя характер их „поведения“
  (функционирования и развития) … Поскольку… в природе действуют существенные
  связи (объективные законы), ее поведение не является случайным, хаотичным;
  она функционирует и развивается закономерным образом и наряду с
  изменчивостью, ей присущи относительная устойчивость и гармоничность».
\item \textbf{Необходимость}. «…всякий объективный закон (закон природы) носит
  необходимый характер; закон, закономерная связь всегда является в тоже время
  необходимой связью, которая, в отличие от случайной связи, при наличии
  определенных условий неизбежно должна иметь место (произойти, наступить) …
  Следовательно, существенная закономерная связь (закон) является в то же
  время и необходимой связью. Другими словами, необходимость — это важнейшая
  черта закона, закономерности. Всякий закон природы представляет собой, таким
  образом, выражение необходимого характера существенных связей в объективном
  мире».
\item \textbf{Всеобщность}. «Другая важнейшая черта всякого объективного
  закона — его всеобщность. Любой закон природы присущ всем без исключения
  явлениям или объектам определенного типа или рода… Всеобщность — это,
  следовательно, вторая важнейшая черта объективных законов, законов природы.
  Поскольку всякий закон носит необходимый и всеобщий характер, поскольку он
  осуществляется всегда и везде, когда и где для этого имеются схожие объекты
  и соответствующие условия, постольку, следовательно, закономерные связи
  будут устойчивыми, стабильными, повторяющимися… Закон инвариантен
  относительно явлений».
\item \textbf{Повторяющийся характер}. «Легко видеть, какое значение имеет
  существование стабильности, повторяемости, порядка в природе для человека,
  для науки и практической деятельности людей. Если бы в природе ничего не
  повторялось и происходило всякий раз по-новому, ни человек, ни животные не
  могли бы приспособиться к окружающим условиям, стала бы невозможна
  целесообразная деятельность, научное познание, да и сама жизнь… Поскольку
  повторяемость, упорядоченность… составляют важную характеристику объективных
  законов, научные поиски закономерных связей в природе начинаются обычно с
  констатации повторяемости определенной стороны или свойства изучаемых
  объектов…  Следовательно, науку интересуют не любые повторяющиеся связи
  объектов, а лишь такие, которые носят в то же время существенный характер,
  т.е. ее интересуют существенные повторяющиеся связи».
\end{itemize}
«…можем определить объективный закон (закон природы) как существенную связь,
которая носит необходимый, всеобщий, повторяющийся (регулярный) характер».

Б. С. Украинцев сформулировал общие особенности объективных законов техники:
\begin{itemize}
\item \textbf{Целеосуществление — реализация потребностей}. «Все технические
  сооружения или устройства, а также их части, создаются целесообразно цели,
  то есть таким образом, чтобы, функционируя, они выполняли роль средства
  достижения цели человека. Поэтому все технические законы по своей сущности
  являются законы целеосуществления».
\item \textbf{Управляемость техники человеком}. «Законы (техники) объединяются
  принципом сопряжения возможностей техники с возможностями человека или иначе
  говоря, принципом управляемости техники человеком».
\item \textbf{Принцип технологичности}. «…новая конструкция должна быть такой,
  чтобы ее можно было изготовить при помощи существующих средств производства
  и на основе имеющихся навыков производства, как исходных моментов
  дальнейшего технического прогресса».
\item \textbf{Эффективное функционирование техники}. «Законы техники являются
  также законами эффективного функционирования технических средств достижения
  общественных и личных целей… Если общественная ценность трудовых,
  материальных и энергетических затрат на создание и функционирование техники
  превосходит общественную ценность результатов ее применения в качестве
  искусственного материального средства целеосуществления, то данная техники
  малоэффективна и общество нуждается в другой технике, удовлетворяющей
  требованиям и принципам эффективности техники».
\item \textbf{Соответствие экономическим возможностям общества}. «Законы
  техники имеют еще один общий момент, выражаемый принципом соответствия
  техники экономическим возможностям общества на данной ступени его развития».
\end{itemize}
А. И. Половинкин сформулировал требования, которым должны удовлетворять законы
техники:
\begin{itemize}
\item Формулировка закона техники должна быть по форме лаконичной, простой,
  изящной, а по содержанию отвечать данным выше определениям закона.
\item Формулировка закона техники должна быть обобщенной и отражать очень
  большое число известных и возможных факторов. Иначе говоря, закон должен
  допускать эмпирическую проверку на существующих или специально полученных
  факторах, имеющих количественную или качественную форму. При этом
  формулировка закона должна быть настолько четкой, что два человека,
  независимо подбирающие и обрабатывающие фактический материал, должны
  получить одинаковые результаты проверки.
\item Формулировка закона техники должна не только констатировать: «что?,
  где?, когда?» происходит (то есть упорядочить и сжато описать факты), но
  еще, по возможности, дать ответ на вопрос «почему?» так происходит. В связи
  с этим заметим, что в науке немало существовало и существует эмпирических
  законов, которые не отвечают на вопрос «почему?» или отвечают на него
  частично. И, по-видимому, почти нет научных законов (в виду локального
  характера их действия), которые отвечают на вопрос «почему?». На все вопросы
  обычно отвечает теория, опирающаяся на несколько законов.
\item Формулировка закона техники должна быть автономно независимой, то есть к
  законам будем относить такие обобщенные высказывания, которые не могут быть
  логически выведены из других законов техники. Выводимые обобщения будем
  относить к закономерностям техники.
\item Формулировка закона техники должна учитывать взаимосвязи: «техника —
  предмет труда», «человек — техника», «техника — природа», «техника —
  общество».
\item Формулировка закона техники должна иметь предсказательную функцию, то
  есть предсказывать новые неизвестные факты, которые могут быть более или
  менее очевидными, а иногда необычными, парадоксальными.
\item Формулировка всех законов техники должна иметь четко определенную единую
  понятийную основу.
\end{itemize}

\subsection*{1.2. Система}

\subsubsection*{1.2.1. Общие понятия}

В данной книге будем рассматривать законы и закономерности развития систем. В
связи с этим дадим определение системы и некоторых понятий, связанных с ней.

\textbf{Система} (от лат. греч. system, «составленный», целое, составленное из
частей; соединение) — множество \emph{элементов}, \emph{взаимосвязанных} и
\emph{взаимодействующих} между собой, которые образуют \emph{единое целое},
обладающее \emph{свойствами}, не присущими составляющим его элементам, взятым
в отдельности.

Такое свойство называют \textbf{системным эффектом} или
\textbf{эмерджентностью}.

\textbf{Эмерджентность} (от англ. emergent — возникающий, неожиданно
появляющийся) в теории систем — наличие у какой-либо системы особых свойств,
не присущих ее подсистемам и блокам, а также сумме элементов, не связанных
особыми системообразующими связями; несводимость свойств системы к сумме
свойств ее компонентов; синоним — «системный эффект».

Часто такое свойство так же называют \textbf{синергетическим эффектом} (от
греч. synergia — вместе действующий) — возрастание эффективности деятельности
в результате интеграции, слияния отдельных частей в единую систему за счет так
называемого системного эффекта.

Например, обмен вещами не приводит к синергетическому эффекту, так как их
остается тоже количество. Обмен идеями приводит к синергетическому эффекту,
так как в результате у одного человека идей становится больше.

\textbf{Синерг\'{и}я} (греч. synergia — сотрудничество, содействие, помощь,
соучастие, сообщничество; греч. syn — вместе; греч. ergon — дело, труд,
работа, действие) — суммирующий эффект взаимодействия двух или более факторов,
характеризующийся тем, что их действие существенно превосходит эффект каждого
отдельного компонента в виде их простой суммы

\subsubsection*{1.2.2. Дополнительные понятия}

\textbf{Целостность} — характеристика системы, выражающая автономность и
единство системы, противостоящей окружению. Она связана с функционированием
системы и присущими ей закономерностями развития.

Целостность не абсолютное, а относительное понятие, поскольку система имеет
множество связей с окружающими объектами и внешней средой и существует лишь в
единстве с ними.

\textbf{Свойство} — сторона (атрибут) системы. Оно определяет различие или
общность предмета с другими предметами.

Свойство обнаруживается в \emph{отношении} подсистем в системе, поэтому всякое
свойство относительно. Свойства существуют объективно, независимо от
человеческого сознания.

\textbf{Отношение} — взаимосвязь, взаимозависимость и соотношение элементов
системы. Это мысленное сопоставление различных объектов и их сторон.

\paragraph{Пример 1.1. Предложение (в языке).}
Предложение состоит из \emph{слов} и \emph{способа построения предложения —
  грамматики}. 

Ни один из этих элементов не обладает свойством выразить \emph{мысль}.
Соединенные в единую систему — предложение, приобрел новое свойство — мысль —
системный эффект.

Предложение — \emph{целостно}. Оно автономно и имеет свои закономерности
развития — развитие грамматики.

В предложении показана взаимосвязь отдельных слов, их \emph{свойства},
обнаруживаемые в их \emph{отношении} друг к другу.

\subsubsection*{1.2.3. Иерархия систем}

Системам свойственно понятие \textbf{иерархии}. 

\textbf{Иерархия систем}:
\begin{itemize}[noitemsep]
\item собственно система;
\item ее подсистемы;
\item надсистема;
\item внешняя среда.
\end{itemize}
\textbf{Подсистема} — составные части системы.

\textbf{Надсистема} — это объект, куда входит система в качестве подсистемы.

Иерархия может иметь более высокие ранги, например, наднадсистема и более
низкие ранги, например, подподсистема.

Наднадсистема — это объект, куда входит надсистема, а подподсистема — это
элементы, из которых состоит подсистема. Количество рангов может быть
достаточно большое.

\paragraph{Пример 1.2. Компьютер}
\begin{itemize}[noitemsep]
\item Система — персональный компьютер.
\item Подсистемы: системный блок и устройства ввода — вывода (например,
  клавиатура, мышь, монитор, принтер, сканер, камера и т. п.).
\item Подподсистемы системного блока — это процессор, материнская плата,
  видеокарта, оперативная память, жесткий диск, дисковод, звуковая карта,
  сетевая карта, блок питания и т. д.
\item Надсистема — компьютерные сети и т. д.
\item Наднадсистема — это всемирная паутина, Интернет.
\item Внешняя среда — это среда, в которой находится компьютер, например,
  помещение, воздух и т. д.
\end{itemize}

\paragraph{Пример 1.3. Телефон}
\begin{itemize}[noitemsep]
\item Система — телефон.
\item Подсистемы: микрофон и наушник, клавиатура, дисплей, память и т. п.
\item Подподсистемы — это элементы, из которых состоят микрофон и наушник,
  клавиатура, дисплей, память и т. д.
\item Надсистема — АТС, телефонные сети и т. д.
\item Наднадсистема АТС — это региональная и мировая телефонная сеть.
\item Внешняя среда — чаще всего — помещение и воздух.
\end{itemize}

\paragraph{Пример 1.4. Автомобиль}
\begin{itemize}[noitemsep]
\item Система — автомобиль.
\item Подсистемы: колеса, двигатель, бензобак, система управления и т. п.
\item Подподсистемы двигателя — это поршень и цилиндр, шатун, свеча, клапаны,
  коленчатый вал, картер и т. д.
\item Надсистема — дорожное движение, к которой относятся: дороги,
  автозаправочные станции, автостоянки, система управления движением, гаражи,
  ремонтные службы, заводы изготовители и т. д.
\item Наднадсистема — это региональная и мировая сеть дорожного движения.
\item Внешняя среда — открытое пространство и атмосферные явления.
\end{itemize}

\subsubsection*{1.2.4. Искусственные системы}

\textbf{Искусственные системы} также называют \textbf{антропогенные системы}.

\textbf{Антропогенная система} (от греч. anthropos — человек, genesis —
происхождение, становление развивающегося явления) — система, созданная в
результате сознательно направленной человеческой деятельности.

\paragraph{Пример 1.5. Антропогенные системы.}
Это широкий класс систем, созданных человеком: язык, понятия, мысли, знания,
наука, литература и искусство, социальные группы (племена, сообщества,
государства и т. д.), сельскохозяйственные системы, искусственно созданные
объекты фауны и флоры (генная инженерия, биотехнологии и т. п.), технические
системы и т. д.

Основное внимание будет уделено рассмотрению одного класса антропогенных
систем — \textbf{технических систем}.

\textbf{Техническая система (ТС)} — это \emph{система}, создающаяся с
конкретной \emph{целью} для удовлетворения определенной
\emph{потребности}. Она выполняет \emph{функцию}, осуществляя \emph{процесс},
основанный на определенном \emph{принципе действия}.

ТС имеет определенную \emph{структуру} и \emph{потоки}.

\emph{Примечание}. Техническая система может включать, как искусственные, так и
природные элементы.

В качестве примеров технических систем можно назвать: самолет, автомобиль,
кондиционер, телефон, телевизор, компьютер, Интернет и т. д.

\paragraph{Пример 1.6. Самолет.}
Самолет состоит из крыльев, фюзеляжа, двигателя, шасси и т. д.

Ни один из этих элементов не обладает свойством летать. Соединенные в единую
систему — самолет приобрел новое свойство — летать — системный эффект.

\paragraph{Пример 1.7. Телефон.}
Телефон состоит из микрофона, наушника, клавиатуры, дисплея, памяти и т. п.

Ни один из этих элементов не обладает свойством передавать звук на
расстояние. Соединенные в единую систему — телефон приобрел новое свойство —
передавать звук на расстояние — системный эффект.

\paragraph{Пример 1.8. Алгоритм.}
Алгоритм — это определенный порядок выполнения различных операций, приводящий
к конкретному результату.

Алгоритм состоит из отдельных операций, выполняемых в определенном порядке.

Каждая из операций и порядок их выполнения в отдельности не приведут к
необходимому результату. Соединенные в единую систему — алгоритм приобрел
новое свойство — конкретный результат — системный эффект.

\subsection*{1.3. Потребность}

\textbf{Потребность} — нужда в чем-либо, необходимом для поддержания
жизнедеятельности индивида, социальной группы, общества, внутренний побудитель
активности.

\subsection*{1.4. Принцип действия}

\textbf{Принцип действия} — это способ выполнения главной функции системы.

\subsection*{1.5. Функция}

\subsubsection*{1.5.1. Определение}

\textbf{Функция} (от лат. functio — совершение, исполнение) — процесс
воздействия субъекта на объект, имеющий определенный результат.

Кроме того, функцию определяют и как «внешнее проявление свойств какого-либо
объекта в данной системе отношений».

В дальнейшем будем использовать более краткую формулировку функции.

\textbf{Функция} — это действие \emph{субъекта} на \emph{объект}, приводящее к
определенному результату (рис. 1.1).
\begin{center}
  Рис. 1.1. Функция
\end{center}
Результатом действия может быть \emph{изменение} параметра объекта или его
\emph{сохранение}.

Функция записывается в виде \emph{глагола}.

\paragraph{Пример 1.9. Самолет.}
Самолет перевозит (перемещает) пассажиров. Самолет — субъект, перевозит —
функция, пассажиры — объект. Перевозить — это значит изменять объект.

\paragraph{Пример 1.10. Кофе.}
Чашка удерживает кофе. Чашка — субъект, удерживает — функция, кофе —
объект. Удерживать — это значит сохранять объект.

\paragraph{Пример 1.11. Компьютер.}
Компьютер обрабатывает информацию. Компьютер — субъект, обрабатывает —
функция, информация — объект. Обрабатывать — это значит изменять объект
(информацию).
\begin{center}
  Бесплатный фрагмент закончился.
\end{center}
\end{document}
