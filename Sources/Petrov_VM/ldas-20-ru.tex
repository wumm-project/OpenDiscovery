\documentclass[11pt,a4paper]{article}
\usepackage{a4wide,url,graphicx,wrapfig}
\usepackage[utf8]{inputenc}
\usepackage[main=russian,english]{babel}

\parindent0pt
\parskip3pt

\title{Закономерности развития искусственных систем} 

\author{В. Петров}

\date{TRIZ-Summit 2020}

\begin{document}
\maketitle
\begin{abstract}
  Статья посвящена уточнению системы законов и закономерностей развития искусственных
  систем и некоторых из закономерностей.

  \emph{Ключевые слова:} ТРИЗ, системный подход, законы и закономерности
  развития систем.
  
  The article is devoted to clarifying the system of laws and patterns of
  development of artificial systems and some of the patterns.

  \emph{Keywords:} TRIZ, systems approach, laws and patterns of system
  development.
\end{abstract}

\section*{1. Введение}
Первая система законов развития технических систем была описана
Г. Альтшуллером в [1]. Эта система была уточнена Б. Злотиным в [2]. Общая
история законов развития систем изложена в [3].

Видение автора на законы и закономерности развития систем изложено в
монографии [4].

В данной статье будет сделана попытка усовершенствовать систему законов,
изложенную в [4].

\section*{2. Вновь введения в понятия и систему законов}
\subsection*{2.1. Вновь введенные понятия}
Новая система законов и закономерностей разбита на \emph{обязательные} и
\emph{необязательные}. Обязательные будем называть \textbf{законами}, а
необязательные – \textbf{закономерностями}. Обязательные – это те, не
соблюдение которых приводит к неработоспособности системы. Необязательные –
это статистические закономерности, которые в определенных условиях могут и
должны соблюдаться, а при других условиях могут и не соблюдаться.

\subsubsection*{2.1.1. Законы}
К законам мы относим законы диалектики, закон увеличения степени системности,
законы построения системы и закон неравномерности развития системы.

\subsubsection*{2.1.2. Закономерности}
К закономерностям мы относим закономерности эволюции систем.

\subsection*{2.2. Вновь введения в структуре законов и закономерностей}
Изменения внесены только в структура законов построения систем.  К этой группе
относятся законы: соответствия, полноты и избыточности, проводимости и
минимального согласования.

Введен новый закон – \textbf{закон соответствия}. Ранее он рассматривался как
одно из необходимых условий работоспособности системы.

Закон полноты включает функциональную и структурную полноту и избыточность.
Структурная полнота и избыточность включает полноту и избыточность частей и связей
системы.

Вновь введена – \emph{полнота и избыточность связей}.

\subsection*{2.3. Вновь введения в закономерности}
Изменена только одна из закономерностей увеличения степени управляемости. К
общей тенденции увеличения степени управляемости добавлен переход центрального
управления к распределенному и самоорганизующемуся управлению.

\section*{3. Новая система законов и закономерностей}
Законы и закономерности развития систем могут быть:
\begin{itemize}
\item всеобщие – это универсальные законы, справедливые для любой системы
  независимо от ее природы, вследствие единства материального мира. Самые
  общие из них – законы диалектики и закономерность S-образного развития;
\item законы и закономерности развития систем, присущие для всех антропогенных
  систем;
\end{itemize}
Всеобщие законы и закономерности не изменялись по сравнению с [4],

\subsection*{3.1. Законы и закономерности развития систем}

Законы и закономерности развития систем предъявляют требования к построению и
развитию систем.

Общее направление развития систем идет в сторону увеличения степени
системности.

Системность – это свойство, заключающееся в согласовании всех
взаимодействующих объектов, включая окружающую среду.  Такое взаимодействие
должно быть полностью сбалансировано.

Объект будет выполнен системным тогда и только тогда, когда он отвечает
следующим системным требованиям.
\begin{itemize}
\item[1.] Система должна отвечать своему предназначению.
\item[2.] Система должна быть жизнеспособной.
\item[3.] Система не должна отрицательно влиять на расположенные рядом объекты
  и окружающую среду.
\item[4.] При построении системы необходимо учитывать закономерности ее развития.
Системные требования представляют собой составляющие закона увеличения
степени системности.
Закономерности развития систем можно разделить на две группы:
 законы построения систем (определяющие работоспособность системы);
 закономерности эволюции систем (определяющие развитие систем).
Закономерности построения систем должны обеспечивать требования
системности:
‒ предназначение;
‒ работоспособность.
Закономерности эволюции систем должны обеспечивать другие требования
системности:
‒
‒
‒
конкурентоспособность;
не влиять отрицательно на окружение;
учитывать закономерности развития систем.
3.2. Законы построения систем
Системное требование предназначение осуществляется с помощью закона
соответствия. Этот закон говорит о необходимости соблюдения соответствия структуры и
главной функции системы. Структура системы должна обеспечивать выполнение главной
функции системы. Структура обеспечивает необходимый набор частей, связей и
взаимодействий между ними. Связи обеспечивают единство системы и возможность
прохода потоков.
Системное требование работоспособности определяется законами полноты и
избыточности, проводимости и минимального согласования.
3.2.1. Закон полноты и избыточности
Закон полноты и избыточности включает функциональную и структурную полноту
и избыточность.
Закон функциональной полноты определяется набором минимально необходимых
основных функций, а закон функциональной избыточности – набором основных,
вспомогательных и дополнительных функций для обеспечения работоспособности
(выполнения) главной функции системы.
Закон структурной полноты определяет минимально необходимый набор частей и
связей системы, закон структурной избыточности определяет набор дополнительных
частей и связей для обеспечения работоспособности системы.
Минимально необходимый набор элементов системы включает:

рабочий орган;

источник и преобразователь вещества, энергии и информации;

связи;

система управления.Минимально необходимый набор связей системы включает необходимые связи
между минимально необходимыми элементами.
Избыточность – это закономерность, по которой приблизительно 20% функций,
элементов и связей системы выполняют около 80% работы.
При создании работоспособной системы нужно учитывать, что для выполнения
какой-либо работы, кроме основных элементов и связей (выполняющих главную функцию),
необходимо еще приблизительно 80% вспомогательных, причем они, как правило,
выполняют только 20% основной работы. Учитывая это, следует предусмотреть
лишний расход вещества, энергии и информации (приблизительно 20% на обеспечение
главной функции и 80% основных и вспомогательных).
В общем виде закономерность избыточности формулируется как «20% усилий дают
80% результата, а остальные 80% усилий — лишь 20% результата» 1 .
Избыточность особо велика, когда к системе предъявляются повышенные требования.
Это наиболее характерно для систем безопасности и спасательных средств,
медицинского оборудования, военной техники, сложных научных исследований,
спортивного оборудования, предметов роскоши, массовых праздников и т. п. Все они, как
правило, имеют средства дублирования, значительные запасы (мощности, энергии,
провиантов, медицинских препаратов, боеприпасов и т. п.) или «излишества», роскошь.
3.2.2. Закон проводимости потоков
Вещество, энергия и информация должны проходить от источника потока до
требуемого элемента, совершая необходимые преобразования и выполняя
соответствующие полезные функции.
Создание правильных потоков обеспечивает необходимую функциональность и
работоспособность системы. Отсутствие хотя бы одного жизненно-важного потока
делает систему не работоспособной.
Поток может быть:
‒ вещества;
‒ энергии;
‒ информации.
Поток вещества обеспечивает транспортировку вещества в различных агрегатных
состояниях (например, в твердом, гелеобразном, жидком и газообразном) или объектов.
Транспортировка веществ может осуществляться, например, по трубопроводам, с
помощью конвейерной (транспортерной) ленты и т. п., а объектов с помощью
транспортных средств, например, по железной дороге, с помощью автотранспорта, судов,
самолетов, эскалаторов, транспортеров и т. д.
Энергетический поток доставляет энергию от источника к требуемому элементу.
Поток может, например, доставлять механическую, электрическую, оптическую,
химическую, другие виды энергии, различные излучения и т. д.
Информационный поток обеспечивает проход информации от источника к
требуемым элементам, например, от системы управления к органам управления и от них к
системе управления. Информационный поток может осуществляться с помощью,
1
Закон Парето – материал из Википедии.например, проводов, по которым осуществляется передача информации, контроль и
управление и всех видов беспроводной связи и т. д. Они могут распространяться
различными путями: через печатные материалы, Интернет, радио и телевидение и т. д.
Носителями информации является вещество и/или поле (энергия).
3.2.2. Закон минимального согласования
Внешнее согласование:
‒ Согласование потребности и главной функции;
‒ Согласование главной функции и принципа действия;
‒ Согласование принципа действия и рабочего органа (рабочий орган должен
обеспечить главную функцию).
Внутреннее согласование (минимальное согласование):
‒ Минимальное согласование преобразователя с рабочим органом;
‒ Минимальное согласование источника и преобразователя вещества, энергии и
информации между собой и с рабочим органом и системой управления;
‒ Минимальное согласование системы управления с рабочим органом,
источником и преобразователем вещества, энергии и информации;
‒ Согласование всех связей и потоков;
‒ Минимальное согласование всех параметров системы.
3.3. Закономерности эволюции систем
В [4] эти закономерности назывались законами. Так как эти законы носят
статистический и не обязательный характер, поэтому мы их переименовали в
закономерности. Они не претерпели изменений.
 Закономерность изменения степени идеальности;
 Закономерность изменения степени управляемости и динамичности;
 Закономерность изменения степени согласования – рассогласования;
 Закономерность перехода в над- и подсистему;
 Закономерность перехода на микро- и макроуровень;
 Закономерности использования пространства.
3.4. Изменения в закономерности изменения степени управляемости и
динамичности
Изменения касаются только закономерности изменения степени управляемости.
3.4.1. Общая тенденция увеличения степени управляемости
Общая тенденция увеличения степени управляемости – это переходы:
– от неуправляемой к управляемой системе;
– неавтоматического (ручного) управления к автоматическому;– проводного управления к беспроводному;
– непосредственного управления к дистанционному;
– от центрального управления к распределенному и самоорганизующемуся
управлению (вновь введенная тенденция).
Тенденция перехода от центрального к распределенному управлению уже давно
используется в сложных системах, таких как самолеты (особенно военные самолеты),
космические аппараты и станции, корабли, автомобили и т. д.
В последние годы такие системы используются для управления группой объектов,
например, спутников, дронов. Имеются проект создания системы дорожного движения,
где каждая машина будет связываться с другими ближайшими машинами и они будут
вырабатывать безопасное движение.
3.5. Построение новых систем
Для построения новых систем используется системный подход, включающий
системный анализ и системный синтез [4].
Системный анализ имеет два направления:
1. Выявление принципа действия, главной функции и потребности исследуемой
системы;
2. Выявление недостатков.
Новую систему можно строить для существующих или альтернативных принципа
действия, функций и потребностей.
Альтернативные принципы действия можно найти, используя различные виды
эффектов и трансфер технологий. Альтернативные функции можно выявить, применяя
закономерности изменения функций. Альтернативные потребности можно выявить,
используя закономерности развития потребностей.
3.5.1. Закономерности изменения функций
Закономерности изменения функций включат [4]:
–
–
–
–
закономерность идеализации функций;
закономерность динамизации функций;
закономерность согласования функций;
закономерность перехода к моно- или поли-функциональности.
3.5.2. Закономерности развития потребностей
Закономерности развития потребностей включают [4]:
–
–
–
–
–
закономерность идеализации потребностей;
закономерность динамизации потребностей;
закономерность согласования потребностей;
закономерность объединения потребностей;
закономерность специализации потребностей.4. Заключение
В статье автор кратко изложил основные изменения в системе законов и
закономерностей развития искусственных систем.
Законы и закономерности разделены на обязательные и необязательные.
Обязательные названы законами, а необязательные – закономерностями.
К законам отнесены законы диалектики, закон увеличения степени системности,
законы построения систем и закон неравномерности развития систем.
Изменения рассматриваются по сравнению с монографией автора [4]. Новое
появилось только в закономерности увеличения степени управления. В общей тенденции
добавлена тенденция перехода от центрального управления к распределенному и
самоорганизующемуся управлению.
\ssection*{Список литературы}
1. Альтшуллер Г. С. Творчество как точная наука: Теория решения
изобретательских задач. – М.: Сов. радио, 1979. – 184 с. – Кибернетика.
2. Альтшуллер Г. С., Злотин Б. Л., Зусман А. В., Филатов В. И. Поиск новых идей:
от озарения к технологии. – Кишинев: Картя Молдовеняскэ, 1991. – 381 с.
ISBN 5 – 362 – 00147 – 7.
3. Петров Владимир. История развития законов: ТРИЗ / Владимир Петров.
[б. м.]: Издательские решения, 2018. – 90 с. – ISBN 978-5-4493-6079-3.
4. Петров Владимир. Законы развития систем: ТРИЗ, Изд. 2-е. Издательские
решения, 2019. – 894 с. – ISBN 978-5-4490-9985-3.
5. Альтшуллер Г. С. Найти идею: Введение в ТРИЗ – теорию решения
изобретательских задач. – Новосибирск: Наука, 1986. – 209 с.

\section*{Благодарность}

Выражаю благодарность Борису Голдовскому за помощь в переосмыслении понятий и
системы законов, изложенной в [4].

Автор для контакта: Владимир Петров, \texttt{vladpetr@013net.net}

\end{document}
