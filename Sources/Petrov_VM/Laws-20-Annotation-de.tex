\documentclass[11pt,a4paper]{article}
\usepackage{a4wide,url,enumitem}
\usepackage[utf8]{inputenc}
\usepackage[main=german,russian]{babel}

\parindent0pt
\parskip3pt

\title{Gesetze und Gesetzmäßigkeiten der Entwicklung von Systemen} 

\author{Vladimir Petrov}

\date{2020}

\begin{document}
\maketitle

\begin{quote}
  \foreignlanguage{russian}{«Законы и закономерности развития систем»}
  (Gesetze und Gesetzmäßigkeiten der Entwicklung von Systemen) ist eine
  vierbändige Monographie des
  Autors\footnote{\url{https://ridero.ru/books/zakony_i_zakonomernosti_razvitiya_sistem},
    ISBN 978-5-0051-5728-7.}, in welcher er die aktuelle Sicht auf die
  Problematik der Gesetze und Gesetzmäßig\-keiten der Entwicklung von
  (technischen und allgemeinen) Systemen entwickelt, wie sie aus der
  TRIZ-Theorie bekannt ist.

  Dieser Text ist eine deutsche Übersetzung eines kostenfreien Fragments aus
  dem Buch, ausgeführt von Hans-Gert Gräbe, Leipzig.
\end{quote}
\section*{Einleitung}

Die Monographie „Gesetze und Gesetzmäßigkeiten der Systementwicklung“ ist
einzigartig, weil sie die vollständigste Beschreibung der Gesetze und
Gesetzmäßigkeiten der Systementwicklung bietet. Sie besteht aus vier
Büchern. Die Monographie ist eine Verbesserung des Buchs „Gesetze der
Systementwicklung“ (2018?).  Seit jener Zeit hat der Autor einige seiner
Ansichten über Gesetze und Gesetzmäßigkeiten verändert. Darüber hinaus ist die
Monografie in einzelne Bücher unterteilt, damit sie sich leichter lesen lässt.

\textbf{Das erste Buch} enthält eine Einführung in die Monographie. Es
beschreibt die grundlegenden Konzepte und Definitionen, die Struktur der
Gesetze und Gesetzmäßigkeiten der Systementwicklung, jedes Gesetz und jede
Gesetzmäßigkeit, die Bestimmung der Gesetze und Gesetzmäßigkeiten und die
Methodik der Vorhersage der Entwicklung von Systemen.

Dieses Buch ist eine Vogelperspektive auf das gesamte System dieser Gesetze
und Gesetz\-mäßig\-keiten. Im diesem Buch können Sie nicht nur mit der
gemeinsamen Struktur der Gesetze und Gesetzmäßigkeiten bekannt machen, sondern
auch mit jedem einzelnen von ihnen. In diesem Buch werden allerdings keine
Beispiele gegeben. Dazu wird auf die weiteren Bücher verwiesen.  In diesem
Buch werden Beispiele nur zu Begriffen und Definitionen im ersten Kapitel
gegeben. Das Buch 1 ist eine Art monographische Zusammenfassung.

\textbf{Das zweite Buch} beschreibt universelle Gesetze der Systementwicklung
(Gesetze der Dialektik, die Gesetzmäßigkeit der Entwicklung in S-Kurven und
die Gesetzmäßigkeiten der Entwicklung von Bedürfnisse und Funktionsänderungen.

\textbf{Das dritte Buch} ist den Gesetzen und Gesetzmäßigkeiten der
Konstruktion und der Entwicklung von Systemen gewidnet.  Das vierte Buch
beschreibt die Gesetzmäßigkeit der Verän\-derung des Grades der Steuerbarkeit
und Dynamik sowie der Vorhersagbarkeit der Entwicklung von Systemen. Praktisch
ist dies der zweite Teil des dritten Buches. Darüber hinaus werden Anwendungen
besprochen.

Die Monografie richtet sich an ein breites Spektrum von Lesern, die an
Innovationen interessiert sind oder sich mit diesen befassen. In erster Linie
richtet sie sich an Wissenschaftler, Ingenieure und Erfinder, die
schöpferische Aufgaben lösen. Sie kann für Hochschullehrer, Graduierte und
Studierende von Nutzen sein, welche die Theorie des erfinderischen
Problemlösens (TRIZ), technisches Schöpfertum, den systemischen Ansatz und den
Innovationsprozess studieren wollen, sowie für Leiter von Unternehmen und für
Geschäftsleute.

Das Buch könnte für Patentanwälte von besonderem Interesse sein.

\section*{Vorwort}

\begin{flushright}
  \begin{minipage}{.8\textwidth}\it
    Technische Systeme entwickeln sich gesetzmäßig. Diese Gesetzmäßigkeiten
    sind erkennbar, sie können dazu verwendet werden, alte technische Systeme
    bewusst zu verbessern und neue zu schaffen, wodurch der Prozess der Lösung
    erfinderischer Aufgaben zu einer exakten Wissenschaft der Entwicklung
    technischer Systeme wird. Hier verläuft auch die Grenze zwischen Methoden
    zur Aktivierung der Variantensuche und einer modernen Theorie der Lösung
    erfinderischer Aufgaben (TRIZ).

    Man muss die Gesetze der Entwicklung technischer Systeme zu kennen und
    anwenden.
    \begin{flushright}
      G.S. Altschuller
    \end{flushright}
  \end{minipage}
\end{flushright}
Die Gesetze der Entwicklung technischer Systeme bilden das Fundament der
Theorie der Lösung erfinderischer Aufgaben (TRIZ).

Das erste System von Gesetzen der Entwicklung technischer Systeme wurde vom
TRIZ-Autor G.S. Altschuller vorgeschlagen. Später wurde dieses System von
Gesetzen nicht nur von Altschuller verbessert, sondern auch von seinen
Kollegen. Die Geschichte der Gesetze der Entwicklung technischer Systeme wird
im Buch „Die Geschichte der Entwicklung von Gesetzen:
TRIZ“\footnote{V.M. Petrov. \foreignlanguage{russian}{История развития
    законов: ТРИЗ} (Geschichte der Entwicklung der TRIZ-Gesetze).  Ridero,
  2018. ISBN 9785449360793.} beschrieben.

Diese Monographie ist eine Verbesserung des Buches „Gesetzmäßigkeiten der
Entwicklung von Systemen“\footnote{V.M. Petrov.
  \foreignlanguage{russian}{Законы развития систем: ТРИЗ} (TRIZ-Gesetze der
  Systementwicklung). 2. Auflage. Ridero, 2019.  ISBN 9785449099853.}.  Das
Buch war wegen seiner Einzigartigkeit recht weit verbreitet. Es ist die
umfassendste Darstellung von Gesetzen und Gesetzmäßigkeiten der Entwicklung
von Systemen. In dieser Ausführlichkeit waren die Gesetze noch in keinem Buch
dargestellt.

Seit der Veröffentlichung dieses Buches hat der Autor jedoch einige seiner
Ansichten zu Gesetze und Gesetzmäßigkeiten geändert. Darüber hinaus hat sich
der Autor bemüht, die neue Monografie leichter lesbar zu gestalten, indem sie
in einzelne Bücher unterteilt wurde.

Im Allgemeinen werden Gesetze und Gesetzmäßigkeiten verwendet für:
\begin{itemize}[noitemsep]
  \item das Auffinden innovativer Lösungen;
  \item die Entwicklung starken (erfinderischen, talentierten) Denkens;
  \item die Vorhersage der Systementwicklung.
\end{itemize}
\textbf{Das erste Buch} ist eine Einführung in die Monographie. Es beschreibt
die grundlegenden Konzepte und Definitionen, die Struktur der Gesetze und
Gesetzmäßigkeiten der Systementwicklung, alle Gesetze und Gesetzmäßigkeiten,
die Bestimmung der einzelnen Gesetze und Gesetzmäßigkeiten und die Methodik
zur Vorhersage der Entwicklung von Systemen.

Dieses Buch ist wie eine Vogelperspektive auf das gesamte System dieser
Gesetze und Gesetzmäßigkeiten. Aus diesem Buch werden Sie nicht nur die
allgemeine Struktur der Gesetze und Gesetzmäßigkeiten kennenlernen, sondern
auch jede von ihnen. Allerdings werden in diesem Buch keine Beispiele gegeben.
Dies erfolgt erst in den weiteren Bänden. In diesem Buch werden Beispiele nur
zu Begriffen und Definitionen im Kapitel eins gegeben, und Beispiele für die
Tendenz zur Änderung von Feldern, die im Anhang beschrieben wird.

Das erste Buch ist eine Art Zusammenfassung der gesamten Monografie. Es kann
auch als Nachschlagewerk verwendet werden.

In den anderen Büchern wird jedes der Gesetze und jede der Gesetzmäßigkeiten
im Detail beschrieben, die Methodik und Algorithmen ihrer Anwendung für
verschiedene Ziele. Darüber hinaus werden jedes Gesetz und jede
Gesetzmäßigkeit anhand zahlreicher Beispiele, Aufgaben und von Bildmaterial
veranschaulicht.

\textbf{Das zweite Buch} beschreibt universelle Gesetze der Systementwicklung
(Gesetze der Dialektik, die Gesetzmäßigkeit der S-förmigen Entwicklung und
Gesetzmäßigkeit der Entwicklung von Bedürfnisse und von Funktionsänderungen.

\textbf{Das dritte Buch} ist den Gesetzen und Gesetzmäßigkeiten der
Konstruktion und Entwicklung von Systemen gewidmet.

\textbf{Das vierte Buch} beschreibt die Gesetzmäßigkeit der Veränderung der
Steuerbarkeit und Dynamik sowie der Vorhersage der Entwicklung eines Systems.
Praktisch ist dies der zweite Teil des dritten Buches. Darüber hinaus werden
Anwendungen beschrieben.

Die Nummerierung der Kapitel in den Büchern ist fortlaufend.

Die Monographie richtet sich an Forscher, Ingenieure und Erfinder, die
schöpferische Aufgaben zu lösen haben. Es kann für Hochschullehrer nützlich
sein, für Graduierte und Studenten, welche die Theorie der Lösungen
erfinderischer Aufgaben (TRIZ), schöpferische Ingenieursarbeit, den
Systemansatz und Innovationsprozesse studieren wollen.

\section*{Danksagungen}

Ich danke meinem Lehrer, Kollegen und Freund Heinrich Altschuller, vor allem,
weil er die Grundlage für die Theorie der Entwicklung technischer Systeme
schuf -- die Gesetze ihrer Entwicklung, dafür, dass ich das Glück hatte, mit
ihm über verschiedene Dinge zu kommunizieren und über verschiedene Aspekte der
TRIZ und des Lebens zu diskutieren, insbesondere über einige Materialien
dieses Buches.

Ich verdanke Esther Zlotyna sehr viel, meiner Frau und TRIZ-Partnerin.  Viele
Jahre haben wir zusammengearbeitet, um verschiedene TRIZ-Materialien zu
entwickeln, darunter erörterten wir auch die ersten Materialien dieser Arbeit.

Ich möchte meinem Freund und Kollegen Boris Goldowsky (Russland) meinen
aufrichtigen Dank aussprechen für seine wertvollen Ratschläge und Kommentare
während der Ausarbeitung des Buches, die dazu beigetragen haben, meine Meinung
zu bestimmten Aspekten zu ändern, die in diesem Buch beschrieben sind.

\section*{Einführung}

\textbf{Grundlage der TRIZ} sind die Gesetze der Entwicklung technischer
Systeme.  Sie bilden eine zusammenhängende Struktur von Gesetzen,
Gesetzmäßigkeiten und Trends der Technikentwicklung.

Bevor wir die Gesetze der Entwicklung technischer Systeme betrachten, wollen
wir auf den oft vernommenen Einwand antworten „Es kann keine Gesetze der
Technikentwicklung geben. Technik entwickeln die Menschen nach ihren Wünschen,
das ist ein zufälliger Prozess“.

Natürlich wird Technik von Menschen entwickelt.

Die ersten „Erfindungen“ wurden von den Urmenschen unter Nutzung der Natur
gemacht. Für die Jagd war er nicht stark genug, also benutzte er eine
Keule. Für die Bearbeitung von Häuten verwende er einen scharfen Stein usw. So
begann er, seine ersten Bedürfnisse zu befriedigen. Diese „Werkzeuge“ gingen
kaputt oder stellten ihn nicht ganz zufrieden, er perfektionierte sie und
verwendete die alten nicht weiter ... So diktierte selbst in jenen fernen
Zeiten die Realität, welche Technik zu bleiben hat und welche ausstirbt.
Später wurden diese Bedingungen immer rigider.

Das Leben eines technischen Systems hängt von vielen Faktoren ab: vom Umfeld,
in dem es eingesetzt ist, von seinen ergonomischen, ökologischen, ökonomischen
und anderen Charakteristiken. In der nächsten Etappe werden die
Unzulänglichkeiten eines wenig erfolgreichen Systems verbessert.  Hinzu kommt,
dass die menschlichen Bedürfnisse ständig wachsen. Um sie zu befriedigen,
werden neue technische Systeme entwickelt, die miteinander konkurrieren.

Nur Systeme mit den besten Charakteristiken überleben. Auf diese Weise
vollzieht sich eine „natürliche Auslese“, der Prozess der Evolution
technischer Systeme.  Dieser Prozess ähnelt der natürlichen Auslese in der
Natur. Wenn Sie die Geschichte der Entwicklung konkreter Systeme analysieren,
ist es möglich, Gesetzmäßigkeiten ihrer Entwicklung zu bestimmen, und indem
man die Gesetzmäßigkeiten verallgemeinert, erhält man Gesetze. Das ist die Art
von Arbeit, die Heinrich Altshuller leistete, indem er Hunderttausende von
Patenten untersucht hat.

Ein ähnliches Vorgehen ist auch bei anderen künstlichen Systemen üblich.

Von den drei Welten menschlichen Schöpfertums -- \textbf{Wissenschaft,
  Technologie, Kunst} -- war die Wissenschaft die erste, die ihre Aura der
persönlichen Exklusivität verlor. Sie untersucht \textbf{objektive
  Gesetzmäßigkeiten}, und ihr Entwicklungsweg ist vorbestimmt.

Im Gegensatz zu Forschern (Menschen der Wissenschaft) haben viele Menschen,
die Technik entwickeln (Erfinder), nicht einmal die leiseste Ahnung von der
Existenz irgendwelcher Gesetzmäßigkeiten in dieser Entwicklung.

Dabei ist die Bedeutung von Kreativität in Wissenschaft und Technik sehr nahe:
\textbf{Ziel der Wissenschaft} ist die Gewinnung von Erkenntnissen über die
Eigenschaften der Materie, \textbf {Ziel von Technik} ist die Nutzung dieser
Eigenschaften zur Befriedigung der Bedürfnisse der Menschen und der
Gesellschaft.

\section*{Kapitel 1. Begriffe und Definitionen}

\subsection*{1.1. Gesetz}

Wir tragen einige Definitionen zusammen.

Ein \textbf{Gesetz} ist ein notwendiges, wesentliches, nachhaltiges, sich
wiederholendes Phänomen. Ein Gesetz drückt eine Beziehung zwischen
Gegenständen (\foreignlanguage{russian}{предмет}), den Bestandteilen dieses
Gegenstands, zwischen den Eigenschaften von Dingen
(\foreignlanguage{russian}{вещи}) als auch zwischen den
Eigenschaften innerhalb dieser Dinge aus.  

Aber nicht alle Beziehungen sind Gesetze. Beziehungen können notwendig und
zufällig sein.  Ein Gesetz ist eine \textbf{notwendige Beziehung}. Es bringt
die wesentliche Beziehung zwischen im Raum koexistierenden Dingen zum
Ausdruck. Es ist ein Gesetz des Funktionierens.

Gesetze existieren \emph{objektiv}, unabhängig vom Bewusstsein der Menschen.

Eine \textbf{Gesetzmäßigkeit} ist eine Bedingtheit durch objektive Gesetze;
eine Existenz und Entwicklung in Übereinstimmung mit den Gesetzen\footnote{Das
  bleibt missverständlich, hier deshalb die Formulierung im Original:
  \foreignlanguage{russian}{Закономерность, обусловленность объективными
    законами; существование и развитие соответственно законам.}}.

V.P. Tugarinov gibt die folgende Definition eines Gesetzes: 
\begin{quote}
  Ein Gesetz ist eine solche Beziehung zwischen wesentlichen Eigenschaften
  oder Stufen der Entwicklung von Phänomenen der objektiven Welt, die
  universellen und notwendigen Charakter haben und sich in der relativen
  Stabilität und Wiederholbarkeit dieser Beziehungen manifestieren.

  Der Begriff «Gesetz» dient zur Bezeichnung wesentlicher und notwendiger,
  gemeinsamer oder allgemeiner
  Beziehungen\footnote{\foreignlanguage{russian}{общей или всеобщей связи}}
  zwischen Gegenständen, Erscheinungen, Systemen, ihrer Seiten oder anderer
  Bestandteile im Prozess der Existenz und Entwicklung. Diese Beziehungen und
  Verbindungen sind objektiv. Die Gesetze der Wissenschaft sind ihr
  Spiegelbild im menschlichen Bewusstsein.

  Der Begriff der „Gesetzmäßigkeit“ unterscheidet sich inhaltlich vom Gesetz
  seinem Inhalt und seiner akzeptierten Verwendung nach. Wenn man von der
  Gesetzmäßigkeit der einen oder anderen Erscheinung
  (\foreignlanguage{russian}{явление}) spricht, wird damit sehr häufig nur die
  Tatsache betont wird, dass der Prozess oder die Erscheinung nicht zufällig
  ist, sondern der Wirkung eines bestimmten eines Gesetzes oder einer Reihe
  von Gesetzen unterworfen ist. Letzteres gilt insbesondere für eine
  Gesetzmäßigkeit, die von ihrem Inhalt her weiter gefasst ist als ein Gesetz,
  und bezeichnet auch die gemeinschaftliche Wirkung
  (\foreignlanguage{russian}{совокупное действие}) einer Reihe von Gesetzen und
  deren Ergebnis ((\foreignlanguage{russian}{итоговый результат})).

  Der Unterschied zwischen Gesetzen und Gesetzmäßigkeiten ist nicht
  ausschließlich, sondern ist als teilweises Zusammenfallen des Inhalts dieser
  Konzepte zu verstehen.
\end{quote}

Die Geschichte der Entstehung und Herausbildung des Begriffs \emph{Gesetz}
wird ausführlich von L. A. Drujanow beschrieben. Darüber hinaus hebt er zwei
Merkmale hervor, die dem Gesetzesbegriff innewohnen, und beschreibt vier (die
Hierarchie dieser Merkmale und die Texthervorhebungen sind durch den Autor
dieses Aufsatzes ausgeführt):
\begin{itemize}
\item \textbf{Wesentliche Beziehung}. «Das objektive Gesetz... ist eine
  wesentliche Beziehung von Phänomenen (oder Seiten ein und desselben
  Phänomens). Ein objektives Gesetz bezieht sich nicht auf ein einzelnes
  Objekt, sondern auf einen Komplex (\foreignlanguage{russian}{совокупность})
  von Objekten, die eine bestimmte Klasse, Art, Menge konstituieren, welcher
  die Art ihres „Verhaltens“ (Funktionieren und Entwicklung) definiert.
  ... Da ... in der Natur wesentliche Beziehungen wirken (objektive Gesetze),
  ist ihr Verhalten nicht zufällig, chaotisch; sie funktionieren und
  entwickeln sich auf gesetzmäßige Weise und zusammen mit Veränderlichkeit
  zeichnen sie sich durch relative Stabilität und Harmonie aus.»
\item \textbf{Notwendigkeit}. «...jedes objektive Gesetz (Naturgesetz) hat ein
  notwendiger Charakter; das Gesetz, die gsetzmäßige Beziehung ist immer
  gleichzeitig eine notwendige Beziehung, die, im Gegensatz zu einer
  zufälligen Beziehung, in Anwesenheit von bestimmten Bedingungen zwangsläufig
  stattfinden (auftreten, eintreten) muss ...  Folglich ist eine wesentliche
  gesetzmäßige Beziehung (Gesetz) zugleich auch eine notwendige Beziehung.
  Mit anderen Worten, die Notwendigkeit ist der wichtigste Charakterzug eines
  Gesetzes, einer Gesetzmäßigkeit. Jedes Naturgesetz ist auf diese Weise
  Ausdruck des notwendigen Charakters der wesentlichen Beziehungen in
  der objektiven Welt.»
\item \textbf{Universalität}. «Ein anderer wichtigster Charakterzug eines
  jeden objektiven Gesetzes ist seine Universalität
  (\foreignlanguage{russian}{всеобщность}). Jedes Naturgesetz ist ausnahmslos
  allen Erscheinungen oder Objekten eines bestimmten Typs oder einer
  bestimmten Art inhärent ...  Universalität ist damit ein zweiter wichtiget
  Charakterzug objektiver Gesetze, der Naturgesetze.  Da jedes Gesetz
  notwendigen und universellen Charakter hat, da es sich immer und überall
  verwirklicht, wo entsprechende Objekte und Bedingungen vorliegen, sind
  folglich gesetzmäßige Beziehungen dauerhaft, stabil, sich wiederholend...
  Das Gesetz ist invariant bezüglich der Erscheinungen».
\item \textbf{Wiederholender Charakter}. «Es ist leicht zu erkennen, wie
  wichtig für den Menschen die Existenz von Stabilität, Wiederholbarkeit,
  Ordnung in der Natur ist, für die Wissenschaft und praktische Tätigkeit der
  Menschen. Wenn es in der Natur nichts wiederholte, sondern jedes Mal auf
  eine neue Art und Weise geschähe, könnten sich weder Mensch noch Tier an die
  umgebenden Bedingungen anpassen, wäre eine zilegerichtete Tätigkeit nicht
  möglich, wissenschaftliche Erkenntnisse und das Leben selbst... Weil
  Wiederholbarkeit, Ordnung... wichtige Merkmale der objektiven Gesetze sind,
  beginnt die wissenschaftliche Suche nach gesetzmäßigen Beziehungen in der
  Natur gewöhnlich mit der Feststellung der Wiederholbarkeit bestimmter Seiten
  oder Eigenschaften der untersuchten Objekte... Deshalb ist die Wissenschaft
  nicht an beliebigen sich wiederholenden Beziehungen zwischen Objekten
  interessiert, sondern nur an solchen, die zugleich wesentlichen Charakter
  haben, d.h. sie ist an wesentlichen sich wiederholenden Beziehungen
  interessiert».
\end{itemize}
«... können wir ein objektives Gesetz (das Naturgesetz) definieren als
wesentliche Beziehung, die notwendigen, universellen, sich wiederholenden
(regelmäßigen) Charakter hat».

B.S. Ukrainzev formulierte als gemeinsame Besonderheiten der objektiven
Gesetze der Technik:
\begin{itemize}
\item \textbf{Zielumsetzung -- Realisiereung von Bedürfnissen}. «Alle
  technischen Konstruktionen oder Gerätschaften sowie auch deren werden nach
  Zielvorstellungen geschaffen, d.h. in einer Weise, dass sie in ihrer
  Funktionsweise Mittel zur Erreichung eines menschlichen Ziels sind.  Deshalb
  sind alle technischen Gesetze in ihrem Wesen Gesetze der
  Zweckverwirklichung».
\item \textbf{Steurbarkeit der Technik durch den Menschen}. «Die Gesetze (der
  Technik) vereint das Prinzip der Kopplung von Möglichkeiten der Technik und
  Möglichkeiten des Menschen oder, anders ausgedrückt, das Prinzip der
  menschlichen Kontrolle über die Technik».
\item \textbf{Das Prinzip der Technologität}. «... eine neue Konstruktion muss
  so gestaltet sein, dass sie mit vorhandenen Produktionsmitteln und basierend
  auf vorhandenen Produktionsfähigkeiten als Ausgangspunkte des weiteren
  technischen Fortschritts hergestellt werden kann».
\item \textbf{Effektives Funktionieren der Technik}. «Die Gesetze der Technik
  sind auch Gesetze des effektiven Funktionierens der technischen Mittel zur
  Erreichung der gesellschaftlichen und persönlichen Ziele... Wenn der
  gesellschaftliche Wert der arbeitsmäßigen, materiellen und energetischen
  Aufwendungen für die Herstellung und den Betrieb von Technik den
  gesellschaftlichen Wert der Ergebnisse ihrer Anwendung als künstliches
  materielles Mittel der Zielverwirklichung übersteigt, dann ist diese Technik
  wenig effektiv ist und die Gesellschaft benötigt eine andere Techniken,
  welche die Anforderungen und Prinzipien der Technik-Effizienz erfüllt».
\item \textbf{Kompatibel mit den ökonomischen Möglichkeiten der Gesellschaft}.
  «Die Gesetze der Technik haben noch eine weitere Gemeinsamkeit, die im
  Prinzip der Entsprechung zwischen Technik und den ökonomischen
  Möglichkeiten der Gesellschaft im gegebenen Stadium ihrer Entwicklung zum
  Ausdruck kommt».
\end{itemize}
A.I. Polovinkin formulierte Anforderungen, denen technische Gesetze genügen
müssen:
\begin{itemize}
\item Die Formulierung eines technischen Gesetzes soll von der Form her
  lakonisch, einfach und elegant und vom Inhalt her den obigen Definitionen
  eines Gesetzes entsprechen.
\item Die Formulierung eines technischen Gesetzes soll verallgemeinernd sein
  eine große Anzahl bekannter und möglicher Faktoren widerspiegeln. Mit
  anderen Worten, ein Gesetz muss eine empirische Überprüfung an vorhandenen
  oder speziell gewonnenen Faktoren zulassen, die von quantitativer oder
  qualitativer Form sind. Dabei muss die Formulierung des Gesetzes so klar
  sein, dass zwei Personen, die unabhängig voneinander faktisches Material
  auswählen und verarbeiten, die gleichen Überprüfungsergebnisse erhalten.
\item Die Formulierung eines technischen Gesetzes soll nicht nur konstatieren:
  «was? wo? wann?» auftritt (d.h., die Fakten ordnen sowie kurz und knapp
  beschreiben), sondern, wenn möglich, auch die Frage nach dem «Warum?»
  beantworten. In diesem Zusammenhang stellen wir fest, dass es in der
  Wissenschaft viele empirische Gesetze gab und gibt, die nicht oder nur
  teilweise auf die Frage «Warum?» antworten. Und es scheint fast keine
  wissenschaftlichen Gesetze zu geben (wegen des lokalen Charakters ihres
  Wirkens), elche die Frage «warum?» beantworten.  Auf solche Fragen antwortet
  in der Regel erst eine Theorie, die auf mehreren Gesetzen beruht.
\item Die Formulierung eines technischen Gesetzes soll autonom unabhängig
  sein, d.h. zu den Gesetzen werden nur solche verallgemeinerten Aussagen
  gezählt, die sich nicht logisch aus anderen technischen Gesetzen ableiten
  lassen. Die herleitbaren Schlussfolgerungen fassen wir unter dem Begriff der
  technischen Gesetzmäßigkeiten zusammen.
\item Die Formulierung eines technischen Gesetzes soll gegenseitige
  Zusammenhänge berück\-sich\-tigen: «Technik -- Arbeitsgegenstand», «Mensch
  -- Technik», «Technik -- Natur», «Technik -- Gesellschaft».
\item Die Formulierung eines technischen Gesetzes soll eine voraussagende
  Funktion haben, d.h. neue unbekannte Fakten vorhersagen, die mehr oder
  weniger offensichtlich, und manchmal ungewöhnlich, paradox, sind.
\item Die Formulierung aller Technik-Gesetze soll eine einheitliche, klar
  definierte begriffliche Basis haben.
\end{itemize}

\subsection*{1.2. System}
\subsubsection*{1.2.1. Allgemeine Begriffe}

In diesem Buch werden wir die Gesetze und Gesetzmäßigkeiten der
Systementwicklung betrachten.  Dazu definieren wir zunächst den Begriff eines
Systems und einige damit verbundene Konzepte.

Ein \textbf{System} (lat. griech. System, „zusammengesetzt“, ein Ganzes
bestehend aus Teilen; Verbindung) ist eine Menge von \emph{Elementen}, die
\emph{untereinander verbunden} sind und \emph{untereinander integrieren}, die
ein \emph{einheitliches Ganzes} bilden, das \emph{Eigenschaften} besitzt, die
nicht bereits in den konstituierenden Elementen, einzeln betrachtet, enthalten
sind.

Eine solche Eigenschaft wird als \textbf{Systemeffekt} oder \textbf{Emeregenz}
bezeichnet.

\textbf{Emergenz} (engl. emergent -- erscheinen unerwartet auftreten) bedeutet
in der Systemtheorie das Vorhandensein besonderer Eigenschaften eines Systems,
die seinen Teilsystemen und Blöcken nicht inhärent sind, such nicht der
einfachen Summe der nicht verbundenen Elemente ohne deren besonderen
systembildenden Beziehungen; Nichtreduzierbarkeit der Systemeigenschaften auf
die Summe der Eigenschaften seiner Komponenten; Synonym -- „Systemeffekt“.

Häufig wird diese Eigenschaft auch als \textbf{Synergieeffekt} bezeichnet (aus
grieh. Synergia -- zusammen wirken) -- Steigerung der Effizienz der Wirkung als
Ergebnis der Integration, Verschmelzen der einzelnen Teile zu einem einzigen
System, was auch als systemischer Effekt bezeichnet wird.

Zum Beispiel führt der Austausch von Dingen nicht zu einem Synergieeffekt,
weil ihre Anzahl gleich bleibt. Der Austausch von Ideen führt dagegen zu einem
Synergieeffekt, denn im Ergebnis hat eine Person mehr Ideen hat.

\textbf{Synergie} (griech. synergia -- Zusammenarbeit, Unterstützung, Hilfe,
Mitgefühl, Gemeinschaft; griech. syn -- zusammen; griech. ergon -- Sache,
Arbeit, Tätigkeit, Aktion) -- summierender Effekt der Wechselwirkung von zwei
oder mehr Faktoren, charakterisiert dadurch, dass ihre Wirkung deutlich über
die Wirkung jeder einzelnen der Komponente als deren einfache Summe
hinausgeht.

\subsubsection*{1.2.2. Zusätzliche Begriffe}

\textbf{Integrität} -- Merkmal eines Systems, das Autonomie ausdrückt und
Einheit des Systems gegenüber der Umwelt ausdrückt. Sie steht im Zusammenhang
mit dem Funktionieren des Systems und der ihm innewohnenden
Entwicklungsgesetzmäßigkeiten.

Integrität ist kein absolutes, sondern ein relatives Konzept, denn das System
hat viele Beziehungen zu umliegenden Objekten und der Umwelt und existiert nur
in Einheit mit ihnen.

\textbf{Eigenschaft} -- Seite (Attribut) eines Systems. Sie definiert 
eine Unterscheidung oder eine Gemeinsamkeit des Objekts mit anderen Objekten.

Eigenschaften werden in der \emph{Beziehung} der Subsysteme im System erkannt,
deshalb ist jede Eigenschaft relativ. Eigenschaften existieren objektiv,
unabhängig vom menschlichen Bewusstsein.

\textbf{Beziehung} -- Wechselseitige Verbindung, wechselseitige Abhängigkeit
und Beziehung der Elemente eines Systems untereinander. Das ist ein mentaler
Vergleich verschiedener Objekte und ihrer Seiten.

\paragraph{Beispiel 1.1. Satz (in der Sprache).}
Ein Satz besteht aus \emph{Worten} und \emph{der Art und Weise des Satzbaus --
  der Grammatik}.

Keines dieser Elemente hat die Eigenschaft, einen \emph{Gedanken}
auszudrücken.  Erst in der Vereinigung in ein einziges System, den Satz,
entsteht die neue Eigenschaft „Gedanke“ als systemische Wirkung.

Der Satz ist \emph{ganzheitlich}. Er ist autonom und hat seine eigenen
Entwicklungsgesetzmäßig\-keiten, die Gesetzmäßigkeiten der Entwicklung der
Grammatik.

Im Satz zeigt die Beziehung der einzelnen Wörter, ihre \emph{Eigenschaften},
die in ihrer \emph{Beziehung} zueinander gefunden werden.

\subsubsection*{1.2.3. Systemhierarchie}

Systemen ist der Begriff \textbf{Hierarchie} eigen. 

\textbf{Hierarchie eines Systems}:
\begin{itemize}[noitemsep]
\item das System selbst;
\item seine Subsysteme;
\item sein Obersystem;
\item die äußere Umgebung.
\end{itemize}
\textbf{System} - Komponenten des Systems.

\textbf{System} - ist das Objekt, in das das System als Subsystem eintritt.

Die Hierarchie kann höhere Ränge haben, zum Beispiel ein Oberobersystem und
niedrigere Ränge, wie ein Subsubsystem.

Das Oberobersystem ist das Objekt, zu dem das Obersystem gehört, und das
Subsubsystem sind die Elemente, aus denen das Subsystem besteht. Die Anzahl
der Ränge kann genügend groß sein.

\paragraph{Beispiel 1.2. Computer.}
\begin{itemize}[noitemsep]
\item System -- der Personalcomputer.
\item Subsysteme: Systemeinheit und Ein-/Ausgabegeräte, (z.B.
  Tastatur, Maus, Monitor, Drucker, Scanner, Kamera usw.).
\item Subsubsysteme der Systemeinheit: Prozessor, Motherboard, Grafikkarte,
  RAM, Festplatte, Diskettenlaufwerk, Soundkarte, Netzkarte, Netzteil, etc.
\item Obersystem -- Computernetzwerke, etc.
\item Oberobersystem -- das World Wide Web, das Internet.
\item Äußere Umgebung -- die Umgebung, in der sich der Computer befindet,
  z.B. der Raum, Luft, etc. 
\end{itemize}

\paragraph{Beispiel 1.3. Telefon.}
\begin{itemize}[noitemsep]
\item System -- Telefon.
\item Subsysteme: Mikrofon und Kopfhörer, Tastatur, Bildschirm, Speicher usw.
\item Subsubsysteme sind die Elemente, aus denen ein Mikrofon oder ein
  Kopfhörer bestehen, Tastatur, Bildschirm, Speicher usw.  
\item Obersystem -- automatisches Vermittlungsssystem (AVS), Telefonnetze,
  etc.
\item Das Oberobersystem des AVS ist das regionale und weltweite Telefonnetz.
\item Äußere Umgebung -- meistens Raum und Luft.
\end{itemize}

\paragraph{Beispiel 1.4. Automobil.}
\begin{itemize}[noitemsep]
\item System -- das Auto.
\item Subsysteme: Räder, Motor, Benzintank, Lenkung usw.
\item Subsubsysteme des Motors sind Kolben und Zylinder, Pleuelstange, Kerze,
  Ventile, Kurbelwelle, Kurbelgehäuse usw.
\item Obersystem -- Straßenverkehr, dazu gehören: Straßen, Tankstellen,
  Parkplätze, Verkehrsleitsystem, Garagen, Reparaturdienste,
  Produktionsstätten usw.
\item Oberobersystem -- das regionale und weltweite Straßennetz.
\item Äußeres Umfeld -- der offene Raum und atmosphärische Phänomene.
\end{itemize}

\subsubsection*{1.2.4. Künstliche Systeme}

\textbf{Künstliche Systeme} werden auch als \textbf{anthropogene Systeme}
bezeichnet.

Ein \textbf{anthropogenes System} (griech. anthropos -- Mensch, genesis --
Ursprung, Herausbildung eines sich entwickelnden Phänomens) ist ein System,
das im Ergebnis der bewusst gerichteten menschlichen Tätigkeit geschaffen
wurde.

\paragraph{Beispiel 1.5. Anthropogene Systeme.}
Es handelt sich um eine breite Klasse von Systemen, die vom Menschen
geschaffen wurden: Sprache, Konzepte, Gedanken, Wissen, Wissenschaft,
Literatur und Kunst, soziale Gruppen (Stämme, Gemeinschaften, Staaten usw.),
landwirtschaftliche Systeme, künstlich geschaffene Objekte der Fauna und Flora
(Gentechnik, Biotechnologie usw.), technische Systeme, etc.

Das Hauptaugenmerk dieser Monographie wird auf die Betrachtung einer
speziellen Klasse von anthropogenen Systemen gerichtet -- auf
\textbf{technische Systeme}.

Ein \textbf{technisches System (TS)} ist ein \emph{System}, das mit dem
konkreten \emph{Zweck} geschaffen wurde, ein bestimmtes \emph{Bedürfnis} zu
erfüllen. Es erfüllt eine \emph{Funktion} aus, indem es einen \emph{Prozess}
auf der Grundlage eines bestimmten \emph{Wirkprinzips} ausführt.

TS sind durch eine gewisse \emph{Struktur} und \emph{Flüsse} gekennzeichnet.

\emph{Anmerkung:} Ein technisches System kann sowohl künstliche als auch
natürliche Elemente umfassen.

Beispiele für technische Systeme sind: Flugzeug, Auto, Klimaanlage, Telefon,
Fernseher, Computer, Internet usw.

\paragraph{Beispiel 1.6. Flugzeug.}
Das Flugzeug besteht aus Flügeln, Rumpf, Motor, Fahrwerk usw.

Keines dieser Elemente hat die Eigenschaft zu fliegen. Vereint in einem System
hat das Flugzeug eine neue Eigenschaft erworben -- fliegen als systemischen
Effekt.

\paragraph{Beispiel 1.7. Telefon.}
Das Telefon besteht aus Mikrofon, Kopfhörer, Tastatur, Display, Speicher usw.

Keines dieser Elemente hat die Eigenschaft, Töne über Entfernung zu
übertragen. In einem gemeinsamen System vereint, hat das Telefon eine neue
Eigenschaft erworben -- die Übertragung von Tönen über eine Entfernung als
systemischen Effekt.

\paragraph{Beispiel 1.8. Algorithmus.}
Ein Algorithmus ist eine Abfolge verschiedener Operationen, die zu einem
konkreten Ergebnis führt.

Der Algorithmus besteht aus einzelnen Operationen, die in einer bestimmten
Reihenfolge ausgeführt werden.

Jede der Operationen und die Reihenfolge selbst, in der sie durchgeführt
werden, führen nicht zum gewünschten Ergebnis. In einem System vereint, hat
der Algorithmus eine neue Eigenschaft erworben, ein konkretes Ergebnis als
systemischen Effekt zu produzieren.

\subsection*{1.3. Bedürfnis}

Ein \textbf{Bedürfnis} ist der Bedarf an etwas, das für die Aufrechterhaltung
der Lebensaktivität eines Individuums, einer sozialen Gruppe, der Gesellschaft
erforderlich ist, der innere Motivator der Aktivität.

\subsection*{1.4. Wirkprinzip}

Ein \textbf{Wirkprinzip} ist die Art und Weise, wie die Hauptfunktion des
Systems ausgeführt wird.

\subsection*{1.5. Funktion}
\subsubsection*{1.5.1. Definition}

Als \textbf{Funktion} (lat. functio -- Ausführung, Ausführung) wird der
Prozess der Einwirkung des Subjekts auf ein Objekt bezeichnet, die ein
bestimmtes Ergebnis hat.

Darüber hinaus ist die Funktion definiert als „eine äußere Manifestation der
Eigenschaften eines beliebigen des Objekts in diesem Beziehungssystem“.

Im Weiteren werden wir eine kürzere Formulierung des Funktionsbegriffs
verwenden.

Eine \textbf{Funktion} ist eine Aktion eines \emph{Subjekts} auf einem
\emph{Objekt}, die zu zu einem bestimmten Ergebnis führt. Das Ergebnis der
Aktion kann eine \emph{Änderung} eines Objektparameters oder seine
\emph{Erhaltung} sein.

Eine Funktion wird als \emph{Verb} angeschrieben.

\paragraph{Beispiel 1.9. Flugzeug.}
Ein Flugzeug transportiert (bewegt) Passagiere. Das Flugzeug ist das Subjekt,
„bewegen“ die Funktion und die Passagiere das Objekt. Transportieren bedeutet,
das Objekt zu verändern.

\paragraph{Beispiel 1.10. Kaffee.}
Die Tasse enthält den Kaffee. Die Tasse ist das Subjekt, „enthalten“ die
Funktion, der Kaffee das Objekt. Enthalten bedeutet, das Objekt zu bewahren.

\paragraph{Beispiel 1.11. Computer.}
Der Computer verarbeitet Informationen. Der Computer ist das Subjekt,
„verarbeitet“ die Funktion, die Information das Objekt. Verarbeiten bedeutet,
das Objekt (die Information) zu ändern.

\begin{center}
  Ende des kostenlosen Buchfragments.
\end{center}
\end{document}
