\documentclass[11pt,a4paper]{article}
\usepackage{a4wide,url,graphicx,enumitem}
\usepackage[utf8]{inputenc}
\usepackage[main=german,russian]{babel}

\setcounter{secnumdepth}{-2}
\parindent0pt
\parskip3pt

\title{Gesetzmäßigkeiten der Entwicklung künstlicher Systeme} 

\author{V.M. Petrov}

\date{TRIZ-Summit 2020}

\begin{document}
\maketitle

\begin{quote}
  Der Aufsatz wurde auf dem TRIZ-Summit 2020
  präsentiert\footnote{\url{https://r1.nubex.ru/s828-c8b/f3139_da/Petrov-TDS-2020-regularities.pdf}}.  

  Übersetzung ins Deutsche von Hans-Gert Gräbe, Leipzig.
\end{quote}

\begin{abstract}
  Der Artikel ist der Präzisierung des Systems der Gesetze und
  Gesetzmäßigkeiten der Entwicklung künstlicher Systeme und einiger
  der Gesetzmäßigkeiten gewidmet.

  \emph{Schlüsselworte:} TRIZ, Systemansatz, Gesetze und Gesetzmäßigkeiten
  der Systementwicklung.
\end{abstract}
%\tableofcontents

\section{1. Einführung}
Das erste System von Gesetzen der Entwicklung technischer Systeme wurde
G. Altschuller in [1] beschrieben. Dieses System wurde von B. Zlotin in [2]
präzisiert.  Die allgemeine Geschichte der Gesetze der Systementwicklung wird
in [3] beschrieben.

Die Sicht des Autors auf die Gesetze und Gesetzmäßigkeiten der
Systementwicklung ist in der Monographie [4] dargelegt.

In diesem Artikel wird der Versuch unternommen, das System der Gesetze aus [4]
zu verbessern.

\section{2. Neuerungen an den Begriffen und am System der Gesetze}

\subsection{2.1. Neu eingeführte Begriffe}

Das neue System von Gesetzen und Gesetzmäßigkeiten zerfällt in
\emph{unbedingte} und \emph{bedingte}. Die unbedingten nennen wir
\textbf{Gesetze}, die bedingten \textbf{Gesetzmäßigkeiten}.  Unbedingt sind
diejenigen, deren Nichteinhaltung zur Funktionsunfähigkeit des Systems führt.
Bedingt sind statistische Gesetzmäßigkeiten, die unter bestimmten Bedingungen
respektiert werden können und müssen, und unter anderen Bedingungen aber nicht
unbedingt respektiert werden.

\subsubsection{2.1.1. Gesetze}
Zu den Gesetzen zählen wir die Gesetze der Dialektik, das Gesetz der Erhöhung
des Grades der Systemität, die Gesetze der Konstruktion von Systemen und das
Gesetz der ungleichen Entwicklung eines Systems.

\subsubsection{2.1.2. Gesetzmäßigkeiten}
Zu den Gesetzmäßigkeiten gehören die Gesetzmäßigkeiten der Systemevolution.

\subsection{2.2. Neuerungen an der Struktur der Gesetze und Gesetzmäßigkeiten} 

Änderungen werden nur an der Struktur der Gesetze der Konstruktion von
Systemen vorgenommen.  Zu dieser Gruppe gehören die Gesetze der Entsprechung,
der Vollständigkeit und Redundanz, der Leitfähigkeit und der minimalen
Abgestimmtheit.

Es wurde ein neues Gesetz eingeführt -- \textbf{das Gesetz der Entsprechung}.
Zuvor galt dies als eine der notwendigen Voraussetzungen für die
Arbeitsfähigkeit eines Systems.

Das Gesetz der Vollständigkeit schließt die funktionale und strukturelle
Vollständigkeit und Redundanz ein.  Die strukturelle Vollständigkeit und
Redundanz schließt Vollständigkeit und Redundanz von Teilen und Beziehungen
des Systems ein.

Neu aufgenommen wurde die \emph{Vollständigkeit und Redundanz der
  Beziehungen}.

\subsection{2.3. Neuerungen an den Gesetzmäßigkeiten}

Es wurde nur eine der Gesetzmäßigkeiten der Erhöhung der Steuerbarkeit
modifiziert.  Zum allgemeinen Trend der Erhöhung der Steuerbarkeit hinzugefügt
wurde der Übergang vo einer zentralen zu verteilter und selbstorganisierender
Steuerung.

\section{3. Das neue System von Gesetzen und Gesetzmäßigkeiten}
Gesetze und Gesetzmäßigkeiten der Systementwicklung können 
\begin{itemize}
\item \textbf{allgemeingültig} sein. Dies sind \emph{universelle Gesetze}, die
  für jedes System gelten, unabhängig von seiner Natur. Sie sind Folge der
  Einheit der materiellen Welt.  Die allgemeinsten unter ihnen sind die
  \textbf{Gesetze der Dialektik} und \textbf{die Gesetzmäßigkeit der
    S-Kurven-Entwicklung};
\item \textbf{Gesetze und Gesetzmäßigkeiten der Entwicklung von Systemen}
  sein, die allen anthropogenen Systemen gemeinsam sind.
\end{itemize}
Die allgemeingültigen Gesetze und Gesetzmäßigkeiten haben sich im Vergleich zu
[4] nicht verändert.

\subsection{3.1 Gesetze und Gesetzmäßigkeiten der Systementwicklung}

Die \textbf{Gesetze und Gesetzmäßigkeiten der Entwicklung von Systemen}
formulieren Anforderungen an die Konstruktion und Entwicklung von Systemen.

Die allgemeine Richtung der Systementwicklung führt zur \textbf{Erhöhung der
  Systemität}.

\emph{\textbf{Systemität} ist die Eigenschaft der Abstimmung aller
  interagierenden Objekte, die Umwelt eingeschlossen}.

\textbf{\emph{Diese Interaktion sollte völlig ausgewogen sein}}.

\textbf{\emph{Ein Objekt wird systemisch dann und nur dann, wenn es die
    folgenden Systemanforderungen erfüllt}}.
\begin{itemize}[noitemsep]
\item[1.] Das System muss seine \textbf{Zweckbestimmung} erfüllen.
\item[2.] Das System muss \textbf{lebensfähig} sein.
\item[3.] Das System sollte die in der Nachbarschaft befindlichen Objekte
  und die Umwelt \textbf{nicht negativ beeinflussen}.
\item[4.] Bei der Konstruktion des Systems sollten die
  \textbf{Gesetzmäßigkeiten seiner Entwicklung} berücksichtigt werden.
\end{itemize}
Die \textbf{Systemanforderungen} sind Bestandteile des \textbf{Gesetzes der
  Erhöhung des Grads der Systemität}.

Die Gesetzmäßigkeiten der Systementwicklung lassen sich in zwei Gruppen
einteilen: 
\begin{itemize}[noitemsep]
\item \textbf{Die Gesetze der Konstruktion von Systemen} (diese bestimmen die
  \textbf{\emph{Arbeitsfähig\-keit des Systems}});
\item \textbf{Die Gesetzmäßigkeiten der Systemevolution} (diese bestimmen die
  \textbf{\emph{Entwicklung von Systemen}}).
\end{itemize}
Die \textbf{\emph{Gesetzmäßigkeiten der Konstruktion von Systemen}} müssen die
folgenden Anforderungen der \textbf{Systemität} erfüllen:
\begin{itemize}[noitemsep]
\item \emph{Zweckbestimmtheit}
\item \emph{Arbeitsfähigkeit}.
\end{itemize}

Die \textbf{\emph{Gesetzmäßigkeiten der Systementwicklung}} müssen die
folgenden weiteren Anforderungen der \textbf{Systemität} erfüllen:
\begin{itemize}[noitemsep]
\item \emph{Wettbewerbsfähigkeit};
\item \emph{die Umwelt nicht nachteilig beeinflussen};
\item \emph{die Gesetzmäßigkeiten der Systementwicklung berücksichtigen}.
\end{itemize}

\subsection{3.2 Gesetze der Konstruktion von Systemen}

Die Systemforderung der \emph{Zweckbestimmtheit} wird durch das \textbf{Gesetz
  der Entsprechung} erreicht. Dieses Gesetz drückt die Notwendigkeit der
Beachtung der Entsprechung von Struktur und Hauptfunktion des Systems aus. Die
Struktur des Systems muss die Hauptfunktion des Systems sicherstellen.  Die
Struktur umfasst die notwendigen Teile sowie die Verbindungen und
Interaktionen zwischen ihnen.  Die Verbindungen sichern die Einheit des
Systems und die Fähigkeit des Flusses durch das System.

Die Systemanforderung der \emph{Arbeitsfähigkeit} wird durch die
\textbf{Gesetze der Vollständigkeit und Redundanz, Leitfähigkeit und minimalen
  Abgestimmtheit} definiert.

\subsubsection{3.2.1 Das Gesetz der Vollständigkeit und Redundanz}

Das \textbf{Gesetz der Vollständigkeit und Redundanz} umfasst
\emph{funktionale} und \emph{strukturelle} Vollständigkeit und Redundanz.

Das \textbf{Gesetz der funktionalen Vollständigkeit} ist gerichtet auf die
minimal notwendigen Grundfunktionen des Systems, und das \textbf{Gesetz der
  funktionalen Redundanz} auf die Grund-, Hilfs- und Unterstützungsfunktionen,
die zur Sicherstellung der Arbeitsfähigkeit (Ausführung) der Hauptfunktion des
Systems erforderlich sind.

Das \textbf{Gesetz der strukturellen Vollständigkeit} definiert den minimal
erforderlichen Satz von Teilen und Verbindungen im System, das \textbf{Gesetz
  der strukturellen Redundanz} die zusätzlichen Teile und Verbindungen, die
zur Arbeitsfähigkeit des Systems erforderlich sind.

Der minimal notwendige Satz von Systemelementen umfasst:
\begin{itemize}[noitemsep]
\item das Arbeitsorgan;
\item Quelle und Transformation von Stoff, Energie und Information;
\item Verbindungen;
\item die Steuerung.
\end{itemize}
Der minimal erforderliche Satz von \emph{Verbindungen} umfasst die
Verbindungen zwischen den minimal notwendigen Elementen.

\textbf{Redundanz} ist eine Gesetzmäßigkeit, nach der etwa \textbf{20\% der
  Funktionen, Elemente und Verbindungen} des Systems etwa \textbf{80\% der
  Arbeit} ausführen.

Bei der Herstellung der Funktionsfähigkeit eines Systems muss berücksichtigt
werden, dass für die Durchführung jeder Arbeit außer den Grundelementen und
Verbindungen (die eine primäre Funktion erfüllen), weitere etwa \textbf{80\%
  Hilfselemente} benötigt werden, die in der Regel, nur \textbf{20\% der
  Basisarbeit} leisten. Vor diesem Hintergrund sollte der \textbf{zusätzliche
  Verbrauch von Stoff, Energie und Information} vorgesehen werden (ca. 20\%
für die Absicherung der Primärfunktion und 80\% für Grund- und
Sekundärfunktionen).

Im Allgemeinen wird die Gesetzmäßigkeit der Redundanz folgendermaßen
formuliert: „20\% des Aufwands lifern 80\% des Ergebnisses und die
verbleibenden 80\% des Aufwands nur 20\% des
Ergebnisses“\footnote{Pareto-Gesetz, siehe Wikipedia.}.

Die Redundanz ist besonders hoch, wenn an das System \emph{erhöhte
  Anforderungen} gestellt werden.

Dies ist am häufigsten bei Sicherheits- und Rettungssystemen, bei
medizinischer Ausrüstung, Militärtechnik, komplizierter wissenschaftlicher
Forschung, Sportgeräten, Luxusgütern, Massenfeiern usw. der Fall.  Derartige
Systeme verfügen in der Regel über Duplizierungsmittel, erhebliche Reserven
(an Leistung, Energie, Proviant, Medikamente, Munition usw.)  oder
„Redundanzen“, Luxus.

\subsubsection{3.2.2 Gesetz der Leitfähigkeit von Flüssen}

Stoff, Energie und Informationen müssen von der Quelle zum verbrauchenden
Systemelement geleitet werden, wobei die erforderlichen Transformationen und
die jeweilige nützliche Funktionen ausgeführt werden müssen.

Das Erstellen der richtigen Flüsse sichert die erforderliche
\textbf{Funktionalität} und \textbf{Arbeits\-fähigkeit}. Fehlt auch nur ein
lebenswichtiger Fluss, ist das System arbeitsunfähig.

Flüsse können sein:
\begin{itemize}[noitemsep] 
\item Stoffe;
\item Energie;
\item Informationen.
\end{itemize}
Der \textbf{Stofffluss} gewährleistet den Transport von \emph{Stoff} in
verschiedenen Aggregatzustände (z.B. fest, gelartig, flüssig oder gasförmig)
oder von \emph{Objekten}.  Der Transport von \textbf{Stoffen} kann z.B. durch
Pipelines erfolgen, mit Hilfe eines Förderbandes usw., und \textbf{Objekte}
mit Hilfe von Transportmitteln, z.B. mit der Bahn, mit Kraftfahrzeugen, mit
Schiffen, Flugzeugen, Rolltreppen, Transportern usw.

Der \textbf{Energiefluss} transportiert Energie von der Quelle zum
verbrauchenden Element.  Der Fluss kann z.B. mechanische, elektrische,
optische, chemische oder andere Energiearten transportieren, verschiedene
Strahlungen usw.

Der \textbf{Informationsfluss} gewährleistet den Fluss der Information von der
Quelle zu den Zielelemente, zum Beispiel vom Steuerungssystem zu den
Steuerorganen und von diesen zurück zum Steuerungssystem. Der
Informationsfluss kann zum Beispiel mit Hilfe von Kabeln realisiert werden,
über welche die Übertragung von Information, Kontrolle und Steuerung umgesetzt
wird, oder über alle Arten der drahtlosen Kommunikation usw. Information kann
auf verschiedene Weise verbreitet werden: durch gedrucktes Material, über
Internet, Radio und Fernsehen usw.  Informationsträger sind Stoff und/oder
Feld (Energie).

\subsubsection{3.2.3 Gesetz der minimalen Abstimmung}
Externe Abstimmung:
\begin{itemize}
\item Abstimmung von Bedarf und Hauptfunktion;
\item Abstimmung der Hauptfunktion und des Funktionsprinzips;
\item Abstimmung des Funktionsprinzips und des Arbeitsorgans (das Arbeitsorgan
  muss die Hauptfunktion sicherstellen).
\end{itemize}
Interne Abstimmung (minimale Abstimmung):
\begin{itemize}
\item Minimale Abstimmung der Umwandlung mit dem Arbeitsorgan;
\item Minimale Abstimmung von Quelle und der Umwandlung von Stoff, Energie und
  Informationen untereinander sowie mit dem Arbeitsorgan und dem
  Steuerungssystem;
\item Minimale Abstimmung des Steuerungssystems mit dem Arbeitsorgan, der
  Quelle und der Umwandlung von Stoff, Energie und Information;
 \item Abstimmung aller Verbindungen und Ströme;
 \item Minimale Abstimmung aller Systemparameter.
\end{itemize}

\subsection{3.3 Gesetzmäßigkeiten der Systementwicklung}

In [4] wurden diese Gesetzmäßigkeiten Gesetze genannt. Da diese Gesetze
statistisch und unverbindlichen Charakter haben, haben wir sie in
Gesetzmäßigkeiten umbenannt. Sie haben sich nicht geändert.
\begin{itemize}
\item Die Gesetzmäßigkeit der Veränderung des Grades der Idealität;
\item Die Gesetzmäßigkeit der Veränderung des Grads der Steuerbarkeit und
  Dynamik;
\item Die Gesetzmäßigkeit der Veränderung des Grades der Abstimmung --
  Divergenz;
\item Die Gesetzmäßigkeit des Übergangs zum Ober- und Untersystem;
\item Die Gesetzmäßigkeit des Übergangs auf die Mikro- und Makroebene;
\item Die Gesetzmäßigkeiten der Ausnutzung des Raumes.
\end{itemize}

\subsection{3.4 Veränderungen in den Gesetzmäßigkeiten der Veränderung des
  Grades der Steuerbarkeit und Dynamik}

Die Änderungen betreffen nur die Gesetzmäßigkeiten der Veränderung des Grades
der Steuerbarkeit.

\subsubsection{3.4.1 Allgemeine Tendenz der Erhöhung des Grades der
  Steuerbarkeit} 

Die allgemeine Tendenz der zunehmenden Steuerbarbeit besteht im Übergang
\begin{itemize}
\item von unkontrollierten zu steuerbaren Systemen;
\item von nicht-automatischer (manueller) Steuerung zu automatischer;
\item von drahtgebundener Steuerung zu drahtloser;
\item von direkte Steuerung zur Fernsteuerung;
\item von zentraler Steuerung zu verteilter und selbstorganisierender
  Steuerung (neu eingeführte Tendenz).
\end{itemize}

Der Trend des Übergangs von zentraler zu verteilter Steuerung wird seit langem
in komplexen Systemen wie Flugzeugen (insbesondere Militärflugzeugen)
eingesetzt, Raumfahrzeuge und -stationen, Schiffe, Autos usw. verwendet.

In den letzten Jahren wurden solche Systeme zur Steuerung einer Gruppe von
Objekten eingesetzt, wie Satelliten, Drohnen. Es gibt ein Projekt zur
Schaffung eines Verkehrssystems, wo jede Maschine mit nahegelegenen Maschinen
verbunden ist und so ein sicheres Bewegungsregime herausgearbeitet wird.

\subsection{3.5 Konstruktion neuer Systeme}

Um neue Systeme zu konstruieren, wird ein Systemansatz verwendet, der
Systemanalyse und Systemsynthese umfasst [4].

Die Systemanalyse hat zwei Richtungen:
\begin{enumerate}
\item Identifizierung des Funktionsprinzips, der Hauptfunktion und der
  Bedürfnisse des untersuchten Systems;
\item Identifizierung von Mängeln.
\end{enumerate}

Das neue System kann auf bestehende oder alternative Prinzipien von Aktionen,
Funktionen und Bedürfnissen aufgebaut werden.

Alternative Handlungsprinzipien können durch die Verwendung verschiedener
Arten von Effekten und Technologietransfer gefunden werden. Alternative
Funktionen können durch Anwendung der Gesetzmäßigkeiten von
Funktionsänderungen identifiziert werden. Alternative Bedürfnisse können unter
Verwendung von Gesetzmäßigkeiten der Bedarfsentwicklung identifiziert werden.

\subsubsection{3.5.1 Gesetzmäßigkeiten von Funktionsänderungen}

Die \emph{Gesetzmäßigkeiten zur Änderung der Funktionen} umfassen [4]:
\begin{itemize}
\item die Gesetzmäßigkeit der Idealisierung von Funktionen;
\item die Gesetzmäßigkeit der Dynamisierung von Funktionen;
\item die Gesetzmäßigkeit der Abstimmung von Funktionen;
\item die Gesetzmäßigkeit des Übergangs zur Mono- oder Polyfunktionalität. 
\end{itemize}

\subsubsection{3.5.2 Gesetzmäßigkeiten der Bedarfsentwicklung}
Die \emph{Gesetzmäßigkeiten  der Bedarfsentwicklung} umfassen [4]:
\begin{itemize}
\item die Gesetzmäßigkeit der Idealisierung von Bedürfnissen;
\item die Gesetzmäßigkeit der Dynamisierung von Bedürfnissen;
\item die Gesetzmäßigkeit der Abstimmung von Bedürfnissen;
\item die Gesetzmäßigkeit der Vereinigung von Bedürfnissen;
\item die Gesetzmäßigkeit der Spezialisierung von Bedürfnissen.
\end{itemize}

\section{4. Schlussfolgerung}

In dem Aufsatz hat der Autor kurz die wichtigsten Änderungen im System der
Gesetze und Gesetzmäßigkeiten der Entwicklung künstlicher Systeme beschrieben.

Gesetze und Gesetzmäßigkeiten werden in obligatorische und nicht
obligatorische unterteilt.  Die obligatorischen werden als Gesetze bezeichnet,
die nicht obligatorischen als Gesetzmäßig\-keiten.

Zu den Gesetzen gehören die Gesetze der Dialektik, das Gesetz der Erhöhung des
Systemität, die Gesetze des Systemaufbaus und das Gesetz der ungleichen
Entwicklung von Systemen.

Änderungen werden im Vergleich zur Monographie [4] des Autors
betrachtet. Neues ist nur in der Gesetzmäßigkeit der Erhöhung des Grades der
Steuerbarkeit enthalten. Zu den allgemeinen Tendenz wird die Tendenz des
Übergangs von zentralen zu verteilten und zu selbstorganisierenden Steuerungen
hinzugefügt.

\section{Literaturliste}
\begin{itemize}
\item[1.] G.S. Altschuller. \foreignlanguage{russian}{Творчество как точная
  наука: Теория решения изобретательских задач} (Schöpfertum als exakte
  Wissenschaft). Moskau, 1979.
\item[2.] G.S. Altschuller, B.L. Zlotin, A.V. Zusman, V.I. Filatov.
  \foreignlanguage{russian}{Поиск новых идей: от озарения к технологии} (Suche
  nach neuen Ideen: Von ersten Einsichten zur Technologie). Kischinjow,
  1991. ISBN 5362001477.
\item[3.] V.M. Petrov. \foreignlanguage{russian}{История развития законов:
  ТРИЗ} (Geschichte der Entwicklung der TRIZ-Gesetze).  Ridero, 2018. ISBN
  9785449360793.
\item[4.] V.M. Petrov. \foreignlanguage{russian}{Законы развития систем: ТРИЗ}
  (TRIZ-Gesetz der Systementwicklung). 2. Auflage. Ridero, 2019.  ISBN
  9785449099853.
\item[5.] G.S. Altschuller. \foreignlanguage{russian}{Найти идею: Введение в
  ТРИЗ – теорию решения изобретательских задач} (Eine Idee finden: Einführung
  in die TRIZ -- Theorie der Lösung erfinderischer Aufgaben).  Novosibirsk,
  1986. 
\end{itemize}

\section{Danksagungen}

Ich bedanke mit bei Boris Goldovsky für seine Hilfe beim Überdenken von
Konzepten und dem System von Gesetzen, das in [4] entwickelt wurde.

\end{document}

Volodya,

ich komme auf deinen Beitrag zu Entwicklungsgesetzen zurück, da ich dort
bereits die grundlegenden Begriffe (System, künstliches System,
Systemkonstruktion, Systemevolution, Unterschied zwischen Systementwicklung
und Systemevolution, der oft wechselnde Gebrauch der Begriffe Gesetz und
Gesetzmäßigkeit, etwa im Abschnitt 3.1, wo zunächst von Gesetzen der
Konstruktion von Systemen und zwei Zeilen weiter von Gesetzmäßigkeiten der
Konstruktion von Systemen die Rede ist) nicht verstehe.

In meinem Konferenzbeitrag ist mein Verständnis des Systembegriffs als
„Reduktion auf das Wesentliche“ in drei Abgrenzungsdimensionen genauer
dargestellt, das muss ich hier also nicht wiederholen.

Die Konzentration von TRIZ auf widersprüchliche Systementwicklungen verstehe
ich - ich denke, da sind wir uns einig - als Reflex auf eine ebensolche
Entwicklung der realen Welt (im Hegelschen Sinne des „Werden“ und der
„Dialektik“).  Da die TRIZ-Methodik Teil dieser realen Welt ist, sollte auch
diese widersprüchlich sein. Ein Blick auf die „Gesetze und Gesetzmäßigkeiten“
vermeidet es gerade, solche Widersprüche zu thematisieren. Dafür mag es gute
und nachvollziehbare Gründe geben.

Da die TRIZ-Methodik (nach gemeinsamer Auffassung) bestens geeignet ist,
systemische Widersprüche zu identifizieren, ist es deshalb mein Ansatz, die
TRIZ-Methodik auf TRIZ selbst anzuwenden, und zwar auf TRIZ im Allgemeinen und
deinen Aufsatz im Speziellen. Mit meiner Frage nach der Einordnung der
Erfindung des Giftgaseinsatzes durch Fritz Haber im 1. Weltkrieg in diese
systemische Landschaft hatte ich bereits ein solches TRIZ-Prinzip angewendet -
treibe Parameter ins Extreme, um Unsichtbares sichtbar zu machen.

Ich möchte das zunächst noch in ein paar Stichpunkten nach TRIZ-Art ausbauen:

[Start der TRIZ-Analyse des Problems]
* Hauptfunktion des (militärischen) Obersystems (?): Geländegewinn
* NE: Bestreben nach Geländegewinn der Gegenseite
* OZ (Raum): Schützengräben beider Seiten und das Gelände dazwischen
* Lösungsvorschläge: verschiedene militärische Taktiken, u.a.
  * Schwächung des Gegners in direkter Konfrontation.

Für letzteres ist der Sturmangriff das probate Mittel, wozu zum Beispiel das
TRIZ-Prinzip der Asymmetrie angewendet werden kann - Ressourcen unbemerkt an
einem Frontabschnitt konzentrieren usw.

Haber geht dem TRIZ-Prinzip der vorherigen Aktion nach: Mit welchen Mitteln
kann man den Feind vorab schwächen. Wie kann man ihn in seinem Schützengraben
dezimieren, ohne eigene Soldaten zu opfern? Ansatz: Ressourcen verwenden, die
Leben im gegnerischen Schützengraben unmöglich machen usw. Der Einsatz von
Chlorgas zu diesem Zweck ist eine weitgehend ideale Lösung, wenn der Wind
günstig steht, denn es sich schwerer als Luft und verätzt die Lungen tödlich.
[Ende der TRIZ-Analyse des Problems]

Dies beschreibt eine prototypische Lösung, die an verschiedenen
Frontabschnitten eingesetzt werden kann.

(1) Was ist hier System nach deiner Lesart?
(1.a) Die Beschreibung der prototypischen Lösung?
(1.b) Eine oder (1.c) alle Realisierungen der Lösung an konkreten
Frontabschnitten?

Ich selbst unterscheide deutlich zwischen Beschreibung und Umsetzungen
(Mehrzahl) der Lösung. Wie ist das bei dir? Deine Ausführungen legen nahe,
dass du dich ausschließlich auf die Beschreibungsebene fokussierst, allerdings
gibt es bei dir den Begriff „Arbeitsfähigkeit“ eines Systems. Das ist
allerdings noch nicht das realweltliche Arbeiten. „Arbeitsfähigkeit“
thematisiert „begründete Erwartungen, nicht aber deren Differenz zu den
„erfahrenen Ergebnissen“ (beides meine Terminologie).

(2) Wie ist in dem Zusammenhang der Idealitätsbegriff zu verstehen?  Fritz
Habers Lösung ist nach meinem Verständnis eine TRIZ-Lösung hoher Idealität,
denn sie ist weitgehend eine „Von-Selbst“-Lösung durch Schaffen von Umständen,
unter denen sich das Problem „von selbst“ (im Sinne des TRIZ-Prinzips 25)
löst. 


Володя,

Я вернусь к вашей статье о развитии законов и закономерностей, так как я даже
основные понятия не понимаю - система, искусственная система, развитие
системы, эволюция системы, разница между развитием системы и эволюции системы,
меняющееся использование терминов закон и закономерностиь, например, в разделе
3.1, где ты пишешь о законе построение систем и две линии дальше о
закономерности построение систем.

В моем вкладе в конференцию, мое понимание концепции систем как "Сокращение до
существенного" в трех измерениях обьясняется более детально, так что нет нужды
повторять это здесь.

Концентрацию ТРИЗ на разработку противоречивых системных положений я понимаю -
и я думаю, мы согласны в этом - как рефлекс на такую же структуру развития
реального мира (в Гегелевском смысле "становления" и "диалектики").  Так как
методология ТРИЗ является частью этого реального мира, то оня сама должна быть
противоречивой. Изучение "законов и закономерностей" как раз - намеренно -
избегает изучение таких противоречий. Существуют причины, почему это под
какими-то аспектами целесообразно и полезно.

Так как методология ТРИЗ (по общему мнению) лучше всего подходит к изучению
системных противоречий, то мой подход заключается в применении методологии
ТРИЗ к самому ТРИЗу, а именно как к ТРИЗ в целом, так и в частности, к твоему
тексту. С моим вопросом о вписании изобретения Фрица Хабера об использовании
отравляющего газа в Первой мировой войне в твой системный ландшафт я уже
применял такой принцип ТРИЗ - гнать параметры до предела, чтобы сделать
невидимое видным.

Я сначала развиваю поставленную Хаберу проблему немножко более подробно по
манере ТРИЗ:

[Начало ТРИЗ-анализа проблемы]
* Главная функция (военной) надсистемы (?): местный прорыв
* НЭ: Враг тоже хочет этому добиться.
* ОЗ (пространство): окопы на обеих сторон и пространство между ними 
* Предлагаемые (частичные, подлежащие более подробному анализу) решения:
различные военные тактики, включая 
  * Ослабление врага в прямом столкновении. 

Для последнего из них нападение является подходящим средством, для которого,
например можно использовать ТРИЗ-приём асимметрии - ресурсы незамеченным для
врага сконцентрировать в одном месте и т.д.

Хабер следует ТРИЗ-приёму предварительного действия: Какими средствами 
можно предварительно ослабить врага. Как можно его уничтожить в его в окопах,
не жертвуя собственными солдатами? Подход: Использовать ресурсы, которые
делают невозможной жизнь во вражеском окопе и т.д. Использование
Хлор газа для этой цели является в значительной степени идеальным решением,
когда ветер дует в правильную сторону, потому что он тяжелее воздуха и сжигает
легкие смертельно. 
[заканчивает ТРИЗ-анализ проблемы]

Это описывает прототипное решение, которое может быть реализовано (повторно)
на разных участках фронта.

(1) Что тут система в твоем понимании?
(1.а) Описание прототипа решения?
(1.б) Одно или (1.в) все реализации решения на конкретных участках фронта?

Я сам четко различаю описание и реализации (во множественном числе) решения. А
ты? Судя по тому, что ты написал, кажется, ты концентрируешься исключительно
на уровне описания, но с другой стороны используешь термин "работоспособность"
системы.  Но это еще не реальная работа системы. "Работоспособность"
тематизирует "обоснованные ожидания“, но не их противоречие с "испытанными
результатами" (обе моя терминология).

(2) Как следует понимать концепцию идеальности в этом контексте?  В моем
понимании, решение Хабера - это ТРИЗ-решение высокой идеальности, так как это
во многом решение "Приём самообслуживания" путем создания обстоятельств, среди
которых проблема "сама по себе" (в смысле ТРИЗ-приёма 25) решается.

Вот мои вопросы на сегодня. По моему идеальность не имеет ни субьективный и ни
обьективный характер, а тесно связано со специальной "редукции к
существенному" и неотделима от этого контекста.

С уважением
Ханс
