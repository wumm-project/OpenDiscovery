\documentclass[a4paper,11pt]{article}
\usepackage{od}
\usepackage[utf8]{inputenc}
\usepackage[ngerman]{babel}

\title{Privatheit, das Wissen des Ichs\\ und die digitale Revolution.\\[12pt]
  \large Eine Einführung in Simulationstheorien des digitalen Wandels}

\author{Ken Pierre Kleemann, Leipzig}
\def\theauthor{Ken P. Kleemann}
\date{25.\,12.\,2018}
\begin{document}
\maketitle 

\begin{abstract}
  Der digitale Wandel bringt neue Herausforderungen, aber auch neue
  Möglichkeiten mit sich, unser Tun sprechbar zu machen.  Damit rücken Fragen
  der \emph{Produzierbarkeit} von Begriffs- und Bedeutungsbildungsprozessen in
  den Vordergrund von Praxis \emph{und} Theorie. Eine angemessene Bewertung
  der hierbei möglichen und zu erwartenden Gestaltungspotenziale steht nicht
  erst seit den Wahlbeeinflussungsversuchen von und mit Cambridge Analytica
  auf der Tagesordnung und erfordert ein besseres Verständnis des
  Zusammenhangs zwischen realen Prozesse und deren „simulativen Abbildern“.
  In diesem Aufsatz werden die Argumentationen in der aktuellen Debatte zum
  digitalen Wandel unter einem solchen Aspekt genauer analysiert.
\end{abstract}

In unserem \emph{Interdizsiplinären Lehrprojekt zum digitalen
  Wandel}\footnote{Hans-Gert Gräbe, Ken Pierre Kleemann: Interdisziplinäres
  Lehrangebot der Abteilung Softwaresysteme.
  \url{http://bis.informatik.uni-leipzig.de/de/Lehre/Graebe/Inter}} wird
herausgearbeitet, dass eine der großen Fragen oder eher Probleme des digitalen
Wandels die \emph{Privatheit} ist, nicht nur in ihrer rechtlich fixierten und
fixierbaren Form, sondern sehr wohl als Vollzug einer Lebenspraxis. Der Umgang
mit den sich vollziehenden Änderungen ist nicht allein ein Problem rechtlicher
Verfahrensweisen und des Schutzes, den man dem bürgerlichen,
vertragsschlussfähigen Subjekt zukommen lässt, sondern auch ein spezifisches
Selbstverhältnis dieses ominösen Subjektes und damit auch ein spezifischer
individueller Umgang dieses vermeintlichen Subjektes mit den „neuen“ Medien.

Die Bestimmung wie auch die Selbstwahrnehmung sind dabei stark abhängig vom
Menschenbild, das man als Grundlage des eigenen Verstehens und Tuns annimmt.
Es ist sehr wohl ein Unterschied zu erkennen, ob ich den Menschen auf der
einen Seite als Phantasie begabt, kreativ spontan, nonverbal assoziativ, als
Konstrukteur, als Künstler seiner Selbst -- als
\emph{Lebenskünstler}\footnote{Rafael Capurro (1992). Informatik: Von der
  Technokratie zur Lebenskunst. \url{http://www.capurro.de/zuerich.htm}
  (25.12.2018)} -- verstehe, oder ob ich andererseits den Menschen als
Ausdruck seines Milieus, des Systems, der Umweltbedingungen fasse. Ersteres
verlangt geradezu den Schutz einer Privatheit und damit einer Privatsphäre,
die gegen technologische, administrative, rechtliche und ökonomische
Durchgriffe geschützt werden muss. Zweiteres verlangt eher die Durchsetzung
einer gewissen technokratischen, disziplinierenden als auch Selbstregulation
fördernden externen Strukturiertheit. Privatheit in ihrer ersten, niemals
greifbaren Form erkennt die vermeintliche Spontanität der geistigen Produktion
an, das Konzept des geistigen Eigentums als ökonomische Aneignungsform ergibt
sich hier als tiefe innere Notwendigkeit. Privatheit in ihrer zweiten Form
sieht den individuellen einmaligen Einfallsreichtum als Teil eines \emph{panta
  rhei}, begreift ihn als scheinbaren Ausdruck gegebener Neukombinationen, für
die Abschottung in der Form geistigen Eigentums kontraproduktiv bis destruktiv
wirkt; das Konzept des freien und jedem zugänglichen Wissens -- der Commons --
folgt hier ebenso aus innerer Notwendigkeit. Beide Möglichkeiten strukturieren
und bestimmen nicht nur verschiedene theoretische Menschenbilder, sondern auch
die Gestaltungskämpfe unserer Zeit.

Derartige Gestaltungsdiskussionen und Auseinandersetzungen sollen in diesem
Aufsatz thematisiert werden.  Der Titel \emph{Provatheit, das Wissen des Ichs
  und die digitale Revolution} ist mit Bedacht gewählt, um den Blick zu
öffnen, aber auch eine gewisse Übersicht zu gewinnen und diese zwei Einstiege,
zwei Wege, zwei Probleme und zwei Perspektiven zu verfolgen. Das \emph{Wissen
  des Ichs} soll auf die immer wieder zu beleuchtende und zu
problematisierende Grundlage der folgenden Theorien, Stimmen und Positionen
aufmerksam machen. \emph{Die Revolution} soll uns aufmerksam machen auf das
Erbe, das explizit und implizit tradiert wird, wie auch auf vielleicht zu
große Erwartungen, welche als Grundlage mit dem Menschenbild kombiniert
werden. Auf der einen Seite werden wir versuchen, das Menschenbild hinter den
verschiedenen Zugängen zu extrahieren und zu problematisieren, auf der anderen
Seite den praktischen Moment dieser Annahmen herausstellen. All diese Zugänge
werden sich um einen Begriff drehen, der in den unterschiedlichsten Farben und
Konnotationen auftritt: \emph{Die Simulation}.

Die Kybernetik wurde in den sechziger Jahren von Wiener als Wissenschaft der
Selbstregulation von Systemen beschrieben, hier insbesondere in
Datenverarbeitungssystemen. \emph{Simulation} ist die Turings
Universalmaschine eigene \emph{sozio-praktische} Arbeitsweise und tritt in
dieser Wissenschaft mit der Unterscheidung von \emph{Designzeit} und
\emph{Laufzeit} an die Stelle des Experiments. In dieser Form werden Fragen
und Probleme bearbeitet, welche nicht nur an das System gestellt werden,
sondern sich auch durch den Eigenbetrieb aus den verarbeiteten Daten
ergeben. Die Kybernetik reduziert damit nicht nur den Menschen auf ein
quantitativ berechenbares Objekt, sondern gleich die ganze menschliche
Gesellschaft -- genauer: die je spezifischen Gesellschaften.

Für Sozialwissenschaftler ist die autopoietische Betrachtung der Gesellschaft
spätestens seit Luhmanns Arbeiten kein ferner Vorgang von Algorithmen und
Rechenmaschinen mehr. Simulation wird hier nicht nur als die quantitative
Berechnung einer reduktiven Wissenschaft verstanden, sondern als notwendige
Form der Darstellung der Dynamik des Systems selbst.

Ursprünglich warnten Kybernetiker in Ost und West, Georg Klaus wie auch Karl
Steinbuch, vor einer derartigen Adaption und Überschätzung des kybernetischen
Ansatzes, nur um selbst den Vorwurf des methodischen Reduktionismus auf sich
zu ziehen. Kybernetik wurde zum Inbegriff einer Verrechnung des Menschen und
zur Ideologie eines instrumentellen Spätkapitalismus, egal, ob in
liberal-westlicher oder in planerisch-östlicher Form, wie
Pollock\footnote{Friedrich Pollock (1975). \emph{Stadien des
    Kapitalismus}. Beck: München. } 1975 feststellt. Weizenbaum fasste jenen
Vorwurf 1976 kurz und eloquent zusammen: Die Macht der Computer, die Ohnmacht
der Vernunft\footnote{Joseph Weizenbaum (1977). \emph{Die Macht der Computer
    und die Ohnmacht der Vernunft}.  Suhrkamp: Frankfurt/M. Original (1976)
  \emph{Computer Power and Human Reason. From Judgement to Calculation.}  }.
Die eigentliche geistige Freiheit des Menschen lasse sich nicht durch eine
algorithmische quantitative Reduktion fangen, im Gegenteil zerstöre sie den
eigentlichen Wesenskern des Schöpfers der Maschine.  Diesen Bedenken wurde
entgegengehalten, dass eine technologisch-technokratische Revolution, wie sie
Kybernetiker anstrebten, die gesellschaftlichen Grundlagen (des Kapitalismus)
nicht zerstöre, sondern im Gegenteil mit den Freiheiten der Simulation neue
\emph{freiheitliche} Potenziale eröffne, welche den quantitativen und
instrumentellen Aspekten der späten Neuzeit entsprächen. Simulation wird zum
Begriff sowohl für einen technischen Vorgang als auch zu einem Begriff der
gesellschaftlichen Verblendung.

Henri Lefèbvre konnte in der \emph{Metaphilosophie}\footnote{Henri Lefèbvre
  (1965).  \emph{Métaphilosophie}. Prolegomenes: Paris.} 1965 schon sagen,
Kybernetik sei die Ideologie dieser Zeit, insoweit sie durch ihre Art und
Weise der experimentellen Simulation eine befreite, technologisierte
Menschheit \emph{vorspiele}. Die Simulation als Erscheinungsform und innere
Notwendigkeit der Datenverarbeitungssysteme erzeuge die technokratische
Simulation einer besseren und gelenkten Gesellschaft, wie auch einer
effizienteren und flexibleren Menschheit. Im Gegensatz aber zu Weizenbaum, der
den Untergang des nicht zu verrechnenden Wesens des Menschen prophezeit,
trennt Lefèbvre konsequent zwischen der Kybernetik als Wissenschaft und der
Kybernetik als Ideologie -- eher wird Luhmann zum Problem als Steinbuch. Für
Lefèbvre ist die Annahme eines nonverbalen, unlogischen, kreativen
Schöpfungsereignisses selbst Ausdruck einer bürgerlich-kapitalistischen Logik,
die den Menschen nicht durch algorithmische Quantifizierung verblendet,
sondern durch die naturrechtliche Verbrämung bürgerlicher Verfahrensweisen.
Die Annahme einer freien, schöpferischen Kreativität, wie sie Weizenbaum
verwendet, wird selbst zum Ausdruck eines Formalismus, welcher den Menschen
gerade \emph{keine} inhaltliche Auf"|ladung von Grundrechten geben will.

Lefèbvre schrieb dies unter den Vorzeichen des Mai 1968, welcher in Nanterre,
und damit im eigenen Arbeitsfeld dieses Mannes, deutlich eine Revolution
vorspielte. Spielte? Sehr wohl bezog man sich auf Guy Deboers
\emph{Gesellschaft des Spektakels}\footnote{Guy Debord (1967). \emph{Die
    Gesellschaft des Spektakels}.  Paris. Original: \emph{La société du
    Spectacle}. }, um weder mit bürgerlich-rechtlichem Reformismus noch mit
marxistisch-leninistischer Revolution verwechselt zu werden. Gegen die
Simulation bürgerlicher Freiheitsrechte, wie auch gegen die Simulation einer
planmäßig gelenkten Funktionärsherrschaft, wurden Sprüche wie: „Die Fantasie
an die Macht“; „Traum ist Wirklichkeit“; „Kunst existiert nicht, Kunst bist
du“ an die Wände der Universitäten geschrieben.  Weder eine wesensschützende
Erklärung der schöpferischen Möglichkeiten des Menschen noch eine
quantifizierte Verrechnung sollten und könnten der Weg für eine ganz andere
Revolution sein.  Gar für die Revolution, welche nun endlich mit den Idealen
der eigentlichen, der französischen Revolution nach fast zweihundert Jahren
ernst machen würde. Der Geist von 68, eine Revolution jenseits von Reform und
Revolution?

Geschichte wird zwar von Menschen gemacht, fällt nur nicht mit dem expliziten
Wollen dieser Menschen zusammen. Der Ost-West-Gegensatz ist selbst Opfer einer
Revolution geworden. Die Revolution frisst ihre Kinder -- auch die Kybernetik
ist Opfer ihrer eigenen Revolution geworden. Der Mensch das eigentliche Opfer?

Fast fünfzig Jahre später ist die Welt eine andere, aber die Ebenen der
Diskussionen und der Missverständnisse kaum verändert.  Mit dem Ende der
siebziger Jahre und endgültig in den achtziger Jahren sehen wir das Aufkommen
der Heimcomputer, der Personal-Computer, der Spielekonsolen und letztlich der
Mobiltelefone.  Datenverarbeitung ist nicht mehr nur Verarbeitung von Daten
mit Lochkarten, sondern durch Bitströme, höhere Programmiersprachen und
Visualisierungen wird eine neue Ebene der experimentellen Simulation, einer
selbstbezogenen Simulation zugänglich, welche nicht nur durch Kybernetiker
eingepflegte Daten verarbeitet, sondern durch die Einbeziehung des Nutzers --
des „Konsumenten“ -- geformt und damit auf sich selbst zurückgeworfen wird.

Gibt es heute noch Kybernetiker? Im eigentlichen Sinne des Datenverarbeiters
scheinen sie ausgestorben zu sein. Im Silicon Valley findet sich -- neben
Programmierern und Projektkoordinatoren -- Technikphilosophen und
transhumanistische Theoretiker. Ray Kurzweil ist heute Chef of Engineering bei
Google. Mit einschlägigen Büchern wie \emph{The Age of Spiritual
  Machines}\footnote{Ray Kurzweil (1999). \emph{The Age of Spiritual
    Machines}. Viking: New York.}, \emph{The Singularity is Near}\footnote{Ray
  Kurzweil (2005). \emph{The Singularity is Near. When Humans Transcend
    Biology}. Viking: New York. } oder \emph{How to Create a
  Mind}\footnote{Ray Kurzweil (2012). \emph{How to Create a Mind: The Secret
    of Human Thought Revealed}. Viking Penguin: New York. } setzt der Traum
der Kybernetiker und Systemtheoretiker zu neuen Höhenflügen an.

Für Kurzweil ist die Welt ein Informationszusammenhang hochkomplexer
Musterbildung wie auch das menschliche Gehirn. Menschen produzieren jetzt
künstliche Intelligenzen, welche nicht wie Weizenbaums Eliza mit vorgegebenen
Algorithmen versuchen, menschliche Reaktionen zu imitieren, sondern wie Siri
oder Alexa menschliches Verhalten imitierend interpretieren. Nicht mehr die
eingegeben Daten des Kybernetikers werden verwendet, sondern die durch
hochkomplexe und selbst-verweisende Protokolle aus dem Internet gesammelten
Daten der \emph{realen} digitalen Welt. Der Mensch hat mit den Prozessen im
„digitalen Universum“ nicht einfach eine Simulation seiner realweltlichen
Handlungsvollzüge -- und damit seines Abbildes -- geschaffen, sondern auch das
Potenzial zur Simulation dieser Simulation. Es ist für Kurzweil nur eine Frage
der Zeit, einmal 2029, einmal 2045, wann es zur „Singularität“ kommt -- zur
Verschmelzung der handlungsmorphen künstlichen Intelligenz mit den handelnden
Wesen selbst, die diese konstituieren und ständig verändern.  Transhumanismus
erscheint hier nicht nur als Gefahr, sondern als unumgängliche
Revolution. Eric Schmidt, der ehemalige CEO von Google, brachte es auf die
resignative, fast schon pessimistisch zugespitzte Formulierung: „Wenn wir die
Privatheit nicht schützen, so wird sie uns verloren gehen.“\footnote{Eric
  Schmidt am 30.05.2013 auf einem Vortrag an der Universität Leipzig.} Und für
Kurzweil wie auch für Schmidt ist diese Revolution nicht aufzuhalten, aber
eine rechtliche Verfassung, insbesondere privatrechtlicher Maßstäbe, soll von
diesem wirtschaftlichen globalen Player nicht gefährdet werden. Mag zwar der
Mensch nur ein hochstufiger selbstbezogener Algorithmus sein, ein Produkt
seines Milieus, seiner Umwelt, so hat er doch in seiner Geschichte einen
Verblendungszusammenhang erarbeitet, ein Wissen seiner Selbst, welches heute
erst den Betrieb eines Unternehmens wie Google rechtlich, produktiv und
distributiv möglich macht. Aus dieser Perspektive erfordert die Revolution des
Transhumanismus eine gesellschaftliche Reform, welche die
strukturell-komplexen Grundlagen des Zusammenlebens in bürgerlich-rechtlicher
Form anerkennt, aber andererseits diese bürgerlichen Illusionen mit jenem Bild
der technologischen Entwicklung vereinigt. Treiber dieses Reformismus ist eine
informierte Elite -- oder mit Jason Brennan\footnote{Jason Brennan (2017).
  \emph{Gegen Demokratie. Warum wir die Politik nicht den Unvernünftigen
    überlassen dürfen.} Ullstein: Berlin.}: vergesst die aktuelle Demokratie
und ihre postfaktische Einfältigkeit und lasst eine Epistokratie an die Macht.
Brauchen wir also eine elitäre, technokratische Revolution, um den Reformismus
gegen die transhumane Revolution zu schützen?  Jaron Lanier fand in seinem
Insiderbuch „Wem gehört die Zukunft“\footnote{Jaron Lanier (2004). Wem gehört
  die Zukunft? Hoffman und Campe: Hamburg.} über und zu diesem geistigen Klima
des Silicon Valley die wohl passendste Betitelung für die Postkybernetiker:
„Digitale Maoisten“.

Spätestens durch Snowdens Enthüllungen sind die Möglichkeiten und Gefahren
einer vermeintlich technokratischen Elite und einer Opferung des Menschen im
Fokus des Feuilletons wie auch der akademischen Diskussionen angelangt. 2014
gestaltete Frank Schirrmacher für die FAZ eine Auseinandersetzung über die
problematischen Ansätze des Silicon Valley, wie auch über die problematischen
Änderungen im Allgemeinen unter dem vielsagenden Buchtitel
\emph{Technologischer Totalitarismus}\footnote{Frank Schirrmacher (Hrsg.,
  2014). \emph{Technologischer Totalitarismus -- Eine Debatte}.  Suhrkamp:
  Frankfurt/M. }. Über Martin Schulz, Sigmar Gabriel, Hans Magnus
Enzensberger, Ranga Yogeshwar bis Karin Göring-Eckhardt, Sascha Lobo und Jaron
Lanier reicht die noch nicht vollständige Aufzählung der prominenten Autoren.
In einer fast schon seltsam anmutenden Einhelligkeit wird, wie es Weizenbaum
tat, die nie verrechenbare Qualität des menschlichen Wesens gegen die
gefährliche Überschätzung der Technifizierung in Stellung gebracht. Die
Privatheit zu schützen ist nicht nur ein bürgerlich-rechtliches Anliegen, um
demokratische Verfahrensweisen zu bewahren, sondern geradezu humanistisch,
eine ethische Verpflichtung.  Schirrmachers „Schülerin“ Yvonne Hofstetter
lässt \emph{Das Ende der Demokratie. Wie die künstliche Intelligenz die
  Politik übernimmt und uns entmündigt}\footnote{Yvonne Hofstetter (2016).
  \emph{Das Ende der Demokratie. Wie die künstliche Intelligenz die Politik
    übernimmt und uns entmündigt}. Bertelsmann: München. } nicht nur als Titel
ihres Buches erscheinen, sondern als allgemeines Problem stehen, welches in
dieser feuilletonistischen Debatte immer wieder thematisiert wird.  Dabei wird
ein kreativer, nonverbaler, unlogischer, den Algorithmen sich entziehender
Kern oder eine solche Fähigkeit des Menschen in Stellung gebracht, ja fast
schon als Schützengraben instrumentalisiert. Der Gegner ist klar, der
Transhumanist oder die technokratischen Eliten, welche einem technologischen
Totalitarismus, wenn sie ihn nicht schon fordern, so doch Vorschub
leisten. Das \emph{Wissen des Ichs} soll mit Verweis auf die unlogische und
unformalisierbare Grundebene des menschlichen Wesens mehr sein als
hochkomplexe Musterbildung.

David Gelernter, noch ein enger Freund des konservativen Schirrmacher, hatte
mit \emph{Gezeiten des Geistes}\footnote{David Gelernter (2016).
  \emph{Gezeiten des Geistes.  Die Vermessung unseres Bewusstseins}.
  Ullstein: Berlin. } dieser Position des Feuilletons nicht nur monographische
Ehre bereitet, sondern sie vor Kurzem in akademische Kreise getragen. Als ob
es einen Freud, Jung, Adler, Fromm, Reich oder Lacan nicht gegeben hat,
revitalisiert Gelernter eine Theorie des Unterbewussten, welche seit Eduard
von Hartmann, also Ende des 19. Jahrhunderts, eigentlich als erledigt
angesehen werden konnte. Die unterste Stufe des menschlichen Erlebens, gar des
Unbewussten, ist hier die voraussetzungslose Kreativität, die nonverbale,
nicht formalisierbare Phantasie. Mathematisierung, gar Technifizierung ist
hier nur eine Simulation, welche höhere Aufmerksamkeitsgrade auf die
eigentliche Stufe des menschlichen Erlebens aufpfropfen. Und so wie das Gehirn
sich eine quantifizierbare Gestaltbarkeit der Welt simuliert, so simuliert die
Technik als Ausdruck dieser Verblendung eine technokratische
Beherrschbarkeit. Hier müssen bürgerliche Werte nicht als historische
Errungenschaften geschützt werden, sondern als natürliche. Der Mensch ist
natürlich hier ein Schöpfer ohne fest setzbare Voraussetzungen. Rutschten die
Transhumanisten in ein angebliches Menschenbild des Systems, des Milieus, der
Umweltbedingungen, so wird hier explizit das Ich als frei und schöpfend
gesetzt.

Kybernetik und Weizenbaum, Kurzweil und Gelernter -- welche Wiederholung einer
alten Auseinandersetzung. Gegen die implizite Revolution der Technokraten, die
nur reformistisch vor der digitalen Revolution schützen wollen, braucht es
hier fast schon die Revolution der Besinnung auf die bürgerlichen
Verfahrensweisen.  Für beide Seiten ist \emph{das Wissen des Ichs} und die
Revolution eine Machtfrage, eine Frage der Willensdurchsetzung gegen
Widerstände. „Die“ Revolution bleibt aus, denn es läuft schon eine, die eine
andere nötig macht. Ein Geist der Vor-68er?

Die vermeintliche Degradierung des Menschen zur komplexen Rechenoperation, als
auch die Überhebung zum schöpferischen Genius, konstatiert eine individuelle
Natur. Diese Naturalisierung durch quantifizierende Algorithmen wie auch die
Naturalisierung der bürgerlich-rechtlichen Privatheit haben Hardt und
Negri\footnote{Michael Hardt, Antonio Negri (2004). \emph{Multitude.  Krieg
    und Demokratie im Empire}.  Campus: München. } in einem fulminanten
Angriff nun selbst zum eigentlichen Problem erklärt und diesem Vor-68er Geist
den Krieg erklärt. Beide Strategien erscheinen als gesellschaftliche Probleme
eines Kapitalismus, der vom alten Imperialismus zum Empire übergegangen ist.
Machtfragen ließen sich je nach der einen oder anderen Seite bearbeiten, als
der Mensch noch in seiner individuellen, körperlichen Form diszipliniert und
selbstreguliert werden konnte. Doch die neuen Formen der Globalisierung und
digitalen Technifizierung erzeugen nicht nur die Simulation der Verrechnung
oder unabhängigen Kreativität, sondern verwandeln das Subjekt selbst in
Virtualität.  Entgrenzung, Zeitgleichheit und Beschleunigung führen aber nicht
nur zu einem fluiden und trans\-identitären Wissen des Ichs, sondern die
Gesellschaft selbst ist zentrumslos und dynamisch.  Gestaltungskämpfe sind
hier keine Machtfragen im Sinne individueller Widerstandsbeseitigung, sondern
intersubjektive Probleme von Klassen, welche mit dem klassischen marxistischen
Klassenbegriff wenig zu tun haben. So virtuell wie das Subjekt ist, so
virtuell ist die Gesellschaft in der Wissensproduktion, welche weder dem
Technokraten noch der Phantasie untersteht. Wissen des Ichs ist
gesellschaftliches Wissen, welches nur als Gemeingut (Commons), als Allmende
jenseits systemischer Imperative oder genialer Einfälle funktioniert.  Die
Politik der bürgerlich-rechtlichen Verfahrensweisen wie auch das Politische
einer sich immer neu austarierenden Zivilgesellschaft verschwindet hier,
sobald die Revolution der virtuellen Subjekte, die \emph{Multitude}, ihre
simulierte Gesellschaft auf Basis geteilter Commons nun ihrerseits als
„Gesellschaft“ algorithmisch simuliert. Die Simulation der Simulation auf der
Basis hochkomplexer selbstbezüglicher „Daten“ ermögliche die \emph{reale}
Simulation der bisher versprochenen oder verweigerten Freiheiten des Menschen
aus der verblendeten Simulation und verspricht damit deren gestalterische
Zugänglichkeit -- eine alternative Version eines \emph{Sozialismus aus dem
  Computer}\footnote{W. Paul Cockshott, Allin Cottrell (2012). Alternativen
  aus dem Rechner. Für sozialistische Planung und direkte
  Demokratie. PapyRossa: Köln.}? Der Mensch als System und der Mensch als
Genie werden von Hardt und Negri durch ein intersubjektives Menschenbild
ersetzt, welches in seiner gesellschaftlichen Fundierung weder durch
individuelle Verrechnung noch durch individuelle Kreativität gefasst wird. Die
Revolution ist hier eine notwendige Folge der gesellschaftlichen Natur des
Menschen. Der arabische Frühling erschien nicht ohne Grund diesen Denkern als
Beginn vom Ende des Empires, die Multitude bewege sich ja schon.
Natürlichkeit nur intersubjektiv gefasst? Systemisch hochkomplexe
Eigenverrechnung als gesellschaftliches Trägheitsmoment verstanden, welches
immer schon gesellschaftliche Wissensproduktion ist?

In gewisser Weise sehen wir hier eine technokratische Variante auf einem
anderen Niveau.  Natürlich werden keine informierten Eliten gefordert, aber
nur der Protestler, der diese simulierten Natürlichkeitssimulationen in seiner
eigenen „Virtualität“ begreift, sei handlungsfähig.  Die digitale Revolution
bringt aus einer solchen Perspektive nicht die Revolution der Eliten hervor,
um einen Reformismus vor der digitalen Revolution zu schützen, noch die
Revolution der Denkungsarten, sondern eine Revolution, welche aus dem
gesellschaftlichen System selbst entstehen muss und wird.  Aus dieser
Perspektive sind die neuen Eliten keine, da ihre Wissensproduktion und damit
das Wissen ihres Ichs selbst den \emph{gesellschaftlichen Bedingungen} der
Wissensproduktion unterliegen.  Auch jenseits der Grabenkämpfe zwischen
geistigem Eigentum und Freier Software gilt, dass neues Wissen altes
voraussetzt, anschlussfähig sein und bleiben muss und damit gesellschaftlich
\emph{nur aus geteiltem und freiem Wissen} produziert werden kann. Nur
„stehend auf den Schultern von Riesen ist es den Zwergen vergönnt, mehr und
Entfernteres zu schauen.“\footnote{Siehe
  \url{https://de.wikipedia.org/wiki/Zwerge_auf_den_Schultern_von_Riesen}
  (25.12.2018).}  Weder der Technokrat noch das Genie produzieren Neues,
sondern die Gesellschaft als Multitude im Ganzen. Der Mensch in seinen
kooperativen Praxen ist Teil und zugleich Ausdruck dieses Systems, der Mensch
ist damit Akteur und Umwelt zugleich.

Gibt es damit auch die Variante der voraussetzungslosen Phantasie auf diesem
anderen Niveau?  Tatsächlich könnte man sagen, es existiert diese Richtung,
verbunden mit Lefèbvres Schüler Jean Baudrillard. Simulakrum\footnote{Jean
  Baudrillard (1995). \emph{Simulacra and Simulation}. Ann Arbor 1995.} ist
der Begriff, welcher nicht nur als theoretische Kategorie auftaucht, sondern
geradezu als Kampfbegriff. In den letzten Jahren erfährt dieser Situationist
eine Aufmerksamkeit, aber auch eine Interpretationsgewalt, welche sich selten
mit seinem Werk deckt.  \emph{Das System der Dinge}\footnote{Jean Baudrillard
  (2007). \emph{Das System der Dinge. Über unser Verhältnis zu den
    alltäglichen Gegenständen}. Campus: München.} als auch \emph{Der
  symbolische Tausch und der Tod}\footnote{Jean Baudrillard (2011). \emph{Der
    symbolische Tausch und der Tod}. Matthes \& Seitz: Berlin.} erfahren,
insbesondere in den letzten drei Jahren, ein stetig wachsendes Interesse --
seltsamerweise sowohl aus einer linken Position, welche Hardt und Negri weiter
denkt, als auch aus einer rechten Ecke, welche Baudrillard mit Breitbart 
instrumentalisieren will, um die vermeintliche Hegemonie der sogenannte
„Fake-News“ verbreitenden Globalisierer zu entlarven.

Für Baudrillard stellt sich die Welt der Dinge als System von Zeichen dar.
Nicht als symbolische Formen, wie sie Cassirer benutzte, sondern als
historisch generiertes System von Signifikat und Signifikant, wobei das
Signifikat selbst schon als intersubjektiver, fast schon implikativer
Signifikant auftritt. Die Mythen des Alltags werden für Baudrillard, mit
Barthes\footnote{Roland Barthes (1957). \emph{Mythen des Alltags}. Suhrkamp:
  Frankfurt/M. }, mehr sein, als nur referenzielle Namensgebungsgeschichten
von sichtbaren Dingen. Natürlich ist hier nichts, denn selbst direkte Referenz
ist gesellschaftlich und historisch bedingt. Die Zeichen simulieren immer
schon einen Sinn, der sich aus den gesellschaftlichen Prozessen ergibt. Die
Verblendung des Menschen ist immer schon sprachlich und damit performativ
bedingt und ist Produkt wie auch Bedingung der jeweiligen Stufe der
gesellschaftlichen Differenzierung und Arbeitsteilung. Die Neuzeit zeichnet
allerdings ein besonderer Umstand aus: die selbst-reflektorische und
selbstregulierende Anwendung und Rückbeziehung dieses
Simulationszusammenhangs.  In der Neuzeit verwendet die Gesellschaft diese
Simulation, diese Verblendung, um sich selbst zu begreifen und selbst zu
gestalten. Es ist eine Ordnung, welche simulierend diesen
Simulationszusammenhang auf sich bezieht, eine Ordnung der Simulakren.

Die erste Ordnung der Simulakren reicht, nach Baudrillard, von der Renaissance
bis zur französischen Revolution. Diese erste Ordnung beruht auf dem Schema
der \emph{Imitation}, sie handelt vom Naturgesetz des Wertes, sie folgt der
Demokratie der Konkurrenz. Es ist eine Imitation des Menschen, theatralisch,
mechanisch und wie ein Uhrwerk, hier gehorcht seine Technik der Analogie. In
der zweiten Stufe der Simulakren steigert sich der Zusammenhang zur
Äquivalenz, die \emph{Produktion} ist das bestimmende Schema des industriellen
Zeitalters. Die Ordnung wird bestimmt vom Marktgesetz des Wertes, freies Spiel
der Konkurrenz ist zur Kalkulation von Kräften geworden. Das Simulakrum
zweiter Ordnung ist operativ. Das Simulakrum der dritten Ordnung nun ist
operational.  Nicht mehr Imitation oder Produktion bilden das Schema der
dritten Ordnung, der jetzigen Ordnung, sondern \emph{Simulation} selbst. Das
\emph{Struktur}gesetz der Werte ersetzt \emph{Natur}gesetz als auch
\emph{Markt}gesetz der Werte.  Es gibt keine Imitation des Originals mehr wie
in der ersten Ordnung, aber auch keine Serie mehr wie in der zweiten Ordnung
-- dort gab es noch Modelle, aus denen alle Formen durch leichte Modulation
von Differenzen hervorgehen. Nicht mehr Imitation oder Produktion, sondern
Reproduzierbarkeit ist der Kern einer Ordnung der seriellen Simulation.
Interdeterminismus und Code sind die neuen Zauberwörter einer Ordnung, welche
nicht mehr nur Simulation der Simulation, das Simulakrum imitiert oder
produziert, sondern das Simulakrum selbst simuliert. Die \emph{Sprachform
  selbst} rückt ins Zentrum der Produktion.

Die Realität ist damit scheinbar zur \emph{Hyperrealität} mutiert, verkommen
oder geworden. Die systemische Betrachtung des Menschen wie auch die frei
kreative Variante sind beide Ausdruck einer Ordnung, welche heute in bipolarer
Ausprägung nur die oligarchischen Grundlagen der Gesellschaft verschleiert und
selbst reproduziert. Aber auch die andere Betrachtung, welche von Machtfragen
und individuellen Natürlichkeiten auf eine gesellschaftliche Perspektive geht
wie Hardt und Negri, erscheint in diesem „hyperrealen“ Licht nur als eine
Seite einer höher geschraubten Bipolarität, deren andere Seite die
voraussetzungslose Phantasie ist.  Systemische Betrachtung wie auch
voraussetzungslose Phantasie auf \emph{gesellschaftlicher} Ebene erscheinen
bei Baudrillard nur als höhere Form der Simulation der Simulakren.  Die
Vorstellung der Multitude ist dabei genauso Ausdruck einer scheinbar
unentrinnbaren Ordnung wie eine technokratische Elite oder ein vernünftiger
Verfassungspatriotismus.

Dies erscheint pessimistisch-apokalyptisch und lässt liberale, sozialistische
oder avantgardistische Lösungen in einem Licht der Selbstbefangenheit
erscheinen oder gar schlimmer als unbewusst böswillige Verstärkungen der
angenommenen Hyperrealität. In dieser Leseart wird es fast notwendig, einen
Mythos des 21. Jahrhunderts zu fordern, der jenseits von Rechts und Links
agieren können sollte. Nicht ohne Grund wird diese fast schon mystische -- und
damit nicht mehr voraussetzungslose -- Phantasie als
ekstatisch-aktionistisches Erleben für die Neue Rechte interessant. Kann man
doch nun mit einem dystopischen Baudrillard jeglichen Fakten-Bezug bestreiten
und sogar noch den schöpferischen Führer konstruieren. Weg mit dem Geist von
68.

Es dürfte klar sein, dass dies eine unhaltbare Leseart ist. Für Baudrillard
gibt es eine Alternative -- die Poesie und den symbolischen Tausch. Doch
sollte man sich hüten, hier die zu erwartende Variante der voraussetzungslosen
Phantasie auf gesellschaftlichem Level zu entdecken. Für Baudrillard ist die
Poesie ein Spiel mit Zeichen, welches durch dieses Spielen seine Referenz wie
auch alle Inferenzen aufhebt, indem es diese aktiv zerstört. Der symbolische
Tausch teilt weder Werte noch Begriffe, sondern vernichtet diese, um eine
unaussprechbare Hypothek bestehen zu lassen, die Hypothek einer gemeinsamen
geteilten Welt. Weder eine Revolution der Eliten, noch eine Revolution der
Denkungsart, noch eine Revolution der Multitude oder eine Revolution eines vom
Schicksal geküssten Führers lässt sich hier begründen. Poesie und der Witz
sind Subversionen durch Revisionen. „Allein jene Subjekte, die wie die Wörter
ihre Identität aufgeben, sind der sozialen Reziprozität im Lachen und im
Genuss geweiht“\footnote{Jean Baudrillard (2011). a.a.O.}, heißt es am Ende
des Buches „Der symbolische Tausch und der Tod“. Baudrillard starb 2007 ohne
Revolution, mit Revolution und gegen „die“ Revolution.

Dieser kleine Ausflug zeigt uns mehreres.

\emph{Erstens} wird das Problem der Privatheit weiter eine der großen
Herausforderungen bleiben und die Gestaltungskämpfe nicht nur unserer
Generation prägen, sondern hoffentlich auch der nach uns Kommenden.

\emph{Zweitens} dürfte klar geworden sein, dass die Bestimmung dieser
vermeintlich einfachen Kategorie \emph{Privatheit} nicht nur abhängig vom
verwendeten Menschenbild ist, sondern auch davon, auf welcher Variante und
Ebene dieses verortet wird. Das Wissen des Ichs ist eine ungeklärte und
unabgegoltene Problematik. In dieser kurzen Darstellung von Zusammenhängen
zwischen Theorien mussten Verschiedenheiten und andere Ansätze ausgeblendet
bleiben. Das Feld der theoretischen Erforschung ist größer als mit einer
schematischen Einführung im Kontext eines kurzen Aufsatzes ausgelotet werden
kann. Für ganz andere Ansätze sei auf Stefan Riegers \emph{Kybernetische
  Anthropologie}\footnote{Stefan Rieger (2003). \emph{Kybernetische
    Anthropologie}. Suhrkamp: Frankfurt/M.}  und Felix Stalders \emph{Kultur
  der Digitalität}\footnote{Felix Stalder (2016). \emph{Kultur der
    Digitalität}. Suhrkamp: Berlin.} hingewiesen.

\emph{Drittens} wurde deutlich, dass die Suche nach tragfähigen epistemischen
Grundlagen nicht nur eine theoretische ist, sondern eine zutiefst politische.
Welche Reform, Revolution oder Gestaltung sich findet und einstellt, ist
abhängig von unserem Wissen, von unserer Vorstellung des Ichs, unseres eigenen
Ichs.

\emph{Viertens} lässt sich ein Spalt, ein Licht in der Ferne, ein offener
Abgrund erkennen, welcher nicht mit Identität, Identitäten oder Ähnlichem
operieren muss. Vielleicht sollten wir tatsächlich mehr lachen in Anbetracht
von Ängsten, vermeintlichen Ängsten oder jeglicher irrationaler
Rationalitäten.

\emph{Fünftens} zeigen die Diskussionen der sechziger Jahre wie auch der
heutigen Zeit eine unheimliche Parallelität, wenn nicht gar eine flache
Wiederholung. Seltsamerweise scheint Geschichte still zu stehen, wenn es um
den Menschen und um das vermeintlich Andere seiner selbst geht -- die Technik.

2011 erlebten wir den arabischen Frühling. Zum ersten Mal bewegten sich
soziale Proteste über Ländergrenzen hinweg, getragen von sozialen Netzwerken,
welche erst seit 2007, also mit Baudrillards Tod, durch das Smartphone direkt
koordiniert werden konnten. Wir erlebten 2014 die Gründung des sogenannten
Kalifats des IS. Der Terrorismus, welcher für Baudrillard noch staatenlos sein
musste, um Ausdruck der Hyperrealität zu sein, koordinierte sich selbst zum
Ordnungsraum, getragen und verwaltet durch soziale Netzwerke und die neuen
Gadgets, welche dieses Jahrzehnt aus den erwartbaren Angeln hebt. 2016
erlebten wir Brexit und Trump. Die Hoffnung der Multitude-Revolution wurde
beendet durch dieselben Strukturen und Geräte, welche diese Hoffnung 2011 erst
möglich machten.  Man rettet sich in postfaktische
Emotionalisierungserklärungen und sieht nicht das Aufbrechen alter tradierter
Denkweisen durch eine Interoperationalität der Geräte, Anbieter und Nutzer,
welche eben \emph{nicht} nur postkonventionelle Moral unterstützen, sondern
auch fast schon vergessene rückwärtsgewandte Sittenkonformität.

Ungleichzeitigkeit der Entwicklung, die internen Spaltungen der Gesellschaft,
nicht nur in Schichten oder Klassen, sondern in Unterklassen und die Hoffnung
auf „die“ Revolution waren und sind die eigentlich problematischen Felder
dieser Analysen und politischen Gestaltungen. Mensch, Technik, das Andere
seines Selbst. Geister des Geistes von 68.  Rechts und Links sind nicht tot,
sondern umkämpfter und unschärfer denn je. Der \emph{Geist} und seine Geister
spuken weiter.

Vielleicht sind wir heute -- 2017 -- in der Lage, den digitalen Wandel nicht
als „die“ Revolution zu begreifen und dennoch den Weg für Künftiges --
reflektorisch als auch aktionistisch -- offen zu lassen. Schützen wir nicht
nur die Privatheit, sondern auch die Möglichkeit, über unsere eigenen Theorien
zu lachen.

\ccnotice
\end{document}
