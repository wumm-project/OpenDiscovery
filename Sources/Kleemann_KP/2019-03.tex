\documentclass[a4paper,11pt]{article}
\usepackage{od}
\usepackage[utf8]{inputenc}
\usepackage[ngerman]{babel}

\title{Agile Methoden, der digitale Wandel und widerspruchsbasierte
  Managementmethoden – ein Bericht über empirische Erfahrungen} 

\author{Hans-Gert Gräbe, Ken Pierre Kleemann, Leipzig}

\date{März 2019, Versio  vom 6. Mai 2020}

\begin{document}
\maketitle

\section*{Die Erfordernisse infrastrukturellen Handelns}

Der digitale Wandel bedeutet nicht allein eine massive Veränderung der
infrastrukturellen oder habituellen Anforderungen an eine sich mehr und mehr
pluralisierende Lebenswelt, sondern bringt auch fundamentale Veränderungen
klassisch tradierter Arbeitsabläufe mit sich. Für wissenschaftliches wie auch
wirtschaftliches Agieren werden Austausch und Koordinierung über bestehende
Strukturgrenzen hinweg wichtiger und zu einem echten Problem der Organisation
und des Managements.

Die Notwendigkeit zu derartigem \emph{kooperativen Handeln} in
strukturübergreifenden Kontexten führt im Kern zu einem
\emph{Vermittlungsproblem} sowohl über Fachgrenzen als auch über territoriale,
sprachliche und kulturelle Grenzen hinweg. Schon für die Verständigung in
interdisziplinären Teams \emph{innerhalb} bestehender Strukturen ist eine
kooperative Übersetzungsleistung erforderlich.  Kooperationen erfordern heute
aber immer mehr strukturübergreifendes Handeln, wobei sich gemeinsames
Verständnis in dezentralen Entscheidungsstrukturen entwickeln muss. Solche
Workflows sind stark von Organisations- und Koordinierungserfordernissen
getragen, welche zum einem eine innovationsfähige Selbststeuerung ermöglichen
sollen, zum anderen aber auch eine begrifflich überfachliche
Gegenstandserfassung und -adaption erfordern. Es ist sowohl notwendig, eine
flexible und agile Selbstverwaltung der Projektteams zu ermöglichen, damit –
unter dem Begriff Kreativität – deren Potentiale sich selbst entfalten können,
als auch eine kooperative Transdisziplinarität zu entwickeln für Sichtweisen
und Kontexte, welche die \emph{Ganzheitlichkeit} des kooperativen
Zusammenhangs reflektiert und so überhaupt erst einen dem angestrebten
Ergebnis entprechenden Entwicklungsgang ermöglicht.

Eine solche Kopplung von Innen und Außen braucht mehr als problematisierende
agile Prozesse und auch mehr als nur übersetzenden Austausch. Die „Autonomie“
in den Kontexten agiler Prozesse und infradisziplinäres Arbeiten über
Systemgrenzen hinweg bilden die beiden Widerspruchspole moderner
projektgestützter Arbeit im digitalen Wandel, die ein dialektisches,
widerspruchsorientiertes methodisches Herangehen erfordern.

\section*{Unser Interdisziplinäres Lehrprojekt}

Seit 2011 wird durch Kooperation der Institute für Informatik und Philosophie
an der Universität Leipzig in einem Interdisziplinären Lehrprojekt
„Gesellschaftliche Strukturen im digitalen Wandel“ versucht, eine solche
inhaltliche Problematisierung der aktuellen Entwicklungen über Fächer- und
Kulturgrenzen hinweg für Studierende der informatik und der Humanities auf
theoretischer und praktischer Ebene auszuloten, in dem die verschiedenen
lebensweltlichen Erfahrungen der beteiligten Studierenden selbst in einem
solchen Grenzen überschreitenden Lehr- und Lernprozess eingespeist werden, um
ein grobes gemeinschaftliches Begriffsverständnis zu entwickeln, das nicht nur
anschlussfähig an die je privaten Erfahrungswelten ist, sondern auch eine
\emph{gemeinsame} Reflexion über diese Grenzen hinweg ermöglicht.

Das Lehrprojekt hat also zwei Ziele – einerseits eine \emph{theoretische
  Reflexion} dieser Entwicklungen und andererseits die Möglichkeit der
\emph{praktischen Übung} in einem solchen interdisziplinären Diskurs.
Schwierig stellte sich die Identifizierung prüfungsrelevanter Anforderungen
heraus, da sich beide Ziele an hochgradig individuellen akademischen Prozessen
orientieren, die sich einer klassischen Leistungsbewertung entziehen. Maßstab
der Prüfungsanforderungen ist deshalb die Fähigkeit zu akademischer
Argumentation, die Studierende der Informatik in einer Seminararbeit und
Studierende der Humanities in einer mündlichen Prüfung nachweisen.

Um die disziplinären disparaten und persönlich verschiedenen Hintergründe
sowohl zum inhaltlichen Forschen als auch zu produktiven praktischen
Lernerfolgen zu führen, wurde eine umfassende organisatorische wie auch
koordinierende Struktur geschaffen, die moderne Didaktik mit projektgestützter
Arbeitsweise in direkten Kontakt bringt. Für die Studierenden beider Kohorten
wird eine inhaltlich auf aktueller Forschung basierende Vorlesungsreihe mit
einem studentisch gestalteten Seminar gekoppelt, wobei die
Vermittlungsschwerpunkte der Vorlesung bei den Lehrenden und im Seminar bei
den Studierenden liegen. Für letzteres gibt es eine Handreichung mit einem
umfangreichen Angebot von Seminarthemen als Anregung für die konkrete Auswahl
von Themen, die von den Studierenden zum Vortrag gebracht und anschließend
ausführlich diskutiert werden. Die entsprechenden Materialien sowie
weiterführende Anmerkungen werden im Internet\footnote{Siehe
  \url{http://www.dorfwiki.org/wiki.cgi?HansGertGraebe/SeminarWissen/Bisher}.}
veröffentlicht und bilden inzwischen einen beachtlichen Erfahrungskorpus als
zusätzliche Informationsquelle für künftige Seminare. Studierende der
Humanities belegen weiterhin ein Projektpraktikum, in denen sie agile Methoden
in einem unserer Drittmittelprojekte kennenlernen. Damit wird zum einen eine
inhaltlich didaktisch akkurate Forschung garantiert und zum anderen die
Möglichkeit eröffnet, direkte Erfahrungen mit flexibel gestalteter
Projektorganisation zu gewinnen.

Damit werden \emph{erstens} die neuen Herausforderungen des digitalen Wandels
an die Abläufe gemeinsamen Arbeitens sowohl theoretisch thematisiert und
entwickelt als auch in ihren technischen und sozio-technischen Dimensionen für
die Studierenden praktisch beobachtbar. \emph{Zweitens} wird so Studierenden
über Fächergrenzen und kulturelle Eigenheiten hinweg ein produktives Bewahren
und Fortentwickeln des eigenen Kontextes praktisch aufgezeigt. \emph{Drittens}
ließen sich Ansprüche akademischen Arbeitens, insbesondere des rationalen
Argumentierens, auf die transdisziplinäre Selbstorganisation des studentischen
Forschens und Präsentierens übertragen, was eine fortgesetzte kooperative
Arbeit an verwendeten Begriffen und Konzepten im Sinne einer
Infradisziplinarität ermöglichte. Mit der Praktikumstätigkeit wird es
\emph{viertens} möglich, auf der einen Seite aktuelle
Projektmanagementmethoden der Softwareentwicklung kennenzulernen und zum
anderen projektspezifisches, aber selbstgesteuertes Arbeiten zu erfahren, was
ungeahnte innovative Potenziale weckt.

Wir greifen bei dieser in neueren hochschuldidaktischen Konzepten als „Service
Learning“ bezeichneten Art der Organisation von Projektarbeit auf lange eigene
Lehrerfahrungen in der Gestaltung von Praktikumsprojekten zurück, in denen die
stimulierende Wirkung derartiger Konstellationen auch über den eigentlichen
Kurs hinaus vielfach beobachtet werden konnte. Dies trifft auch auf eine Reihe
von Projekten aus diesem Kontext zu, die von Studierenden über den Kurs hinaus
weiter verfolgt und dabei teilweise auch neue finanzielle Förderung
erschlossen werden konnte.

Besonders erfolgreich waren Projektpraktika, die eine infradisziplinäre
Kontextualisierung des eigenen Projektes mit einem Widerspruchsmanagement
innerhalb agiler Methoden kombiniert haben. Es soll nun genauer beschrieben
werden, wie erstens die Anforderungen des digitalen Wandels eine derartige
Begriffsarbeit und Problemkontextualisierung ermöglichen und fordern und
zweitens derartige Probleme zu \emph{dialektischen Widersprüchen}
transformieren werden können, die ein selbstgesteuertes, innovatives,
kreatives und produktspezifisches Entwickeln möglich machen.

\section*{Infradisziplinarität}

Dass der digitale Wandel mehr bedeutet als Breitbandausbau, wird spätestens
durch die Thematisierung der notwendigen Verarbeitung riesiger Datenmengen
deutlich. Allein durch eine verstärkte Beachtung der Big Data Analyse lässt
sich aber das strukturelle Moment der soziotechnischen Veränderungen kaum
fassen. Durch eine immense Änderung der habituellen Verhaltensweise und der
durch entsprechende Gadgets gestützten gesellschaftlichen Vollzugsformen haben
wir eine Entwicklung vor uns, welche nicht nur territorial tradierte Grenzen
verschiebt, sprachlich umgrenzte Räume öffnet, sondern alltägliche
Gewohnheiten aufsprengt. Die Erfassung und das Begreifen einer vor sich
gehenden Entwicklung, welche den Betrachter mit inkludiert, fordert nicht nur
ein Umdenken wirtschaftlicher Sichtweisen, sondern auch
wissenschaftlich-gesellschaftlich eine weitere Perspektive. Während sich
wirtschaftliche Sichtweisen in großen Unternehmen immer mehr von
Marktführerschaft und Konkurrenz zu Technologieführerschaft verschieben, geht
es in der weiteren Perspektive darum, die Bedingtheiten des eigenen Handelns
in Kooperationen mit anderen überhaupt sprechbar zu machen. In der freien
Wirtschaft, in der akademischen Forschung und Lehre wie auch im
zivilgesellschaftlichen Diskurs ist die Wahrnahme der eigenen Kontextualität
\emph{Voraussetzung} für eine gemeinsame Begriffs- und
Kontextproblematisierung.

Für den akademischen Bereich sind dies kaum neue Forderungen, da insbesondere
der interdisziplinäre Austausch bereits mehr als hundert Jahre vor dem Schisma
der Geistes- und Naturwissenschaften in der ersten Hälfte des 20. Jahrhunderts
thematisiert wurde. Durch die Entstehung großer Massen"|universitäten und
durch den Druck technologiegestützter Modernisierung wurde im Ausgang des
zwanzigsten Jahrhunderts die interkontextuelle Übersetzungsleistung alleine
als zu gering für kooperative Praxen erkannt.  Entscheidend wurde nicht nur
eine zu erstrebende Transdiziplinarität, bei der sich auf gemeinsame Begriffe
und Konzepte auf angemessenem Abstraktionsniveau konsensual geeinigt wird,
sondern das kooperative fortgesetzte Arbeiten an diesen Begriffen und
Konzepten im Abgleich mit empirischen, sich ständig verändernden Erfahrungen.
\emph{Infradisziplinarität} erfordert damit auch die fortgesetzte Adaption und
Veränderung des eigenen \emph{innerfachlichen} Begriffsapparates und somit der
Perspektive auf den eigenen Gegenstand. Dieses Problematisieren des
übersetzten eigenen Kontextes wird heute oft auch implizit im
projektgestützten wissenschaftlichen Arbeiten umgesetzt.

Die Herausforderungen durch technologisch gestützte Entwicklung, durch
transnationalen Austausch und durch veränderte globale Handlungsvollzüge
verknüpft diese Erfahrung mit den Anforderungen freier Wirtschaft und
zivilgesellschaftlicher Gestaltung. Projekte, welche nicht allein
infrastrukturelle technische Ebenen berühren, sind allzu oft mit den Barrieren
unterschiedlicher Sozialisation, kultureller Hintergründe und fachspezifischer
Ausbildung konfrontiert. Ein Empowerment des Teams und dementsprechender
adaptiver Organisations- und Managementmethoden braucht somit sowohl die
Einsicht in die Grenzen und Bedingtheiten der \emph{eigenen Perspektive} und
der damit einhergehenden Übersetzungsleistung als auch die projektspezifische
kooperative Anpassung einer zu entwickelnden \emph{gemeinsamen Perspektive}.
Infradisziplinarität ist somit nur ein erstes Mittel, wenn nicht gar nur
Katalysator, um einen fortgesetzten Prozess der Projektzieladaption und der
Problematisierung selbst zu erreichen.

Jenseits des akademischen Bereiches, in dem eine gewisse Sensibilität durch
die Erfordernisse des rationalen Argumentierens gegeben ist, wird
Infradisziplinarität als prozessuales gemeinsames Problematisieren selbst zum
Problem. Für einen studentischen Kontext sind Motivation und Antrieb meist
klar und manifestieren sich im Zuschnitt von Forschungsvorhaben wie auch im
Benotungssystem. Für andere Kontexte der heutigen Projektarbeit wird die
Organisation und Koordination der Interdependenzen des Projektteams zu
eigentlichen Herausforderung, flexibel, aber zugleich motivierend auf sich
dynamisch entwickelnde Projektkontexte zu reagieren. Dies wird durch die
Entwicklung agiler Methoden versucht.

\section*{Agile Methoden}

Im Bereich der Softwareentwicklung ist man seit Jahrzehnten mit derartigen
Probleme der Projektarbeit konfrontiert. Produktzyklen werden nicht nur durch
wechselnde Kundenanforde"|rungen geprägt, sondern in noch größerem Maße durch
die Anpassung an technische Entwicklungen. Dies gilt besonders für den großen
Bereich anwendungsspezifischer Spezialsoftware im Betrieb von IT-Strukturen im
B2B-Bereich.  Damit ist es von Anfang an unumgänglich, einen
Produktentwicklungszyklus zu entwerfen, der nicht linear ist und Brüche
inkludieren kann. Software ist nicht allein von den initialen Vorgaben des
Kunden abhängig, sondern auch von den schwer prognostizierbaren
Änderungserfordernissen im Laufe des gesamten Software-Lebenszyklus und damit
von der Anpassungsfähigkeit des Projektes und der Teamleistung selbst. Agile
Methoden versuchen, eine solche Flexibilität der Produktentwicklung mit der
Offenheit und Selbststeuerung des Teams zu verbinden.

Im universitären Kontext des Praktikums erwies sich die Umsetzung der agilen
Methoden auf der Basis einer an die Spezifika des universitäen Lehrbetriebs
angepassten \emph{Scrum-Methodik} als besonders fruchtbar. Dabei wird
versucht, den Abstand und die dennoch notwendige Verknüpfung zum
Produkterfordernis mit der flexiblen Selbststeuerung der Gruppe in einem
strukturierten methodischen Kontext zu verbinden.

Hierfür wird der \emph{Product Owner} als derjenige, der die
technologisch-sozialen Anforderungen des Produktes formuliert und
kontrolliert, von der eigentlichen \emph{Teamarbeit} und damit der technisch
funktionalen Umsetzung separiert. Er ist für die Formulierung der
Anforderungen zuständig, aber die Kontrolle der weiteren Umsetzung der
Produktentwicklung ist ihm weitgehend entzogen. Somit wird dem Team ein
motivierender Abstand zum eigentlichen Auftrag suggeriert und eine flexible
Detaillierung der fachlichen Dimension der Kontext- und Anforderungsanalyse
sowie die Organisation der Umsetzung dieser funktionalen Anforderungen
übertragen.

Eine koordinierende Rolle für festzulegende Entwicklungsabschnitte („Sprints“
in der Scrum-Terminologie), Deadlines und Ergebnisüberprüfungen spielt der
\emph{Scrum-Master}. Dieser gehört ebenfalls nicht zum Team, ist zum einen für
die zeitliche und formale Organisation verantwortlich, zum anderen aber mehr
als Tutor und „Problemlöser“ denn als Manager des Teams tätig und dafür
zuständig, die \emph{Arbeitsbedingungen} des Teams zu stabilisieren. Diese
Entkopplung erlaubt es dem Team, sich auf die fachlichen Aspekte der
Produktentwicklung zu konzentrieren und diese in kleinere Sprints zu zerlegen,
an deren Ende im \emph{Sprint Review} und der \emph{Sprint Retrospektive} eine
Analyse und Problematisierung sowohl der fachlichen Anforderungen als auch der
teaminternen Arbeitsbedingungen erfolgt, um diese mit technisch-funktionalen
wie auch projektdynamischen Entwicklungen gegen Erfahrungen der „Außenwelt“
abzugleichen. Infradisziplinarität setzt genau an dieser Stelle an und soll
vom Scrum-Master nicht moderierend, sondern katalysierend zur Anwendung
gebracht werden. Die Scrum-Methodik erwies sich damit für uns als eine
wirksame Methodik, in den studentischen Projektpraktika eine
kontextualisierende und zugleich technisch-funktionale Arbeitsweise zu
entwickeln, welche die innere (technisch-funktionale) und äußere
(technologisch-soziale) Dimension der Problemdynamik separiert und damit die
Motivation und den Eigenanteil der Projektteilnehmer deutlich aktivieren
konnte.

Mit einer solchen Separierung von Verantwortlichkeiten lassen sich diese
beiden Ebenen der Problemdynamik zwar gut voneinander abheben, deren
dialektisches Verhältnis ist damit aber noch nicht thematisiert oder gar
entwicklungsmethodisch eingebunden.

\section*{WUMM – Widersprüche und Managementmethoden}

In der Scrum-Entwicklungsmethodik werden Widersprüche zwischen der inneren
Konstitution des Teams und den äußeren Bedingtheiten des Handelns dieses Teams
in den Scrum-Master ausgelagert und es dessen Fähigkeiten, Erfahrungen und oft
auch seiner Autorität im Unternehmen überlassen, diese Probleme zu
lösen. Damit kann das Aufzeigen von derartigen Problemen auch für eine agile
und flexible selbstgesteuerte Arbeitsweise zum Problem werden, wenn eine
solche „kreative“, aber wenig methodisch untersetzte Lösung nicht gelingt.

Um die Innovationsfähigkeit und Offenheit des Teams in agiler Weise zu
erhalten, ist es notwendig eine solche infradisziplinär übersetzende
Problematisierung sowie deren Lösung selbst methodisch weiter zu begleiten.
\emph{Methodisch} getriebenes Innovationsmangement wird hier zum zusätzlichen
Erfolgsfaktor, damit die kreativen Prozesse der Gruppe in selbstgesteuerten
Bahnen nicht ins Stocken kommen. Hierbei können widerspruchsorientierte
Innovationsmethodiken aus der TRIZ-Forschung zur Anwendung gebracht werden.

TRIZ geht wesentlich auf den sowjetischen Erfindungsforscher Genrich
Saulowitsch Altschuller zurück und war ursprünglich der theoretische und
praktische Versuch, die Abläufe und Kontexte von technischen Erfindungen und
Innovationen aufzuklären, zu abstrahieren und für die Erfinderpraxis fruchtbar
zu machen. Altschuller hat seit Mitte der 1940er Jahre auf der Basis
systematischer Patentrecherchen TRIZ als „Theorie des Lösens von
Erfinderaufgaben“ entworfen und ein umfangreiches methodisches Instrumentarium
entwickelt, mit dem diese theoretischen Ansätze in der Arbeit von Ingenieuren
und Erfindern erfolgreich eingesetzt werden.

TRIZ ist als Methodik auf dem Gebiet der ehemaligen Sowjetunion sehr
verbreitet und hat sich mit der Auswanderungswelle nach 1991 auch in anderen
Ländern (USA, China, Südkorea, Israel) etabliert. In der DDR verbreitete sich
diese Methodik in den 1980er Jahren im Rahmen der Erfinderschulbewegung. Diese
Erfinderschulen schafften es in relativ kurzer Zeit, eine bis heute wenig
aufgearbeitete Forschungsbasis über Abläufe und Bedingungen innovativer
Prozesse zu schaffen und gleichzeitig Leitfäden und Anweisungen zu erstellen,
die zum einen diese Erfahrungen verwendeten, aber zum anderen ständig
veränderten. Entsprechend groß ist mittlerweile die Bandbreite von
systematischen Handlungsempfehlungen und strategischen Leitfäden, welche alle
nach Gegenstandsbezug stark variieren können. Dabei spielten insbesondere
ebenen-übergreifende Widerspruchskonstellationen wie die von uns weiter oben
herausgearbeiteten eine zentrale Rolle.

Die wichtigste Leistung dieser Forschung war die Herausarbeitung einer
Methodik, derartige Widersprüche zu identifizieren, deren dialektischen Gehalt
herauszuarbeiten und auf dieser Basis geeignete Lösungsprinzipien zur
Anwendung zu bringen. Auf diese Weise kann man im gesamten
Produktentwicklungszyklus Probleme von Anfang an auf ihren Widerspruchsgehalt
hin analysieren und Zugänge zu diesen Widersprüchen identifizieren.

Eine solche Option erweitert die Selbststeuerungsmöglichkeiten des Teams,
indem bisher auf den Scrum-Master delegierte Problemlösebedarfe und
-kompetenzen ins Team zurück"|geholt werden. Alternativ kann man die bisherige
Arbeitsteilung zwischen Team und Scrum-Master beibehalten, aber die Arbeit des
Scrum-Masters selbst methodisch genauer strukturieren.

Ein methodischer Zugang zu Problemlösebedarfen, der nicht nur auf den
Entwicklungsgegenstand gerichtet ist, sondern auch die Widersprüche in der
eigenen Teamarbeit berücksichtigt, lässt erkennen, dass es bei
interdisziplinärer Teamarbeit nicht nur um Übersetzungsleistungen oder die
Berücksichtigung kultureller Unterschiede geht, sondern die differenten
Erfahrungshorizonte selbst Quelle von Widersprüchen sind.  Derartige
Verschiedenheiten in den Kulturen sind oft ein \emph{in Verfahrensweisen
  verfestigter} Umgang mit Widersprüchen in den Reflexionsformen einer
gesellschaftlichen Totalität, die im konkreten interdisziplinären Projekt
wieder „verflüssigt“ werden muss.  Damit wird ein solches interdisziplinäres
Projekt aber nicht allein durch Übersetzungsleistungen oder infradisziplinäre
Verfahrensweise agil gehalten, sondern in sich verwandelt und zum
motivierenden selbststeuernden Motor der weiteren Arbeit. Derartige
Widerspruchs- und Managementmethoden ermöglichen, insbesondere in der
Kombination mit der agilen Scrum-Methodik, eine andere Selbstwahrnehmung und
Mobilisierung jedes einzelnen Teammitgliedes.

Die Anwendung von WUMM zeigte die erfolgreiche Umgehungsmöglichkeit eines
problematischen Haltens des Produktentwicklungszyklus auf und öffnete
zusätzliche Perspektiven durch die Verschiebung zur faktoralen
Widerspruchsanalyse. Durch die Entfernung vom Product Owner, vom Scrum Master
und vom unbedingten Lösenwollens eines Problemes konnten innovative Wege
geöffnet und kreative Ideen des Teams in Eigenverantwortung geweckt werden.
WUMM bietet in diesem Sinne eine offene Systematisierung von
Innovationsprozessen zur Selbstsystematisierung am spezifischen Produkt und
dessen Anwendungskontexten.

\section*{Empirischer Befund}

Der digitale Wandel stellt die Gesellschaft gleich vor mehrere
Herausforderungen.  Neben infrastrukturellen Erfordernissen stellt eine
veränderte Habitusstruktur erhebliche Anforderungen an eine technologische
Infrastruktur auf semantisch-technologischer Basis und führt zu einer
Veränderung sozio-technischer Bedingungen. Globalisierung und Digitalsierung
sind Schlagworte die auf ein stärkeres Gebot der Kooperation und
internationalen Zusammenarbeit verweisen. Digitale Kompetenzen als Verstehen
der technischen Seite der Entwicklung helfen alleine nicht weiter und brauchen
die Dynamisierung der sowie Eigenverantwortlichkeit für die Prozesse in
theoretischer wir in praktischer Hinsicht. Transkulturelle und
fächerübergreifende Diskurse sind nicht mehr nur zu erstrebende Ziele, sondern
Bedingungen der Möglichkeit einer nachhaltigen technischen und sozialen
Entwicklung. Projektarbeit, welche produktspezifisch die Anforderungen des
Gegenstandes, aber auch des Kontextes verwenden muss, kann heute nur über
Methoden erreicht werden, die bisher nicht berücksichtigten Faktoren und
Zusammenhänge instrumentalisierbar machen. Da eine zentralistische
Instrumentalisierung durch hierarchische und horizontale Organisations- und
Koordinationsformen nicht nur dem technischen Stand zuwiderläuft, sondern auch
dem Selbstverständnis moderner Projektarbeit, ist ein anderer Workflow und ein
dementsprechendes Storytelling für die Projektarbeit unumgänglich.

Dabei können Motivationen geschaffen werden und Grenzen des fachspezifischen
und persönlichen Hintergrundes überwunden werden. Durch ein gemeinsames
Thematisieren und Bearbeiten von Begriffen und Konzepten wird eine
\emph{eigene Story} sowohl über den Gegenstand und das zu entwicklende Produkt
erreichbar als auch eine \emph{Story des Teams} möglich, welche übersetzend
die eigene und gemeinsame Perspektive offen und weiter ausbauen helfen
kann. Die Selbstverantwortung kann zusätzlich durch agile Methoden gesteigert
werden und durch die Entfernung klassischer hierarchischer Ebenen zu einer
noch stärkeren innovativen Eigenleistung geführt werden. Das Problematisieren
möglicher Entwicklungshemmungen kann durch kluges methodisches Herangehen an
widersprüchliche Anforderungssituationen noch weiter gesteigert werden und das
kreative Potential des Einzelnen wie auch der Gruppe aktivieren. Faktorale
Analysen praktischer Widersprüche ermöglicht so die adaptive und dennoch
rekursive \emph{Reduktion} der Produktentwicklungszyklen auf die je
bewältigbare und kontextnahe Erstellung eines Produktes, in welchem sich
sowohl die Erwartungen und Erfahrungen des Auftraggebers als auch des
Projektteams wiederfinden.  Der Kombination von Infradisziplinarität und
Widerspruchsverfahren in agilen Methoden kommt somit nicht allein ein
akademischer Wert oder eine zivilgesellschaftlich besondere Bedeutung zu,
sondern sie kann als Managementmethode einer Kooperation in multilateralen
Projekten in einer technologie-orientierten Wirtschaft Standortvorteile
sichern, ohne direkte Konkurrenzverhältnisse gegen mögliche internationale
zukünftige Partner ausspielen zu müssen. Eine derartige Kombination von
Infradisziplinarität und WUMM ist eine effiziente Organisations- und
Koordinierungsform und gleichzeitig eine Möglichkeit, die kreativen Potentiale
von Menschen zu wecken, die mit ihrer Arbeit direkten Anteil am digitalen
Wandel haben. Innovationen und Innovationsprozesse erscheinen somit nicht
allein als Bedingung der Möglichkeit für ein erfolgreiches Projektarbeiten,
sondern als Befähigung und Stärkung einer offenen und sich globalisierenden
Gesellschaft.

\end{document}
