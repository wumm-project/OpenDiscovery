\documentclass[a4paper,11pt]{article}
\usepackage{od}
\usepackage[utf8]{inputenc}
\usepackage[ngerman]{babel}

\title{Das Wissen des Ichs und die Revolution – Einführung in
  Simulationstheorien des digitalen Wandels }

\author{Ken Pierre Kleemann, Leipzig}

\date{Vorlesung vom 2.5.2017}

\begin{document}
\maketitle

Meine Damen und Herren,

ich bedanke mich für ihr zahlreiches Erscheinen, auch wenn wir es heute mit
einer eher außerplanmäßigen Vorlesung zu tun haben. Dennoch werde ich mit
Ihnen nicht ein außerplanmäßiges Thema besprechen, sondern zwei Wege
verfolgen, die uns einen Anschluss und eine Vertiefung der Probleme der
letzten Vorlesung ermöglichen. Es wurde deutlich, dass eine der großen Fragen
oder eher Probleme des digitalen Wandels die Privatheit ist, nicht nur in
ihrer rechtlich fixierten und fixierbaren Form, sondern sehr wohl als Vollzug
einer Lebenspraxis. Der Umgang mit den stattfindenden Änderungen ist nicht
allein ein Problem rechtlicher Verfahrensweisen und des Schutzes, den man dem
bürgerlichen vertragsschlussfähigen Subjekt zukommen lässt, sondern auch ein
spezifisches Selbstverhältnis dieses ominösen Subjektes und damit auch ein
spezifischer individueller Umgang dieses vermeintlichen Subjektes mit den
„neuen“ Medien.

Wir sahen, dass die Bestimmung als auch die Selbstwahrnehmung stark abhängig
sind vom Menschenbild, das man zur Grundlage des eigenen Verstehens und Tuns
annimmt. Es ist sehr wohl ein Unterschied zu erkennen, ob ich den Menschen als
phantasiebegabt, kreativ spontan, nonverbal assoziativ, als Konstrukteur, als
Künstler seiner Selbst oder aber als Ausdruck seines Milieus, des Systems, der
Umweltbedingungen verstehe. Ersteres verlangt geradezu den Schutz einer
Privatheit und damit einer Privatsphäre, die gegen technologische,
administrative, rechtliche und ökonomische Imperative geschützt werden muss.
Zweiteres verlangt eher eine gewisse technokratische, disziplinierende als
auch Selbstregulation fördernde Struktur. Privatheit in ihrer ersten, niemals
greifbaren Form, erkennt die vermeintliche Spontanität der geistigen
Produktion an, das Konzept des geistigen Eigentums folgt hier aus tiefer
innerer Notwendigkeit. Privatheit in ihrer zweiten Form negiert den
individuellen einmaligen Einfallsreichtum, um ihn als scheinbaren Ausdruck
gegebener Neukombinationen zu begreifen. Das Konzept des freien und jedem
zugänglichen Wissens, der Commons, folgt hier ebenso aus innerer
Notwendigkeit. Beide Möglichkeiten strukturieren und bestimmen nicht nur ein
theoretisches Menschenbild, sondern Gestaltungskämpfe unserer Zeit. Diese
Gestaltungsdiskussionen und Auseinandersetzungen sind es, was uns heute
interessieren wird. Und zwei Wege werden uns sowohl einen Zugang als auch die
Möglichkeit der Einführung eröffnen.

Der Titel der heutigen Vorlesung ist deswegen mit Bedacht gewählt, um uns
sowohl einen Blick zu eröffnen als auch die Gewinnung einer gewissen Übersicht
zu ermöglichen. „Das Wissen des Ichs und die Revolution“ soll uns zwei
Einstiege, zwei Wege, zwei Probleme und zwei Perspektiven an die Hand
geben. Das \emph{Wissen des Ichs}, Sie mögen die Formulierung entschuldigen,
soll uns aufmerksam machen auf die immer wieder zu beleuchtende und zu
problematisierende Grundlage der folgenden Theorien, Stimmen und Positionen.
Die \emph{Revolution} soll uns aufmerksam machen sowohl auf das Erbe, das
explizit und implizit tradiert wird, als auch auf vielleicht zu große
Erwartungen, welche als Grundlage mit dem Menschenbild verbunden werden. Auf
der einen Seite werden wir versuchen, das Menschenbild zu extrahieren und zu
problematisieren, auf der anderen Seite werden wir versuchen, den praktischen
Moment dieser Annahme herauszustellen. All diese Zugänge werden sich um einen
Begriff drehen, der in den unterschiedlichsten Farben und Konnotationen
auftritt: die Simulation.

Die Kybernetik wurde in den sechziger Jahren von Wiener als Wissenschaft der
Selbstregulation von Systemen beschrieben, hier insbesondere als
Datenverarbeitungssysteme. Die Simulation ist nun deren eigene Arbeitsweise
wie auch das eigentliche Experiment dieser Wissenschaft. Simulation ist die
Bearbeitung von Fragen und Problemen, welche nicht nur an das System gestellt
werden, sondern sich durch den Eigenbetrieb aus den verarbeiteten Daten
ergeben. Die Kybernetik reduziert somit nicht nur den Menschen auf einen
quantitativ berechenbaren Algorithmus, sondern gleich die ganze menschliche
Gesellschaft wie auch deren spezifischen Gesellschaften. Für
Sozialwissenschaftler ist die autopoietische Betrachtung der Gesellschaft,
spätestens seit Luhmanns Arbeiten, kein ferner Vorgang von Algorithmen und
Rechenmaschinen mehr. Simulation wird hier nicht nur als die quantitative
Berechnung einer reduktiven Wissenschaft verstanden, sondern als die
notwendige Selbstdarstellung der Dynamik des Systems.

Ursprünglich warnten Kybernetiker in Ost und West, Georg Klaus wie auch Karl
Steinbuch, vor einer derartigen Adaption und Überschätzung des kybernetischen
Ansatzes, nur um selbst den Vorwurf des vergessenen methodischen
Reduktionismus auf sich zu ziehen. Kybernetik wurde zum Inbegriff einer
Verrechnung des Menschen und zur Ideologie eines instrumentellen
Spätkapitalismus, egal, ob in liberal-westlicher oder in
planerischer-östlicher Form, wie es nach Pollock heißen wird. Weizenbaum
fasste in dieser Zeit diesen Vorwurf kurz und eloquent zusammen: Die Macht der
Computer, die Ohnmacht der Vernunft. Die eigentliche geistige Freiheit des
Menschen lasse sich nicht durch eine algorithmische quantitative Reduktion
fangen, mehr noch zerstöre sie den eigentlichen Wesenskern des Schöpfers der
Maschine. Eine technologisch-technokratische Revolution, wie sie Kybernetiker
angeblich anstrebten, zerstöre nicht nur den Kapitalismus nicht, sondern
stelle erst eine simulierte Freiheit her, welche den quantitativen und
instrumentellen Aspekten der späten Neuzeit entsprechen sollte. Simulation
wird zum Begriff für einen technischen Vorgang wie auch zu einem Begriff der
gesellschaftlichen Verblendung.

Henri Lefevbre konnte in der „Metaphilosophie“ 1965 schon sagen, dass
Kybernetik die Ideologie dieser Zeit sei, insoweit sie durch ihre Art und
Weise der experimentellen Simulation die Simulierung einer befreiten,
technologisierten Menschheit vorspiele. Die Simulation als Experiment und
innere Notwendigkeit der Datenverarbeitungssysteme erzeuge die technokratische
Simulation einer besseren und gelenkten Gesellschaft wie auch einer
effizienteren und flexibleren Menschheit. Im Gegensatz aber zu Weizenbaum, der
den Untergang des nie zu verrechnenden Wesens des Menschen beklagt, trennt
Lefevbre konsequent zwischen der Kybernetik als Wissenschaft und der
Kybernetik als Ideologie, eher wird Luhmann zum Problem als Steinbuch. Für
Lefevbre ist die Annahme eines nonverbalen, unlogischen, kreativen
Schöpfungsereignisses selbst Ausdruck einer bürgerlich-kapitalistischen Logik,
die den Menschen nicht durch algorithmische Quantifizierung verblendet,
sondern durch bürgerlich-naturrechtliche Verfahrensweisen. Die Annahme einer
freien, schöpferischen Kreativität, wie sie Weizenbaum verwendet, wird selbst
zum Ausdruck eines Formalismus, welcher den Menschen gerade keine inhaltliche
Aufladung von Grundrechten geben will.

Lefevbre schrieb dies unter den Vorzeichen des Mai 1968, welcher in Nanterre
und damit im eigenen Arbeitsfeld dieses Mannes deutlich eine Revolution
vorspielte. Spielte? Sehr wohl bezog man sich auf Guy Deboers „Gesellschaft
des Spektakels“, um weder mit bürgerlich-rechtlichem Reformismus noch mit
marxistisch-leninistischer Revolution verwechselt zu werden. Gegen die
Simulation bürgerlicher Freiheitsrechte wie auch gegen die Simulation einer
planmäßig gelenkten Funktionärsherrschaft wurden Sprüche wie: „Die Fantasie an
die Macht“; „Traum ist Wirklichkeit“; „Kunst existiert nicht, Kunst bist du“
an die Wände der Universitäten geschrieben. Weder eine wesensschützende
Erklärung der schöpferischen Möglichkeiten des Menschen noch eine
quantifizierte Verrechnung sollten und könnten der Weg für eine ganz andere
Revolution sein. Gar für die Revolution, welche nun endlich mit den Idealen
der eigentlichen, der französischen Revolution nach fast zweihundert Jahren
ernst machen würde. Der Geist von 68, eine Revolution jenseits von Reform und
Revolution?

Nun wird Geschichte von Menschen gemacht, allerdings hängt sie nicht immer vom
expliziten Wollen dieser Menschen ab. Der Ost-West-Gegensatz ist selbst Opfer
einer Revolution geworden. Die Kybernetik ist Opfer ihrer eigenen Revolution
geworden. Der Mensch das eigentliche Opfer? Fast fünfzig Jahre später ist die
Welt eine andere, aber die Ebenen der Diskussionen und der Missverständnisse
haben sich kaum verändert.

Seit dem Ende der 1970er Jahre und endgültig in den 1980er Jahren sehen wir
das Aufkommen der Heimcomputer, der Personal-Computer, der Spielekonsolen und
letztlich der Mobiltelefone. Datenverarbeitung ist nicht mehr nur
Verarbeitung von Daten mit Lochkarten, sondern mit Bits und Bytes, durch
Programmiersprachen und Visualisierungen eine neue Form der experimentellen
Simulation, einer selbstbezogenen Simulation, welche nicht nur durch
Kybernetiker eingepflegte Daten verarbeitet, sondern durch die Einbeziehung
der Konsumenten umgeformt und auf sich selbst zurückgeworfen wird.

Gibt es heute noch Kybernetiker? Im eigentlichen Sinne des Datenverarbeiters
wird es schwer einen zu finden. Im Silicon Valley findet sich neben dem
Programmierer oder Projektkoordinator der Technikphilosoph, der
transhumanistische Theoretiker. Ray Kurzweil ist heute Chef of Engineering bei
Google. Mit einschlägigen Büchern wie „The Age of Spiritual Machines“, „The
Singularity is Near“ oder „How to Create a Mind“ sehen wir in gewisser Weise
den angeblich unreflektierten Traum der Kybernetiker und Systemtheoretiker
wieder auferstehen.

Für Kurzweil ist die Welt ein Informationszusammenhang hochkomplexer
Musterbildung und damit folglich das menschliche Gehirn ebenfalls. Menschen
produzieren jetzt künstliche Intelligenzen, welche nicht wie Weizenbaums Eliza
mit vorgegebenen Algorithmen versuchen menschliche Reaktionen zu imitieren,
sondern wie Siri oder Alexa menschliches Verhalten imitierend
interpretieren. Nicht mehr die eingegebenen Daten des Kybernetikers werden
verwendet, sondern die durch hochkomplexe und selbstverweisende Protokolle
gestützten Daten des Internets. Der Mensch hat nicht einfach eine Simulation
seines Abbildes geschaffen, sondern die Simulation dieser Simulation. Es ist
für Kurzweil nur eine Frage der Zeit, einmal 2029, einmal 2045, wann es zur
Singularität kommt; zur Verschmelzung der handlungsmorphen K.I. mit den
handelnden Wesen, die diese konstituieren und ständig
verändern. Transhumanismus erscheint hier nicht nur als Gefahr, sondern als
unumgängliche Revolution. Eric Schmidt, der CEO von Google, brachte es auf die
resignative, fast schon pessimistisch zugespitzte Formulierung: „Wenn wir die
Privatheit nicht schützen, so werden wir sie verlieren.“ Und für Kurzweil als
auch für Schmidt ist diese Revolution nicht aufzuhalten, aber eine rechtliche
Verfassung, insbesondere privatrechtlicher Maßstäbe, soll von diesem
wirtschaftlichen globalen Player nicht gefährdet werden. Mag zwar der Mensch
nur ein hochstufiger selbstbezogener Algorithmus sein, ein Produkt seines
Milieus, seiner Umwelt, so hat er aber in seiner Geschichte einen
Verblendungszusammenhang erarbeitet, ein Wissen seiner Selbst, welcher heute
erst den Betrieb eines Unternehmens wie Google rechtlich, produktiv und
distributiv möglich macht. Die Revolution des Transhumanismus macht einen
Reformismus nötig, der die strukturell-komplexen Grundlagen des Menschen in
bürgerlich-rechtlicher Form anerkennt und andererseits diese bürgerlichen
Illusionen mit diesem Bild der technologischen Entwicklung vereinigt. Einen
Reformismus einer informierten Elite, oder mit Jason Brennan; vergesst die
aktuelle Demokratie und ihre postfaktische Einfältigkeit und lasst eine
\emph{Epistokratie} an die Macht; also eine elitäre, technokratische
Revolution, um den Reformismus gegen die transhumane Revolution zu schützen?

Jaron Lanier fand in seinem Insiderbuch über und aus dem geistigen Klima des
Silicon Valley „Wem gehört die Zukunft“ den wohl passendsten Titel für die
Postkybernetiker: „Digitale Maoisten“.

Spätestens durch Snowdens Enthüllungen sind die Möglichkeiten und Gefahren
einer vermeintlichen technokratischen Elite und einer Opferung des Menschen
im Fokus des Feuilletons wie auch im Fokus der akademischen Diskussionen
angelangt. 2014 gestaltete Frank Schirrmacher für die F.A.Z. eine
Auseinandersetzung über die problematischen Ansätze des Silicon Valley wie
auch über die problematischen Änderungen im Allgemeinen unter dem vielsagenden
Buchtitel: „Technologischer Totalitarismus“. Über Martin Schulz, Sigmar
Gabriel, Hans Magnus Enzensberger, Ranga Yogeshwar bis Karin Göring-Eckardt,
Sascha Lobo und Jaron Lanier reicht die noch nicht vollständige Aufzählung
der prominenten Autoren. In einer fast schon seltsam anmutenden Einstimmigkeit
wird, wie es Weizenbaum tat, die nie verrechenbare Qualität des menschlichen
Wesens gegen die gefährliche Überschätzung der Technifizierung in Anschlag
gebracht. Die Privatheit zu schützen ist nicht nur ein bürgerlich-rechtliches
Anliegen, um demokratische Verfahrensweisen zu schützen, sondern geradezu
humanistisch-ethische Verpflichtung. Schirrmachers „Schülerin“ Yvonne
Hof-stetter lässt „Das Ende der Demokratie – Wie die künstliche Intelligenz
die Politik übernimmt und uns entmündigt“ nicht nur als Titel ihres Buches
erscheinen, sondern als allgemeines Problem stehen, welches in dieser
feuilletonistischen Debatte immer wieder thematisiert wird. Allgemein wird
hier ein kreativer, nonverbaler, unlogischer, sich den Algorithmen
entziehender Kern von Fähigkeiten des Menschen in Stellung gebracht, ja fast
schon als Schützengraben instrumentalisiert. Der Gegner ist klar – der
Transhumanist oder die technokratischen Eliten, welche einen technologischen
Totalitarismus, wenn schon nicht fordern, so doch Vorschub leisten. Das Wissen
des Ichs soll mehr sein als hochkomplexe Musterbildung und gerade die
unlogische unformalisierbare Grundebene des menschlichen Wesens thematisiert.

David Gelernter, noch ein enger Freund des konservativen Schirrmacher, hatte
mit „Gezeiten des Geistes“ dieser Position des Feuilletons nicht nur
monographische Ehre bereitet, sondern sie vor Kurzem in akademische Kreise
getragen. Als ob es einen Freud, Jung, Adler, Fromm, Reich oder Lacan nicht
gegeben hat, revitalisiert Gelernter eine Theorie des Unterbewussten, welche
seit Eduard von Hartmann, also seit Ende des 19. Jahrhunderts, eigentlich als
erledigt angesehen werden konnte. Die unterste Stufe des menschlichen
Erlebens, gar des Unbewussten, ist hier die voraussetzungslose Kreativität,
die nonverbale, nicht formalisierbare Phantasie. Mathematisierung, gar
Technifizierung, ist hier nur eine Simulation, welche höhere
Aufmerksamkeitsgrade auf die eigentliche Stufe des menschlichen Erlebens
aufpfropfen. Und so wie das Gehirn sich eine quantifizierbare Gestaltbarkeit
der Welt simuliert, so simuliert die Technik als Ausdruck dieser Verblendung
eine technokratische Beherrschbarkeit. Hier müssen bürgerliche Werte nicht als
historische Errungenschaften geschützt werden, sondern als natürliche. Der
Mensch ist natürlich hier ein Schöpfer ohne festsetzbare Voraussetzungen.
Rutschten die Transhumanisten in ein angebliches Menschenbild des Systems, des
Milieus, der Umweltbedingungen, so wird hier explizit das Ich als frei und
schöpfend gesetzt.

Kybernetik und Weizenbaum, Kurzweil und Gelernter, welche Wiederholung einer
alten Auseinandersetzung. Gegen die implizite Revolution der Technokraten,
die nur reformistisch vor der digitalen Revolution schützen wollen, braucht es
hier fast schon die Revolution der Besinnung auf die bürgerlichen
Verfahrensweisen. Für beide Seiten ist das Wissen des Ichs und die Revolution
eine Machtfrage, eine Frage der Willensdurchsetzung gegen Widerstände. „Die“
Revolution bleibt aus, denn es läuft schon eine, die eine andere nötig
macht. Ein Geist der Vor-68er?

Die vermeintliche Degradierung des Menschen zur komplexen Rechenoperation wie
auch die Überhebung zum schöpferischen Genius konstatiert eine individuelle
Natur. Diese Naturalisierung durch quantifizierende Algorithmen wie auch die
Naturalisierung der bürgerlich-rechtlichen Privatheit haben Hardt und Negri in
einem fulminanten Angriff als eigentliches Problem ausgemacht und diesem
Vor-68er-Geist den Krieg erklärt. Beide Strategien erscheinen als
gesellschaftliche Probleme eines Kapitalismus, der vom alten Imperialismus zum
Empire übergegangen ist. Machtfragen ließen sich je nach der einen oder der
anderen Seite bearbeiten, als der Mensch noch in seiner individuellen,
körperlichen Form diszipliniert und selbstreguliert werden konnte. Doch die
neue Form der Globalisierung und digitalen Technifizierung erzeugt nicht nur
die Simulation der Verrechnung oder unabhängigen Kreativität, sondern
verwandelt das Subjekt selbst in Virtualität. Entgrenzung, Zeitgleichheit und
Beschleunigung führen aber nicht nur zu einem fluiden und transidentitären
Wissen des Ichs, sondern die Gesellschaft selbst ist zentrumslos und
dynamisch. Gestaltungskämpfe sind hier keine Machtfragen im Sinne
individueller Widerstandsbeseitigung, sondern intersubjektive Probleme von
Klassen, welche mit marxistischen Klassen im klassischen Sinne nichts zu tun
haben. So virtuell wie das Subjekt ist, so virtuell ist die Gesellschaft in
der Wissensproduktion, welche weder dem Technokraten noch der Phantasie
untersteht. Wissen des Ichs ist gesellschaftliches Wissen, welches nur als
Commons, als Allemende jenseits systemischer Imperative oder genialer Einfälle
funktioniert. Die Politik der bürgerlich-rechtlichen Verfahrensweisen wie auch
das Politische einer sich immer neu austarierenden Zivilgesellschaft
verschwindet hier, sobald die Revolution der virtuellen Subjekte, die
Multitude, ihre simulierte Gesellschaft auf Basis geteilter Commons
simuliert. Die Simulation der Simulation hochkomplexer selbstbezüglicher Daten
ermöglicht die reale Simulation der bisher versprochenen oder verweigerten
Freiheiten des Menschen aus der verblendeten Simulation. Der Mensch als System
und der Mensch als Genie werden hier durch ein intersubjektives Menschenbild
ersetzt, welches in seiner gesellschaftlichen Fundierung weder durch
individuelle Verrechnung noch durch individuelle Kreativität gefasst wird. Die
Revolution ist hier eine notwendige Folge der gesellschaftlichen Natur des
Menschen. Der arabische Frühling erschien nicht ohne Grund diesen Denkern als
der Anbeginn des Endes des Empires, die Multitude bewege sich ja
schon. Natürlichkeit nur intersubjektiv gefasst? Systemisch hochkomplexe
Eigenverrechnung als gesellschaftliches Trägheitsmoment verstanden, welche
immer schon \emph{gesellschaftliche} Wissensproduktion ist?

In gewisser Weise sehen wir hier eine technokratische Variante auf einem
anderen Niveau. Natürlich werden keine informierten Eliten gefordert, aber nur
der Protestler, der diese simulierten Natürlichkeitssimulationen durch seine
eigene Virtualität begreift, hat sie damit auch überkommen. Die digitale
Revolution bringt hier weder die Revolution der Eliten, um einen Reformismus
vor der digitalen Revolution zu schützen hervor, noch die Revolution der
Denkungsarten, sondern eine Revolution, welche aus dem gesellschaftlichen
System selbst entstehen muss und wird. Die neuen Eliten sind keine, da ihre
Wissensproduktion und damit das Wissen ihres Ichs selbst den
gesellschaftlichen Bedingungen der Wissensproduktion unterliegen. Jenseits von
geistigem Eigentum oder freier Software gilt, dass Wissen nur aus
\emph{geteiltem} und \emph{freiem} Wissen in der Gesellschaft produziert
werden kann. Weder der Technokrat noch das Genie produzieren Neues, sondern
die Gesellschaft als Multitude im Ganzen. Der Mensch ist als System Ausdruck
des Systems, der Mensch als umweltbedingte Variante auf höherer Stufe.

Gibt es damit nun auch die Variante der voraussetzungslosen Phantasie auf
anderem Niveau?

Tatsächlich könnte man sagen, es existiert diese Richtung, verbunden mit
Lefevbres Schüler Jean Baudrillard. \emph{Simulakra} ist der Begriff, welcher
nicht nur als theoretische Kategorie auftaucht, sondern geradezu als
Kampfbegriff. In den letzten Jahren erfährt dieser Situationist eine
Aufmerksamkeit, aber auch eine Interpretationsgewalt, welche sich selten mit
seinem Werk deckt.  „Das System der Dinge“ als auch „Der symbolische Tausch
und der Tod“ erfahren, insbesondere in den letzten drei Jahren, ein immer
größer werdendes Interesse. Und seltsamerweise sowohl aus einer linken
Position, welche Hardt und Negri, so könnte man sagen, weiterdenkt, als auch
aus einer rechten Ecke, welche mit Breitbart Baudrillard instrumentalisieren
will, um die vermeintliche Hegemonie der sogenannten Fake-News verbreitenden
Globalisierer zu entlarven.

Für Baudrillard stellt sich die Welt der Dinge als System von Zeichen dar.
Nicht als symbolische Formen, wie sie Cassirer benutzte, sondern als
historisch generiertes System von Signifikat und Signifikant, wobei das
Signifikat selbst schon als intersubjektiver, fast schon implikativer
Signifikant auftritt. Die Mythen des Alltags werden für Baudrillard, mit
Barthes, mehr sein, als nur referenzielle Namensgebungsgeschichten von
sichtbaren Dingen. \emph{Natürlich} ist hier nichts, denn selbst direkte
Referenzen sind gesellschaftlich und historisch bedingt. Die Zeichen
simulieren immer schon einen Sinn, der sich aus den gesellschaftlichen
Prozessen ergibt. Die Verblendung des Menschen ist immer schon sprachlich und
damit performativ bedingt und ist Produkt wie auch Bedingung der jeweiligen
Stufe der gesellschaftlichen Differenzierung und Arbeitsteilung. Die Neuzeit
zeichnet allerdings ein besonderer Umstand aus: die selbstreflektorische und
selbstregulierende Anwendung und Rückbeziehung dieses
Simulationszusammenhangs. Die Neuzeit verwendet diese Simulation, diese
Verblendung, um sich selbst zu begreifen und selbst zu gestalten. Es ist eine
Ordnung, welche simulierend diesen Simulationszusammenhang auf sich bezieht,
eine Ordnung der Simulakren.

Die erste Ordnung der Simulakren reicht nach Baudrillard von der Renaissance
bis zur französischen Revolution. Diese erste Ordnung beruht auf dem Schema
der Imitation, sie handelt vom Naturgesetz des Wertes, sie folgt der
Demokratie der Konkurrenz. Es ist eine Imitation des Menschen, theatralisch,
mechanisch und wie ein Uhrwerk, hier gehorcht seine Technik der Analogie. In
der zweiten Stufe der Simulakren steigert sich der Zusammenhang zur
Äquivalenz, die Produktion ist das bestimmende Schema des industriellen
Zeitalters. Die Ordnung wird bestimmt vom Marktgesetz des Wertes, das freie
Spiel der Konkurrenz ist zur Kalkulation von Kräften geworden. Das Simulakrum
zweiter Ordnung ist operativ. Das Simulakrum der dritten Ordnung nun ist
operational. Nicht mehr Imitation oder Produktion bilden das Schema der
dritten Ordnung, der jetzigen Ordnung, sondern Simulation selbst. Das
Strukturgesetz der Werte ersetzt das Naturgesetz wie auch das Marktgesetz der
Werte. Es gibt keine Imitation des Originals mehr wie in der ersten Ordnung,
aber auch keine Serie mehr wie in der zweiten Ordnung: Es gibt Modelle, aus
denen alle Formen durch leichte Modulation von Differenzen hervorgehen. Nicht
mehr Imitation oder Produktion, sondern Reproduzierbarkeit ist der Kern einer
Ordnung der seriellen Simulation. Interdeterminismus und Code sind die neuen
Zauberwörter einer Ordnung, welche nicht mehr nur Simulation der Simulation,
das Simulakrum imitiert oder produziert, sondern das Simulakrum selbst
simuliert. Die Realität ist zur Hyperrealität mutiert, verkommen oder
geworden. Die Variante der systemischen Betrachtung des Menschen wie auch die
frei kreative Variante sind Ausdruck einer Ordnung, welche heute in bipolarer
Ausprägung nur die oligarchischen Grundlagen der Gesellschaft verschleiert
und selbst reproduziert. Mehr noch erscheint auch die andere Ebene, welche von
Machtfragen und individuellen Natürlichkeiten auf eine gesellschaftliche
Perspektive geht, also Hardt und Negri, hier nur als eine Seite einer höher
geschraubten, aber interdeterministischen Bipolarität. Systemische Betrachtung
als auch voraussetzungslose Phantasie auf gesellschaftlicher Ebene erscheinen
hier nur als simulierte Höhersetzung des simulierten Simulakrum. Die
Vorstellung der Multitude ist hier genauso Ausdruck einer unentrinnbaren
Ordnung wie eine technokratische Elite oder ein vernünftiger
Verfassungspatriotismus.

Dies erscheint pessimistisch-apokalyptisch und lässt liberale, sozialistische oder avantgardistische Lösungen in einem Licht der Selbstbefangenheit erscheinen oder gar schlimmer als unbewusst böswillige Verstärkungen der angenommenen Hyperrealität. In dieser Leseart wird es fast notwendig, einen Mythos des 21. Jahrhunderts zu fordern, der jenseits von rechts und links agiert. Nicht ohne Grund wird diese fast schon mystische, nicht nur als voraussetzungslose Phantasie, sondern als extatisch-aktionistisches Erleben für die Neue Rechte interessant. Kann man doch nun mit einem dystopischen Baudrillard jeglichen Fakten-Bezug bestreiten und sogar noch den schöpferischen Führer konstruieren. Weg mit dem Geist von 68.
Es dürfte klar sein, dass dies eine unhaltbare Leseart ist. Für Baudrillard
gibt es eine Alternative; die Poesie und den symbolischen Tausch. Doch sollte
man sich hüten, hier nun die zu erwartende Variante der voraussetzungslosen
Phantasie auf gesellschaftlichem Level zu entdecken. Für Baudrillard ist die
Poesie ein Spiel mit Zeichen, welches durch dieses Spielen seine Referenz wie
auch alle Inferenzen aufhebt, indem es diese aktiv zerstört. Der symbolische
Tausch teilt weder Werte noch Begriffe, sondern vernichtet diese und
hinterlässt eine unaussprechbare Hypothek, die Hypothek einer gemeinsam
geteilten Welt. Weder eine Revolution der Eliten, noch eine Revolution der
Denkungsart, noch eine Revolution der Multitude oder eine Revolution eines vom
Schicksal geküssten Führer lässt sich hier begründen. Poesie und Witz sind
Subversionen durch Revisionen. „Allein jene Subjekte, die wie die Wörter ihre
Identität aufgeben, sind der sozialen Reziprozität im Lachen und im Genuss
geweiht“, heißt es am Ende des Buches „Der symbolische Tausch und der
Tod“. Baudrillard starb 2007 ohne Revolution, mit Revolution und gegen „Die“
Revolution.

Meine Damen und Herren, dieser kleine Ausflug zeigt uns mehrere Punkte.
Erstens wird das Problem der Privatheit weiter eine der großen
Herausforderungen bleiben und die Gestaltungskämpfe nicht nur unserer
Generation fesseln, sondern hoffentlich auch der nach uns kommenden.

Zweitens dürfte es klar geworden sein, dass die Bestimmung dieser vermeintlich
einfachen Kategorie nicht nur abhängig ist vom verwendeten Menschenbild,
sondern auch davon, auf welcher Variante und Ebene dieses verortet wird. Das
Wissen des Ichs ist eine ungeklärte und unabgegoltene Problematik.

Drittens hoffe ich, sie konnten erkennen, dass diese Entscheidung nicht nur
eine theoretische ist, sondern eine zutiefst politische. Welche Reform,
Revolution oder Gestaltung sich findet und einstellt, ist abhängig von unserem
Wissen, von unserer Vorstellung des Ichs, unseres eigenen Ichs.

Viertens lässt sich ein Spalt, ein Licht in der Ferne, ein offener Abgrund
erkennen, welcher nicht mit Identität, Identitäten oder ähnlichem operieren
muss. Vielleicht sollten wir tatsächlich mehr lachen in Anbetracht von
Ängsten, vermeintlichen Ängsten oder jeglicher irrationaler Rationalitäten.

Fünftens zeigen die Diskussionen der 1960er Jahre als auch der heutigen Zeit
eine unheimliche Parallelität, wenn nicht gar eine flache
Wiederholung. Seltsamerweise scheint Geschichte still zu stehen, wenn es um
den Menschen und um das vermeintlich Andere seiner selbst geht; die
Technik. 2011 erlebten wir den arabischen Frühling. Zum ersten Mal bewegten
sich soziale Proteste über Ländergrenzen hinweg, getragen von sozialen
Netzwerken, welche erst seit 2007, also mit Baudrillards Tod, durch das
Smartphone direkt koordiniert werden konnten. Wir erlebten 2014 die Gründung
des sogenannten Kalifats des I.S. Der Terrorismus, welcher für Baudrillard
noch staatenlos sein musste, um Ausdruck der Hyperrealität zu sein,
koordiniert sich selbst zum Ordnungsraum, getragen und verwaltet durch soziale
Netzwerke und die neuen Gadgets, welche dieses Jahrzehnt erwartbar aus den
Angeln hebt. Wir erlebten 2016 Brexit und Trump. Die Hoffnung der
Multitude-Revolution wurde beendet durch dieselben Strukturen und Geräte, die
diese Hoffnung 2011 erst möglich machten. Man rettet sich in postfaktische
Emotionalisierungserklärungen und sieht nicht das Aufbrechen alter tradierter
Denkungsweisen durch eine Interoperationalität der Geräte, Anbieter und
Nutzer, welche eben nicht nur postkonventionelle Moral unterstützen, sondern
auch fast schon vergessene rückwärtsgewandte Sittenkonformität.

Ungleichzeitigkeit der Entwicklung, interne Spaltungen der Gesellschaft, nicht
nur in Schichten oder Klassen, sondern in Unterklassen und die Hoffnung auf
„Die“ Revolution waren und sind die eigentlich problematischen Felder dieser
Analysen und politischen Gestaltungen. Geister des Geistes von 68.

Rechts und Links sind nicht tot, sondern umkämpfter als je und unschärfer den
je. Der Geist und seine Geister spuken weiter.

Vielleicht ist der digitale Wandel heute, im Jahr 2017, in der Lage, sich
nicht als „Die“ Revolution zu begreifen und dennoch den Weg für Künftiges –
reflektorisch als auch aktionistisch – offen zu lassen. Schützen wir nicht nur
die Privatheit, sondern auch die Möglichkeit, über derartiges zu lachen.

Ich bedanke mich und bin gespannt auf ihre Fragen und Anmerkungen.

\end{document}
