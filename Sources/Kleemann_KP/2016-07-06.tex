\documentclass[12pt,a4paper]{article}
\usepackage{lifis}
\usepackage[utf8]{inputenc}

\title{Tragödie der Soziologie oder die Schwierigkeit der Interdisziplinarität}

\author{Ken Pierre Kleemann, Leipzig}
\def\theauthor{Ken Pierre Kleemann}
\date{06.\,07.\,2016}
\begin{document}
\maketitle

\begin{quote}
  Eine Replik auf Alfred Fuhrs Beitrag \emph{„In Parasocial Media we
    Trust“}\footnote{Alfred Fuhr (2015). Das Abenteuer des medienvermittelten
    Vertrauens in der Hochmoderne. Parasozialität in Big und Smart Open Data
    Public Relations.  Vortrag im 8.~Interdisziplinären Gespräch „Wege des
    digitalen Wandels“ am 30.\,01.\,2015 an der Universität Leipzig.
    \url{http://mint-leipzig.de/2015-01-30.html}. Siehe auch Alfred Fuhr
    (2016). Das Abenteuer des medienvermittelten Vertrauens in der
    Hochmoderne. LIFIS ONLINE [05.\,07.\,2016].}  zum 8.~Interdisziplinären
  Gespräch „Wege des digitalen Wandels“ am 30.\,01.\,2015 am Institut für
  Informatik der Universität Leipzig.
\end{quote}

\subsection*{Die Schwierigkeit der Interdisziplinarität}

Aus den Kathedralen der hohen Bildung ist seit Jahren die Forderung nach dem
Ausbau der gemeinsamen akademischen Tätigkeit zu vernehmen. Lehre und Forschung
im 21.~Jahrhundert sollen nicht nur in Anbetracht der ach so neuen
Globalisierung stärker verzahnt, sondern überhaupt in ein engeres Verhältnis
gebracht werden, weil allein schon das „Neuland“ der Digitalität eine stärkere
Kombination der Fächer fordere. „Die Technik“ verlange eine intensivere
Zusammenarbeit nicht nur auf der Ebene sich erschließender Möglichkeiten,
sondern auch auf der Ebene einer geradezu erscheinenden Selbstverständlichkeit
von Entwicklung.

Tatsächlich verändern sich die Lebensweisen und somit die Universität als
Ausbildungs-, aber auch als Reflexionsstätte. Und tatsächlich ist leider die
verständliche \emph{Forderung} nach Interdisziplinarität nicht dasselbe wie die
\emph{real stattfindende} Arbeit. Es ist zwar ein positives Zeichen, dass
einschlägige Konferenzen immer häufiger stattfinden und somit, so sollte man
meinen, die Möglichkeiten des Austauschs von Argumenten vermehrt werden, doch
sind Präsentieren und Miteinander-Reden noch immer zwei unterschiedliche
Tätigkeiten. Interdisziplinarität ist und darf sich nicht einfach auf das
Präsentieren von noch so guten Argumenten reduzieren, sondern muss vor allem
auf die \emph{gemeinsame} Arbeit an \emph{gemeinsamen} Inhalten und damit auf
die Arbeit an den Begriffen und ihren jeweiligen Bedeutungen gerichtet sein.
Infradisziplinarität, wie Lorenzer schon in den 1970ern sah, umfasst das
gemeinsame Lernen und somit das Verändern der eigenen Position. Argumente
auszutauschen ist zu wenig, wenn es zu keiner Änderungen der eigenen Argumente
kommt. Lehre und Forschung im 21.~Jahrhundert benötigen die Auseinandersetzung
mit den neuen Möglichkeiten, aber auch das reflektierte Verstehen der
Veränderungen. Beides kann nur im kollegialen Arbeiten bewerkstelligt werden.

Fuhr ist zu danken, dass er sich die wertvolle Zeit der institutionellen
Einbindung frei nimmt, um den Forderungen der Institution nachzukommen. Es ist
bis heute ein Problem, jenseits der etablierten Konferenzen einen
Argumentenaustausch zu Wege zu bringen, insbesondere dann, wenn es um eine
Verknüpfung von MINT-Fächern und geisteswissenschaftlichen Professionen geht.
Eigentlich sollte es in Anbetracht der impliziten Annahmen, die in einem Bild
wie der „digitalen Globalisierung“ transportiert werden, selbstverständlich
sein, eine derartige Kooperation zustande zu bringen. Doch gilt wohl auch hier
das alte Shibboleth der Philosophie, dass Realität und Wirklichkeit nicht eins
sind.

Fuhr ist weiterhin zu danken, dass er nicht nur alternative soziologische
Theorien benutzt und verbreitet, sondern gerade solche verwendet, die sich den
spezifischen Problemen der heutigen gesellschaftlichen Entwicklung stellen
wollen. „Es gibt bis heute keine wissenschaftliche, empirisch exakte
konsistente Soziologie der Massenmedien – das ist der eigentliche
Forschungsskandal“, meint Fuhr und verweist auf Konzepte der Parasozialität,
eine „andere Theorie der Internetnutzung“.

Ein derartiger Ansatz verweist auf zwei problematische Punkte und ist damit
sowohl Kritik als auch Theorie. Gesellschaftstheorie begreift sich hier als
Gesellschaftskritik, welche mehr als nur negativ-kritisches Kritisieren sein
möchte, denn sie will Momente der Veränderung positiv greifbar machen.

\subsection*{Grenzen einer Bewusstseinsphilosophie}

Erstens, und dies wird bei Fuhr sehr deutlich, ist die Annahme eines
\emph{Ich-Kerns}, eines \emph{homo oeconomicus}, eines \emph{homo faber} oder
einer sonstwie geprägten „nach außen abgeschlossenen Instanz in uns“
unzureichend, wenn nicht gar ein hinderliches Paradigma. „Dieser Typ des
Selbsterlebnisses und der Individualisierung [ist] aber selbst etwas
Gewordenes, Teil eines sozialen Prozesses {\ldots}.“ In diesem Prozess wird der
„Zusammenhalt von Menschen aber eher als ein Ergebnis von Kommunikation“
hergestellt, also letztlich, so Fuhr, durch den Austausch von
Informationen. „Unsere Lebensbereiche {\ldots} sind durchdrungen von
Informationstechnik“.

Es ist kaum verwunderlich, dass Fuhr in Anbetracht einer nunmehr fast
fünfzigjährigen Tradition der expliziten Kritik an den Versuchen, ein
abgeschlossenes Subjekt zur Grundprämisse gesellschaftlicher Kritik und
Theoriebildung zu machen, an einer heterogenen Diskussion teilnimmt. Nicht nur
die so oft verpönte Postmoderne hat ihr Augenmerk auf das Menschenbild gelegt,
sondern auch der philosophische Soziologe oder soziologische Philosoph
Habermas. In geradezu meisterlicher Erzählweise erfahren wir bei diesem
löblichen interdisziplinären Versuch, dass die Bewusstseinsphilosophie, die
gerade den Ich-Kern voraussetze, nicht nur das Problem der zeitgenössischen
Wissenschaft ist, sondern das Grundproblem der Moderne selbst darstellt. Die
Kommunikation als intersubjektive Tätigkeit müsse zur Grundlage der
Handlungstheorie werden, also eine praktische Bestimmung des Wesens des
Menschen darstellen. Kant, Hegel und selbst noch Marx waren zu verfangen in
diesem alten solipsistischen Bild, und die analytische Philosophie, mit einem
Schuss Max Weber, ist geradezu eine Antwort auf die sinnentleerte Moderne, auf
die technische Entfremdung und systemische Kolonisierung der kommunikativen
harmonischen Lebenswelt.

\paragraph{1.} 
Der Versuch von Habermas, die „kontinentale und angelsächsische“ Philosophie zu
vereinen, ist ein umfassender und inspirierender Ansatz. Im Allgemeinen ist es
anzuerkennen, dass der Versuch unternommen wird, das vielleicht bisher zu enge
Menschenbild den intersubjektiven Tätigkeiten dieses Wesens näher zu bringen.

\paragraph{2.}
Kommunikation ist der wohl entscheidende Schlüssel, darf aber nicht auf einen
idealen Sender-Empfänger-Austausch reduziert werden. Die Searlsche
Sprachakt-Theorie ist gegenüber dem Austinschen Grundansatz schon defizitär und
wird mit der emanzipatorischen Einführung der Ja-Nein-Stellungnahme von
Habermas nur verschärft.

\paragraph{3.} 
Die Dualismen Mensch und Natur, Ich und Der Andere, System und Lebenswelt sind
logische Folgerungen, welche mit Annahmen von Kohlberg und Piaget nur zum
Wiederauferstehen der skeptischen Haltung gegenüber den Massenmedien und der
Technik führen. Vielleicht ist die Ideologie der „bösen“ Positivisten und
Szientisten ja doch etwas komplizierter und mit Dahm ist gerade darauf
aufmerksam zu machen, dass es eine solche technisierende Menschenbildverwendung
im gescholtenen Positivismus nicht gibt.

\paragraph{4.} 
Technik wird trotz oder gerade mit der kritischen Position am doch so flachen
„Szientismus“ zum \emph{Durchdringer}, zu etwas Fremdem, Unnatürlichem.
Information und Kommunikation müssen folgerichtig als \emph{Versand
  konstruierter Signale} zum \emph{Penetrieren von außen} durch ihre eigene
Vergegenständlichung führen. Der \emph{Informationsbegriff} holt von hinten den
Ich-Kern zurück ins Boot, diesmal zwar als eliasianisch-prozessuale
Verhältnisbestimmung der zivilisatorischen Entwicklung, aber dennoch als
durchdrungen.

\paragraph{5.} 
Das \emph{Ziel} aller bisherigen Perspektiven ist aber die Tätigkeit, die
Handlung, die \emph{Nutzung} des Internets. Es geht also nicht nur um eine
Prozessualisierung der vermeintlichen Bewusstseinsphilosophie, sondern um eine
andere Auffassung. Es ist geradezu traurig zu sehen, dass die Soziologie die
Ergebnisse der anderen Fächer nicht zu verwenden weiß, oder dies nur abseits
des Mainstreams geschieht. Es muss nicht Foucault sein, denn gerade Habermas
zeigte eine Linie. Die Analytische Philosophie hat nicht nur an sprachlichen
Phänomenen gearbeitet, sondern an der Kritik der eigenen Position. Sellars,
Brandom, McDowell, Pinkard oder Pippin sind nur ein paar der Namen, die man
vermisst, wenn man die Enge heutiger soziologischer Theoriebildung betrachtet.
In der Philosophie, der Lingusitik, der Kulturwissenschaft, der
Literaturwissenschaft und gar noch der evolutionären Anthropologie ist ein
Begriff wie Performanz, Pragmatik oder Praxis ein hoffnungsvoller Strahl auf
echte Möglichkeiten einer alternativen Theoriebildung.

\paragraph{6.} 
Hier werden nicht nur andere Möglichkeiten herausgestellt, sondern ein neues
Licht auf die Grundlagen unserer eigenen Tradition geworfen. Man müsste fast
fordern, erneut den Ruf erklingen zu lassen: „Zurück zu Kant“ und damit zu
Hegel und Marx. Bewusstseinsphilosophie ist zwar ein schönes Label, hat aber
mit den eigentlichen Argumenten der Autoren wenig zu tun. Ob Idealismus,
Materialismus oder das so schädliche Produktionsparadigma, alle sind nette
Ismen, die für ein emanzipatorisches Narrativ nützlich sind, aber den
eigentlichen Weg auf eine Sicht der Moderne verstellen, die weder
technikgläubig noch technikfeindlich verfährt. Bis heute steht eine Rezeption
und Interpretation des Weberschen „Charisma der Vernunft“ jenseits aller
stählernen Gehäuse aus.
\medskip

Als eigentliches Problem bleibt, einen handlungsorientierten, gar
„empirischen“, Ansatz der Tätigkeit des Menschen zu entwerfen, bei dem keine
Begriffe wie „die Technik“ dem Organismus gegenübergestellt werden, sondern
selbst als Handlungsvollzüge, als Praxen in den Fokus genommen werden
können. Worte wie „Durchdringung“, „Information“, „Kommunikation“ zeigen eher
einen prekären Theoriestand an als eine Lösung für Probleme, die wir gewillt
sind, gemeinsam zu erschließen, weil sie uns gemeinsam betreffen.

\subsection*{Das Ich und der Andere}

Zweitens, und dieser Punkt ist eher eine Fortführung und Intensivierung des
ersten Punkts von Fuhr als ein eigenständiger Punkt, müssen wir in dieser Sicht
folgerichtig „den Anderen“ mitbetrachten.

In den wuselnden Großstädten der Moderne seien „der Fremde und der Fremde“
durch ein spezielles Vertrauen verbunden. Schon der marxistisch angehauchte
Nichtneukantianer-Neukantianer Simmel beschrieb die Vermittlung der anonymen
Massengesellschaft des massenmedialen Zeitalters durch das „parasoziale
Vertrauen“. „In der modernen Gesellschaft übernehmen die Massenmedien“ die
alltägliche Hineinversetzung in die Rolle des Fremden, also letztlich die
„Funktion der Integration und Organisation der Gesellschaft“. Parasoziales
Vertrauen sei „eine menschliche Grundeigenschaft“. Zuschauer- und
Teilnehmerperspektive greifen ineinander, wenn es darum geht „Öffentlichkeit
als aktive praktische parasoziale Teilhabe“ zu konstituieren und zu
verstehen. Informationstechnologie braucht „das parasoziale Vertrauen als
Steuerungssystem für sein Wachstum“ und Theorien des Verfalls können nur
Verschwörungstheorien sein.

\paragraph{1.} 
Es ist nicht schwer zu erkennen, dass wir es hier mit derselben problematischen
Ebene zu tun haben, die wir bei der Ich-Kern Diskussion gesehen haben. Man
möchte eine \emph{Handlungsperspektive} haben, arbeitet aber mit einem Begriffe
wie Vertrauen, der notwendigerweise eine individuelle Entscheidung impliziert.
Ob diese nun durch die „böse“ Anonymität der Metropole oder die „unmenschliche“
Ferne der Massenmedien getragen ist, spielt dabei kaum eine Rolle. Rolle ist
damit ein weiterer höchst problematischer Begriff.

\paragraph{2.} 
Eine Grundeigenschaft des Menschen zu bestimmen ist immer schon die Festsetzung
der \emph{condition humaine}, also eines dezidierten Menschenbilds. Man mag ja
an die „Hineinversetzung in die Rolle des Fremden“ glauben, doch sagt dies
nichts über die Tätigkeit aus. Sprachakte sind nicht zuerst die empathische
Aufnahme einer vermeintlichen Verwendung von Vorurteilen zum Verständnis eines
Gegenübers. Ein derartiges Bild kann auch wieder mit dem Ich-Kern arbeiten,
denn dieser transportiert die Vorstellung eines Individuums gegenüber einem
„Etwas“. Mag deshalb die Pointe der Großstadt sein, dass wir nicht jede Sekunde
auf die Rolle des Anderen schauen, sondern aktiv handeln.

\paragraph{3.} 
Jedes Urteil, jeder Schluss, jeder Sprechakt trägt schon ein gesellschaftlich
praktisch gewonnenes Urteil in sich. Ich versetzte mich nicht erst meditativ in
die Rolle des Gegenübers, sondern habe durch Teilhabe an menschlicher Tätigkeit
die Tätigkeit als Voraussetzung. Gerade \emph{weil} wir kommunikative Wesen
sind, haben wir den gemeinsamen Vollzug der menschlichen Praxen im praktischen
Vollzug gelernt. Ich lerne keine Rollen, ich vollziehe sie. Dies ist zu
berücksichtigen, wenn ich einen Begriff wie \emph{Vertrauen} fruchtbar machen
möchte. Die Frage der grundsätzlichen Möglichkeit der Indienstnahme eines
solchen Begriffs ist damit gestellt.

\paragraph{4.} 
Der Begriff \emph{Vertrauen} ist nicht weniger janusköpfig als der Begriff der
\emph{Rolle} oder der \emph{Information}. Wenn man den scheinbaren Automatismus
des Großstädters beschreiben möchte, kann man einen derartigen Begriff
fruchtbar machen. Wenn man aber das Problem des Ich-Kerns als zentral
betrachtet, ist zu fragen, ob ein Konzept, das immer mit einer gewissen
individuellen Haltung arbeitet – egal, ob substanzialistisch oder
zivilisatorisch-prozessual gefasst –, nicht den Weg versperrt, auf den man
gelangen möchte. Es kann natürlich nicht um die Aushebelung eines vernünftigen
Redens über das Selbstbewusstsein und damit der Verantwortung des Einzelnen in
der Gesellschaft gehen, denn gerade dieses Reden ist und muss von uns
sinnkritisch und sinnanalytisch \emph{verstanden} werden.

\paragraph{5.} 
Das ganze Gebilde dieses soziologischen Ansatzes zu einem Narrativ umzuwandeln,
an dem die Massenmedien die Rolle der Grundeigenschaft des Menschen übernehmen
sollen, wollen oder tun, ist damit nicht nur problematisch, sondern wird selbst
zur „Verschwörungstheorie“, die zwar keinen Verfall postuliert, aber
emanzipatorische Momente nicht fassen kann. Wenn wir anfangen, selbst aus
explikatorischer Absicht, einer „Grundeigenschaft des Menschen“ unabhängig von
deren Vollzügen die Bürde der Erklärung aufzuladen, haben wir den Boden
wissenschaftlicher Arbeit verlassen und die real-existierenden
Gestaltungsvarianten der Wissenschaft aufgegeben.

\paragraph{6.} 
„Parasoziales Vertrauen“ nun endgültig als Steuerungssystem des
„Informationszeitalters“ zu verstehen, entzieht sich letztlich jeder
gesellschaftlich möglichen Gestaltung. Wenn wir auch nur halbwegs diese Natur
des Menschen akzeptieren und die geschichtliche Beschreibung der medialen
Ersetzung nachvollziehen, ist es beim technisch-rechtlichen Stand der
Zivilisation nicht verständlich, wie die Mobilisierung des parasozialen
Vertrauens aussehen soll, außer natürlich, man spielt mit dem radikalen Bruch
alles Bestehenden. Revolution muss ja nichts Schlechtes sein, nur zeigt gerade
der arabische Frühling, dass es einen Unterschied gibt zwischen einer
„digitalen Revolution“ und „Der Revolution“.  

Das zu sehen braucht es aber nicht die Erfahrungen der neueren Geschichte, die
letzten zweihundert Jahre liefern genug Anschauungsmaterial.

\subsection*{Die Herausforderungen von Interdisziplinarität}

Das Grundproblem bleibt somit dasselbe wie beim ersten Punkt: das Menschenbild.
Handlungstheorie kann und ist natürlich nicht nur Theoriebildung, sondern immer
auch Kritik der Verhältnisse, die wir analysieren und an denen wir gleichzeitig
teilnehmen. Dass die Soziologie heute ein besonderes Verhältnis zur
Gesellschaft hat, kommt ihr nicht nur durch den Beschreibungscharakter der
Verhältnisse zu, in denen sie stattfindet; sind doch ihre Ergebnisse, nicht nur
der quantitativen Forschung, Grundlage und Gestaltungsmoment ihres eigenen
Gegenstandes.

Dennoch hat es etwas geradezu Tragisches, wenn eine so wichtige Disziplin wie
die Soziologie die Entwicklungen der anderen Richtungen von Außen betrachtet.
Man kann von Habermas halten, was man möchte, doch sein Bestreben, am
problematischsten Punkt anzusetzen, der Bestimmung des Wesens des Menschen, ist
vorbildhaft. Eine derartige Interdisziplinarität ist akademisch löblich und
darüber hinaus ein Meilenstein der soziologischen Arbeit.

Gleichzeitig wird eine Perspektive erkennbar, die nach aller gebotenen Kritik
des vorliegenden Textes, eine ernst zu nehmende Möglichkeit auch für die
heutige Soziologie bereithält. Die Erforschung der sprachlichen Tätigkeit des
Menschen ist heute auf einem neuen Level angekommen. Der \emph{linguistic turn}
ist ein weit verbreitetes Narrativ, um die Änderung des Fokus in den 1970er
Jahren zu beschreiben, und auch heute gibt es erste Autoren, die von etwas wie
dem \emph{practical turn} reden. Tatsächlich ändert sich durch
interdisziplinäre Diskussionen der Fokus, und es ist zu hoffen, dass es nicht
eines Kuhnschen „Aussterbens der Träger des alten Paradigmas“ bedarf, um einen
Schritt weiter zu gehen. Die Entwicklung der letzten zwanzig Jahre in der
Philosophie in Auseinandersetzung mit der Lingusitik, der Neurowissenschaft
oder der Wissenschaftstheorie haben zu einem anderen Verständnis von Pragmatik
geführt und sind es wert, auch von der Soziologie aufgenommen zu werden.
\clearpage

Die Soziologie hat, und darauf weist Fuhr zu Recht hin, alle Mittel in der
Hand, einen neuen Schritt zu wagen, wenn sie sich nicht den alten Dogmen
verschreibt. Die Perspektive auf das eigentliche Problem haben die Disziplin
als auch Fuhr in der Hand. Die Probleme des Ich-Kern-Bildes sind erkannt und
Wege liegen offen.

Doch diese Wege können nicht das Heraufbeschwören alter Simmeleien sein,
sondern erfordern die kollegiale Fruchtbarmachung der Möglichkeiten in
differierenden Ansätzen der anderen Kollegen. Die größte Schwierigkeit bereitet
Interdisziplinarität damit gerade beim \emph{Verlassen} des eigenen
Fächerkanons.  \emph{Ergebnisse} der Wissenschaft und insbesondere der anderen
Geisteswissenschaften \emph{gibt es}, auch wenn dies wie eine alte
positivistische Leier klingen mag.

Fuhr ist damit nicht nur zu danken für den argumentativen Austausch. Es ist ihm
zu danken, dass er am Mainstream der heutigen Soziologie vorbei Alternativen
aufzeigt, Ergebnisse unorthodoxer Ansätze auswertet und am zentralen Problem
arbeitet. Es ist zu danken, dass sich ein Kollege die Zeit genommen hat,
Kollegen anderer Fächer Argumente zu präsentieren und diese mit ihnen zu
diskutieren.

Sicher bin ich mir, dass er auch das mit diesem Text vorgelegte Angebot zur
infradisziplinären Arbeit freudig aufnimmt und einen nächsten Schritt
unternimmt, an den problematischen Begriffen umfassend zu
arbeiten. Interdisziplinarität ist der Schritt, den wir heute nicht nur als
Forscher brauchen, um infradisziplinär für uns als Menschen etwas zu
leisten. Man mag eine „Tragik der heutigen Soziologie“ in einem problematischen
Konzept wie dem der Parasozialität sehen, aber der interdisziplinäre Diskurs
darum und das infradisziplinäre Ringen um Begrifflichkeiten liefert den Beweis,
dass es auch anders geht.

Somit kann man nur sagen: Zurück zu Kant – \emph{sapere audem}.

\ccnotiz
\end{document}
