\documentclass[11pt,a4paper]{article}
\usepackage{od}
\usepackage[utf8]{inputenc}
\usepackage[russian,main=ngerman]{babel}

\title{Element-funktionale Konfliktmodellierung: EFM.K.}

\author{Evgeni E. Smirnov, St. Petersburg}
\date{1983}

\begin{document}
\maketitle
\begin{quote}
  Original: \foreignlanguage{russian}{Элементно-функциональное моделирование
    конфликтов: ЭФМ.К}.

  Quelle: \url{https://cloud.mail.ru/public/4a7c/WX37Yg1JB}
  
  Übersetzt von Hans-Gert Gräbe, Leipzig.
\end{quote}
\begin{quote}
  In diesem Artikel geht es darum, eine der am meisten unterschätzten Methoden
  der Konfliktmodellierung zu verbessern: Die SF-Modellierung.  Es werden Fragen
  der Erhöhung der Instrumentalität von Methoden der Konfliktmodellierung
  betrachtet.

  Es werden Empfehlungen zur Steigerung der Wirksamkeit von Techniken gegeben
  und ihr Platz bei der Lösung erfinderischer Probleme aufgezeigt.

  Schlüsselwörter: TRIZ, Methodologie, SF-Modelle, SF-Analyse,
  Funktionalanalyse, Funktionale Systeme, EF-Modellierung von Konflikte,
  EFM.K, Techniken, Widersprüchliche Anforderungen.
\end{quote}

\section*{Einführung}

Ein kurzer Überblick über die Grundprinzipien des Elementaren funktionale
Modellierung, die nicht nur ersetzt Feldanalyse, wird aber auch eine
verbindliche Methode für andere Instrumente im Rahmen der TRIZ.

Ziel dieser Arbeit ist es, den aktuellen Stand der Dinge in hervorzuheben
Verbesserung der Vepol-Analyse, Vorschläge für seine Radikalität ändern, sowie
die Effizienz der Anwendung und andere zu verbessern TRIZ-Tools am Beispiel
von Konfliktlösungstechniken.

\section*{Hintergrund}

Heutzutage wird am häufigsten die Vepol-Analyse verwendet beschränkt auf
Schulungen. Seine praktische Anwendung beschränkt auf eine Reihe von
wesentlichen Nachteilen. Ich denke, das macht hier keinen Sinn Sprechen Sie
ausführlich darüber, was diese Art der Analyse ist. Hier nur einige
Definitionen.

Eine der ersten Definitionen von Vepol wurde in [1] gegeben: „Vepol ist ein
minimal vollständiges Modell eines idealen technischen Systems.“ Hier besucht:
Feld (P), Substanz (B) und Medium (C). Als eine Option - ein Wildschwein mit
einem Feld (P) und zwei Substanzen (2B), jeweils von Welches ist das Medium in
Bezug auf ein anderes. Diese letzte Option, wo jetzt zwei Substanzen -
"Werkzeug" und "Produkt" - verwendet werden. Jedoch, nach modernen Konzepten
als Teil eines funktionalen Systems (FS) Das Produkt ist nicht im Lieferumfang
enthalten. Aber die Definitionen haben sich unwesentlich geändert, Zum
Beispiel: „Vepol ist das Mindestmodell eines technischen Systems: es fünf
umfasst Produkt, Werkzeug und Energie (Feld). notwendig für die Exposition
Werkzeug pro Produkt“ [2, S. 91].

Für die Durchführung einer Vepol-Analyse wird sie nun als konstruiert genannt
das Dreieck "2B + 1P", das sogar von Lösern verwendet wird wenn Aufgaben nicht
zu diesem Modell passen.

Hier ist ein Beispiel für eine andere Definition, die erstens nicht klar ist
Die Grundlage dieser Regel deutet auf das Vorhandensein eines Widerspruchs
hin.  zweitens „schaltet“ es die mentale Trägheit des Denkens ein und stimmt
klar ab für die Konstruktion eines Dreiecks nach der Formel „2B + 1P“: „
Vepol-Analyse im Betriebsbereich der Aufgabe ausgeführt werden, d.h. wo
identifiziert physischer Widerspruch. Es müssen zwei an diesem Ort sein
Substanzen B1 und B2, die vorteilhaft oder schädlich miteinander interagieren,
und das Feld P, das diese beiden Substanzen bindet “[3, S. 89].

Schließen Sie die Lösungssuche durch Formulierung und Auflösung physikalische
Widersprüche in [4] vorgeschlagen. Und der Hauptnachteil Hier ist die
Formulierung des inhärenten Subsystems (physisch) Widersprüche auf
Systemebene, unmittelbar bei der Analyse einer Problemsituation.  Der
Ausschluss eines solchen Übergangs zwischen Systemebenen beraubt die Methode
zusätzliche heuristische Kraft.

Es wird auch versucht, Vepol zu verbessern Analyse, aber diese neuen Versionen
haben keine Massenverteilung.

\section*{Ein kurzer Überblick über Arbeiten zur Verbesserung der
  Vepol-Analyse}

In letzter Zeit wurden nicht viele Arbeiten der Feldanalyse gewidmet und
System von Standards. Und die meisten von ihnen wurden auf dem Gipfel
vorgestellt.  2013 Entwickler im Thema "Weiterentwicklung der Vepol-Analyse".

Alle Artikel werden als allgemeine Schwachstellen dieses Tools aufgeführt.
sowie Nachteile, die die Lösung sehr spezifischer Probleme behindern. Also
rein Arbeiten [5, 6] präsentieren Optionen für die Modifikation des
klassischen Vepol Analyse für Informationssysteme:

Die „Substanz“ wird durch das „Element“ ersetzt und die Konzepte werden
verwendet "Elepole", und es wurde auch versucht, mit Funktionsanalyse zu
kombinieren: „ Die Gemeinsamkeit von elepole mit einem Funktionsmodell
ermöglicht es Ihnen, Werkzeuge zu übertragen Funktionsanalyse in der
Elepolanalyse. Wir werden diese Art nennen Analyse der Funktionsfeldanalyse
von Systemen "; und mit funktionalen Suchorientierte Suche (FOP). Im Vergleich
zu " Durch die Pluralität" wird die folgende Aussage gemacht: "sowohl in der
Pluralität als auch im Elepol Funktionen können sein oder nicht “[5];

Das Wildschwein wurde in El-Action umbenannt (genau so - auf Englisch):
element- Aktion und führte auch eine andere Komponente ein - Wissen (Wissen):
„ Das Modell, einschließlich des Elements, der Aktion und des Wissens, wird
EAK genannt.  Die EAK-Analyse- und Transformationsmethode wird als EAK-Analyse
bezeichnet. “ Ebenfalls Merkmale der Anwendung der EAK-Analyse für
Verarbeitungssysteme werden berücksichtigt Information: „In diesen Systemen
wird ein Element als Daten dargestellt (Daten - D), Handlung ist eine Funktion
(Funktion - F) und Wissen (Wissen - K). Modell, einschließlich Daten,
Funktion, Wissen werden wir DFK nennen. Methodik Die DFK-Analyse und
-Transformation wird als DFK-Analyse bezeichnet “[6].

In [7] wurde eine Variante der Modellierung von Interaktionen in biologische
(lebende) Systeme, aber ohne über das Klassische hinauszugehen Ideen zur
Vepol-Analyse.

In der Rezension des Buches [8] sprechen wir über die Anpassung der
Vepol-Analyse und des Vepol-Systems Standards für die schnelle Ausbildung von
Ingenieuren - hier der Einfachheit halber Transformationen der
Standard-Vepol-Formel werden nur mit durchgeführt 5 Regeln aus 76 Standards
ausgewählt, wahrscheinlich basierend auf dem Prinzip ihrer Häufigkeit
verwenden. Darüber hinaus erwog der Autor von [8] die Option Darstellungen des
Systems als eine Reihe von Feldern und Substanzen, die vollständig ist ein
Analogon der strukturellen Energiesynthese von Systemen aus früheren Arbeiten
[9, S. 73].

Ein Artikel über Eber [10] schlägt die Anwendung eines Prozessansatzes mit vor
Ersetzen von "Substanzen" und "Feldern" durch "Ressource". Und am Ende glaubt
der Autor das ähnliche Prozessdiagramme: „1. In allen betrachteten Fällen
überhaupt nicht weniger heuristisch als klassische Vepol-Designs. 2. Lass los
aus der Vielzahl von protovolepny Schemata des Problems, das heißt aus
Subjektivität. 3. Kombinieren Sie die klassische VA logisch funktional
perfekte Modellierung und Prozessansatz als solche. 4. Angenommen Ersetzen des
Begriffs "Vepol-Analyse" durch etwas wie "System-Prozess" Modellierung "(SPM)
. " Eine solche Darstellung der Modellierungsmethode ist jedoch Konflikte
haben keine Fortsetzung.

In anderen Werken beschränken sich die Autoren hauptsächlich auf die
Auflistung die Mängel der klassischen Materialfeldanalyse und des Individuums
Empfehlungen für seine Verbesserung. Insbesondere werden Wünsche in
ausgedrückt Verbesserung der Visualisierung des strukturellen und dynamischen
Beschreibungsprozesses (technische) Systeme, die in erster Linie mit
Unvollkommenheit verbunden sind grafische Mittel der Vepol-Analyse. Zu diesem
Zweck wird vorgeschlagen, eine Ausarbeitung vorzunehmen Leistungsbeschreibung
für Forschungsarbeiten [11].

Es wird empfohlen, Vepol-Analyse und FSA [11, 12] zu kombinieren und zu
kombinieren Stärken sowie die Axiomatik dieses Werkzeugs ausrichten Axiomatik
der Systemanalyse und der TRIZ insgesamt [13]. Diese jedoch Empfehlungen haben
keine wirkliche Umsetzung.

Darüber hinaus wird vorgeschlagen, das Studium der Vepol-Analyse im
Allgemeinen zu reduzieren.  TRIZ-Kurs [11] oder sogar die Entwicklung der
Vepol-Analyse aufgeben und Standardsysteme als eigenständiges Instrument [12].

\section*{Liste der Hauptnachteile verschiedener Versionen der Analoga der
  Vepol-Analyse:}
\begin{itemize}
\item Es gibt immer noch eine Verwechslung der Konzepte "System" und
  "Konflikt" mit Optionen Modellierung basierend auf Vepol-Analyse;
\item
  die Nutzung von Existenzbedingungen ist nicht vollständig realisiert und
  Gesundheit der Systeme, sowie fast keine funktionale ein Ansatz;
\item
  die Begriffe „Substanz“ und „Feld“, wenn durch andere ersetzt, dann oder zu
  allgemein [10] oder umgekehrt - hochspezialisiert [6];
\item
  Optionen zum Abrufen von Lösungsbildern aus konstruierten Modellen oder
  fehlt oder reduziert auf die Verwendung eines Standardsystems: ein wenig
  modifiziert [5, 9] oder signifikant reduziert [8], dh sie basieren nach dem
  gleichen Prinzip wie die klassische Vepol-Analyse;
\item
  Keine der vorgeschlagenen Modellierungsoptionen schlägt vor Visualisierung
  von Widersprüchen.
\end{itemize}

\section*{Wichtige Punkte}

Bei der Lösung von Problemen durch klassische Vepol-Analyse ist dies nicht
immer der Fall Es ist möglich, den gewünschten Standard zu verwenden oder den
meisten zu wählen geeignet. Vor allem, wenn es keine grafische Interpretation
hat.  Zum Beispiel ist die bekannte Aufgabe, eine Nadel im Heuhaufen zu
finden, allen gemeinsam. Öfters Insgesamt wird es nach der Vepol-Syntheseregel
(Standard 1.1.1 [14]) gelöst, wenn es zwei gibt Substanzen: Nadel und Heu, und
es gibt kein Feld [3, S. 90]. Es scheint, dass solche Die Konstruktion des
Modells ist offensichtlich und bedarf keiner Erklärung. Aber in Wirklichkeit
Es gibt eine Ersetzung der Aufgabe: statt suchen (erkennen) ausführen Trennung
von Heu und Nadeln. Ebenso können Sie das Problem des Findens von Fischen
lösen Ozean - werden Sie entscheiden, indem Sie den Fisch vom Wasser trennen
und einen anwenden oder Felder?

Eine der Hauptursachen für solche Modellierungsfehler ist Mangel an
funktioneller Komponente in der Vepol-Analyse. Findet nicht Dies spiegelt sich
in den in der Rezension vorgestellten Arbeiten wider. Der zweite Grund -
schlechte Anwendung eines systematischen Ansatzes, der höchstwahrscheinlich
damit verbunden ist dass Modellierungssysteme (funktionale Systeme oder in den
Klassikern - technische Systeme) ist nur beschreibend.

Ein Versuch, einen systematischen Ansatz anzuwenden, wurde von den Autoren von
[9, S. 82] mit unternommen die Entwicklung der strukturellen Energiesynthese
von Systemen. Aber wie „Energieketten“, die aus elementaren Struktureinheiten
aufgebaut sind sind ein wichtiger Schritt zur Verbesserung der Effizienz des
Wildschweins Analyse, enthalten auch erhebliche Mängel, die weit stören
Verteilung dieser Methode, nämlich:
\begin{itemize}
\item Dieselben Komponenten werden als Elementarkomponenten verwendet.
"Substanzen" und "Felder" und damit die Prinzipien der Interaktion
aus der Vepol-Analyse entlehnt;
\item Manager Kraftfluss begrenzt auf physikalische Effekte: " Ähnliche"
Substanz-Feld "-Paare bieten Kontrollierbarkeit Systeme, in der Natur ist
ziemlich viel bekannt. Insbesondere können sie ferromagnetische Substanzen
oder Ferropartikel mit magnetische oder elektromagnetische Felder. Und auch
elektrisches Feld und elektrorheologische Flüssigkeiten (d.h.  Gemische, die
ihre Viskosität unter elektrischem Einfluss ändern Felder) ”;
\item Das Prinzip der Vollständigkeit des Systems, einschließlich 4 Arten von
  Elementar Struktureinheiten, und in der Beschreibung der Methode
  dargestellt, nicht voll in der Konstruktion von Modellen verwendet, was dazu
  führt Weglassen einer Reihe möglicher Richtungen zum Konvertieren dieser
  Modelle;
\item "Produkt" wird als Feldempfänger betrachtet - in "Ändern" Systeme oder
  als Konverter - in "Messen", aber Nirgendwo erreicht es laut ein
  vollständiges Funktionssystem Analogien zum "Werkzeug", das den Bereich
  erheblich einschränkt Anwendung solcher Modelle und schränkt das Suchfeld
  möglich ein Entscheidungen - in diesem Fall - aufgrund nicht genutzter
  Ressourcen "Produkte";
\item Fehlende Funktionalität: elementare Interaktionen
  Links werden nicht angegeben, auch wenn einer von ihnen ein "Produkt" ist.
\end{itemize}
Was getan werden muss? Für den Anfang die meisten beseitigen oben erwähnte
Nachteile. Insbesondere um stereotype loszuwerden Beschreibung des Modells in
Form eines Dreiecks "2B + 1P" eindeutig bestimmen Das Gebiet der Modellierung
ist eine Konfliktsituation und vielleicht die wichtigste - Anwendung der im
Rahmen des Bewährten verwendeten Konzepte und Modelle Ansätze: systemisch und
funktional.

In der Praxis ist der systematische Ansatz daher mit dem funktionalen
verbunden dass Objekte (Geräte, Prozesse) der realen Welt vom Menschen benutzt
werden, Jedes hat seinen eigenen Zweck, ausgedrückt durch den Funktionsbegriff
.  Somit werden Objekte als funktionale Systeme (FS) modelliert .  In TRIZ
besteht ein Funktionssystem aus mehreren Teilen. Zum Beispiel in einem Buch
[15, S. 94-97] Der Autor bestimmt das obligatorische Vorhandensein eines
technischen Systems (TS) aus vier Teilen: Motor (D), Getriebe (T), Steuerung
(OU) und der Arbeitskörper (RO). Die Energiequelle (IE) stimmt hier entweder
überein Der Motor oder aus dem System genommen, wenn die Energie von außen
kommt, auch von einer Person.

Und nun, aus der beschreibenden Natur der Modellierung von Objekten in Form
von FS, es ist notwendig, zu seiner Instrumentalversion zu gehen. Aber dafür
folgt es Es ist besser zu verstehen, was die ausgewählten Teile sind (IE, D,
T, RO, OS) und welche Rolle spielen sie für die Funktionsweise der Systeme.

Hier ist, was K. Marx zu diesem Thema schreibt [16]: „ Jeder entwickelt Ein
Maschinengerät besteht aus drei wesentlich unterschiedlichen Teilen:
Maschinenmotor, Getriebe, schließlich Maschinengewehre oder
Arbeitsmaschine. Der Maschinenmotor ist die treibende Kraft von allem
Mechanismus. Sie selbst erzeugt ihre Motorleistung wie Dampf Maschine,
Kalorienmaschine, elektromagnetische Maschine usw. oder empfängt einen Impuls
von außen, von jeder vorgefertigten Naturgewalt wie Wasser ein Rad aus
fallendem Wasser, ein Flügel einer Windkraftanlage vor dem Wind usw.  ein
Mechanismus bestehend aus Schwungrädern, beweglichen Wellen, Zahnrädern,
Exzenter, Stangen, Transferbänder, Riemen, Zwischenprodukte Geräte und Zubehör
verschiedener Art, regelt Bewegung, ändert bei Bedarf seine Form, zum Beispiel
dreht sich von senkrecht zu kreisförmig, verteilt es und überträgt es an die
Arbeiter Autos. Diese beiden Teile des Mechanismus existieren nur zu Sagen Sie
die Bewegung des Maschinengewehrs, damit es erfasst das Thema Arbeit und
ändert es zweckmäßigerweise . “

Bei der Analyse vorhandener Tools und ihrer Anwendungen auf die Lösung
Aufgaben - es wurde offensichtlich, dass Teile des Systems Energiewandler sind
(genauer gesagt - seine Eigenschaften), die für die Umsetzung durch den
Arbeitskörper erforderlich sind Zweck des Systems (Funktion):
\begin{itemize}
\item Konverter der ersten Art - ändern Sie die Energiequalität, dh ihren Typ;
\item Konverter der 2. Art - ändern Sie die Menge und räumliche vorübergehende
  Organisation (Intensität, Richtung, Anwendungsort) Energiefluss.
\end{itemize}
Im Allgemeinen zu verstehen, was jeder von Teile des Systems (Konverter) ist
es bequemer, ein solches Konzept wie zu verwenden ein Element , das
ausgedrückt werden kann durch:
\begin{itemize}
\item[1.] Die Substanz
\item[2.] Energie
\item[3.] Informationen.
\end{itemize}
Daher ist es offensichtlich, dass das funktionale System durch das
Vorhandensein von bestimmt wird Satz von Basiselementen (mit seiner Struktur
und Funktionalität) und eine externe Funktion , d. h. die Fähigkeit zu
produzieren eine Änderung eines Merkmals der Elemente anderer Systeme:
Zustände, Eigenschaften, Parameter.

Um den Zustand des Systems zu gewährleisten, ist dies erforderlich Sie hatte
ein exekutives Element sowie die Fähigkeit, dies zu empfangen ein
Energieelement, in ausreichendem Volumen und in der für die Umsetzung
erforderlichen Form gegebene Funktion.

In den meisten Fällen muss das System mindestens zwei Elemente enthalten:
Arbeitskörper und Energiequelle. Bei Bedarf werden sie eingeführt zusätzliche
Energiewandler sowie ein Minimalsystem Steuerung, dh Bereitstellung der
Möglichkeit zum Ein- und Ausschalten Energieversorgung der RO.

Und natürlich kann die Konstruktion nicht ohne das Vorhandensein eines
„Produkts“ auskommen.  das erstreckt sich auch auf volle FS, ähnlich - durch
Hinzufügen Energiequelle zu seinem Aktuator.  Wenn es unbefriedigende
Funktionen gibt, Konflikt, was bedeutet, dass es ein Konfliktmodell gibt und
ein solches Modell sein kann Es genannte Element funktionelle oder EFM.K .

\begin{center}
  Bild einfügen\\ EPM-Präsentationsoption
\end{center}
Somit sind 2 Grundsteine davon jedes funktionale System: 1) Ein
Elementkonverter von Energie und 2) Funktion.

Wenn die elementar-funktionale Modellierung groß genug ist Systeme, genauer
gesagt Systeme mit einer großen Anzahl von Elementen, werden zum einen
berücksichtigt nur diejenigen Elemente, die an den betrachteten Interaktionen
beteiligt sind, und zweitens werden die Grenzen der Elemente sowie die Grenzen
der Systeme basierend auf festgelegt Aufgabenbedingungen und gesunder
Menschenverstand. Das heißt, wenn wir uns verbessern Komponenten, zum Beispiel
eines elektrischen Generators, können wir alle als betrachten einzelne
Elemente mit dem erforderlichen Detaillierungsgrad. Aber wenn der Generator
ist eine elektrische Energiequelle des Systems, dann kann es als einzelnes
IE-Element vorhanden.

Das heißt, das Prinzip der Fraktalität gilt hier, wenn jedes Element, bis zu
einer gewissen Grenze kann in seine Bestandteile zerlegt werden, Beziehungen
davon sind auch funktionale Systeme.

Darüber hinaus können wir über verschiedene Arten solcher Modelle und ihre
Varianten sprechen Transformationen, die auf die Regeln hinauslaufen. Das
heißt: zuerst brauchen Sie die Situation zu analysieren und ein
elementar-funktionales Modell aufzubauen Konflikt Danach wird das Modell
analysiert und durch Anwendung der Regeln Es gibt eine Synthese neuer Modelle:
Modelle von Richtungen, um Lösungen zu finden.
\begin{center}
  EFM.K $\to$ EFM.NPR
\end{center}
Dann kommt die Synthese von Lösungen. Dafür ist es aber schon nötig Verwenden
Sie heuristische Methoden, Analogien und figuratives Denken (Phantasie),
beitragen Überwindung der endgültige eine Barriere Entscheidungsmodelle von
ihren realen Inkarnationen trennen.

Die beliebteste heuristische Methode ist die Verwendung Empfänge. Aber wie
kann man diese Techniken am effektivsten einsetzen?

\section*{Empfänge}

Die Aufgabe, eine Nadel im Heu zu finden, ist recht einfach. Aber es gibt noch
mehr komplexe Aufgaben, die hier nicht berücksichtigt werden. Die Frage ist
also ungefähr Der Einsatz von Techniken gliedert sich in zwei Teile:
\begin{itemize}
\item[1.] Bei einfachen Problemen wird die Lösung auf der Ebene der Regeln
  erhalten und steht auf die Frage "Warum brauchen wir Empfänge?".
\item[2.] Bei komplexen Aufgaben stellt sich eine etwas andere Frage: „Wie
  Empfänge benutzen? "oder" Wie man sie erreicht? "
\end{itemize}
In der klassischen TRIZ gibt es 2 analoge Werkzeuge, bei denen Tricks:
\begin{itemize}
\item[1.] Um physische Inkonsistenzen im Rahmen von ARIZ zu beheben .
\item[2.] Als Ergebnis der Arbeit mit der Altshuller-Matrix, die den Zugriff
  auf beinhaltet Techniken durch die Formulierung sogenannter technischer
  Widersprüche.
\end{itemize}
Gibt es eine Möglichkeit, diese Optionen für zu verwenden
EFM.K-Transformationen?

Zunächst definieren wir im Rahmen des Modells die Essenz des Problems und
Anweisungen für seine Beseitigung. Die Spitze des Eisbergs ist seitdem eine
Funktion es spielt die Rolle eines Indikators für die Aktivität von
Elementen. Und der Konflikt ist vorhanden, wenn die Funktion dort
unbefriedigend, schädlich oder nicht vorhanden ist, wo soll sie sein.

Dies deutet darauf hin, dass es in unbefriedigende Eigenschaften gibt
Elemente, die an der Implementierung dieser Funktionen beteiligt sind und für
verantwortlich sind seine Implementierung sowohl von der Seite des „Werkzeugs“
als auch von der Seite des „Produkts“.

Trotz der qualitativen Unerschöpflichkeit der Materie, da "jeder ein Ding hat
unzählige verschiedene Qualitäten. Erkunden Bei einer endlichen Anzahl von
Eigenschaften einer Sache können wir nicht sagen, dass diese Sache von uns
stammt bereits vollständig studiert “[17, S. 70], - es ist möglich, sich auf
das Finale zu beschränken die Anzahl der Merkmale, die für die Bedingungen
einer bestimmten Aufgabe von Bedeutung sind, und in am meisten bestimmen die
implementierten Funktionen. Also ändern Diese Eigenschaften können geändert
werden und funktionieren in die gewünschte Richtung (normalisieren). Und für
einen solchen formalen Ansatz kann es sein Angewandte Techniken zur
Transformation von Elementen und ihren Beziehungen.

Die Anwendung von Techniken im Anfangsstadium kann zu zwei führen
Hauptergebnisse:
\begin{itemize}
\item[1.] Der Konflikt ist gelöst, es gibt keine unerwünschten Folgen - das
  Problem ist gelöst;
\item[2.] Der Empfang beseitigt den anfänglichen Konflikt, wird jedoch
  angezeigt unerwünschte Änderung der normalen Funktion der Elemente - Eine
  unbefriedigende Funktion kann auftreten korrigiert (normalisiert). In diesem
  Fall können wir darüber sprechen das Auftreten eines Widerspruchs der
  Bedingungen ( PU ) - hier sind die Bedingungen verbunden.  Anwendung /
  Nichtbenutzung der Rezeption.
\end{itemize}
Wenn die Tricks nicht sofort funktionierten und der Konflikt nicht ohne gelöst
werden konnte Konsequenzen, dann ist das Problem auf Systemebene nicht gelöst
und es ist notwendig entweder das Supersystem kontaktieren; entweder auf die
Mikroebene gehen - zu Subsystemen, wo PU als Widerspruch der Anforderungen (
PT ) umformuliert wird , um zu beseitigen Das ist auch möglich, auf Tricks
zurückzugreifen.

Erklärung. PU und PT\footnote{Die Begriffe PU und PT wurden 2013 vom Autor
  eingeführt und in praktischen Übungen in erfolgreich getestet SPb MOU TRIZ.}
werden als allgemeinere Begriffe für verwendet Widersprüche anstelle von TP
und FP in der klassischen TRIZ übernommen. Eine solche Bezeichnungen sind auch
universeller und daher anwendbar in verschiedene Tätigkeitsbereiche (und nicht
nur in der Technologie), sondern auch reflektieren das Wesen der Widersprüche
dieser Art: einschließlich Konflikte zwischen, Dementsprechend Bedingungen.

\section*{Vorteile und Super-Effekte}
Eingereicht Möglichkeit Modellieren Konflikte besitzt Die folgenden
Hauptvorteile:
\begin{itemize}
\item Die Ressourcen des Systems und funktionale Ansätze werden verwendet:
implementierte ihre Synthese und Instrumentalisierung.
\item Überwindung psychologischer Barrieren bei der Beschreibung der Regeln
  Gebäudemodelle.
\item Einzigartigkeit und Universalität der Begriffe sowie Klarheit Verwalten
  der Regeln zum Erstellen und Transformieren von Modellen - fertig Auf diese
  Weise können Sie komplexe Konflikte simulieren, die in auftreten jedes
  Tätigkeitsfeld.
\item Der Ort der Empfänge wurde festgelegt und ihre Möglichkeiten erweitert
  Design (siehe Präsentation).
\item Es wurde ein bedeutender Schritt in Richtung eines neuen Verständnisses
  der Essenz unternommen.  erfinderische Werkzeuge und ihre integrierte
  Verwendung.
\end{itemize}

\section*{Literatur}
\begin{itemize}
\item[1.] G. Altschuller, Ch. Gadzhiev,
  I. Flikstein. \foreignlanguage{russian}{Введение в вепольный анализ}. –
  Baku, OLMI, 1973.
\item[2.] G.S. Altschuller, B.L. Zlotin, A.V. Zusman, V.I. Filatov.
  \foreignlanguage{russian}{Поиск новых идей: от озарения к технологии (Теория
    и практика решения изобретательских задач). Кишинев: Картя Молдовеняскэ,
    1989}.
\item[3.] G.I. Ivanonv. \foreignlanguage{russian}{Формулы творчества, или Как
  научиться изобретать: Кн.  для учащихся ст. классов. – М.: Просвещение},
  1994.
\item[4.] V.N. Glasunov. \foreignlanguage{russian}{Параметрический метод
  разрешения противоречий в технике (методы анализа проблем и поиска решений в
  технике) – М.: «Речной транспорт», 1990}.
\item[5.] M.S. Rubin. \foreignlanguage{russian}{Элепольный анализ как развитие
  вепольного и функционального анализа в ТРИЗ}. URL:
  \url{http://triz-summit.ru/file.php/id/f5776/name/Элепольный-Рубин-5.pdf}
\item[6.] V. Petrov, G. Voronov. \foreignlanguage{russian}{Новый подход к
  вепольному (структурному) анализу}. URL:
  \url{http://triz-summit.ru/file.php/id/f5677/name/Petrov%20V.%20Voronov%20G.%20A%20new%20approach%20to%20Su-Field%20_structu.pdf} 
\item[7.] Sara Greenberg. Introducing substance-field, as a method for
  studying living systems. URL:
  \url{http://triz-summit.ru/file.php/id/f5669/name/Introducing%20substance-field%20as%20a%20method%20for%20studying%20liv.pdf}
\item[8.] Yu. Belski. \foreignlanguage{russian}{Инструменты ТРИЗ для ХХI века:
  современный вещественно-полевой анализ}. URL:
  \url{http://triz-summit.ru/file.php/id/f4543/name/SuFieldBelskiPart1-1-RUSS-BK-1.doc}
\item[9.] B.I. Goldovski, M.I. Wainerman.
  \foreignlanguage{russian}{Рациональное творчество. О направленном поиске
    новых технических решений. – М.: «Речной транспорт», 1990. (Методы анализа
    проблем и поиска решений в технике).}
\item[10.] V.A. Koroliov.  \foreignlanguage{russian}{Веполи: 20 лет спустя
  (2)}.  URL: \url{http://coroliov.trizinfor.org/works/ws6.html}
\item[11.] N.B. Feygenson.  \foreignlanguage{russian}{Вепольной анализ и его
  аналоги – прагматические аспекты}. URL:
  \url{http://triz-summit.ru/file.php/id/f5678/name/TDS-2013_Feygenson_Su_Field_notes.pdf}
\item[12.] S.A. Logvinov. \foreignlanguage{russian}{Проблемы обновления
  системы стандартов и вепольного анализа}. URL:
  \url{http://triz-summit.ru/file.php/id/f5679/name/Логвинов%20-%20статья%20ТРИЗ-Саммит%202013.pdf}
\item[13.] V.A. Koroliov. \foreignlanguage{russian}{Веполи: 20 лет спустя}.
  URL: \url{http://coroliov.trizinfor.org/data/c82.htm}
\item[14.] G.S. Altschuller. \foreignlanguage{russian}{Маленькие необъятные
  миры: стандарты на решение изобретательских задач // В сб. "Нить в
  лабиринте". – Петрозаводск: Карелия, 1988}. – pp 165-230. URL:
  \url{http://www.altshuller.ru/triz/standards.asp}
\item[15.] Yu.P. Salamatov. \foreignlanguage{russian}{Как стать изобретателем:
  пособие для учителя. – 2-е изд., дораб. – М.: Просвещение}, 2006.
\item[16.] K. Marx. Kapital, Band 1, Kap. 13. Maschinerie und große Industrie.
  Entwicklung der Maschinerie.
\item[17.] A.I. Uiomov. \foreignlanguage{russian}{Вещи, свойства и
  отношения. – М.: Изд-во Академии наук,} 1963.
\end{itemize}

\end{document}
