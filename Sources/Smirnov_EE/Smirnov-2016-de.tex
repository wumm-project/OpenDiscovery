\documentclass[11pt,a4paper]{article}
\usepackage{od}
\usepackage[utf8]{inputenc}
\usepackage[russian,main=ngerman]{babel}

\title{Element-Funktions-Modelle\\ in der Konfliktmodellierung: EFM.K.}

\author{Evgeni E. Smirnov, St. Petersburg}
\date{1983}

\begin{document}
\maketitle
\begin{quote}
  Original: \foreignlanguage{russian}{Элементно-функциональное моделирование
    конфликтов: ЭФМ.К}.

  Quelle: \url{https://cloud.mail.ru/public/4a7c/WX37Yg1JB}
  
  Übersetzt von Hans-Gert Gräbe, Leipzig.
\end{quote}
\begin{abstract}
  In diesem Artikel geht es um die Verbesserung einer der am meisten
  unterschätzten Methoden der Konfliktmodellierung: Die SuField-Modellierung.
  Es werden Fragen der Erhöhung der Instrumentalität von Methoden der
  Konfliktmodellierung betrachtet.

  Es werden Empfehlungen zur Steigerung der Wirksamkeit von Techniken gegeben
  und ihr Platz bei der Lösung erfinderischer Probleme aufgezeigt.

  Schlüsselwörter: TRIZ, Methodologie, SF-Modelle, SF-Analyse,
  Funktionalanalyse, Funktionale Systeme, EF-Modellierung von Konflikten,
  EFM.K, Techniken, Widersprüch"|liche Anforderungen.
\end{abstract}

\section*{Einführung}

Es wird ein kurzer Überblick über die Grundprinzipien der EF-Modellierung
gegeben, die nicht nur die SF-Analyse ablöst, sondern eine verbindende
Methode für andere Instrumente ist, die im Rahmen der TRIZ angewendet werden. 

Ziel dieser Arbeit ist es, den aktuellen Stand der Dinge in der Verbesserung
der SF-Analyse zu beleuchten, Vorschläge für deren radikale Veränderung sowie
auch zur Erhöhung der Effizienz der Anwendung anderer TRIZ-Tools am Beispiel
von Konfliktlösungstechniken zu unterbreiten.

\section*{Hintergrund}

Heutzutage beschränkt sich die SF-Analyse meist auf den Einsatz in Schulungen.
Ihre praktische Anwendung wird durch eine Reihe von wesentlichen Nachteilen
beschränkt. Ich denke, es ist hier nicht erforderlich, genauer darzustellen,
was sich hinter diese Art der Analyse verbirgt. Hier nur einige Definitionen.

Eine der ersten Definitionen eines SF-Modells wurde in [1] gegeben: \emph{„Ein
  SF-Modell ist ein minimal vollständiges Modell eines idealen technischen
  Systems.“} Hier kommen die folgenden Begriffe vor: Feld (F), Substanz (S)
und Medium (M). Als Variante wurde ein SF-Modell mit einem Feld (F) und zwei
Substanzen (2S) betrachtet, von denen die eine jeweils das Medium in Bezug auf
die andere ist. Diese letzte Variante, mit den zwei Substanzen "Werkzeug" und
"Produkt", wird auch heute noch verwendet. Jedoch ist das Produkt nach
modernen Vorstellungen kein Teil eines funktionalen Systems (FS). Allerdings
haben sich die Definitionen dabei nur unwesentlich geändert, zum Beispiel:
\emph{„Ein SF-Modell ist ein minimales Modell eines technischen Systems: es
  umfasst das Produkt, das Werkzeug und die Energie (Feld), die notwendig ist
  für die Einwirkung des Werkzeugs auf das Produkt“} [2, S. 91].

Damit wird für eine SF-Analyse auch heute noch das Dreieck „2S+1F“
konstruiert, das von Lösern sogar dann verwendet wird, wenn Aufgaben nicht zu
diesem Modell passen.

Hier ist ein Beispiel für eine weitere Definition, wo erstens nicht klar ist,
auf der Grundlage welcher Regel das Vorhandensein des herausgearbeiteten
Widerspruchs vorgeschlagen wird, und zweitens die mentale Trägheit des Denkens
„eingeschaltet“ wird und klar für die Konstruktion eines Dreiecks nach der
Formel „2S+1F“ plädiert wird: \emph{„Die SF-Analyse wird in der operativen
  Zone des Auftretens der Aufgabe durchgeführt, d.h. dort, wo der physische
  Widerspruchaufgedeckt wurde. Es müssen an jenem Ort unbedingt zwei
  Substanzen S1 und S2 vorhanden sein, die nützlich oder schädlich miteinander
  interagieren, und das Feld F, das diese beiden Substanzen verbindet“} [3,
  S. 89].

Eine nahe Variante der Lösungssuche durch Formulierung und Auflösung
physikalischer Widersprüche wird in [4] vorgeschlagen. Als Hauptnachteil wird
hier die Formulierung von inhärent auf der Ebene von Subsystemen existierender
(physischer) Widersprüche auf Systemebene gesehen, die unmittelbar bei der
Analyse der Problemsituation erfolgt.  Der Ausschluss eines solchen Übergangs
zwischen Systemebenen beraubt die Methode zusätzlicher heuristischer Kraft.

Es wird auch versucht, die SF-Analyse selbst zu verbessern, aber auch diese
neuen Versionen haben keine massenhafte Verbreitung erfahren.

\section*{Ein kurzer Überblick über Arbeiten zur Verbesserung der SF-Analyse} 

In letzter Zeit wurden nicht viele Arbeiten zur SF-Analyse und zum System der
Standards veröffentlicht. Der größte Teil von ihnen wurden auf dem
Entwicklergifel 2013 im Thema "Weiterentwicklung der SF-Analyse" vorgestellt.

Alle Artikel enthalten Auflistungen sowohl allgemeiner Schwachstellen dieses
Instruments als auch Nachteile, welche die Lösung spezifischer Probleme
behindern.  So werden in den Arbeiten [5, 6] Varianten der Modifikation der
klassischen SF-Analyse für Informationssysteme vorgestellt:
\begin{itemize}
\item Der Begriff „Substanz“ wird durch „Element“ ersetzt und der Begriff
  „EleFeld“ verwendet, und es wurde auch versucht, dies mit der
  Funktionsanalyse zu kombinieren: \emph{„Die Gemeinsamkeit eines EleFelds mit
    einem Funktionsmodell ermöglicht es, Werkzeuge der Funktionsanalyse in die
    EleFeld-Analyse zu übertragen. Wir nennen diese Art der Analyse der
    Funktionsfeldanalyse von Systemen“}; und mit funktional-orientierter Suche
  (FOS).  Allerdings wird dann im Vergleich zu „Mengen“ die folgende Aussage
  getroffen: \emph{„sowohl in der Menge als auch im EleFeld können Funktionen
    vorhanden sein oder auch nicht“} [5];
\item Das SF-Modell wurde in EL-Action umbenannt (genau so -- auf Englisch):
  Element-Wirkung und es wird eine weitere Komponente eingeführt -- Knowledge
  (Wissen): \emph{„Ein Modell, das Elemente, Aktionen und Wissen umfasst, wird
    EAK genannt.  Die Methodik der Analyse und Transformation von EAK wird als
    EAK-Analyse bezeichnet.“} Ebenfalls werden Besonderheiten der Anwendung
  der EAK-Analyse für Infomationsverarbeitungssysteme berücksichtigt:
  \emph{„In diesen Systemen wird ein Element als Daten dargestellt (Data --
    D), eine Aktion ist eine Funktion (Function -- F) und Wissen (Knowledge --
    K). Das Modell, das Data, Function, Knowledge umfasst, nennen wir DFK.
    Die Methodik der Analyse und Transformation von DFK-Modellen wird als
    DFK-Analyse bezeichnet“} [6].
\end{itemize}
In [7] wurde eine Variante der Modellierung von Interaktionen in biologischen
(lebende) Systeme betrachtet, ohne über die klassischen Vorstellungen einer
SF-Analyse hinauszugehen.

In der Rezension [8] des Buches wird über die Anpassung der SF-Analyse und des
Systems der Standards für die schnelle Ausbildung von Ingenieuren geschrieben
-- hier werden der Einfachheit halber Transformationen der standardmäßigen
SF-Formel nur mit 5 Regeln ausgeführt, die aus den 76 Standards ausgewählt
wurden, wahrscheinlich basierend auf der Häufigkeit ihrer Verwendung. Darüber
hinaus untersucht der Autor von [8] die Variante, ein System als eine
Ansammlung von Feldern und Substanzen zu fassen, was vollständig analog zur
strukturell-energetischen Synthese von Systemen aus einer früheren Arbeit ist
[9, S. 73].

Im Aufsatz [10] über SF-Modelle wird die Anwendung eines Prozessansatzes
vorgeschlagen, in dem die Begriffe "Substanzen" und "Feldern" durch
"Ressource" ersetzt werden. Und am Ende schlägt der Autor ähnliche
Prozessdiagramme vor: \emph{„1. In allen betrachteten Fällen ist dies in
  keiner Weise weniger heuristisch als das klassische SF-Design. 2.  Eine
  Vielzahl von Proto-SF-Schemata der Aufgabe und damit Subjektivität lassen
  sich so vermeiden. 3.  Es werden die klassische SF-Modellierung, die
  funktional-ideale Modellierung und ein Prozessansatz als solcher logisch
  vereinigt. 4. Es wird vorgeschlagen, den Begriff „SF-Analyse“ durch etwas
  wie „System-Prozess-Modellierung“ (SPM) zu ersetzen.“} Allerdings hat auch
eine solche Darstellung der Methode der Konfliktmodellierung keine Fortsetzung
erfahren.

In anderen Arbeiten beschränken sich die Autoren hauptsächlich auf die
Auflistung von Mängeln der klassischen SF-Analyse und auf einzelne
Verbesserungsvorschläge. Insbesondere werden Wünsche zur Verbesserung der
Visualisierung des Prozesses der strukturellen und dynamischen Beschreibung
(technische) Systeme geäußert, was in erster Linie mit der Unvollkommenheit
der grafischen Mittel der SF-Analyse verbunden ist. Zu diesem Zweck wird
vorgeschlagen, eine Leistungsbeschreibung für entsprechende Forschungsarbeiten
zusammenzustellen [11].

Es wird empfohlen, die SF-Analyse und die funktionale Wertanalyse (FWA) [11,
  12] zu vereinigen, indem ihre starken Seiten kombiniert und auch die
Axiomatik dieses Werkzeugs mit der Axiomatik der Systemanalyse und der TRIZ
insgesamt abgeglichen werden [13]. Aber auch diese Empfehlungen haben keine
wirkliche Umsetzung gefunden.

Darüber hinaus wird vorgeschlagen, das Studium der SF-Analyse im allgemeinen
TRIZ-Kurs zu reduzieren [11] oder sogar auf die Entwicklung der SF-Analyse und
des Systems der Standards als eigenständiges Instrument ganz zu verzichten
[12].

\section*{Liste der Hauptnachteile verschiedener Versionen der Analoga der
  SF-Analyse:}
\begin{itemize}
\item Es gibt immer noch eine Vermischung der Konzepte „System“ und „Konflikt“
  in Modellierungsvarianten, die auf der SF-Analyse basieren.
\item Die Nutzung von Bedingungen der Existenz und Arbeitsfähigkeit von
  Systemen ist nicht vollständig realisiert, auch ist ein funktionaler Ansatz
  so gut wie nicht vertreten.
\item Die Begriffe „Substanz“ und „Feld“, wenn sie auch durch andere ersetzt
  werden, dann eher zu allgemein [10] oder umgekehrt - hochspezialisiert [6];
\item Varianten zur Ableitung von Lösungsbildern aus den konstruierten
  Modellen fehlen entweder oder werden auf die Verwendung des Systems der
  Standards reduziert: ein wenig modifiziert [5, 9] oder signifikant reduziert
  [8], d.h. sie basieren auf dem gleichen Prinzip wie die klassische
  SF-Analyse.
\item Keine der vorgeschlagenen Modellierungsvariationen schlägt eine
  Visualisierung von Widersprüchen vor.
\end{itemize}

\section*{Grundlegende Beobachtungen}

Bei der Lösung von Problemen durch die klassische SF-Analyse kann man nicht
immer den gewünschten Standard verwenden oder einen geeigneten auswählen,
besonders, wenn er keine grafische Interpretation hat.  Zum Beispiel ist die
Aufgabe, eine Nadel im Heuhaufen zu finden, allgemein bekannt.  In de n
meisten Fällen wird sie mit der SF-Syntheseregel (Standard 1.1.1 [14]) gelöst,
in der es zwei Substanzen -- Nadel und Heu -- und kein Feld gibt [3, S. 90].
Es scheint, dass eine solche Konstruktion des Modells offensichtlich ist und
keiner Erklärung bedarf. Aber in Wirklichkeit wird die Aufgaben durch eine
andere ersetzt: statt eine Suche (Finden) auszuführen, erfolgt die Trennung
von Heu und Nadeln.  Genauso kann man auch das Problem der Suche eines Fisches
im Ozean lösen -- werden Sie diese durch Trennen von Fisch und Wasser durch
Anwenden eines Feldes lösen?

Eine der Hauptursachen für solche Modellierungsfehler ist das Fehlen
funktioneller Komponenten in der SF-Analyse. Dies spiegelt sich aber in den
Übersichten der vorgestellten Arbeiten nicht gebührend wider. Der zweite Grund
ist die schlechte Anwendung eines systematischen Ansatzes, was am ehesten
damit verbunden ist, dass die Modellierung von Systemen (funktionaler Systeme
oder klassisch -- technischer Systeme) nur beschreibenden Charakter hat.

Ein Versuch, einen systematischen Ansatz anzuwenden, wurde in [9, S. 82] bei
der Entwicklung einer strukturell-energetischen Synthese von Systemen
unternommen. Aber solche „Energieketten“, die aus elementaren Strukturgliedern
aufgebaut sind, auch wenn sie ein wichtiger Schritt zur Verbesserung der
Effizienz der SF-Analyse sind, enthalten auch wesentliche Mängel, die der
weiteren Verbreitung dieser Methode im Wege stehen, nämlich:
\begin{itemize}
\item Als Elementarkomponenten werden immer noch dieselben „Stoffe“ und
  „Felder“ verwendet, womit die Prinzipien der Interaktion aus der SF-Analyse
  entlehnt werden.
\item Der gerichtete Energiefluss ist durch physikalische Effekte begrenzt:
  \emph{„Substanz-Feld-Paare, mit denen Systeme gesteuert werden können, sind
    in der Natur viele bekannt. Insbesondere können ferromagnetische
    Substanzen oder Ferropartikel mit magnetischen oder elektromagnetischen
    Feldern verwendet werden; ebenso elektrisches Feld und elektrorheologische
    Flüssigkeiten (d.h.  Gemische, die ihre Viskosität unter elektrischem
    Einfluss ändern Felder)”}.
\item Das Prinzip der Vollständigkeit von Systemen, das 4 Arten von
  elementaren Strukturgliedern einschließt, und in der Beschreibung der
  Methode vorgestellt wird, wird nicht in vollem Umfang in der Konstruktion
  von Modellen verwendet, was zum Auslassen einer Reihe möglicher Richtungen
  der Transformation dieser Modelle führt.
\item Das „Produkt“ wird als Empfänger des Feldes betrachtet -- in „sich
  ändernden“ Systemen -- oder als dessen Transformator - in "messenden"
  Systemen, aber nirgendwo wird es zu einem vollständigen Funktionssystem
  ergänzt, analog zum „Werkzeug“, was den Anwendungsbereich derartiger Modelle
  erheblich begrenzt und das Suchfeld möglicher Lösungen einschränkt -- in
  diesem Fall -- aufgrund nicht genutzter Ressourcen des „Produkts“.
\item Es fehlt Funktionalität: Die Interaktionen elementarer Kettenglieder
  wird nicht konkretisiert, sogar wenn eines von ihnen das „Produkt“ ist.
\end{itemize}
Was muss getan werden? Für den Anfang die meisten oben erwähnten Nachteile
beseitigen. Insbesondere muss man sich von der stereotypen Beschreibung des
Modells als Dreiecks "2S+1F" trennen, das Gebiet der Modellierung -- die
Konfliktsituation -- eindeutig bestimmen und, bitteschön, das Allerwichtigste
-- Begriffe und Modelle anwenden, die aus bewährten Ansätzen bekannt sind: dem
systemischen und dem funktionalen.

In der Praxis kommt der systematische Ansatz daher mit dem funktionalen, weil
Objekte (Geräte, Prozesse) der realen Welt, die von Menschen benutzt werden,
jedes seinen eigenen Zweck hat, der durch den \textbf{Funktionsbegriff}
ausgedrückt wird.  Somit werden Objekte als \textbf{funktionale Systeme} (FS)
modelliert.  In der TRIZ besteht ein funktionales System aus einer Reihe von
Teilen.  Zum Beispiel wird postuliert in [15, S. 94-97] der Autor das
obligatorische Vorhandensein von vier Teilen in einem technischen System (TS):
Antrieb (A), Transmission (T), Steuerungsorgan (S) und Arbeitskörper (W). Die
Energiequelle (E) fällt hier entweder mit dem Antrieb zusammen oder wird aus
dem System herausgenommen, wenn die Energie von außen kommt, auch von einem
Menschen.

Und nun ist es erforderlich, von der beschreibenden Natur der Modellierung von
Objekten als FS zu einer Instrumentalversion überzugehen. Aber dafür muss man
besser verstehen, was die ausgewählten Teile (E, A, T, S, W) genau darstellen
und welche Rolle sie für die Funktionsweise des Systems spielen.

K. Marx schreibt zu diesem Thema [16]: „Alle entwickelte Maschinerie besteht
aus drei wesentlich verschiednen Teilen, der Bewegungsmaschine, dem
Transmissionsmechanismus, endlich der Werkzeugmaschine oder Arbeitsmaschine.
Die Bewegungsmaschine wirkt als Triebkraft des ganzen Mechanismus. Sie erzeugt
ihre eigne Bewegungskraft, wie die Dampfmaschine, kalorische Maschine,
elektro-magnetische Maschine usw., oder sie empfängt den Anstoß von einer
schon fertigen Naturkraft außer ihr, wie das Wasserrad vom Wassergefäll, der
Windflügel vom Wind usw. Der Transmissionsmechanismus, zusammengesetzt aus
Schwungrädern, Treibwellen, Zahnrädern, Kreiselrädern, Schäften, Schnüren,
Riemen, Zwischengeschirr und Vorgelege der verschiedensten Art, regelt die
Bewegung, verwandelt, wo es nötig, ihre Form, z.B. aus einer perpendikulären
in eine kreisförmige, verteilt und überträgt sie auf die Werkzeugmaschinerie.
Beide Teile des Mechanismus sind nur vorhanden, um der Werkzeugmaschine die
Bewegung mitzuteilen, wodurch sie den Arbeitsgegenstand anpackt und zweckgemäß
verändert.“

Bei der Analyse des vorhandenen Instrumentariums und seiner Anwendungen auf
die Lösung von Aufgaben wurde offensichtlich, dass die Teile des Systems
Energiewandler sind (genauer gesagt -- deren Charakteristik), welche der
Arbeitskörper benötigt, um den Zweck des Systems (die Funktion) zu
realisieren:
\begin{itemize}
\item Umwandler der ersten Art ändern die Energiequalität, d.h. deren Typ.
\item Umwandler der zweiten Art ändern die Menge und raum-zeitliche
  Organisation (Intensität, Richtung, Anwendungsort) des Energieflusses.
\end{itemize}
In Gänze gesprochen, um zu verstehen, was jedes Teil des Systems (als
Umwandler) darstellt, ist es bequemer, ein solches Konzept wie
\textbf{Element} zu verwenden, das ausgedrückt werden kann als
\begin{itemize}
\item[1.] Substanz,
\item[2.] Energie,
\item[3.] Information.
\end{itemize}
Von daher ist es offensichtlich, dass das \textbf{funktionale System bestimmt
  ist} durch das Vorhandensein eines Satzes von grundlegenden
\textbf{Elementen} (mit eigener Struktur und innerer Funktionalität) und einer
\textbf{externen Funktion}, d.h. der Fähigkeit, eine Änderung eines Merkmals
der Elemente anderer Systeme zu bewirken: Zustände, Eigenschaften, Parameter.

Um die Arbeitsfähigkeit des Systems zu gewährleisten, ist es erforderlich, das
dieses über ein ausführendes Element verfügt sowie die Fähigkeit, dass dieses
Element Energie in ausreichendem Umfang und geforderter Form für die
Realisierung der gegebenen Funktion empfangen kann.

In den meisten Fällen muss das System mindestens zwei Elemente enthalten:
Arbeitskörper und Energiequelle. Bei Bedarf werden zusätzliche Energiewandler
sowie ein Minimalsystem der Steuerung eingeführt, d.h. die Möglichkeit, die
Energieversorgung zum Arbeitsorgan ein- und auszuschalten.

Und natürlich kann die Konstruktion nicht ohne das Vorhandensein eines
„Produkts“ auskommen, das ebenfalls zu einem vollen FS ergänzt werden muss,
analog durch Hinzufügen einer Energiequelle zu seinem ausführenden Element.

Beim Vorhandensein unbefriedigend ausgeführter Funktionen entsteht ein
Konflikt, was bedeutet, dass es ein Konfliktmodell gibt und ein solches Modell
kann als \textbf{Element-Funktions-Modell} oder \textbf{EFM.K} bezeichnet
werden.

\begin{center}
  Bild einfügen\\ Variante der Darstellung eines EFM
\end{center}
Somit sind zwei Grundbausteine identifiziert, aus denen jedes funktionale
System besteht: 1) Ein Element als Energieumformer und 2) eine Funktion.

Bei der Element-Funktions-Modellierung hinreichend großer Systeme, genauer
gesagt von Systemen mit einer großen Anzahl von Elementen, werden erstens nur
diejenigen Elemente berücksichtigt, die an den betrachteten Interaktionen
beteiligt sind, und zweitens werden die Grenzen der Elemente sowie die Grenzen
des Systems basierend auf den Aufgabenbedingungen und dem gesunden
Menschenverstand festgelegt. Das heißt, wenn wir Komponenten verbessern, zum
Beispiel eines elektrischen Generators, können wir alle als einzelne Elemente
mit dem erforderlichen Detaillierungsgrad betrachten. Aber wenn der Generator
die elektrische Energiequelle des Systems ist, dann muss er als einzelnes
I-Element dargestellt sein.

Das heißt, hier wirkt das Prinzip der Fraktalität, wenn jedes Element, bis zu
einer gewissen Grenze, in seine Bestandteile zerlegt werden kann, deren
Beziehungen untereinander auch funktionale Systeme darstellen.

Darüber hinaus kann man über verschiedene Arten solcher Modelle und Varianten
ihrer Transformationen sprechen, die auf Regeln hinauslaufen. Das heißt:
zuerst ist die Situation zu analysieren und ein Element-Funktions-Modell des
Konflikts aufzubauen. Danach wird das Modell analysiert und durch Anwendung
der Regeln werden neue Modelle synthetisiert: Modelle der Richtungen der Suche
nach Lösungswegen (RSL).
\begin{center}\bf   EFM.K $\to$ EFM.RSL \end{center}
Danach kommt die Synthese von Lösungen. Dafür sind aber schon heuristische
Methoden, Analogien und figuratives Denken (Phantasie) anzuwenden, die zur
Überwindung der finalen Barriere beitragen, welche Entscheidungsmodelle von
ihren realen Inkarnationen trennen.

Die beliebteste heuristische Methode ist die Verwendung der Prinzipien. Aber
wie kann man diese Prinzipien am effektivsten einsetzen?

\section*{Prinzipien}

Die Aufgabe, eine Nadel im Heuhaufen zu finden, ist recht einfach. Aber es
gibt auch schwierigere Aufgaben, die hier nicht betrachtet werden. Deshalb
gliedert sich Frage des Einsatzes der Prinzipien in zwei Teile:
\begin{itemize}
\item[1.] In einfachen Fällen erhält man die Lösung auf der Ebene der Regeln
  und es stellt sich die Frage „Warum brauchen wir Prinzipien?“.
\item[2.] Bei schwierigeren Aufgaben stellt sich eine etwas andere Frage: „Wie
  die Prinzipien benutzen?“ oder „Wie auf sie kommen?“
\end{itemize}
In der klassischen TRIZ gibt es zwei \textbf{Werkzeug-Analoge}, bei denen
Prinzipien angewendet werden:
\begin{itemize}
\item[1.] Um \textbf{physische Widersprüche} im Rahmen von ARIZ aufzulösen.
\item[2.] Als Ergebnis der Arbeit mit der Altschuller-Matrix, die den Zugriff
  auf Prinzipien über die Formulierung sogenannter \textbf{technischer
    Widersprüche} kodiert.
\end{itemize}
Kann man diese Varianten irgendwie für eine Transformation der \textbf{EFM.K}
verwenden? 

Eingans definieren wir im Rahmen des Modells das Wesen des Problems und die
Richtungen für seine Beseitigung. Die Spitze des Eisbergs ist eine
\textbf{Funktion}, da diese die Rolle eines Indikators für die Aktivität von
Elementen hat. Und der Konflikt ist vorhanden, wenn die Funktion
unbefriedigend, schädlich oder nicht dort vorhanden ist, wo sie sein sollte.

Dies spricht dafür, dass es unbefriedigende Eigenschaften in den Elementen
gibt, die an der Realisierung dieser Funktionen beteiligt sind und für deren
Ausführung sowohl von der Seite des „Werkzeugs“ als auch von der Seite des
„Produkts“ verantwortlich sind.

Trotz der qualitativen Unerschöpflichkeit der Materie, da „jedes Ding
unzählige verschiedene Qualitäten hat. Nach der Untersuchung jeder endlichen
Anzahl von Eigenschaften einer Sache können wir nicht sagen, dass diese Sache
von uns bereits vollständig studiert wurde“ [17, S. 70], -- ist es möglich,
sich auf eine endliche Anzahl von Merkmalen zu beschränken, die für die
Bedingungen einer bestimmten Aufgabe von Bedeutung sind und am meisten die zu
realisierenden Funktionen beeinflussen. Das heißt, wenn diese Eigenschaften
geändert werden, ändern sich auch die Funktionen in die gewünschte Richtung
(normalisieren sich). Und genau für einen solchen formalen Ansatz können
Prinzipien angewendet werden, die auf die Transformation von Elementen und
ihrer Beziehungen ausgerichtet sind.

Die Anwendung von Techniken in der Anfangsetappe kann zu zwei Hauptergebnissen
führen:
\begin{itemize}
\item[1.] Der Konflikt ist beseitigt, es gibt keine unerwünschten Folgen --
  das Problem ist gelöst. 
\item[2.] Das Prinzip beseitigt den anfänglichen Konflikt, es ergibt sich
  jedoch eine unerwünschte Änderung in der normalen Funktion der Elemente --
  eine unbefriedigende Funktion kann auftreten, die mit der korrigierten
  (normalisierten) verbunden ist. In diesem Fall können wir vom Auftreten
  eines \textbf{Widerspruchs der Bedingungen} (WB) sprechen -- hier sind die
  Bedingungen mit der Anwendung oder Nichtbenutzung des Prinzips verbunden.
\end{itemize}
Wenn die Prinzipien nicht sofort funktionieren und der Konflikt nicht ohne
Folgen gelöst werden konnte, dann heißt das, dass das Problem auf der ebene
des Systems nicht gelöst werden kann und es ist notwendig, entweder auf die
Ebene des Obersystems oder auf die Mikroebene zu gehen, wo der \textbf{WB} als
\textbf{Widerspruch der Anforderungen (WA)} umformuliert wird, zu dessen
Lösung ebenfalls auf die Prinzipien zurückgegriffen werden kann.

\emph{\textbf{Erklärung.} Die Begriffe \textbf{Widerspruch der Bedingungen
    (WB)} und \textbf{Widerspruch der Anforderungen (WA)}\footnote{Die
    Begriffe WB und WA wurden 2013 vom Autor eingeführt und erfolgreich in
    praktischen Übungen am MOU TRIZ in St. Petersburg getestet.}  werden als
  allgemeinere Begriffe für Widersprüche anstelle von technischer Widerspruch
  (TW) und Physikalischer Widerspruch (PW) der klassischen TRIZ verwendet.
  Solche Bezeichnungen sind auch universeller und daher in verschiedenen
  Tätig"|keits"|bereichen (nicht nur in der Technologie) anwendbar, drücken
  aber auch das Wesen der Widersprüche dieser Typen aus: einschließlich der
  Konflikte zwischen Bedingungen und Anforderungen.}

\section*{Vorteile und Super-Effekte}
Es wurde eine Variante der Konfliktmodellierung vorgestellt, die folgende
Hauptvorteile besitzt:
\begin{itemize}
\item Es werden Ressourcen systemischer und funktionaler Ansätze verwendet:
  deren Synthese und Instrumentalisierung wurde verwirklicht.
\item Es werden psychologische Barrieren in der Etappe der Beschreibung der
  Modellbildungs-Regeln überwunden.
\item Eindeutigkeit und Universalität der Begriffe sowie Genauigkeit bei der
  Steuerung der der Regeln des Erstellens und Transformierens von Modellen --
  all das erlaubt es, schwierige Konflikte zu modellieren, die in beliebigen
  Tätigkeitsfeldern auftreten können.
\item Der Platz der Prinzpien wurde bestimmt und Möglichkeiten ihrer
  Konstruktion erweitert (siehe Präsentation).
\item Es wurde ein bedeutender Schritt in Richtung eines neuen Verständnisses
  des Wesens von Erfindungs-Werkzeugen und ihrer integrierten Verwendung
  gegangen.
\end{itemize}

\section*{Literatur}
\begin{itemize}
\item[1.] G. Altschuller, Ch. Gadzhiev, I. Flikstein.
  \foreignlanguage{russian}{Введение в вепольный анализ} (Einführung in die
  SF-Modellierung). – Baku, OLMI, 1973.
\item[2.] G.S. Altschuller, B.L. Zlotin, A.V. Zusman, V.I. Filatov.
  \foreignlanguage{russian}{Поиск новых идей: от озарения к технологии. Теория
    и практика решения изобретательских задач.}  (Such nach neuen Ideen: Von
  der Erleuchtung zur Technologie. Theorie und Praxis des Lösens von
  Erfindungsaufgaben). Kischinjow. Kartja Moldovenjsake, 1989.
\item[3.] G.I. Ivanonv. \foreignlanguage{russian}{Формулы творчества, или Как
  научиться изобретать: Кн.  для учащихся ст. классов.} (Formeln des
  Schöpfertums, oer wie man Erfinden lernt. Lehrbuch für Schüler höherer
  Klassen). Moskau, Verlag Prosweschtschenie, 1994.
\item[4.] V.N. Glasunov. \foreignlanguage{russian}{Параметрический метод
  разрешения противоречий в технике. Методы анализа проблем и поиска решений в
  технике.} (Der parametrische Zugang zum Lösen von Widersprüchen in der
  Technik.  Problemanalysemethoden und Lösungssuche in der Technik) Moskau,
  Verlag Retschnoj Transport, 1990.
\item[5.] M.S. Rubin. \foreignlanguage{russian}{Элепольный анализ как развитие
  вепольного и функционального анализа в ТРИЗ}. (EleFeld-Analyse als
  Entwicklung der SF-und funktionalen Analyse in der TRIZ). TRIZ Developers
  Summit 2013.  URL: \url{https://triz-summit.ru/confer/tds-2013/articles/}
\item[6.] V. Petrov, G. Voronov. \foreignlanguage{russian}{Новый подход к
  вепольному (структурному) анализу}. (Ein neuer Zugang zur
  SF-(Struktur)-Analyse).  TRIZ Developers Summit 2013.  URL:
  \url{https://triz-summit.ru/confer/tds-2013/articles/}
\item[7.] Sara Greenberg. Introducing substance-field, as a method for
  studying living systems. (Einführung der SF-Methode als Methode zum Studium
  lebender Systeme). TRIZ Developers Summit 2013.  URL:
  \url{https://triz-summit.ru/confer/tds-2013/articles/}
\item[8.] Yu. Belski. \foreignlanguage{russian}{Инструменты ТРИЗ для ХХI века:
  современный вещественно-полевой анализ}. (TRIZ-Instrumente für das
  21. Jahrhundert: Moderne SF-Analyse). 2008. URL:
  \url{https://refdb.ru/look/2055665.html}
\item[9.] B.I. Goldovski, M.I. Wainerman.
  \foreignlanguage{russian}{Рациональное творчество. О направленном поиске
    новых технических решений.} (Rationales Schöpfertum. Über die gezielte
  Suche neuer technischer Lösungen) – Moskau, Verlag Retschnoj Transport,
  1990.
\item[10.] V.A. Koroljov.  \foreignlanguage{russian}{Веполи: 20 лет спустя
  (2)}.  (SF-Modelle: 20 Jahre später. Teil 2).  URL:
  \url{http://coroliov.trizinfor.org/works/ws6.html}
\item[11.] N.B. Feygenson.  \foreignlanguage{russian}{Вепольной анализ и его
  аналоги – прагматические аспекты}. (SF-Analyse und ihre Analogien --
  pragmatische Aspekte).  TRIZ Developers Summit 2013.  URL:
  \url{https://triz-summit.ru/confer/tds-2013/articles/}
\item[12.] S.A. Logvinov. \foreignlanguage{russian}{Проблемы обновления
  системы стандартов и вепольного анализа}. (Probleme der Systemerneuerung und
  SF-Analyse). TRIZ Developers Summit 2013.  URL:
  \url{https://triz-summit.ru/confer/tds-2013/articles/}
\item[13.] V.A. Koroljov. \foreignlanguage{russian}{Веполи: 20 лет спустя}.
  (SF-Modelle: 20 Jahre später.) URL:
  \url{http://coroliov.trizinfor.org/data/c82.htm}
\item[14.] G.S. Altschuller. \foreignlanguage{russian}{Маленькие необъятные
  миры: стандарты на решение изобретательских задач // В сб. "Нить в
  лабиринте".} (Kleine unermessliche Welten: Standards der Lösung von
  Erfindungsaufgaben). Petrosawodsk, 1988. – S. 165-230. URL:
  \url{http://www.altshuller.ru/triz/standards.asp}
\item[15.] Yu.P. Salamatov. \foreignlanguage{russian}{Как стать изобретателем:
  пособие для учителя.} (Wie wird man Erfinder. Handreichung für
  Lehrer). Moskau, Verlag Prosweschtschenie, 2006.
\item[16.] K. Marx. Kapital, Band 1, Kap. 13. MEW 23.
\item[17.] A.I. Uiomov. \foreignlanguage{russian}{Вещи, свойства и отношения.}
  (Dinge, Eigenschaften und Beziehungen). Moskau, Verlag der AdW, 1963.
\end{itemize}

\end{document}
