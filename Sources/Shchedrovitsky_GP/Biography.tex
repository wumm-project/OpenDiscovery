\documentclass[a4paper,11pt]{article}
\usepackage[utf8]{inputenc}
\usepackage[main=english,russian]{babel}
\usepackage{od}

\title{Biography of G.P. Shchedrovitsky}
\author{Translated by Hans-Gert Gräbe, Leipzig}

\date{December 7, 2021}

\begin{document}
\maketitle

\begin{quote}
  Source: \foreignlanguage{russian}{ Биография Г. П. Щедровицкого}\\
  \url{https://www.fondgp.ru/schedrovitsky-gp/biography/} (2021-12-07)

  Translated with support of \texttt{deepl.com}.
\end{quote}

Text prepared by G.A. Davydova on the basis of autobiographical materials.

Georgy Petrovich Shchedrovitsky was born on 23 February 1929 in Moscow. His
father, Pyotr Georgiyevich, a graduate of the Moscow Higher Technical School,
worked in the aviation industry. His mother, Kapitolina Nikolaevna, worked as
a doctor after graduating from the First Moscow Medical Institute.

In 1946 G.P. Shchedrovitsky entered the faculty of physics at Moscow
University (specializing in theoretical physics). His field of scientific
interests had already been formed during his first years at the Physics
Department -- structure of scientific theories with an intention to work out a
project of the theory of thinking. This predetermined his move to the
Philosophy Faculty (1949), where he got involved in philosophical problems of
natural science and then in logic and methodology of science. While still a
fourth year student, he began teaching at school (logic, psychology, physics).
In 1953, he graduated with honours from the Faculty; his thesis was devoted to
the mechanisms of development of scientific notions.

Immediately after graduation G.P. Shchedrovitsky begins working on his
doctoral dissertation \emph{Linguistic Thinking and Methods of its Analysis},
trying to combine in a special way the ideas, means and methods of logic,
linguistics, psychology and sociology. After defending his thesis, he was
awarded a Ph.D. degree in Philosophy (1964).

Georgy Petrovich's first scientific articles (on problems of thinking and
thinking activity) date back to 1957. At the same time, the programme of
complex research of thinking and activity developed by him together with
N.G. Alexeev and I.S. Ladenko was published in the journal \emph{Doklady APN
  RSFSR}\footnote{APN RSFSR -- Academy of Pedagogical Sciences of the Russian
  Federation.}.  The program is still being implemented in many researches.

His seminars occupy a special place in his scientific and pedagogical
activities because this was the way to involve students and young scientists
in his research programs. Since 1955 he held a seminar on methodology and in
the same year he organised a seminar on the complex and systematic study of
thinking (at the Department of Logic in the Philosophical Department of
MSU\footnote{Moscow State University}). In March 1958 Georgy Petrovich
organized (jointly with V.V. Davydov and with the support of Professor
P.A. Shevarev) the Commission on the Psychology of Thinking and Logic within
the AllUnion Society of Psychologists, whose activity he maintained until the
end of his life. Finally, in 1962 at the Council on Cybernetics of the USSR
Academy of Sciences he established (jointly with V.N. Sadovsky and E.G. Yudin)
an interdisciplinary seminar on structural-system analysis methods in science
and technology, which he headed until 1976. The results of this seminar were
published, in particular, in \emph{Problems of Systems and Structure
  Research}\footnote{\foreignlanguage{russian}{Проблемы исследования систем и
    структур.}}  (1965) and a brochure of G.P. Schedrovitsky \emph{Problems of
  System Research Methodology}\footnote{\foreignlanguage{russian}{Проблемы
    методологии системного исследования.}} (1964).

Since April 1958 Georgy Petrovich starts working in the APN RSFSR publishing
house: he is in charge of psychology, physiology and industrial training
sections of the Pedagogical Dictionary; later he edits the works of
N.K. Krupskaya, P.P. Blonsky, books on theory and history of pedagogy in the
Pedagogical Editorial Board. During those years he also works in the theory
department (as editor) of the journal \emph{Questions of Psychology}.

His main area of scientific interest during those years is the
structural-systemic analysis of knowledge and thinking activity, the place and
limits of logical and normative methods of thinking analysis, and their
relation to psychological and psychological-pedagogical research of thinking.

From October 1960 to August 1965 G.P. Shchedrovitsky worked as a junior
research associate at the Laboratory of Psychology and Psychophysiology of the
Research Institute of Preschool Education at the OPN of the RSFSR, where he
was engaged in problems of mental development of preschool children, playing
of childs and its role in the development of "child's society", and
development of preschool and school children in terms of learning as well as
the analysis and typology of teaching-learning situations. The main topics of
research of this period include \emph{Investigations of Children's Thinking on
  the Material of Solving Arithmetic Tasks} (publications of 1960, 1961, 1964,
1965, 1974), \emph{The Methodology of Pedagogical Investigation of Playing}
(1963, 1964, 1966, 1973), \emph{Interrelation of Learning and Development from
  a Systemic Perspective} (1966, 1968, 1974). At the same time Georgy
Petrovich (together with B.V. Sazonov, V.M. Rozin, N.I. Nepomniashchaya,
N.G. Alexeev and A.S. Moskaeva) prepared for publication his fundamental work
\emph{Pedagogy and Logic} reflecting basic research directions in the
framework of substantial-genetic logic and activity theory.

Since 1960 G.P. Shchedrovitsky pays much attention to the problems of speech
and language and outlines a program of building system-activity semiotics. The
main works of this trend are: \emph{On the Method of Semiotic Research of Sign
  Systems} (published in 1963, 1965, 1967), \emph{The Natural and the
  Artificial in Sign Systems} (1965, 1966, 1967), \emph{On Basic Approaches in
  the Studying of Signs and Sign Systems} (1964, 1965, 1967).

In VNII\footnote{Allunion Institute of Scientific Research} of Technical
Aesthetics of the USSR State Committee for Science and Technology (August 1965
-- March 1969), G.P. Shchedrovitsky as a senior researcher leads the Design
Methodology group. The basic notions of activity theory, in particular of
project thinking and planning activities, methods of historical and
historical-critical analysis become the main target of his research. The
studies of this period are reflected in the works \emph{Design in a system of
  independent planning}\footnote{\foreignlanguage{russian}{Дизайн в системе
    обособляющегося проектирования}} (1965--1967) and \emph{Thinking of a
  Designer} (1966-1969). At the same time, an attempt is made to transfer the
notions of a sphere of activity developed on the material of design to the
sphere of pedagogy and linguistics: \emph{A System of Pedagogical Research}
(publications 1966 and 1970), and \emph{The Methodological Sense of the
  Problem of Linguistic Universals} (1966 and 1969). Additionally,
sociological approaches to the study of activity and thought are being
developed and deepened, and problems of organisation, leadership, and
management are becomig increasingly important. Studies of scientific and
technological movements -- design, systemic, organisational and managerial,
etc. -- are started.

In February 1968 a trial was held in Moscow against A. Ginzburg and
Y. Galanskov, who had studied at the school where Georgy Petrovich had taught
at the same time. For this or for some other reason, he signed a collective
letter of cultural and scientific figures to the leaders of the CPSU and the
government in defence of the accused. In July-August of the same year by the
decisions of the Regional and Moscow City Committee of the Communist Party of
the Soviet Union G.P. Shchedrovitsky was expelled from the CPSU (that he
entered in 1956) "for actions used to the detriment of the Party and country“.
As consequence the printing of \emph{Pedagogy and Logic} was stopped, such
that the book not came out in full circulation until 1993.

Nevertheless, up to March 1969 Georgy Petrovich continued to work in the
department of theory and methodology of design of VNIITE\footnote{Allunion
  Institute of Scientific Research on Technical Aesthetics}, headed the
preparation of a collective monograph \emph{Methodological Problems of Design
  Theory} -- until the sudden dismissal from the Institute due to staff cuts.
The immediate cause was an "answer" in the newspaper \emph{Pravda}, signed by
the then editor-in-chief V. Afanasyev, to an article by G.P. Shchedrovitsky
entitled \emph{Scientific Data or Self-Delution} in the \emph{Literaturnaya
  Gazeta}.  The article (the title was given by the newspaper) asserted that,
until sociology has formed its own scientific subject, the specific
sociological research conducted cannot be regarded as scientifically
substantiated.

As a result, Georgy Petrovich was left without a job and -- in the long run --
without means of subsistence, since in those years few people would have dared
to hire a PhD in philosophy who had been expelled from the Party. However,
such people were found: in April 1969, Shchedrovitsky joined the staff of the
Central Training and Experimental Studio of the Union of Artists of the USSR,
first as a methodist, and later as head of the teaching-methodological
laboratory.

During this period, disregarding the "circumstances", Georgy Petrovitch
continues to work on problems of semiotics and theory of understanding
(hermeneutics), studies the specifics of projecting, planning and programming
thinking, analyses the prospects of methodological thinking and methodology,
specific forms of methodics and methodical work, using every opportunity to
publish the results of his (and the collective) research.

The main works of this period include \emph{Sense and meaning in the structure
  of sign}\footnote{\foreignlanguage{russian}{Смысл и значение в структуре
    знака}} (publications of 1969, 1970, 1971 and 1974), \emph{Problems of
  Historical Development of
  Thinking}\footnote{\foreignlanguage{russian}{Проблемы исторического развития
    мышления}} (1973, 1974 and 1975), and \emph{Systemic Movement and
  Perspectives of Systemic-Structural
  Methodology}\footnote{\foreignlanguage{russian}{Системное движение и
    перспективы развития системно-структурной методологии}} (1974, 1979, 1981
and 1985). That was also the time when he started intensive development of the
bases for a general theory of activity, reflexion processes and their role in
the development of activity, and a more detailed analysis of communication
processes. Major publications within the framework of this trend were
\emph{Meaning and Knowledge}\footnote{\foreignlanguage{russian}{Значения и
    знания}} (1971), \emph{Activity, Communication,
  Reflexion}\footnote{\foreignlanguage{russian}{Деятельность, коммуникация,
    рефлексия}} (1974), \emph{Sense and
  Understanding}\footnote{\foreignlanguage{russian}{Смысл и понимание}}
(1977).

In 1975, a collective (A. Rappoport, O. Genisaretsky, B. Sazonov and others)
monograph \emph{The Development and Implementation of Automatic Systems in
  Design. Theory and Methodology} was published. In the section \emph{Initial
  Concepts and Categorical Means of Activity Theory}, written by Georgy
Petrovich, the main conceptions of the activity theory and of the system
approach were assembled.

In October 1974 G.P. Shchedrovitsky joins the Moscow Regional State Institute
of Physical Education as a senior lecturer at the Department of Pedagogical
Disciplines. He gave lectures in pedagogy and the history of pedagogy,
introduction to speciality, and conducted courses on the methodology of
scientific-pedagogical research in the domain of sports and on the methodology
of design of sports training systems, as well as special courses on the
socio-psychological structure of sports collectives and teams. Since 1977, he
has organised research on \emph{Methodological Recommendations for Designing
  an Annual Training Cycle (Content and Models of Organising Training sessions
  for Olympic Reserve Coaches)}.

The interest shown in the methodology by sports managers is understandable:
the Moscow Olympic Games are approaching. Since 1974 Georgy Petrovich is a
member of the Scientific Council of the Sports Committee of the USSR and Head
of the Commission for Structural-Systemic Research and Development in Physical
Culture\footnote{\foreignlanguage{russian}{физкультура}} and Sport. During
three years the Commission held five All-Union meetings on problems of
structured analysis of physical culture and sports. During the same years
G.P. Shchedrovitsky led (together with L.N. Zhdanov) the
Scientific-Methodological Seminar on problems of physical culture and sports
at the Moscow State University of Physical Culture and Sport (since 1976) and
the implementation of the complex scientific-methodological research programme
of industrial practices and practical preparation of students (since 1979)
organised by the USSR Ministry of Higher Education.

During this period the main subjects of research are forms of organisation of
collective thinking and activity, organisational and socio-psychological
structure of collectives, communication in groups and in collectives,
interdisciplinary relations. His principal texts are: \emph{On the basic
  aspects of sociological research in physical culture and sport as a field of
  activity} (1977), \emph{On the basic problems and directions of
  scientific-methodological research in the field of sport} (1977), \emph{The
  complex organisation of scientific research as a socio-technical system}
(1979) and others.

In 1979 Georgy Petrovich began research on \emph{Analysis of Techniques of
  Solving Complex Problems and Tasks under Conditions of Incomplete
  Information and Collective Action}. On their basis he developed a new form
of organising collective thinking and activity aimed at solving complex
interdisciplinary, industrial, scientific-technical and managerial problems,
which was called Organisational Activity
Games\footnote{\foreignlanguage{russian}{организационно-деятельностные игры}}
(OAG). In the period of 1979--1991 G.P. Shchedrovitsky conducted more than 90
OAGs, that served themselves as material and means of further research in the
field of thinking, activity, understanding and reflexion. In 1983
Shchedrovitsky and a group of coauthors prepared within the framework of the
theme \emph{Prospects and Ways to Automate the Systems of Thinking
  Activities}\footnote{\foreignlanguage{russian}{Перспективы и пути
    автоматизации систем мыследеятельности}} the following works:
\emph{Situational Analysis and Analysis of Situations}, \emph{Schemes and
  Signs in Thinking and Activities}, \emph{Schemes in Thinking and Signs in
  Communication}, \emph{Working with Schemes in Organisational Activity Games}.

In 1980--1983 G.P. Shchedrovitsky is working (as senior researcher) in the
Psychology Department of the Research Institute of General and Pedagogical
Psychology of the APN of the USSR. In December 1983, due to changes in the
head of NIIOPP\footnote{Scientific Research Institute of General Pedagogy and
  Rsychology} and the decision of the APN Presidium to close all research
related to the psychology of organisational management, Shchedrovitsky moves
to the Department of Methodology and Theory of Engineering Studies at the
Central Research Institute of the USSR Gosstroi, which in 1986 was transferred
(in the process of reorganisation) to the Institute of Production and
Scientific Research of Engineering Studies and Construction of the USSR
Gosstroi.

The main research topics of this period were: \emph{Category of complexity of
  design of explorational
  projects}\footnote{\foreignlanguage{russian}{Категория сложности
    производства проектно-изыскательских работ}} and \emph{Typology of
  situations of explorational
  projects}\footnote{\foreignlanguage{russian}{Типология ситуаций
    проектно-изыскательских работ}}. At the same time Georgy Petrovich
conducted an interuniversity seminar on the system approach in geology at the
Gubkin Moscow Institute of Petrochemical and Gas Industry.

Shchedrovitsky's next place of work was the All-Union Research Institute of
Theory of Architecture and Urban Planning where he was the head of the
Laboratory of Organisation of Projecting and Construction (December 1988 --
April 1992), recovering the former line of projecting and program development.

The last place of Georgy Petrovich's work was the International Academy of
Business and Banking (now Togliatti Management Academy) (April 1992 -- 1994).
Here he headed a group of his students who created the Network of
Methodological Laboratories for Designing Modern Education System and
Methodological Training and began to implement this project.

All in all Georgy Petrovich published about 150 works, among them in the USA,
England, Germany, Bulgaria and other countries. But this is only a part of
what was written by him during 40 years of active work, unceasing reflections,
discussions and conversations with colleagues, friends and pupils.

The last years of his life, which coincided with the period of perestroika,
led to the socialisation of the methodology. Within the Union of Scientific
and Engineering Societies of the USSR G.P. Shchedrovitsky created the
Committee on
SMD-Methodology\footnote{\foreignlanguage{russian}{СМД-методология =
    системо-мыследеятельностная методология} methodology of systemic thinking
  and acting} and Organisational Activity Games, organized and conducted five
All-Union Congresses of Methodologists (Kiev, Samara, Moscow), the last of
which (1993) was dedicated to the 40th anniversary of the Moscow
Methodological Circle.

G.P. Shchedrovitsky died on 3 February 1994.

\end{document}
