\documentclass[11pt,a4paper]{article}
\usepackage{od}
\usepackage[utf8]{inputenc}
\usepackage[main=english,russian]{babel}

\title{Principles and General Scheme of the Methodological Organization of
  System-Structural Research and Development.  }

\author{Shchedrovitsky, G. P. (1981)}

\begin{document}
\maketitle
\begin{quote}  
  Shchedrovitsky G.P. Selected Works. – M., Publishing School of Cultural
  Policy, 1994.  Electronic publication: Centre for Humanitarian Technologies. –
  20.02.2011.  URL: \url{https://gtmarket.ru/library/basis/3961/3967}

  Translated from the Russian original by Hans-Gert Gräbe, Leipzig We
  acknowledge support from the free version of Deepl.com and several remarks
  on the translation by Alexander Solodkin, Minsk.
\end{quote}

\section*{I. The current socio-cultural situation and the systemic movement} 

\paragraph{1.}
In the last 10-15 years, systems and systems analysis have become one of the
most fashionable topics and are discussed in different ways and from different
perspectives. There are a lot of different expressions and terms in use: For
example, the "systems revolution" that has engulfed the world of science,
engineering, and practice \cite{Ackoff1971, Ackoff1972}, the "systems
approach" that characterizes a new style and new methods of scientific
thinking \cite{Blauberg1969, Blauberg1973}, the "general theory of systems" as
a scientific theory of a special type that performs methodological functions
\cite{GTS1966, Zadeh1970, Mesarovich1973, Uemov1978}, the "general theory of
systems" as a metatheory \cite{Trends1972, Sadovsky1974}, and the "systems
analysis of operations" \cite{Quaid1969, Otdel1974, Optner1969}, "system
orientations" \cite{Yudin1972}, and so on.

However, it remains unclear whether all these expressions capture what has
already been created and actually exists, or only the projects and programmes
put forward by different groups of researchers.

In any case, with such an abundance of different points of view, we are forced
to ask what is actually happening now in this whole systemic area, and if it
turns out that it includes all the above-mentioned formations, then we will
have to somehow correlate and link them to each other to get an objective and
concrete picture of what is happening. But for this, of course, we need
special means and, in particular, some general idea, which would cover and
unite all the above-mentioned.

In our view, the most general yet most accurate notion, covering everything
that is currently happening in the 'systems field', would be the notion of a
systems movement.

This paper is based on the texts of papers given at the seminar "Structures
and Systems in Science and Technology" of the philosophical section of the
Scientific Council on Cybernetics of the Presidium of the Academy of Sciences
of the USSR (Moscow, October 1970), at the Commission on System Research of
the Scientific Council of the USSR (Moscow, December 1974) and at the VII
All-Union Symposium on Logic and Methodology of Science (Kiev, October 1976).

For us this thesis means that the analysis of everything that belongs to the
systems field should not start from the systems approach or the general
systems theory, but from the systems movement, and everything else – systems
analysis, systems engineering, systems orientations, and everything else -
should be considered as various elements, functional components and
organisations of the systems movement \cite{Shchedrovitsky1974b}.

The main feature and characteristic of the systems movement (making it a
"movement" rather than a "direction", an "approach" or the like) is that it
brings together representatives of different professions (engineers, military,
teachers, scientists, philosophers, mathematicians, organisers and managers),
who have different means and styles of thinking, different values and points
of view.

The motives for such association are not so much content-related as
socio-cultural (or even socio-organisational).

The representatives of the different professions, when they join the systems
movement, nevertheless remain oriented towards the standards and norms of
their profession, still strive to obtain such products which have been set as
models in their profession, and work in the professional means and methods
they are accustomed to. Moreover, representatives of each profession interpret
the meaning and content of the systems movement according to their
professional canons and strive to transform and organize the entire systems
field so that it corresponds to their familiar schemes, and even insist that
all other participants in the systems movement work only according to these
schemes. In other words, each profession within the systems movement seeks to
absorb and assimilate all the material of the systems movement and the systems
field into its specific forms of thinking and activity.

At this stage in the development of the systems movement, such a strategy is
natural and justified, because the structure and organisation of the systems
movement itself has not yet been established, and the products it must create
are not set or defined in any way. And so each profession has the right to put
forward its own professional ideal of organisation and its own idea of the
final product of all work as a model.

Accordingly, a very complex and internally contradictory range of ideas
emerges in the systems movement, on the one hand, and on the other a multitude
of different system orientations. They express ideas about the
cultural-historical products that the systemic movement can and must produce.
And herein lies the main source of conflict between the actors of the systemic
movement.

\paragraph{2.}
To highlight only the most visible and sufficiently formed, we can name eight
main proposals and respectively eight draft cultural products of the systemic
movement:
\begin{itemize}
\item[1] The development and improvement of already existing private sciences
  and fields of engineering and practice by introducing systems concepts,
  notions and methods of analysis into them \cite{Evenko1970, Kosygin1970,
    LargeSystems1971, Lyubishchev1971, Ackoff1972, Guishiani1972}.
\item[2] "General systems theory", similar to already existing natural science
  theories, such as physics, chemistry, biology, and so on \cite{Bogdanov1925,
    Sadovsky1972, Mesarovich1973, Uemov1973, Uemov1978, GTS1966}.
\item[3] "General systems theory", similar to traditional mathematics like
  geometry or algebra, or new ones like Shannon's information theory
  \cite{LargeSystems1971, Zadeh1970, Kalman1971, GTS1966}.
\item[4] "General systems theory" of the type of meta-mathematics in the sense
  of D. Hilbert and S. Kleene \cite{Trends1972, Sadovsky1974}.
\item[5] A practical methodology or methodology along the lines of disciplines
  such as operations research, decision analysis and the like \cite{Quaid1969,
    Optner1969, Evenko1970, Johnson1971}.
\item[6] Engineering methodology such as Hood and R. Makol's Systems
  Engineering \cite{Hood1962, Nikolaev1970, Simon1972}.
\item[7] The so-called 'systems philosophy' \cite{Laszlo1972}.
\item[8] System-structural methodology as a section or part of "general
  methodology". \cite{Shchedrovitsky1964a, Shchedrovitsky1965a,
    Shchedrovitsky1967g, Shchedrovitsky1969b, Spirkin1964, Dubrovsky1971,
    Gushchin1969, Kuzmin1976, Development1975}.
\end{itemize}
The first seven sentences have a historical prototype already implemented on
other material.

This is their strength. But at the same time it also, in our view, raises a
major objection. When each participant of the systems movement offers his
professional solution to systemic problems, he acts as an agent of an already
existing and functioning sphere of thought and activity – science,
engineering, mathematics, philosophy, and so on, within which he has been
formed as a "systems person", and as such he is always bound and limited by
the private cultural and historical situation in which he understood the
meaning and importance of systemic problems and tasks. Consequently, in the
end, he always only develops the professional organisation of his initial
thought activity at the expense of systemic means and methods. But it is well
known (and can even be considered generally accepted) that the systems
movement has emerged and is developing as an interdisciplinary and
interprofessional entity. This means that it must form and create an
organization that transcends the boundaries of every single scientific
discipline and every single profession. Consequently, the systemic movement,
in its formation and development, must take into account the entire
contemporary socio-cultural situation and proceed from an extremely broad
understanding of the possibilities and prospects of its development. Thus, we
are faced with the need to discuss the contemporary socio-cultural situation
as a whole.

\paragraph{3.}
In our view, there are at least eight points in the current socio-cultural
situation that have the most direct connection to the systemic movement.

The first of these is a process of increasing differentiation between the
sciences and the professions. Progressive in the eighteenth and nineteenth
centuries, it has now led to a mass of isolated scientific subjects, each
developing almost independently of the others. These subjects now not only
organise, but also limit the thinking of researchers. Techniques and ways of
thinking, new techniques and new methods created in one subject do not extend
to others. Each of the scientific subjects creates its own ontological
picture, which does not join the ontological pictures of other subjects. All
attempts to build a unified or at least coherent picture of our reality
encounter great difficulties.

The second point is the existence of highly specialised channels for
transmitting a compartmentalised subject culture. The modern mathematician has
little knowledge and understanding of physics, the less of biology or history.

A philologist is usually ignorant of mathematics and physics, but just as
ignorant of history and its methods. Already at school we are beginning to
divide children into those capable of mathematics and those capable of
literature. The idea of general education is increasingly being destroyed by
the idea of specialised schools.

The third point is the crisis of classical non-Marxist philosophy, caused by
the realization of the fact that this philosophy had lost its means of
controlling science and lost its role of coordinator in the development of
sciences, the role of mediator, transferring methods and means from one
sciences to another. This circumstance became clear already in the first
quarter of the nineteenth century and became a subject of special discussion.
Much attention was paid to it by Karl Marx and Friedrich Engels in their
works, who redefined the function of philosophy in relation to natural
sciences and humanities. The loss of direct connection with philosophy has
forced different sciences to develop their own forms of comprehension, their
own proprietary philosophy. This gave rise to various forms of positivism and,
more recently, to so-called 'scientism'.

The fourth point is the formation of engineering as a special activity
combining constructing with various forms of quasi-scientific analysis.
Traditional academic sciences, developed immanently in many respects, turned
out to be disconnected from the new directions of engineering, and this forced
engineers to create new type of knowledge systems, not corresponding to the
traditional patterns and standards. Information theory and cybernetics are
only the most striking examples of such systems.

At the same time, the problem of the relationship between construction and
research emerged and began to be intensively debated.

The fifth (very important) point is the continuing separation and isolation
within the activity sphere of various production technologies, which are
gaining an independent significance, becoming a new principle and objective
law in the organisation of all our activities and, ultimately, subordinating
the activities, nature and behaviour of people. The maintenance of these
technologies is becoming an essential necessity and almost the main purpose of
all social activity.

At the same time, technological forms of activity organisation are continually
being formalised and becoming more and more important, extending also to
thinking.

The sixth point is the emergence, formalisation and partial isolation of
designing as a special kind of activity. As a result, the question of the
connection and relationship between designing and research has become even
more acute. Designing directly and with all power sticked with the problem of
the relationship between the natural and the artificial in the objects of our
activity \cite{Shchedrovitsky1967g, Simon1972}. Neither of these problems has
been solved within the traditional sciences.

The seventh point is the increasing importance and role of organizational and
managerial activity in all our social life. Its effectiveness depends first
and foremost on scientific support. However, traditional sciences do not
provide the knowledge necessary for this activity; this is primarily due to
the complicated, synthetic, or, as they say, complex, nature of this activity
and the analytical, or "abstract", nature of traditional scientific
disciplines.

The eighth point (also particularly important) is the emergence and formation
of a new type of science, which roughly could be called "complex sciences".
This includes the sciences serving pedagogy, engineering, military science,
management, and so on. Now these complex practices are served by
unsystematised agglomerations of knowledge from different scientific
disciplines. But the very complexity and versatility of this practice, its
orientation to both normative, artificial and realisational, natural plans of
activity require a theoretical unification and theoretical systematization of
artificial and natural knowledge, which cannot be achieved.

All these points, which are characteristic of the contemporary socio-cultural
situation, give rise to a common "counter-intelligence". The differentiation
of the sciences gives rise to an attitude of unification and the creation of a
bridgehead that corresponds to this goal. The professionalization of education
gives rise to a direction towards common polytechnic and university education
and stimulates the development of the generalized and universal systems of
knowledge necessary for this. The crisis of traditional philosophical
consciousness and the loss of the old classical philosophy's governing role
for science gave rise to the idea of a restructuring of philosophy itself and
of all sciences in which philosophy could reconnect with the sciences and
regain its former leading role in the world of thought. Similarly, from the
opposition of the emerging situation, the demand for the establishment of
organic and effective links between engineering and science has been put
forward, followed by the demand for a complex organisation of natural,
technical, humanitarian and social sciences \cite{Ackoff1972, Volkov1973,
  Development1975}.

All these aspects of the contemporary socio-cultural situation are generally
well known, and we note them here only to point out the connection between
them and the systems movement. The fact is that the systems approach (whether
fixed or not) carried hope from the very beginning that it would solve all
these problems, integrate the disintegrated parts of science and technology,
develop a common language and homogeneous methods of thinking for all fields
and spheres of activity and, finally, in the limit, create a single reality
for modern science, technology and practice. In essence, these are the same
hopes that were pinned on physicalism in the 1930s and on cybernetics in the
1950s.

\paragraph{4.}
From our point of view, all these hopes for the present variants of the
systems approach are as unjustified as the previous hopes for physicalism and
cybernetics. But what matters to us here is not whether or not the current
variants of the systems approach justify the hopes placed in them, but the
other, one might say opposite, aspect of the problem: those requirements for
the systems approach that the current socio-cultural situation puts forward,
and it is these requirements that we want to base our reasoning on. If the
attitude towards the integration and synthesis of different activities is
fixed as a fact and if it is accepted as a value (at least for thought work),
then next we must wrap the task and discuss the structure of the product to be
obtained in the systems movement, if its goal is indeed to achieve such
synthesis. And only after resolving this question can we begin to analyse the
means of systemic thinking, its categories, basic concepts, methods and the
like. And in this way we can obtain data to answer the question: can the
systems movement create such a product?

It should be emphasised that this turning around of the task creates a very
different plan and style of analysis: it will not be about what is actually
being created now in the systems movement, but about the programmes and
projects put forward by different groups of professionals involved in the
systems movement, the validity of these programmes and projects and their
feasibility. It will be, on the one hand, a criticism of existing programmes
and, on the other hand, the nomination of new programmes which we consider to
be more promising.

\paragraph{5.}
The first, critical part of this work has already been done to some extent by
us and has been published in some of its parts \cite{Shchedrovitsky1964a,
  Shchedrovitsky1974b, Shchedrovitsky1976, Development1975}. Therefore, here
we shall dwell only on the second part of it: we shall try to outline the
essence of our own programme, which can be discussed within the systemic
movement along with all other programmes and projects. This is a programme to
develop a "system-structural methodology".

The main idea of our proposal is to combine the development of a systems
approach with the development of new methods and ways of thinking, which we
call "methodological" \cite{Shchedrovitsky1964a},
\cite[pp. 50-84]{Shchedrovitsky1969b}, \cite{Development1975}. Hence, we
proceed from the fact that, by their origin and specific character, the system
problems and tasks are considered not from an object standpoint, but from a
subject perspective. They arise in a situation where it is necessary to relate
and connect different subject views of one object with another
\cite{Shchedrovitsky1964a, Shchedrovitsky1964h, Shchedrovitsky1971i}.
Particularly, these problems and tasks generate, from our point of view, a
specifically systematic technique of thinking, in particular in research,
designing, planning and management, and this technique remains efficient and
effective only in the movement from a multitude of disparate unilateral
representations of an object to a unified and coherent representation. When
these conditions disappear and we obtain a homogeneous constructively
deployable representation of the object, then the systemic thinking technique
becomes unnecessary and the systemic problems and challenges disappear
\cite{Shchedrovitsky1974b}.

In other words, systemic problematics and systemic thinking, from our point of
view, exist only where there are several different subjects, and we must work
with these different subjects, moving as if over them and through them,
achieving a coherent description of the object in the diversity and
multiplicity of the subjects fixing it. In these cases, obviously, we can no
longer be inside these domains and act according to the laws immanent to them,
but have to "jump out" beyond their borders, working in some special way,
linking elements of different subjects together either for private practice or
for broad theoretical purposes.

But then, naturally, we come to the question of what are the organisational
scopes of research and project work, more generally, the organisational scopes
of thinking, that enable us to assimilate scientific subjects and describe an
object not through the prism of a single subject, but by taking into account
many subjects at once, the special features of each of them and at the same
time having a special perspective that differs from each subject and turns
these subjects themselves both into functional elements of our “thinking
machine” and into objects of our thinking and operational activity.

From our point of view, the specific organizational scope solving these
problems is the organizational scope of methodological thinking and
methodological work, which should not be identified neither with the
philosophical nor with the special scientific forms of organization of
thinking and activity. Therefore, further on, we have to elaborate the
specific characteristics of methodological work and the possible project of
organizing and constructing systemic-structural methodology.

\section*{II. General description of the methodological work}

\paragraph{1.}
Let's start with a few important, but so far purely verbal, characteristics of
methodological work as such.

In this context, it can be distinguished and contrasted with
concrete-scientific and philosophical work in six main ways:
\begin{itemize}
\item[1] Methodological work is not "pure research"; it also includes
  criticism and schematisation, programming and problematisation, construction
  and designing, ontological analysis and standardisation as deliberately
  delineated forms and stages of work. The essence of methodological work is
  not so much cognition as it is the creation of methods and projects; it does
  not only reflect but also, to a greater extent, creates, initiates anew,
  including through construction and project. And this defines the basic
  function of methodology: it serves the entire universe of human activity,
  above all through projects and prescriptions. But this also means that the
  main products of the methodological work – constructions, projects, norms,
  methodological prescriptions, etc. – cannot be and never are checked for
  truth. They are only tested for feasibility. The situation here is the same
  as in any kind of engineering or architectural design. When we project any
  city, it is meaningless to ask whether our project is true: after all, the
  latter corresponds not to the city that was, but to the city that will be;
  not the project, therefore, reflects the city, but the city will be the
  realisation of the project.
  
  This is a very important and fundamental point in understanding the nature
  of methodology: the products and results of methodological work in their
  bulk are not knowledge verifiable for truth, but projects, project schemes
  and prescriptions. This is an inevitable conclusion, as soon as we abandon a
  too narrow, purely cognitive attitude, accept Marx's thesis about a
  revolutionary-critical, transformative character of human activity, and
  begin to consider, along with cognitive activity, also engineering,
  practical, organizational, and managerial activities, which can never be
  reduced to the acquisition of knowledge. And it is natural that methodology,
  as a new form of organization of thought and activity, should encompass and
  cover all the types of thought activity mentioned.
  
\item[2] All these strong statements are not to say that research and
  knowledge are excluded from the field of methodology. On the contrary,
  methodology differs from methods precisely because it is knowledge-rich (in
  the precise sense of the word) to the extreme and involves clearly
  delineated, dedicated and, one might say, sophisticated research;
  methodological work and methodological thinking connect designing, criticism
  and standardisation with research and cognition. In doing so, research is
  subordinate to designing and standardisation, although it may be organised
  as an autonomous system; but in the end, research within methodology always
  serves designing and standardisation, it is guided by their specific
  objectives.
  
\item[3] Methodology does not only not reject the scientific approach, but on
  the contrary, it continues and extends it to areas where it was previously
  impossible.
  
  First and foremost, this manifests itself in the fact that methodology
  creates very complex compositions of different types of knowledge that are
  inaccessible to traditional science. In particular, it combines and connects
  natural scientific, constructive-technical, historical and
  practical-methodological knowledge in a new way. Traditional science avoided
  combining these four types of knowledge, and in this it was right, as its
  main task was to create a "pure image" of the natural object. Science (in
  the narrow and precise sense of the word) is oriented towards the
  separation of truly objective, "natural" knowledge from all other knowledge,
  in particular from that which determines what should or must be done to
  achieve a particular practical goal. Science assumes that the story of how
  to measure fields is a pre-scientific story. And while ancient Egyptian
  practical-methodical knowledge, which captures how to measure fields of
  various shapes, does fall into the section of the history of mathematics,
  the section itself and the corresponding stage of history are considered
  pre-scientific in contrast to ancient Greek mathematics, which all
  unanimously attribute already to science. Methodology supports this line of
  demarcation of different types of knowledge and their corresponding types
  of thinking. Moreover, for the first time it gives scientific
  (epistemological) grounds for such a division. But in parallel it creates
  more complex superstructures, linking knowledge of different types, and
  constantly uses such links.
  
  In addition, as already mentioned, methodology creates and uses knowledge
  about knowledge, it is as if it is always aware of itself, of its own
  structures, and this is necessary, because without such awareness of the
  form and structure of knowledge in general and the specificity of different
  types of knowledge in particular, the link and coordination of different
  types of knowledge that has just been mentioned cannot be realised.

\item[4] At the same time, methodology strives to connect and merge knowledge
  about activity and thinking with knowledge about the objects of this
  activity and thinking, or, to invert this relation, directly object
  knowledge with reflexive knowledge. Therefore, the object, with which
  methodology deals, resembles a matryoshka doll. In fact, it is a special
  type of alignment of two objects in which into the initial for methodology
  object – activity and thinking – is put another object – the object of this
  activity or this thinking. Therefore, methodology always deals with a
  duality of objects, not with the activity as such and not with the object of
  this activity as such, but with their "matryoshka" like connection. If we
  simply describe and fix activity in our knowledge, presenting it as an
  object of a special type, it would be a natural-scientific view of activity
  and the latter would appear as one of the objects of the natural-scientific
  type in the same line with such objects as the physical and biological ones.
  
  Methodological knowledge, by contrast, should consist of two knowledges –
  knowledge about the activity and knowledge about the object of this
  activity. If we break this bond and consider its constituent knowledge as
  autonomous, we would have to say that it is simply different knowledge about
  different things. But the essence of the methodological approach is
  precisely that we link and connect this knowledge. And just how the ways of
  connecting these different kinds of knowledge are defined and established
  the most important feature of methodology. After all, there is no
  "whole-part" relation between the activity and its object: the activity is
  not added to the object as a second, supplementary part and in the same way
  the object is not just a part of the activity; the object of activity is
  included in the activity many times – as its element, and as the content of
  other elements, for example, knowledge, and as material.
  
  In this way, methodological knowledge combines and unifies many different
  and heterogeneous knowledge; it is internally heterogeneous and
  heterarchicalised.
  
  But at the same time, it must be unified and coherent, despite its internal
  complexity and heterogeneity. In methodological work, we must have knowledge
  that integrates both our imagination about the activity and the object of
  the activity, and they must be connected so that we can use this connection
  in our practical work.
  
  Lets repeat that this way of connecting heterogeneous knowledge with the
  help of knowledge about activity and through this knowledge is the
  specificity of methodological knowledge. Thus, one can say that methodology
  defines the logic of reflection, i.e., the logic and rules of such a
  connection of distinct knowledge.

\item[5] For methodology it is characteristic to take into account the
  differences and multiplicities of different positions of the actor in
  relation to the object; hence working with different perceptions of the
  same object, including different professional perceptions: in this knowledge
  itself and the fact of its multiplicity are seen as an objective moment of
  the thought-activity situation.
  
  This is an extremely important point. Classical philosophy, like all science
  built on it, was based on the notion of one single true knowledge. If the
  same situation was described differently in different knowledge, the
  question was usually posed as to which knowledge was true. Methodology, by
  contrast, assumes that the same object can correspond to many different
  perceptions and knowledge and there is no point in testing them for truth in
  relation to one another, because they are simply different. This is the most
  important principle of modern methodological thinking, which is called the
  principle of multiplicity of perceptions and knowledge related to one
  object. But since the object itself is always taken subjectively, i.e.
  always in connection with its representations, the plurality of different
  representations turns out to be a fact of an active and communicative
  situation, which unites different professionals. Methodology starts its work
  with the professionals' perceptions of an object, and initially the object
  is defined only by this multitude of perceptions. Only then, starting from
  this whole set of representations, the methodologist can put the question
  about the reconstruction of the object as it exists "in fact" and make this
  reconstruction, supposing that all the available representations
  characterise the object from different sides, as if in its different
  projections \cite{Shchedrovitsky1964a, Shchedrovitsky1964h,
    Shchedrovitsky1971i}.
  
  Of course, this approach can be accused of a lack of autocriticism: after
  all, the ontological representation of the object created in this way will
  be such only for a strictly defined set of selected knowledge and
  professional activities, and if we choose another set of knowledge and
  professional positions, we will get another ontological representation. But
  these considerations do not prove the subjectivity of ontological
  representations at all, but only their historically transient nature. So
  anyone who speaks of an object as it "really is" must always remember that
  any ontological representation of an object is authentic only from a
  historically limited point of view. And since we can never escape this
  limitation, we must always consider the object in conjunction with a set of
  knowledge about it, and always relate and link together knowledge of
  different types – knowledge about the object and knowledge about
  knowledge. Therefore methodological thinking always makes use of schemes of
  many knowledges and fixes many different knowledges of one object in its
  images; this is called the reception of many knowledges
  \cite{Shchedrovitsky1964a, Shchedrovitsky1971i}.  To each of the images
  alternately the index of objectivity may be ascribed, that is, it is claimed
  that this particular knowledge corresponds to the object, and then all other
  knowledges are evaluated in relation to it and transformed so as to
  correspond to it. Then we can transfer the objectivity index to another
  knowledge or representation, and then all other knowledges are evaluated
  according to it. O.I. Genisaretsky called this method of work a "rafter's
  strategy": it's like running along logs, stepping on one and pushing those
  floating nearby, then jumping from this log to another, to a third,
  constantly changing the fulcrum and thus moving the entire raft forward.
  
\item[6] In methodology, it is above all not the schemes of the object of
  activity, but the schemes of the activity itself that connect and integrate
  different knowledge. To reconstruct the object on the basis of the different
  representations of the professionals we have no other way than to find out,
  what was the "active interest" of these professionals. Only after we have
  described the thought-activity that made professionals imagine the object in
  that way and not otherwise, and thus have determined the foci in the view of
  which they built their perceptions, only after that we can begin to collect
  and coorganise all these perceptions, but again not directly through the
  perception of the object, but foremost, through the perception of an
  activity, since really different perceptions are to be assembled and
  coorganised into a whole only when the activities with which they are
  connected enter into cooperation with one another, when they begin to
  process from different sites the object that has become one for all of
  them. This is the basic principle of methodological thinking: the concept of
  a complex cooperative activity serves as a means of linking together
  different conceptions of the object of this activity
  \cite{Shchedrovitsky1965a, Shchedrovitsky1967g},
  \cite[pp. 50-84]{Shchedrovitsky1969b}, \cite{Development1975}. And this
  binding goes not so much on the logic of the structure and life of the
  object in question, as on the logic of the use of diverse knowledge in
  collective cooperative activity.
  
  For this reason, in methodological work there is always not one ontological
  representation, but at least two: one of them depicts the structure of a
  professional cooperative activity – this is the so-called ontology of
  organizational activity, and the other depicts the object of this
  cooperative activity – this is the ontology of the natural object. The
  particular relation and connection of these two ontological representations
  constitute each time a specific feature of a particular methodological work
  (cf. \cite{Shchedrovitsky1979b}).
\end{itemize}

\paragraph{2.}
All the above-mentioned points can be summed up in one thesis: methodological
work is directed not on nature as such, but on thought-activity and its
organisational scope, and organisational scope of thought-activity has a
seemingly double existence: once as elements and components of thinking and
activity, and once as independent and autonomous formations (as a rule,
artificial-natural ones), multiplied in different forms and connected with
each other by thought-activity processes. The "natural objects" themselves are
seen in this case as special organisational scopes of thought activity,
created within philosophy and natural science domains together with others:
the natural science orientation towards the so-called natural object turns out
to be only one of many subdivisions in the organisation of our knowledge and
our thinking.

But this circumstance – the change from a natural reality to an activity-based
one in the transition to methodological forms of work – confronts us with a
new range of very complicated problems: In order to learn to work with complex
knowledge structures that combine methodological, constructive-technical,
natural-scientific, historical and philosophical knowledge, on the one hand,
and knowledge about objects and knowledge about knowledge and
thought-activity, on the other hand, a new logic of thinking must be
developed, which can be in summa called as logic of reflection; from this
perspective, modern methodology will be characterised as being based on the
logic of reflection.

It may be added that the logic of reflection itself presupposes special
knowledge about reflection \cite[pp. 131-143]{Elaboration1975}.

When we discuss this whole range of questions, we are moving into another,
special type of knowledge, which can be called methodologically reflexive.
Many of the statements made above were not deployed in the reality of
methodology, but in the reality of metamethodology: instead of carrying out
and demonstrating a thinking or activity procedure, we described either the
procedure itself or the transformation it carried out, its possible products
and results. This is what made appearing the difference between the reality of
methodology and the reality of methodological reflection (metamethodology).
This circumstance, too, has to be constantly kept in mind.

Many of the statements made above will have different meanings depending on
how we interpret them; as directly objectifiable or as belonging to the
specific reality of the meta-methodologist. To some extent, this distinction
can be accounted for and captured by the technique of double (or generally
multiple) knowledge. In particular, it is possible to set certain images of an
object and say that it is the object as it is "really"; in this way
objectification will be produced and we can then question how such an object
can and is actually described depending on certain research tasks, and we will
construct these descriptions, obtaining a second knowledge of the object. But
in the same way we can, having given a certain image of an object, say that
this is only our subjective representation of it, obtained in a certain
professional position, and then we will need to raise the question of what the
object is "really" like, and look for an image for it. And although in the
second case, by introducing a certain image of the object, we thus introduce a
representation of the object itself, its properties and characteristics, its
structure as an object will be problematized, while the structure and
character of knowledge and its reality will be dogmatized; in the first case,
by contrast, the structure of the object will be dogmatized, while the
structure of knowledge will be problematized. But this methodological
reflection is a necessary and organic part of methodological thinking as
research, construction, designing, criticism, and the like are.

After this summary characterisation of methodology, we can move on to our main
question: to characterise the systems approach from a methodological point of
view and outline a sketchy project for organising a systems-structural
methodology. 

\section*{III. Basic scheme of organisation of system-structural methodology}

\paragraph{1.}
So far, we have avoided questions about the specifics of the systems approach.
And this was not by chance, for we did not have a framework within which to
answer them. Now there is a framework and we can move on to discussing the
"systems approach" itself.

Our first claim in this regard (in line with all that has been said above) is
that the specificity of the systems approach can only be defined in describing
the structure and forms of organization of methodological work, for, in our
conviction, the systems approach exists only as a subdivision and special
organization of methodology and methodological approach. It arises in
conditions where we have to combine several different subjects – we have
already mentioned this – and move in accordance with the means and norms of
methodology. And if the very expression "systems approach" and the organizing
of thinking and activity corresponding to it appear also for representatives
of special sciences, it occurs, in our opinion, only due to the fact that they
borrow the means, methods and ontology of methodological work.

Consequently, only describing the structure of methodological work and
methodology we can approach the question of the specificity of the systems
approach. Prior to this, we could not even attempt to answer this question at
all. Moreover, since different systems of representations can be used in the
methodological position, the specificity of the systems approach, even if we
look for it in the reality of methodology and methodological work, will also
be defined differently, depending on which system of descriptions we choose.
If we choose a description in a theory of thought, we will define the
specifics of systems thinking. But it is possible to describe the systemic
approach also in the means of a theory of activity, and then its specificity
will be expressed and fixed differently. Thus, here, too, we must take into
account the plurality of possible representations.

Once this has been captured, the next step can be done and we can try to
collect and imagine in the scheme the special features or principles of the
methodological approach that were formulated earlier. In other words, it is
now time to draw a scheme of system-structural methodological work, taking
into account the principles formulated above.

\paragraph{2.}
In the preceding considerations, it has been established that methodological
work is directed towards activities – practical, engineering designing,
research, management and the like – and their organisational scopes; it should
ensure their construction, organisation and further development
(cf. \cite{Shchedrovitsky1969b, Development1975}). This work is of a
substantive character and is carried out on the material of individual
subjects – scientific, engineering, management and the like. Therefore, in the
scheme, the blocks of subjects growing over practices of various kinds are
covered by particular system-structural methodological developments (see
Scheme 1).

But it is natural that methodological work cannot be limited to this: after
all, particular systems-methodological developments, whether in physics,
biology, management theory or psychology, cannot provide a general concept of
system and cannot lead to general methods of systems work, equally applicable
in all domains. Consequently, more layers of methodological work are needed to
provide all particular-methodological developments with common concepts,
common ontological pictures and the logic of systems thinking. Thus we get
four layers of activities, each of which seems to build on and assimilate the
preceding one; these are:
\begin{itemize}[noitemsep]
\item layer of practice (including engineering and designing, organisation and
  management, construction, pedagogical and other developments);
\item layer of science, engineering, management, project and other domains; 
\item layer of particular methodological developments, and finally; 
\item layer of general methodology. 
\end{itemize}

Now we need to take the next step and answer the question of how we can
imagine the structure of a general system-structural methodology.

We have already emphasised above that the product of methodological
developments should be not only and not so much knowledge (especially
scientific) as methodological prescriptions, projects, programmes, norms and
the like, which will be used in the lower layers of thinking and activity – in
private-methodological developments, in domains of various kinds and in
practice.

Therefore, the first and main part of the general system-structural
methodology should not be research, but construction and designing. In
schematising this conclusion, we have depicted in the "body" of general
methodology over a set of particular methodological developments a layer of
general methodological system-structural construction and designing (in Scheme
1, the arrows running from this block depict the process of providing
particular methodological and domain-specific elaborations with common tools).

The relationship of the layer of methodological system-structural construction
and designing to the underlying layers of thinking and activity can be
explained using the example of work in a scientific domain (which has so far
been better analysed than other types of domain work).

It has been established in special logico-methodological studies (see, in
particular, \cite[pp. 106-190]{Problems1967}) that every scientific domain has
at least nine different epistemological units:
\begin{enumerate}[noitemsep]
\item Problems. 
\item Tasks.
\item "Experiential Facts”. 
\item "Experimental Facts”.
\item The body of general knowledge that constitutes this scientific domain.
\item Ontological schemata and pictures. 
\item Models. 
\item Means (languages, concepts, categories). 
\item Methods and methodics (see Scheme 2). 
\end{enumerate}
This is a set of basic building blocks of a scientific domain.

With this list at hand, we can now ask which of the above-mentioned
organisational scopes are formed and created directly in the scientific
domain, and which, on the contrary, are borrowed from methodology and shaped
under its determining influence. The historical-scientific analysis gives a
very definite answer here: at least four elements of any scientific subject –
ontological schemes and pictures, means and methods, and problems – have
always been developed either entirely outside the scientific domain (in
philosophy and in the incipient structures of natural science methodology), or
formally within science, but in fact in the systems of philosophical and
methodological thinking captured by it. 

Therefore, we have to double these four blocks and place them – in another
connection and in another co-organisation – also in methodology itself (first
of all in its constructive and designing parts) and show by arrows that the
main content of these blocks within scientific subjects is generated by their
counterparts in the system of methodology (see Scheme 2). And approximately
the same we find when studying the history of formation and development of
engineering, organizational and managerial and other subjects.

But for the blocks of constructing and designing projects represented in the
general scheme of the system-structural methodology (see Scheme 1) to work,
one must still have at least two groups of special knowledge: firstly, various
knowledge (constructive-technical, project-technical, natural-scientific etc.)
of the objects that are created by constructive-technological and
project-methodological thought-activity
\cite[pp. 211-227]{Shchedrovitsky1966a}, \cite{Dubrovsky1971},
\cite[pp. 393-408]{Development1975}; it is an obligatory requirement of any
productive work without prototypes: since the block of methodological
construction and designing supplies to scientific, engineering and managerial
subjects certain organisational scopes which further function according to the
laws of these subjects, it is necessary for designing to know the purpose and
functions of these organisational scopes, the requirements for their
morphology and the like \cite[pp. 50-84]{Shchedrovitsky1969b},
\cite{Dubrovsky1971}, \cite[pp. 299-302]{Development1975}; second, the methods
and conceptual tools of methodological construction and designing itself.
  
These two types of knowledge need to enter the 'body' of methodological
construction and designing and be used as tools there; but clearly they need
to be obtained somewhere before that.

We have already stressed above that methodological work cannot be reduced to
mere construction an designing. It connects construction and designing with
research. Therefore, besides the layer of methodological construction and
designing, there should be at least one more layer of methodological work in
the system of methodological work – the layer of research. By its structure,
methodological research is a special type of research, because its objects
are not physical, chemical, or biological phenomena, but scientific domains,
i.e. knowledge from various sciences together with the objects of this
knowledge and with the activity of producing and using knowledge; therefore we
should talk here about research, which differs from the natural science first
of all by the specificity of its object. But the specificity of the object of
study entails the specificity of the means and methods of research, and
therefore we can and should also talk here about the specificity of the
technology of methodological research.

In order to relate these statements to the discussions currently taking place
within the systems movement, we recall the theses made by J. Clear and
V.N. Sadovsky \cite{Trends1972, Sadovsky1974}: "general systems theory" (GST)
is not a theory, but a metatheory; this means that the question of what is the
object of the "general systems theory" thus understood must be and should be
answered: concepts, languages, methods, problems of other sciences.

Leaving aside the question of the appropriateness and correctness of the term
"metatheory" here and considering only the essence of the case, we can say
that the main point here is felt and expressed: although GST is not natural
science research, it is still research, and being research, it is very
different from traditional natural science research.

In our opinion, J. Clear and V. Sadovsky are referring to methodological
research; this research is entirely part of the system of methodological work
– and this defines its specificity, but it by no means exhausts methodological
work in general, nor even methodological analysis, for along with it in
methodology there are other forms of analysis, which we will talk about below.
And this form, called methodological research, is defined, firstly, by its
orientation towards scientific, engineering, organisational-managerial, and
other subjects, and, secondly, by its function of serving methodological,
constructive and designing work.

Given the reflexive origins of research work, we need to present it as a block
covering all that what is being researched (see Figure 1).

In addition, system-structural methodology must include at least one more
layer of work, the purpose of which is to realize and systematize its own
organization of methodological work in the systems domain: this block,
therefore, organizes system-structural methodology as a whole, linking and
uniting methodological system-structural construction and designing with all
the system-structural knowledge and methodological system-structural research
that serves it. Therefore, we can call it the "meta-methodology" layer, or,
more precisely, the systemic autoreflection of methodology. This layer of
work connects the system-structural methodology with broader, encompassing
systems – the philosophy of dialectical materialism and the whole culture of
humanity accumulated in the course of historical development. In essence, this
is the layer of proper methodological reflection and methodological thinking,
which covers all other components of methodological work and creates the
specificity of the methodological organization of thinking and activity. We
cannot characterize it so far through the specifics of language, concepts, and
procedures of methodological thinking, but we have already grasped and
expressed it in a certain way in the cooorganization and relations of the
objects of methodological reflection and methodological thinking, and the next
task will be to form the means and methods of methodological thinking as
commensurate with the organization of its object domain, or the space of its
objects.

Thus, the meaning of the whole scheme we have described can be summarised in
one statement: If we want to consider and characterize the structure and forms
of organization of the methodology of systemic-structural research, we should
proceed not from the scheme of the scientific subject and its main functional
units presented in Scheme 2, but from a quite different scheme of organization
of thought activity, namely the one presented in Scheme 1, and consider
methodology as a supersubject structure covering both subjects and practices
of different kinds and involving not only one single relation to them, but a
mass of different relations – not only research, but also constructive,
projectal, reflexive, organisational etc. relations.

By virtue of this, structural-systematic methodology turns out to be not just
a complicated structure and complicated system, but a heterogeneous and
heterarchicalised system that has both a level-hierarchicalised and a
'matryoshka' structure.

The basic "substance" (if only it is possible to put it that way) of this
system is formed by the methodological reflection, which covers practices of
different kinds and the subjects serving them or being independent of them –
say, geotechnics and geology, electrical engineering and the theory of
electricity, psychotechnics and psychology, etc.; In these practices and
subjects of different kinds, the systemic-structural methodological reflection
highlights systemic problems of different kinds, and then (according to
different thought relations) is shaped into different kinds and types of
systemic-structural thinking: programming, designing, constructing,
researching, organisational, and so on.

All these various types of methodological thinking are identified, shaped and
organized within reflection, out of its own substance and that of the
practices and objects it captures. In addition, all these organisational
scopes of methodological thinking are also co-organized with each other into
certain cooperative structures, which correspond to the lines of circulation
of their products in the space of methodology. Methodological programming
supplies all the other branches of methodology with programs of thinking and
practical work; methodological designing – projects of practices and subjects
of various kinds; methodological designing – systemic-structural ontologies,
tools of systemic-structural analysis, i.e., systemic graphics and concepts
describing the use of this graphics in mental work, main categories,
procedures and methods of systemic thinking, etc.; while methodological
research – knowledge about the systemic-structural aspects of practical and
subject-centered work.

For a proper understanding of all this organisation, it is very important to
keep in mind that systems-structural methodological research is not aimed at
systemic objects, but at systemic-structural thinking activity and describes
its processes, mechanisms and structure; therefore, in addition to "systems
scientists" working on various special subjects and materials of practice,
there should be "pure systems scientists" or "systems methodologists" who
carry out systems-structural methodological programming, designing and
research and in these activities create and investigate what we call
"structures in general" and "systems in general".

Generalising this point, which is already related to the difference in
positions and types of work within methodology, we can now say that within
systems-structural methodology there exist and should exist many different
types and ways of thinking and mental work and, accordingly, many different
positions and, one might even say, specialisations. These would be:
\begin{enumerate}[noitemsep]
\item Organisation of systemic practices of various kinds. 
\item Development of systemic problems within the special subjects of science,
  engineering, management and the like.
\item System-structural programming of research and development.
\item System-structural designing.
\item System-structural constructing.
\item Methodological systems-structural research describing systemic
  development within scientific, engineering and management subjects and
  practices of various kinds.
\item Methodological autoreflection of the whole field of systems-structural
  development in general.
\end{enumerate}

And if we want to establish order on our "workbench" of systemic-structural
methodology, we must take into account, on the one hand, the fundamental
difference between all these types and activities, and on the other hand,
their organic connection within the framework of systemic-structural
methodology. If any of these areas is eliminated, there will be no
systemic-structural methodology as a whole and, at the very end,
systemic-structural research in scientific, engineering and
organizational-managerial subjects and practice will be undermined and cease
to be deployed.

\section*{IV. Organisation of methodological work and the challenges of
  building a systems approach}

\paragraph{1.}
All that we have said above and presented in Scheme 1 is a certain project of
organization of methodological thinking and methodological work in the
systemic field. And this begs the question: what does all this have to do with
the systems approach, a systemic approach, which should give us specific
systemic categories, systemic methods of analysis and systemic representations
for different fields of practice and scientific research? And in this
question at the same time there will swing a doubt that all told so far has
direct and immediate relevance, that it sets and defines the specificity of
the systemic approach: after all these are some general schemes of the
organization of methodological work, and they seem to be not directly
connected with special features of systemic-structural representations which,
after all, obviously define and set a systemic approach itself; approximately
in this way here the main objection will be formulated.

From a traditional naturalistic perspective, it is perfectly legitimate. It is
so from a naturalistic point of view, based on the assumption that 'it already
knows' what a systemic approach is, but not from a methodological and
activity-based perspective, which are developing under the assumption that we
do not currently have an adequate and effective systemic-structural
perspective, that it still needs to be developed, obtained, and this, in
particular, is the task of the systemic movement.

But if these latter assertions are plausible, then we can only have two
strategies: 1) get down immediately to "business" and start constructing
systemic-structural notions without knowing how to do it and what the result
should be, or 2) design and create such an organisation, or "activity
machine", which in the process of its functioning would start processing the
current germs of systemic-structural notions into a coherent and consistent
system of systemic views and systemic developments. There is no third
strategy, although there is always a way (by the way, the most massive and
most widespread) of re-negotiating and reformulating already existing
perceptions created by others, but it does not give genuine contributions to
culture.

So, there are two possible strategies for productive work itself. The first
strategy cannot suit us for purely professional reasons (although we are well
aware that no work can do without it or its elements, including the most
refined methodological constructions). Therefore, without denying the
importance of the first strategy, we choose the second to organize our work.
Our task is to create a special "machine of thinking activity" that will
produce systemic-structural representations; and this, in our opinion, is the
essence of the methodological approach to the development of a
systemic-structural methodology.

For a naturalistic worldview, as already mentioned, such a move seems
unreasonable. Methodologists are constantly asked: do you have schemes or
plans of those systemic-structural representations which this "machine" should
create? After all, if you don't know these products, you can't construct the
"machine" either! In fact, the task here is as follows: Give us
systemic-structural representations and we will construct a "machine"
corresponding to them. We answer: if we already had systemic-structural ideas,
we would not need to build this "machine"; this is the point – we do not yet
have these ideas and, moreover, we do not even know what they should be, and
to somehow get out of this hopeless for a "naturalist" situation we build a
"methodological machine" that will produce the systemic-structural ideas we
need. That they will be systemic-structural representations is guaranteed by
the fact that the "machine" will be oriented to systemic problems and will
process the material of the systemic domain, and that they will be
methodological representations is guaranteed by the methodological structure
of the "machine" itself. The design of the "machine" and the character of the
material it processes should, therefore, guarantee us the necessary quality of
the resulting products.

Here, however, the following question arises (and the answers to it may be
very different): what exactly is the material of the systemic domain and how
should this "methodological machine" be oriented or directed? But in our
opinion, the answer is already given by the scheme we propose for the
organisation of methodological work. If someone thinks that methodological
thinking, like scientific thinking, is directed at natural objects, he will
naturally consider the systemic representation of natural objects as such
material; if someone thinks that methodological thinking is directed at
scientific subjects and knowledge, he will consider systemic knowledge and
problems as the main material of the systemic approach; and who considers
procedures, methods and techniques of research and project work as the subject
of methodological analysis, will naturally bring their systemic analysis to
the forefront. All these variants are equally acceptable to us in the
framework of the idea of methodological organization of systemic-structural
research and development: they all fit into the proposed scheme of
organization. And this seems to be the main thing.

An important advantage of this organisation of systemic-structural research
and development is that it does not reject any of the existing variants of
subject-centered and methodological work, accepts them all and shows the
place, role and necessity of each of them. But it also takes them in their
connections and relations to one another, in their involvement with the whole
and in their dependencies on the whole, and on this basis additionally deepens
and develops each of these types of work.

In addition – and this is very important for understanding the essence of the
case – this scheme establishes a special relationship between the structure
(or layout) of the "methodological machine" and the material it captures. The
character of the "machine" is determined at least by both; the material it
includes influences the nature and quality of its product as much as the
structure itself (or the order and sequence of processing the material by the
corresponding forms); and, moreover, the material itself, through the specific
layout of this "machine" (especially through the operation of the
autoreflection unit) is always pressing on the machine's layout, always being
processed into the machine's layout, into its forms.

And if we dwell on the question of why the proposed project of organization of
a systemic-structural methodology and all the ideas associated with it seem
usually strange and raise many objections, we should point first of all to
this solution to the question of the relationship between the design of the
"machine" and the material it captures: in our proposed project of
systemic-structural methodology, the construction of the "machine" is designed
not only to process the material it captures, but also to imitate and
reproduce the morphology of this material (in fact, this principle is a
further generalization of the principle of the content-awareness of logical
forms as basically for a content-aware genetic logic); specifically, this
relation is realized in the "methodology machine" through methodological
reflection and a unit of methodological research of systemic work in all kinds
and types of human activities.

\paragraph{2.}
Finally, there is another basis for the objections usually raised against our
proposed framework for organising systemic-structural methodology.

It has to do with a misunderstanding, in our view, of the processes of history
and the mechanisms of development of human activities. It is often asked how
it can be justified that the proposed system of methodological work will solve
the set of problems that currently exist in various fields of science and
practice, usually characterized as systemic-structural problems. However, the
essence of our point of view is precisely that the whole system of
methodological work described above is not created and organized in order to
solve today's problems referred to as "systemic-structural" ones (although in
the process it should solve or most often remove these problems, too); the
system of methodological work is created in order to develop the whole set of
thinking and human activity. The immediate reason for its creation are today's
problems, but if we limited our goals and tasks to them, it would be a largely
empty or, in any case, ineffective work. Therefore, the real aim of the
systemic-structural methodology should not be to eliminate and overcome one or
another group of particular problems, but to ensure a constant and continuous
systematic development of activities. At the same time, of course, emerging
problems should be continuously identified and fixed.

But it would be a mistake to think that tensions and discontinuities in
activities (or problems) unambiguously determine directions and ways of
resolving them, or, in other words, transitions to tasks. Not at all. In the
abstract possibility there is always an infinite number of solutions to each
problem, and in practical terms a sufficiently large number of substantially
different solutions. If we combine problems and look for one solution for each
of these combined groups, it is of course harder to find a practically
meaningful solution than for each individual problem, but still there can
always be several different such solutions. Thus, a tension, disruption or
problem in thought-activity does not yet determine unambiguously the task of
thought-activity; to a great extent, the task is determined by the means we
use, and the means are always the result of our "depravity", our individual
contribution to history, and they determine how and with what constructions a
particular set of difficulties, disruptions and problems in activity will be
overcome and resolved.

All of this applies fully to the systemic movement as well. One cannot ask
whether the proposed organisation of systemic-structural methodology will
produce the systemic-structural representations we need, for no one can say in
advance what kind of systemic-structural representations are needed. There is
a certain set of tensions, difficulties and problems in activity that we
consider to be systemic-structural.

But this is only a reason for creating a systemic approach and a
systemic-structural methodology, and when the latter is created, the
representations and means of analysis it produces will be systemic-structural
in the exact sense of the word.

Thus, the criticism proceeds from the assumption that the specificity of
systemic-structural representations and the systemic approach can be given
without reference to the means we use to create these representations, while
we, on the contrary, argue that it is unthinkable that the character of
systemic-structural representations and the systemic approach in general will
be determined primarily by the character of the means we use and therefore
suggest that appropriate systemic-structural representations are those that
will be produced by our "machine” of systemic-structural methodology.

This approach follows directly from the characterisation of the systemic
movement we have given above: there is an attitude to systemic developments,
but what is "system" and "systemic" is unknown; at any rate, representatives
of different groups in the systems movement understand it all differently.
These differences stem from differences in means and value attitudes.
Therefore, we must first catalogue and define these means and attitudes. For
our part, we propose the concept of methodological organisation of systemic
work. And for us, therefore, it is very natural to assume that genuine
systemic-structural representations will be those produced by this
organisation, just as it is natural for representatives of other groups to
assume that genuine systemic-structural representations will be produced by
the models they propose.

At the same time, we do not consider the path we have outlined to be the only
one; we only consider it to be the broadest and most effective in terms of the
idea of continuous development of thought activity. Every rupture in the
historical situation must be filled with a construction, but there is no
requirement that only a single construction is possible, and, as we now
understand it, there cannot be a single one in history. Figuratively speaking,
we can go in different directions from the ruptured situation, and where we
should most appropriately go is not determined by this situation, but by the
perspectives of the trajectories of our further movement.

Our program is to create a new formation of thinking, which we call
methodological, and new forms of organization of thought activity that will
produce, like "machines", new systemic-structural notions. And if we are asked
whether this thinking and these forms of organization of thought activities
will correspond to the old situations (from which we start), to the old
problems and to the ideas emerging in these situations, then we answer that,
of course, they will not: what is the sense in creating new thought formations
and new "machines of activity" in order to return in the end to old systems
and old problems?

\paragraph{3.}
Thus we once again arrived, but with different ideas, at the major and
decisive point in the contemporary debate. The development of a systemic
approach is not and cannot, in our view, be self-contained.

The systemic approach in the current socio-cultural situation can be created
and will be effective only if it is included in the more general and broader
task of creating and developing the tools of methodological thinking and
methodological work. And this way, as we have sought to show, corresponds to
the conditions of the emergence of the systemic approach and the traditions of
its development. The converse statement is also true. We believe that the
systemic approach is one of the most important points in modern methodological
thinking and contemporary methodological work, without it methodology today
can neither be formed, nor exist. Therefore, the most important socio-cultural
task at the present stage is to combine the systemic approach with the
methodological approach and its various variants, such as activity-based,
normative, typological approaches, and vice versa – to enrich and develop the
methodological approach and all its various variants by the specific means of
the systemic approach. And this two-way task can be solved, in our opinion,
with the help and within the framework of the methodological organisation of
systems of thought-activity described above. 

\begin{thebibliography}{xxx}
\bibitem{Ackoff1971} Robert Ackoff (1971). \foreignlanguage{russian}{О природе
  систем. // Изв. АН СССР. Техн. кибернетика. 1971. № 3.  }
\bibitem{Ackoff1972} Robert Ackoff
  (1972). \foreignlanguage{russian}{Планирование в больших экономических
  системах. — М., 1972.  }
\bibitem{Blauberg1969} Blauberg et al., 1969
  \foreignlanguage{russian}{Блауберг И. В., Садовский В. Н., Юдин
    Э. Г. Системный подход: предпосылки, проблемы, трудности. — М., 1969.}
\bibitem{Blauberg1973} Blauberg, Yudin, 1973
  \foreignlanguage{russian}{Баублерг И. В., Юдин Э. Г. Становление и сущность
    системного подхода. — М., 1973. }
\bibitem{Bogdanov1925} Bogdanov, 1925-1929 \foreignlanguage{russian}{Богданов
  А. А. Всеобщая организационная наука (тектология). В 3-х т. Восьмое
  издание. Москва-Берлин, 1925–1929.  }
\bibitem{Development1975} Development, 1975
\bibitem{Dubrovsky1971} Dubrovsky, Shchedrovitsky L., 1971
  \foreignlanguage{russian}{Дубровский В. Я., Щедровицкий Л. П. Проблемы
    системного инженерно-психологического проектирования. — М., 1971 а.}
\bibitem{Elaboration1975} Elaboration..., 1975
\bibitem{Evenko1970} Evenko, 1970 \foreignlanguage{russian}{Евенко
  Л. И. Системный анализ — инструмент обоснования управленческих решений. //
  США: экономика, политика, идеология. 1970. № 8.  }
\bibitem{GTS1966} General theory of systems, 1966
\bibitem{Guishiani1972} Guishiani, 1972
\bibitem{Gushchin1969} Gushchin et al, 1969 \foreignlanguage{russian}{Гущин
  Ю. Ф., Дубровский В. Я, ЩедровицкийЛ. П. К понятию «системное
  проектирование». // Большие информационно-управляющие системы. — М., 1969.}
\bibitem{Hood1962} Hood, Makol, 1962 
\bibitem{Johnson1971} Johnson et al, 1971 \foreignlanguage{russian}{Джонсон
  Ф. и другие. Системы и руководство (теория систем и руководство
  системами). — М., 1971.}
\bibitem{Kalman1971} Kalman and others, 1971 \foreignlanguage{russian}{Калман
  Р. и другие.  Математическая теория систем. — М., 1971.}
\bibitem{Kosygin1970} Kosygin, 1970 \foreignlanguage{russian}{Косыгин
  Ю. А. Методологические вопросы системных исследований в геологии. //
  Геотектоника. 1970. № 2.}
\bibitem{Kuzmin1976} Kuzmin, 1976 \foreignlanguage{russian}{Кузьмин
  В. П. Принцип системности в теории и методологии К. Маркса. — М., 1976.}
\bibitem{LargeSystems1971} Large Systems, 1971
\bibitem{Laszlo1972} Laszlo, 1972
\bibitem{Lyubishchev1971} Lyubishchev, 1971 \foreignlanguage{russian}{Любищев
  А. А. Значение и будущее систематики. // Природа. 1971. № 2.}
\bibitem{Mesarovich1973} Mesarovich and others, 1973
  \foreignlanguage{russian}{Месарович М. и другие. Теория иерархических
    многоуровневых систем. — М., 1973. }
\bibitem{Nikolaev1970} Nikolaev, 1970 \foreignlanguage{russian}{Николаев
  В. В. Состояние и некоторые проблемы развития системотехники. //
  Методологические проблемы системотехники. — Л., 1970.  }
\bibitem{Optner1969} Optner, 1969 \foreignlanguage{russian}{Оптнер
  С. Системный анализ для решения деловых и промышленных проблем. — М., 1969.
}
\bibitem{Otdel1974} Otdel, 1974
\bibitem{Problems1967} Problems of Research on the Structure of Science, 1967
  \foreignlanguage{russian}{Проблемы исследования структуры науки (материалы к
    симпозиуму). — Новосибирск, 1967.}
\bibitem{Quaid1969} Quaid, 1969 \foreignlanguage{russian}{Квейд Э. Анализ
  сложных систем (методология анализа при подготовке военных решений). — М.,
  1969.}
\bibitem{Sadovsky1972} Sadovsky, 1972 \foreignlanguage{russian}{Садовский
  В. Н. Некоторые принципиальные проблемы построения общей теории систем. //
  Системные исследования. Ежегодник 1971. — М., 1972.  }
\bibitem{Sadovsky1974} Sadovsky, 1974 \foreignlanguage{russian}{Садовский
  В. Н. Основания общей теории систем. — М., 1974.  }
\bibitem{Shchedrovitsky1964a} G.P. Shchedrovitsky 1964a
\bibitem{Shchedrovitsky1964h} G.P. Shchedrovitsky 1964h
\bibitem{Shchedrovitsky1965a} G.P. Shchedrovitsky 1965a
\bibitem{Shchedrovitsky1966a} G.P. Shchedrovitsky 1966a
\bibitem{Shchedrovitsky1967g} G.P. Shchedrovitsky 1967g
\bibitem{Shchedrovitsky1969b} G.P. Shchedrovitsky 1969b
\bibitem{Shchedrovitsky1971i} G.P. Shchedrovitsky 1971i
\bibitem{Shchedrovitsky1974b} G.P. Shchedrovitsky 1974b
\bibitem{Shchedrovitsky1976} G.P. Shchedrovitsky 1976
\bibitem{Shchedrovitsky1979b} G.P. Shchedrovitsky 1979b
\bibitem{Simon1972} Simon, 1972 \foreignlanguage{russian}{Саймон Г. Науки об
  искусственном. — М., 1972.}
\bibitem{Spirkin1964} Spirkin, Sazonov, 1964 \foreignlanguage{russian}{Спиркин
  А. Г., Сазонов Б. В. Обсуждение методологических проблем исследования систем
  и структур. // «Вопросы философии», 1964. № 1.}
\bibitem{Trends1972} Trends, 1972
\bibitem{Uemov1973} Uemov, 1973 \foreignlanguage{russian}{Уемов А. И. Методы
  построения и развития общей теории систем. // Системные
  исследования. Ежегодник 1973. — М., 1973.  }
\bibitem{Uemov1978} Uemov, 1978 \foreignlanguage{russian}{Уемов
  А. И. Системный подход и общая теория систем. — М., 1978.}
\bibitem{Volkov1973} Volkov 1973 \foreignlanguage{russian}{Волков
  Г. Научно-техническая революция: естествознание и обществоведение. //
  Правда. 1973, 25 февр.}
\bibitem{Yudin1972} Yudin, 1972 \foreignlanguage{russian}{Юдин
  Б. Г. Становление и характер системной ориентации. // Системные
  исследования. Ежегодник 1971. — М., 1972.}
\bibitem{Zadeh1970} Zadeh, Desauer, 1970 \foreignlanguage{russian}{Заде Л.,
  Дезоер Ч. Теория линейных систем. — М., 1970.  }
\end{thebibliography}

\end{document}
