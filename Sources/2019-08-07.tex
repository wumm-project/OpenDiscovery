\documentclass[11pt,a4paper]{article}
\usepackage{od}
\usepackage[utf8]{inputenc}

\title{On a Discussion about TRIZ and System Thinking}

\author{Text compiled and translated by Hans-Gert Gr\"abe, Leipzig}
\date{July 8, 2019}

\begin{document}
\maketitle

The discussion was compiled from a thread in the post
\url{https://www.facebook.com/valeri.souchkov/posts/10212251519998280} in the
Facebook account of Valeri Souchkov. 

\section*{The Original Discussion}

Valeri Souchkov Ruin Aleksei - а что ее искать? Истина уже высказана, в
презентации. Истина, представленная субъектом, не может быть объективна, по
определению. Поэтому спорить можно всегда. Ибо ничто в этом мире не
абсолютно. В таких спорах истина не ищется, в них каждый пытается защитить
свою точку зрения - свою истину. Которая, по определению, тоже субъективна. А
слушающие дискуссию пусть выбирают ту истину, от которой им польза есть. :)

Andrei Kuryan Nikolay Shpakovsky, есть ещё такое понятие, как решение
(solution).  Если ты решил с помощью молотка подпевать дверь на балконе, чтобы
она не закрывалась, то ты создал решение. Ты это имеешь ввиду?

Nikolay Shpakovsky Andrei Kuryan ну если мы решает задачу как подпереть дверь
с помощью молотка, то мы сначала выстраиваем в голове систему, включающую
дверь, молоток, пол там, стену...  Если выстроилась система, мы можем на ее
основе построить реальное устройство для подпирания двери.  Выстроилась
частично - решаем задачу по получению работоспособной системы

Andrei Kuryan Nikolay Shpakovsky, смотри, в твоей системе подпирания двери
присутствует молоток. Он выполняет некоторую функцию, удовлетворяющую твоим
требованиям здесь и сейчас. Решение включает ещё и операции по вставке молотка
в зазор между дверью и стеной, а потом доставали молотка оттуда.  Если ты
покажешь это решение, например, соседу, то он предложит тебе вместо молотка
использовать подходящий брусок из дерева или пластика.  Система подпора двери
стала более дешевой.  Системы подпора двери с молотком и брусков - это разные
системы? Это вообще системы?

Nikolay Shpakovsky Andrei Kuryan система не может быть дешевой или дорогой...
дешевым или дорогим может быть устройство, реализующее эту систему в реале не
понял про операции, но можно построить и систему, включающую операции, какие
проблемы? Она так и называется процессная

Andrei Kuryan Nikolay Shpakovsky, в системной инженерии система
рассматривается в рамках всего ее жизненного цикла. Согласно многоэкранке ТРИЗ
система - это не только ее версия КАК ЕСТЬ или КАК НАДО, но и КАК
БЫЛО. Другими словами, в ТРИЗ система подпора двери это и сборка системы и
эволюция системы: сначала с молотком, потом с бруском и т.д.

Nikolay Shpakovsky на здоровье...  в каждом этом случае можно построить разные
системы (технические там или искусственные)

Sergey Simakov 1.Мне кажется, что выше и писал, и Валерий пишет о
классификации, когда говорим о системе - говорим о языке. Который позволяет
сжать информацию о мире. Отбросить лишнее, которое не нужно для решения какой
то проблемы. К стати, не только с его помощью мы это делаем :) Это просто
построение моделей.  В этом смысле системы везде и всегда. Как свойство нашего
мышления. И скорее всего не только нашего. Как выделил объект -
система. Классификация. В этом смысле и собака системно соображает. :)
Т.е. это отправная точка для дальнейшего злостного определительства. Система -
это это некая модель, в которой выкинуты предположительно несущественные для
будущего действия части. Разбиение системы на подсистемы и т.п. - процесс
упрощения модели. Когда выше спорилось о линзе, где граница - это о построении
более практичной модели. Ну и естественно все во времени. Как свойства модели
во времени меняются.  2. И сразу возникает вопрос, нафиг делить системы на
естественные и искусственные или естественные и не естественные. К стати
вопрос. Чем отличаются эти два деления :) Что это дает для практического
применения.  Лично я вижу необходимость только в случае когда кто то может
изменить систему по своей воле, т.е. в качестве элемента надсистемы появляется
человек и это важно для ее описания.  Т.е. когда мы говорим про искусственную
систему подразумевается, что человек ее может существенно изменить. И не буду
в подробности вдаваться, т.к. человек -социальный - за ним прицепом социум
тащится.  Где то так. В моем разумении.

Nikolay Shpakovsky Я хочу объяснить уважаемому собранию мою возможную
назойливость по системам. Просто понятие система (техническая, полная,
функционирующая) очень активно и эффективно используется в нашем подходе при
работе с задачей.  Поэтому пришлось хорошо разобраться с такими вещами, что
такое объект, что такое система, где живет система, какой тип системы наиболее
эффективен для решения задачи... для себя, конечно.  Естественно эта
дискуссия, хотя и оффтоп (сорри, Валерий), была для меня полезной.

Sergey Simakov Nikolay Shpakovsky К стати, когда я к ЕГЭ по физике готовлю,
тоже очень активно использую понятия системы, функциональной модели. :) И
методы работы с ними. Естественно в контексте своем.  Большинству больше
инклюзивно, а некоторым сильно продвинутым или как раз с очень большими
проблемами эксклюзивно все объясняю.  Не было бы ЕГЭ - не было бы проблем - не
было бы обучения системному мышлению. :) Единственно плохо и хорошо, что
т.к. репетиторство, то для каждого клиента совершенно индивидуальная
траектория, которая здесь и сейчас. Очень тесно переплетающаяся с материалом
конкретным. В этом смысле руки не дошли до нормальной систематизации.  Я
пытался по умному делать, с планированием детальным обучения, но пошли
сплошные провалы. Поэтому как то есть некие направления, а детали -как
получится здесь и сейчас.  Но это мои проблемы. Мои особенности. Лентяй я...

Valeri Souchkov Nikolay Shpakovsky Николай, эта дискуссия как раз о системном
мышлении - поэтому совершенно в тему. Если мы не определимся с тем, как мы
понимаем систему, о каком-либо системном мышлении говорить бессымсленно.  3

Boris Stroganov Зачем же так! Системное мышление - рассмотрение чего-либо во
взаимодействии с окружающим пространством!

Nikolay Shpakovsky Valeri Souchkov ну давайте попробуем сравнить позиции...
1. система - мыслеобраз, живет в сознании 2. главный признак системы - это
цель ее создания 3. при решении ИЗ наиболее рассматривать систему, дающую
какой-то продукт 4. объект и система - это разные вещи 5. нельзя сказать вот
поехала система автомобиль далее по желанию

Alexey Schinnikov Nikolay Shpakovsky почему нельзя сказать "поехала система
автомобиль"?

Nikolay Shpakovsky можно конечно но это будет неправильно

Valeri Souchkov Так... сейчас начнется... :D :D :D

Valeri Souchkov Nikolay Shpakovsky Для меня система это совокупность
компонентов (элементов, подсистем), которая обладает свойствами и
функциональностью, которыми не обладает ни один из ее компонентов, взятый
отдельно. Границы которой обозначены условно вокруг тех элементов, которые
образуют жизненный цикл системы и реализуют цель системы в ее надсистеме
(соответственно, любая система это часть надсистемы, в которую она
входит). Автомобиль, жук, ножницы, молекулы, атом - это все системы. Все
зависит от границ. Объект вполне может быть системой. Разве нельзя сказать,
что молекула, ножницы или автомобиль это объект? Только вот опредление объекта
иное. Есть и нюансы. Если взять нож и отделить лезвие от ручки, в принципе, им
можно управлять и им же можно резать. И лезвие (если не переходить на
молекулярный уровень) не является системой. Но исчезнут важные свойства -
управляемость и безопасность.

Nikolay Shpakovsky Valeri Souchkov объект конечно может быть (стать) системой
и, скорее всего, не одной но тогда он будет называться система, а не объект
объект - это несистематизированная система, если можно так выразиться 
Valeri Souchkov

Valeri Souchkov Nikolay Shpakovsky Николай, я чуть расширил свой предыдущий
ответ.

Valeri Souchkov Nikolay Shpakovsky Все зависит от уровня делимости. Если мы
доходим до подсистемы, которая является элементом - то есть, неделимой на
другие элементы компонентом, мы вправе сказать - все, хватит делить. Этот
элемент уже не является системой относительно нашего контекста.  

Nikolay Shpakovsky Valeri Souchkov дада... я понял, можно оставить подсистемы
на уровне объектов, чтобы не загромождать анализ с чем не согласен -
Автомобиль, жук, ножницы, молекулы, атом - это все системы по моему -
Автомобиль, жук, ножницы, молекулы, атом - это все может быть представлено как
системы

Valeri Souchkov Nikolay Shpakovsky, да, ты прав. Именно представлены. Потому
что система это не какой-то физически классифицированный объект. Это границы,
которые мы условно очерчиваем вокруг некоего количество взаимодействующих
элементов.

Nikolay Shpakovsky Valeri Souchkov снимаю шляпу...

Valeri Souchkov Nikolay Shpakovsky Взаимно!

Andrei Kuryan Nikolay Shpakovsky, 1. "система - это мыслеобраз...".  1.1. Но
ведь то же самое я могу сказать о механизме, машине и т.д. В этом смысле
молоток - это мыслеобраз.  1.2. Если это утверждение нужно для того, чтобы
указать, что систем нет в реале, то не соглашусь. Правильно определенное
система состоит из реально существующих элементов и взаимодействий между
ними. На стадии проектирования мы оперируем описаниями системы, на основе
которых можем создать систему в реале.

Valeri Souchkov Andrei Kuryan Да, все верно, но это когда мы говорим об
искусственно создаваемых (designed) системах. Здесь все просто. Сложности
иногда возникают, когда пытаемся определять границы природных систем.  

Nikolay Shpakovsky Andrei Kuryan не буду спорить... если так удобнее -
пожалуйста.

Alexey Schinnikov Nikolay Shpakovsky я без задней мысли: почему неправильно
(хочу понять) говорить, что поехала система автомобиль.

Andrei Kuryan Valeri Souchkov, у природных систем границы определяются
областью действия тех законов, которыми мы описывает эти системы. Границы
солнечной системы - это действие закона всемирного тяготения на звезду,
планеты и прочие объекты. Цель описания солнечной системы как системы - это
проверка соответствия реального движения объектов описанию через закон. Если
соответствие есть, то мы достигли понимания солнечной системы.

Valeri Souchkov Andrei Kuryan А вот кузнечик, например? :)

Andrei Kuryan Valeri Souchkov, а про кузнечик пока ещё не всё известно :)

Valeri Souchkov Andrei Kuryan а вот я как раз представляю его как систему :)
Прыгает, стрекочет. Для чего обладает совокупностью компонентов. Составляет
звено в пищевой цепочке - надсистеме :) 2

Andrei Kuryan Valeri Souchkov, ну, если ты можешь по такому описанию
предсказать, когда и куда он будет прыгать и как стрекотать в любой момент
времени, то да, ты достиг понимания системы кузнечик и можешь использовать его
для получения какой-то пользы, не знаю, щекотать нос вместо будильника по
утрам :)

Valeri Souchkov Andrei Kuryan То есть, поскольку мы не знаем, о чем думает
кузнечик и не можем предсказать его поведение, мы не имеем права представить
его системой? Хотя вроде бы можем достаточно четко очертить ее границы и даже
разбить на подсистемы? Кстати, здесь мы сталкиваемся еще с одним существенным
признаком и свойством системы - ее поведением.

Sergey Simakov Возвращаясь к тому, что система - это о сжатии информации,
модели. Которую потом изменять будем. И в этом смысле собака мыслит
системно. Осознанно-не осознанно - второе дело. Понятие пользы - по мне - это
тоже дело совершенно второе.  Я к тому, что то, что мы называем объектом, это
замороженный процесс. В том смысле что для упрощения иззменения ВСЕЙ модели,
что то не меняется. Выделение объектов - это когда есть описание с
границами. Другой объект - другое описание. И между ними происходит
взаимодействие - еще одни правила и результаты.  Так просто мозги устроены. И
как я могу предположить, судя по тому как устроено зрение, которое на уровне
нейронов выделяет некие простейшие сущности, а потом усложняет, это совершенно
древнейший биологический механизм моделирования.  Ну просто у человек дорос до
мыслей о мыслях как мыслим :)

Nikolay Bogatyrev Мне не вполне понятно, почему уважаемые коллеги усомнились в
том, что кузнечик - система... Конечно же система. Это - организменный
структурно-функциональный уровень животного мира. У кузнечика есть уровни /
подсистемы: системы органов, органы, ткани, клетки, органеллы клеток,
молекулярный уровень (ДНК и РНК). Глубже - уже атомный уровень.  Помимо этого,
у кузнечика есть несколько слоёв надсистем: популяционный, видовой, гильдий,
экосистемный, биосферный. Биологи могут меня упрекнуть в том, что я не все
уровни перечислил (их можно делать и более дробными). Но в целом я, кажется,
ничего не упустил...

Andrei Kuryan Valeri Souchkov, в качестве системы можно рассматривать все что
угодно, даже кузнечика. Зачем это делать? Чем система кузнечик лучше
оригинального кузнечика? Мы вроде бы об этом, не?

Sergey Simakov Мы сейчас пытаемся системными средствами описать систему :)
Т.е. найти границы где уже не система.  Боюсь, что этим понятийным аппаратом
это внутреннее противоречие :) логически не объяснить.  Нужно во что то
поверить. Осознанно. :) Например то, что есть объективная реальность, которую
мы по крайней мере осознанно воспринимаем как систему. Всегда. И тогда о чем
базар - мы просто все системно видим ибо как только осознаем - то делаем
границы.

Sergey Simakov И тогда сравнение оригинального кузнечика и система кузнечик -
это системный процесс изменения модели "система кузнечик". А лучше или хуже -
это опять что то надсистемное. Но процедуру измерения придумать не могу.  Тут
только бы в схоластку не свалиться. Бессмысленную и беспощадную.  Иногда лучше
сказать - ВЕРУЮ :) Сделать редукцию ( как работаем с моделями) и работать.

Sergey Simakov Проблема в том, что, как мы кажется, вляпываемся в логические
петли и незнание чего то. Наше знание это в каком то смысле реализация,
выборка некого вероятностного состояния мозга. Да еще модель беднее
оригинала. Поэтому нужно силой воли заглушки веры ставить. И переходить к
практике. Может быть потом, по мере усложнения модели методом упрощения вера
куда то в более дальнее место перейдет.  И может более практичным будет
попытка понять, во что мы Верим. Где границы. Хотя...это опять до
бесконечности...

Alexey Schinnikov Судя по всему есть разногласие по поводу является ли Система
моделью реальной системы или Система это и есть реальная система, то есть
система - это или модель или реальность :) 1

Nikolay Bogatyrev Добавлю... Реальный кузнечик и система кузнечик, как правило
совпадают почти всегда и почти на 100\%... А вот биологические понятия /
конструкты "Вид", "Род", "Семейство", "Отряд", "Класс", "Тип", "Царство"
(расположены в порядке возрастания) очень часто ОЧЕНЬ могут сильно отличаться
- в реальности и в модели... С этой ситуацией ведут всю жизнь нестихаемый бой
систематики и таксономисты...

Alexey Schinnikov Nikolay Bogatyrev в книгах по диалектике я встречал простое
объяснение: есть система Вообще и система Конкретно. Система Вообще - это
идея, в программировании - класс, а система Конкретно - конкретный экземпляр
класса, материализованная идея, объект реальности

Alexey Schinnikov Система Вообще (класс) - кузнечики, а система Конкретно -
кузнечик, которого я сейчас наблюдаю в траве

Nikolay Bogatyrev Alexey Schinnikov, ну, да... То самое же - в фонетике: мы
знаем, как в русском языке звучит звук "А"... Но даже один и тот же человек не
сможет дважды повторить этот звук абсолютно идентично. (Приборы это легко
подтверждают, хотя мы и окружающие этого не замечают, точнее, наш мозг -
обобщает сигнал).

Alexey Schinnikov Nikolay Bogatyrev больше всего мне нравится подход
объектно-ориетированного программирования (класс-объект). Класс автомобиль
Волга описывает его свойства и функции, и с классом можно проделывать
манипуляции по Девятиэкранке, прогнозировать изменения класса в новый класс. А
объекты класса - это конкретные автомобили Волга, находящиеся в собственности
автолюбителей, с номерами. Удобно. Также по Девятиэкранке можно анализировать
конкретный объект - жизненный цикл автомобиля у дяди Васи, например

Alexey Schinnikov В своем системном подходе (Архитектура Событий) я так и
сделал: системы разделил на системы-классы и системы-экземпляры (объекты)

Alexey Schinnikov Потому я и спрашивал Nikolay Shpakovsky почему нельзя
сказать, что система автомобиль поехала. Да, неправильно так говорить, если
речь идёт и системе-классе, но, правильно так говорить, если речь идёт о
системе-экземпляре класса

Sergey Simakov Да. Системы без модели не бывает. А модель это отображение
одной физики в другую. В этом смысле для нас (физика - человеческое тело с
мозгом, шире социум -как коллективный разум) все система. В том числе
конкретный кузнечик и его обобщенная модель. Я снова предлагаю, насколько это
возможно, вернуться к хотя бы простейшим моделям работы мозга. Как модели
построителя.  Просто надсистемы разные. Конкретный кузнечик запускает один
набор моделей. А модель кузнечика - другой. При этом они во многом случайны и
как то фильтруются контекстом.  Например конкретный кузнечик на лугу в
контексте "красота" запускает модели солнечного восхода, а в контексте "на
булавке" и "красота" - совсем другие.  Модель кузнечика без кузнечика в натуре -
запускает воспоминания о красивой функциональной схеме :) А та о красивой
женщине. :) Собственно говоря запуск разных контекстов и порождает
различия. Но для этого нужна некая мета надсистема в которой все измеряется. И
осуществляется торможение.  Я к тому, что умничать и злостно определять можно
до бесконечности. Все связано со всем.  И опять получается что без Веры (и так
сойдет), которая прекращает процесс определительства и запускает переход к
деятельности - никак..

Nikolay Shpakovsky Sergey Simakov ментальная модель (система) кузнечика,
различные модели кузнечика и сам кузнечик - это разные вещи.  тоже самое и про
автомобиль путаница возникает, потому что система существует сама по себе, она
же присутствует в модели, и система присутствует в реальном ее воплощении.

Sergey Simakov Nikolay Shpakovsky И я примерно об этом. Если правильно понял.
Но все таки, я в практике настаиваю на том, что бы понять, как это все
создавалось. И зачем мы эти конкретные модели строим.  Ибо сначала создавались
набор моделей кузнечика в различных контекстах, объедененных потом в некую
метамодель. А потом реальное нечто про проведении измерений был
классифицирован как кузнечик который сам по себе, гораздо более сложный, но с
определенными параметрами которые для нас важны.

Nikolay Shpakovsky Sergey Simakov проблема понимания понятия система
запутывается еще тем, что на практике решения ИЗ мы используем его в двух
основных случаях 1. имеется некое устройство или явление, техническое,
биологическое, любое. Надо разобраться какова его структура и как оно
работает. С целью улучшения там или еще какой. Вот и строим системы,
подсистемы.  2. нужно создать новое устройство или технологию. В этом случае
сразу начинаем со строительства систем-подсистем с целью понять, какая будет
будущая машина. Затем представляем их в виде каких-то графических,
математических, физических и любых моделей. Изучаем их и - наконец - получаем
реальную новую машину.  Кузнец относится к первому случаю.

Sergey Simakov Nikolay Shpakovsky По моему здесь очень тонкий случай. Как
классифицировать. В любом случае строительство идет не на пустом
месте. Т.е. нужны начальные модели, которые мы будем менять. То ли, понимая их
изначальную недостаточность для преобразований, изучая "физику" с последующим
изменением, то ли считая, что и так хватит, строя на этой базе новые модели. А
там как получится.  В этом смысле изобретательские задачи - это об уточнении
знания. Новое устройство и технология - это новое знание, реализованное в
"физике".  В этом смысле ТРИЗ в широком смысле- это один из наборов наборов
инструкций общего плана по изучению окружающего мира, сфокусированный на
удовлетворении человеческих потребностей более техническими средствами.  Ну а
сравнивать эти наборы с другими и друг с другом, где границы - в социальном
смысле себе дороже :)

Nikolay Shpakovsky Sergey Simakov пока реализовать новое знание в реале, надо
сначала построить его в уме, то есть построить модели, прежде всего ментальную
модель - систему Конечно же с учетом очень многих вещей, кто спорит. Но сути
дела это не меняет. Система - основа всего

Alexander Veres Одно из главных свойств изобретателя - воображение. Именно в
нем и существуют системы :)

Sergey Simakov Nikolay Shpakovsky Так и я о том же. Сначала в голове
(головах). Снова. Немного расширить.  Я даже больше скажу. Эта модель наверное
больше чем на 90\% не вербализуема. Если попытаться все важное вербализовать -
чекнуться можно. Ранее вербализуемое из осознаваемого в автоматическое
перешло. Т.е. система как модель, с которой реально работает совокупность
мозгов на порядок-порядки больше описываемой. Ибо в систему, которая в голове,
как один из важнейших фактов может попасть ругачки с дочкой. Которая ни в
каком списке стейхолдеров не значится и значится не может:) Оффтоп - именно
поэтому 11 лет в школе учатся, что бы хоть 10\% автоматизировать. :) Ибо если
попытаться построить хотя бы граф знаний - шизануться можно. :) И именно
поэтому я к ИИ для изобретательских целей аккуратно отношусь.  К стати, как
метод борьбы со сложностью - различные программы численного
моделирования. Когда уже и не важно - почему. Это для того, что бы головы
разгрузить. И научиться с ними активно "сотрудничать" - тема.  Т.е. очень
существенное, повторяясь, осознаваемая модель системы -очень малая толика от
реально применяемой.  Попытки построить более осознанную модель имеют и
негативные последствия. Собственно говоря об этом и был разговор про
линзу. Когда при осознании модели много что интересного оказалось. Которое не
нужно было для конкретных целей построения модели, как поясниловке к основным
понятиям для новоинтересантов.

Nikolay Shpakovsky при проклятом царизме, когда в школах учили логику,
говорили так: "револьвер системы Наган .модели тульского завода" кто-то может
прокомментировать в рамках нашей дискуссии?

Alexander Veres Nikolay Shpakovsky дикие люди, не знали что такое система,
пока ГСА все не разрулил.

Valeri Souchkov Nikolay Shpakovsky Конечно. В данном случае речь идет о более
абстрактной модели системы, обобщающией конкретные экземпляры системы. Вот
например, абстрактная система "слон". Слон может быть африканский, индийский -
морфология и анатомия будут в принципе одинаковы, отличия будут уже в
конкретной реализации. То есть, имеются некие общие свойства и
признаки. Например, у одного слона овальные уши, у другого круглые. Но у всех
слонов обязательно будет хобот. А вот бегемот в систему "слон" не впишется -
нет существенного компонента - хобота.

Nikolay Shpakovsky то есть система одна - модели разные, правильно?  

Nikolay Bogatyrev Valeri Souchkov, можно добавить - и каждый индивидуальный
африканский слон, и кузнечик одного вида, и наган Тульского завода будут иметь
индивидуальные различия (по типу индивидуальных отпечатков пальцев и формы
ушной раковины у человека).

Nikolay Bogatyrev Nikolay Shpakovsky, "то есть система одна - модели разные,
правильно?" - Здесь просто ИЕРАРХИЧЕСКАЯ СТРУКТУРА: 0) автомашина, 1) легковая
машина, 2) машина "Форд", 3) модель "Форд-Фиеста", 4) "Форд-Фиеста" немецкого
/ американского / российского производства 5) индивидуальный экземпляр
автомашины...  То самое же - со слонами, наганами, кузнечиками и людьми...

Sergey Simakov А можно и по другому. Даже хуже. Я бы сказал - имя одно
системы, а для каждого контекста - разные модели. Тема для нас известная,
когда в практике интуитивно отрабатывается. Когда для металлоломщика и гонщика
:)

Sergey Simakov В варианте с иерархией как раз модели более менее
одинаковые. Просто происходит сужение допустимого диапазона изменения
параметров сверху вниз. Ну или в более сложном - разные наборы,
взаимсвязанные.

Nikolay Bogatyrev Sergey Simakov, со сменой контекста будет меняться и
систематика / таксономия / иерархия...

Sergey Simakov Nikolay Bogatyrev Вот вот!!! А если учесть, что систематика и
прочее - это очень часто выбор в каком то смысле случайный из вариантов. Ибо
толи так толи сяк...:) Я опять о гибкой логике, превращающейся в аристотелеву
:) Которая при детализации разваливается....  И ведь как то в этом бардаке
работаем. И даже большею Куча народа даже не осознает, а решения эффективные
принимает :)

Nikolay Bogatyrev Sergey Simakov, да-а-а... Я тоже иногда думаю, как это всё
существует, работает и как нам ещё иногда удаётся понять друг друга
(правда....... не всегда... :) ). А изящные решения зачастую находят именно
при смене контекста: на том построена в известной мере ТРИЗ и анекдоты... :) 2

Boris Stroganov Nikolay Shpakovsky Тот, кто создал естественные системы,знает
зачем он это создал,

Alexey Schinnikov Nikolay Shpakovsky в философии диалектики есть деление на
системы-вообще (классы) и системы-конкретно (объекты, экземпляры класса). Этот
же подход используется в объектно-ориентированном программировании, и хорошо
бы использовать в ТРИЗ, чтобы исключить путаницу абстрактного и конкретного 1

Andrei Kuryan Nikolay Shpakovsky, класс систем - система - модель - версия -
партия - изделие№ и т.п. - это классификация и одновременно идентификация,
принятая в традиционной технике. Существуют международные стандарты,
устанавливающие порядок такой классификации/идентификации.  Но это не совсем
то, что предлагает ТРИЗ.  В системном операторе отдельная ось - ось эволюции
системы, как раз предназначена для того, чтобы рассматривать систему с точки
зрения того, как она появилась и изменялась. Другими словами, как изменялись
ее состав, структура и функциональность.  Поэтому под системой мы понимаем не
просто какое-то представление реального объекта, но и всю совокупность
изменений в нем.

Nikolay Bogatyrev Andrei Kuryan, в биологии эволюционный взгляд на то, что в
ТРИЗе называют системным оператором, весьма широко-применяемый метод в
палеонтологии (в сочетании с реверсивным "инжинирингом"... :) ).  

Andrei Kuryan Nikolay Bogatyrev, да, но есть смутное подозрение, что револьвер
системы Нагана не относится к биологии :))

Sergey Simakov Andrei Kuryan Ну почему же...В контексте взаимодействия с
биотелом - Самое прямое. :) Еще можно много биологических контекстов
посмотреть. :)

Nikolay Bogatyrev Andrei Kuryan, увы, имеет... :( Самое прямое... Функция
нагана - прекращать биологическое функционирование крупного наземного
животного, например, человека.  Кстати, биологическая и техническая эволюции
имеют сходные стадии, только идут они в разных направлениях... :)

Sergey Simakov Вот вот. Наган -как элемент в эволюции огнестрельного оружия -
эволюция оружия - эволюция ТС - эволюция человеческого социума - эволюция
биологических систем. Ну а дальше можно вниз до бактерий и ДНК искать. :) А
внизу (или верху), как наиболее стабильное и основополагающее - физические
модели мира.  Пойдешь в подсистемы - попадешь в надсистемы. И наоборот. Просто
когда вниз идут - появление других надсистем идет чаще на бессознательном
уровне. Иначе крыша поедет.  Я так, поиграться. И как комментарий к
вышестоящим разговорам. Намек на то, что это безобразие единообразно не
опишешь. И все время приходится держать баланс между точностью модели и
временем на ее создание. Прикапывание к точности выполнения неточных правил
имеет смысл если это влияет на результат существенно. Ибо в процессе все равно
будут задействованы куча обратных связей. И самое мерзкое, как повлияет,
известно станет, да и то далеко не всегда, когда все уже закончится. :)
Т.е. непонятно и как баланс держать осознанно :)

Andrei Kuryan Nikolay Bogatyrev, все же функция нагана - метать
пули. Применение нагана в решениях по прекращению биологического
функционирования крупных наземных животных, включая человека, - это далеко не
единственное применение. ИМХО, угроза воспользоваться наганом в спорах сделала
для эволюции социума гораздо больше, чем его применение по прямому
назначению. И ещё им можно орехи колоть))

Nikolay Shpakovsky Andrei Kuryan метать пули - назначение нагана, предписанное
разработчиками его и изготовителями. Это еще очень неудачно называют ГПФ.
Вообще объекта нет никакой функции (кроме назначенной ему субъективно), она
появляется в зависимости от места объекта в системе.  Допустим, мы пытаемся
выключить свет, не дотягиваемся рукой и используем для удлинения руки
незаряженный наган - то функция в системе его будет такой - часть трансмиссии
и рабочий орган для воздействия на выключатель (человек, видевший наган
впервые, скажет - о, это штуковина для выключения света).

Andrei Kuryan Nikolay Shpakovsky, это очень интересный вопрос - является ли
нагана частью системы включения света.  С точки зрения многоэкранки - нет,
потому что нагана не замышлялся в качестве элемента этой системы.  Тогда что
это?  В ИТ это принято называть решением, когда результат достигается сборкой
готовых компонентов, изначально для такого результата не предназначено. В
жизни это называется life-hack (не знаю аналога в русском).  Вот не знаю,
можно ли наган рассматривать в качестве элемента включения света. Или система
- это нечто более структурно устойчивое.

Sergey Simakov Что изменится в решении практической задачи если мы это не
назовем системой?

Valeri Souchkov Sergey Simakov Изменится то, что будет сложнее увидеть
ресурсы, которые можно использовать для поиска решения с более высокой
степенью идеальности. Выделяя систему, подсистемы и надсистему, производить
поиск ресурсов можно более полноценно и - что важно - системно.

Sergey Simakov Когда я слышу ГПФ, моя рука невольно тянется к пистолету :).
Если серьезно, выделение ГПФ позволяет сфокусироваться на главном здесь и
сейчас, переведя возникшие по ходу рения вторичные задачи на потом. Это все
хорошо раньше было, во времена появления ТРИЗ, когда системы простые были. И
достаточно было изменить один-два параметра, что бы функционал стал лучше.
Сейчас вытянешь хвост - нос увяз. Поэтому конечно полезно попытаться найти
ГПФ, но зацикливаться на этом опасно. Набор функций с границами.

Sergey Simakov Valeri Souchkov Так и я об этом. :) Т.е. всегда система. Что бы
не рассматривали. Тем паче, согласно гипотезе, все это прошито
биологически. :)

Valeri Souchkov Sergey Simakov ГПФ хрроша, когда система простая, как
молоток. Сегодня развиваются сложные многофункциональные системы. Взять тот же
кухонный комбайн. У которого несколько рабочих органов, и в зависимости от
того, что нам нужно в данный момент, ГПФ меняется. Конечно, можно сказать, что
ГПФ комбайна "обрабатывать продукт", что, в общем-то, корректно, но слишом
абстрактно. А ГПФ смартфона? Запускать приложения? Разоваривать? Снимать
видео? Все зависит от контекста применения системы в конкретный момент.

Sergey Simakov Valeri Souchkov Да. Проблема в том, что стремление ВСЕГДА найти
ГПФ приводит к тому, как Вы и пишите, нахождению его на высших уровнях
абстракции. Слишком высоко в надсистеме. Слишком далеко от физики конкретной
системы.  Формальный выход - расширить систему так, что она начинает включать
в себя кучу черт знает чего. Ну да. Монстрообразно. Но не решабельно.

Sergey Simakov Да простят меня ТРИЗовцы, я точно также к выявлению рабочего
органа, двигателя и прочее отношусь. Ни разу в жизни на практике не
пользовался. Есть функциональная модель. Достаточно.

Valeri Souchkov Sergey Simakov Да, когда в качестве ГПФ мы определяем уж
слишком абстрактную функцию, она теряет привязку к конкретным компонентам и
процессам. Это уже цель, а не функция. Поэтому в многофункциональных системах
я не определяю ГПФ вообще, а функции, направленнные на те элементы надсистемы,
для осуществления которых создавалась система.

Valeri Souchkov Sergey Simakov В функциональной модели они все равно
выявляются, пусть и в неназванном виде. Потом уже можно выявлять,
анализировать.... если есть такая цель. А она иногда возникает в задачах
оптимизации.

Nikolay Shpakovsky Вот и я про то, что не совсем понятно куда девать эту
систему пр решении задачи, и зачем она нужна... кроме ресурсов, пожалуй.  Всё
сильно меняется, если решении задачи гпф идет лесом. Надо понимать функцию,
которую выполняет объект в системе в данной конкретной ситуации. Системы ведь
разные применяются при решении...  Например, какой ГПФ у вредной системы?

Andrei Kuryan Sergey Simakov, ГПФ позволяет понять, для чего изначально была
придумана система.

Nikolay Shpakovsky Andrei Kuryan за-чем?

Andrei Kuryan Nikolay Shpakovsky , чтобы понять, зачем изначально была
придумана система.  Когда ты наганом тыкаешь в выключатель, то ГПФ нагана -
"метать пули" должно намекнуть тебе, что ты используешь наган не по прямому
назначению. Следовательно, это повод вместо нагана использовать в системе
что-то попроще, например, деревянную палочку.

Nikolay Shpakovsky Andrei Kuryan понятно для поиска ресурсов... узковато, но
тоже надо... FOS всё нормально проблема в том, наверное, что мы говорим от
разных подходов в нашем подходе понятие система (техническая, функционирующая,
полная, вредная и т.п) играет мало не ключевую роль.

Nikolay Bogatyrev Конечно, наган имеет множество над-систем: наградное оружие,
"палочка" для выключения света, товар для получения прибыли в магазине,
средство угрозы, грузило для ловли рыбы, и т.д. - Но это - известно для любого
объекта, процесса, явления...

Nikolay Bogatyrev Ну, и ещё раз повторюсь. Для простейших искусственных систем
можно найти / понять / обнаружить / предписать её цель, ГПФ... А вот для
естественных, сложных и /или биологических систем найти / понять / доказать
функцию, цель, назначение часто бывает весьма непросто...

Nikolay Shpakovsky Nikolay Bogatyrev и не нужно.... нужно понять ее функцию в
обстоятельствах решаемой задачи

Nikolay Bogatyrev Nikolay Shpakovsky, ну, да... Примерно так и делают многие
биологические системы и те, кто их описывает... Правда, здесь можно
ошибиться... :) (Но ошибиться можно каждому, всегда, везде и во всём... :) ).

Andrei Kuryan Nikolay Bogatyrev, цели для искусственной системы определяются
надсистемой. Функция системы определяется её структурой, которая, в свою
очередь, определяется ограничениями природы. На уровне системы происходит
согласование требований и функций.  Пока для естественной системы не
определена надсистема, мы не можем говорить о ее цели и требованиях к ней. Но
ничто не мешает определять её функции. Вот только за-чем? :))

Sergey Simakov Тут такое дело...  1. Мозги требуют завершенности модели. Когда
она не завершена и это чувствуется - начинается расколбас. В состав модели
входит ответ на вопрос - "а нафига" с точки зрения высших сил. Так сказать
смысл. И обязательно - почему так делаю. Но это в меньшей степени.  Вот и
причина поиска смысла. Как "сердце требует".  2. Другая причина - появление
цели позволяет резко упростить предсказание. Собственно говоря ИКР - об этом.
Если мы говорим что целью является сохранение вида, то то, что этому не
соответствует -отбраковываем. Но в жизни все сложнее...:)

Александр Кармазанов Andrei Kuryan цель для искусственной системы определяется
создателем системы. А надсистема в роли источника ограничений,и то если
надсистема распознана верно. А для естественной системы, или какой то
сторонней, наблюдатель определяет цель чужой системы.

Nikolay Bogatyrev Andrei Kuryan, "Но ничто не мешает определять её
функции. Вот только за-чем? :))" - Ну как же?! - Чтобы, "объяснив" этот
непонятный, невнятный, изменчивый, равнодушный, враждебный мир, попытаться
найти способ избежать ударов, потерь, разочарований, боли, а если повезёт, то
и поиграть и испытать упоительное чувство, что ты чем-то владеешь, имеешь и
управляешь!... Ведь человек - властелин природы, человек - это звучит гордо,
ну и далее в том же духе... :)

Andrei Kuryan Nikolay Bogatyrev, если в качестве надсистемы для естественной
системы мы определяем наблюдателя и общество, а цели - возможность предсказать
будущие состояния этой системы, то остаётся всего один маленький шаг до того,
чтобы извлекать из этой сиистемы пользу. Собственно, знание о будущих
состояниях естественной системы уже наносит пользу обществу хотя бы тем, что
снижает риски.  Тогда граница между искусственной и естественной системами
становится исчезающе тонкой. Более того, естественные системы можно
рассматривать как подмножество искусственной, которое пока не приспособлен
обществом для своих нужд :))

Andrei Kuryan Александр Кармазанов, у системы много создателей
(заинтересованных сторон), которые как раз и находятся в надсистеме. Более
того, у системы ровно столько надсистем, сколько у неё создателей.

Александр Кармазанов Andrei Kuryan то есть если конструкторское бюро создало
чертежи, проектное бюро проект, завод произвел ...и заказ чик использовал
систему, теперь надсистем очень много у созданной системы... По-моему вы
оторвались от земли с этим тезисом.

Andrei Kuryan Александр Кармазанов, вы перечислили далеко не все этапы
жизненного цикла системы. Но в целом верно: на разных этапах и стадиях ЖЦ у
системы разные надсистемы.

Александр Кармазанов Andrei Kuryan на разных этапах жц и сама система разная,
она меняется создателями..так что надо фиксировать систему и создателя.  Как в
программировании, любой алгорит содержит минимум одну ошибку, кто то потом
поправил, новая система, новый создатель ...

Andrei Kuryan Александр Кармазанов , в ТРИЗ мы рассматриваем разные этапы и
стадии системы как одну систему. Более того, даже эволюцию системы мы
рассматриваем как одну систему.

Александр Кармазанов Andrei Kuryan ну если в триз вы рассматриваете все этапы
как одну систему, значит и создатель один, обобщенный.

Sergey Simakov Александр Кармазанов Как раз об этом выше и плакались. Что если
попытка создать всеобъемлющую да еще осознанную модель требует бесконечного
времени и ресурсов.  Поэтому столько и путей к Богу. Ибо у каждого свой
контекст. Кто то без ИКР кушать не может, а кто то в гробу его видал. Кто то
алгоритмы по 10 листов фигачит с книгами комментариев и портянки рисует
необозримые им самим, а кто в 4 строчки :) Как удобно, так и делайте. Только
если вдруг не идет - ищите причины и меняйте подход.

Sergey Simakov К сожалению, выше указанный подход имеет изнанку. Слабою
договороспособность ТРИЗовцев, плохую взаимозаменяемость.  Если в инженерии в
принципе один строитель другого заменит, то в проектах с ТРИЗ - все хуже
будет. Клиенты взвоют. :) Вот и причины еще одного плача...  Так что такие
обсуждения как то позволяют синхронизировать понимание границ у разнородных
товарищей. Не приверженцев одного стиля ТРИЗ кунг - фу. Ибо одно дело, когда
есть список терминов, а другое -живое общение.

Александр Кармазанов Sergey Simakov да вот не говорите организовали Ма ТРИЗ,
другие общества ... Разработок общих чтоб коллективы авторов делали нет...
Каждый перетасовывает , добавляет по своему ..  Ученные уже не могут в
современном мире делать открытия в одиночку. Это только они выбирают кого то
кто будет "лицом". Где коллективная разработка и развитие ТРИЗ?

Andrei Kuryan Александр Кармазанов "ну если в триз вы рассматриваете все этапы
как одну систему, значит и создатель один, обобщенный."  Из чего следует этот
странный вывод? Я же выше писал, что заинтересованных сторон у системы много,
как и надсистем.

Александр Кармазанов Andrei Kuryan логически рассуждайте. Если система каждый
раз меняется, меняются создатели. Разные надсистемы. А потом мы рассматриваем
одну обобщенную систему в своем развитии...-> следовательно один обобщенный
создатель , и одна обобщенная надсистема должна быть, а то получается слева у
нас за деревьями лес, а с права за деревьями, другие деревья.

Andrei Kuryan Александр Кармазанов, жизненный цикл и эволюция системы - это
разные оси измерения системы. Я пока не пойму, что не так?

Nikolay Bogatyrev Andrei Kuryan, "Собственно, знание о будущих состояниях
естественной системы уже наносит пользу обществу хотя бы тем, что снижает
риски." - Ну, да, с/х, метеорология, медицина, баллистика, химия - все эти
области имеют более или менее сильный прогностические аппараты, чем и
привлекательны... :)

"Тогда граница между искусственной и естественной системами становится
исчезающе тонкой. Более того, естественные системы можно рассматривать как
подмножество искусственной, которое пока не приспособлен обществом для своих
нужд :))" - Да, в области биологии, медицины, генной инженерии, бионики это
особенно заметно.

Sergey Simakov Александр Кармазанов По поводу ТРИЗ Сушков выше очень хорошо
написал.

Александр Кармазанов Andrei Kuryan не так то, что весь триз построен на
обобщении или индукции. И началось с того, что Альтшуллер решил выявить общие
методы, и законы...  Нельзя где то применить частное, а где то общее..Должен
быть переход..  От частного к общему, и наоборот..у вас получается...в разных
местах используются утверждения без перехода..

Александр Кармазанов вот так (picture added)

Ruin Aleksei Александр, выше по треду было четко сказано, кажется, Валерием,
фигурально выражаясь: как хочу, так кручу, это не ТРИЗ, пошутил, а "ты вообще
в армии служил?", что такой грубый? Ваши вопросы и наблюдения, к сожалению, не
имеют смысла/пользы/толка. Как впрочем и вся эта дискуссия ;)

Александр Кармазанов Ruin Aleksei ну и ваше тоже..но все мы люди и поэтому так
все...  1

Andrei Kuryan Александр Кармазанов, жизненный цикл системы является аналогом
онтогенеза, эволюция - аналогом филогенеза. 3-ая ось системного оператора
описывает иерархические связи между надсистемой, системой и подсистемами, 4-ая
ось - системы и ее анти-системы.  Современный системный оператор представляет
собой 4-мерное пространство, в котором мы рассматриваем систему. (в отличие от
3-х мерного в классическом ТРИЗ). К слову, он хорошо синхронизируется с
подходами в традиционной системной инженерии.  Н. Хоменко в ОТСМ-ТРИЗ
предложил обобщенный N-мерный системный оператор и его 7-мерную
реализацию. Хотя его подход не получил широкого распространения.

Ruin Aleksei Андрей, вы серьезно? "Не получил широкого распространения" это
пять. Я даже знаю почему.

Александр Кармазанов Andrei Kuryan здорово все это...Только начали мы с того,
что есть обобщенная система, но почему, то у этой обобщенной системы по вашей
новой оси отсчета, появилось много надсистем, и много создателей, что
противоречит, тому что был сделан переход к обобщенной "системе", без
обобщения всего остального. Сожалею, что мне не удалось вам показать и донести
эту ошибку.

Andrei Kuryan Александр Кармазанов, смотрите, 1) если мы рассматриваем систему
на разных стадиях ЖЦ, выделяя для каждой стадии некоторое состояние системы,
то для каждого такого состояния мы можем определить надсистему. Это будут
разные надсистемы.  2) анализ использования такой операции к системам
продемонстрировал, что эти надсистемы отличаются друг от друга разными
стейкхолдерами.  3) Реверсивный анализ такой операции показал, что мы можем
выделять в рамках традиционных этапов ЖЦ системы некоторые промежуточные
стадии. Например, смартфон на этапе использования может использоваться как
фотоаппарат, карта или телефон. Такие состояния мы можем рассматривать как
мелкие стадии ЖЦ в рамках общего этапа. И для каждой стадии мы можем выделить
разных стейкхолдеров: фотограф, турист, абонент. (Физически это может быть
один человек, но с разными ролями.)  4) В результате было сформулировано
следующее правило: появление (еще одного) стейкхолдера является признаком
наличия у системы (еще одной) надсистемы.  Где вы видите здесь противоречие?

Александр Кармазанов Andrei Kuryan Противоречие в том, что новый стейкхолдер
это частное явление в какой-то промежуток времени, при какой-то новой частной
системе, или новой точки отсчета. Когда будет сделано обобщение все равно
останется один стейкхолдер как общее понятие. Иначе теория не сформулирована
получается. Возможно это какое-то начало новой теории, но не той старой.

Valeri Souchkov Ruin Aleksei Алексей, не нужно передергивать. Я просил
говорить по существу. Но снова лишь личные нападки и нулевая ценность. Если ты
знаешь априори и везде громогласно провозглашаешь, что все эти дискуссии
бесполезны, что ты в них участвуешь? Всего доброго.  1

Andrei Kuryan Александр Кармазанов, я не понимаю, о каком обобщении системы вы
говорите.

Ruin Aleksei Да крышняк у меня едет. Жду, когда до-... 7мерная реализация
Н-мерной модели. И что с ней делать потом? В какой другой инструмент
"засунуть"? Это же синдром самозванца. Наверное не заболевание, но однозначное
расстройство психики. Все эти попытки спрятаться за наукоподобность это,
скорее всего, этот самый синдром!Это не про кого-то конкретно, это общее
впечатление от треда.

Александр Кармазанов Andrei Kuryan вы рассматриваете разные системы в разный
момент времени, так как она меняется...Говорите, что есть разные стекхолдеры,
и разные надсистемы....Потом вы делает вывод, что есть эта обобщенная система,
у которой если появился новый стейкхолдер, и значит есть новая надситема. НО с
какой стати? Система была, каждый раз разная. Если мы в едином жизненном цикле
рассматриваем как обощенную систему, то почему надсистема не обобщена,
стейкхолдер тоже? И да таких понятий в ТРИЗ не было..соответственно это даже
не ТРИЗ.  НУ есть геометрия Лобачевского и другие никто не спорит..Но это не
ТРИЗ.

Ruin Aleksei И не Нескафе! ;)

Andrei Kuryan Александр Кармазанов, я же привел в качестве примера смартфон. В
разные моменты времени смартфон (система) все тот же, но его использование
разное (разные состояния смартфона). В рамках ЖЦ мы рассматриваем конкретную,
а не обобщенную систему. Собственно, понятие системы и ее ЖЦ (сотсояний
системы) в ТРИЗ заимствовано из системной инженерии; мы ничего не придумывали
и используем эти понятия в ТРИЗ в их оригинальной трактовке.

Александр Кармазанов Andrei Kuryan Ну и что, если у вас система =
"много-рук-много-ног", то и у вас один стейкхолдер = "много-рук-много-ног",
такая же и над система...Это вот мы обобщили...

Andrei Kuryan Александр Кармазанов, для целей решения ИЗ стейкхолдеров лучше
различать, а не обобщать.

Sergey Simakov Александр Кармазанов Я просто предлагаю, что бы каждый свою
логику применил к решению конкретной проблемы. Если у каждого получится, ну и
слава Богу. Нельзя рассматривать это все без решения конкретных задач.
Повторяюсь, речь идет о создании с помощью неких правил которые почему то ТРИЗ
называются, который у каждого свой, более точных моделей с ограничениям по
срокам.  Если работает - ну и ладно.  Различать - не различать...Это все
контекстом определяется. Задачей. Опытом. Знанием. Временем. И т.д.  Проблемы
решаются. И ладно.

Andrei Kuryan Sergey Simakov, мы занимаемся не только применением личного
опыта для решения задач, но и тиражирование такого опыта.

Александр Кармазанов Sergey Simakov Да уж печально это..что нет совместных
коллективов, которые бы разрабатывали теорию...Если система взаимовложенная,
или какая-то многомерная еще...ее называют старым понятием, делают какие-то
выводы, включая новые понятия...в итоге все гуру, у каждого свой Дзен...

Sergey Simakov Александр Кармазанов Времена, когда бренд ТРИЗ был в одних
крепких руках прошли. И не вернутся. Так что что есть то есть.

Valeri Souchkov Александр Кармазанов Это совершенно естественно. Теория в ТРИЗ
вообще почти не развивается. Развиваются инструменты, в основном, уже
существующие. Развивают их сами ТРИЗ-практики - компании или фрилансеры,
которые получают доход от своих клиентов. Размер таких бизнесов не позволяет
заниматься теоретическими исследованиями. Кроме того, они заинтересованы в
создании своей интеллектуальной собственности. В бюджетные организации,
специализирующиеся на исследованиях, например, университеты, ТРИЗ пока что
проник слабо, и то, в основном, на уровне преподавания. О причинах я писал
выше. Есть ряд ассоциаций, объединяющих тризовцев со схожими взглядами
(скорее, идеологическими, чем теоретическими), но и там уровень каких-либо
исследований пока еще очень и очень низкий. Это происходит по одной
причине. Несмотря на столь долгое время существования, ТРИЗ пока не достиг
критической массы, которая бы позволила осуществить качественный прыжок,
перейти на новый уровень развития, новую S-кривую.

Sergey Simakov Andrei Kuryan Если другой не воспринимает этот опыт, то
насильно мил не будешь. Пусть решает задачи по своему. Есть результат ну и
ладно. Будет свой опыт тиражировать.

Александр Кармазанов Valeri Souchkov Это знаете как с свободным программным
обеспечением...Если бы люди по всему миру не помогли открытую свободную
операционную систему разработать, такую как Linux...То ее бы не было ..Да были
бы какие-то отдельные программки свободные, но толку от них было бы мало...
Валерий подскажите есть ли сейчас наработки или изыскания на тему того, есть
ли какая-то зависимость формы представления продукта, формы представления там,
например рабочего органа, и методов решения возникшего конфликта требований?

Valeri Souchkov Александр Кармазанов Я отслеживал разработки Торвальдса и
всего сообщества с самого начала, работал и в Red Hat, и в Debian, и в
Ubuntu. Но там у сообщества была конкретная цель - создать альтернативу очень
дорогой в то время Unix. Я сам c 1995 начинал в Solaris на Sun Sparcstation,
ценник там на все был адский. А потребность была массовой и очень большой.
Есть ли массовая потребность в ТРИЗ? нет. Есть ли большая потребность?  Нет.
ТРИЗ до сих пор остается очень нишевым продуктом. Поэтому ожидать массового
open source невозможно.  Что вы понимаете под формой представления - скажем,
рабочего органа? Геометрической? Насколько я знаю, работ по зависимости такого
уровня сегодня нет, кроме старой работы И. Викентьева "Геометрический
пространственный оператор".

Sergey Simakov Александр Кармазанов Вообще то ТРИЗ начиналась как раздел
психологии. И с ней получилось как с психологией. Ибо предмет такой же
мутный. В психологии тоже куча школ и направлений. Только сейчас в нее
начинают проникать по настоящему научные методы и она начинает приближаться к
естественным наукам. НО плата за это - офигенный ценник
исследований. Исчезновение термина психология. Ибо это уже какой то синтез
всего корпуса знаний.  Разработка ПО типа ОС - на порядки более простая
задача, чем разработка научно обоснованной, надежной, слабо зависящей от
субьективных факторов методики изменения систем. Если на создание Линукс
столько сил, времени и средств вложено, то что о ТРИЗ говорить...

Александр Кармазанов Sergey Simakov наверно вы хотели сказать, что для решения
реальной проблемы, мозгу необходимо распознать явления, использовать понятия,
для абстрагирования, чтоб привести к задаче, удобной для мозга, что б можно
было по заранее известному алгоритму мозгу получить эффективное решение. При
этом сам алгоритм для структур мозга должен быть исполнимым. Только опять тут
проблема алгоритмы в триз имеющиеся, не сводятся к точному решению. А могут
алгоритмы расходиться, то есть приводят к разным решениям.

Sergey Simakov Александр Кармазанов Можно и так. Только слово "алгоритм" мне
не нравится. Это слово в обычном понимании связано с механистическим
пониманием систем. Формальной логикой.  ТРИЗовские алгоритмы - это ближе к
художественному. Даже не мягкой логике. Не зря мы столько здесь воду в ступе с
системами толчем и договориться не можем до конца:).  И есть у меня очень
большое подозрение, что любые алгоритмы не сводятся к точному решению. Ибо по
определению наши знания неточны. И точно по алгоритму работать не можем. И
т.д.  Единственные решения - это совершенно дичайшее исключение...

Александр Кармазанов Sergey Simakov Более того дать точно определение самому
понятию "Алгоритм" нельзя. Но точность решения зависит от исполнителя - мозга,
если ошибки исполнителя в сумме дают не существенное расхождение, то можно
сказать, что алгоритм сходиться. И если бы в ТРИЗ алгоритмы были полностью
художественные, то не было бы никаких стандартов, приемов, и АРИЗ-85. И
соответственно на основе решений задач по этим алгоритмам не было бы
изобретений воплощенных в жизнь. Остается вопрос, а они точно есть?
воплощенные в жизнь, на основе этих решений изобретения?

Sergey Simakov Александр Кармазанов Вообще то, есть такие искусствоведческие
науки. И когда обучают художествам, там жесткая муштра и правила :) Как кисть
держать, композиция, правила, правила, правила...  Только одни после этой
муштры точно скопировать пейзаж могут. В стиле какого то художника. Не
цепляет.  А другие что то там намажут небрежно - и у народа в животе
бабочки...

Hans-Gert Gräbe Alexey Schinnikov "больше всего мне нравится подход
объектно-ориетированного программирования (класс-объект)".

The subtitle of Szyperski's book "Component Software"
<https://dl.acm.org/citation.cfm?id=515228> is "beyond object oriented
programming". The reason is simple since he observed that in IT service
structures the "customer" is responsible for the state (the "data") and the
supplier for functionality (the "behavior"). Hence there is a fundamental (!)
contradiction from the business point of view and OOP is the wrong (!)
solution. It is mere a compromise (once more). Time to apply TRIZ to solve
\emph{that} contradiction in a more sound way?

Alexey Schinnikov Hans-Gert Gräbe к сожалению я не владею английским и другими
иностранными языками...

Valeri Souchkov Alexey Schinnikov Google translate! :)

Alexey Schinnikov Valeri Souchkov гугл тот ещё переводчик )

Hans-Gert Gräbe Alexey Schinnikov "В своем системном подходе (Архитектура
Событий) я так и сделал: системы разделил на системы-классы и
системы-экземпляры (объекты)"

Since you are a very expert in OOP I suppose you know about the three ways
objects come into live: constructors, factory objects and factory methods.
Only the first is bound to the notion of class in your sense. For the second
and third only the definition of \emph{interfaces} is required, i.e., you need
only a \emph{description} of the function. In modern architectures (APIs)
\emph{this} is the main way objects come to live. Factory objects are used in
the most cases even to monitor object live cycles - just another concept
(evolution) that was used elsewhere in the discussion in a way completely
unrelated to the questions discussed here. Time to compile a better "system
theory" (using also the "well forgotten" old ideas)?

Valeri Souchkov Alexey Schinnikov Он постоянно совершенствуется. Кайзен :) И
уже работает весьма прилично. Проблема в авторах. Они иногда так пишут, что и
на родном языке, как говорится, без 100 грамм морковного сока не
разобраться. Правда, в каждой стране свой сок... :D 

Hans-Gert Gräbe Alexey Schinnikov Поскольку у меня нет русской клавиатуры, я
пишу свои тексты с помощью Google Translate, а затем позволяю им переводить
меня на кириллицу и исправлять текст, созданный Goo gle, с помощью
<https://www.russtast.de/>. Я сохранил последний шаг здесь. (сохраняет =
сохранено = не выполняется на немецком языке) So much about Google
Translate. Of course it is much easier to write in English (or German - for me
even better).

\section*{English Translation}

Valeri Souchkov Ruin Aleksei - and what to look for her? Truth has already
been expressed, in presentations. The truth presented by the subject cannot be
objective, according to definition. Therefore, you can always argue. For
nothing in this world absolutely. In such disputes, truth is not sought; in
them everyone is trying to defend their point of view - their truth. Which, by
definition, is also subjective. And let those who listen to the discussion
choose that truth from which they benefit. :)

Andrei Kuryan Nikolay Shpakovsky, there is still such a thing as a solution
(solution). If you decide with a hammer to sing along the door to the balcony,
it didn’t close, then you created a solution. Do you mean that?

Nikolay Shpakovsky Andrei Kuryan Well, if we solve the problem of how to
support the door with a hammer, then we first build in our head a system that
includes door, knocker, floor there, wall ... If the system is lined up, we
can on it basis to build a real device to support the door. Lined up partially
- we solve the problem of obtaining a workable system

Andrei Kuryan Nikolay Shpakovsky, look, in your door locking system a hammer
is present. It performs some function that satisfies yours.  requirements here
and now. The solution also includes hammer insertion operations into the gap
between the door and the wall, and then they got a hammer out of there. If you
are if you show this solution, for example, to a neighbor, he will offer you
instead of a hammer use a suitable block of wood or plastic. Door support
system became cheaper. Door and hammer support systems are different system?
Is this a system at all?

Nikolay Shpakovsky Andrei Kuryan system cannot be cheap or expensive ...
cheap or expensive may be a device that implements this system in real life is
not I understood about operations, but you can build a system that includes
operations, which Problems? It’s called process

Andrei Kuryan Nikolay Shpakovsky, System Engineering System considered
throughout its entire life cycle. According to TRIZ multiscreen a system is
not only its version AS IS or AS NEEDED, but AS IT WAS. In other words, in
TRIZ, the door support system is both an assembly of the system and system
evolution: first with a hammer, then with a bar, etc.

Nikolay Shpakovsky on health ... in each case, you can build different systems
(technical or artificial there)

Sergey Simakov 1. It seems to me that I wrote above, and Valery writes about
classifications, when we speak of a system, we speak of a language. Which
allows squeeze world information. Discard the excess that is not needed to
decide which then problems. By the way, not only with his help we do it :)
It's just building models. In this sense, systems are everywhere and
always. As a property of our thinking. And most likely not only ours. How to
select an object - system. Classification. In this sense, the dog also thinks
systematically. :) Those. it is the starting point for further malicious
determination. System - this is a certain model in which the allegedly
insignificant for future action of the part. Partitioning a system into
subsystems, etc. - process simplification of the model. When arguing above
about the lens, where the border is about building more practical model. Well,
of course, everything is in time. How model properties change over
time. 2. And immediately the question arises, what for divide systems into
natural and artificial or natural and not natural. By the way question. What
is the difference between these two divisions :) What does this give for
practical application. Personally, I see the need only when someone can change
the system of your own free will, i.e. appears as an element of the
supersystem man and this is important for its description. Those. when we talk
about artificial the system implies that a person can significantly change
it. And I will not go into details, as social man - followed by a society
trailer drags along. Somewhere like that. In my mind.

Nikolay Shpakovsky I want to explain to the distinguished assembly my possible
system importunity. Just the concept of a system (technical, complete,
functioning) is very actively and effectively used in our approach when work
with the task. Therefore, I had to deal well with such things that such an
object, what is a system, where does the system live, what type of system is
the most effective for solving a problem ... for yourself, of
course. Naturally this the discussion, although offtopic (sorry, Valery), was
useful to me.

Sergey Simakov Nikolay Shpakovsky By the way, when I prepare for the exam in
physics, I also very actively use the concepts of a system, a functional
model. :) And methods of working with them. Naturally in its context. Most
more inclusive, and some very advanced or just very large I explain everything
exclusively by problems. There would be no exam - there would be no problems -
not would be learning systems thinking. :) The only bad and good thing is that
because tutoring, then for each client is completely individual trajectory,
which is here and now. Very interwoven with material specific. In this sense,
the hands did not reach normal systematization. I tried to do cleverly, with
detailed training planning, but let's go continuous failures. Therefore, as it
is, there are certain directions, and details as get here and now. But these
are my problems. My features. Idler ...

Valeri Souchkov Nikolay Shpakovsky Nikolay, this discussion is just about the
system thinking - therefore completely in topic. If we do not decide how we we
understand the system, talking about any kind of systemic thinking is
meaningless. 3

Boris Stroganov Why so! Systemic thinking - looking at something in
interaction with the environment!

Nikolay Shpakovsky Valeri Souchkov well, let's try to compare the positions
...  1. the system is a thought-image, lives in consciousness 2. the main sign
of the system is the purpose of its creation 3. when deciding from the most to
consider the system that gives some product 4. an object and a system are two
different things 5. one cannot say this went the car system further optional

Alexey Schinnikov Nikolay Shpakovsky why not say "the system went car"?

Nikolay Shpakovsky is possible of course, but it will be wrong

Valeri Souchkov So ... now begins ...: D: D: D

Valeri Souchkov Nikolay Shpakovsky For me, a system is an aggregate components
(elements, subsystems), which has the properties and functionality that none
of its components has taken separately. The boundaries of which are marked
arbitrarily around those elements that form the life cycle of the system and
realize the goal of the system in its supersystem (accordingly, any system is
part of the supersystem into which it included). A car, a bug, scissors,
molecules, an atom - these are all systems. Everything Depends on the
boundaries. An object may well be a system. Can't you say What molecule,
scissors or car is this an object? Only here is the definition of the object
otherwise. There are also nuances. If you take a knife and separate the blade
from the handle, in principle, they can be controlled and it can also be
cut. And the blade (if you don’t go to molecular level) is not a system. But
important properties disappear - manageability and safety.

Nikolay Shpakovsky Valeri Souchkov object of course can be (become) a system
and, most likely, not one but then it will be called a system, not an object
an object is an unsystematized system, so to speak Valeri Souchkov

Valeri Souchkov Nikolay Shpakovsky Nikolay, I slightly expanded my previous
answer.

Valeri Souchkov Nikolay Shpakovsky It all depends on the level of
divisibility. If we we get to the subsystem, which is an element - that is,
indivisible by other elements are a component, we have the right to say -
that's enough to share. This the element is no longer a system relative to our
context.

Nikolay Shpakovsky Valeri Souchkov Dada ... I understand, you can leave the
subsystem at the object level, so as not to clutter up the analysis with which
I disagree - A car, a bug, scissors, molecules, an atom - these are all
systems in my opinion - A car, a bug, scissors, molecules, an atom - all this
can be represented as the system

Valeri Souchkov Nikolay Shpakovsky, yes, you're right. It is
represented. because that the system is not some kind of physically classified
object. These are the boundaries which we conditionally outline around a
certain amount of interacting elements.

Nikolay Shpakovsky Valeri Souchkov take off my hat ...

Valeri Souchkov Nikolay Shpakovsky Mutually!

Andrei Kuryan Nikolay Shpakovsky, 1. "The system is a mental image
...". 1.1. But because the same thing I can say about the mechanism, car,
etc. In this sense the hammer is a mental image. 1.2. If this statement is
necessary in order to indicate that there are no systems in real life, I do
not agree. Correctly defined the system consists of real-life elements and
interactions between them. At the design stage, we operate on system
descriptions, based on which we can create a system in real life.

Valeri Souchkov Andrei Kuryan Yes, that's right, but that's when we talk about
artificially created (designed) systems. Everything is simple
here. Difficulties sometimes arise when we try to determine the boundaries of
natural systems.

Nikolay Shpakovsky Andrei Kuryan I will not argue ... if it is more convenient
- you are welcome.

Alexey Schinnikov Nikolay Shpakovsky I'm without a second thought: why is it
wrong (I want to understand) to say that the car system went.

Andrei Kuryan Valeri Souchkov, in natural systems, boundaries are defined the
scope of the laws by which we describe these systems. Borders the solar system
is the effect of the law of universal gravitation on a star, planets and other
objects. The purpose of describing the solar system as a system is checking
the compliance of the real movement of objects with the description through
the law If a there is conformity, then we have reached an understanding of the
solar system.

Valeri Souchkov Andrei Kuryan But a grasshopper, for example? :)

Andrei Kuryan Valeri Souchkov, but not everything is known about the
grasshopper :)

Valeri Souchkov Andrei Kuryan, but I just imagine it as a system :) Jumping,
chirping. Why has a combination of components. Makes up link in the food chain
- the supersystem :) 2

Andrei Kuryan Valeri Souchkov, well, if you can by such a description to
predict when and where he will jump and how to chirp at any moment time, then
yes, you have reached an understanding of the grasshopper system and you can
use it for some benefit, I don’t know, tickle my nose instead of an alarm
clock by in the morning :)

Valeri Souchkov Andrei Kuryan That is, since we do not know what he is
thinking grasshopper and cannot predict his behavior, we have no right to
present his system? Although it seems that we can quite clearly outline its
boundaries and even split into subsystems? By the way, here we are faced with
another significant a sign and property of the system is its behavior.

Sergey Simakov Returning to the fact that the system is about information
compression, models. Which we will change later. And in this sense, the dog
thinks systemically. Consciously-not consciously - the second thing. The
concept of benefit - for me - is also a completely second matter. I mean, what
we call an object is frozen process. In the sense that to simplify the change
of the whole model, something does not change. Selecting objects is when there
is a description with the boundaries. Another object is another
description. And between them happens interaction is another rule and
outcome. So just the brains are arranged. AND as I can assume, judging by how
the vision is arranged, which is at the level neurons selects some simple
entities, and then complicates, it’s completely ancient biological modeling
mechanism. Well, it’s just that a person has grown to thoughts about thoughts
as we think :)

Nikolay Bogatyrev I don’t quite understand why dear colleagues doubted
that the grasshopper is a system ... Of course, a system. It is organismic
structural and functional level of the animal world. Grasshopper has levels /
subsystems: organ systems, organs, tissues, cells, cell organelles,
molecular level (DNA and RNA). Deeper is the atomic level. Besides,
the grasshopper has several layers of supersystems: population, species, guilds,
ecosystem, biosphere. Biologists can blame me for not being all
listed levels (they can be made more fragmentary). But overall I seem
missed nothing ...

Andrei Kuryan Valeri Souchkov, as a system you can consider all that anything,
even a grasshopper. Why do this? Why the grasshopper system is better the
original grasshopper? We seem to be talking about this, no?

Sergey Simakov We are now trying to describe the system using system tools :)
Those. find boundaries where there is no longer a system. I'm afraid that this
conceptual apparatus this is an internal contradiction :) cannot be logically
explained. Need something to believe. Consciously. :) For example, that there
is an objective reality that we at least consciously perceive it as a
system. Is always. And then what Bazaar - we just see everything
systematically, because as soon as we realize, we do borders.

Sergey Simakov And then the comparison of the original grasshopper and the
grasshopper system - this is the systemic process of changing the grasshopper
system model. Is it better or worse - it is again something supersystem. But I
can’t think of a measurement procedure. Here if only not to fall into a
scholarship. Meaningless and merciless. Sometimes better say - BELIEVE :) Make
a reduction (how we work with models) and work.

Sergey Simakov The problem is that, as it seems, we plunge into logical loops
and ignorance of something. Our knowledge is in a sense implementation, sample
of a certain probabilistic state of the brain. Yes, the model is poorer the
original. Therefore, it is necessary by force of will to put the stubs of
faith. And go to practice. Maybe later, as the model becomes more complex, the
method of simplification of faith will go somewhere further away. And it may
be more practical an attempt to understand what we believe. Where are the
borders. Although ... it's up to infinity ...

Alexey Schinnikov Apparently there is disagreement about whether the System
model of a real system or system this is a real system, that is the system is
either a model or reality :) 1

Nikolay Bogatyrev I will add ... The real grasshopper and the grasshopper
system, as a rule coincide almost always and almost 100 \% ... But biological
concepts / constructs "View", "Rhode", "Family", "Order", "Class", "Type",
"Kingdom" (arranged in ascending order) very often VERY can be very different
- in reality and in the model ... With this situation, an unstoppable battle
is waged throughout life taxonomists and taxonomists ...

Alexey Schinnikov Nikolay Bogatyrev in books on dialectics I met a simple
explanation: there is a system Generally and a system Specifically. System
Generally is idea, in programming - a class, and the system Specifically - a
specific instance class, materialized idea, reality object

Alexey Schinnikov System Generally (class) - grasshoppers, and the system
Specifically - the grasshopper that I am now observing in the grass

Nikolay Bogatyrev Alexey Schinnikov, well, yes ... The same thing is in
phonetics: we we know how the sound "A" sounds in Russian ... But even the
same person doesn’t will be able to repeat this sound twice absolutely
identical. (Instruments are easy confirm, although we and others do not notice
this, or rather, our brain - summarizes the signal).

Alexey Schinnikov Nikolay Bogatyrev most of all I like the approach
object-oriented programming (object class). Class car Volga describes its
properties and functions, and you can do with a class Nine-screen
manipulations, predict class changes to a new class. BUT class objects are
specific Volga cars owned motorists with numbers. Conveniently. Also on
Nine-screen you can analyze a specific object - the life cycle of a car at
Uncle Vasya, for example

Alexey Schinnikov In my systematic approach (Event Architecture), I made:
divided the system into class systems and instance systems (objects)

Alexey Schinnikov That's why I asked Nikolay Shpakovsky why not say the car
system went. Yes, it’s wrong to say that if we are talking about a
system-class, but it’s right to say so, if we are talking about class instance
system

Sergey Simakov Yes. There is no system without a model. A model is a mapping
one physics to another. In this sense, for us (physics is a human body with
brain, wider society - as a collective mind) the whole system. Including
specific grasshopper and its generalized model. Again I suggest how much this
perhaps return to even the simplest models of the brain. Like models the
builder. Just the supersystems are different. A particular grasshopper
launches one set of models. And the grasshopper model is different. Moreover,
they are largely random and somehow filtered by context. For example a
particular grasshopper in a meadow in in the context of "beauty" launches
models of sunrise, and in the context of "on pin "and" beauty "are completely
different. The model of a grasshopper without a grasshopper in kind - launches
memories of a beautiful functional diagram :) And that of a beautiful to the
woman. :) Actually, the launch of different contexts gives rise to
differences. But for this we need some meta-supersystem in which everything is
measured. AND braking is carried out. I mean, that you can be clever and
maliciously determined to infinity. Everything is connected with
everything. And again it turns out that without Faith (and so will come down),
which stops the determination process and starts the transition to activity -
no way ..

Nikolay Shpakovsky Sergey Simakov, the grasshopper’s mental model (system),
different grasshopper models and the grasshopper itself are two different
things. same thing about confusion arises because the system exists on its
own, it it is present in the model, and the system is present in its real
embodiment.

Sergey Simakov Nikolay Shpakovsky And I'm about it. If I understood correctly.
But still, in practice, I insist that I would understand how it all
created. And why are we building these specific models. For at first they were
created a set of grasshopper models in various contexts, then combined into a
certain metamodel. And then the real something about taking measurements was
classified as a grasshopper which in itself is much more complex, but with
certain parameters that are important to us.  Nikolay Shpakovsky Sergey
Simakov the problem of understanding the concept of a system confused by the
fact that in practice the solutions of IZ we use it in two main cases 1. there
is a certain device or phenomenon, technical, biological, any. We need to
understand what its structure is and how it works. In order to improve there
or something else. So we build the systems, subsystems. 2. You need to create
a new device or technology. In this case immediately start with the
construction of subsystem systems in order to understand what will be future
car. Then we present them in the form of some kind of graphic, mathematical,
physical and any models. We study them and - finally - get real new car. The
blacksmith refers to the first case.

Sergey Simakov Nikolay Shpakovsky In my opinion there is a very delicate
case. how to classify. In any case, construction is not on empty
location. Those. we need initial models that we will change. Or, understanding
them initial failure for transformations, studying "physics" followed by a
change, or believing that it’s enough, building new models on this basis. BUT
there how it goes. In this sense, inventive tasks are about refinement
knowledge. New device and technology is new knowledge implemented in
"physics". In this sense, TRIZ in the broad sense is one of the sets of sets
general environmental study instructions focused on meeting human needs with
more technical means. well and compare these sets with others and with each
other, where the boundaries are in social I mean more expensive :)

Nikolay Shpakovsky Sergey Simakov while realizing new knowledge in real life,
it is necessary first build it in the mind, that is, build models, especially
the mental model - system Of course, given so many things who argue. But the
essence it does not change matters. The system is the foundation of everything

Alexander Veres One of the main properties of the inventor is
imagination. Exactly at him and there are systems :)

Sergey Simakov Nikolay Shpakovsky So I am about the same. First in the head
(heads). Again. Expand a bit. I will even say more. This model is probably
more than 90 \% is not verbalizable. If you try to verbalize everything
important - You can check in. Previously verbalized from conscious to
automatic has passed. Those. system as a model with which the aggregate really
works brains an order of magnitude greater than described. For the system
that’s in the head, as one of the most important facts swearing with a
daughter can get. Which not in which list of stakeholders does not appear and
cannot be listed :) Offtop - exactly therefore, they have been studying at
school for 11 years in order to automate at least 10%. :) For if try to build
at least a graph of knowledge - you can get into a mess. :) And exactly
therefore, I carefully relate to AI for inventive purposes. By the way, how
complexity control method - various numerical programs modeling. When it’s not
important - why. This is to make the heads unload. And learning how to
actively "collaborate" with them is a topic. Those. highly the significant,
repeating, conscious model of the system is a very small fraction of really
applicable. Attempts to build a more informed model have and Negative
consequences. Actually speaking about it there was a conversation about a
lens. When, with the awareness of the model, a lot of interesting things
turned out to be. Which not it was necessary for the specific purposes of
building the model, as explained to the main concepts for newcomers.

Nikolay Shpakovsky under damned tsarism, when logic was taught in schools,
they said this: "a revolver of the Nagan system. models of the Tula plant"
someone can comment on as part of our discussion?

Alexander Veres Nikolay Shpakovsky wild people, did not know what the system
is, until the GAW resolves everything.

Valeri Souchkov Nikolay Shpakovsky Of course. In this case, we are talking
about more An abstract system model that summarizes specific instances of the
system. Here for example, the abstract elephant system. Elephant can be
African, Indian - morphology and anatomy will be basically the same, the
differences will already be specific implementation. That is, there are some
general properties and signs. For example, one elephant has oval ears, the
other round. But everyone elephants will certainly be the trunk. But the
hippopotamus does not fit into the "elephant" system - there is no essential
component - the trunk.

Nikolay Shpakovsky that is, there is only one system - the models are
different, right?

Nikolay Bogatyrev Valeri Souchkov, you can add - and each individual an
African elephant, and a grasshopper of the same species, and a gun of the Tula
plant will have individual differences (by type of individual fingerprints and
shape auricle in humans).

Nikolay Bogatyrev Nikolay Shpakovsky, "that is, there is only one system - the
models are different, right? "- There is simply a HIERARCHIC STRUCTURE: 0) a
car, 1) a car car, 2) Ford car, 3) Ford Fiesta model, 4) Ford Fiesta German /
American / Russian production 5) individual copy cars ... The same thing -
with elephants, guns, grasshoppers and people ...

Sergey Simakov But it’s possible in another way. Even worse. I would say one
name systems, and for each context - different models. The theme is famous for
us, when practiced intuitively. When for a metalbreaker and racer :)

Sergey Simakov In a variant with a hierarchy just models more or less the
same. Just a narrowing of the permissible range of change parameters from top
to bottom. Well or in more complex - different sets, interconnected.

Nikolay Bogatyrev Sergey Simakov, with a change of context will change and
taxonomy / taxonomy / hierarchy ...

Sergey Simakov Nikolay Bogatyrev Here it is !!! And when you consider that the
taxonomy and other - this is very often a choice in a sense, a random
option. For toli so toli this way ... :) I again about flexible logic turning
into Aristotelian :) Which falls apart when detailing .... And after all,
something like this in this mess work. And even more Heap of people does not
even realize, and effective decisions accepts :)

Nikolay Bogatyrev Sergey Simakov, yeah ... I also sometimes think how it all
exists, works and how do we sometimes manage to understand each other (true
....... not always ... :)). And elegant solutions are often found precisely
when changing context: TRIZ and jokes are built to a certain extent ... :) 2

Boris Stroganov Nikolay Shpakovsky Anyone who created natural systems knows
why did he create it

Alexey Schinnikov Nikolay Shpakovsky in the philosophy of dialectics is a
division into systems-in general (classes) and systems-specifically (objects,
class instances). This the same approach is used in object-oriented
programming, and well would use in TRIZ to eliminate confusion of the abstract
and concrete 

Andrei Kuryan Nikolay Shpakovsky, systems class - system - model - version -
batch - product№, etc. - this is a classification and at the same time
identification, adopted in traditional technology. There are international
standards establishing the procedure for such classification /
identification. But it’s not really what TRIZ offers. In the system operator,
a separate axis is the evolution axis systems, just designed to view the
system from a point view of how she appeared and changed. In other words, how
did its composition, structure and functionality. Therefore, by a system we do
not mean just some kind of representation of a real object, but also the
totality changes in it.

Nikolay Bogatyrev Andrei Kuryan, in biology an evolutionary view of what in
TRIZ is called a system operator, a very widely used method in paleontology
(in combination with reverse "engineering" ... :)).

Andrei Kuryan Nikolay Bogatyrev, yes, but there is a vague suspicion that the
revolver Nagan system does not apply to biology :))

Sergey Simakov Andrei Kuryan Well, why ... In the context of interaction with
Biotel - The most direct. :) You can still have many biological contexts
look. :)

Nikolay Bogatyrev Andrei Kuryan, alas, has ... :( The most direct ... Function
Nagana - stop the biological functioning of a large ground animal, for
example, human. By the way, biological and technical evolution have similar
stages, only they go in different directions ... :)

Sergey Simakov Here. Nagan - as an element in the evolution of firearms -
weapon evolution - vehicle evolution - human society evolution - evolution
biological systems. Well, then you can look down to bacteria and DNA. :) BUT
down (or up), as the most stable and fundamental - physical models of the
world. If you go to subsystems, you will fall into supersystems. And vice
versa. Simply when they go down - the appearance of other super-systems goes
more often on the unconscious level. Otherwise, the roof will go. I'm so, play
around. And as a comment to higher conversations. The hint that this disgrace
is not uniformly you will describe. And all the time you have to keep a
balance between the accuracy of the model and time to create it. Dropping to
the accuracy of inaccurate rules it makes sense if it affects the result
significantly. For in the process anyway a bunch of feedbacks will be
involved. And the most vile, how will it affect it will become known, and even
then not always, when everything is already over. :) Those. it is not clear
how to keep the balance consciously :)

Andrei Kuryan Nikolay Bogatyrev, yet the function of Nagan is to throw
bullets. The use of Nagan in decisions to terminate biological the functioning
of large land animals, including humans, is far from single use. IMHO, the
threat to use a gun in disputes made for the evolution of society, much more
than its direct application destination. And they can chop nuts))

Nikolay Shpakovsky Andrei Kuryan throwing bullets - the appointment of Nagan
prescribed its developers and manufacturers. It is still very unsuccessfully
called the GPF.  In general, an object has no function (except subjectively
assigned to it), it appears depending on the location of the object in the
system. Let's say we try turn off the light, do not reach for the hand and use
to extend the arm uncharged gun - then the function in his system will be like
this - part of the transmission and a working body for acting on the switch (a
person who saw a gun for the first time, he says - oh, this is a contraption
for turning off the light).

Andrei Kuryan Nikolay Shpakovsky, this is a very interesting question - is it
Nagana part of the light inclusion system. In terms of multi-screen - no,
because nagana was not intended as an element of this system. What then this?
In IT, this is called a decision when the result is achieved by the assembly.
ready-made components, initially not intended for such a result. AT life it is
called life-hack (I do not know the analogue in Russian). I don’t know Can
Nagan be considered as an element of the inclusion of light. Or system - This
is something more structurally stable.

Sergey Simakov What will change in solving a practical problem if we don’t
let's call the system?

Valeri Souchkov Sergey Simakov What will be harder to see will change
resources that you can use to find a solution with a higher degree of
ideality. Allocating a system, subsystems and supersystem, produce the search
for resources can be more fully and - importantly - systematically.

Sergey Simakov When I hear the GPF, my hand involuntarily reaches for the gun
:).  Seriously, highlighting the GPF allows you to focus on the main thing
here and now, transferring secondary tasks that arose during rhenium to
later. It's all it used to be good, at the time of the appearance of TRIZ,
when the systems were simple. AND it was enough to change one or two
parameters to make the functionality better.  Now you stretch your tail - your
nose is stuck. Therefore, it is of course useful to try to find GPF, but it’s
dangerous to focus on it. A set of functions with borders.

Sergey Simakov Valeri Souchkov So I about it. :) I.e. always a system. What
would not considered. Moreover, according to the hypothesis, all this is
stitched biologically. :)

Valeri Souchkov Sergey Simakov GPF hrosha when the system is simple, how
hammer. Today complex multifunctional systems are developing. Take the same
food processor. Which has several working bodies, and depending on What we
need at the moment, the GPF is changing. Of course, we can say that GPF of the
processor "process the product", which, in general, is correct, but too
abstractly. What about the smartphone’s GPF? Launch applications? To talk?
Take off video? It all depends on the context of the application of the system
at a particular moment.

Sergey Simakov Valeri Souchkov Yes. The problem is that the desire is ALWAYS
to find GPF leads to, as you write, finding it at the highest levels
abstraction. Too high in the supersystem. Too far from physics specific
system. The formal solution is to expand the system so that it starts to
include a bunch of hell knows what. Well yes. Monstrous. But not decisively.

Sergey Simakov TRIZovites will forgive me, I’m also identifying a worker body,
engine and stuff. Never in life in practice have used. There is a functional
model. Enough.

Valeri Souchkov Sergey Simakov Yes, when we define as a GPF too abstract
function, it loses its binding to specific components and processes. This is a
goal, not a function. Therefore, in multifunctional systems I don’t define the
GPF at all, but functions aimed at those elements of the supersystem, for the
implementation of which a system was created.

Valeri Souchkov Sergey Simakov In the functional model, they are all the same
come to light, albeit in an unnamed form. Then it’s already possible to
identify analyze .... if there is such a purpose. And it sometimes arises in
tasks optimization.

Nikolay Shpakovsky Here I am about the fact that it is not clear where to put
this a system for solving a problem, and why it is needed ... apart from
resources, perhaps. Everything changes dramatically if the solution to the GPF
problem goes through the woods. You need to understand the function, which is
performed by the object in the system in this particular situation. Systems
after all different ones are used when deciding ... For example, what is the
harmful function of the harmful system?

Andrei Kuryan Sergey Simakov, GPF allows you to understand what was originally
invented a system.

Nikolay Shpakovsky Andrei Kuryan for what?

Andrei Kuryan Nikolay Shpakovsky to understand why was originally invented a
system. When you poke a nagan into a switch, then the GPF nagan - "throwing
bullets" should hint to you that you are not using the gun directly
destination. Therefore, this is a reason to use instead of Nagan in the system
something simpler, like a wooden stick.

Nikolay Shpakovsky Andrei Kuryan is understandable for finding resources ... a
little narrow but also necessary ... FOS everything is normal the problem is
probably what we are talking from different approaches in our approach, the
concept of a system (technical, functioning, complete, harmful, etc.) plays a
key role.

Nikolay Bogatyrev Of course, Nagan has many over-systems: premium weapons,
"stick" to turn off the light, goods for profit in the store, threat, sinker
for fishing, etc. - But this is - known to anyone object, process, phenomenon
...

Nikolay Bogatyrev Well, and again I repeat. For the simplest artificial
systems you can find / understand / discover / prescribe its purpose, GPF
... But for natural, complex and / or biological systems to find / understand
/ prove function, purpose, purpose is often very difficult ...

Nikolay Shpakovsky Nikolay Bogatyrev and do not need .... you need to
understand its function in circumstances of the task

Nikolay Bogatyrev Nikolay Shpakovsky, well, yes ... That's what many do
biological systems and those who describe them ... True, here you can make a
mistake ... :) (But everyone can make a mistake, always, everywhere and in
everything ... :)).

Andrei Kuryan Nikolay Bogatyrev, goals for the artificial system are defined
supersystem. The function of the system is determined by its structure, which,
in its the turn is determined by the limitations of nature. System level going
coordination of requirements and functions. While for the natural system is
not a supersystem is defined, we cannot talk about its purpose and
requirements for it. But nothing interferes with defining its
functions. That's just what? :))

Sergey Simakov There is such a thing ... 1. Brains require completeness of the
model. When it is not completed, and it is felt - the fuss begins. The
composition of the model includes the answer to the question - "what for" from
the point of view of higher forces. So to speak meaning. And surely - why am I
doing this. But this is less so. That's reason for seeking meaning. Like a
"heart requires." 2. Another reason is the appearance goals allows you to
dramatically simplify the prediction. In fact, RBI is about that.  If we say
that the goal is to preserve the species, then that which is not Corresponds
to reject. But life is more complicated ... :)

Alexander Karmazanov Andrei Kuryan the goal for the artificial system is
determined creator of the system. A supersystem as a source of constraints,
and even if the supersystem is recognized correctly. And for the natural
system, or some third-party, the observer determines the purpose of an alien
system.

Nikolay Bogatyrev Andrei Kuryan, "But Nothing Prevents Defining Her
functions. That's just what? :)) "- Well then !? - To," explaining "this
incomprehensible, slurred, volatile, indifferent, hostile world, try find a
way to avoid blows, losses, disappointments, pain, and if you're lucky, then
and play and experience the delightful feeling that you own something, have
and you control! ... After all, man is the sovereign of nature, man - it
sounds proudly, Well and further in the same vein ... :)

Andrei Kuryan Nikolay Bogatyrev, if as a supersystem for the natural systems
we define the observer and society, and the goals are the ability to predict
future states of this system, then there is only one small step left before to
benefit from this system. Actually, knowledge about the future states of the
natural system is already beneficial to society at least by the fact that
reduces risks. Then the boundary between artificial and natural systems
becomes vanishingly thin. Moreover, natural systems can considered as a subset
of artificial, which is not yet adapted society for their needs :))
Andrei Kuryan Alexander Karmazanov, the system has many creators
(interested parties), which are just in the supersystem. More
Moreover, the system has exactly as many super-systems as its creators.

Alexander Karmazanov Andrei Kuryan that is, if the design bureau created
blueprints, design bureau, project, factory produced ... and the order used
system, now there are a lot of super-systems in the created system ... I think
you got off the ground with this thesis.

Andrei Kuryan Alexander Karmazanov, you have listed far from all the stages
system life cycle. But in general, it is true: at different stages and stages
of the life cycle of systems are different supersystems.

Alexander Karmazanov Andrei Kuryan at different stages of the life cycle and
the system itself is different, it changes by the creators .. so you need to
fix the system and the creator. How in programming, any algorithm contains at
least one error, someone later fixed, new system, new creator ...

Andrei Kuryan Alexander Karmazanov, in TRIZ we consider different stages and
system stages as one system. Moreover, even the evolution of the system we We
consider as one system.

Alexander Karmazanov Andrei Kuryan well, if in triz you consider all stages as
one system, so the creator is one, generalized.

Sergey Simakov Alexander Karmazanov Just about it above and wept. What if an
attempt to create a comprehensive and even conscious model requires an
infinite time and resources. Therefore, there are so many ways to God. For
each has his own context. Someone can’t eat without RBI, and someone saw him
in a coffin. Someone draws algorithms of 10 sheets with books of comments and
footcloths draws boundless by himself, and who is in 4 lines :) How
convenient, do it. Only if suddenly it doesn’t, look for the reasons and
change the approach.

Sergey Simakov Unfortunately, the above approach has the wrong side. Weak
negotiable TRIZovtsev, poor interchangeability. If in engineering in In
principle, one builder will replace another, then in projects with TRIZ -
everything is worse will be. Customers will howl. :) Here are the reasons for
another crying ... So such discussions somehow allow you to synchronize
understanding of the boundaries of heterogeneous comrades. Not adherents of
one style of TRIZ kung fu. For it is one thing when There is a list of terms,
and the other is live communication.

Alexander Karmazanov Sergey Simakov, but don’t say they organized Ma TRIZ,
other societies ... There are no common developments so that groups of authors
do ...  Each shuffles, adds in its own way .. Scientists can no longer in the
modern world is making discoveries alone. It’s only they who choose someone
who will be the "face." Where is the collective development and development of
TRIZ?

Andrei Kuryan Alexander Karmazanov, "Well, if in Trize you consider all stages
as one system, it means the creator is one, generalized. "From which this
strange conclusion? I wrote above that the system has many stakeholders,
like supersystems.

Alexander Karmazanov Andrei Kuryan logically reason. If the system is each
times is changing, creators are changing. Different supersystems. And then we
look at one generalized system in its development ...-> therefore one
generalized creator, and one generalized supersystem should be, otherwise it
turns out on the left us behind the trees is a forest, and from the right
behind the trees, other trees.

Andrei Kuryan Alexander Karmazanov, the life cycle and evolution of the system
are different axis measurement system. I still do not understand what is
wrong?

Nikolay Bogatyrev Andrei Kuryan, "Actually, knowledge of future conditions the
natural system is already benefiting society, even if it reduces risks. "-
Well, yes, agricultural, meteorology, medicine, ballistics, chemistry - all
these areas have more or less strong prognostic devices than and attractive
... :)

"Then the border between artificial and natural systems becomes disappearingly
thin. Moreover, natural systems can be considered as a subset of artificial
that is not yet adapted by society to its needs :)) "- Yes, in the field of
biology, medicine, genetic engineering, bionics it especially noticeable.

Sergey Simakov Alexander Karmazanov Regarding TRIZ Sushkov above is very good
wrote.

Alexander Karmazanov Andrei Kuryan is not that the whole trize is built on
generalization or induction. And it began with the fact that Altshuller
decided to identify common methods, and laws ... You can’t apply the
particular somewhere, but somewhere the general .. Must to be a transition
.. From private to general, and vice versa .. you get ... in different
locations are used approval without transition ..

Alexander Karmazanov like this (picture added)

Ruin Aleksei Alexander, above the thread it was clearly said, it seems, by
Valery, figuratively speaking: as I want, I’m twisting, it’s not TRIZ, I
joked, but “you’re served in the army? "so rude? Your questions and
observations, unfortunately, are not have meaning / benefit / sense. As well,
and this whole discussion;)

Alexander Karmazanov Ruin Aleksei and yours too .. but we are all human and
therefore all ... 

Andrei Kuryan Alexander Karmazanov, the life cycle of the system is analogous
ontogenesis, evolution - an analogue of phylogenesis. 3rd axis of the system
operator describes hierarchical relationships between a supersystem, a system
and subsystems, 4th axis - systems and its anti-systems. The modern system
operator presents is a 4-dimensional space in which we consider the
system. (Unlike 3-dimensional in classic TRIZ). By the way, it syncs well with
approaches in traditional systems engineering. N. Khomenko at OTSM-TRIZ
proposed a generalized N-dimensional system operator and its 7-dimensional
implementation. Although his approach was not widespread.

Ruin Aleksei Andrey, are you serious? "Not widespread" is five. I even know
why.

Alexander Karmazanov Andrei Kuryan great all of this ... We just started by
that there is a generalized system, but why, then this generalized system
according to your a new axis of reference, there were many super-systems, and
many creators that contrary to the fact that a transition to a generalized
"system" was made, without generalizing everything else. I regret that I was
not able to show you and convey this mistake.

Andrei Kuryan Alexander Karmazanov, see, 1) if we consider the system at
different stages of the life cycle, highlighting for each stage a certain
state of the system, then for each such state we can define a supersystem. It
will be different supersystems. 2) analysis of the use of such an operation to
systems demonstrated that these super-systems are different from each other
stakeholders. 3) A reverse analysis of such an operation has shown that we can
in the framework of the traditional stages of the life cycle system to
highlight some intermediate stage. For example, a smartphone at the stage of
use can be used as camera, card or phone. We can consider such states as small
stages of life cycle as part of the general stage. And for each stage we can
highlight different stakeholders: photographer, tourist,
subscriber. (Physically, it could be one person, but with different roles.) 4)
As a result, it was formulated following rule: the appearance of (another)
stakeholder is a sign the presence of the system (one more) supersystem. Where
do you see the contradiction here?

Alexander Karmazanov Andrei Kuryan The contradiction is that the new
stakeholder this is a private phenomenon in a certain period of time, with
some new private system, or a new reference point. When the generalization is
done anyway one stakeholder will remain as a general concept. Otherwise, the
theory is not formulated it turns out. Perhaps this is some beginning of a new
theory, but not the old one.

Valeri Souchkov Ruin Aleksei Alexey, no need to distort. I requested speak to
the point. But again, only personal attacks and zero value. If you are you
know a priori and everywhere loudly proclaim that all these discussions
useless that you participate in them? All the best. one

Andrei Kuryan Alexander Karmazanov, I do not understand what generalization of
the system you are talking about say it.

Ruin Aleksei Yes, I have a lid. Waiting for before ... 7-dimensional
implementation H-dimensional model. And then what to do with it? What other
tool "shove"? This is impostor syndrome. Probably not a disease, but
unequivocal mental disorder. All these attempts to hide behind the
science-like is, most likely this same syndrome! This is not about anyone
specifically, this is common impression of the thread.

Alexander Karmazanov Andrei Kuryan you consider different systems in different
point in time, as it changes ... Say that there are different stackers, and
different supersystems .... Then you conclude that there is this generalized
system, which if a new stakeholder has appeared, which means there is a new
super-system. BUT with why? The system was different each time. If we are in a
single life cycle we consider as a generalized system, then why the
supersystem is not generalized, stakeholder too? And yes there were no such
concepts in TRIZ .. accordingly, it’s even not TRIZ. Well, there is
Lobachevsky’s geometry and no one else argues .. But it’s not TRIZ.

Ruin Aleksei And not Nescafe! ;)

Andrei Kuryan Alexander Karmazanov, I brought a smartphone as an example. AT
different points in time the smartphone (system) is still the same, but its
use different (different states of the smartphone). As part of the LC, we are
considering a specific, not a generalized system. Actually, the concept of a
system and its life cycle (hundred states systems) in TRIZ is borrowed from
system engineering; we didn’t invent anything and we use these concepts in
TRIZ in their original interpretation.

Alexander Karmazanov Andrei Kuryan So what if you have a system =
"many-hands-many-legs", then you have one stakeholder =
"many-hands-many-legs", the same system over ... We summarized this ...

Andrei Kuryan Alexander Karmazanov, for the purpose of solving IZ stakeholders
it is better distinguish, not generalize.

Sergey Simakov Alexander Karmazanov I just suggest that each his own applied
logic to solve a specific problem. If everyone succeeds, well, thank God. You
can’t consider all this without solving specific problems.  I repeat, this is
about creating using some rules that for some reason TRIZ each one has its
own, more accurate models with restrictions on deadlines. If it works - well,
okay. To distinguish - not to distinguish ... That's all context is
determined. Task. Experience. Knowledge. Time. Etc. Problems are being
decided. And okay.

Andrei Kuryan Sergey Simakov, we are not only dealing with personal experience
to solve problems, but also replication of such experience.

Alexander Karmazanov Sergey Simakov Yes, it's sad .. that there are no joint
collectives that would develop a theory ... If the system is interdependent,
or some multidimensional yet ... it is called the old concept, they make some
conclusions, including new concepts ... in the end, all gurus, everyone has
their own Zen ...

Sergey Simakov Alexander Karmazanov Times when the TRIZ brand was in one
strong hands passed. And will not return. So what is what is.

Valeri Souchkov Alexander Karmazanov This is completely natural. Theory in
TRIZ almost does not develop at all. Tools are being developed, mostly already
existing ones. TRIZ practices themselves develop them - companies or
freelancers, who receive income from their customers. The size of such
businesses does not allow engage in theoretical research. In addition, they
are interested in creating your intellectual property. To budget
organizations, specializing in research, for example, universities, TRIZ so
far penetrated weakly, and that, basically, at the level of teaching. I wrote
about the reasons above. There are a number of associations uniting Trizovites
with similar views.  (rather ideological than theoretical), but even there the
level of any Research is still very very low. This happens one at a time.
reason. Despite such a long time of existence, TRIZ has not yet reached
critical mass, which would allow a high-quality leap, go to a new level of
development, a new S-curve.

Sergey Simakov Andrei Kuryan If the other does not perceive this experience,
then you will not be forcibly sweet. Let him solve the problems in his own
way. There is a result, well, okay. It will replicate its experience.

Alexander Karmazanov Valeri Souchkov You know this as with free software
providing ... If people around the world did not help open free to develop an
operating system, such as Linux ... That it wouldn’t exist .. Yes there were
some separate programs would be free, but there would be little sense from
them ...  Valery tell me if there are any developments or research on the
topic of is there any kind of dependence on the presentation of the product,
the presentation on there, for example, a working body, and methods for
resolving a conflict of requirements?

Valeri Souchkov Alexander Karmazanov I tracked the development of Torvalds and
the whole community from the very beginning, and worked in Red Hat, and in
Debian, and in Ubuntu But there the community had a specific goal - to create
an alternative to a very dear Unix at that time. I myself started in 1995 at
Solaris on the Sun Sparcstation, the price tag there for all was hellish. And
the need was massive and very large.  Is there a massive need for TRIZ?
not. Is there a big need? Not.  TRIZ is still a very niche product. Therefore
expect massive open source is not possible. What do you mean by presentation
form - let's say working body? Geometric? As far as I know, works according to
such there is no level today, except for the old work of I. Vikentyev
"Geometric spatial operator. "

Sergey Simakov Alexander Karmazanov Actually TRIZ started as a section
psychology. And with it it turned out as with psychology. For the subject is
the same turbid. In psychology, too, a lot of schools and directions. Just now
into her truly scientific methods begin to penetrate and it begins to approach
natural sciences. BUT the fee for it is an awesome price tag research. The
disappearance of the term psychology. For this is already some kind of
synthesis whole body of knowledge. OS type software development - orders of
magnitude simpler task than developing scientifically sound, reliable, weakly
dependent on subjective factors of the methodology for changing systems. If on
the creation of Linux so much effort, time and money has been invested, what
can we say about TRIZ ...

Alexander Karmazanov Sergey Simakov you probably wanted to say that to solve a
real problem, the brain needs to recognize phenomena, use concepts, to
abstract, to lead to a task convenient for the brain, so that you can it was
according to a previously known algorithm to the brain to get an effective
solution. At In this, the algorithm itself for brain structures must be
executable. Only here again Problem algorithms in the freeze available are not
reduced to an exact solution. Or maybe algorithms diverge, that is, lead to
different solutions.

Sergey Simakov Alexander Karmazanov It is possible and so. Just the word
"algorithm" to me I do not like. This word in the usual sense is associated
with a mechanistic understanding of systems. Formal logic. TRIZ algorithms are
closer to artistic. Not even soft logic. No wonder we have so much water here
in a mortar with systems push and can not agree to the end :). And I have a
very there is a big suspicion that any algorithms are not reduced to an exact
solution. For by our knowledge is inaccurate by definition. And we can’t work
according to the algorithm. AND etc. The only solutions are the wildest
exception ...

Alexander Karmazanov Sergey Simakov Moreover, to give an exact definition to
yourself the concept of "Algorithm" is impossible. But the accuracy of the
decision depends on the performer - the brain, if the errors of the contractor
in the amount give no significant discrepancy, then say the algorithm
converge. And if in TRIZ the algorithms were completely artistic, there would
be no standards, techniques, and ARIZ-85. AND accordingly, based on the
solution of problems by these algorithms, there would not be inventions
brought to life. The question remains, but do they definitely exist?  embodied
on the basis of these solutions of the invention?

Sergey Simakov Alexander Karmazanov Actually, there are such art science. And
when they teach art, there’s a tough drill and rules :) Like a brush keep,
composition, rules, rules, rules ... Only one after this drills accurately
copy the landscape can. In the style of some kind of artist. Not touches. And
others will smear something carelessly - and the people’s stomach butterflies
...

Hans-Gert Gräbe Alexey Schinnikov "Most of all I like the approach
object-oriented programming (object-class). "

The subtitle of Szyperski's book "Component Software"
<https://dl.acm.org/citation.cfm?id=515228> is "beyond object oriented
programming ". The reason is simple since he observed that in IT service
structures the "customer" is responsible for the state (the "data") and the
supplier for functionality (the "behavior"). Hence there is a fundamental (!)
contradiction from the business point of view and OOP is the wrong (!)
solution. It is mere a compromise (once more). Time to apply TRIZ to solve
\emph{that} contradiction in a more sound way?

Alexey Schinnikov Hans-Gert Gräbe unfortunately I do not speak English and
others foreign languages ...

Valeri Souchkov Alexey Schinnikov Google translate! :)

Alexey Schinnikov Valeri Souchkov google that translator)

Hans-Gert Gräbe Alexey Schinnikov "In his Systems Approach (Architecture
Events) I did just that: divided the systems into class systems and instance
systems (objects)"

Since you are a very expert in OOP I suppose you know about the three ways
objects come into live: constructors, factory objects and factory methods.
Only the first is bound to the notion of class in your sense. For the second
and third only the definition of \emph {interfaces} is required, i.e., you
need only a \emph {description} of the function. In modern architectures
(APIs) \emph {this} is the main way objects come to live. Factory objects are
used in the most cases even to monitor object live cycles - just another
concept (evolution) that was used elsewhere in the discussion in a way
completely unrelated to the questions discussed here. Time to compile a better
"system theory" (using also the" well forgotten "old ideas)?

\ccnotice

\end{document}
