\documentclass[11pt,a4paper]{article}
\usepackage{od}
\usepackage[utf8]{inputenc}
\usepackage[main=german,russian]{babel}
\usepackage{tikz}
\usetikzlibrary{shapes.misc}
\usetikzlibrary{arrows.meta}

\title{Zum Zusammenhang zwischen\\ den Entwicklungsgesetzen allgemeiner
  Systeme und\\ den Entwicklungsgesetzen technischer Systeme}

\author{M.S.Rubin, Moskau}
\date{Version vom 6. November 2019}

\begin{document}
\maketitle
\begin{quote}
  Original: \foreignlanguage{russian}{О связи комплекса законов развития
    систем с ЗРТС}.  Vorabversion. Übersetzt von Hans-Gert Gräbe, Leipzig.

  ZRTS steht für „Gesetze der Entwicklung technischer Systeme“. 
\end{quote}

\begin{abstract}
  Eine präzisierte Fassung des Komplexes der Gesetze der Entwicklung
  technischer Systeme (ZRTS) wird vorgeschlagen.  Ein Komplex universeller
  Gesetzen der Entwicklung von Systemen (ZRS) für beliebige Systeme --
  materiell oder immateriell -- wird ausgearbeitet. Es wird die Beziehung von
  ZRS zu ZRTS aufgezeigt sowie die Beziehung von ZRTS zu Werkzeugen zur
  Analyse der Entwicklung technischer Systeme und der Lösungen von
  Erfindungsaufgaben. Es werden Möglichkeiten der Entwicklung von ZRTS auf der
  Basis von ZRS aufgezeigt.

  \emph{Schlüsselwörter:} TRIZ; Entwicklungsgesetze technischer Systeme
  (ZRTS), Entwicklungsgesetze von Systemen (ZRS), Evolutionäre
  Systemwissenschaft
\end{abstract}

\section*{1. Zu Gesetzen der Entwicklung technischer Systeme (ZRTS)}

Die Idee, Gesetze der Entwicklung von Maschinen abzuleiten, wurde bereits 1946
von Rafael Shapiro geäußert. Im Jahr 1977 formulierte G.S. Altschuller die
folgenden Gesetze der Entwicklung technischer Systeme (ZRTS):
\begin{itemize}
\item[1.] Das Gesetz der Vollständigkeit der Teile des Systems. Notwendige
  Voraussetzung der Funktionsfähigkeit eines technischen Systems ist die
  Verfügbarkeit und minimale Funktionsfähigkeit der Hauptteile des Systems.
\item[2.] Das Gesetz der „Energieleitfähigkeit“ des Systems. Eine notwendige
  Voraussetzung der grundlegenden Lebensfähigkeit eines technischen Systems
  ist der Energiedurchsatz durch alle Teile des Systems.
\item[3.] Das Gesetz der Harmonisierung der Rhythmik der Teile des Systems.
  Notwendige Voraussetzung grundlegenden Lebensfähigkeit eines technischen
  Systems ist die Resonanz (oder bewusste Dissonanz) der Schwingungsfrequenzen
  (der Betriebsfrequenzen) aller Teile des Systems.
\item[4.] Das Gesetz der Erhöhung des Idealitätsgrades des Systems. Die
  Entwicklung aller Systeme geht in Richtung der Erhöhung des Grads der
  Idealität.
\item[5.] Das Gesetz der ungleichmäßigen Entwicklung der Teile des Systems.
  Die Entwicklung der Teile des Systems erfolgt ungleichmäßig: Je komplexer
  das System ist, desto ungleichmäßiger ist die Entwicklung seiner Teile.
\item[6.] Das Gesetz des Übergangs zum Obersystem. Die Entwicklung eines
  Systems, das an seine Grenzen stößt, setzt sich auf der Ebene des
  Obersystems fort.
\item[7.] Das Gesetz der Dynamisierung technischer Systeme. Starre Systeme
  müssen dynamisch werden, um ihre Effizienz zu verbessern, d.h. müssen zu
  einer flexibleren, sich schnell ändernden Struktur übergehen und zu einem
  Betriebsregime, das sich an die Veränderungen der äußeren Umgebung anpasst.
\item[8.] Das Gesetz des Übergangs von der Makroebene zur Mikroebene. Die
  Entwicklung der Arbeitsorgane erfolgt zuerst auf der Makro- und dann auf der
  Mikroebene.
\item[9.] Das Gesetz der Erhöhung der Stoff-Feld-Interaktionen. Die
  Entwicklung technischer Systeme geht in die Richtung der Erhöhung der
  Stoff-Feld-Interaktionen: Systeme mit geringem Interaktionsgrad streben
  danach, diesen Interaktionsgrad zu erhöhen, und Systemen mit hohem
  Interaktionsgrad entwickeln sich in Richtung der Erhöhung der Anzahl der
  Verbindungen zwischen Elementen, der Erhöhung der Reaktionsfähigkeit
  (Empfindlichkeit) der Elemente, der Erhöhung der Anzahl der Elemente.
\end{itemize}
Ausgehend von diesem System von Gesetzen wurde die ARIZ-Methodik als TRIZ
bekannt -- Theorie des Lösens erfinderischer Probleme.

A. Lyubomirsky und S. Litvin schlugen eine eigene Hierarchie der
Entwicklungsgesetze technischer Systeme vor (Abb. 1). Ihre wichtigste
Besonderheit besteht darin, dass sie für die Durchführung von Analysen
technischer Systeme entsprechend den Gesetzen der Entwicklung bequemer ist,
hat aber ihre Nachteile, die unten beschrieben sind.

Bevor wir die Mängel beschreiben, geben wir zwei Definitionen aus
enzyklopädischen Wörter\-büchern.

Als \emph{Gesetz} bezeichnet man eine notwendige, substanzielle, nachhaltige,
wiederkehrende Beziehung zwischen Phänomenen in Natur und Gesellschaft. Der
Begriff des Gesetzes ist mit dem Begriff des Wesens verwandt.

Als \emph{Trend} -- ein Anglizismus -- wird die Haupttendenz einer Veränderung
von etwas bezeichnet: Zum Beispiel in der Mathematik die Zeitreihe.

Der grundsätzliche Unterschied zwischen Gesetzen und Trends der Entwicklung
ist offensichtlich, die Begriffe sind klar auseinanderzuhalten. Daher muss man
zum Beispiel die Einführung eines \emph{Trends der Evolution längs einer
  S-Kurve} in ein System von Gesetzen als nicht korrekt betrachten.
G.S. Altschuller beschrieb den Trend der Entwicklung technischer Systeme längs
einer S-Kurve, gab ihm aber nicht den Status eines Gesetzes. Der „Trend der
Evolution längs einer S-Kurve“ ist also kein Gesetz, sondern eine
Entwicklungstendenz.

Die Erhöhung der Multifunktionalität ist auch kein Gesetz, sondern nur eine
der Richtungen zur Erhöhung der Idealität.
\newpage

\begin{center}
  \begin{tikzpicture}[
      box/.style={draw,align=center,rounded corners=2pt,text width=4cm},
      >={Triangle[length=3pt 9, width=3pt 3]}]
    \node[box] at (5,10) (A0) {Trend der Evolution nach S-Kurven};
    \node[box] at (5,8) (A1) {Gesetz der Erhöhung der Idealität};
    \node[box] at (10,4) (A2) {Gesetz der Erhöhung der Vollständigkeit TS};
    \node[box] at (0,4) (A3) {Gesetz der Erhöhung der Abstimmung};
    \node[box] at (10,6) (A4) {Gesetz der Erhöhung der Effektivität der
      Ausnutzung von Flüssen};
    \node[box] at (5,4) (A5) {Gesetz der ungleichmäßigen Entwicklung der
      Teile des Systems};
    \node[box] at (10,2) (A7) {Gesetz der Verdrängung des Menschen aus TS};
    \node[box] at (0,2) (A8) {Gesetz der Erhöhung der Steuerbarkeit};
    \node[box] at (0,0) (A9) {Gesetz der Erhöhung der Dynamisierung};
    \node[box] at (0,6) (A10) {Gesetz des Übergangs ins Obersystem};
    \node[box] at (5,6) (A12) {Gesetz der Erhöhung der Multifunktionalität};

\draw[->] (A0) -- (A1) ;
\draw[->] (A1) -- (-2.5,8) |- (A3) ;
\draw[->] (1,8) -| (A10) ;
\draw[->] (A1) -- (7.5,8) |- (A4) ;
\draw[->] (7.5,8) |- (A2);
\draw[->] (7.5,8) |- (A4);
\draw[->] (7.5,8) |- (A5);
\draw[->] (A1) -- (A12);
\draw[->] (A2) -- (A7);
\draw[->] (A5) -- (A3);
\draw[->] (A3) -- (A8) ;
\draw[->] (A8) -- (A9) ;
  \end{tikzpicture}
\end{center}
\textbf{Abbildung 1:} Gesetze technischer Systeme nach A. Lyubomirsky und
S. Litvin
\medskip

\begin{center}
  \begin{tikzpicture}[scale=.95,transform shape,
      box/.style={draw,align=center,rounded corners=2pt,text width=4cm},
      >={Triangle[length=3pt 9, width=3pt 3]}]
    \node[box,text width=3cm] at (5,7) (A1) {Gesetz der Erhöhung der
      Idealität};
    \node[box,text width=5cm] at (-0.4,6) (A2) {Gesetz der Erhöhung der
      Vollständigkeit der Teile des Systems (Gesetz der Vollständigkeit der
      Umsetzung des Wirkprinzips)};
    \node[box,text width=5cm] at (-0.4,2) (A3) {Gesetz der Erhöhung der
      Abstimmung zwischen den Teilen des Systems};
    \node[box,text width=5cm] at (-0.4,3.7) (A4) {Gesetz der Leitfähigkeit: von
      Energie, Flüssen usw.};
    \node[box,text width=3cm] at (5,5) (A5) {Gesetz der ungleichmäßigen
      Entwicklung der Teile des Systems};
    \node[box] at (10,-.2) (A6) {Tendenz der Entwicklung auf einer S-Kurve};
    \node[box,text width=5cm] at (-0.4,0.4) (A7) {Tendenz der Verdrängung des
      Menschen aus dem TS};
    \node[box] at (10,4) (A8) {Gesetz der Erhöhung der Steuerbarkeit und der
      Stoff-Feld-Interaktionen};
    \node[box,text width=3cm] at (5,1) (A9) {Gesetz der Erhöhung der
      Dynamisierung};
    \node[box,text width=3cm] at (5,3) (A10) {Gesetz des Übergangs zum
      Obersystem};
    \node[box] at (10,6) (A11) {Gesetz des Übergangs von der Makro- zur
      Mikroebene};
    \node[box] at (10,1.8) (A12) {Tendenz der Entwicklung hin zum Zusammenlegen
      oder Teilen von Funktionen};

\draw[->] (A1) -- (2.8,7) |- (A2) ;
\draw[->] (2.8,7) |- (A3);
\draw[->] (2.8,7) |- (A4);
\draw[->] (2.8,7) |- (A5);
\draw[->] (2.8,7) |- (A7);
\draw[->] (2.8,7) |- (A9);
\draw[->] (2.8,7) |- (A10);
\draw[->] (A2) -- (-3.7,6) |- (A7);
\draw[->] (A5) -- (7.2,5) |- (A9);
\draw[->] (7.2,5) |- (A10);
\draw[->] (7.2,5) |- (A6);
\draw[->] (A1) -- (7.5,7) |- (A8) ;
\draw[->] (7.5,7) |- (A11) ;
\draw[->] (7.5,7) |- (A12) ;
\draw[->] (A8) -- (12.5,4) |- (A12) ;
  \end{tikzpicture}
\end{center}
\textbf{Abbildung 2:} Der Komplex der ZRTS unter Berücksichtigung der
bestehenden Unterscheidung zwischen Gesetzen und Trends

Es ist notwendig, auch das Gesetz der Vollständigkeit der Teile eines Systems
zu präzisieren -- es kann nicht reduziert werden nur auf den notwendigen Satz
von Teilen: Energiequelle, Motor, Getriebe, Arbeitsorgan, Steuerung. Dies gilt
nur für Maschinen und nicht für alle technischen Systeme. In allgemeinerer
Betrachtung muss es in diesem Gesetz um die Umsetzung des Wirkprinzips gehen.

Unter Berücksichtigung der Beseitigung dieser und anderer Mängel des
betrachteten ZRTS-Komplexes erhalten wir die Version der ZRTS-Hierarchie in
den Abbildungen 2 und 3, die als Arbeitsvariante zur Analyse der Entwicklung
technischer Systeme im Rahmen von ZRTS vorgeschlagen wird.

\begin{center}
  \begin{tikzpicture}[
      box/.style={draw,align=center,rounded corners=2pt,text width=5cm},
      >={Triangle[length=3pt 9, width=3pt 3]}]
    \node[box] at (0,-0.1) (A1) {Gesetz der ungleichmäßigen Entwicklung der
      Systemteile};
    \node[box] at (6,-0.1) (B1) {Modelle und Schablonen von Widersprüchen
      (technische, physikalische) };
    \node[box] at (0,1.2) (A2) {Gesetz der Erhöhung der Idealität};
    \node[box] at (6,1.2) (B2) {Ideales Endresultat (IER)};
    \node[box] at (0,2.4) (A3) {Gesetz der Leitfähigkeit: Energie, Flüsse
      usw.};
    \node[box] at (6,2.4) (B3) {Flussmodelle, Flussanalysis};
    \node[box] at (0,3.8) (A4) {Gesetz der Erhöhung der Steuerbarkeit und von
      Stoff-Feld-Interaktionen};
    \node[box] at (6,3.8) (B4) {Stoff-Feld-Interaktionen, Standardlösungen für
      Erfindungsaufgaben};
    \node[box] at (0,5.6) (A5) {Tendenz der Entwicklung durch Verschmelzen
      oder Trennen von Funktionen};
    \node[box] at (6,5.6) (B5) {Funktionsmodelle, Funktionsanalyse};
    \node[box] at (6,7) (B5a) {Linie der Mono-Bi-Poly-Verschmelzungen};
    \node[box] at (0,8) (A6) {Gesetz des Übergangs zum Obersystem};
    \node[box] at (6,8.5) (B6) {Systemoperator};
    \node[box] at (0,10) (A7) {Tendenz der Entwicklung längs einer S-Kurve};
    \node[box] at (6,10) (B7) {Etappen der Entwicklung technischer Systeme
      entsprehend einer S-Kurve der Systementwicklung};
    \node[box,text width=3cm] at (11,3.8) (C) {Algorithmus zum Lösen von
      Erfindungs\-aufgaben (ARIS)};

\draw[->] (A1) -- (B1) ;
\draw[->] (A2) -- (3,1.2) |- (B2) ;
\draw[->] (A3) -- (B3) ;
\draw[->] (A4) -- (B4) ;
\draw[->] (A4) |- (3,4.7) -| (B5) ;
\draw[->] (A5) -- (B5) ;
\draw[->] (B5a) -- (2,7) -| (A5) ;
\draw[->] (B5a) -- (B5) ;
\draw[->] (A6) -- (3,8) |- (B6) ;
\draw[->] (A6) -- (6,8) -- (B5a) ;
\draw[->] (A7) -- (B7) ;
\draw[->] (B1) -| (C) ;
\draw[->] (B2) -| (C) ;
\draw[->] (B3) -| (C) ;
\draw[->] (B4) -- (C) ;
\draw[->] (B5) -| (C) ;
\draw[->] (B6) -| (C) ;
  \end{tikzpicture}
\end{center}
\textbf{Abbildung 3:} Das Verhältnis der grundlegenden Werkzeuge zur Analyse
technischer Systeme und Lösungen erfinderischer Aufgaben mit dem Komplex ZRTS
(Fragment)

\section*{2. Über die Gesetze der Systementwicklung (ZRS)}
In der evolutionären Systemwissenschaft (Evolutionswissenschaft) steht die
Aufgabe der Ausweitung der TRIZ-Ansätze auf Entwicklungsprozesse und die
Lösung erfinderischer Probleme in beliebigen Systemen. Dazu wurde ein Komplex
universeller Gesetze der Systementwicklung (ZRS) formuliert, der hier als
Komplex von 4 Blöcken und 12 Gesetzen dargestellt wird.

\begin{center}
  \begin{tikzpicture}[scale=.85,transform shape,
      bigbox/.style={draw,rounded corners=4pt},
      box/.style={draw,align=center,rounded corners=4pt},
      >={Triangle[length=0pt 6,width=0pt 5]},
    rounded corners=2pt,line width=.8pt]
    \node at (-3.5,15.5) {Allgemeine Gesetze}; 
    \node[box,text width=6cm] at (-2,14) (A1) {1. Gesetz des Strebens des
      Systems nach Erhöhung des Niveaus und der Effektivität der Aneignung von
      Ressourcen} ;
    \node[box,text width=9cm] at (7,14) (A2) {2. Gesetz der
      Entstehung von Kräften, die Änderungen entgegenwirken und Erhaltung
      unterstützen, die Entwicklung bremsen und der Zerstörung des Systems
      entgegenwirken} ; 
    \node at (-2,12) {Wechselwirkung mit der Umgebung} ;
    \node[box,text width=6cm] at (-2,10.5) {3. Gesetz der Induktion
      (gegenseitigen Beeinflussung) von System und Umgebung} ;
    \node[box,text width=6cm] at (-2,8.7) {4. Gesetz des Übergangs
      zu Ober- und Teilsystemen} ;
    \node[box,text width=6cm] at (-2,7) {5. Gesetz der Ausbildung
      von Hierarchieebenen in der Organisation von Systemen} ; 
    \node[box,text width=6cm] at (-2,5) {6. Gesetz des Bestrebens
      des Systems unabhängig von der Umgebung zu werden} ;
    \node[text width=5cm,align=center] at (5,11.7) {Strukturentwicklung der
      Systeme} ;
    \node[box,text width=5cm] at (5,9.7) {7. Gesetz des Übergangs
      von Ressourcennutzung zu Selbstorganisationsformen und zu funktionalen
      Systemen};
    \node[box,text width=5cm] at (5,7.4) {8. Gesetz des Strebens
      nach idealen und funktionalen Systemen};
    \node[box,text width=5cm] at (5,5.1) {9. Gesetz der
      Entwicklung von Aneingungs-Mechanismen von starren zu flexiblen sowie
      von ständigen zu gesteuerten} ;
    \node at (10,12) {Statik} ;
    \node[box,text width=3.3cm] at (10,9.8) {10. Gesetz der Erhaltung
      der strukturellen Integrität und funktionalen Vollständigkeit} ;
    \node[box,text width=3.3cm,dashed] at (10,6) {Gesetze der Formierung
      von Kräften und Energie im Prozess der Systementwicklung} ;
    \node at (-1.5,2.5) {Gesetze der Auf\-lösung von Widersprüchen} ;
    \node[box,text width=16.5cm] at (3.3,1.5) {11. Gesetz der
      Entwicklung durch das Auftreten und Lösen von Widersprüchen in den
      Anforderungen} ;
    \node[box,text width=16.5cm] at (3.3,0) {12. Gesetz der Auf\-lösung
      von Widersprüchen im Raum, in der Zeit, durch Systemübergänge und in den
      Beziehungen} ;

    \draw[<->] (A1) -- (A2) ;
    \draw[bigbox] (-5.8,16) -- (12,16) -- (12,12.7) -- (-5.8,12.7) -- cycle  ;
    \draw[bigbox] (-5.8,12.5) -- (1.8,12.5) -- (1.8,3.5) -- (-5.8,3.5) --
    cycle  ; 
    \draw[bigbox] (2.2,12.5) -- (7.8,12.5) -- (7.8,3.5) -- (2.2,3.5) -- cycle  ;
    \draw[bigbox] (8,12.5) -- (12,12.5) -- (12,8) -- (8,8) -- cycle  ;
    \draw[bigbox] (-5.8,3) -- (12,3) -- (12,-1) -- (-5.8,-1) -- cycle  ;
    
  \end{tikzpicture}
\end{center}
\textbf{Abbildung 4:} Komplex der Gesetze der Systementwicklung
\medskip

Als die zwei grundlegenden werden das \emph{Gesetz der Inbesitznahme von
  Ressourcen} und das \emph{Gesetz der Trägheit des Systems} vorgeschlagen.
Der Kampf dieser beiden Gesetze (das Streben der Systeme nach Inbesitznahme
und die systemische Trägheit) sind die treibende Kraft der Entwicklung von
Systemen.

Da sich Systeme nicht isoliert von der äußeren Umgebung entwickeln können,
besteht der nächste Block von Gesetzen aus dem \emph{Gesetz der Induktion},
dem \emph{Gesetz des Übergangs zu Ober- und Untersystemen}, dem \emph{Gesetz
  der Herausbildung von Systemebenen} und dem \emph{Gesetz der Zunahme der
  Unabhängigkeit des Systems}.

In Interaktion mit der äußeren Umgebung verändert sich gesetzmäßig auch die
innere Struktur der Systeme, der Charakter ihrer Aktivitäten, und diese
Veränderungen werden im dritten Block von Gesetzen beschrieben: das
\emph{Gesetz der Selbstorganisation}, das \emph{Gesetz der Idealisierung} und
das \emph{Gesetz der zunehmenden Flexibilität}.  Als eigenständiger Block
stehen die \emph{Gesetze der Selbsterhaltung}, die derzeit nur durch das
\emph{Gesetz zur Aufrechterhaltung der Integrität und Vollständigkeit}
vertreten werden.

Der Prozess der Systementwicklung vollzieht sich in ständiger Auflösung von
Widersprüchen, mit denen das System konfrontiert ist, was die letzten beiden
Gesetze widerspiegeln: Das \emph{Gesetz der Entwicklung durch Auftreten und
  Auflösen von Widersprüchen} und das \emph{Gesetz der Auflösung von
  Widersprüchen auf vier Wegen}.

Die spezifischsten sind die Begriffe der Inbesitznahme, der Induktion und der
systemischen Trägheit, deren Verwendung erhebliche Unterschiede bewirken im
Vergleich zu den Entwicklungsgesetzen technischer Systeme, wie sie im Rahmen
der TRIZ eingesetzt werden. Unter \emph{Inbesitznahme} wird das Streben eines
Systems zur Aneignung externer (und möglicherweise auch interner) Ressourcen
verstanden, um was für Ressourcen es sich auch handelt. Unter \emph{Induktion}
wird der Einfluss der Umgebung (einschließlich anderer Systeme) auf das System
wie auch der umgekehrte Einfluss des Systems auf die Umwelt verstanden, und
die \emph{Systemische Trägheit} ist das Ergebnis der Selbstinduktion des
Systems, d.h. der Einfluss seiner Teile aufeinander, was zur Unmöglichkeit der
sofortigen Entwicklung des Systems auch in einem durchaus günstigen äußeren
Umfeld führt. Potenziell erlauben es diese drei Begriffe, systemweite
Parameter einzuführen: \emph{Systemenergie} als Maß der Bestrebungen des
Systems zur Inbesitznahme, dessen \emph{Passionarität} und \emph{systemische
  Masse} als Maß der internen Trägheit des Systems sowie ein Analogon der
\emph{Systemreibung} (oder vielleicht allgemeiner -- systemischer Kräfte),
welche die Wechselwirkung des Systems mit der äußeren Umgebung beschreiben.

\section*{3. Verknüpfung der Entwicklungsgesetze von Systemen,\\ ZRTS- und
  TRIZ-Tools}

Da ZRS allgemeiner als ZRTS sind, kann gezeigt werden, aus welchen Gesetzen
einer ZRS Gesetze einer ZRTS folgen. Ein Fragment derartiger Verbindungen ist
in Abb. 3 dargestellt. In Wirklichkeit gibt es natürlich mehr solche
Verknüpfungen und sie sind allgemeiner.

Aus Abb. 3 ist ersichtlich, dass es Gesetze zur Entwicklung technischer
Systeme gibt (zum Beispiel das Gesetz des Aufgebens terrestrischer
Bedingungen), die aus dem ZRS hervorgehen, aber derzeit im ZRTS nicht
verfügbar sind. Ein weiteres Merkmal dieses Schemas: Im ZRTS gibt es keinen
direkten Übergang von diesem oder jenem Gesetz zu solchen Werkzeugen zur
Lösung erfinderischer Probleme wie Methoden und Effekte. Im Komplex der ZRS
ist eine solche Beziehung leicht zu sehen. Das heißt, der Komplex ZRS ist
allgemeiner im Vergleich zu ZRTS, ist inhaltlich vollständiger und enthält
auch Gesetze, die bisher noch nicht in modernen Version der ZRTS enthalten
sind.

\section*{4. Zur Weiterentwicklung der Entwicklungsgesetze}
Die Entwicklung des Komplexes ZRS geht in verschiedene Richtungen:
\begin{itemize}
  \item Bestätigung und Präzisierung des gebildeten Komplexes ZRS.
  \item Präzisierung des Komplexes der Gesetze im ZRTS auf der Basis von ZRS. 
  \item Erstellung eines vollständigen Bildes der logischen Zusammenhänge
    zwischen den Gesetzen ZRS, ZRTS und den TRIZ-Werkzeugen.
  \item Konstruktion von Entwicklungsgesetzen in verschiedenen Bereichen der
    menschlichen Tätigkeit (in Business, Ökonomie, Informationssysteme, Kunst,
    Kultur usw.) auf der Grundlage des Komplexes ZRS.
\end{itemize}
Wir geben nur ein Beispiel für die mögliche Entwicklung von ZRTS, die auf ZRS
basieren. Im Komplex der Gesetze der Systementwicklung (ZRS) gibt es Gesetze,
für die es in ZRTS keine Analoga gibt.  Zum Beispiel gibt es für das Gesetz
der Induktion (gegenseitige Beeinflussung) von Systemen und ihrer äußeren
Umgebung Bestätigungen in Physik, Chemie, Biologie, Wirtschaft, Business usw.
Es ist logisch davon ausgehen, dass für die Gesetze der Entwicklung
technischer Systeme ein ähnliches Induktionsgesetz (gegenseitige Beeinflussung
des Systems und seiner äußeren Umgebung) geben sollte. Zum Beispiel begannen
sich unter dem Einfluss der Umwelt (der wissenschaftlichen und industriellen)
Kameras zu entwickeln.  Unter dem Einfluss von Kameras und ihren Möglichkeiten
begann sich das äußere Umfeld zu verändern (das wissenschaftliche,
industrielle, soziale, rechtliche ...). Beeinflusst von diesem neuen äußeren
Umfeld entwickeln sich Kameras weiter, sie erhalten neue Funktionen, neue
Möglichkeiten.

\section*{Danksagung}
Der Autor bedankt sich bei Misyuchenko I.L., Rubina N.V., Shchedrin N.A. und
anderen Kollegen, mit denen dieses Material vorbereitet und diskutiert wurde.

\section*{Liste der Referenzen (alle Texte in russischer Sprache)}
\begin{itemize}
\item[1.] G.S. Altshuller. Über die Gesetze der Entwicklung technischer
  Systeme. - Baku, 1977 (Manuskript).
\item[2.] G.S. Altshuller. Kreativität als exakte Wissenschaft - M.:
  Sowjetisches Radio, 1979.
\item[3.] A. Lubomirsky, S. Litvin. Gesetze zur Entwicklung technischer
  Systeme, GEN3 Partners, Februar 2003.
  \url{https://metodolog.ru/00767/00767.html}
\item[4.] M.S. Rubin. Zu den Gesetzen der Entwicklung technischer Systeme.
  Thesen eines Vortrags auf der wissenschaftlich-praktischen
  Allunions-Konferenz „Probleme der Entwicklung und Steigerung der Effizienz
  wissenschaftlicher und technischer Kreativität der
  Arbeiter“. (2.--4. Oktober 1979), Nowosibirsk.\\
  \url{http://www.temm.ru/ru/section.php?docId=3400}
\item[5.] M.S. Rubin. Über den Einfluss terrestrischer Bedingungen auf die
  technologische Entwicklung, Baku, 1980.
  \url{http://www.temm.ru/ru/section.php?docId=3420}
\item[6.] M.S. Rubin. Studien zu den Gesetzen der Technologieentwicklung,
  2006.\\  \url{http://www.temm.ru/en/section.php?docId=3432}
\item[7.] M.S. Rubin. Mythen über die Gesetze der Entwicklung technischer
  Systeme, 2009.\\  \url{http://www.temm.ru/ru/section.php?docId=4384}
\item[8.] M.S. Rubin. Studien zu evolutionären Systemwissenschaft.
  Evolutionswissenschaft.\\ TRIZ-Summit Sankt Petersburg 2015.\\
  \url{https://triz-summit.ru/confer/tds-2015/paper/science/300497/}
\item[9.] M.S. Rubin, I.L. Misyuchenko, N.V. Rubina. „Forschen in der TRIZ.
  Evolutionswissenschaft und die Gesetze der Systementwicklung.  Präsentation
  auf dem Workshops des TRIZ-Summit 2018.\\
  \url{https://triz-summit.ru/file.php/id/f303809-file-original.pdf}
\item[10.] I.L. Misyuchenko. Systemweite Gesetze der Entwicklung und
  Entwicklung der Physik der Mikrowelt.  TRIZ Developer Summit.  Sankt
  Petersburg 2018.\\
  \url{https://triz-summit.ru/file.php/id/f303772-file-original.pdf}
\end{itemize}
\end{document}

Abbildung 4: Komplex von Gesetzen der Entwicklung von Systemen

\begin{quote}
  Allgemeine Gesetze:
  \begin{itemize}
  \item [1.] Gesetz des Strebens des Systems zur Erhöhung des Niveaus und der
    Effektivität der Aneignung von Ressourcen.
  \item [2.] Gesetz der Entstehung von Kräften, die Änderungen entgegenwirken
    und Erhaltung unterstützen, die Entwicklung bremsen und der Zerstörung des
    Systems entgegenwirken.
  \end{itemize}
  Wechselwirkung mit der Umgebung:
  \begin{itemize}
  \item [3.] Gesetz der Induktion (gegenseitigen Beeinflussung) von System und
    Umgebung.
  \item [4.] Gesetz des Übergangs zu Ober- und Teilsystemen.
  \item [5.] Gesetz der Ausbildung von Hierarchieebenen in der Organisation
    von Systemen.
  \item [6.] Gesetz des Bestrebens des Systems unabhängig von der Umgebung zu
    werden.
  \end{itemize}
  Strukturentwicklung der Systeme:
  \begin{itemize}
  \item [7.] Gesetz des Übergangs von Ressourcennutzung zu
    Selbstorganisationsformen und zu funktionalen Systemen.
  \item [8.] Gesetz des Strebens nach idealen und funktionalen Systemen.
  \item [9.] Gesetz der Entwicklung von Aneingungs-Mechanismen von starren zu
    flexiblen sowie von ständigen zu gesteuerten.
  \end{itemize}
  Statik:
  \begin{itemize}
  \item [10.] Gesetz der Erhaltung der strukturellen Integrität und funktionalen
  Vollständigkeit. 
  \end{itemize}
  Gesetze der Formierung von Kräften und Energie im Prozess der
  Systementwicklung\footnote{Im Original gestrichelt und nicht weiter
    ausgeführt}.\medskip

  Gesetze der Auflösung von Widersprüchen
  \begin{itemize}
  \item [11.] Gesetz der Entwicklung durch das Auftreten und Lösen von
    Widersprüchen in den Anforderungen.
  \item [12.] Gesetz der Auflösung von Widersprüchen im Raum, in der Zeit,
    durch Systemübergänge und in den Beziehungen.

    
\end{quote}

