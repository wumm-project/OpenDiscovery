\documentclass[11pt,a4paper]{article}
\usepackage{od}
\usepackage[utf8]{inputenc}
\usepackage[russian,ngerman]{babel}

\title{Zum Zusammenhang der Gesetzgebung zur Entwicklung von Systemen mit ZRTS}

\author{M.S.Rubin, Moskau, Russland}
\date{Version vom 6. November 2019}

\begin{document}
\maketitle
\begin{quote}
  Original: \foreignlanguage{russian}{О связи комплекса законов развития
    систем с ЗРТС}.  Vorabversion. Übersetzt von Hans-Gert Gräbe, Leipzig.

  ZRTS steht für „Gesetze der Entwicklung technischer Systeme“. 
\end{quote}

\begin{abstract}
  Eine präzisierte Fassung des Komplexes der Gesetze der Entwicklung
  technischer Systeme (ZRTS) wird vorgeschlagen.  Ein Komplex universeller
  Gesetzen der Entwicklung von Systemen (ZRS) für beliebige Systeme --
  materiell oder immateriell -- wird ausgearbeitet. Es wird die Beziehung von
  ZRS zu ZRTS aufgezeigt sowie die Beziehung von ZRTS zu Werkzeugen zur
  Analyse der Entwicklung technischer Systeme und der Lösungen von
  Erfindungsaufgaben. Es werden Möglichkeiten der Entwicklung von ZRTS auf der
  Basis von ZRS aufgezeigt.

  \emph{Schlüsselwörter:} TRIZ; Entwicklungsgesetze technischer Systeme
  (ZRTS), Entwicklungsgesetze von Systemen (ZRS), Evolutionäre
  Systemwissenschaft
\end{abstract}

\section*{1. Zu Gesetzen der Entwicklung technischer Systeme (ZRTS)}

Die Idee, Gesetze der Entwicklung von Maschinen abzuleiten, wurde bereits 1946
von Rafael Shapiro geäußert. Im Jahr 1977 formulierte G.S. Altschuller die
folgenden Gesetze der Entwicklung technischer Systeme (ZRTS):
\begin{itemize}
\item[1.] Das Gesetz der Vollständigkeit der Teile des Systems. Notwendige
  Voraussetzung der Funktionsfähigkeit eines technischen Systems ist die
  Verfügbarkeit und minimale Funktionsfähigkeit der Hauptteile des Systems.
\item[2.] Das Gesetz der „Energieleitfähigkeit“ des Systems. Notwendige
  Voraussetzung der grundlegenden Lebensfähigkeit eines technischen Systems
  ist der Energiedurchsatz durch alle Teile des Systems.
\item[3.] Das Gesetz der Harmonisierung der Rhythmik der Teile des Systems.
  Notwendige Voraussetzung grundlegenden Lebensfähigkeit eines technischen
  Systems ist die Resonanz (oder bewusste Dissonanz) der Schwingungsfrequenzen
  (der Betriebsfrequenzen) aller Teile des Systems.
\item[4.] Das Gesetz der Erhöhung des Idealitätsgrades des Systems. Die
  Entwicklung aller Systeme geht in Richtung der Erhöhung des Grads der
  Idealität.
\item[5.] Das Gesetz der ungleichmäßigen Entwicklung der Teile des Systems.
  Die Entwicklung der Teile des Systems erfolgt ungleichmäßig: Je komplexer
  das System ist, desto ungleichmäßiger ist die Entwicklung seiner Teile.
\item[6.] Das Gesetz des Übergangs zum Obersystem. Die Entwicklung eines
  Systems, das an seine Grenzen stößt, setzt sich auf der Ebene des
  Obersystems fort.
\item[7.] Das Gesetz der Dynamisierung technischer Systeme. Starre Systeme
  müssen dynamisch werden, um ihre Effizienz zu verbessern, d.h. müssen zu
  einer flexibleren, sich schnell ändernden Struktur übergehen und zu einem
  Betriebsregime, das sich an die Veränderungen der äußeren Umgebung anpasst.
\item[8.] Das Gesetz des Übergangs von der Makroebene zur Mikroebene. Die
  Entwicklung der Arbeitsorgane erfolgt zuerst auf der Makro- und dann auf der
  Mikroebene.
\item[9.] Das Gesetz der Erhöhung der Stoff-Feld-Interaktionen. Die
  Entwicklung technischer Systeme geht in die Richtung der Erhöhung der
  Stoff-Feld-Interaktionen: Systeme mit geringem Interaktionsgrad streben
  danach, diesen Interaktionsgrad zu erhöhen, und Systemen mit hohem
  Interaktionsgrad entwickeln sich in Richtung der Erhöhung der Anzahl der
  Verbindungen zwischen Elementen, der Erhöhung der Reaktionsfähigkeit
  (Empfindlichkeit) der Elemente, der Erhöhung der Anzahl der Elemente.
\end{itemize}
Ausgehend von diesem System von Gesetzen wurde die ARIZ-Methodik als TRIZ
bekannt -- Theorie des Lösens erfinderischer Probleme.

A. Lyubomirsky und S. Litvin schlugen eine eigene Hierarchie der
Entwicklungsgesetze technischer Systeme vor. Ihr Hauptmerkmal ist, dass sie
für die Durchführung von Analysen technischer Systeme nach den Gesetzen der
Entwicklung bequemer ist, hat aber seine Nachteile, die unten beschrieben
sind.
\begin{quote}
  HGG: An dieser Stelle ist noch ein Diagramm einzubauen.
\end{quote}
Bevor wir die Mängel beschreiben, geben wir zwei Definitionen aus
enzyklopädischen Wörterbüchern.

Als \emph{Gesetz} bezeichnet man eine notwendige, substanzielle, nachhaltige,
wiederkehrende Beziehung zwischen Phänomenen in Natur und Gesellschaft. Der
Begriff des Gesetzes ist mit dem Begriff des Wesens verwandt.

Als \emph{Trend} -- ein Anglizismus -- wird die Haupttendenz einer Veränderung
von etwas bezeichnet: Zum Beispiel in der Mathematik die Zeitreihe.

Der grundsätzliche Unterschied zwischen Gesetzen und Trends der Entwicklung
ist offensichtlich, die Begriffe sind klar auseinanderzuhalten. Daher kann man
zum Beispiel in das System von Gesetzen die Einführung eines \emph{Trends der
  Evolution längs einer S-Kurve} nicht als korrekt betrachten.
G.S. Altschuller beschrieb den Trend der Entwicklung technischer Systeme längs
einer S-Kurve, gab ihm aber keinen Status eines Gesetzes. Der „Trend der
Evolution längs einer S-Kurve“ ist also kein Gesetz, sondern eine
Entwicklungstendenz.

Koagulation ist auch kein Gesetz, sondern nur eine der Richtungen zur Erhöhung
der Idealität.

Es ist notwendig, das Recht der Vollständigkeit von Teilen des Systems zu
präzisieren - es kann nicht reduziert werden nur auf den notwendigen
Teilesatz: Energiequelle, Motor, Getriebe, Funktion Aufbau, Steuerung Dies
gilt nur für Maschinen und nicht für alle technischen Systeme. In Eine
breitere Sichtweise in diesem Gesetz sollte die Umsetzung des Aktionsprinzips
betreffen.

Unter Berücksichtigung der Beseitigung dieser und anderer Mängel des
betrachteten ZRTS-Komplexes haben wir Wir erhalten die folgende Version der
Hierarchie ZRTS, die als Arbeit vorgeschlagen wird Möglichkeiten zur Analyse
der Entwicklung technischer Systeme für ZRTS.

\begin{quote}
  Abb. 1. Das komplexe ZRTS unter Berücksichtigung der bestehenden
  Unterschiede zwischen Gesetzen und Trends. HGG: zu ergänzen
\end{quote}
\begin{quote}
  Abb. 2. Das Verhältnis der grundlegenden Werkzeuge zur Analyse technischer
  Systeme und Lösungen erfinderischer Aufgaben mit einem Komplex von ZRTS
  (Fragment). HGG: zu ergänzen
\end{quote}

\section*{2. Über die Gesetze der Systementwicklung (ZRS)}
In der evolutionären Systemologie (Evolutionsstudien) die Aufgabe Ausweitung
der TRIZ-Ansätze auf Entwicklungsprozesse und Lösung erfinderischer Probleme
in beliebigen Systemen. Zu diesem Zweck wurde eine Reihe universeller Gesetze
gebildet Systementwicklung (ZRS). In Abb. 3 zeigt eine Reihe von Gesetzen für
die Entwicklung von Systemen, bestehend aus 4 Blöcke und 12 Gesetze.
\begin{quote}
  Abb. 3. Der Satz von Gesetzen für die Entwicklung von Systemen. HGG: zu
  ergänzen. 
\end{quote}
Das Gesetz der Ressourcenerfassung und das Gesetz des Systems Trägheit. Der
Kampf dieser beiden Gesetze (der Wunsch der Systeme, systemische Trägheit
einzufangen) sind die treibende Kraft hinter der Entwicklung von Systemen.

Da sich Systeme also nicht isoliert von der externen Umgebung entwickeln
können der nächste Block von Gesetzen ist das Gesetz der Induktion, das Gesetz
des Übergangs zu Supersystemen und Subsysteme, das Gesetz der Bildung von
Systemebenen und das Gesetz der Zunahme Systemunabhängigkeit.

In Interaktion mit der äußeren Umgebung, der inneren Die Struktur der Systeme,
die Art ihrer Aktivitäten und diese Veränderungen werden im dritten Abschnitt
beschrieben Gesetzesblock: Gesetz der Selbstorganisation, Gesetz der
Idealisierung und Gesetz der Zunahme Flexibilität. Die bisher vorgestellten
Selbsterhaltungsgesetze bilden einen eigenen Block das einzige Gesetz zur
Aufrechterhaltung der Integrität und Vollständigkeit.

Der Prozess der Systementwicklung vollzieht sich in der ständigen Auflösung
von Widersprüchen mit was ihr begegnet, was sich in den letzten beiden
Gesetzen widerspiegelt: Das Gesetz der Entwicklung durch Auftreten und
Auflösen von Widersprüchen und Widerspruchsauflösungsgesetz 4 Wege.

Die spezifischsten sind die Konzepte der Erfassung, Induktion und systemischen
Trägheit, deren Verwendung unterscheidet sich erheblich von den
Entwicklungsgesetzen technische Systeme, die im Rahmen der TRIZ eingesetzt
werden. Capture bezieht sich auf das Verlangen Systeme für die Entwicklung
externer (und möglicherweise auch interner) Ressourcen, was würden diese
Ressourcen waren nicht. Mit Induktion ist die Auswirkung seiner Umgebung auf
das System gemeint Existenz (einschließlich anderer Systeme) sowie die
umgekehrte Auswirkung des Systems auf die Umwelt und Systemische Trägheit ist
das Ergebnis der Selbstinduktion des Systems, d.h. der Einfluss seiner Teile
gegeneinander und führt zu der Unmöglichkeit der sofortigen Entwicklung des
Systems auch bei einem durchaus günstigen äußeren umfeld. Potenziell diese
drei Konzepte Ermöglichen die Eingabe systemweiter Parameter: "Systemenergie"
als Maß Bestrebungen des Systems, dessen "Antrieb" und "systemische Masse" als
Maß zu erfassen interne Trägheit des Systems sowie ein Analogon der
"Systemreibung" (oder vielleicht allgemeiner - systemische Kräfte), die die
Wechselwirkung des Systems mit der äußeren Umgebung beschreiben.

\section*{3. Verknüpfung von Entwicklungsgesetzen von Systemen, ZRTS- und
  TRIZ-Tools}

Aufgrund der Tatsache, dass Luftverteidigungssysteme allgemeiner in Bezug auf
Luftverteidigungsraketensysteme sind, ist dies möglich zeigen, welche Gesetze
des Luftverteidigungssystems von den Gesetzen des Luftverteidigungssystems
herrühren. Ein Fragment des Aufbaus eines solchen Die Verknüpfungen sind in
Abb. 4 dargestellt. In Wirklichkeit gibt es natürlich mehr Verknüpfungen und
es gibt mehr Verknüpfungen gemeinsam.
\begin{quote}
  Abb. 4. Das Verhältnis der Gesetze der Systementwicklung (SAM) zu ZRTS
  (Fragment). HGG: zu ergänzen.
\end{quote}
Abb. 4 zeigt, dass es Gesetze zur Entwicklung technischer Systeme gibt (zum
Beispiel das Gesetz Abweichung von den terrestrischen Bedingungen), die aus
dem Luftverteidigungssystem hervorgehen, aber derzeit nicht verfügbar sind in
ZRTS. Ein weiteres Merkmal dieses Schemas. In ZRTS gibt es keinen direkten
Übergang von diesem oder jenem Gesetz zu solchen Werkzeugen zur Lösung
erfinderischer Probleme wie Techniken und Wirkungen. In komplexe ZRS eine
solche Beziehung ist leicht sichtbar. Das heißt, das komplexe SAM ist
allgemeiner Im Vergleich zu ZRTS ist es inhaltlich vollständiger und enthält
auch Gesetze, die bisher gelten noch nicht in der modernen Version von ZRTS
reflektiert.

\section*{4. Zur Weiterentwicklung der Entwicklungsgesetze}
Die Entwicklung des komplexen SAM wird in verschiedene Richtungen gehen:
\begin{itemize}
  \item Bestätigung und Klärung des gebildeten Komplexes SAM
  \item Klärung des auf ZRS basierenden Gesetzeskomplexes von ZRTS
  \item die Erstellung eines vollständigen Bildes der logischen Zusammenhänge
    zwischen Luftverteidigungssystemen, Luftverteidigungssystemen und
    Werkzeugen TRIZ
  \item die Konstruktion von Entwicklungsgesetzen in verschiedenen Bereichen
    der menschlichen Tätigkeit (Wirtschaft, Wirtschaft, Informationssysteme,
    Kunst, Kultur usw.) auf der Grundlage des Komplexes ZRS.
\end{itemize}
Wir geben nur ein Beispiel für die mögliche Entwicklung von
Luftverteidigungssystemen, die auf Luftverteidigungssystemen basieren. In Der
Komplex der Gesetze zur Entwicklung von Systemen (ZRS) hat Gesetze, für die es
in ZRTS keine Analoga gibt.  Zum Beispiel für das Induktionsgesetz
(gegenseitige Beeinflussung) von Systemen und ihrer äußeren Umgebung gibt es
Bestätigungen in Physik, Chemie, Biologie, Wirtschaft, Business usw. Ist
logisch davon ausgehen, dass für die Gesetze der Entwicklung von technischen
Systemen ähnlich sein sollte das Induktionsgesetz (gegenseitige Beeinflussung
des Systems und seiner äußeren Umgebung). Zum Beispiel unter dem Einfluss von
Kameras (wissenschaftliche und industrielle) entstanden und begannen, Kameras
zu entwickeln.  Unter dem Einfluss von Kameras und ihren Fähigkeiten begann
sich das äußere Umfeld zu verändern.  (wissenschaftliche, industrielle,
soziale, rechtliche ...). Beeinflusst von diesem neuen Äußeren Kameras ändern
sich weiter, sie haben neue Funktionen, neue Chancen.

\section*{Danksagung}
Der Autor ist Misyuchenko I.L., Rubina N.V., Shchedrin N.A. und andere an
Kollegen, mit denen dieses Material vorbereitet und diskutiert wurde.

\section*{Referenzliste}
\begin{itemize}
\item[1.] Altshuller G. Über die Gesetze der Entwicklung technischer
  Systeme. - Baku, 1977, 15 p.  (Manuskript).
\item[2.] Altshuller G.S. Kreativität als exakte Wissenschaft - M .:
  Sowjetisches Radio. Jahr Ausgaben, 1979 - Kybernetik.
\item[3.] A. Lubomirsky, S. Litvin, Gesetze zur Entwicklung technischer
  Systeme, GEN3-Partner Februar 2003,
  \url{https://metodolog.ru/00767/00767.html} 
\item[4.] Rubin Zu den Gesetzen der Entwicklung technischer
  Systemzusammenfassungen bei der All-Union wissenschaftlich-praktische
  Konferenz „Probleme der Entwicklung und Steigerung der Effizienz
  wissenschaftliche und technische Kreativität der Arbeiter“. (2.-4. Oktober
  1979), Nowosibirsk, \url{http://www.temm.ru/ru/section.php?docId=3400}
\item[5.] Rubin MS, Über den Einfluss terrestrischer Bedingungen auf die
  technologische Entwicklung, Baku, 1980.
  \url{http://www.temm.ru/ru/section.php?docId=3420}
\item[6.] Rubin MS, Studien zu den Gesetzen der Technologieentwicklung, 2006.
  \url{http://www.temm.ru/en/section.php?docId=3432}
\item[7.] Rubin MS, Mythen über die Gesetze der Entwicklung technischer
  Systeme. 2009.  \url{http://www.temm.ru/ru/section.php?docId=4384}
\item[8.] Rubin M.S. Studien zu evolutionären
  Systemstudien. Evolutionsstudien. TRIZ-Summit Sankt Petersburg 2015.
  \url{https://triz-summit.ru/confer/tds-2015/paper/science/300497/}
\item[9.] Rubin M. S., Misyuchenko I. L., Rubina N. V. „Forschen in der TRIZ.
  Evolution und die Gesetze der Systementwicklung. Vorstellung des
  Workshops. TRIZ-Summit 2018.
  \url{https://triz-summit.ru/file.php/id/f303809-file-original.pdf}
\item[10.] Misyuchenko I. Systemweite Gesetze der Entwicklung und Entwicklung
  der Physik der Mikrowelt.  TRIZ Developer Summit.  Sankt Petersburg 2018.
  \url{https://triz-summit.ru/file.php/id/f303772-file-original.pdf}
\end{itemize}
\end{document}
