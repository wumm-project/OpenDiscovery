\documentclass[11pt,a4paper]{article}
\usepackage{a4wide,url,graphicx,wrapfig}
\usepackage[utf8]{inputenc}
\usepackage[main=german,russian]{babel}

\parindent0pt
\parskip3pt

\title{TRIZ-Anwendung beim Architekturentwurf eines\\ Steuerungssystems auf
  der Grundlage von\\ Daten in der Energietechnik}

\author{A.P. Tjukow, Staatliche Technische Universität Wolgograd, Russland}

\date{TRIZ-Summit 2020}

\begin{document}
\maketitle

\begin{quote}
  Der Aufsatz wurde auf dem TRIZ-Summit 2020 präsentiert. Das Original ist
  unter
  \url{https://r1.nubex.ru/s828-c8b/f3137_9d/Tyukov-TDS-2020.pdf}
  zu finden.

  Übersetzung ins Deutsche von Hans-Gert Gräbe, Leipzig.
\end{quote}

\begin{abstract}
  In diesem Aufsatz werden die Erfahrungen mit der Visualisierung der
  Architektur eines Steuerungssystems auf der Grundlage von Energiedaten
  (SUND) und der Strategie seiner weiteren Entwicklung unter Anwendung von
  TRIZ-Instrumenten beschrieben: Funktionsanalyse, Ursache-Wirkungs-Analyse
  CECA, Suche nach Unzulänglichkeiten und Methoden zu deren Beseitigung.  SUND
  dient der Verwaltung von Informationsflüssen in der Energiewirtschaft:
  Heizungsmanagement auf der Basis von Wettervorhersagen, Lademanagement von
  Elektrofahrzeugen, Management von Energieflüssen in verteilten
  Energiesystemen, technisch-ökonomische Machbarkeitsstudien. Bei der Analyse
  und Beschreibung der neuen Systemarchitektur wird mit Notationen aus den
  Bereichen UML, BPMN, Archimate, Datenflussdiagramme (DFD) in der
  Systemmodellierungsumgebung \emph{Visual Paradigm} unter Verwendung der
  TOGAF-Methodologie gearbeitet.  Der Autor stellt eine Hypothese über
  verallgemeinerte Mängel und Methoden zu deren Beseitigung bei der Bildung
  von datenbasierten Steuerungssystemarchitekturen auf.

  \emph{Schlüsselworte:} Informationssysteme, Architekturen, TRIZ,
  datengetriebene Steuerung, System Engineering.
\end{abstract}


\end{document}
