\documentclass[12pt,a4paper]{article}
\usepackage{od}
\usepackage[utf8]{inputenc}
\usepackage[main=english,russian]{babel}
\setlist{noitemsep}
\usetikzlibrary{arrows.meta, positioning}
\usetikzlibrary{shapes.geometric, snakes}

\title{\raggedright ProHEAL – Social Needs and Sustainability Aspects in the
  Methodology of 
  the GDR Inventor Schools}  

\author{Hans-Gert Gräbe, Leipzig, Rainer Thiel, Storkow}
\def\theauthor{Hans-Gert Gräbe, Rainer Thiel}
\date{15.\,08.\,2021}
\begin{document}
\maketitle

\begin{abstract}
  This paper presents the basic conceptual elements of ProHEAL, a version of
  TRIZ developed in the 1980s within the framework of the GDR inventor
  schools. It elaborated already at that time on the embedding of technical
  solutions in the organisation of production and identified the resulting
  technical-economic contradictions in a much more structured way than in the
  ARIZ-85C variant still used in the TRIZ mainstream. In addition to a
  presentation of the conceptual approach, the explanations are also relevant
  for a history of TRIZ ideas. The developments came to an abrupt end in 1990
  after German unification and are widely unknown in the TRIZ community.
  \bigskip
  
  \textbf{Keywords:} ProHEAL, GDR inventor schools, TRIZ history of ideas.
\end{abstract}

\section*{1 The GDR Inventor Schools}

The GDR inventor schools in the 1980s mark an important early independent TRIZ
development outside the Russian-speaking community. The specific conditions in
the GDR at the beginning of the 1980s were characterised by a greater economic
autonomy of the big plants (combinats) at the one side and growing overall
economic problems on the other, which had accumulated during 15 years of
Honecker's “unity of economic and social policy” \cite{6,7}. Thus
technical-economic aspects already played an important role in that TRIZ
variation. Such aspects were taken up only 20 years later with “Business TRIZ”
in the TRIZ mainstream.

Unfortunately, this heritage is largely unknown in the international TRIZ
community, certainly also because relevant materials \cite{13,14} have so far
only been available in German and have not been translated into one of the two
leading TRIZ languages, Russian and English.

Within the WUMM project \cite{16} we started to digitise historical materials
from this development and make them publicly available, initially at least in
scanned form \cite{17}. The 90th birthday of Rainer Thiel in September 2020
was an occasion to reprint the KdT teaching material \cite{13} from 1989 in an
annotated version \cite{15}. It systematises the experiences of that time in a
detailed algorithmic variant ProHEAL (a German abbreviation for \emph{Program
  for the elaboration of inventive tasks and solution approaches} – in short:
the invention program), which is on par with ARIZ-85C in its level of detail.

Within the space restrictions of this publication, we present the central
ProHEAL-specific concepts – the path model, the decision model and the
versions of the ABER matrix on the three problem field levels. In an appendix
we reproduce English translations of the ProHEAL thinking field structure, an
uncommented verbal and graphical representation of the ProHEAL algorithm, and
the ProHEAL decision tree.

The explanations in sections 3 and 4 follow \cite[part 2, ch. 4]{14}, the
detailed algorithmic presentation of the ProHEAL path model in the appendix is
taken from \cite{13} and \cite{15}.  \cite[ch. 5]{15} contains a detailed
explanations of the 13 steps of the path model, which are also available in
English translation with \cite{8}.

\section*{2 ProHEAL Basics – a Short Overview}

The starting point of the TRIZ influence on ProHEAL were the German
translation of three of Altshuller’s publications \cite{1,2,3}. These ideas
fell on fertile ground, prepared on the one hand by the structure of
\emph{Honorable Inventors} existing since 1950 and on the other hand by the
\emph{Systematic Heuristics} of Johannes Müller \cite{10,11}. The latter
experienced a short but intensive institutional boom in the early 1970s with
lasting influence on a whole generation of engineers \cite{9}. Details of the
organisational unfolding of the inventor schools are presented in more detail
in \cite{6,7} and \cite{14}.

Due to the specific scope of application in socio-economic practices of large
production units (combinates), ProHEAL differs significantly in some
approaches from TRIZ in Altshuller’s variant available at that time.

This refers \textbf{firstly} to the more detailed elaboration of
\emph{technical-economic contradictions} between social needs and
technological possibilities. Although Altshuller is also aware of
administrative contradictions, they are not seriously addressed in his work.

\textbf{Secondly}, ProHEAL early abandoned a monofunctional orientation, which
still plays a central role in the TRIZ system concept as MPV (for example
\cite{12}). In ProHEAL, value determinations are recorded under different
aspects as evaluation figure\footnote{In German “Zielgröße” but here the
  notion “evaluation figure” is used to emphasise its multidimensional
  structure.} at all levels of detail in the ABER matrices. Thus
contradictions in the problem description are already identified during
requirements elicitation. Such concepts of \emph{generalised contradictions},
that are formed from a larger number of action and evaluation parameters,
nowadays is elaborated in the IDM approach \cite{5} in more detail.

\textbf{Thirdly}, in addition to solving a contradictory problem situation,
the transfer of the solution into production also plays an important role in
ProHEAL. Thus, even the solutions of contradictions on levels 2 and 3 are
being returned to level 1 in node E2 to decide about the transfer to
production, see the figure in the appendix.

As in TRIZ, the \emph{ProHEAL Path Model} distinguishes three levels of
contradiction. On the first level, a \emph{basic variant} of the system as
required is developed from the technical-economic requirements, and the
(external) \emph{technical-economic contradiction} (TEC) is identified. This
contradiction can either already be solved at this level or a \emph{critical
  functional area} of the basic variant is identified in which the problems
caused by the TEC are concentrated. On the second level, the ideal technical
subsystem for the \emph{core variant} and the harmful technical effects are
detailed. They meet in this area in the (internal)
\emph{technical-technological contradiction} (TTC). This too can either
already be solved or the \emph{critical operational area} of the core variant
is identified. There a deeper \emph{technical-scientific contradiction} (TSC)
does manifest itself. Finally, on the third level, the ideal natural process
of the core variant and its harmful effects are opposed to each other in the
critical operational area.

Solutions on the second level often lead to unexpected low-tech inventions
that are easy to implement in production – \emph{surprisingly simple
  solutions} (SSS) or a \emph{surprising impact} (SI). Solutions on the third
level often lead to high-tech inventions. They have to be verified more
comprehensively before transferring them to production. If no solution is
found on the third level either, serious scientific research is required that
goes beyond the possibilities of a company innovation project. The agenda to
be worked on (C6-C9 in the appendix) lies outside of ProHEAL.

\section*{3 The Problem Field Levels in the\\ ProHEAL Path Model}

\subsection*{3.1 The Technical-Economic Problem Field Level}

At this first level, all problem-determining facts come into consideration
that relate the social need as potential need for a solution and the status of
technology as a system of available technical products and processes as
potential solution.

The consideration is person- and process-related and determined by the
product-goods-purpose relationship.  Results at this problem field level are
\begin{itemize}
\item the \emph{technical-economic objectives} of an innovation project,
\item the \emph{basic variant} of a process or product innovation that is
  tailored to the technological requirements,
\item the \emph{critical functional area} in the multi-dimensional
  optimisation behavior of this basic variant,
\item the \emph{TEC} that prevents an optimal design and tailoring of the
  basic variant.
\end{itemize}

If the basic variant cannot be optimised in terms of the technical-economic
objectives, we are faced with an inventive problem that has to be analysed at
the next level on which the solution of the TEC is the goal of the invention.

\subsection*{3.2 The Technical-Technological Problem Field Level}

At this next level, all the facts are considered that affect the technical
system of the basic variant, its structure, function, its behavior and its
immediate technological environment.

The consideration is object- and function-related and determined by the
technical means-action-counteraction relations.  Results at this problem field
level are
\begin{itemize}
\item the \emph{ideal technical subsystem} as an ideal solution of the TEC in
  the critical functional area of the basic variant,
\item the \emph{undesired effects} as not intended, technically
  disadvantageous influence of the ideal subsystem on the functional behavior
  of the basic variant,
\item the \emph{critical operational area} in the functional structure where
  the causal interdependence of the ideal subsystem and the undesired effects
  are located,
\item the \emph{TTC}, that prevents to eliminate or suppress the undesired
  effect by varying the parameters of the functional principle of the ideal
  subsystem.
\end{itemize}

If a technical subsystem with an \emph{alternative} functional principle in
the critical functional area of the basic variant can be found without causing
significant undesired side effects, then we obtained an invention as a
solution to the TEC. Due to the heuristic approach, this often turns out to be
located in the low-tech area, as a \emph{surprisingly simple solution}.  In
the best case it only requires a technical trial run before productive
roll-out.

If the solution at this problem field level is not achieved, the problem
situation has to be formulated as inventive task that contains the TTC as well
as a solution strategy tailored to this contradiction. The goal is to
determine the harmful natural law effects in the critical operational area of
the functional structure and to replace them with an alternative, known
operating principle, at the third problem field level.

\subsection*{3.3 The Technical-Scientific Problem Field Level}

At this third level, all facts come into consideration that concern the
operating principle of the basic variant, the requirements for its technical
use as well as its theoretical and experimental basics.

The consideration is model- and event-related and determined by the
field-factor-effect relationships.  Results at this problem field level are
\begin{itemize}
\item the \emph{ideal operating principle} that solves the TTC in the critical
  operational area of the functional structure,
\item the \emph{harmful natural law effects} which prevent the technical
  deployment of the ideal operating principle,
\item \emph{new technical-constructive boundary conditions} in the critical
  operational area, which suppress the harmful natural law effects,
\item the \emph{TSC}, which prevents the deployment of the ideal operating
  principle by varying the technical-constructive boundary conditions in the
  critical operational area.
\end{itemize}

If the new operating principle can be technically unfolded in the necessary
dimensions to ensure the fulfillment of the function in the critical range, we
are faced with an invention as a solution to the TTC. Since this enters new
technical-scientific territory, the solution is usually in the high-tech
area. It requires further application-oriented fundamental research for its
verification.

If a solution to the problem cannot be found in this way, we are faced with a
\emph{system-immanent TSC}, that questions the development and viability of
the system as a whole. The solution strategy then requires to search for a
suitable, so far unknown operating principle or a fundamental process
innovation. Both problem solving strategies usually go beyond the scope of a
timely and financially definable innovation project. They were therefore not
subject to further methodological considerations in ProHEAL, since they could
not be based on corrections of the existing process and a corresponding new
solution for the basic variant.

\section*{4 The ABER Matrices as a Strategic Tool\\ in the Invention
  Methodology}

The ProHEAL invention methodology proposes a set of methodological
instruments, which includes three categories of tools and techniques:
\begin{itemize}
\item \emph{Strategic tools} for goal and path planning, for working out the
  problem-determining contradiction at each level and to find solution
  strategies to overcome such a contradiction. 
\item \emph{Tactical tools} for the procurement and processing of information,
  for the generation of solution variants and their evaluation according to
  given solution strategies.
\item \emph{Creativity techniques} to activate and strengthen intuition,
  imagination, fluency in thinking and the ability to abstract, associate and
  for lateral thinking.
\end{itemize}

The strategic tools differ at the three problem field levels and have
inventive method specifics.  The tactical tools and creativity techniques do
not have inventive methodical specifics and can be used at all three problem
field levels in a similar way. The choice is determined solely by the
heuristic specifics of the respective working situation and the activities
related to the situation.

\subsection*{4.1 The Evaluation Matrix (ABER(1) Matrix)\\ at Problem Field
  Level 1}

It is used to systematically record the goal-determining
\begin{itemize}
\item \emph{Requirements} (\textbf{A}nforderungen),
\item \emph{Conditions} (\textbf{B}edingungen),
\item \emph{Expectations} (\textbf{E}rwartungen) and
\item \emph{Restrictions} (\textbf{R}estriktionen)
\end{itemize}
related to
\begin{itemize}
\item \emph{Functionality},
\item \emph{Profitability},
\item \emph{Controllability} and
\item \emph{Usefulness}
\end{itemize}
of the investigated technical system.

The need for innovation is explicitly or implicitly expressed in a
\emph{technical-economic problem situation}. It results, for example, from
increased requirements, changed conditions, new expectations and tightened
restrictions with regard to production, distribution, use, abrasion or removal
of the technical system.

\begin{center}
  \begin{tabular}{|l|c|c|c|c|}\hline
    & Functionality & Profitability & Controllability & Usefulness\\\hline
    A: Requirements &&&&\\\hline
    B: Conditions &&&&\\\hline
    C: Expectations &&&&\\\hline
    D: Restrictions &&&&\\\hline
  \end{tabular}
  
  A template of the ABER(1) matrix.
\end{center}

The ABER(1) matrix has 16 entries and contains at least as many evaluation
parameters as elements. It is used to systematically explore the actual need
for action, the action goals, the project idea on which the innovation project
is based. It converts this information into technical-economic system
properties of the technical product or service with direct reference to the
corresponding evaluation parameters.

\emph{Extensive parameter variations} are used to elaborate negative feedback
in the ABER matrices at different levels. E.g. an improvement in functional
requirements may cause increased costs and thus have negative impact on
profitability. In this way, TEC are extracted from the variation of parameters
in the ABER(1) matrix.

Working with the ABER(1) matrix also requires a \emph{process analysis} beyond
the scope of the actual action goals. As result of this analysis the technical
system with its overall function is \emph{delimited} as black box model and
its interfaces are sufficiently defined within the overall process. It is
important that no process is skipped to capture also hidden facts, that not
immediately trigger need for action and therefore are not mentioned in the
goals, but may cause additional problems.

This already may result in a more precise definition of the action goals and
in a modification of the project idea, which can be decisive for the later
invention. The intention of the ABER(1) matrix is to anticipate all
conceivable ”yes, but” ("ja, aber" in German), which are expected to be
opposed against an invention when it comes to introduce it into production and
to the market.

The heuristic goal of further work with the ABER(1) matrix is first to find
out the main technical-economic parameter that serves as \emph{guiding
  parameter} for the action goals if it is varied as independent variable, and
the variation behavior of the system of parameters of the evaluation figure as
a whole. In the further analysis of the evaluation figure it is important to
define the systemic, technical-economic problem situation that results from
it.  The technical-economic problem situation results from the fact that
improving the guiding parameter deteriorates other, high-ranking evaluation
parameters to an inadmissible degree or they go beyond given limit values.

ProHEAL assumes that the discussion of the technical-economic problem
situation starts with an already contoured specific technical system. This can
be an existing technical system in terms of the required overall function
(\emph{reference variant}) or one composed of components of the known and
commercially available state of the art (\emph{basic variant}). For a
reference variant optimisation algorithms as well as manufacturing and
operational experience are available. The potential for error is therefore
relatively small. But the potential for contradiction is high as the system as
a whole may be out of date. It is the other way in the case with a basic
variant. A decision goes usually for a basic variant with a balanced ratio of
potential for error and contradiction.

\subsection*{4.2 The Critical Function Matrix (ABER(2) Matrix)\\ at Problem
  Field Level 2}

It serves to systematically delimit the \emph{critical functional area} and to
define the technical-technological innovation objective in the form of the
\emph{ideal subsystem of the basic variant} by defining
\begin{itemize}
\item the functional requirements (A),
\item the design and manufacturing conditions (B),
\item the technological influences (E = \textbf{E}influss) as well as
\item natural law restrictions and their fulfillment (R)
\end{itemize}
in relation to the elementary components of the subsystem:
\begin{itemize}
\item \emph{Operand} (object that is acted on),
\item \emph{Operation} (way of acting)
\item \emph{Operator} (means to act),
\item \emph{Counter-operation} (way of counter-action in the sense of creating
  an equilibrium which realises the function) and
\item \emph{Counter-operator} (means to stabilise the function).
\end{itemize}

As result the technical-scientific solution needs are determined in terms of
new functional requirements, other design and manufacturing conditions,
changed technological influences or other types of natural law restrictions in
the functional realisation. Further such solution needs are to be considered
for which neither suitable means-effect relationships nor function-fulfilling
technical arrangements are known in the system-related state of the art.

The work with the ABER(2) matrix is based on a \emph{function-related
  structural analysis} of the system considered as a whole. It aims at
delimiting the critical functional area and defining the interface conditions
for the ideal subsystem in both structural and functional direction. This
makes the interrelations transparent and manipulable, which cause the
undesired effect in the functional behavior of the ideal subsystem.

The ABER(2) matrix has 20 entries and at least as many functional or
structural parameters as elements for the ideal subsystem. When it is created,
the \emph{need for innovation} and the \emph{technical-technological
  innovation goal} are explored. At the same time the inventive innovation
idea is shaped as the new functional principle of the ideal subsystem. The
considerations are to be extended beyond the ideal subsystem also to its
interrelationships with the technical system as a whole. This is fixed in the
definition of the design conditions and the technological influences in the
ABER(2) matrix.

The work with the ABER(2) matrix is not only directed towards a
contradiction-free inventive solution idea for the ideal subsystem. The result
also may be the formulation of a \emph{TTC} that prevents such a solution
based on known operating principles. In this case, the contradicting
structural and functional parameters in the critical operational area of the
ideal subsystem have been found and based on this, a solution strategy can be
generated that is oriented on a new operating principle.

\subsection*{4.3 The Operational Field Matrix (ABER(3) Matrix)\\ at Problem
  Field Level 3}

It is based on a scientific-mathematical model and a working hypothesis based
on that model concerning the processes in the \emph{critical operational area}
of the ideal subsystem. It serves to systematically record
\begin{itemize}
\item Requirements (A),
\item Conditions (B),
\item Findings (E = \textbf{E}rkenntnisse),
\item Restrictions (R)
\end{itemize}
\newpage
in relation to
\begin{itemize}
\item technically usable effects,
\item technologically to be controlled side effects and accompanying effects,
\item constructively required counter-effects and guiding effects in the
  functional structure of the ideal subsystem
\end{itemize}
as well as the elaboration of the causal relationships between those
operational parameters.

The demand on application-oriented scientific research results from previously
unrealised effectiveness and efficiency requirements, completely new usage
conditions, not yet available scientific knowledge or ethical or ecological
restrictions.

The operational field matrix has 12 entries and at least as many operational
parameters as elements to transform the problem and the solution goal from the
technical to the level of scientific observation and representation.

Now the solution goal is a new functional principle according to the solution
strategy and the operating principle. The solution goal is therefore no longer
immediately oriented towards the invention, but primarily towards the
acquisition of scientific knowledge, which opens up new space for inventive
thinking.

The operational field matrix also serves to critically question inventive
innovation ideas and needs for technical-scientific solutions from a point of
view of natural science restrictions. This can lead to a new view on the
problem and a new inventive innovation idea, which no longer implies an
undesired effect and therefore is free of contradictions in the
technical-technological meaning.

For the critical, solution-oriented exploration of the inventive innovation
idea from this scientific point of view, substance-field analysis can be
applied. Within ProHEAL substance-field analysis was developed further from a
more phenomenological to an analytical tool to create effect-related solution
modules.

For this purpose, a system of scientific effects in different forms was
developed as a knowledge store on electronic media, that could be used to
search for suitable solution variants starting with a problem- and
contradiction-oriented menu. Also Manfred Ardenne’s monograph \cite{4} was
used in the inventor schools.

\section*{5 Conclusion}

To understand the specific experiences of the GDR inventor schools and the
algorithmic TRIZ variant ProHEAL presented here it is indispensable to take
into account the larger picture of the specific economic development
conditions in the GDR of the 1980s. Altshuller's \emph{Theory of the
  Development of Creative Personalities} and his term \emph{heresy} used in
this context \cite{H} refer to special experiences and observations of the
founder of TRIZ as such a personality. At the same time, they point to a
special position and mechanisms of social exclusion of such "troublemakers"
when they are not needed to solve mature problems that resist "normal"
problem-solving approaches. Bundles of such problems in times of crisis -- and
this is what the GDR economy of the 1980s was all about -- opens up scope for
the application of contradiction-oriented problem-solving methodologies in a
wider range.

There was a \emph{practical} fertile ground and roots already prepared in the
1970s which could now grow more intensively in a number of GDR combinates. At
a first level, this required broadening the personnel base of appropriately
trained experts. The practical organisational and methodological approaches of
this movement of trainings are described in more detail in \cite{14}.  Even
driven by strong traits of self-organisation, it would not have reached the
dimensions that were ultimately achieved without the provision of time and
material resources by a number of combinates. The dissolution of this
economic-structural basis after 1990 led to the rapid collapse of these
training structures.

Even the few efforts of far-sighted representatives of a West German culture
of innovation don't change this general picture -- what value can be generated
in a capitalist market economy from experiences gained in a socialist planned
economy? Such a judgement shows ideological blindness, as witnesses the
experience of the introduction of TRIZ methods in South Korea, which was also
essentially triggered by the open-mindedness of strategic management and the
provision of resources by large economic units as SAMSUNG.

Germany was faced with a second TRIZ wave at the beginning of the 2000s, which
was triggered by Russian-speaking TRIZ experts emigrating to Germany.  This
wave was already unable to pick up these older developments, neither in terms
of personnel nor organisation, and the TRIZ expertise of representatives of
this first generation -- such as Michael Herrlich, Hansjürgen Linde,
Hans-Joachim Rindfleisch, Rainer Thiel, Dietmar Zobel -- is still perceived as
marginal even in the inner German discourse.

It should be all the more interesting to compare those developments, which
date back more than 30 years, with current trends -- above all with
\emph{Business TRIZ} -- to identify moments of the uncompensated and let them
become effective. This paper aims to contribute to such a process.

\section*{6 Postscript}

The paper was accepted by the reviewers for presentation at the \emph{TRIZ
  Future Conference 2021}, but it does not meet the "novelty" criteria for a
paper to be included in the Conference Proceedings, as 61\% of the material
presented here\footnote{According to the analysis of the chairs of the
  conference.} can also be found on the pages of the WUMM project and hence
„is not new“.  Such rules massively hinder the further development of
scientific ideas and call into question the discursive character of scientific
work. \emph{LIFIS-Online} is a scientific journal that stands on clearly
different positions. Hence this survey is published in this journal.

\newpage
\begin{thebibliography}{xxx}
\bibitem{1} G.S. Altshuller (1973). Erfinden – (k)ein Problem? Verlag Tribüne,
  Berlin.  Translation of \foreignlanguage{russian}{Алгоритм Изобретения}
  (1969).
\bibitem{2} G.S. Altshuller, A.B. Seljuzki (1983). Flügel für Ikarus.
  Urania-Verlag, Leipzig. Translation of \foreignlanguage{russian}{Крылья для
    Икара: как решить изобретательские проблемы} (1980).
\bibitem{3} G.S. Altshuller (1986). Erfinden. Wege zur Lösung technischer
  Probleme. Verlag Technik, Berlin. Translation of
  \foreignlanguage{russian}{Творчество как точная наука. Теория решения
    изобретательских задач} (1979).
\bibitem{H} G.S. Altshuller, I.M. Vertkin (1991). How to become a heretic: A
  life strategy for a creative personality (in Russian).  
\bibitem{4} M. von Ardenne, G. Musiol, S. Reball (1989). Effekte der Physik
  und ihre Anwendungen. Verlag der Wissenschaften, Berlin.
\bibitem{5} D. Cavallucci, P. Lutz, F. Thiebaud (2000). Intuitive Design
  Method (IDM): A new framework for design method integration. Journal for
  Manufacturing Science and Production, 3(2-4), 95-102.
  \url{https://doi.org/10.1515/IJMSP.2000.3.2-4.95}
\bibitem{6} H.-G. Gräbe (2019). The Development of the GDR Inventor Schools
  and the Evolution of TRIZ (in Russian). In: Online material of the TRIZ
  Developers Summit 2019, Minsk.
  \url{https://triz-summit.ru/confer/tds-2019/}. 
\bibitem{7} H.-G. Gräbe (2019). The Contribution to TRIZ by the Inventor
  Schools in the GDR.  Proceedings of the 15th MATRIZ TRIZfest, 346-352.
\bibitem{8} H.-G. Gräbe, R. Thiel (2021). ProHEAL Basics – Extended
  Version. Manuscript.
  \url{https://wumm-project.github.io/GDR-InventorSchools/ProHEAL-en.pdf}
\bibitem{9} P. Koch, K. Stanke (2021). 50 Jahre Systematische Heuristik.
  Rohrbacher Manu\-skripte, Heft 23. LIFIS, Berlin.\\
  \url{https://nbn-resolving.org/urn:nbn:de:bsz:15-qucosa2-755699}
\bibitem{10} J. Müller, P. Koch et al. (1973). Programmbibliothek zur
  systematischen Heuristik für Naturwissenschaftler und Ingenieure. In:
  Wissenschaftliche Abhandlungen des Zentralinstituts für Schweißtechnik
  Halle, Band 97-99.
\bibitem{11} J. Müller (1990). Arbeitsmethoden der Technikwissenschaften.
  Systematik, Heuristik, Kreativität. Springer, Berlin. ISBN
  978-3-642-93442-1. 
\bibitem{12} V.P. Petrov (2020). Laws and patterns of systems development.
  Book in 4 vol. (in Russian).  Ridero, Moscow (2020).
\bibitem{13} H.-J. Rindfleisch, R. Thiel, G. Zadek (1989). KDT-Erfinderschule,
  Lehrbrief 2: Erfindungsmethodische Arbeitsmittel. Lehrmaterial zur
  Erfindungsmethode. Berlin.
\bibitem{14} H.-J. Rindfleisch, R. Thiel (1994). Erfinderschulen in der DDR.
  Trafo Verlag, Berlin. ISBN 3-930412-23-3.
\bibitem{15} R. Thiel (2020). Dialektik, TRIZ und ProHEAL. Rohrbacher
  Manuskripte, Heft 21.  LIFIS, Berlin.\\
  \url{https://nbn-resolving.org/urn:nbn:de:bsz:15-qucosa2-749491}
\bibitem{16} WUMM-Project. Github Pages at
  \url{https://wumm-project.github.io/}
\bibitem{17} WUMM-Project. Material about the German Inventor Schools in the
  1980s. \\ \url{https://wumm-project.github.io/GIS}
\end{thebibliography}
\newpage

\section*{Appendix. The ProHEAL Building Blocks}

In the following, we reproduce an English translation of the main blocks of
the ProHEAL approach, the \emph{Thinking Field Structure}, the \emph{Path
  Model} and the \emph{Decision Model} as published (in German) in
\cite[Appendix]{13} and republished in \cite[Appendix]{15}.  They give the
TRIZ expert a good overview of similarities and differences of the ProHEAL
approach to other TRIZ versions.

\subsection*{Appendix 1. The ProHEAL Thinking Field Structure}

\begin{center}
  \begin{tikzpicture}[scale=.9,transform shape,
    bbox/.style={draw,fill=white,rectangle,align=center},
    >={Triangle[length=0pt 6,width=0pt 5]},
    rounded corners=2pt,line width=.8pt]
  \draw[dashed,line width=.3pt] (4,11) -- (4.5,3) ;
  \draw[dashed,line width=.3pt] (12.5,11) -- (12,3) ;
  \node[bbox,text width=4cm] at (8,13) (A0)
       {Technical-economic Problem Situation};
  \node[text width=3cm,align=center] at (8,11) (B0)
       {Technical-economic Operational Field};
  \node[bbox,text width=3cm] at (3,11) (B0a)
       {Societal Need};
  \node[bbox,text width=3.5cm] at (13,11) (B0b)
       {State of the Art. Basic Variant};
  \node[bbox,text width=5cm] at (8,9) (A1)
       {Technical-economic (external) Contradiction};
  \node[text width=3cm,align=center] at (8,7) (B1)
       {Critical Functional Area of the Basic Variant};
  \node[bbox,text width=3cm] at (3.3,7) (B1a)
       {IDEAL of the Core Variant};
  \node[bbox,text width=3.5cm] at (13,7) (B1b)
       {Harmful Technical Effect};
  \node[bbox,text width=5cm] at (8,5) (A2)
       {Technical-technological (internal) Contradiction};
  \node[text width=3cm,align=center] at (8.5,3)  (B2)
       {Critical Operational Area of the Core Variant};
  \node[bbox,text width=4cm] at (3.6,3) (B2a)
       {Ideal Scientific Effect of the Core Variant};  
  \node[bbox,text width=3.5cm] at (13,3) (B2b)
       {Harmful Scientific Effect};
  \node[bbox,text width=6cm] at (8,1) (A3)
       {Technical-scientific (internal) Contradiction};
  \node[bbox,text width=3cm] at (8,-1) (A4)
       {Solution Strategy};
  \node[rotate=90,text width=3.5cm] at (0,3) (D1)
       {Problem solution imaginable based on the state of the art of science}; 
  \node[rotate=90,text width=3.5cm] at (0,7) (D2)
       {Problem solution conceivable based on the state of the art of
         science and technology};   
  \node[rotate=90,text width=3.5cm] at (0,11) (D2)
       {Problem solution feasible based on the state of the art of
         technology}; 
  \draw[->] (A0) -- (B0a) ;
  \draw[->] (B0a) -- (A1) ;
  \draw[->] (A0) -- (B0b) ;
  \draw[->] (B0b) -- (A1) ;
  \draw[->] (A1) -- (B1a) ;
  \draw[->] (B1a) -- (A2) ;
  \draw[->] (A1) -- (B1b) ;
  \draw[->] (B1b) -- (A2) ;
  \draw[->] (A2) -- (B2a) ;
  \draw[->] (B2a) -- (A3) ;
  \draw[->] (A2) -- (B2b) ;
  \draw[->] (B2b) -- (A3) ;
  \draw[->] (A3) -- (A4) ;
  \draw[->,dashed] (B0) -- (B0a) ;
  \draw[->,dashed] (B0) -- (B0b) ;
  \draw[->,dashed] (B1) -- (B1a) ;
  \draw[->,dashed] (B1) -- (B1b) ;
  \draw[->,dashed] (B2) -- (B2a) ;
  \draw[->,dashed] (B2) -- (B2b) ;
\end{tikzpicture}
\end{center}
\newpage

\subsection*{Appendix 2. ProHEAL – The Algorithm}

The following presentation follows \cite[Appendix]{13}, in which the algorithm
is presented in short form as a programme flow chart. The numbers in brackets
refer to the detailed version of the algorithm in \cite[ch. 3]{15} (30 printed
pages). Since the presentation of the details would go far beyond the scope of
this paper, we refer the interested reader to this (German) publication.  See
also the diagrammatic presentation of the path model in part D of this
appendix.

\subsubsection*{A. The Technical-Economic Part of the Program}

\textbf{Objective:} Critique of the state of the art from a technical-economic
point of view. Determine the relevant evaluation and reference variables.

\begin{itemize}[leftmargin=35pt,align=left]
\item[(A1)] Specify the societal need (SN) according to operational tasks of
  the enterprise
  \begin{itemize}[leftmargin=20pt]
  \item[(A1a)] Determine the overall SN (1.3), (1.6)
  \item[(A1b)] Determine the special SN (1.1), (1.2), (1.4)
  \end{itemize}
\item[(A2)] Find the ABER (1.4), (1.6), (1.7)
\item[(A3)] Determine the required usage properties (1.4), (1.5)
\item[(A4)] Define the components in the evaluation figure (1.8), (1.9),
  (1.10)
\item[(A5)] Choose the technical-technological principle (2.1)
\item[(A6)] Determine the basic variant of the technical system starting from
  the state of the art (2.2), (2.3), (2.4)
\item[(A7)] Formulate the technical-economic objective (2.5), (4.3b)
\item[(A8)] Black box analysis of the technical system (2.6), (2.7), (2.8),
  (2.9), (2.10), (2.11), (2.14), (2.15), (3.4)
\item[(A9)] Delimit the technical-economic field of operation (2.12), (2.13),
  (3.1), (3.2), (3.3)
\item[(A10)] Determine the guiding parameter (2.5f), (4.1)
\item[(E1)] \textbf{Decide:} Is the technical system appropriately delimited?
  (4.2), (3.4)
  \begin{itemize}[leftmargin=20pt]
  \item \textbf{Yes:} Go to (E2)
  \item \textbf{No:} Back to (A8)
  \end{itemize}
\item[(E2)] \textbf{Decide:} Is an optimisation solution possible? (2.9),
  (2.14), (2.15)
  \begin{itemize}[leftmargin=20pt]
  \item \textbf{Yes:} Work out the optimisation solution → STOP
  \item \textbf{No:} Go to (A11)
  \end{itemize}
\item[(A11)] Find and formulate the TEC that determines the problem (4.2),
  (4.3), (4.4)
\item[(E3)] \textbf{Decide:} Is there a case of ”business blindness”?
  \begin{itemize}[leftmargin=20pt]
  \item \textbf{Yes:} Back to (E2)
  \item \textbf{No:} Proceed with part B
  \item \textbf{Unknown:} Back to (A5)
  \end{itemize}
\end{itemize}
\newpage
\subsubsection*{B. The Technical-technological Part of the Program}
        
\textbf{Objective:} Critique of the state of the art from a
technical-technological point of view. Determine the decisive functional
parameters.

\begin{itemize}[leftmargin=35pt,align=left]
\item[(B1)] Find and formulate the undesired effect (2.10), (2.11), (2.15c),
  (2.15e), (5.1), (5.4)
\item[(B2)] Delimit the critical functional area in the structure of the
  technical system (2.8), (2.15c), (2.15d) (3.4), (5.2), (5.3)
\item[(B3)] Draft the ideal subsystem for the core variant (in the critical
  functional area of the technical system) – IDEAL – (6.1)
\item[(B4)] Develop ideas about the necessary technical requirements (ABER)
  for the usefulness of the IDEAL (ideal conceptions) (6.2)
\item[(B5)] Conceptual modification of the technical system with regard to
  required functional properties outside the critical functional area
  according to the IDEAL on the ABER (6.3), (6.4)
\item[(E4)] \textbf{Decide:} Does a harmful technical effect reappear? (6.5)
  \begin{itemize}[leftmargin=20pt]
  \item \textbf{Yes:} Back to (B2)
  \item \textbf{No:} Go to (E5)
  \end{itemize}
\item[(E5)] Decide: Are the ABER sufficiently determined? (6.2a)
  \begin{itemize}[leftmargin=20pt]
  \item \textbf{Yes:} Go to (B6)
  \item \textbf{No:} Back to (B4)
  \end{itemize}
\item[(B6)] Extract the ideal final result (6.4)
\item[(E6)] \textbf{Decide:} Is the ideal vision in the ABER technically
  feasible? (6.2a), (9.5)
  \begin{itemize}[leftmargin=20pt]
  \item \textbf{Yes:} An unexpected approach to a surprisingly simple solution
    (SSS) is found (6.5). Back to (E2).
  \item \textbf{No:} Go to (B7)
  \end{itemize}
\item[(B7)] Find and formulate the TTC (6.2d), (7)
\item[(E7)] \textbf{Decide:} Is it a prejudice of the professional world?
  (6.2a), (9.5)
  \begin{itemize}[leftmargin=20pt]
  \item \textbf{Yes:} Transition to the elimination of the TTC with surprising
    impact (6.2a), (9.5). Back to (E2).
  \item \textbf{No:} Go to part C
  \item \textbf{Unknown:} Back to (B1)
  \end{itemize}
\end{itemize}
\newpage
\subsubsection*{C. The Technical-scientific Part of the Program}
        
\textbf{Objective:} Critique of the state of the art from a
technical-scientific point of view. Determination of the decisive operational
parameter.

\begin{itemize}[leftmargin=35pt,align=left]
\item[(C1)] Derive the technical-scientific cause of the harmful technical
  effect from the ABER (8.1a)
\item[(C2)] Find the critical operational area in the technical system (2.8)
\item[(C3)] Model the critical operational area.
\item[(C4)] Formulate a search query to the database of scientific effects to
  realise the ABER according to the ideal vision (ideal scientific effect)
  (8.3)
\item[(E8)] \textbf{Decide:} Is there an appropriate scientific effect?
  \begin{itemize}[leftmargin=20pt]
  \item \textbf{Yes:} Consider it as basis for a new technical approaches.
    Back to (E6).
  \item \textbf{No:} Go to (C5)
  \end{itemize}
\item[(C5)] Formulate the technical-scientific contradiction (8.1b), (10.1)
\item[(E9)] \textbf{Decide:} Is it a matter of blindness in the professional
  world? (8.2), (10.1)
  \begin{itemize}[leftmargin=20pt]
  \item \textbf{Yes:} Consider technical approaches from a foreign
    domain. Back to (E2).
  \item \textbf{No:} Go to (C6)
  \item \textbf{Unknown:} Back to (C1)
  \end{itemize}
\item[(C6)] Find suitable solution strategies in the technical system to
  overcome the TSC (8.2), (9.1), (9.2), (9.4a), (10.2)
\item[(C7)] Formulate the invention task with the goal of a radical renewal of
  the structure of the technical system (9.4b)
\item[(C8)] Find suitable solution principles to solve the problem of the
  invention task (9.3), (9.4a)
\item[(C9)] Find fundamentally new approaches to solutions (creation of a new
  generation of the technical system) (10.2) → Back to (A5)
\end{itemize}
\newpage
\subsubsection*{D. Diagrammatic Presentation of the Algorithmic Structure of
  ProHEAL}  

\begin{center}
\begin{tikzpicture}[scale=1.2,transform shape,
    >={Triangle[length=0pt 6,width=0pt 5]},
    rounded corners=2pt,line width=.8pt]
  \node[draw] at (0,15) [rectangle] (A0) {Start};
  \node[draw] at (0,14) [rectangle] (A1) {A1--A5};
  \node[draw] at (0,13) [rectangle] (A6) {A6--A8};
  \node[draw] at (0,12) [rectangle] (A9) {A9--A10};
  \node[draw] at (2,14) [circle] (E1) {E1};
  \node[draw] at (4,14) [circle] (E2) {E2};
  \node[draw] at (6,14) [rectangle] (A11) {A11};
  \node[draw] at (4,12.7) [rectangle] (A4) {\emph{Optimisation}};
  \node at (4,11.7) [rectangle] (A5) {STOP};
  \node[draw] at (8,14) [circle] (E3) {E3};
  \node[draw] at (0,10) [rectangle] (B1) {B1};
  \node[draw] at (0,9) [rectangle] (B2) {B2};
  \node[draw] at (0,8) [rectangle] (B3) {B3--B4};
  \node[draw] at (0,7) [rectangle] (B5) {B5};
  \node[draw] at (2,9) [circle] (E4) {E4};
  \node[draw] at (4,9) [circle] (E5) {E5};
  \node[draw] at (5,9.7) [rectangle] (B6) {B6};
  \node[draw] at (6,9) [circle] (E6) {E6};
  \node[draw] at (7,8.3) [rectangle] (B7) {B7};
  \node[draw] at (6,10.3) [rectangle] (Rel) {\emph{SSS}};
  \node[draw] at (8,10.3) [rectangle] (UW) {\emph{SI}};
  \coordinate (Z1) at (7,12) ;
  \node[draw] at (8,9) [circle] (E7) {E7};
  \node[draw] at (0,4.7) [rectangle] (C1) {C1--C4};
  \node[draw] at (0,3.7) [rectangle] (C6) {C6--C9};
  \node[draw] at (3,4.7) [circle] (E8) {E8};
  \node[draw] at (7,4.7) [circle] (E9) {E9};
  \node[draw] at (5,4.7) [rectangle] (C5) {C5};
  \node[draw] at (3,6) [rectangle] (NTL) {\emph{NTS}};
  \node[draw] at (7,6) [rectangle] (FTL) {\emph{TSD}};
  \coordinate (Z2) at (7,7) ;
  
  \draw[->] (A0) -- (A1) ;
  \draw[->] (A1) -- (A6) ;
  \draw[->] (A6) -- (A9) ;
  \draw[->] (A9) -- (E1) ;
  \draw[->] (E1) -- (E2) ;
  \node at (2.8,14.3) {yes};
  \draw[->] (E1) -- (2,12) -- (A9) ;
  \node at (2.5,13.4) {no};
  \draw[->] (E2) -- (A11) ;
  \node at (4.8,14.3) {no};
  \draw[->] (E2) -- (A4) ;
  \node at (4.4,13.3) {yes};
  \draw[->] (A4) -- (A5) ;
  \draw[->] (A11) -- (E3) ;
  \draw[->] (E3) -- (7.5,15) -- (4,15) -- (E2) ;
  \node at (7.2,14.6) {yes};
  \draw[->,dashed] (E3) -- (8,15.5) -- (2,15.5) -- (A1) ;
  \node at (3,15.3) {unknown};
  \draw[->] (E3) -- (8,11.1) -- (0,11.1) -- (B1) ;
  \node at (7.6,13.4) {no};
  \draw[->] (B1) -- (B2) ;
  \draw[->] (B2) -- (B3) ;
  \draw[->] (B3) -- (B5) ;
  \draw[->] (B5) -| (E4) ;
  \draw[->] (E4) -- (B2) ;
  \node at (1.2,9.2) {yes};
  \draw[->] (E4) -- (E5) ;
  \node at (2.9,9.2) {no};
  \draw[->] (E5) -- (B6) ;
  \node at (4.3,9.7) {yes};
  \draw[->] (E5) |- (B3) ;
  \node at (3.6,8.3) {no};
  \draw[->] (B6) -- (E6) ;
  \draw[->] (E6) -- (Rel) ;
  \node at (6.4,9.7) {yes};
  \draw[->] (E6) -- (B7) ;
  \node at (6.3,8.4) {no};
  \draw[->] (B7) -- (E7) ;
  \draw[->] (E7) -- (UW) ;
  \node at (8.4,9.7) {yes};
  \draw[->,dashed] (E7) -- (7,10.7) -- (2,10.7) -- (B1) ;
  \node at (3.4,10.4) {unknown};
  \draw[-,dashed] (UW) -- (Z1) ;
  \draw[-,dashed] (Rel) -- (Z1) ;
  \draw[->,dashed] (Z1) -- (7,13) -- (E2) ;
  \draw[->] (E7) -- (8,6.5) -- (0,6.5) -- (C1) ;
  \node at (8.4,8.4) {no};
  \draw[->] (C1) -- (E8) ;
  \draw[->] (E8) -- (NTL) ;
  \node at (2.6,5.3) {yes};
  \draw[->] (E8) -- (C5) ;
  \node at (3.9,5) {no};
  \draw[->] (C5) -- (E9) ;
  \draw[->] (E9) -- (FTL) ;
  \node at (7.4,5.4) {yes};
  \draw[->] (E9) |- (C6) ;
  \node at (7.4,4) {no};
  \draw[-,dashed] (NTL) |- (Z2) ;
  \draw[-,dashed] (FTL) |- (Z2) ;
  \draw[->,dashed] (Z2) -- (E7) ;
  \draw[->,dashed] (C6) -- (-1.5,3.7) -- (-1.5,14) -- (A1) ;
  \draw[-,dotted] (-1,15.9) -- (9,15.9) -- (9,11.3) -- (-1,11.3) --
  (-1,15.9) ; 
  \draw[-,dotted] (-1,10.9) -- (9,10.9) -- (9,6.6) -- (-1,6.6) --
  (-1,10.9) ; 
  \draw[-,dotted] (-1,6.4) -- (9,6.4) -- (9,3.3) -- (-1,3.3) --
  (-1,6.4) ; 
\end{tikzpicture}
\end{center}

\subsubsection*{Legend}
\begin{tabular}{ll}
SSS & Surprisingly Simple Solution\\
SI & Surprising Impact\\
NTS & Novel Technical Solution\\
TSD & Technical Solution from another Domain
\end{tabular}
\newpage

\subsection*{Appendix 3. The ProHEAL Decision Tree}

In \cite[Appendix]{13} the decision tree is also presented as a diagram and
announced as \emph{Renewal Passport, Part I -- Elaboration of Inventive Tasks
  and Solution Approaches within the Nomenclature Framework of the
  achievements and work stages of the Plan Science and Technology}.  This
formulates a normative claim as to how ProHEAL fits into more general planning
documents, which play an important role not only in a socialist planned
economy.  Whether this normative claim was realised in practice is another
matter.

\begin{itemize}[leftmargin=30pt,itemsep=6pt]
\item[(P1)] Can the \textbf{technical-economic contradiction} be solved by
  multidimensional optimisation based on the state of the art?
  \begin{itemize}
  \item Yes. Derive a \textbf{Draft Specification}\footnote{German:
    "Pflichtenheft".} of the realisation without inventive objective.
  \item No. Continue with (P2).
  \end{itemize}
\item[(P2)] Can the \emph{harmful technical effect} be determined and
  explained using sufficiently secured hypotheses or models based on the state
  of the technical-technological experience and the technical-scientific
  knowledge?
  \begin{itemize}
  \item Yes. Continue with (P4).
  \item No. Continue with (P3).
  \end{itemize}
\item[(P3)] Hypothesis generation. Derivation of the target question for
  hypothesis testing.  Derive a \textbf{Draft Specification} for the required
  research with a discovery-oriented question.

  Return to (P2) with the results.
\item[(P4)] Does the \emph{ideal final result} appear realisable as a complete
  elimination of the \emph{harmful technical effect} without substantial
  change of the technical system as a whole?
  \begin{itemize}
  \item Yes, under certain conditions.  Derive a \textbf{Draft Specifiction}
    of the realisation with inventive objective.
  \item Not realisable, even taking into account all feasible options.
    Continue with (P5).
  \end{itemize}
\item[(P5)] Does the \textbf{technical-technological contradiction} appear to
  be solvable based on the state of the art in technology or at least a
  solution is hypothetically conceivable?
  \begin{itemize}
  \item Yes.  Continue with (P7). 
  \item No, and even hypothetically not conceivable.  Continue with (P6).
  \end{itemize}
\item[(P6)] Solve the \textbf{technical-scientific contradiction} by finding
  hypotheses, building models and deriving the search question for effective
  operational principles.  Derive a \textbf{Draft Specification} of the
  research project with inventive objective.

  Return to (P5) with the results.
\newpage
\item[(P7)] Is a \textbf{solution strategy for the technical-technological
  contradiction} feasible? 
  \begin{itemize}
  \item Yes, possibly in more distant analogy areas and/or on the basis of on
    the basis of sufficiently secured hypotheses. Continue with (P9).
  \item No. Continue with (P8).
  \end{itemize}
\item[(P8)] The search question has to be formulated on the basis of
  insufficiently secured hypotheses concerning the technical applicability of
  operational and working principles.  Derive a \textbf{Draft Specification}
  of research with an inventive objective.

  Return to (P7) with the results.
\item[(P9)] Derive a \textbf{Draft specification} for realisation from the
  principal solution approaches with an inventive objective.
\end{itemize}

\ccnotice
\end{document}

