\documentclass[11pt,a4paper]{article}
\usepackage{od}
\usepackage[ngerman]{babel}
\usepackage[utf8]{inputenc}

\setlist{noitemsep}

\begin{document}
\begin{quote}
  Quelle: \url{https://www.deutsche-erfinder-akademie.de/satzung/}

  Textaufnahme: Hans-Gert Gräbe, Leipzig
\end{quote}

\section*{\centering Satzung der deutschen Erfinder-Akademie e.V.}

\subsection*{§1}
Zur Förderung hochkreativer aufwands-nutzens-optimaler und erfinderischer
Lösungen technoökonomischer Probleme durch Ingenieure, Techniker,
Naturwissenschaftler, Meister und Interessierte aus allen Schichten der
Bevölkerung wird ein Verein unter dem Namen „Deutsche Erfinder-Akademie e.V.”,
Sitz: 04275 Leipzig, Kochstraße 1\,B gebildet. Der Verein verfolgt
ausschließlich und unmittelbar gemeinnützige Zwecke im Sinne des Abschnitts
“Steuerbegünstigte Zwecke” der Abgabenordnung.

\subsection*{§ 2}
Zweck des Vereins ist die praxisnahe Qualifizierung von Bewerbern durch
Training bezüglich der
\begin{itemize}
\item Schlüsselqualifikationen:
  \begin{itemize}
  \item Rationelles Informieren, Kommunikations- und Kooperationsfähigkeit
  \item kybernetisches sowie bionisches Denken und Handeln
  \item Eigeninitiative
  \item Kritikfähigkeit
  \item Fähigkeit zur Folgenabschätzung u.a.
  \end{itemize}
\item Systematische Aufwand-Nutzen-Optimierung (SANO)
\item Erfindermethodik.
\end{itemize}
Dieser Zweck wird vornehmlich verfolgt durch
\begin{itemize}
\item Informations- und Schulungsveranstaltungen (Kurse, Seminare,
  Fachtagungen usw.)
\item vereinsinterne und -externe Beratungen sowie Erfahrungsaustausche
\item Erarbeitung und Herausgabe von Lehr- und Hilfsmitteln
\item Organisation von Technologie- und Innovationsvereinen.
\end{itemize}
Zum Erreichen dieses Zwecks ist der Verein selbstlos tätig. Er verfolgt nicht
in erster Linie eigenwirtschaftliche Ziele.
\subsection*{§ 3}
Mitglieder des Vereins können natürliche und juristische Personen werden.

Die Mitgliedschaft beginnt nach Bestätigung des Aufnahmeersuchens durch den
Vorstand und Zahlung des Jahresmitgliedsbeitrages, welcher 
\begin{itemize}
\item für Personen mindestens 300 DM und
\item für Institutionen midestens 3\,000 DM beträgt.
\end{itemize}
Einkommensschwache können beim Vorstand eine Ermäßigung beantragen.
\subsection*{§ 4}
Neben den Mitgliedsbeiträgen erhält der Verein auch Mittel aus kostendeckenden
Teilnehmergebühren von Bildungsveranstaltungen sowie aus Zuwendungen, wie
Zuschüsse, Spenden und Verfügungen von Todes wegen.

Mittel des Vereins dürfen nur für den Vereinszweck verwendet werden.

Die Mitglieder erhalten keine Zuwendungen aus Mitteln des Vereins.

Es darf niemand durch vereinsfremde Ausgaben oder durch unverhältnismäßig hohe
Vergü\-tun\-gen begünstigt werden.
\subsection*{§ 5}
Die Mitgliedschaft endet durch Austritt, Tod oder Ausschluß.

Der Austritt muß schriftlich dem Vorstand gegenüber erklärt werden.
Ausschließungsgründe sind:
\begin{itemize}
\item grober Verstoß gegen die Satzung oder die guten Sitten
\item Nichtzahlung des Beitrages trotz Mahnung.
\end{itemize}
\subsection*{§ 6}
Über Aufnahme und Ausschluß entscheidet der Vorstand.

Gegen die Entscheidung des Vorstandes ist Berufung an die
Mitgliederversammlung möglich. Diese entscheidet endgültig.
\subsection*{§ 7}
Organe des Vereins sind
\begin{itemize}
\item die Mitgliederversammlung
\item der Vorstand
\end{itemize}
\subsection*{§ 8}
Die Mitgliederversammlung tritt mindestens einmal pro Jahr an einem vom
Vorstand zu bestimmenden Ort zusammen.

Die Mitglieder werden hierzu vom Vorstand zwei Wochen vorher unter Angabe der
Tagesordnung schriftlich eingeladen.

Außerordentliche Mitgliederversammlungen werden auf Antrag mindestens eines
Viertels der Mitglieder durchgeführt.

Die Leitung der Mitgliedsversammlung obliegt dem Vorsitzenden oder dem
Stellvertreter.

Über jede Mitgliederversammlung ist eine Niederschrift anzufertigen und durch
die nachfolgende Mitgliederversammlung zu billigen. Die Niederschrift ist
nach Billigung durch den Leiter der Mitgliederversammlung und durch ein
weiteres Vorstandsmitglied zu unterzeichnen.
\subsection*{§ 9}
Die Mitgliederversammlung
\begin{itemize}
\item beschließt über die Satzung und die Geschäftsordnung,
\item nimmt den Rechenschaftsberichtdes Vorstandes über das abgelaufene
  Kalenderjahr einschließlich Kassenbericht entgegen und erteilt ihm
  gegebenenfalls Entlastung,
\item beschließt über den Haushalt,
\item entscheidet über den Jahresbeitrag,
\item wählt den Vorstand und
\item die Kassenprüfer.
\end{itemize}
\subsection*{§ 10}
Beschlüsse der Mitgliederversammlung werden mit einfacher Mehrheit der
erschienenen Mitglieder gefaßt.

Beschlüsse über die Auflösung des Vereins oder über die Änderung der Satzung
bedürfen der Anwesenheit von mindestens der Hälfte der Mitglieder und einer
Mehrheit von mindestens drei Vierteln der abgegebenen Stimmen. Sofern nicht
mehr als die Hälfte der Mitglieder anwesend waren, kann eine zweite
Mitgliederversammlung zum selben Tagungsordnungspunkt einberufen werden, die
dann mit einer Mehrheit von mindestens drei Vierteln der abgegebenen Stimmen
der Anwesenden beschlußfähig ist.

Stimmberechtigt sind alle Mitglieder.  Juristische Personen haben je eine
Stimme.  Stimmüber\-tragung ist nicht möglich.

Abstimmungen sind geheim durchzuführen, wenn dies von einem anwesenden
Mitglied beantragt wird.
\subsection*{§ 11}
Der Vorstand besteht aus dem Vorsitzenden, seinem Stellvertreter und
mindestens zwei Beisitzern und wird aller drei Jahre gewählt.

Der Vorsitzende und sein Stellvertreter vertreten den Verein im Rechtsverkehr.
Jeder von ihnen ist stets einzelvertretungsberechtigt.

Der Vorstand beschließt über alle Maßnahmen, die nicht der
Mitgliederversammlung obliegen, insbesondere über
\begin{itemize}
\item die Planung und Durchführung von Seminaren, Informations- und
  Schulungsveranstaltungen sowie
\item die Herausgabe von Lehrmitteln
\end{itemize}
Die Beschlüsse des Vorstandes werden mit einfacher Mehrheit aller
Vorstandsmitglieder gefaßt. Bei Stimmengleichheit gibt die Stimme des
Vorsitzenden oder bei dessen Abwesenheit die seines Stellvertreters den
Ausschlag.
\subsection*{§ 12}
Bei Auflösung oder Aufhebung des Vereins oder bei Wegfall des bisherigen
Zwecks fällt das Vereinsvermögen in die Verfügung der gemeinnützigen
\begin{quote}
  Gesellschaft zur Förderung des Erfinderwesens in Deutschland e.V. (GFEW)\\
  90429 Nürnberg\\
  Spittlertorgraben 15 
\end{quote}
die dieses unmittelbar und ausschließlich für gemeinnützige oder
wissenschaftliche Zwecke im Sinne ihrer Satzung zu verwenden hat.
\end{document}
