\documentclass[11pt,a4paper]{article}
\usepackage{od}
\usepackage[ngerman]{babel}
\usepackage[utf8]{inputenc}

\setlist{noitemsep}

\title{Entstehung und Entwicklung von Erfinderschulen}

\author{Dr.-Ing. Michael Herrlich und Prof. Dr.-Ing. habil. Peter Koch}

\date{2. Juni 2016}


\begin{document}
\maketitle

\begin{quote}
  Quelle: \url{http://problemloesendekreativitaet.de/} Historie, Abschnitt~10.

  Textaufnahme: Hans-Gert Gräbe, Leipzig
\end{quote}

\section*{0. Vorbemerkung}

Die Erfinderschulen waren in der DDR mit vielen Hunderten Seminaren eine fast
typische Weiterbildung für die F/E Bereiche vorwiegend der Industrie ab Ende
der 70er bis zur Liquidierung der eigenständigen F/E Kapazitäten der
Industriekombinate in Folge der Wende. Ihr Beitrag zur praktischen
Erfindertätigkeit war hoch und wurde geschätzt, was damals zu der großen
Verbreitung republikweit führte. 

Ein herangebildeter Trainerstamm lehrte bereits ab 1975 (vgl. [20, He 14]) in
einer Vorstufe auch die Erkenntnisse Altshullers (ARIZ und später TRIZ) – oft
mit eigenen Adaptionen und spezifischen Lehrmaterial.

Eine für die Erfinderschulen herausragende und im besten Sinne agitatorisch
wirkende Per\-sön\-lich\-keit dieser Zeit war Dr.-Ing. Michael Herrlich, der
mit vielen anderen Trainern die Erfinderschulen konzipierte und umsetzte. Der
nachfolgende, etwas längere Beitrag gemeinsam mit dem ebenfalls an der
Gründung der Erfinderschulen mitwirkenden Prof. Dr.-Ing. Peter Koch gibt einen
Überblick über das wichtigste zu „Erfinderschulen“. Der Beitrag zeigt
zugleich, dass „Erfinderschulen“ zu recht einen besonderen Platz in der
„Geschichte/Historie der problemlösenden Kreativität“ finden.

Verweisen sei auch zu diesem Thema auf den Beitrag von Dr. R. Thiel
„Erfinderschulen -- Problemlöse-Workshops. Projekt und Praxis“ auf der Seite
„ProblemlösenmitSystem“ dieser Homepage.

\section*{1. Einführung}

\subsection*{1.1. Zielsetzung der Erfinderschulen}
Die sogenannten Erfinderschulen waren und sind darauf ausgerichtet
kreativitäts- und effektivitätsfördernde Prinzipien, Methoden, Denk- und
Arbeitsweisen sowie Kenntnisse zum Patentwesen und zu Patentrecherchen für
das Stimulieren von Ideen und der Erfindertätigkeit, vor allem für Mitarbeiter
aus Forschung und Entwicklung (F/E), praxisbezogen zu vermitteln und ihre
Anwendung an aktuellen, noch nicht gelösten, problemorientierten
Aufgabenstellungen aus der Praxis für die Praxis zu trainieren.

Erfinden im Sinne der Erfinderschulen ist das Entwickeln überraschend neuer,
noch nie dagewesener, funktionsfähiger, wirtschaftlich attraktiver,
schutzrechtlich durch Patente oder Gebrauchsmuster gesicherter, realisierbarer
\emph{Problemlösungen}. Die Erfindungsmethodik und das Konzept der
Erfinderschulen sind eine wichtigen Säule und ein bedeutender Beitrag
\emph{zur Förderung der problemlösenden Kreativität}.

\subsection*{1.2. Notwendigkeit, Zielgruppe und Ansatz}
Die Beschleunigung der internationalen industriellen Entwicklung, vor allem in
der exportorientierten Industrie der DDR, erforderte in den 1970er Jahren
einerseits mehr Erfindungen und Patente mit großer Erfindungshöhe sowie hoch
produktive, wirtschaftlich effiziente Neuerungen auf Weltniveau. Es galten in
progressiven F/E-Kreisen z.B. die Thesen „Phantasie und Erfinden ist Pflicht“,
„Wer nicht erfindet verschwindet“ oder „Geht nicht, gibt’s nicht“.

Anderseits waren zu dieser Zeit national und international
kreativitätsfördernde Prinzipien, Methoden sowie Denk- und Arbeitsweisen für
F/E herangereift. Mit ihrer Anwendung in der Industriepraxis konnte eine
nachhaltige Beschleunigung und Qualifizierung der Innovationsprozesse durch
Rationalisierung der F/E-Tätigkeit und durch die Kreativitätsförderung in den
F/E-Prozessen in vielen Fällen erreicht werden.

In diesem Umfeld entwickelte sich das Ziel und der Ansatz, F/E- Mitarbeiter,
F/E-Teams, jugendliche Forscherkollektive und andere kreative Kräfte für die
Weiterentwicklung ihrer Arbeitsweise durch das Vermitteln und Trainieren einer
kreativen, methodisch-systematischen Arbeitsweise in Erfinderschulen zu
qualifizieren und erfolgreicher zu machen.

Klar war, dass gutes fachliches Wissen, Fleiß und Gründlichkeit in F/E allein
nicht mehr ausreichen, um in dem zunehmenden Wettbewerb mehr, schneller und
gezielt planbar \emph{anspruchsvollere Erfindungen und Patente} zu generieren.
Ausgehend von der Notwendigkeit und den gewonnenen Erkenntnissen und
Erfahrungen wurde nach 1975 von „Pionieren“\footnote{Als in der Gründungsphase
aktive Mitglieder werden genannt: Dipl.-Ing. Päd. Jürgen Bausdorf,
Dr. sc. techn. Burghard, Dr.-Ing. Klaus Henning Busch,
Prof. Dr. sc. techn. Frank Heinrich, Dr. sc. techn. Dieter Herrig,
Prof. Dr. sc. techn. Peter Koch, Dr. Lange, Dr. Leichsenring,
Dr. sc. päd. Hans-Georg Mehlhorn, Dr. sc. techn. Müller, Dr.-Ing. Hans-Joachim
Rindfleisch, Ing. Karl Speicher, Dr. habil. Rainer Thiel.} ein Konzept für
technikorientierte Erfinderschulen erarbeitet und in den Folgejahren
umgesetzt.

Die Entstehung und Entwicklung der Erfinderschulen kann in drei Phasen
gegliedert werden:
\begin{itemize}
\item Akkumulations- und Ideen-Phase im Vorfeld der Erfinderschulen.
\item Gründungs- und Entwicklungs-Phase in der DDR bis 1990.
\item Fortsetzungs-Phase ab 1990 in der wiedervereinigten BRD.
\end{itemize}

\section*{2. Die Akkumulations- und Ideen-Phase im Vorfeld der
  Erfinderschulen} 
 
In den Jahren von 1950 bis 1980 wurden zunehmend intensiver und klarer
Methoden zur Rationalisierung der geistigen Arbeit und zur
Kreativitätsförderung für die schöpferische Produkt- und Verfahrensentwicklung
von mehreren Disziplinen und unter verschiedenen Aspekten ver\-öffent\-licht,
in der Praxis zur Anwendung gebracht, weiterentwickelt und als wirksam in
Problem\-lösungs\-prozessen erkannt -- sowohl für die individuelle
schöpferische Denk- und Arbeitsweise als auch für kreative interdisziplinäre
Teamarbeit.

Schulen, Ansätze und Beiträge aus dem deutschsprachigen Raum, die in diesem
Zeitabschnitt u.a. maßgeblicher Ausgangspunkt für die Erfindungsmethodik und
die Erfinderschulen waren, sind z.B. in der Beitragsfolge „Historie der
problemlösenden Kreativität“, Beitrag Nr. 9 „Entwicklung der
Konstruktionswissenschaft von 1950 bis 1990“ mit über 50 Quellenangaben
dargestellt \cite{1}. Darüber hinaus bewirkte schon das bereits 1973 von
Dr. Kurt Willimczki ins Deutsche übersetzte Buch von Altschuller zum ARIZ
\cite{3.1}, dass die Widerspruchsdialektik als wirksame Quelle für Erfindungen
in die Erfindungsmethodik der Erfinderschulen einging und weiterentwickelt
wurde.

Die Anwendung dieser Ansätze, Prinzipien, Methoden, Denk- und Arbeitsweisen in
der F/E-Praxis und die daraus resultierenden Ergebnisse und Erfahrungen trugen
im Vorfeld für die Entwicklung der Erfinderschulen wesentlich bei. So
wurden z.B. wirksam:
\begin{itemize}
\item Die Anwendung der Systematischen Heuristik \cite{2}, die Anfang 1970 bis
  1972 gestützt durch die kreativitätsfördernden Methoden in ihren
  Problemseminaren für F/E-Mit\-ar\-bei\-ter und vor allem in
  unternehmens-spezifischen, methodisch moderierten
  Problem\-lösungs\-prozessen in großen Industriezentren und F/E-Einrichtungen
  überraschend attraktive Ergebnisse in großer Breite erreichte. In dieser
  Phase wurden nachhaltige Erkenntnisse und Ergebnisse zur Anwendung und zum
  Training methodisch-systematischer Arbeits- und Denkweisen für
  interdisziplinäre, methodisch moderierte Teamarbeit gewonnen.
\item Ab 1973 wurde das Gedankengut der Systematischen Heuristik u.a. für die
  AUTEVO-Projekte (AUtomatisierte TEchnischen VOrbereitung der Produktion) in
  großen Unternehmen, z.B. Kombinaten, genutzt und weiterentwickelt.
\item Die Arbeitsgemeinschaft „Rationalisierung der geistigen Arbeit und
  Methodik des Erfindens“, angesiedelt bei der Kammer der Technik der DDR, in
  der u.a. die Erfindungsmethodik von Altschuller (TRIZ) \cite{3.1,20} in
  Deutschland zur Anwendung kam und zur Verbreitung führte.
\item Die Anwendung der Erfindungsmethodik durch Verdiente Erfinder in ihren
  Unternehmen, z.B. durch Dr. M. Herrlich im Institut für Süßwarenindustrie
  Leipzig.
\item Die Methoden, Denk- und Arbeitsweisen des Institutes für Schweißtechnik
  Halle/Saale, durch das jährlich 50 bis 80 Patente angemeldet wurden.
\item Präsentation von repräsentativen Erfindungsprozessen, die durch sehr
  erfolgreiche Erfinder, wie z.B. den Maschinenbau-Ingenieur und Verdienten
  Erfinder Ing. Karl Speicher aus dem Dampfturbinenbau, an Hand von
  Fallbeispielen in Workshops bzgl. des Vorgehens und der Motivation
  vorgestellt wurden.
\item Der Fachausschuss Konstruktion bei der Kammer der Technik, dessen
  Mitglieder die modernen Konstruktionsmethoden in ihren industriellen
  Wirkungsbereichen zur Anwendung brachten, auf Tagungen publizierten und
  später im Lehrmaterial darstellten.
\item An einigen Hochschulen, die z.B. mit der Konstruktionstechnik oder der
  Technischen Entwicklungslehre die Vermittlungsfähigkeit von
  Konstruktions-Methoden im Studium erkannten und diese Erfahrungen nutzten
  \cite{4,5,6}.
\item Von der KDT wurde durch Dr. Heyde und Dr. Paetzold Lehrmaterial für die
  KDT-Lehrgänge „Förderung des schöpferischen Denkens und der Initiative zur
  rationellen Lösung wissenschaftlich-technischer Aufgaben“ bereits 1975
  \cite{20} herausgegeben, die auch den ARIS-68 von Altshuller und seine
  damals 35 technische Prinzipien zur „Lösung wissenschaftlich-technischer
  Aufgaben“ mit umfassten und folglich in die durchgeführten Seminare getragen
  wurden.
\end{itemize}
Die Reihe der positiven Erfahrungen bei der Anwendung kreativer Methoden in
Problem\-lösungs\-prozessen in der F/E-Praxis ließen sich weiter fortsetzen.
Es wurde durch diese praxisbezogenen Arbeitsprozesse und ihre Ergebnisse die
Überzeugung gewonnen, dass Schöpfertum, Kreativität und die
Erfindungstätigkeit sehr wesentlich durch kreative,
methodisch-systema\-ti\-sche Arbeitweisen für die Problemerkennung,
Problempräzisierung und Problemlösung in Wissenschaft und Technik gefördert
werden können und eine bedeutende Leistungssteigerung, vor allem bei
Teamarbeit, erreichbar ist.

Es ging letztendlich darum, bei den Teilnehmern Grundlagen zu einem funktions-
und widerspruchsbezogenen, sich von der „Enge“ des Bekannten befreienden, sehr
flexiblen, methodischen Denkstil zu legen und die Vorgehensweise und Stärken
kreativer, methodisch-systematischer Arbeitsweise im Team anzuwenden und
erlebbar zu machen.

Es wurde weiterhin sichtbar und als Arbeitsthesen postuliert, dass
\begin{itemize}
\item kreative, methodisch-systematische Arbeitsweisen durch die Synthese von
  moderiertem Training und individueller Arbeit an realen
  technisch-wissenschaftlichen, \emph{noch nicht gelösten} Problemstellungen
  aus der Praxis für die Praxis lehrbar und erlernbar sind und
\item die Motivation zum bewussten kreativen, schöpferischen Arbeiten und
  Erfinden nachhaltig gefördert werden kann, um noch erfolgreicher zu werden.
\end{itemize}

\section*{3. Die Gründungs- und Entwicklungs-Phase der Erfinderschulen in der
  DDR bis 1989} 
 
\subsection*{3.1. Gründung}
\emph{Die offizielle Gründung der Erfinderschulen} wurde im Jahr 1980 auf
Initiative von Dipl.-Ing. Michael Herrlich und Dipl. oec. Ing. Gerhard Zadek
durch den Patentamtpräsidenten der DDR, Prof. Dr. Hemmerling und den
Präsidenten der Kammer der Technik (KDT), Prof. Dr. Schubert eingeleitet. Es
wurde der Auftrag erteilt, Erfinderschulen, begleitet durch die
organisatorisch-materiellen Möglichkeiten der KDT, vorzubereiten und
durchzuführen. Die KDT war die Ingenieur-Dachorganisation in der DDR, die in
fachlicher Hinsicht ähnliche Aufgaben wie der VDI in der BRD verfolgte.

Im Rahmen dieses Auftrages entwickelte ein kreatives Team das Konzept und
Programm für die Erfinderschulen, die als Erfinder-Seminare gestaltet werden
sollten. Die erste Erfinderschule wurde vom 27. bis 31.05.1980 in einem
KDT-Schulungsheim mit 21 Teilnehmern unter Leitung von Dr. Herrlich und
mehreren in der Erfindermethodik erfahrenen Ingenieuren und Erfindern
durchgeführt.

\subsection*{3.2. Entwicklung}
Nach den überzeugenden Erfolgen der ersten Erfinderseminare wurden
Erfinderschulen in allen Bezirken der DDR durch die KDT-Bezirkserfinderschulen
gebildet und damit in die Breite getragen. So gelang es z.B. einige Monate
später durch Initiative von Dr. Rainer Thiel und Hans-Joachim Rindfleisch im
Bezirk Berlin der KDT dort die erste Erfinderschule zu organisieren,
unterstützt durch den Berliner Werkzeugmaschinenbau Marzahn. Gleichzeitig
wurden Erfinderschulen in den Bezirken der DDR durchgeführt, so z.B. in
Rostock, Schwerin, Potsdam, Leipzig, Dresden, Karl-Marx-Stadt, Suhl. Aus den
ersten Erfinderschulen qualifizierten sich Teilnehmer als Workshop-Trainer für
neue Erfinderschulen.

Das Konzept und der Inhalt mit dem Methodenangebot wurden in den Folgejahren
schrittweise bis 1990 weiterentwickelt. Dazu trugen auch die von Michael
Herrlich einberufenen jährlichen, mehrtägigen Trainertagungen bei. Es wurden
Vorbereitungsmaterialien, Broschü\-ren, Lehrbriefe, Fachbeiträge besonders von
Verdienten Erfindern erarbeitet, so z.B.
\begin{itemize}
\item Das erste offizielle Erfinderschul-Lehrmaterial \cite{7}, das von einem
  Autorenkollektiv unter Leitung von M. Herrlich und G. Zadeck erstellt wurde.
  (siehe auch \cite{19}).
\item H.-J. Rindfleisch entwickelte unter Mitwirkung von R. Thiel bis 1988
  ausgehend vom Altshullerschen Widerspruchsgedanken seine
  „Erfindungsmethodischen Grundlagen“ als KDT-Lehrmaterial, das ein Programm
  zum Herausarbeiten von Erfindungsaufgaben und Lösungsansätzen vermittelt
  \cite{8}.
\item Dietmar Zobel veröffentlichte nach seinen Erfahrungen in Erfinderschulen
  sein Buch „Erfinderfibel – systematisches Erfinden für Praktiker“ \cite{9.1}
  (1984) und später „TRIZ FÜR ALLE“ \cite{9.3}. Er vermittelt, wie
  Widersprüche herausgearbeitet und wie sie gestützt auf 35 generierende
  Lösungsprinzipien geknackt werden können, um damit neuartige Lösungen mit
  großer Erfindungshöhe gezielt zu gewinnen.
\end{itemize}
Quasi zeitparallel entwickelte sich das Kreativitätszentrum für
wissenschaftlich-technisches Schöpfertum in F/E (ctc) im Umfeld der
Bauakademie der DDR und des Kombinates CARL ZEISS JENA, in dem erfolgreiche
F/E-Mitarbeiter in Trainingsseminaren das Erkennen und Lösen von
Problemstellungen aus ihren Unternehmen \emph{in der Einheit} von kreativen
Denk- und Arbeitsweisen und Kreativitätstechniken zur Förderung der
\emph{wissenschaftlich-technischen Kreativität} (problemlösende Kreativität)
und Kreativität fördernden Verhaltensweisen im Team durch die bewusste
Einbeziehung und Entwicklung der \emph{sozialen Kreativität} mit
erfinderischer Zielstellung trainierten \cite{10,11}. Die Erfinderschulen und
das Kreativitätszentrum (ctc) befruchteten sich gegenseitig.
    
\subsection*{3.3. Konzept}
Das Konzept der Erfinderschulen orientierte
\begin{itemize}
\item einerseits auf das \emph{Vermitteln der Erfindungsmethodik}, der
  notwendigen Kenntnisse zum Patentwesen und auf die Methoden für
  Patentrecherchen und
\item andererseits vor allem auf \emph{Problemlösungsworkshops}, in denen in
  kleinen Arbeitsgruppen von 6 bis 8 Teilnehmern vor allem die Anwendung der
  Erfindungsmethodik, Prinzipien und Denkweisen für das Lösen realer,
  problemhaltiger Aufgabenstellungen, gestützt auf die Moderation der Trainer,
  trainiert wurde.
\end{itemize}
Wesentlich hierbei war, dass die Problemstellungen, deren \emph{Lösungen noch
nicht bekannt} sind, für die Workshops von den Teilnehmern aus ihren
Unternehmen eingebracht wurden und dass wirkungsvolle, für die Praxis nutzbare
Ergebnisse erarbeitet werden sollten und wurden.

Der wirksamste und zeitlich größte Teil der Erfinderseminare bezog sich auf
das schöpferische Herausarbeiten und Lösen der erfinderischen
Aufgabenstellungen in den Workshops. Die Erfinderseminare wurden unter
Klausurbedingungen vollzogen. In den Problembearbeitungs-Gruppen wurde
intensiv, aufgeschlossen und hoch engagiert von 8:00 bis ca. 22:00 Uhr mit
geeigneten Pausen gearbeitet und gelernt, einschließlich themengerechter, das
Blickfeld erweiternder, die Motivation fördernder Abendprogramme.

Große F/E-Einrichtungen und Industriekombinate beschickten komplette
Erfinderseminare mit eigenen Mitarbeitern und Problemstellungen. Damit wurden
ergänzend zu den offenen Erfinderschulen, in denen Teilnehmer aus
verschiedenen Unternehmen zusammentrafen, auch die geschlossenen,
unternehmensspezifischen Erfinderschulen erfolgreich praktiziert. Gerade sie
gewannen schnell an Bedeutung. Mit dem Ziel, fundierte, für Patentanmeldung
gereifte, schutzfähige Lösungen zu generieren, wurden vor allem die
geschlossenen Erfinderschulen in mehreren Etappen strukturiert.
\begin{itemize}
\item[1.] \emph{Vorbereitung} mit Information der Unternehmensleitung und der
  Teilnehmer zu den Anforderungen und zum Ablauf, Klärung der einzubringenden
  Problemstellungen und der Seminar-Organisation sowie die Ausgabe und
  Besprechung des Vorbereitungsmaterials.
\item[2.] \emph{Erste Seminarwoche in Klausur} mit bis zu 25 Teilnehmern mit
  dem Ziel, die erfinderische Aufgabenstellung zu erarbeiten, die
  erfinderischen Ideen und Lösungsansätze zu generieren und die
  Patentrecherche einzuleiten.
\item[3.] \emph{Erste Selbstarbeitsphase im Unternehmen}, in der die
  Teilnehmer selbstständig im Team oder individuell an ihren Problemstellungen
  im Unternehmen fachlich und methodisch arbeiten, z.B. mit Recherchen zum
  Stand der Technik, der Literatur, der Patente, Marktanalysen und kritischen
  Analysen, Modellen, Berechnungen, Versuchen, Vergleichen zum Objekt und
  Gegenstand der Aufgabenstellung.
\item[4.] \emph{Zweite Seminarwoche in Klausur}, in der ein
  Erfahrungsaustausch, die Wiederholung und Anwendung der Methodik zur
  weiteren Vertiefung der kreativen Lösungsfindung, die kritische Analyse und
  Bewertung der entwickelten Lösungen sowie die Aktivitäten zur
  Schutzrechtsarbeit und zur Anfertigung von Patentschriften im Mittelpunkt
  standen.
\item[5.] \emph{Zweite Selbstarbeitsphase im Unternehmen} zur Konkretisierung
  und Aufbereitung der Problemlösungen, die Patentanmeldung in Form eines
  Rechercheantrages und die Verteidigung der Ergebnisse im Unternehmen.
\end{itemize}
Beim Durchhalten diese Struktur wurden in der Regel attraktive Ergebnisse
erreicht. Dies ließ sich jedoch in diesem Umfang nicht immer umsetzen, so in
den offenen Erfinderseminaren. 
   
\subsection*{3.4. Inhalt}   
Es wurden zu folgenden vier Schwerpunkten das Vorgehen, die Methoden,
Prinzipien sowie die Denk- und Arbeitsweisen für den kreativen
Problembearbeitungsprozess vermittelt und \emph{vor allem durch Anwendung
trainiert}.

\paragraph{(1) Aufgabenentwicklung.}
Ausgehend von der Zielorientierung und dem Bedürfnissen standen für diesen
Schwerpunkt folgende Tätigkeiten im Mittelpunkt:
\begin{itemize}
\item den Problemsachverhalt erfassen,
\item anspruchsvolle Ziele setzen durch das Erarbeiten zu beachtender, aber
  auch anspruchsvoller Forderungen, Bedingungen, Restriktionen und Umstände,
\item den Zweck und die zu erfüllende Funktion herausarbeiten,
\item die angestrebten Wirkungen und Effekte definieren,
\item den Stand der Technik (Welthöchststand) erarbeiten,
\item das Erkennen von sich gegenseitig bedingenden Gegensätzen, Hindernissen,
  Schwachstellen, Risiken, Lücken sowie
\item das Herausarbeiten der zu lösenden Teil-Probleme und Teil-Aufgaben, die
  im Problembearbeitungsprozess bewältigt werden müssen, um die Zielsetzung zu
  erfüllen.
\end{itemize}

\paragraph{(2) Erfinderische Aufgabenstellung und Arbeitsplan herausarbeiten.}
\emph{Im ersten Schritt dieses Schwerpunktes} kam es darauf an,
\begin{itemize}
\item mit anspruchsvollen Anforderungen und
\item bewusster Zuspitzung der Anforderungen bis hin zur „Idealen Lösung“
\item den Problemkern zu erkennen,
\item die Widersprüche, die der Lösung durch gegenseitig bedingte Gegensätze
  entgegenstehen, zu erfassen und
\item sie kreativ und lösungsorientiert zu formulieren.
\end{itemize}
Von der „idealen Lösung“ kann oft rückwärts schreitend auf die konkreten
Lösungen hingearbeitet werden. Die Teilnehmer können hier erkennen, dass die
intensive Auseinandersetzung mit dem Problem notwendig und sehr wirksam ist.
Sie sollten und konnten erfahren: „Das klare Erkennen und Formulieren des
Problems oder die richtige Fragestellung kann die halbe Lösung sein“.

\emph{Im zweiten Schritt dieses Schwerpunktes} wurde angestrebt, aus den
Ergebnissen von (1) und (2) einen \emph{Pflichtenheftansatz} zu erarbeiten und
die \emph{Teilaufgabenstellungen (Probleme und Aufgaben) abzuleiten}, zu
präzisieren und erfindungsorientiert zu formulieren, die zur Erfüllung der
Zielsetzung zu lösen sind. Davon ausgehend sind sie nach ihrer
inhaltlich-fachlichen Reihenfolge und unter Beachtung der verfügbaren
Ressourcen sowie zeitlichen Erfordernisse zu ordnen und zu vernetzen, um
daraus einen Operations- oder Arbeitsplan zu gewinnen.

\paragraph{(3) Lösungsfindung und Bewertung.}
In diesem Schwerpunkt wurden die Methoden, heuristischen Prinzipien und
kreativen Denkweisen zur Lösungsfindung mit der Demonstration an
Fallbeispielen vermittelt und das Auswählen und Anwenden der geeigneten
Methoden für die kreative Lösungsfindung an der eigenen Aufgabenstellung aus
der Praxis für die Praxis trainiert.

Für die Lösungsfindung standen besonders im Mittelpunkt:
\begin{itemize}
\item Das Formulieren der erfinderischen Suchfrage für die Lösungsfindung und
  das Variieren des Suchraums, z.B. durch Suchfelderweiterung, Erweiterung des
  Blickwinkels, Feldforschung, Abstraktion, Umkehrung, Vereinfachung.
\item Die Nutzung der Prinzipien zur Lösung von technischen Problemen bzw.
  Widersprüchen z.B. nach Altschuller und später D. Zobel (TRIZ)
  \cite{9.1,9.3}.
\item Die intuitiven und diskursiven Lösungssuchmethoden, z.B. Suchen
  vorhandener Lösun\-gen, Suchen physikalischer Effekte, Suchen analoger
  Bereiche und Lösungen, aber auch Methoden wie Synektik, Delphitechnik oder
  die Brainstormingvarianten.
\item Die Variationsmethoden und die Kombinationsmethoden unter Nutzung der
  Variationsprinzipien und Lösungssuchmethoden für die erkannten Teilprobleme.
  Für die Auswahl und Entscheidung zum Erkennen der optimalen Lösung wurden
  die Methoden zur dualen, gewichteten oder ungewichteten mehrwertigen
  Bewertung angewendet.
\end{itemize}
 
\paragraph{(4) Patentwesen und Patentrecherchen.} 
Vermitteln der Grundlagen und Erarbeiten des Schutzrechtsentwurfs am eigenen
Fallbeispiel.

\subsection*{3.5. Ergebnisse}

Die Entwicklung der Erfinderschulen von 1980 bis 1990 hat deutlich positive
Wirkungen gebracht, sowohl bzgl. der Teilnehmerqualifikation und der in den
Seminaren gewonnenen Problemlösungen als auch für die Entwicklung der
Erfindungsmethodik und die Seminarkonzeption. Die Erfinderschulen erwiesen
sich als eine wirksame Trainingsstätte für schöpferische Tätigkeit und
erfinderisches Schaffen.

Die Mitwirkung der Teilnehmer war überraschend aufgeschlossen, aktiv,
engagiert, kooperativ, teamorientiert. An der Gewinnung kreativer,
anspruchsvoller Lösungen wurde sehr ernsthaft und zielstrebig gearbeitet. Die
Anwendung der Methoden war im Training durch die Begleitung der Trainer
erfolgreich. Die Teilnehmer bewerteten die Seminare zum Abschluss mit sehr
großer Mehrheit als sehr hilfreich und nützlich für ihre Tätigkeit in der
Praxis.

Sie bewirkten weiterhin eine Welle der Aufgeschlossenheit in den Unternehmen
der DDR. Die methodisch-systematischen Arbeitsweisen haben in vielen Betrieben
und Unternehmen „Fuß“ gefasst. Sie waren Anstoß für neue Initiativen und
Formen der Weiterbildung der Ingenieure, wirkten in Jugendforscher-Kollektiven
und befruchteten den Erfinder-Wettbewerb \cite{13,14}. Auch an vielen
Hochschulen und Universitäten wurden die methodisch-systematischen Denk- und
Arbeitsweisen in die Ausbildungsprogramme immer besser integriert.

Trotz der aufgeführten Erfolge gab es noch viel zu tun, um die Erfinderschulen
weiter zu qualifizieren und die Anwendung der Methodik, Denk- und Arbeitsweise
in der Praxis auf breiterer Basis zu festigen. Es war z.B. zu beobachten, dass
die zweite Seminarwoche \emph{nicht immer belegt} werden konnte oder die
erfinderischen Lösungen wegen Investitionsschwäche, unzureichender
Zuständigkeiten oder fehlender F/E-Kapazitäten nicht immer umgesetzt wurden.
Es war nicht überraschend, dass auch auf sich allein gestellte Teilnehmer die
Umsetzung der Erfindungsmethodik in ganzer Breite für sich oder in ihrem Team
nicht erreichten. Dazu wären weitere Aufbauseminare förderlich gewesen.

Exakte Statistiken zur Zahl der durchgeführten Erfinderschulen, der
Teilnehmerzahlen und des Nutzens der erarbeiteten Problemlösungen sind heute
für die Autoren nicht mehr verfüg\-bar. Die Trainer waren ehrenamtlich tätig,
wurden von ihren Arbeitsstellen nur für die Seminare freigestellt und konnten
die Entwicklung der Teilnehmer und die Umsetzung der erarbeiteten
Problemlösungen nur bedingt verfolgen.

Es liegen heute folgende Angaben für eine grobe Einschätzung vor. Von 1980 bis
1985 besuchten mehr als 2\,500 Teilnehmer die Erfinderschulen und wiesen einen
Nutzen im zweistelligen Mio-Betrag in Mark pro Jahr aus. Von 1980 bis 1990
wurden ca. 300 Erfinderschulen mit ca. 7\,000 Teilnehmern registriert. Hierzu
wurden ca. 600 Patentanmeldungen und ca. 1\,000 praxiswirksame Problemlösungen
registriert. Die Dunkelziffer wird erheblich größer geschätzt.

An vielen Technischen Hochschulen arbeiteten progressive Hochschullehrer mit
ihren Studenten mit der Erfindungsmethodik bzw. der methodisch-systematischen
Arbeitsweise erfolgreich an konkreten Projekten der Industrie, vor allem in
Konstruktionslabors, im Ingenieurpraktikum und bei der Diplomarbeit.

\section*{4. Die Fortsetzungs-Phase in der BRD ab 1990}

Mit der Wiedervereinigung zerfielen die Strukturen für die Durchführung der
Erfinderschulen. Die KDT als Dachorganisation wurde abgeschafft, so dass die
Gesamtorganisation, Teilnehmergewinnung, Trainerverpflichtung, die
Bereitstellung der Seminarunterlagen, der Räumlich\-keiten und Versorgung und
nicht zuletzt die Kostenklärung nicht mehr erfolgen konnten. Ebenso kritisch
wirkte die Umstrukturierung und Auflösung der F/E-Einrichtungen und
Unternehmen.  Die Trainer der Erfinderschulen suchten wegen der
Auflösungserscheinungen ihrer Arbeitsstätten neue Wirkungsbereiche, in denen
für eine Mitwirkung an Erfinderschulen kein Platz war.

Analoge Einrichtungen der BRD, wie z.B. das Deutsche Patent- und Markenamt
oder der VDI, die diese Aufgabe hätten wahrnehmen können, fühlten sich für
diese Aufgabe nicht zuständig. Unter diesen Umständen war die breitenwirksame,
den neuen Rahmenbedingungen angepasste Weiterführung der Erfinderschulen nur
bedingt möglich.

Allerdings bestand auch nach 1990 in der BRD die Notwendigkeit, die
Erfindertätigkeit nachhaltig zu aktivieren. Die Notwendigkeit und der Weg dazu
wurden von vielen deutschen Autoren (z.B. Dr. Matthias Heister, Dr. Paul
Krüger vom damaligen Bundesforschungsministerium) behandelt und dargestellt.
So z.B. durch die Arbeiten der Deutschen Aktionsgesellschaft „Bildung –
Erfindung – Innovation e.V.“ (DABEI) im DABEI-Handbuch für Erfinder und
Unternehmer, an dem später auch Aktivisten der Erfindungsmethodik mitwirkten
\cite{15}.

Im Februar 1990 wurde der \emph{Europäische Erfinderverband} im Schloss
Saarbrücken gegründet, initiiert u. a. durch Dr. M. Heister und
Dr. M. Herrlich, sowie im Beisein des damaligen Präsidenten des Deutschen
Patent- und Markenamtes (DPMA), Prof. Dr. Häußer.

\emph{Dr. M. Heister} stellt in einer fundierten Analyse die heute noch
bestehenden Mängel und Rückstände bei der Förderung der Erfindertätigkeit und
problemlösenden Kreativität in Deutschland dar. Er führt dieses Dilemma in
\cite{16} u.a. zurück auf eine Vernachlässigung und einen zu geringen
Stellenwert der Erfindertätigkeit, auf eine zu geringe Bekanntheit des
Vorhandenen in Fachkreisen, nicht hinreichend stimulierende Faktoren und
unzureichende Vergütungen für die Erfindertätigkeit sowie eine ungenügende
Einbindung der Erfindertätig\-keit und problemlösenden Kreativität in die
Wirtschaft, die staatliche Förderung und in die Bildung, das Studium und die
Fortbildung.

Deutschland ist im Feld der Patentanmeldungen und erteilten Patente, die ein
markanter Indikator zukünftiger Innovationskraft sind, international weit
abgeschlagen worden. Von 1840 bis 1960 war Deutschland eine Erfindernation.
Trotz der auf 1960 bezogenen Verdreifachung der F/E-Aufwendungen bis 2014 hat
sich die Zahl der Erfindungen der beim DPMA erteilten Patente von 22\,030 pro
Jahr auf nur 15\,022 im Jahr 2014 reduziert. Die Erteilungsquote der
angemeldeten Patente fiel auf bedenkliche 22,8\%. Alle Universitäten und
Hochschulen des Landes meldeten rd. 670 Patente pro Jahr an, genau so viele
wie die Universität Tokio. China hat Deutschland längst überholt und ist heute
an der Weltspitze. Staat und Universitäten tragen dort gebündelt und
maßgeblich zu dieser Entwicklung bei. Die Erfindungsmethodik und
problemlösende Kreativität war und ist an den deutschen Universitäten und
Hochschulen nur in wenigen Fällen präsent. Ihre breite, wirkungsvolle Nutzung
in der Wirtschaft und Bildung ist bis heute in Deutschland nicht maßgeblich
vorangekommen.

Die Weiterentwicklung der Erfinderschulen und Erfindungsmethodik als konkreter
Bestandteil der problemlösenden Kreativität unter Beachtung der heutigen
Erfordernisse und ihre gesellschaftliche Anerkennung und breite Umsetzung wäre
ein wirkungsvoller Beitrag zur Stärkung der Erfindertätigkeit, Kreativität und
Innovationskraft unseres Landes. Es sollte in Anbetracht dieser kritischen
Entwicklung ein „Weckruf“ durch die potentiell verantwortlichen Einrichtungen
gehen.

Unabhängig davon entstanden anspruchsvolle „Insellösungen“, z.B.: gründete

\paragraph{a) Dr.-Ing. Michael Herrlich (Verdienter Erfinder)} 
hat nach der Wiedervereinigung Deutschlands trotz aller Hemmnisse und auf
Grund der Notwendigkeit, die Erfindertätigkeit in Deutschland zu aktivieren,
die \emph{Deutsche Erfinderakademie
e.V.}\footnote{\url{https://www.deutsche-erfinder-akademie.de/}} in Leipzig
für die Durch\-füh\-rung von \emph{Erfinder- und Innovationsmanager-Seminaren}
gegründet.

Herrlich erreicht mit seinem aktualisierten Seminarkonzept regen Zuspruch und
attraktive Ergebnisse \cite{17}. Die Absolventen dieser Seminare haben meist
doppelt so viel Erfolg wie vorher oder gegenüber dem Durchschnitt im Land. Von
den nach 1990 qualifizierten rd. 7\,000 Erfinderschülern in Deutschland, der
Schweiz und Österreich haben 23\% bereits ein Jahr nach den Erfinder- und
Innovationsmanager-Seminaren so starke Patente angemeldet, dass von diesen bis
zu 80\% erteilt wurden.

Die Seminare werden vorwiegend von ehemaligen Seminaristen in Zusammenarbeit
mit den örtlichen Innovationsbeauftragten der Kommunen, IHK, HWK,
Volkshochschulen, Unternehmen und Instituten organisiert. Die Zielgruppe für
die Seminare sind kreative Leiter und Mitarbeiter, die Probleme besser
erkennen und lösen wollen.

Mit dem von M. Herrlich modifizierten Konzept der Erfinderschulen wird das
Ziel angestrebt, die Teilnehmer in Einheit von „Wissen – Wollen – Können --
Handeln“ zum niveauvollen Erfinden zu führen. Es werden vorwiegend
geschlossene Seminare für Unternehmen und Einrichtungen angeboten. Die
Erfinder- und Innovationsseminare umfassen analog zu den Erfinderschulen meist
zwei Wochenseminare, die von einer einmonatigen Selbstarbeitsphase an der
eigenen Aufgabenstellung in den Unternehmen unterbrochen wird. Je nach
Kundenbedarf kann der zeitliche Umfang für die Teilnehmer auch variiert
werden.

Diese Erfinder- und Innovationsseminare befassen sich mit folgenden
inhaltlichen Schwerpunkten:
\begin{itemize}
\item Der erfinderisch denkende und handelnde Mensch.
\item Gesetzmäßigkeiten der internationalen Bedürfnis- und Trendentwicklung.
\item Rationelles Informieren zum Stand der Technik und Zuspitzung des
  Problems durch Aufbereiten, Defektliste, Arbeitsplan, Generieren des
  Prinzips, Pflichtenheft, Entscheiden.
\item Unterstützung des methodengestützten Erfindens durch Nutzung der
  dialektischen Widersprüche und ihrer Lösung mit Hilfe der bekannten
  Lösungsprinzipien nach Altschuller und Zobel \cite{3.1,9.1}.
\item Erarbeiten eines Schutzrechtsentwurfes.
\item Überleitung der erfinderischen Lösung in die gewinnbringende Verwertung.
\item Das optimale Erfinderteam und Unterstützungsmöglichkeiten für Erfinder.
\end{itemize}

\paragraph{b) Dr.-Ing. Hansjürgen Linde (Verdienter Erfinder)}
aus Gotha gründete nach der Wiedervereinigung und nach seiner
Ingenieurtätigkeit bei BMW in München und ab 1992 als Professor an der
Fachhochschule Coburg nebenamtlich ein privates
Institut\footnote{\url{https://www.wois-innovation.de/}}, das Seminare unter
dem Thema „Erfolgreich Erfinden -- Widerspruchsorientierte
Innovationsstrategien für Entwickler und Konstrukteure“ bundesweit
durchführte.  Dazu erschien 1993 auch sein gleichnamiges Buch im
Hoppenstedt-Verlag Darmstadt \cite{18}. Linde baute ein Team mit jungen
Mitarbeitern auf. Diese Seminare wurden auch von namhaften Unternehmen aus der
gesamten BRD genutzt.  Die inhaltliche Basis und Orientierung der Seminare
war, bezogen auf die Erfinderschulen und die Seminare von M. Herrlich,
gleichartig. Nach dem Ableben von Prof. Dr. Linde ist kein
erfindungsmethodischer Lehrstuhl bekannt.

\section*{5. Zusammenfassung}
Im Rahmen der Erfinderschulen entwickelte sich die Erfindungsmethodik und das
Seminarkonzept auf einen Stand, mit dem eine wirksame, nachhaltige Förderung
der Kreativität, des Schöpfertums und nicht zuletzt einer niveauvollen
Erfindertätigkeit im Bereich Wissenschaft und Technik erreichbar ist. Die
Erfinderschulen erwiesen sich als echte, erfolgreiche Trainingstätten für
schöpferische Tätigkeiten, auch für Problemlösungsprozesse im
interdisziplinären Team.

Es wurde nachvollziehbar erkennbar: Kreative Erfindertätigkeit und
schöpferisches Innovationsmanagement sind im Rahmen von trainingsorientierten
Workshopseminaren lehr- und lernbar. Engagierte und begabte Teilnehmer können,
unterstützt durch die Erfindungsmethodik, in Erfinderseminaren bzgl. ihrer
Erfindertätigkeit nachhaltig erfolgreicher werden und sich durch Training
befähigen, an kreativer, interdisziplinärer, methodisch moderierter Teamarbeit
im Unternehmen erfolgreich teilzunehmen oder als Moderatoren wirksam zu
werden.

Die Fortsetzung der Erfinderschulen nach der Wiedervereinigung ist in
einzelnen Fällen erfolgreich gelungen. Die breite Anwendung in der Wirtschaft
und Bildung wurde bisher noch nicht durchgesetzt. Dazu führten auch heute noch
die Hemmnisse, wie z.B. der geringe Stellenwert und die unzureichende
Anerkennung und Förderung der problemlösenden Kreativität und
Erfindungsmethodik in der Gesellschaft, Wirtschaft und Bildung sowie ein
geeigneter organisatorischer Gesamtrahmen. Zusätzlich zur Überwindung dieser
Hemmnisse ist auch die Weiterentwicklung des Konzeptes für Erfinderschulen
notwendig.

\begin{quote}
  Die Erfindungsmethodik und das Seminarkonzept samt einer adaptiven
  Weiterentwicklung eignen sich auch heute für das in Deutschland sehr
  notwendige Aktivieren einer effizienten Erfindertätigkeit mit
  anspruchsvollen Ergebnissen in größerer Breite.
\end{quote}

Ebenso besteht die Chance, an den Universitäten und Hochschulen die
praxisorientierte Aus- und Fortbildung im Sinne einer verbesserten
Kompetenzbildung durch die Integration der kreativen,
methodisch-systematischen Denk- und Arbeitsweisen im Allgemeinen und der
Erfindungsmethodik im Speziellen so zu gestalten, dass die Befähigung der
Absolventen zu bewusster schöpferischer Arbeit gezielt verbessert wird, um sie
damit erfolgreicher zu machen.

Für die Förderung einer anspruchsvollen Erfindertätigkeit und der
Innovationsprozesse in Deutschland ist die \emph{breitere Anwendung und
Weiterentwicklung} der bisher gewonnenen Ergebnisse und Erfahrungen mit den
Erfinder- und Innovations-Seminaren eine anspruchsvolle Herausforderung, für
die vor allem eine geeignete, nah an der Praxis tätige, leistungsstarke
\emph{Trägerschaft} für die organisatorische und materielle Unterstützung
gewonnen werden sollte.

Vorschläge an die Bundeskanzlerin, die Zuständigkeit für die
erfindungsorientierte Fortbildung in F/E dem DPMA zu übertragen und für diesen
Zweck angemessene Fördermittel bereit zu stellen, hatten bisher noch keinen
Erfolg, obwohl bekannt ist, dass „Erfinderunternehmen“ in der Tendenz im
Vergleich zum jeweiligen Branchendurchschnitt einen um den Faktor 3 bis 25
höheren Gewinn erreichen und auch in Zukunft wesentlich bessere
Erfolgsaussichten haben. Es bedarf in diesem Sinne auch in der Gesellschaft
zukünftig weitere koordinierte Bemühungen, um das erfinderische Schaffen in
Deutschland in größerer Breite nachhaltig zu beleben. Das gilt natürlich für
den gesamten Innovationsprozess, denn in Fachkreisen ist klar, dass eine
Erfindung erst erfolgreich ist, wenn der „Markt HURRA schreit“.

\begin{thebibliography}{xxx}
\bibitem{1} Peter Koch. Entwicklung der Konstruktionswissenschaften von 1950
  bis 1990. Beitrag Nr. 9 im Kapitel Geschichte/Historie der Homepage
  „Problemlösende Kreativität“.
\bibitem{2} Johannes Müller, Peter Koch (Hrsg.). Programmbibliothek zur
  Systematischen Heuristik für Naturwissenschaftler und Ingenieure. ZIS Halle,
  TWA Nr. 97, 98, 99, Halle/Saale 1973.
\bibitem{3.1} Genrich S. Altschuller. Erfinden -- (k)ein Problem (ARIS).
  Verlag Tribüne, Berlin 1973.
\bibitem{3.2} Genrich S. Altschuller. Erfinden -- Wege zur Lösung technischer
  Probleme (TRIZ). Verlag Technik, Berlin 1984.
\bibitem{4} Peter Koch. Der Konstruktionsprozess und das Analysieren der
  Aufgabenstellung und technischer Gebilde. Konstruktionstechnik,
  1.~Lehrbrief, Verlag Technik, Berlin 1974.
\bibitem{5} Peter Koch.  Lösungsfindung in der Prinzipphase.
  Konstruktionstechnik, 2.~Lehrbrief, Verlag Technik,  Berlin 1974.
\bibitem{6} Günther Höhne, Peter Koch. Anwendung der Variationsmethode beim
  Konstruieren. Maschinenbautechnik, Berlin 23 (1976), S. 183-186.
\bibitem{7} Michael Herrlich, Gerhard Zadek (Hrsg.). KDT-Erfinderschule --
  Lehrmaterial, 2~Teile. KDT, Berlin 1982.
\bibitem{8} Hans-Jochen Rindfleisch, Rainer Thiel. Programm zum Herausarbeiten
  von Erfindungsaufgaben und Lösungsansätzen in der Technik. Lehrbrief der
  KDT-Erfinderschule, Berlin 1989.
\bibitem{9.1} Dietmar Zobel. Erfinderfibel. Deutscher Verlag der
  Wissenschaften, Berlin 1985.
\bibitem{9.2} Dietmar Zobel. Erfinderpraxis -- Ideenvielfalt durch
  Systematisches Erfinden.  Deutscher Verlag der Wissenschaften, Berlin 1991.
\bibitem{9.3} Dietmar Zobel. TRIZ FÜR ALLE. Der systematische Weg zum
  Problemlösen. Expert Verlag, Renningen, 2. Auflage 2007.
\bibitem{10} Volker Heyse, Jürgen Bausdorf (Hrsg.). Grundlagen des
  wissenschaftlich-technischen Schöpfertums in Forschung und Entwicklung.
  Berlin/Jena: Bauakademie, Carl Zeiss Jena. Lehrbriefe, 3.~Auflage mit 17
  Heften (1986).  Besonders die Lehrbriefe 2, 5, 6, 10-12.
\bibitem{11} Peter Koch. Zur Entwicklung erfinderischer Aufgabenstellungen
  durch die Nutzung der Widerspruchsanalyse bei der Problemerkennung und
  -Präzisierung. Maschinenbautechnik, Berlin 37 (1988), 340-343.
\bibitem{12} Michael Herrlich. Erfinden als Informationsverarbeitungs- und
  Generierungsprozess. Diss. TH Ilmenau 1987.
\bibitem{13} Michael Herrlich. Kapitel 2.2, „Methodische Grundlagen des
  niveauvollen erfinderischen Schaffens“ in: Hemmerling (Hrsg.).
  Erfinderhandbuch. Verlag die Wirtschaft, Berlin 1988.
\bibitem{14} Hans-Jochen Rindfleisch, Rainer Thiel. Erfinderschulen in der
  DDR.  Trafo-Verlag, Berlin 1994
\bibitem{15} DABEI-Handbuch für Erfinder und Unternehmer. Von der Idee zum
  Produkt und zur Vollbeschäftigung. VDI-Verlag, Düsseldorf 1987.
\bibitem{16} Matthas Heister. Bildung, Erfindung, Innovation. Bd. 2,
  Expertenwissen für Erfinder und Unternehmer. Verlag Iduso GmbH, Bonn 2013.
\bibitem{17} Michael Herrlich: „Ingenieure und Naturwissenschaftler im
  Strukturwandel – Warum und wie man Erfinden lernen sollte“. In Staudt,
  E. (Hrsg.): Strukturwandel und Karriereplanung. Berlin, Springer Verlag
  1988.
\bibitem{18} Hansjürgen Linde, Bernd Hill. Erfolgreich Erfinden –
  Widerspruchsorientierte Innovationsstrategien für Entwickler und
  Konstrukteure. Hoppenstedt Technik, Darmstadt 1993.
\bibitem{19} Michael Herrlich: Was können Erfinderschulen für die
  schöpferische Befähigung leisten. In Neuner, G. (Hrsg.): Leistungsreserve
  Schöpfertum.  Dietz-Verlag, Berlin 1986
\bibitem{20} Heyde, Pätzold. Förderung des schöpferischen Denkens und der
  Initiative zur rationellen Lösung wissenschaftlich-technischer Aufgaben.
  Internes Lehrmaterial der Kammer der Technik.
\end{thebibliography}
\end{document}
