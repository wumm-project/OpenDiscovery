\documentclass[11pt,a4paper]{article}
\usepackage{od}
\usepackage[utf8]{inputenc}
\usepackage[ngerman]{babel}

\title{Erfindungsmethodische Arbeitsmittel.\\ Lehrmaterial zur
Erfindungsmethode} 
\author{Jochen Rindfleisch, Rainer Thiel, Gerhard Zadek}
\date{KdT-Präsidium, Berlin 1989} 

\newcommand{\derTitel}{
%Schmutztitel
Baustein
KDT-Erfinderschule
Lehrbrief 2

Erfindungsmethodische Arbeitsmittel 
Lehrmaterial zur Erfindungsmethode

Verdienter Erfinder Dr.-Ing. Jochen Rindfleisch
Dr. habil. Rainer Thiel
Dipl.-0k. Ing. Gerhard Zadek (Leiter des Autorenkollektivs)

KAMMER DER TECHNIK
Präsidium
Sekretariatsbereich Weiterbildung
Kommission wissenschaftlich-technisches Schöpfertum 
1989
%Zweite Seite
Erfindungemethodische Arbeitsmittel Lehrmaterial / Rindfleisch, 
Jochen; Thiel, Rainer ; Zadek, Gerhard. - Berlin : Präsidium
d. Kammer d. Technik. 1989 -- 96 S.
(Bausteinesystem Leitung, Planung, Erhöhung der Produktivität, 
Effektivität und Qualität wissenschaftlich-technischer Arbeit; 
KDT-Erfinderschule, Lehrbrief 2)

1. Auflage
(C) by Präsidium der KDT, Clara-Zetkin-Str. 115/117. Berlin 1086.
I 12 4 Ag 238/114/89
Printed in the German Democratic Republic
Redaktionsschluß: 15. Februar 1989
Jede Vervielfältigung -- auch auszugsweise -- ist nur mit Genehmigung dee
Herausgebers gestattet. 
}

\begin{document}
\maketitle

\section*{Inhaltsverzeichnis}                                   
\begin{itemize}
\item[A.] Programm „Herausarbeiten von Erfindungeaufgaben und Lösungsansätzen
  in der Technik“ -- Erfindungsprogramm der KDT-Erfinderschulen
\item [B.] Erfindungsmethodische Arbeitsblätter
\end{itemize}
Zum Gelingen dieses Lehrmateriels haben durch förderliche Hinweise
beigetragen:
\begin{itemize}[noitemsep]
\item Teilnehmer und Trainer der KDT-Erfinderschulen
\item Mitglieder der AG(Z) „Erfindertätigkeit/Schöpfertum“
\end{itemize}
sowie
\begin{itemize}[noitemsep]
\item Dr. E. Heyde, AfEP
\item Dr.-Ing. H.-J. Linde, VEB Ingenieurbüro und Mechanisierung Gotha
\end{itemize}
\newpage

\section*{A. Programm „Herausarbeiten von Erfindungsaufgaben und
Lösungsansätzen in der Technik“ -- Erfindungsprogramm}
\begin{itemize}
\item[1.] Das gesellschaftliche Bedürfnis. Vorläufige Systembenennung
\item[2.] Stand der Technik. Vorauswahl und Systemanalyse einer
Startvariante. Die bedürfnis\-gemäße Variation der Systemparameter
\item[3.] Das Operationsfeld des Erfinders
\item[4.] Der technisch-ökonomische Widerspruch
\item[5.] Der schädliche technische Effekt (stE)
\item[6.] Das IDEAL. Anstoß und Orientierung zu vertiefter Systemanalyse
\item[7.] Der technisch-technologische Widerspruch (ttW)
\item[8.] Der technisch-naturgesetzmäßige Widerspruch 20 (tnW)
\item[9.] Die Strategie zur Widerspruchslösung
\item[10.] Die eigene Erfindung als Schrittmacher in der internationalen
Entwicklung
\item[A.] Drei Abbildungen (5 Blatt)
\end{itemize}
\newpage

\subsection*{1. Das gesellschaftliche Bedürfnis. Vorläufige Systembenennung.}

Das auftragsgemäß zu erneuernde technische System und seine Unterstellung
unter gesellschaftliche Bedürfnisse -- die technisch-ökonomische Zielstellung.

1.1. Welche Funktion soll das auftragegemäß zu erneuernde technische System
erfüllen? In welchen übergeordneten Nutzungsprozeß soll diese Funktion als
Teilfunktion eingebunden sein?  Benutze zum Aufzeichnen die
Black-box-Darstellung.

1.2. Welchem speziellen Bedürfnis der Gesellschaft (bzw. des Exportkunden)
soll dieser über\-geordnete Nutzungsprozeß dienen?  Welche
Gebrauchseigenschaften und Eignungsmerkmale dieses Systeme sind notwendig und
hinreichend, damit es dem übergeordneten Nutzungsprozeß besser als bisher
entspricht? Welches spezielle Bedürfnis kommt darin zum Ausdruck? Welche
Nutzungsprozesse sind in In- und Ausland bekannt, die einem vergleichbaren
Bedürfnis dienen?

1.3. Analysiere Literatur, Patente, Forschungsberichte, Marktinformationen,
Reiseberichte.

1.4. Wie lange gibt es das spezielle Bedürfnis schon? Wie hat es sich
entwickelt? Welche Bedingungen für die Verwendung und welche Anforderungen an
die Gebrauchseigenschaften und Eignungsmerkmale haben sich mit der Entwicklung
des Nutzungsprozeesee verändert? Zeige mögliche Tendenzen der weiteren
Entwicklung auf. Läßt sich eine Tendenz finden, die bisher nicht gesehen
wurde?

1.5. Mit welcher Hauptfunktion erfüllt das auftragsgemäß zu erneuernde
technische System seinen spezifischen Zweck im übergeordneten Nutzungsprozeß?
Welche seiner Gebrauchseigenschaften sind dafür kennzeichnend?  Welchen
Anforderungen und Bedingungen müssen sie genügen?

1.6. Welche \emph{allgemeinen}, übergreifenden \emph{gesellschaftlichen}
Bedürfnisse sind zu beachten?  Warum sind sie entstanden? Wie haben sie sich
entwickelt? Wie werden sie sich voraussichtlich entwickeln? Welche
Restriktionen in bezug auf die Nutzung von Ressourcen und welche Erwartungen
in bezug auf den Nutzungseffekt ergeben sich daraus?

1.7. Welche \textbf{A}nforderungen, \textbf{B}edingungen, \textbf{E}rwartungen
und \textbf{R}estriktionen (ABER) bestimmen die erforderliche Entwicklung der
\emph{gesellschaftlichen} Effektivität des technischen Systeme? Nenne die ABER
vollständig und begründe sie. Prüfe, ob sie nicht aus subjektiven Auffassungen
oder Vorurteilen resultieren.  Welches Entwicklungsziel folgt aus den ABER?

1.8. Welche spezifischen ABER bestimmen die Zielgrößenkomponenten
\begin{itemize}[noitemsep]
\item Zweckmäßigkeit    (Z1) 
\item Wirtschaftlichkeit(Z2) 
\item Beherrschbarkeit  (Z3) 
\item Brauchbarkeit     (Z4) 
\end{itemize}
des zu schaffenden technischen Systems als Ganzes? (Zielgrößenmatrix)

1.9. Welche Prioritäten ergeben sich aus den ABER für die einzelnen 
Merkmale des auftragegemäß zu erneuernden technischen Systems, die 
den vier Zielgrößenkomponenten zugeordnet sind?

1.10. Welche Zusammenhänge zwischen diesen Merkmalen bzw. Eigenschaften --
kooperative oder gegenläufige -- lassen sich in der Zielgrößenmatrix abheben?

\subsection*{2. Stand der Technik. Vorauswahl und Systemanalyse einer
  Startvariante. Bedürfnisgemäße Variation der Systemparameter} 

2.1. Welches ist das für die Realisierung der Zielgröße am besten 
geeignete \emph{technisch-techno\-logische Prinzip}?

a) Untersuche die auf dem Stand der Technik bekannten Prinzipien der
Herstellung und/oder Nutzung technischer Objekte aus Deinem Technologiebereich
auf Eignung in bezug auf die ABER. Wähle das technisch-technologische
Prinzip,
\begin{itemize}
\item das dem Zweck des zu schaffenden technischen Systems (Zielkomponente Z1)
  am meisten entspricht,
\item und mit dessen Anwendung voraussichtlich nicht oder vergleichsweise
  wenig gegen Anforderungen und Restriktionen verstoßen wird,
\item und das die Bedingungen und Erwartungen ohne wesentlichen Zusatzaufwand
  zu erfüllen erlaubt.
\end{itemize}
Hierbei sind die verfügbaren und alle machbar erscheinenden Mittel und
Verfahren auf dem Stande der Technik in Betracht zu ziehen.

b) Ist ein technisch-technologisches Prinzip mit der Aufgabenstellung
verbindlich vorgegeben, so überprüfe es auf seine Eignung und vergleiche es
mit anderen bekannten Prinzipien. Nimm gegebenenfells Rücksprache mit dem
Auftraggeber.

c) Ist ein geeignetes technisch-technologisches Prinzip im eigenen
Technologiebereich nicht auffindbar, ist die Suche auf weitere, auch fern
liegende Bereiche auszudehnen.

d) Formuliere \emph{technisch-ökonomische Parameter} (Effektivitäteparameter)
so, daß sie dem technisch-technologischen Prinzip gemäße Maßgrößen für die
Eigenschaften sind, welche durch die Zielgrößenkomponenten gefordert werden.

2.2. Welche Arten von Objekten müssen in Betracht gezogen werden, um das
technische System dem technisch-technologischen Prinzip entsprechend nutzbar
zu machen?

a) Welche Gebrauchseigenschaften müssen die Vertreter der einzelnen
Objektarten haben, damit sie entsprechend Z1 für die Verwendung im technischen
System geeignet sind?

b) Welche Objektart trägt am meisten zu den Effektivitäts- und
Eignungsmerkmalen des technischen Systeme bei?

c) Sind weitere Objektarten mit spezifischen Gebrauchseigenschaften in
Betracht zu ziehen, um allen notwendigen Eignungs- und Effektivitätsmerkmalen
des technischen Syeteme hinreichend im Hinblick auf Z3 und Z4 Rechnung tragen
zu können?

2.3. Welche Hauptfunktion hat der Nutzungsprozeß des technischen Systems?

a) Mit welchen notwendigen Teilfunktionen ist die Hauptfunktion gemäß
technisch-techno\-logischem Prinzip zu verwirklichen?

b) Durch welche Teilfunktionen werden welche Objekte auf welche Weise in den
Nutzungsprozeß einbezogen?

c) Wie werden dadurch ihre Gebraucheeigenschaften aktiviert?

d) Gibt es eine Teilfunktion, durch die besondere viele Objekte in den Prozeß
einbezogen und aktiviert werden?

e) Gibt es Objekte, welche durch mehrere Teilfunktionen auf unterschiedliche
Weise in den Prozeß einbezogen werden?

f) Durch welche notwendigen Funktionseigenschaften läßt sich die
prozeßgerechte Wirkungsweise, und durch welche Struktureigenschaften läßt sich
der erforderlicha Aufbau und die zweckmäßige Anordnung der einzelnen Objekte
(technischen Mittel) kennzeichnen?

2.4. Welche für den Nutzungsprozeß gemäß 2.3. geeigneten technischen Mittel
sind zu den einzelnen Objektarten gemäß 2.2. auf dem Stand der Technik
verfügbar oder bekannt?

a) Gibt es ein technisches System auf dem internationalen Stand der Technik,
welches die notwendigen Eignungsmerkmale gemäß Z1, Z3, Z4 prinzipiell besitzt?
Ist dieses System verfüg\-bar?  Wähle dieses System als Referenzvariante, auch
wenn es nicht auf dem gewählten technisch-technologischen Prinzip beruht.

b) Welche der einzelnen erforderlichen Mittel gemäß 2.3.f) gibt es auf dem
Stand der Technik? Welche sind verfügbar? Welche sind machbar?

c) Welche gemäß 2.3.f) notwendigen technischen Mittel sind auf dem Stande der
Technik weder verfügbar noch bekannt?

d) Wie wären technische Mittel gemäß 2.4.c) auf dem Stand der
Technikwissenschaften denkbar?

e) Welche der in Betracht gezogenen technisches Mittel lassen sich aufgrund
ihrer Mittel-Wirkungs-Beziehungen miteinander zu einer Basisvariante
verknüpfen? (Eventuell morphologisches Schema) 

f) Welche funktionellen Anforderungen, strukturellen Bedingungen sowie
naturgesetzmäßigen Einflüsse und Restriktionen (ABER) sind dabei zu
berücksichtigen?

g) Bei welchen neuartigen technischen Mitteln treten demgemäß die meisten
Unvereinbarkeiten auf? Bei funktionsbestimmenden oder bei untergeordneten
technischen Mitteln?

h) Worin bestehen diese Unvereinbarkeiten? Lassen sie sich durch Verlagerung
auf untergeordnete technische Mittel und geeignete Variation ihrer Funktions-
und Struktureigenschaften beheben?

2.5. Welchc technisch-ökonomischen Mängel hzw. technisch-technologischen
Defekte besizt die mit bekannten technischen Mitteln bestenfalls erreichbare
Basisvariante?

a) In welchen technisch-technologischen Eignungsmerkmalen weicht die
bevorzugte Basisvariante von der Zielgröße voraussichtlich am stärksten ab?

b) In welchen technisch-ökonomischen Hauptleistungsdaten weicht sie
voraussichtlich von der Sollgröße am stärksten ab?

c) In welchen Eignungs- und Effektivitätsmerkmalen ist die Basisvariante der
Referenzvariante prinzipiell überlegen?

d) Welche neuen technischen Mittel sind notwendig und denkbar, um mit der
Basisvariente die Sollgröße zu erreichen und die Referenzvariante in allen
Hauptleistungsdaten zu übertreffen?

e) Wie lautet die technisch-ökonomieche Zielstellung der notwendigen
technischen Entwicklung?

f) Welcher Hauptleistungsparameter liegt ihr als Führungegröße zu Grunde?

2.6. Fasse das technische System, das die Zielgröße realisieren soll,
insgesamt als black box auf. Mit welchen Eingängen und Ausgängen realisiert
das technische System in der gewählten bzw. vorgefundenen Aueführungsform das
spezielle gesellschaftliche Bedürfnis?

Beschreibe die Ein- und Ausgangsgrößen in auftragerelevanten Bestimmungen der
Art, der Zusammensetzung, der Struktur und des Zustandes von Stoff, Energie
urd Information.

2.7. Durch welches Verfahrensprinzip wird bei der gewählten Basisvariante die
zweckbestimmte Eingangs-/Ausgangsrelation (Überführungsfunktion) realisiert?

a) Nenne die funktionellen Merkmale der wesentlichen Teilsysteme für die
Realisierung der Hauptfunktion!

b) Nenne die dabei zu erzielenden notwendigen Zwischenstadien der
Eingangs-Ausganga-Tranaformation der Zustandagrößen von Stoff, Energie und
Information.

2.8. Welche technischen Wirkprinzipe liegen bei der Basisvariante der
Hauptfunktion zugrunde?

a) Untersuche das technische Wirkprinzip jeder Elementarfunktion:

Durch welchen Operator soll welche Einwirkung (welche Operatien auf welches
Objekt (Operand) ausgeübt werden?

Welche Rückwirkung (Gegenoperation) ist dafür erforderlich und durch welchen
Gegenoperator wird sie hervorgerufen?

Welche Auswirkungen ergeben sich aus dem Zusammenwirken von Operator und
Gegenoperator in dem zu verändernden Objekt?

b) Kennzeichne Art und Weise der konstruktiven bzw. verfahrenstechnischen
Verknüpfung der (elementaren) Funktionseinheiten zur Struktureinheit der
Hauptfunktion.

c) Konfrontiere die technische Wirksamkeit -- den Funktionswert -- der
einzelnen Elementarfunktionen und der Hauptfunktion als Ganzem mit den
Anforderungen an die Gebrauchseigenschaften des technischen Systems.

2.9. Enthält das technische System für den vorgesehenen Verwendungzweck
überflüssige Elementarfunktionen?

2.10. Welche Nebenwirkungen der Hauptfunktion treten auf bzw. sind bei
vorgesehenen technisch-technologischen Maßnahmen zu erwarten?

a) Untersuche die einzelnen Elementarfunktionen in der Wirkungskette der
Hauptfunktion und die ihnen zugrunde liegenden Wirkprinzipe auf technisch
und/oder naturgesetzlich bedingte Nebenwirkungen.

b) Unterscheide dabei nützliche, verfügbare und schädliche, zu unterdrückende
Nebenwirkungen.

2.11. Wodurch sind die Nebenwirkungen verursacht?

a) Nenne die konstruktiv bzw. technologisch und die naturgesetzlich
determinierten Anforderungen, Bedingungen, Einflüsse und Restriktionen (ABER),
auf Grund derer die Nebenwirkungen entstehen bzw. nicht ohne weiteres
unterdrückt werden können.

Diese ABER ergeben sich für ein technisches Gebilde aus den
technisch-konstruktiven Merkmalen seines Aufbaus und/oder den
technisch-technologischen Merkmalen seiner Herstellung, und für ein
technisches Verfahren aus den technisch-technologischen Merkmalen seines
Ablaufs und/oder den technisch-konstruktiven Merkmalen des Aufbaus des mit ihm
herzustellenden technischen Gebildes.

b) Konfrontiere die Nebenwirkungen und den Grad ihrer Nutzung bzw.
Unterdrückung mit den gesellechaftlich-ökonomischen Anforderungen,
Restriktionen, Erwartungen und Bedingungen (ABER), welche sich aus den
übergreifenden gesellschaftlichen Bedürfnissen ergeben.

c) Ermittle die nachteiligste Nebenwirkung.

2.12. Gibt es technische Mittel (Operatoren) zur Realisierung der
Hauptfunktion, für die international bereits andere Wirkprinzipe genutzt
werden?

2.13. Welche Anforderungen, Bedingungen und Restriktionen
gesellschaftlich-ökonomischer, technisch-technologischer und/oder
schutzrechtlicher Art behindern die Einführung international bekannter
Lösungen in das technische System?  

2.14. Untersuche die Funktionseinheiten des technischen Systems, ob sie
Nebenfunktionen enthalten, die geeignet sind, Nebenwirkungen besser als bisher
nutzbar zu machen oder schädliche Nebenwirkungen zu unterdrücken oder sogar in
nützliche zu verwandeln.

2.15. Welches Verhalten dee technischen Systeme ist zu erwarten, wenn Werte
der technisch-ökonomischen Parameter erhöht werden?

a) Variiere die Werte jedes'einzelnen technisch-ökonomischen Parameters gemäß
technisch-ökonomischer Zielstellung bis an die im Auftrag geforderten
Grenzwerte und darüber hinaus. Beachte dabei die Rangfolge in der
gesellschaftlich-ökonomischen Wichtung der Parameer der Zielgröße Z.

b) Untersuche, welche technischen Systemparameter -- Leitgrößen
(Führungsgröße, Strukturgröße, Wirkgröße; vgl. Lehrmaterial
„Erfindungsmethodische Grurdlagen“, Abschnitt 1.5.) -- dazu in welcher
Richtung und in welchem Maße verändert werden müßten.

c) Untersuche, ob dann die technisch-technologische Wirksamkeit der einzelnen
Elementarfunktionen in der Wirkungskette der Hauptfunktion den ABER gemäß
gewährleistet bleibt, oder ob schädliche Effekte und damit
technisch-ökonomische Widersprüche entstehen.

d) Untersuche, welche Elementarfunktion auf Grund ihres Wirkprinzips und/oder
auf Grund der vorliegenden ABER die Verbesserung der Parameterwerte primär
begrenzt.

e) Untersuche, wie sich das Verhältnis von Haupt- und Nebenwirkungen (bezogen
auf jede einzelne Elementarfunktion und auf das gesamte technische System) mit
der Variation der technisch-ökonomischen Parameter verändert. Stelle fest, ob
die schädlichen Nebenwirkungen durch die Nutzung vorhandener Nebenfunktionen
besser beherrscht werden können.

3. Das Operationsfeld des Erfinders

3.1. Welche Teilsysteme (Baugruppen, Bauteile, Verfahrensstufen,
Verfahrensschritte) des technischen Systems sind auf Grund von
gesellschaftlich-ökonomischen Restriktionen und technisch-technologischen
Bedingungen einer Veränderung nicht zugängig und daher der
tech\-nisch-technologischen Umgebung zuzuordnen?

3.2. Welche stofflichen, energetischen und/oder informellen Komponenten des
technisch-tech\-nologischen Umfeldes können bzw. müssen in die
Systembetrachtung mit einbezogen werden?

Untersuche, ob es bestimmte Komponenten der technisch-technologischen Umgebung
des Systems oder sogar des gesellschaftlichen Obersystems gibt, die als
Operatoren in der Wirkungskette der Hauptfunktion oder die im Sinne von
Nebenfunktionen genutzt werden können.

3.3. Grenze das technische System bzw. die entsprechende Basisvariante
entsprechend den Antworten auf die Fragen 3.1. und 3.2. neu ab. Bestimme seine
Aus— und Eingangsgrößen, seine Hauptfunktion sowie die ihm zugehörigen
technisch-technologischen Bedingungen entsprechend neu.

3.4. Welches Teilsystem stellt für die Erhöhung der Werte der
technisch-ökonomischen Parameter im Sinne des gesellschaftlichen Bedürfnisses
eine primäre Barriere dar?

a) Stelle fest, zu welchem Teilsystem (Baugruppe, Bauteil, Verfahrensstufe,
Verfahrensschritt) das effektivitätsbegrenzende technische Mittel (Operator)
gehört bzw. in welchem Teilsystem sich das Verhältnis von Haupt- und
Nebenwirkung bei Erhöhung der Werte von technisch-ökonomischen Parametern am
stärksten zu Ungunsten der Hauptwirkung verändert.

b) Bestimme die Ein- und Ausgangegrößen dieses Teilsystems und stelle die
Wirkungskette seiner Elementarfunktion dar.

4. Der technisch-ökononomische Widerspruch

4.1. Untersuche, wie die technisch-ökonomischen Parameter der Zielgröße bei
dem in Betracht gezogenen Stand der Technik (Basisvariante) durch das ihnen
zugrunde liegende System der technisch-technologischen Parameter der
Basisvariante miteinander verknüpft sind. Bestimme den
technisch-technologischen Parameter des technischen Systems, der den stärksten
Einfluß auf die technisch-ökonomische Effektivität gemäß Zielgröße hat. Wähle
ihn als Führungsgröße.

4.2. Läßt sich durch Variation der Werte der Führungsgröße das erforderliche
Wachstum aller technisch-ökonomischen Parameter erzielen? Oder ist das
erforderliche Wachstum einzelner Paramter nur bei Abnahme anderer
technisch-ökonomischer Parameter erreichbar?

a) Stelle die Entwicklung der technisch-ökonomischen Effektivität des zu
betrachtenden technischen Systems als Funktion der Verbesserung seiner
technisch-ökonomischen Parameter dar. Gewährleiste, daß dabei die Interessen
der Volkswirtschaft insgesamt zum Ausdruck kommen.

b) Zeige, daß unter dem Gesichtspunkt der zu steigernden Effektivität die
Entwicklung tech\-nisch-ökonomischer Parameter widersprüchlich geworden ist
(Widersprüche \emph{zwischen} Parametern hinsichtlich ihres Beitrages zur
Effektivitätesteigerung und Widersprüche zwischen den Konsequenzen der
Entwicklung des einen oder anderen Parameters):
\begin{itemize} 
\item Nenne die Parameter, deren Einfluß auf des Effektivitätswachstum sich
  zunehmend spaltet in einander entgegengesetzte Einflüsse (innerer
  Widerspruch in der Entwicklung eines technisch-ökonomischen Parameters).
\end{itemize}

Der Rest des Textes steht in der zweiten pdf-Datei.
\end{document}
