\documentclass[a4paper,11pt]{article}
\usepackage{od}
\usepackage[utf8]{inputenc}

\title{Zu Herbert Hörz „Dialektik als Heuristik“ \\[6pt] \large Kritik dem
  Freund und Gleichgesinnten}

\author{Rainer Thiel, Storkow}

\date{Version vom 19.12.2005, mit Kommentaren\\ von Hans-Gert Gräbe,
  24.12.2018}

\begin{document}

\maketitle
\begin{quote}
  Original erschienen als \\
  Rainer Thiel: Wie wird Dialektik nutzbar als Heuristik?\\
  EWE 17 (2006) 2: 230--233. \\
  \url{https://groups.uni-paderborn.de/ewe}
\end{quote}
\section*{Wie wird Dialektik nutzbar als Heuristik?}

\paragraph{1.}
Der Titel „Dialektik als Heuristik“ von (Hörz 2006)\footnote{Solitäre
  Publikationen wie diese verweisen auf ein umfassendes Forschungsprogramm,
  das mit der Wende 1990 ein abruptes Ende fand.  Siehe dazu auch das im
  Max-Stirner-Verlag 2013 herausgegebene Buch „Dialektik der Natur und der
  Naturerkenntnis“
  \url{http://www.max-stirner-archiv-leipzig.de/dokumente/Hoerz-Roeseberg-Dialektik.pdf}
  -- HGG.} bezeichnet auch mein Anliegen. Hörz hat in seinem Absatz ((1))
angedeutet, was auch ich unter Dialektik verstehe. Seine Präliminarien bis
((8)) kritisiere ich nicht, doch seinen Text von ((9)) bis ((28)).

\paragraph{2.}
Hörz und Thiel eint die Überzeugung, dass die ganze Welt prozessiert und
Dialektik zu erkennen uns Erdenbürgern unerlässlich ist. Beide bekennen sich
von Jugend an zum Sozialismus. Beiden war früh schon klar: Vom mechanischen
zum dialektischen Materialismus zu gelangen fällt unsren Freunden schwer.
Dialektik war nur Logo: Gegensatz ist zwischen Arm und Reich, und nichts
bleibt, wie es ist.

\paragraph{3.}
Hörz und Thiel mögen dialektischen Materialismus. Hörz war vorsichtig.  Thiel
grübelte. Hörz kam zu Amt und Ehren. Dennoch hat er -- nach Georg Klaus als
einziger -- den Grübler zu Wort kommen lassen. Thiel und Hörz wollten ab 1973
institutionalisiert kooperieren. Das scheiterte an Hörzens Chef. 1999 förderte
Hörz eine neue Thiel-Edition, und Thiel dankt ihm für Freundlichkeit.

\paragraph{4.}
Dennoch blieb viel Unterschied. Hörz gibt dem Topos „Determinismus“ Vorrang,
auch wenn er das Thema „Dialektik“ wählt. Zudem lässt er unreflektiert, dass
noch heute geglaubt wird, Marx und Engels hätten die Zukunft vorausgesagt.
Erst 1998 wurde diese Interpretation ad absurdum geführt (Thiel 1998:
Kap. 14).  Hörz hätte diese Einsicht früher haben können als der
Außenseiter. Dialektik wäre rechtzeitig ins Blickfeld gelangt.

\paragraph{5.}
Thiel wurde einst als Funktionär des Jugendverbands wegen Kritik am Chef
Honecker vom Studium ausgeschlossen\footnote{Siehe hierzu Thiels Autobiografie
  (Thiel 2010) -- HGG.}. Bauarbeiter prägten ihn. Um 1958 traf er sich in
seiner Stimmung mit einem Mathematiker, der bald Nationalpreisträger und
Akademiemitglied wurde, damals auch Parteisekretär eines Fach-Bereichs war und
über die gängige Lehre spottete: „Da heißt es immer, 'die Dialektik lehrt,
Komma, dass \ldots', doch die Dialektik selbst, die bleibt uns vorenthalten.“
Thiel empfand, am wenigsten werde geklärt, was ein „dialektischer Widerspruch“
und was „Umschlag von Quantität in Qualität“ sei, nach Friedrich Engels
(Engels 1879) Kerne der Dialektik. Nach Lenin (Lenin 1914: S. 212) wäre zu
erwarten gewesen, dass „nicht Abschweifungen, sondern das Ding an sich selbst“
– die Dialektik also selbst (!!!) - abgehandelt werde. Das wollte ich. Doch
oft wurde ich gerügt, „immer gleich ohne Einleitung an die Sache zu gehen“.
Später, als Dozent für Patentingenieure, sah ich, dass selbst gestandne
Ingenieure beim Thema „erfinderische Ingenieur-Arbeit“ nicht zur Sache kamen.
Sachen zu umschweifen und mit Girlanden zu umwinden ist verbreitet.

\paragraph{6.}
1959 begann mein Abenteuer -- Blick in die Regelungstheorie der Kybernetik:
Unterscheidung von Übergangsfunktionen und Stabilitätszuständen, von positiver
und negativer Rückkopplung samt Zielstrebigkeit, alles mathematisch
beschreibbar.  Norbert Wiener hatte die Beschränkung dieser Begriffe auf
Technik überwunden. Und unsre Professoren? Unbemerkt von Hermann Ley und
geschützt durch Georg Klaus zeigte ich, dass Rückkopplung und Zielstrebigkeit
im Werk von Marx konstituierende Momente sind und damit der philosophischen
Dialektik zugehören (Thiel 1962). Die heuristische Rolle dieser Begriffe
beginnt damit, dass Monokausalität bezweifelt und Achselzucken, wonach eine
Wechselwirkung nicht zu hinterfragen sei, überschreitbar wird.

\paragraph{7.}
Widersprüche sind sich entwickelnde Spaltungen von „Einheitlichem“ (Lenin
1914).  Im Alltag der DDR wurden sie als „Fehler“ gesehen. Deshalb wollte sie
Georg Klaus in Analogie zu Regelabweichungen verstehen. Das kann geboten sein.
Dann wird die ereignis-orientierte Sicht durch die duale
potential-orientierte, Wahrnehmungsvielfalt pauschalisierende ergänzt.

\paragraph{8.}
Dialektik als Methodik nutzt Integration von Wahrnehmungen bei andauernder
Bewegung. Je ein Prozess wird nach kennzeichnenden Parametern ins Auge
gefasst, wie stets auch dann, wenn man von „Maß“ spricht. Nach Barometer-Art
sind Indikatoren gefragt. Diese können durch Attribute wie „zuspitzend“ oder
„stagnierend“, „abflauend“ oder „tendenziell“ angedeutet sein.  Damit sind
Wahrnehmungen sehr vieler, als Einzelne in ihrer Masse kaum fassbarer
Ereignisse integriert. Jedes Ereignis müsste selber als ein Integral gesehen
werden. Das Integral betreffend kann z.B. nach Dämpfung, Periodizität und
Verstärkung, nach Stabilitätszuständen, Übergangsfunktionen, nach
Nichtlinearität und Auflösung von Teufelskreisen gefragt werden. So wird ein
Teil der Prozess-Dialektik bedient.

\paragraph{9.}
Regelungstheorie ist in der Sprache der Differentialgleichungen geschrieben.
Diese Sprache ist ein Medium, dialektische Widersprüche in ihrer strukturellen
Spezifik zu sehen (Thiel 1967). Das kontert die Ansicht, Mathematik sei das
Ressort der Quantität. Bald konnte ich auch zeigen, dass die Wörter „Qualität“
und „Quantität“ 13 bzw. 10 verschiedene Bewandtnisse bezeichnen. Leider werden
immer noch die verschiedenen Bedeutungen durcheinander gebracht. Am
absurdesten ist es, „Grad“ als „Qualität“ zu bezeichnen. Schon Hegel
unterschied extensives und intensives Quantum.

\paragraph{10.}
Schwierig bleibt, Parameterwerte zu besorgen. Man kann aber
Entscheidungsvarianten definieren und am „Sandkasten“ mit hypothetischen
Werten erörtern. Andere Begriffe als die Regelungstheorie nutzend liefert die
mathematische Theorie der strategischen Spiele Entscheidungsmuster. Sie
provoziert erfinderische Überlegungen zur Erweiterung der Strategie-Vektoren.

\paragraph{11.}
Kybernetik hatte als Brücke zwischen Mathematik und Philosophie guten Dienst
getan. Ins Zentrum rückte für mich seit 1967 das Verhältnis von Sprache und
Denken.  1975 zeigte ich unterm Titel „Mathematik – Sprache – Dialektik“
(Thiel 1975): Die Mathematik ist ein Fundus problemspezifischer Sprachen.
Mathematiker sehen das (Thiel 1982, 2003). Einige dieser Sprachen eignen sich,
Prozesse, Wechselwirkungen, Widersprüche und Qualitätsumschläge strukturell zu
beschreiben, um praktisches Handeln besser zu bedenken und dialektisches
Denken zu üben. Eine Kuriosität wurde aufgelöst: Einerseits wird Dialektik als
Allerweltsphänomen gepriesen – so weit so gut --, aber man kennt von ihren
universellen Topoi keine Spezifikationen, so, als wenn man sagen würde, „es
gibt Menschen und Tiere, sie entwickeln sich. Punkt.“ Eine so karge Lehre wird
trotz aller Wahrhaftigkeit nicht ernst genommen. Dehnung der Präliminarien,
ohne zur Sache selbst zu kommen, wirkt abstoßend. Lehrbücher boten meist
dieselben drei Beispiele, doch so kraus, dass sie verdunkeln statt zu
leuchten.  Und so blieb es. Gänzlich unreflektiert blieb jene Art von
„Widerspruch“, die Marx -- u.a. in (Marx 1867: Begriff der Ware) -- analysiert
hatte.

\paragraph{12.}
Ein Minimum an Spezifikation der Monstren „dialektischer Widerspruch“ und
„Qualitätsumschlag“ ist aber nötig, um sich im Sinne von „Dialektik als
Heuristik“ dem Leben zu nähern. Der Sache wegen skizzierte ich eine
Nomenklatur dialektischer Widersprüche. Auch „Heuristik des Quale-Umschlagens“
ist möglich geworden.

\paragraph{13.}
Eine Chance zur Entwicklung der Dialektik entsprang der Provokation zu
„technischem Schöpfertum“. Stoff lieferte 1973 Genrich Saulowitsch
Altschuller: „Erfinden, (k)ein Problem“ (Altschuller 1973). Kern war die
Kennzeichnung dialektischer Widersprüche in der technischen Evolution.
Gemeinsam mit erfinderisch erfahrenen Ingenieuren habe ich – im Gegensatz zu
akademischen Instituten – an der Begründung sog. Erfinderschulen des
Ingenieurverbandes gewirkt. Dialektik in technisch-ökonomischer Evolution ist
einfacher als im Politischen -- ein Vorteil fürs Erststudium. Leicht ließen
sich zwei Kompromiss-Arten und „Widerspruchslösung“ unterscheiden, den Blick
für Verhaltensweisen schärfend, wo Philosophen ein Bedürfnis zur
Unterscheidung gar nicht spüren.

\paragraph{14.}
Widerspruch-Lösen erwies sich als Kern jeglicher Kreativität von Erwachsenen.
Lehrmaterialien entstanden, im Zentrum ein heuristisches Programm zum
Herausarbeiten von Entwicklungswidersprüchen und Lösungsansätzen.  Dialektik
als Heuristik.  Philosophen -- um ihre Herrschaft über das Wort „Widerspruch“
besorgt – begannen zu intrigieren. Hörz war Ausnahme. Thiel mit
Erfinder-Kompagnon ließ er in seinem Bereich vortragen, sich selbst und die
Gäste der Arroganz junger Philosophen aussetzend. Ehe es zum Dialog kam, war
die DDR am Ende. Die Sache selbst wurde am Rhein (!) würdig dokumentiert und
von Bayern aus mit großem Erfolg weitergeführt als „Widerspruchsorientierte
Innovationsstrategie“.

\paragraph{15.}
Training von Dialektik und von Kreativität ist auch außerhalb des technischen
Bereichs denkbar (Thiel 2005). So wird Antwort auf Hörz, Philosophie sei zur
Sinnstiftung im Menschenleben berufen. Philosophen äußern sich in Wort und
Schrift dazu.  Doch wenig kommt im Alltag an, und weiter muss Mephisto klagen:
„Die Menschen dauern mich in ihren Jammertagen, Ich mag sogar die Armen selbst
nicht plagen.“ Da wird es Zeit, Dialektik und aufrechten Gang zu üben, um die
freiheitlich-demokratische Grundordnung zu bewahren.

\paragraph{16.}
Das heuristische Programm (Rindfleisch, Thiel 1988) zu kreativem Verhalten in
der technisch-ökonomischen Evolution geleitet zur Analyse echter Bedürfnisse,
zu gedanklicher Parametervariation an tradierten Objekten, zur Antizipation
von Widersprüchen sowie zur Analyse ihrer Geflechte auf
technisch-ökonomischer, technisch-technologischer und eventuell
technisch-naturgesetzlicher Denk-Ebene. Das Programm ist kein abzuspulender
Algorithmus, sondern ein Szenarium, das zum Denken und Prüfen anspornt, den
Denkweg in begehbare Abschnitte zerlegt – Schleifen nicht zu vergessen! -- und
Entscheidungspunkte markiert. Training heißt Übung, sich in solchen Szenarien
zurechtzufinden und Haltungen zu verinnerlichen, die auf Problemlösung zielen.
(Wenn alles gut gegangen ist in unsrer Vita, haben wir ja auch ein wenig Logik
verinnerlicht.) Zum Programm gehören heuristisch nutzbare Lösungs-Strategien.
Einige verdienen das Signum „dialektisch“ (Rindfleisch, Thiel 1988: Kap. 1.).

\paragraph{17.}
Kreative Haltungen auch für staatsbürgerliches Dasein auszuprägen müsste
Anliegen jeglicher Bildung werden, dann könnte die übliche Verehrung von
Gessler-Hüten umschlagen zu Aufrechtem Gang, Obrigkeiten zwingend, die
Bürgerwürde zu achten.

\paragraph{18.}
Leider ist Herbert Hörz auf Determinismus ausgewichen, wo Dialektik angesagt
war. Zudem ist „Determinismus“ nur Hülle ohne Inhalt. Dialektik im
Determinandum ist aber nicht wegen der von Hörz apostrophierten Zufälligkeit,
sondern wegen ihrer Universalität. Zu Hörz ((9)), Widersprüche seien die
Quelle der Entwicklung, sei angemerkt: Bewegung IST, und „Widerspruch“ ist
deren Form.

\paragraph{19.}
Determinismus ist Anerkennung der Tatsache, dass es Myriaden
„Wenn-so-Relationen“ gibt, von denen jeweils zehntausend im Spiele sein
können, in Wirklichkeit weit mehr, weil Dialektik auch innerhalb jeder
Komponente einer Wenn-so-Relation wirkt, wie Hegel wusste und wie sich heute –
Zeit der fraktalen Geometrie – leichter denken lässt. Geahnt war es schon
immer: „Der Teufel steckt im Detail -- die Hölle selbst hat ihre Rechte.“ Wer
als Akademiker auch praktisch arbeitet, sieht es auf Schritt und Tritt.

\paragraph{20.}
Schließlich ist „dialektische Wechselwirkung“ um die Begriffe „Iteration“ und
„Attraktor“ zu erweitern, die uns die Mathematik anbietet.  Was auch immer
„Notwendigkeit“, „Gesetz“ und „Zufall“ im „Determinismus“ sein mag: Dialektik
ist Substanz. „Determinismus“ als Hohlkörper ist überflüssig.

\paragraph{21.}
Natürlich verlangt Dialektik, Wenn-so-Relationen geltend zu machen, auch mit
moralischen Gründen, um sie mit Willen zu verwirklichen. Gesetze zu entdecken
ist geschichtliche Tat (Marx 1867). Doch Ballast ist ein Ismus, der sich auf
Anerkenntnis von „Wenn-so“ beschränkt. Ob sehend oder blind, ob duldend oder
kämpfend -- Geschichte wird erzeugt. Menschen sind Erzeuger.  Geschichte
vorwärts ist so wenig bestimmt wie ein Fußballspiel, wo auch Zufälligkeit im
Sinne von Hörz nicht passt.

\paragraph{22.}
Zweiter Kern der Dialektik ist das sog. Umschlagen quantitativer Veränderungen
in qualitative. Je ein Objekt wird parameter-orientiert gesehen.  1960 begann
mir aufzufallen, dass Marx und Engels darüber anders dachten als Marxisten.
Auch fiel mir auf, dass in der Regelungstheorie zwischen linearen und
nichtlinearen Systemen unterschieden wird. „Nichtlinear“ heißt: In den
Gleichungen treten Glieder mit unterschiedlichen Exponenten auf; bei größeren
Regelabweichungen erfolgt keine Rückkehr zum Sollwert. Nichtlinearität erweist
sich nicht nur durch Inputs. Sie entsteht auch durch Mehrdimensionalität
(z.B. Länge $\to$ Fläche $\to$ Volumen). Als ich das angehen wollte, wurde ich
der Universität verwiesen. Auch wollte ich mit Mathematikern kooperieren.
Später wurden in dieser Arbeitsrichtung Nobelpreise verliehen. Die
Chaos-Ordnungs-Theorie hängt damit zusammen. Gute Mathematiker gab es auch in
der DDR.

\paragraph{23.}
Zusammenhang von Quantität/Qualität und Nichtlinearität, auch das jeweils
Ganze, das mehr ist als die Summe der Teile, beschäftigten mich ab 1966 in
dienstfreien Stunden. Hegel, Marx und Engels hatten längst darüber nachgedacht
-- mathematische Begriffe nutzend, heuristisch und für ihre Werke konstitutiv,
nachlesbar, problemspezifische Sprachmittel nutzend von mir auch erläutert,
nur von den Philosophen nicht bemerkt. Bis heute nicht. Ein Professor für
„marxistisch-leninistische Philosophie“ wollte mir an die Gurgel, als ich ihm
literarische Zeugnisse von Marx und Engels nachwies.  Mathematische Modelle
lassen erkennen, wie die Täuschung von der Plötzlichkeit sog. Qualitätssprünge
zustande kommen konnte. Auch eine tragische Irritation Lenins gab es. Lenin
bekannte ehrlich, ohne „höhere Mathematik“ könne man Hegels Lehre zum
„Qualitätssprung“ nicht verstehen (Lenin 1914: S. 110).  Wie aber kann Herbert
Hörz von Dialektik sprechen und „Nichtlinearität“ nur flüchtig erwähnen?

\paragraph{24.}
Ergebnisse jahrzehntelanger, oft unterbrochner Arbeit sind zusammengefasst in
(Thiel 2000), in der Reihe „Selbstorganisation in der Gesellschaft“
herausgegeben von Herbert Hörz.  Ergebnisse sind: Quale-Wandel erfolgt
permanent.  Wer damit nicht rechnet, den bestraft das Leben. Das kann schon im
Gymnasium gelehrt werden. In seiner Permanenz erweist sich quantitativer
Wandel als das, was er von Anfang an ist, als nichtlinear und qualitativ. Um
das zu bemerken, waren Hegel, Marx und Engels zu lesen und von nichtlinearen
Funktionen zu wissen. Naive Beobachter aber abstrahieren vom permanenten
Quale-Wandel und ignorieren die Nichtlinearität bis zum Exzess, dem Bruch.
Quale-Wandel ist keine Frage der Zeit, also auch nicht von Plötzlichkeit. Wenn
schon Tempora zur Rede stehen: Auch Quale-Wandel ist in der Regel allmählich.
Hegels „Sprung“ bezeichnet eine Metapher und hat nichts mit Plötzlichkeit zu
tun, wohl aber mit Permanenz. Also „Allmählichkeit der Revolution“,
Kreativität und Energie statt Warten im Parteilokal. Die Linken könnten von
Hegel und Marx profitieren.

\paragraph{25.}
Die heuristische Rolle mathematischer Begriffe in Hegels und Marxens Dialektik
liegt zutage. Mathematische Medien helfen, die Schätze jedem Schüler zu
zeigen.  Hegels nicht ganz geglücktes Bild „Knotenlinie von Maßverhältnissen“
wurde – neuere mathematische Sprachmittel nutzend -- rektifiziert. Von Hegel
demonstrierte Spaltung von „Einheitlichem“ in divergierende, ja gegensätzliche
Komponenten wird bewirkt durch Nichtlinearität.  „Dialektik“ ersetzt
„Determinismus“. Das kann beitragen, Menschen zu \emph{aufrechtem} Gang zu
ermutigen. Herbert Hörz hat per Titel „Dialektik als Heuristik“ an ein
wichtiges Thema gerührt.

\section*{Literatur}
\begin{itemize}
\item G.S. Altschuller (1973). Erfinden (k)ein Problem?  Verlag des
  Gewerkschafts-Bundes der DDR.  Übersetzung aus dem Russischen von Kurt
  Willimczik. Original: \emph{Algoritm isobretenija} (Moskau 1969).
\item F. Engels (1879). Manuskript zur „Dialektik“. In MEW 20, Berlin
  1971.
\item H. Hörz (2006).  Dialektik als Heuristik. EWE 17.2:167--176.
\item W.I. Lenin (1914). Konspekt zur 'Wissenschaft der Logik', Die Lehre vom
  Begriff. Werke Band 38, Berlin 1964.
\item K. Marx (1867). Das Kapital, Band I, Erster Abschnitt.
\item H.-J. Rindfleisch, R. Thiel (1988). Erfindungsmethodische Grundlagen –
  Die Methode des Herausarbeitens von Erfindungsaufgaben und Lösungsansätzen.
  Kammer der Technik, Berlin, als Manuskript gedruckt.
\item R. Thiel, G. Klaus (1961). Über die Existenz kybernetischer Systeme in
  der Gesellschaft. Deutsche Zeitschrift für Philosophie 1/1962, Berlin. (Aus
  der Diss. Thiel)
\item R. Thiel (1967). Quantität oder Begriff – Der heuristische Gebrauch
  mathematischer Begriffe. Deutscher Verlag der Wissenschaften, Berlin.
\item R. Thiel (1975). Mathematik – Sprache – Dialektik.  Akademie-Verlag,
  Berlin.
\item R. Thiel, M. Peschel (1982). Warum Mathematik? In \emph{Wissenschaft und
  Fortschritt} 32 (1982). Reprint 2003 in „Systemtheorie -- Gedenkband zum
  Leben und Schaffen von Prof. Manfred Peschel“, Hochschule Zittau/Görlitz,
  ISBN 3-9808089-3-9.
\item R. Thiel (1998). Marx und Moritz – Unbekannter Marx – Quer zum Ismus.
  Trafo Verlag Berlin, ISBN 3-89626-153-3.
\item R. Thiel (2000). Die Allmählichkeit der Revolution – Blick in sieben
  Wissenschaften.  LIT Verlag Münster, London, Hamburg, ISBN 3-8258-4945-7.
\item R. Thiel (2004). Georg Klaus, die Dialektik, die Mathematik und das
  lösbare Problem disziplinärer Philosophie. In \emph{Kybernetik und
    Interdisziplinarität in den Wissenschaften – Georg Klaus zum
    90. Geburtstag. Gemeinsames Kolloquium der Leibniz-Sozietät und der
    Deutschen Gesellschaft für Kybernetik im November 2002 in Berlin}.  ISBN
  3-89626-435-4.
\item R. Thiel (2005). Das vergessene Volk – Mein Praktikum in Landespolitik.
  Regionen Verlag Görlitz, ISBN 3-9809400-3-9.
\item R. Thiel (2010). Neugier, Liebe, Revolution.  Verlag am Park Berlin,
  ISBN 978-3-89793-248-7.
\end{itemize}
\ccnotice
\end{document}
