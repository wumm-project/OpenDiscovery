\documentclass[11pt,a4paper]{article}
\usepackage[utf8]{inputenc}
\usepackage{od}
\usepackage[english]{babel}

\parindent0pt
\parskip4pt
\setlist{noitemsep}

\title{ProHEAL Basics -- Extended Version}

\author{Hans-Gert Gr\"abe, Rainer Thiel}
\date{Version of 2021-08-22} 

\begin{document}
\maketitle

\section{ProHEAL -- the Background}

In order to understand the theoretical approaches of the \emph{Programme for
  Working out Invention Tasks and Solution Approaches} (ProHEAL)\footnote{In
  German: Programm zum Herausarbeiten von Erfindungsaufgaben und
  Lösungsansätzen. } as a GDR-specific TRIZ version, the specific economic
conditions of the GDR in the 1980s must be taken into account. After an
upswing of innovation-theoretical as well as innovation-practical approaches,
especially in the context of the boom in cybernetics and measurement and
control technology in the 1960s and early 1970s, innovation-technical aspects
moved to the background after 1974 in favour of a "unity of economic and
social policy".

This undermining of industrial innovative strength showed its effects in the
1980s with a marked decline in the international competitiveness of GDR
products, especially in the high-tech sector, and resulted in massive
import-export imbalances. Such initially economic contradictions could only be
solved through more far-reaching technological changes, whereby the focus was
not so much on the classic TRIZ issue of patent circumvention, but rather on
import replacements, which became necessary both for reasons of valuta
balances and as a result of increased embargoes.

Such replacement processes do not only intervene deeply in technological
processes, but also require precise knowledge about the existence and concrete
access conditions to required resources, which could hardly be maintained in a
centralised and centrally managed planning process. This need for agile
management significantly upgraded the position of power of local "captains of
industry" (the directors of the combinates) compared to the previously
dominant party-controlled central planning bureaucracy and ultimately led to a
shift in the power balance in GDR society, as explained in more detail in
\cite{Barkleit}.

In the combinates, it was mainly the R\&D directors with an engineering
background, who supported and promoted the use of such approaches.  Many of
them became already familiar with systematic innovation methodologies during
their own graduation \cite{RM-23}. The need for \emph{practical} training in
innovation methodology led to the boom of the GDR inventor schools in the
1980s as places for in-house and inter-company innovation-methodological
training.  During these trainings special problems from within the companies
formed the backbone of the trainings, see \cite[part 1, ch. 3]{RT94}, and the
power of Altshuller's TRIZ methodology had to prove themselves again and again
in such practical contexts.

There were only loose connections between the inventor schools of the
individual combinates, and the interest of most participants was limited to
the experienced methodological support in solving their own intra-company
problems. An overarching connection existed at the level of the lead trainers,
among whom Michael Herrlich, Hans-Jochen Rindfleisch, Hansjürgen Linde, Rainer
Thiel and Dietmar Zobel (in alphabetical order) are particularly worthy of
mention. This group has also been instrumental in generalising, systematising
and publishing the experience gained. Two dissertations \cite{Linde1988,
  Herrlich1988} and various training and other materials \cite{HMT1985,
  Herrlich1982, Herrlich1986, Hill1987, RT1986, Speicher1980, Thiel1977,
  Thiel1980, Thiel1986} were produced in this context.

After the fall of the Berlin Wall, only Hansjürgen Linde and Dietmar Zobel
were able to continue working on invention methods in practice. Linde first
worked at BMW in Munich and later as a professor at the Coburg University of
Applied Sciences. The \emph{Contradiction-Oriented Innovation Strategy}
(WOIS)\footnote{In German: Widerspruchs-orientierte Innovations-Strategien.}
developed in his dissertation \cite{Linde1988} is an alternative
generalisation of the experiences of the GDR inventor schools, which was
published in book form in \cite{LindeHill1993} and is now being further
developed by the WOIS Institute in Coburg. Since the methodology and
algorithmic approaches are protected by trademark and copyright, we will not
go into further detail here, especially since WOIS and ProHEAL come to similar
conclusions. We therefore limit ourselves to a more detailed description of
ProHEAL. Within the framework of the WUMM project, various materials from that
heritage have been compiled in digital form and are publicly
available\footnote{See \url{GDR-InventorSchools}.} under an Open Source
Licence.

Dietmar Zobel plays a special role in this context as a chemical technologist.
TRIZ as a whole and also the ProHEAL approaches mainly generalise experiences
from mechanical engineering with its emphasis on \emph{artefactual},
\emph{structural} and \emph{functional} moments. In contrast, in chemical
technology (and a number of other technology areas little studied by TRIZ),
\emph{processual} moments with flows, flow characteristics and dynamic flow
equilibria play a more important role, for which only a few elaborated TRIZ
tools are available. Zobel -- author of more than 30 patents really applied in
industry -- worked during GDR times as production director at the Piesteritz
nitrogen company. After 1990, he founded an engineering office for systems
technology and worked as an independent expert, consultant, TRIZ trainer and
lecturer for systematic invention. His experience is available in a series of
publications \cite{Zobel1985, Zobel1991, Zobel2001, Zobel2006, Zobel2007,
  Zobel2009}, with many subsequent editions, which, however, have hardly found
any attention beyond a German-speaking community. This, too, cannot be
discussed in detail here, see \cite{Graebe2019a, Graebe2019b, Thiel2016}.

\section{ProHEAL Basics -- a Short Overview}

This paper elaborates on a number of ProHEAL aspects that could only be
touched upon cursorily in \cite{ProHEAL-21}. In this section, a brief summary
of the interrelationships presented in \cite{ProHEAL-21} is given.

\subsection{The ProHEAL Path Model}

ProHEAL, like TRIZ in general, is based on a \emph{three-level model} of
resolution of conflicting requirement situations, which provides for an
increased focusing and deepening of the analysis, see the diagram
\cite[Appendix 1]{ProHEAL-21}. The procedure is schematically described in a
\emph{decision tree} \cite[Appendix 3]{ProHEAL-21}, which leads in each case
to a \emph{Draft Specification}, which in the context of further processing is
either to be developed into a solution suitable for production or into a
"thought product" as the result of a more comprehensive thinking activity (in
the sense of Shchedrovitsky), which further qualifies the planning process.

On the \emph{first level}, the technical-economic problem situation stands as
a technical-economic operational field between societal needs as potential
requirements and the technical state of the art as the field of possible ways
to implement these requirements. The result is a \emph{basic variant} that
roughly outlines such a possible realisation. If a problem solution is
\emph{feasible} at the state of the art it only has to be detailed and
realised.

If this fails due to technical-economic (external) contradictions, this basic
variant must be analysed in more detail on the \emph{second level}. The
critical functional area within the basic variant is identified as the
\emph{core variant} and there the functional IDEAL is contrasted with the
harmful technical effects. If a problem solution is \emph{conceivable}, it
must be worked out in more detail and then returned to the first level with
the correspondingly transformed basic variant.

If this fails due to technical-technological (internal) contradictions, the
process-critical operational principle in the core variant must be identified
on the \emph{third level} and there the operational IDEAL of the natural law
must be contrasted with its harmful effects. If a problem solution is
\emph{imaginable}, it must be worked out in more detail and then returned to
the second level with a correspondingly transformed core variant.

If this fails due to profound technical-scientific (internal) contradictions,
the entire principle of the basic variant is called into question and a
fundamental, disruptive innovation is required, which cannot be realised
within a delimitable innovation project, but requires fundamental research,
which is not further addressed by ProHEAL. If such new research results are
available, this \emph{new operational principle} can be used to return to the
third level.

\subsection{The ABER Matrices}

Details see in \cite{ProHEAL-21}.

\subsection{The Refinement of the Problem Field Levels}

In contrast to other TRIZ versions, in its first phase -- the modelling of the
technical-economic requirements -- ProHEAL discusses also in more detail
methodical aspects for the 80\% of solutions that can be obtained on the
current state of the art solely by tayloring or optimising known approaches.
The ProHEAL methodology can thus be applied more broadly, because at the
beginning and in an early phase of the analysis of the technical-economic
requirements, it is not yet clear whether we will encounter obstacles in the
solution process that require strong inventive thinking.

On the other hand, even in solutions that are ultimately obtained through
adaptation or optimisation, certain inventive moments arise, especially
concerning processual aspects, whose systematic treatment is advantageous.

Finally, the solution of a "hard" inventive problem does not start from
scratch, but draws on the experience of the (finally unsuccessful) modelling
of the concrete problem that precedes the insight that it is a "hard"
inventive problem.

It is therefore only logical to detail this first phase in more
systematic-methodological parts as well. In \cite[Part A]{RT89}, see also
\cite[Part 3]{RM-21}, ten complexes of a advancing analysis are distinguished
for this purpose:
\begin{itemize}[leftmargin=40pt,align=left]
\item[A.1.] The social need. Preliminary system designation.
\item[A.2.] State of the art. Preliminary selection and system analysis of a
  starting variant. The variation of the system parameters according to the
  needs.
\item[A.3.] The operational field of the inventor.
\item[A.4.] The technical-economic contradiction.
\item[A.5.] The harmful technical effect.
\item[A.6.] The IDEAL. Starting point and orientation for a deeper system
  analysis.
\item[A.7.] The technical-technological contradiction.
\item[A.8.] The technical-scientific contradiction.
\item[A.9.] The strategy for the solution of a contradiction.
\item[A.10.] Own invention as a pacemaker in the international
  development of the state of the art.
\end{itemize}
The first four complexes belong to this first phase.  The role of a reference
and basic variant of a possible solution in this part is outlined below.

\section{ProHEAL -- Some Explanatory Notes}

\subsection{ProHEAL -- a Tool for the Engineer and Inventor}

The ABER matrix provides the open-minded observer with many suggestions.
Varying parameters in one or several matrix fields he probably searched for
and perhaps found inventive solutions. The open-minded observer based his
considerations on the so-called \emph{basic variant}. But soon the open-minded
observer can also encounter obstacles and reach limits that prevent him from
advancing. Then the situation gets difficult.

The attempts to improve (increase) common parameter values of technical
objects lead mostly to a situation in which the engineer has to pause and ask:
"Yes, but what shouldn't happen?" That is a question of undesirable
consequences. One objective of inventions is to discover the most accurate
knowledge of all possible "Yes, but ...".  The goal is to find for every "Yes,
but ..."  not an "Either ... or", but an "As well ... as". The observing
engineer feels now caught in a vicious circle. The "Yes, but ..."  signals
that there is a dialectical contradiction: Two tendencies are opposed to each
other -- battle of opposites -- each tendency inevitably produces the other,
often several others as well. The ABER matrix helps to perceive and understand
what is happening. This is the starting point to determine the entries in the
ABER matrix. What the observer refers to from the very beginning are the
societal needs, the manufacturer needs, the user needs. A concrete objective
must be derived from this. If no solution has been found for the contradiction
after the first attempts, further questions about the basic variant must be
asked and answered. ProHEAL shows the way to a solution.

The \emph{mandatory evaluation figure}\footnote{German: "Zielgröße".} (either
completely specified or completed by the responsible engineer) is
characterised in \cite[(1.3)]{RM-21} (of ProHEAL) as expression of the
identified system of ABER entries. The technical-economic parameters are to be
derived from it. That some or many of them should be improving a lot (but none
should deteriorate) is initially just a request or wish. They are rooted in
the network of technical-technological or technical-scientific properties of
the basic variant and are interlinked by these properties. From these
connections in the system of the ABER entries inevitably result the "Yes, but
...", which are to be inventively transformed -- through changes in the basic
variant -- in "as well as". This inevitability is the core difficulty of a
purposeful, effective invention.

The "Yes, but ..." become all the more delicate and acute, 
\begin{itemize}
\item the more the effectiveness $E$ is to be increased, because then limits
  of improvements based only on optimisation of the technical object are
  reached or have to be crossed;
\item the more consistently the evaluation figure -- the ABER system -- is
  respected in its complexity so that improvements regarding some parameters
  no longer can be achieved by "cheap" approaches at the expense of
  deterioration in other parameters that can improperly be ignored.
\end{itemize}

In the technical-economic development, such contradictions can be ignored for
a while. This is possible,
\begin{itemize}
\item if they are not serious in terms of their importance, because the
  contradictory parameters are not so important in the overall system; in
  other words, if the complex target (the system of ABER entries) allows for a
  distinction between fundamental and less important parameters;
\item if the contradiction is in the initial stage of development, where
  improvements of the parameters affect each other in a not yet excluding way.
  In this initial stage, a compromise solution can be found by optimisation
  based on the state of the art which leads to a sufficient increase of value
  for each of the two parameters.
\end{itemize}

The invention method must now evoke awareness,
\begin{itemize}
\item how the technical-economic contradiction to be resolved is determined by
  analyzing the technical object -- more precisely: the basic variant.  Which
  relationships in the technical object are "responsible" for the conflict
  situation in the system of ABER entries?  These primary, often complex
  relationships are usually not easily to be disentangled. More about the
  analysis up to the determination of the technical-economic contradiction see
  \cite[A.1--A.4]{RM-21}.
\item how to find the point or secondary context hidden deeper in the
  technical object, in its structure where parameters can be changed in such a
  way that the primary context, on which the technical-economic contradiction
  is based, can be rendered harmless. This leads to the question about the
  technical-technological and possibly technical-scientific contradiction that
  must be uncovered when the technical-economic contradiction should be
  dissolved.  See \cite[A.5--A.9]{RM-21}.
\end{itemize}

\subsection{The Basic Structure of ProHEAL}
  
At the beginning the \emph{guiding variable} has to be defined. It is a
system-specific parameter of central importance, the variation of which
influences the development of the performance and/or the effectiveness of a
technical system. In different words: Such influence is -- mostly based on
many years of experience -- expected as a result of the variation of the
guiding variable. The guiding variable can be, for example: the unit of
performance of a large transformer or a large generator, the number of
integrated circuits of a microprocessor, the load level of a transport system,
the starting torque of an electrical motor (based on its nominal torque), the
clock speed of a machine tool, the equilibrium concentration of a chemical
process, the specific number of separation stages of an extraction process.
More on the notion of huiding variable in \cite[(2.3.4)]{RM-21}.

A \emph{technical-economic contradiction} (TEC) arises if in the variation of
a technical parameter that is decisive for the achievement of this higher
economic effect -- the guiding variable (GV) -- at least two important
technical-economic parameters $E_1$ and $E_2$ of the technical object behave
mutually opposite.

For example, consider the development of a container. As GV we choose the
thickness of the wall of the container in relation to its edge length.
Reducing the thickness of the wall two essential technical-economic action
parameters of the container are favourably influenced: the specific use of
material, related to the container volume, and the payload ratio, i.e. the
ratio of loadable mass to the self mass of the container.  We can therefore
combine both parameters to the technical-economic action parameter $E_1$. The
opposing technical-economic action parameter $E_2$ is the specific load
capacity of the container, i.e. the loadable load in relation to the load
level of the transport system in question. The latter becomes to low when the
container wall thickness falls below a certain limit thus outweighing by far
the economic advantages of material savings and the low weight of the
container. In addition, slimming the wall thickness makes the container more
susceptible to corrosion and mechanical damage. A further reduction in the
specific wall thickness under a characteristic statistical limit value thus
gives rise to a critical technical-economic contradiction.

A \emph{technical-technological contradiction} (TTC) is given when during
variation of the technical parameter that primarily determines the expression
of the functional technical effect -- the \emph{structural value} (SV) -- at
least two decisive technical-technological action parameters $T_1$ and $T_2$
of the technical object become opposite to each other in their behavior.

In the example of the container with regard to overcoming the
technical-economic contradiction we initially consider the container function
as SV.  A functional technical effect is decisively influenced by it, namely
the distribution of the load forces and the type of their effect (in the form
of compression, tension, shear and/or bending stresses) in relation to the
local strength distribution in the wall of the container. One of the essential
effectiveness parameters of the container is based on this technical effect,
its specific, i.e. related to its volume load capacity $T_1$. By suitably
shaping the container wall the technical effect can be brought into effect in
such a way that material savings and higher payload ratio no longer are in a
technical-economic contradiction with the specific load capacity of the
container.

If this does not succeed, we are faced with a TTC.  It arises due to the fact
that with the container shape as SV not only the load-bearing capacity is
significantly determined, but also the specific usable volume, i.e.  the
portion of the volume of the cargo container that can be filled and its
accessibility, i.e. the way of loading and unloading the container. We can
combine these properties to a technical-technological effectiveness parameter
$T_2$. The technical-technological contradiction can consist in the fact that
a container shape is required to achieve a higher load-bearing capacity, that
leads to a lower specific usable volume and/or less favorable way of loading
or unloading the container. This is the case, for example, when the container
bottom should be given a curved shape to increase the load-bearing capacity
and thereby the usable container height (when transporting bulky goods) or its
unloadability (when transporting loose material) is inadmissibly impaired.

A \emph{technical-scientific contradiction} (TSC) exists if, in the variation
of a technical-scientific parameter that is decisive for the occurrence of a
scientific effect, i.e. the \emph{impact value} (IV), at least two
technical-scientific effectiveness parameters $S_1$ and $S_2$ of the technical
object take values in opposite directions.

Concerning the solution of the TTC in the case of the container we can, at
least locally, consider the elasticity of the material of the container wall
as IV. The impact of the natural law bound to this quantity is the elastic
deformation of the container wall. This impact causes two essential
technical-scientific effectiveness parameters to go in opposite directions:
\begin{itemize}
\item on the one hand, the adaptability of the shape of the container to the
  shape of the cargo and the adaptability of its strength distribution to
  the distribution of the specific load ($S_1$), but also
\item on the other hand, the stability of the form of the container ($S_2$).
\end{itemize}

Both effectiveness parameters will not contradict each other if the elasticity
is appropriately distributed in the container wall or if the shape durability
does not play a decisive role or is even undesirable. The latter is for
example the case with the waste container. That's why the garbage bag is made
of extremely thin, highly elastic and biodegradable plastic film.  Choosing
the IV "elasticity" not only the TSC is solved, but also the TTC.  At the same
time the susceptibility to corrosion due to the extremely thin container wall
is here even desirable because it results in rapid biodegradation of the
container.

A TSC can be given, for example, if the negative influence of the selected
impact value "elasticity" on the stability of the shape is evoked by
vibrations of the container wall under dynamic loading, which lead to
resonance phenomena and as a result to an impairment of the load and/or to
premature destruction of the container wall.

\subsection{Surprisingly Simple Solutions (SSS)}

The SSS are solutions with a particularly favorable ratio of effort and
benefit.  In \cite[(6.4),(9.3)]{RM-21} we pay particular attention to
solutions whose verbal description contains word like "by itself",
"self-movement", "self-fixation" etc.  Already \cite[(2.14)]{RM-21} contains a
question aimed at such solutions: "Which secondary functions in the system are
suitable, to make other side effects usable or to avoid harmful side effects
or to transform them into useful ones?" Very often such a suitability is
given. Then a solution or partial solution of the type "by itself" can already
be formulated during the very beginng of the system analysis, in this case as
self-compensation.  Experience show that such simple and ideal solutions are
ofteb out of the field of imagination.  Therefore inventors rarely search for
them.

Such solutions are characterised by the fact that their material realisation
can be reached predominantly with already existing functional units and energy
potentials, with little equipment effort and/or little operating energy. In
that sense they are simple, elegant, ideal. The technical world is full of
such solutions from ancient times on, which unfortunately we are carelessly
passing because we are using them from very childhood on. A typical example is
the ship's anchor, an extremely simple device with pointed shovels that "by
itself" all the more dig deeper into the ocean floor the stronger the wind or
the waves attack the ship.  The fishhook behaves analogously in the fish's
mouth.

A similar example is provided by Duncker's pendulum, which addresses the
problem of accuracy of a pendulum clock under temperature changes. But even in
physics classes in schools such SSS that can be found en masse throughout the
centuries of history of technology are totally ignored. As a curious child I
(R.T.) was surprised that the simple toilet cistern regulates the water
supply automatically.  So I asked my father, and because he was a craftsman,
he explained it to me. Even G.S. Altshuller does not pay enough attention in
his books to such surprisingly simple solutions. If you have never seen the
opportunities, it becomes difficult.

\section{The General Heuristic of the ProHEAL Path Model}

The following section is essentially a translation of \cite[ch. 5]{RM-21}.

\subsection*{W.1. The Structure of the Path Model}

The path model, see \cite[Appendix 1]{ProHEAL-21}, is displayed as a heuristic
scheme. It shows how from the technical-economic requirements based on the
necessary effectiveness and usage properties of technical objects, their
structural and functional properties are derived and finally, they are
abstracted to operational effects of technical-technological principles and
how to advance -- progressing from abstraction level to abstraction level --
on the search for solution ideas further and further away from the own special
domain to distant analogy areas.  Already here inventions with a high economic
benefit can arise.

\subsection*{W.2. The Social Need and the ABER}

From the more or less vaguely formulated technical-economic problem situation
as starting point -- according to the heuristic path model -- the social needs
and the associated ABER are to be determined. It is indispensable to derive
the causes for the emergence of the social need and the ABER contrasting it
with the current available state of technology and its past development.  In
this course it is always necessary also to check whether the given task is
oriented to overcome the causes or only to eliminate undesirable economic,
social, technical or ecological effects.  During such an analysis the main
technical-technological problem to be solved can be delimited and a
\emph{reference variant} of the technical system can be determined that most
closely matches the ABER.

Now in order to identify and weight the defects and shortcomings of the
reference variant, to find the causes and to create an independent, "tailored"
definition and solution of the specific problem, first the \emph{evaluation
  figure} is determined within a conceptual process of product planning. Along
that target, from the ideal state of the art a representative \emph{basic
  variant} of the technical system is created, identifying through patent
research and analysis of world-class solutions suitable technical means and
combining them to form the overall system. As part of a system analysis the
basic variant is compared to the reference variant to find weak points and
defects that lead to TEC in their behavior which are inventively to be fixed.
From the basic variant a solution is to be developed that has clear
technological and economic advantages compared to the reference variant.

\subsection*{W.3. The Evaluation Figure and the State of the Art}

The ABER are initially available in a verbal-descriptive form and express
social, economic and technological issues that affect a certain social
situation of needs and interests (see \cite[A.1]{RM-21}). From this an
evaluation figure is to be derived, which essentially expresses the usage
properties of the technical system to be created and how its production and
application does meet the social need in the best way. This is done assigning
to the components of the evaluation figure the relevant ABER entries. This
defines and evaluates concrete characteristics of suitability and
effectiveness. On the one hand, these include the respective social, economic
and/or technological specifics of the social need and on the other hand the
objective specificity of the technical object or object area (see
\cite[(1.3)]{RM-21}).

These suitability and effectiveness characteristics must first be described
qualitatively, before parameter values can be specified. A premature and
uncritical commitment to functional or economic parameters that are familiar
or mentioned in the problem description or even limiting oneself to them must
be avoided.

In order to be able to define these parameters correctly, it is necessary to
derive from the state of technology the most suitable
\emph{technical-technological principle} (TTP) for the technical object to be
developed. That is a characteristic principle of manufacturing and/or applying
technical objects in a specific technology domain. With this principle, a
class of methods and means is delimited in the state of the art which
represents the context for further processing of the problem. Thus the choice
of the TTP has decisive importance for the further solution. It should be done
in such a way that a TTP is preferred that fits the purpose of the technical
object to be created (target component $Z_1$) in the best way and that does
not conflict with the ABER or -- in comparison to other principles -- violates
as less as possible A and E from the ABER system. For this, it should be
considered first all procedures and means (see \cite[Appendix 1]{ProHEAL-21})
\begin{itemize}
\item that are \emph{available} on the material state of technology,
\item that \emph{appear feasible} based on the ideal state-of-the-art, and
  finally
\item that \emph{seem conceivable} on the given state of technological and
  that \emph{seem imaginable} on the given state of scientific development.
\end{itemize}
If a TTP is prescribed with the problem setting, it has to be checked if it is
feasible for the evaluation figure and should be compared with other known
principles. If necessary, this must be discussed with the client.

On the basis of the TTP, basic variants of the technical system are designed
using the \emph{methods and means available from the state of the art}.  This
is done transforming the evaluation figure determined by the ABER in two
stages:

In the \emph{first transformation stage}, the types of technical objects are
determined, which are \emph{necessary} according to the TTP to ensure the
suitability of the technical system with respect to the evaluation figure.  To
every object type now such utility properties are attributed, which on the one
hand are typical for the respective object type and on the other hand
correspond to special suitability characteristics of the evaluation figure. In
doing so, it is appropriate first to determine the necessary contributions
specific for that object type to the expediency of the system (component $Z_1$
of the evaluation figure). Then those functional properties are defined that
are characteristic for the respective object type and that guarantee the
suitability of the technical system with regard to its controllability and its
usability. For a sufficient suitability of the technical system -- especially
with regard to its controllability -- it can be necessary to take into account
additional object types, that match specific suitability characteristics with
their functional or operational properties.

In this way, the evaluation figure is transformed from a system of socially
determined \emph{suitability characteristics} into a system of object-related
\emph{usage properties}. This evaluation figure is the basis for a systematic
patent research and world status analysis for the pre-selection of suitable
technical objects, which in their combination according to the TTP are
sufficiently suitable to form a technical system that meets the ABER.

In a \emph{second transformation stage}, the \emph{main function} of the
technical system according to the TTP is defined. It can be assumed that the
main function activates the usage properties of the individual objects and
links them in the process of their use in such a way that the characteristics
of the technical system responsible for its suitability according to the ABER
are produced.  This main function has to be broken down, related to the usage
process, in its \emph{necessary and sufficient subfunctions}. Here, a
hierarchy level of the technical system is selected that on the one hand is as
high as possible, on the other hand takes into account the formation of object
types from the evaluation figure that was already completed.

To the individual sub-functions such objects are assigned, that are activated
by the respective sub-function in the sense of the main function of the
technical system. For each sub-function, those functional properties are
defined that are caused by these technical objects, which means that they
obtain the \emph{character of specific technical means}. The sub-functions
through which object-related usage properties are activated to support the
suitability of the technical system concerning manageability and usability,
are established in the same way, however as \emph{necessary support
  functions}.

In this way, the evaluation figure is transformed from a \emph{system of
  object-related usage properties} into a \emph{system of process-related
  functional properties of technical means}. This evaluation figure is used
for the appropriate selection of technical means from the set of relevant
technical objects and their functional couplings to build up the basic variant
of the technical system. Additionally, in this transformation stage the
evaluation figure forms together with the non-transformed component $Z_2$
(profitability) the basis for the definition of the main technical-economic
performance data of the technical system and for their quantitative
determination in terms of a nominal figure.

The \emph{system analysis} is based on this nominal figure. It is aimed at
determining the effectiveness properties of the technical system in their
combination, especially to uncover contradictory tendencies in their
developmental behavior and to reveal the relevant technical causes in the
context of a developmental weakness analysis. The function-related evaluation
figure can already provide an initial indication of the critical functional
area (critical system area). This is usually the area where the greatest
number of sub-functions meet in a technical object.

\subsection*{W.4. The Basic Variant}

The technical means selected from the available state of the art according to
the evaluation figure are divided into \emph{sub-systems in the form of
  separable structural units} based on their function.  Each structural unit
embodies one of the process-related sub-functions as part of the main function
or a necessary support function that is responsible for control, protection
and/or environmental compatibility of the system. With the functional
combination of the technical means to sub-systems and the sub-systems to the
overall system of the basic variant the ABER -- the functional requirements
(\textbf{A}nforderungen) and structural conditions (\textbf{B}edingungen) as
well as the external influences (\textbf{E}infl\"usse) and restrictions
(\textbf{R}estriktionen) -- have to be taken into account which describe
influence of the individual technical means or subsystems on each other when
combined to form the basic variant. To do this, for their coupling (by means
of a morphological scheme) a \emph{ranking} has to be defined according to the
technical-technological importance of the subsystems in such a way that a
subsystem or technical means of higher rank defines the ABER for the
subsystems or technical means on the respective lower levels of hierarchy.

\subsection*{W.5. The Decisive Defect and the Core Variant}

The basic variant developed according to the state of the art or the technical
sciences usually still has decisive deficiencies. These deficiencies can be of
technical-economic nature, arising from the fact that the utility and
profitability properties could not be aligned with the evaluation figure, i.e.
requirements and/or restrictions had to be neglected. The defects can also be
of "heuristic" nature, e.g. if means are neither available nor feasible, but
at most conceivable or even only imaginable.

A \emph{technical-economic deficiency} is present if the technical means
required according to the evaluation figure are principally available or known,
but at least in a crucial usage property the required performance and/or
effectiveness parameters are not achievable or only at the expense of other
parameters relevant for usability.

A \emph{heuristic deficiency} is present if for at least one of the usage
properties required by the ABER no technical means are known which would be
suitable according to their functional properties to produce the required
means-effect relationships. To become aware of a heuristic deficiency requires
inventive instinct and courage to question conventional and proven technology.

For further problem processing, that basic variant is selected which has the
smallest deficiencies. An inventive approach is characterised by the property
that it does not allow any serious technical-economic deficiencies, but
deliberately accepts serious heuristic deficiencies if they challenge for
inventive solutions.  If there is a serious heuristic deficiency the subsystem
or system area where the deficiency appears is declared as the decisive
subsystem or the \emph{core variant} of the technical system. For the
problematic core of the basic variant in this subsystem or system area new,
conceivable solutions are generated by new modifications or previously unusual
combinations of known technical objects.  From these core variants, the one is
chosen that does best fit into the overall context of the technical system of
the basic variant. This may already be an inventive solution as the result of
a heuristic approach, which can be denoted as \emph{projecting invention}.

If the basic variant consists of an inventive core variant with only small
technological scope and in the remainder of verified and tested system
components according to the available state of the art, and does not show
significant deficiencies in relation to the evaluation figure, it can be
optimised, transferred into an overall operational project and tested in a
pilot series or trial production.

However, if there are still considerable deviations between the usability and
effectiveness of the basic variant on the one hand and the evaluation figure
on the other, and in particular the functional characteristics of the core
variant in the overall context of the technical system are still in question,
then the further processing is aimed at identifying the causes of these
deficiencies more precisely and to investigate and fix them. For this purpose,
first of all, a \emph{technical-economic objective} is derived from the
evaluation figure as a more precise specification, which is aimed at the
increase of those performance and/or profitability parameters (main
performance data), that are still in question.

\subsection*{W.6. The Structurally Prepared Basic Variant}

The cause of the deficiencies is initially searched for in the structure of
the technical system. For this purpose, the basic variant is prepared
according to the aspect of its structure abstracting from usage properties of
individual objects or groups of objects to structural properties of the
technical system. This is done in such a way that the technical objects
combined and functionally linked within the basic variant are considered with
regard to their required structural usage properties (mainly contained in the
components \emph{controllability} and \emph{usability} of the evaluation
figure) and are coordinated in such a way that they can be spatially and/or
temporally combined to form the structural units and the overall system of the
basic variant.

This creates to some extend the system-specific structural properties of the
technical means.  Here, above all, the structural properties in the system
area of the core variant are emphasised that primarily influence the specific
performance and/or profitability parameters of the technical-economic
objective.  A variation of the structural properties of the technical system
in the sense of the technical-economic objectives often results in a
deterioration in specific functional properties that already indicates a
technical-economic contradiction.

In many cases this is a conflict between the requirements of
manufacturability, mountability and/or maintainability (or the continuous
process management, the monitorability and controllability of procedures) and
the requirements of functionality, insensitivity to external disturbances and
internal functional security.

Here the inventive processing is initially aimed to find the critical
structural unit or functional weak point that primarily prohibits an optimal
design and dimensioning of the basic variant. By a clever transformation or
redesign of one or more objects within this critical system area the
functional performance can be increased without changing the function itself.
If this succeeds, then an inventive solution of the contradiction between
structural and functional properties of the basic variant has been found in
the sense of the technical-economic objective.  Such an approach is called
\emph{constructive invention}. The inventive solution has to be realised in a
functional sample and to be tested for its functionality.

If it turns out that the technical-economic objective cannot be met without
changing functions, the basic variant has to be prepared concerning its
functional properties and a corresponding system analysis is required.

\subsection*{W.7. The Preparation of the Basic Variant Concerning its
  Functional Properties}

In the functional preparation of the basic variant, the structural properties
of the technical objects are abstracted to their functional properties. The
aim is to \emph{identify the essential functional relationships} of the basic
variant as technical system with its environment, and the internal functional
relationships (means-effect relationships) between its components which are
decisive for that.

First of all, on has to determine the overall function of the basic variant
and its known or foreseeable side effects as well as the interface conditions
to its technical-technological environment. This is done by a \emph{black box
  analysis}.  The interface conditions (boundary conditions) of the black box
define the input variables of the technical system from the specified output
variables of a preceding system within a higher-level process, and its output
variables from the specified input variables of a following system at the same
higher level. Depending on the type of input and output variables the transfer
or support function is given from this results which has primarly to be
implemented by the basic variant as a technical system. This is therefore
defined as the \emph{main function}.

However, this is \emph{not} process-related -- as with the transformation of
the evaluation figure -- but object-related.  That is, the function is not
considered as necessary, process-related activation of certain usage
properties of technical objects, but as \emph{structurally constrained effect
  of certain functional properties of technical means}. On this basis the
necessary technical prerequisites for the creation and maintenance of the main
function -- i.e. for the viability of the technical system -- are determined.
From this the required auxiliary functions are defined, starting from the
types \emph{interference suppression function} and \emph{protection function}.

Defining the \emph{interference suppression function}, one can first refer to
the usage properties that are contained in the component $Z_3$
(controllability) of the evaluation figure.  In addition, it is necessary to
determine what side effects are triggered from the specific objects of the
basic variant during operation or use. These side effects must be recorded as
completely as possible.  There are harmful as well as useful or usable side
effects.  Necessary measures to suppress the harmful side effects caused by
the overall function to acceptable levels lead to the definition of the
interference suppression function.

Necessary measures to suppress harmful effects caused on the main function and
the interference suppression function of the technical system by the
environment, lead to the definition of the \emph{protection function} of the
system. Defining the protection function we can initially be conducted by the
usage properties and usage conditions, which are combined in the component
$Z_4$ (usability) of the evaluation figure. Concerning the harmful effects
from the environment not only technical, technological and natural effects are
to be considered, but possibly also social (qualification, discipline) and
organisational (supply of means of transport, material, energy and/or
information) are to be taken into account.

An important for the inventor functional class are the \emph{auxiliary
  functions}. This are functions, which are generated or can be generated "for
free" by the objects of the basic variant in addition to their main functional
destination.  They have to be investigated whether and to what extent they can
used to support or even to replace functions of one or more other objects of
the basic variant. This can lead to a \emph{trimming of functions}, the effect
of which goes beyond the sum of the individual effects of the objects in
question. This is an important indicator for an inventive achievement.

Auxiliary functions that cannot be used are considered as \emph{unnecessary
  functions}.  They should be eliminated as completely as possible by a more
suitable choice or design of the objects of the basic variant, at least when
they provoke a disruption of the functional value flow or cause unnecessary
costs.

For a complete recording and advantageous design of all interrelationships
between the system and its environment, it is necessary to delimit an
\emph{operational field} for the inventor in relation to the technical system.
It includes all objects -- technical and natural -- as well as all factors --
social, organisational and technological -- with their harmful and beneficial
effects that the technical system with its function has to take into account
or that can be effectively included in its function. It depends on the correct
delimitation of the operational field and the technical system whether the
protection function is correctly determined and whether objectively available
options to simplify functions or to increase the value of functions are
recognised and used for improvements. (See \cite[A.3]{RM-21}).

Depending on the situation, this can be done in such a way that suitable
objects from the operational field are used to support or simplify functions
including them in the basic variant by structural as well as functional
integration by an adaptation function.  Conversely, it can also be
advantageous, and in some cases even necessary to relocate certain objects
from the basic variant to the outer part of the operational field. A close
functional and structural link across the system boundary as a mediation
function can generate a positive influencing factor in the outer operating
field or reinforce an existing one. In certain cases one gets thus at the same
time a simplification of the function of the basic variant or an increase in
their functional value.

With the black box analysis of the basic variant, its function-related
preparation is essentially completed. The knowledge gained about the
functional characteristics of the basic variant, their mutual dependencies and
the possibilities to optimally coordinate them with each other and with the
system environment are now used to attempt to resolve the contradiction
between structural and functional properties that occurred during the
structure-related preparation of the basic variant.  Here you can in an
inventive way, through an original distribution of the required functions to
the individual objects of the basic variant and the skillful use of so far
neglected structural and functional properties create the prerequisites for an
optimal overall solution.

\subsection*{W.8. The Optimisation of the Basic Variant Compared to the
  Reference Variant and the TEC}

In order to be able to optimise the basic variant, a technical performance
parameters has to be determined as a \emph{guiding variable}, the variation of
which affects to a decisive extent on the one hand the effectivity parameters
of the technical-economic objective and, on the other hand, the necessary
structural and functional properties of the technical system. Choosing the
guiding variable, we decide the direction of the further development of the
technical system and the development trend of its utility and profitability
properties.

The guiding variable must therefore be in agreement with the evaluation figure,
even if it turns out that its variation -- although professonally thought
ahead -- leads to changes in the structural and functional properties of the
technical system, which (at least partially) are still in contradiction to the
technical-economic objectives. As orientation for the correct determination of
the guiding variable can serve the reference variant which was derived from
the worldwide material state of the art.

As guiding variable serves such a technical performance parameter in which the
reference variant still deviates at most from the evaluation figure.  That is,
the requested performance (in the component $Z_1$ of the evaluation figure) is
achieved either not at all or under the given implementation conditions only
with impermissibly high technical ($Z_3$), technological ($Z_4$) and/or
economic effort ($Z_2$).  This way, a follow-up strategy is avoided from the
beginning and a progressive solution strategy is designed that meets the real
social needs.

In addition, the reference variant can be included in the black box analysis
as a suggestion for the functional and structural conceptualisation of the
basic variant. Did we already succeed -- possibly in an inventive way -- to
develop a basic concept that can be optimised then we will find an optimal
overall solution for the basic variant that meets the technical-economic
objective through well-coordinated design and dimensioning of the individual
objects. If it is found, then the functionality and the functional value of
the basic variant has to be tested on a test sample.  For this, it is
sufficient to reconstruct that part of the functional area of the basic
variant, in which the decisive structural and functional changes are located
that were carried out compared to the tried and tested state of the art. As a
rule, it is the core variant and its closer system environment.

If an optimal overall solution has not yet been found or if the design of the
basic variant proves to be not functional, the decisive \emph{TEC} is to be
determined.  In other words, the decisive technical-economic effectiveness
parameters have to be determined that relate to one another in such a way that
the increase in one parameter leads to an impermissible reduction of the other
parameter, if the guiding variable is varied according to the
technical-economic objective (see \cite[A.4]{RM-21}).

The further processing is now no longer possible through \emph{accompanying}
or tactical inventing, but characterised by \emph{forward-looking}, strategic
invention. It is actually inventing in the true meaning of the method, the
\emph{invention itself}. This brings us to stage 2 of the organisational
model, at the beginning of which a \emph{renewal pass} and a \emph{draft
  specification} with a clear inventive task have to be negotiated.  The
subject of the invention is now a technical-economic contradiction, the goal
is to overcome it.

\subsection*{W.9. The Inventive Core Variant (Key Variant)}

To overcome the TEC, it is assumed that its cause is not distributed around
the whole technical system, but essentially focused in a specific system area
-- the area that is critical for the functioning of the technical system, the
\emph{critical functional area}. The inventive goal now is to \emph{discover}
this system area and to produce an inventive solution, that creates in this
critical area new technical conditions, new means-effect relationships,
opening new possibilities for the development of the basic variant and the
corresponding variation of the guiding variable in terms of the
technical-economic objective.

\subsection*{W.10. The Critical Functional Area of the Basic Variant and
  the ABER}

Based on the result of the black box analysis, the findings during the
unsuccessful optimisation of the basic variant and possibly from a functional
trial that was completed with negative test results, the causes of the
technical-economic contradiction are to be explored. (See \cite[A.5]{RM-21}).

For this purpose, continuing the black box analysis the basic variant is first
divided into the object-related sub-functions, which are essential for a
viable chain of intended and/or prevented \emph{state changes of one or more
  objects} that produces in the end a stable and effective main function.  For
the specification of the necessary functional features one can exploit the
usage properties that are summarised in the component $Z_1$ (functionality) of
the evaluation figure in terms of effects of their activation. Thus one can
assign system components contained in the basic variant as objects to
individual sub-functions and assess them with regard to their functional
value.

It will always be possible to delimit an area of the technical system in which
one or several sub-functions are present, which in comparison to the
neighboring system areas have a significantly lower functional value. This
system area acts like a bottleneck in the function value flow of the main
function, which does not take full effect of this and other sub-functions and
thus decisively limits the overall functionality of the technical system. It
is therefore called \emph{critical functional area} of the technical system.

For the inventor, it is not only the question of the technical-technological
causes for the emergence of the functional bottleneck, but also the question
of the \emph{technical-constructive or technical-operational reasons} that
prevent the elimination of these causes on the way to an optimal dimension.
These reasons are to be reduced to a \emph{harmful technical effect} (HTE),
which prevents the development of the technical system according to the
evaluation figure.

The answer to this question about the HTE, which is crucial for the
inventional task, can only be derived gradually. For this purpose, the
sub-functions created in the critical functional area are divided according to
their operating principle into \emph{elementary functions} and their
corresponding \emph{functional units}. The individual functional units are
resolved into their operational parts -- \emph{operation}, \emph{operand},
\emph{operator} and \emph{counter-operator} -- and these parts technically
defined as functional parameter according to the operating principle of the
respective functional unit. In this way the critical functional area can be
clearly and transparently displayed in a morphological scheme.

The \emph{function value flow} is now examined from elementary function to
elementary function.  In doing so, following a suitably selected guiding
variable (structural variable), an optimisation of the functional units is
attempted varying the functional parameters while keeping the operating
principle.

Depending on the result of these optimisation attempts, the core of the
harmful technical effect may be restricted to certain functional units and
their structural and functional properties. This means that the critical
functional area is increasingly narrowed and more precisely defined.  At the
same time, the technical-technological requirements, constraints, influences
and restrictions (the ABER) are determined in their specific for the technical
system form of interrelations that determine the technical-technological core
of the problem.

That is, this set of ABER becomes a \emph{system} which links sub-objects of
the basic variant. This ABER on the technical-technological level is the
analogue of the ABER on the technical-economic level. Note that at this level,
we work with \emph{influences} (German: \textbf{E}inflüsse) rather than
\emph{expectations} (German: \textbf{E}rwartungen) as \textbf{E}.

These new ABER prevent to overcome the TEC. They are to be changed in a
following stage of the inventive process in the sense of a technical ideal
(IDEAL) in such a way that the HTE disappears. In that process it is initially
not allowed to vary technical functional requirements or restrictions by
natural laws.

\subsection*{W.11. The Harmful Technical Effect and the IDEAL}

The (technical) IDEAL primarily refers to the behavior of the technical system
in its critical functional area. Initially, the rest of the technical system
is essentially set immutable. With the IDEAL, such ideal constructive
conditions and/or such ideal operational arrangements are thought ahead the
recognised optimisation limits, that all undesired technical-technological
factors of influence disappear or are at least reduced to such an extend in
their effect that the functional value in the critical functional area
decisive increases. The functional principle or the operational principle are
initially not changed.  (See \cite[A.6]{RM-21}).

In contrast to the technical-economic ABER the technical-technological ABER
are not directly derived from the social supersystem and the technological
environment of the technical system, but rather from its constructive or
operational structure and the functional principle implemented there.  With
these ABER next to \emph{requirements}, \emph{conditions} and
\emph{restrictions} also \emph{influences} (in the sense of side effects) of
technical-constructive and technical-scientific type are recorded, which the
components of the technical system exert on each other, or which affect them
from the system environment.

Opposing new conditions and pushing back the influencing factors has to
respect the technical requirements for structural and functional basic
properties of the basic variant and must not violate restrictions by natural
laws, which are set by the overall function of the basic variant in a
principal way.  Otherwise the IDEAL will cause another harmful technical
effect in another system area, which usually also leads to a specific TEC.

If it turn out that the elimination of a harmful technical effect is only
possible with the emergence of another one, so in any case it is to "scout"
whether there is one of these harmful consequential effects, against which a
supplementary IDEAL may be thought that meets all the requirements and
restrictions of the technical system.  As a rule, however, this requires a
detailed examination of the structural and functional interrelationships of
the technical system -- at least in the environment of the critical functional
area.

In order to avoid an odyssey through the technical system in such a situation,
this exploratory procedure therefore only makes sense as long as it does not
go too far beyond the originally delimited system.  If such an IDEAL approach
is found that can be developed further, a stable \emph{technical effect} and a
mediating \emph{functional principle} have to be searched which correspond to
the new conditions and influencing factors in the system of
technical-technological ABER.  This is the starting point to develop the
\emph{sub-function principles} and the \emph{technical principle} of the
inventive solution for the core variant (key variant) that can be tested in a
test sample.

However, if no viable IDEAL has been found, so the further processing starts
with the approach that is in greatest compliance with the requirements and
restrictions in the system of technical-technological ABER. The findings from
the exploration of the technical system are now summarised as the
technical-technological contradiction (TTC). (See \cite[A.7]{RM-21}).

\subsection*{W.12. The TTC and the New Functional Principle for the Key
  Variant} 

The TTC substantiates the specific technical issue that removing the initially
found harmful technical effect \emph{necessarily} yields another, just as
difficult to remove harmful effect. To resolve this contradiction now the
\emph{general problem-solving principles} are brought into consideration. (See
\cite[A.9]{RM-21}).

If a solution approach has been found to overcome the TTC, it is first to be
checked its \emph{technical effect} at the IDEAL for its principal usability.
Then the technical-technological ABER are to be modified accordingly, and it
is important to ensure that this does not violate the technical requirements
or restrictions by natural laws.  Finally, it must be checked whether the
\emph{harmful technical effect} has actually been eliminated and no new TTC
did appear.  Only then it is time to define -- with reference to the IDEAL --
a more detailed specification of the \emph{new technical effect} and the
system-compatible expression of the \emph{new functional principle} for the
key variant.

If a useful approach to solve the TTC is not found, the technical-scientific
fact is to be determined, which decisively opposes this solution.  For this
purpose, the system analysis is directed to the critical operational point of
the key variant, where those technical-scientific restrictions start that are
decisively involved in the creation of the TTC. This technical-scientific
restriction emanating from the critical operational point is called a
\emph{harmful scientific effect} (HSE). It essentially consists in the fact
that the operational principle of a partial technical effect which should be
evoked at the critical point to provide an effect of functional importance or
an utility value prohibits certain functional and/or structural changes in the
vicinity of this operational point. (See also \cite[A.7]{RM-21}).

\subsection*{W.13. The Technical-Scientific Contradiction (TSC) and the New
  Operational Principle for the Key Variant}

In a \emph{database of scientific effects and principles}, such approaches are
searched for that produce the necessary technical partial effect at the
critical operational point in at least the same strength but the original
restriction by natural law is no more relevant.

Of course, it must always be checked whether only a problematic restriction
has been exchanged against another. This examination can initially be done on
a theoretical basis of a technical-scientific model at the operational point
and its immediate surrounding.  In this process the elementary (functional and
structural) conditions and relationships have to be investigated that are
required to create the necessary conditions for the appearance of the new
technical partial effect at the operational point. It always turns out that at
least one of these conditions must be met without restriction due to the
selected operating principle. That is, it is to be regarded as the \emph{new
  restriction by natural law}.

To determine whether or not this new restriction prevents solving the problem
it can be compared with the IDEAL within the system context of the
technical-technological ABER and examined whether the harmful technical effect
is now eliminated or the TTC can be solved.  If this is the case, starting
from the IDEAL a \emph{specification of the new technical effect} and the
system-compatible \emph{expression of the new functional principle} for the
key variant are to be developed.

However, before the partial function principles and the technical principle
are developed from this, the simplifying assumptions and the neglected
possible secondary effects and subordinate influencing factors of the
\emph{technical-scientific model} have experimentally to be checked for
validity and reliability.  A \emph{laboratory sample} has to be used for this,
i.e. a reproduction of the structure of the technical system in the area of
the critical operational point.

If, even after several approaches, a suitable technical-scientific operational
principle for the solution of the TTC is \emph{not} found, the knowledge
gained is expressed as \emph{technical-scientific contradiction} (TSC). This
substantiates the problem-specific scientific fact that for the technical
system of the basic variant \emph{there is no operational principle}, that
removes a technical-scientific restriction without causing other equally
serious ones.  The reason for this are \emph{restrictive functional and/or
  structural conditions and requirements} of the technical system that do not
allow new operating principles to be developed either.

With the help of the \emph{general problem-solving principles} an attempt is
now made to "weaken" these conditions and requirements in such a way that one
of the considered operational principles no longer leads to a TSC.  This means
that also the TTC and the TEC can be solved in principle. (See
\cite[A.9]{RM-21}).

\section{The ABER(1)-Matrix. An Example}

Evaluation parameters and components. An example of the ABER matrix by
Hans-Jochen Rindfleisch and Rainer Thiel published in \cite[appendix
  III/2]{RT94}.
  
\begin{center}\renewcommand{\arraystretch}{1.5}
  \begin{tabular}{|l|c|c|c|c|}\hline
    & {Functionality} & {Profitablity} & {Controllability} &
    {Usefulness}\\\hline
    \textbf{A:} Requirements &  (A.1) & (A.2) & (A.3) & (A.4)\\\hline
    \textbf{B:} Conditions &  (B.1) & (B.2) & (B.3) & (B.4)\\\hline
    \textbf{E:} Expectations& (E.1) & (E.2) & (E.3) & (E.4)\\\hline
    \textbf{R:} Restrictions& (R.1) & (R.2) & (R.3) & (R.4)\\\hline
  \end{tabular}\par 
\end{center}
\begin{itemize}[noitemsep]
\item[(A.1)] Performance and fitness to drive up to a driving speed of $x$
  km/h
\item[(A.2)] 1. Fuel saving\par 2. Utilising exhaust gas heat
\item[(A.3)] 1. Easy to operate, wearing parts easily accessible\par
  2. Replacement parts available on board (or can be carried)
\item[(A.4)] 1. Adaptable to local traffic conditions\par 
  2. Can be used as truck unit, delivery van and touring van
\item[(B.1)] 1. Approved for road transport\par 2. Can be used as traction unit
\item[(B.2)] 1. Service-friendly\par 2. Well suited fo load transport
\item[(B.3)] 1. Temporarily overloadable to $x$ times normal load\par
  2. Driving behaviour (undelayed), follows steering
\item[(B.4)] 1. Insensitive to stone impact\par 2. Heat repellent\par
  3. Temperature regulating\par 4. Humidity balancing
\item[(E.1)] 1. High acceleration capacity\par 2. Delay-free acceleration
\item[(E.2)] 1. High transport yield\par 2. Low cost
\item[(E.3)] 1. Automatically compensating skidding movements\par
  2. Self-adjusting to changing road conditions\par 3. Self-monitoring 
\item[(E.4)] 1. Independent of service stations\par 2. Insensitive to low
  temperatures (e.g. when starting)
\item[(R.1)] 1. Traction and braking system true to track\par
  2. Lighting and signalling system in conformity with traffic regulations
\item[(R.2)] 1. Undemanding in terms of maintenance\par 2. Frugal in terms of
  fuel quality 
\item[(R.3)] 1. Traffic-safe\par 2. Vibration resistant\par 3. Shock and
  impact resistant\par 4. Theft-proof
\item[(R.4)] 1. Compatible with emission standards\par 2. Corrosion resistant
  to de-icing salt\par 3. Harmless for inner-city traffic
\end{itemize}

\section{German-English Translations of Terms}
\begin{center}
  \begin{tabular}{l|l}
    English & German \\\hline
    adaptation function & Anpassfunktion \\
    auxiliary function & Nebenfunktion \\
    available & verfügbar \\
    basic variant & Basisvariante\\
    conceivable & denkbar \\
    constructive invention & konstruierendes Erfinden \\
    controllability & Beherrschbarkeit \\
    core variant & Kernvariante\\
    critical functional area & kritischer Funktionsbereich\\
    critical funtion matrix & kritische Funktionsmatrix\\
    critical operational area & kritischer Wirkbereich\\
    deficiency & Mangel \\
    developmental weakness analysis & Entwicklungs-Schwachstellen-Analyse\\
    draft specification & Pflichtenheft \\
    evaluation figure & Zielgröße\\
    evaluation matrix& Zielgrößenmatrix \\
    evaluation parameter & Zielparameter \\
    feasible & machbar \\
    follow-up strategy & Nachlaufstrategie \\
    functional area & Funktionsbereich \\
    functionality & Zweckmäßigkeit \\
    functional structure & Funktionsstruktur \\
    functional unit & Funktionseinheit \\
    function value flow & Funktionswertfluss \\
    guiding variable & Führungsgröße \\
    harmful natural laws & schädliche Naturgesetzmäßigkeit\\
    heuristic deficiency & heuristischer Mangel\\
    honorable inventors & Verdiente Erfinder\\
    imaginable & vorstellbar \\
    impact value & Wirkgröße \\
    interference suppression function & Entstörfunktion \\
    main function & Hauptfunktion \\
    main performance data & Hauptleistungsdaten \\
    means-effect relationship & Mittel-Wirkungs-Beziehung \\
    mediation function & Vermittlungsfunktion \\
    nominal figure & Sollgröße \\
  \end{tabular}
  \begin{tabular}{l|l}
    English & German \\\hline
    objectives & Zielstellung \\
    operational area & Wirkbereich\\
    operational effects & funktionstragende Effekte\\
    operational field & Operationsfeld \\
    operating principle & Wirkprinzip\\
    operational field matrix & Wirkfeldmatrix\\
    path model & Wegemodell\\
    problem field level & Problemfeldebene \\
    procedure & Verfahren \\
    profitability & Wirtschaftlichkeit \\
    protection function & Schutzfunktion \\
    reference variant & Referenzvariante\\
    renewal pass & Erneuerungspass \\
    social need & gesellschaftliches Bedürfnis\\
    structural value & Strukturgröße\\
    subfunction & Teilfunktion\\
    suitability & Zweckmäßigkeit\\
    suitability characteristics & Eignungsmerkmale\\
    support function & Hilfsfunktion, Tragfunktion \\
    surprising impact (SI) & überraschende Wirkung \\
    surprisingly simple solution (SSS) & raffiniert einfache Lösung (REL)\\
    target component & Zielkomponente\\
    technical-constructive boundary conditions & technisch-konstruktive
    Randbedingungen \\
    technical-economic deficiency & technisch-ökonomischer Mangel\\
    technical-economic objectives & technisch-ökonomische Zielstellung\\
    technical-operational reason & verfahrenstechnischer Grund \\
    technical-technological principle & technisch-technologisches Prinzip\\
    usability & Brauchbarkeit \\
    usage properties & Gebrauchseigenschaften \\
    viability & Funktionsfähigkeit \\
  \end{tabular}
\end{center}

\begin{thebibliography}{xxx}
\bibitem{Altshuller1973} Genrich S. Altshuller. Erfinden – (k)ein Problem?
  Verlag Tribüne, Berlin (1973).
\bibitem{Altshuller1983}  Genrich S. Altshuller, Alexander B. Seljuzki. Flügel
  für Ikarus. Urania-Verlag, Leipzig (1983).
\bibitem{Altshuller1986} Genrich S. Altshuller. Erfinden. Wege zur Lösung
  technischer Probleme. Verlag Technik, Berlin (1986). 
\bibitem{Barkleit} Gerhard Barkleit. Mikroelektronik in der DDR.  Berichte und
  Studien Nr. 29, Hannah-Arendt-Institut, Dresden 2000.  ISBN 3-931648-32-X.
\bibitem{Graebe2019a} Hans-Gert Gräbe. The Development of the GDR Inventor
  Schools and the Evolution of TRIZ (in Russian). In: Online material of the
  TRIZ Summit, Minsk (2019).
\bibitem{Graebe2019b} Hans-Gert Gräbe. The Contribution to TRIZ by the
  Inventor Schools in the GDR. Proceedings of the 15th MATRIZ TRIZfest, pp.
  346-352 (2019).
\bibitem{ProHEAL-21} Hans-Gert Gräbe, Rainer Thiel. ProHEAL – Social Needs and
  Sustainability Aspects in the Methodology of the GDR Inventor Schools.
  LIFIS Online, 2021. \url{http://dx.doi.org/10.14625/graebe_20210815}
\bibitem{HMT1985} Dieter Herrig, Herbert Müller, Rainer Thiel. Technische
  Probleme -- dialektische Widersprüche -- erfinderische Widerspruchslösung.
  In: Maschinenbautechnik 6/1985, Berlin.
\bibitem{Herrlich1982} Michael Herrlich. KDT-Erfinderschule, Lehrmaterial, 
  1. und 2. Teil.  Berlin 1982.
\bibitem{Herrlich1986} Michael Herrlich. Vorschläge zur künftigen Gestaltung
  der Aus- und Weiterbildung im erfinderischen Schaffen. In: Das
  Hochschulwesen 3 (1986) Heft 7, Berlin.  
\bibitem{Herrlich1988} Michael Herrlich. Erfinden als
  Inforrnationsverarbeitungs- und -generierungsprozeß, dargestellt am eigenen
  erfinderischen Schaffen und am Vorgehen in KDT Erfinderschulen. Dissertation
  A, TH Ilmenau, 1988.\\ \url{http://d-nb.info/900036486}
\bibitem{Hill1987} Bernd Hill. Methoden des Erfindens und ihre Nutzung zur
  Förderung technisch begabter Schüler neunter Klassen. Unveröffentlichtes
  Material, Pädagogische Hochschule Erfurt, 1987.
  \url{http://d-nb.info/890765634}
\bibitem{RM-23} Peter Koch, Klaus Stanke. 50 Jahre systematische Heuristik.
  Rohrbacher Manuskripte, Heft 23. LIFIS, Berlin (2020).
\bibitem{Linde1988} Hansjürgen Linde. Gesetzmäßigkeiten, methodische Mittel
  und Strategien zur Bestimmung von Entwicklungsaufgaben mit erfinderischer
  Zielstellung.  Dissertation A, TU Dresden, 1988.
  \url{http://d-nb.info/890630186} 
\bibitem{LindeHill1993} Hansjürgen Linde, Bernd Hill.  Erfolgreich erfinden:
  Widerspruchsorientierte Innovationsstrategie für Entwickler und
  Konstrukteure.  Darmstadt, 1993.  ISBN: 978-3-87807-174-7
\bibitem{Mueller1973} Johannes Müller, Peter Koch et al. Programmbibliothek
  zur systematischen Heuristik für Naturwissenschaftler und Ingenieure. In:
  Wissenschaftliche Abhandlungen des Zentralinstituts für Schweißtechnik
  Halle, Band 97-99, Halle (1973).
\bibitem{Mueller1990} Johannes Müller. Arbeitsmethoden der
  Technikwissenschaften.  Systematik, Heuristik, Kreativität. Springer, Berlin
  (1990).
\bibitem{RT1986} Hans-Jochen Rindfleisch, Rainer Thiel. Beiträge zur Erhöhung
  des erfinderischen Schaffens. Bauakademie der DDR. Berlin 1986.
\bibitem{RT89} Hans-Jochen Rindfleisch, Rainer Thiel, Gerhard Zadek.
  KDT-Erfinderschule, Lehrbrief 2: Erfindungs­methodische Arbeitsmittel.
  Lehrmaterial zur Erfindungsmethode. Berlin (1989).
  \url{https://wumm-project.github.io/GIS}
\bibitem{RT94} Hans-Jochen Rindfleisch, Rainer Thiel. Erfinderschulen in der
  DDR. Trafo Verlag, Berlin (1994).
\bibitem{Speicher1980} Karl Speicher. Beiträge zur Förderung technischer
  Erfindungen und Spitzenleistungen. Genesen und methodologisch orientierte
  Analysen eigener Erfindungen. Unveröffentlicht. Berlin 1980.
\bibitem{Thiel1977} Rainer Thiel (Hrsg.). Methodologie und Schöpfertum.
  Teilnehmerbeiträge zum Kolloquium am 1./2. Dezember 1977. Berlin 1977.
\bibitem{Thiel1980} Rainer Thiel. Dialektische Widersprüche in der
  alltäglichen Ingenieurarbeit -- Verhältnis von Kompromiß (Optimierung) und
  erfinderischer Widerspruchslösung -- Widerspiegelung dieses Verhältnisses in
  der Hochschulliteratur zur Ingenieurausbildung. Vortrag in der
  Arbeitsgemeinschaft „Erfindertätigkeit und Schöpfertum“ beim Bezirksvorstand
  der KDT, 1980.
\bibitem{Thiel1986} Rainer Thiel.  Methodologische Grundlagen des
  schöpferischen Problemlösungsprozesses. Reihe \emph{Grundlagen des
    wissen\-schaft\-lich-technischen Schöpfertums in Forschungs- und
    Entwicklungsprozessen}, Lehrbrief 2.2., Berlin und Jena, 1986.\\
    \url{http://d-nb.info/1035446324} 
\bibitem{Thiel2016} Rainer Thiel.  Erfinderschulen – Problemlöse-Workshops.
  Projekt und Praxis. LIFIS-Online, 03.07.2016.
  \url{http://dx.doi.org/10.14625/thiel_20160703}.
\bibitem{RM-21} Rainer Thiel.  Dialektik, TRIZ und ProHEAL. Rohrbacher
   Manuskripte, Heft 21. LIFIS, Berlin (2020).
\bibitem{Zobel1985} Dietmar Zobel. Erfinderfibel. Berlin 1985.
\bibitem{Zobel1991} Dietmar Zobel. Erfinderpraxis. Berlin 1991. ISBN
  3-326-00131-2.
\bibitem{Zobel2001} Dietmar Zobel. Systematisches Erfinden -- Methoden und
  Beispiele für den Praktiker. Renningen 2001, 5th edition 2009. ISBN
  978-3-8169-2939-0.
\bibitem{Zobel2006} Dietmar Zobel. TRIZ FÜR ALLE -- Der systematische Weg zur
  Problemlösung. Renningen 2006. ISBN 978-3-8169-2760-0.
\bibitem{Zobel2007} Dietmar Zobel. Kreatives Arbeiten – Methoden, Erfahrungen,
  Beispiele.  Renningen 2007. ISBN 978-3-8169-2713-6. 
\bibitem{Zobel2009} Dietmar Zobel, Rainer Hartmann. Erfindungsmuster -- TRIZ:
  Prinzipien, Analogien, Ordnungskriterien, Beispiele.  Renningen 2009. ISBN
  978-3-8169-2904-8.
\end{thebibliography}

\end{document}
