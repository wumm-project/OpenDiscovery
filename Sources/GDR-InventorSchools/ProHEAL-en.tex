\documentclass[11pt,a4paper]{article}
\usepackage[utf8]{inputenc}
\usepackage{od}
\usepackage[english]{babel}

\parindent0pt
\parskip4pt
\setlist[itemize]{noitemsep}

\title{ProHEAL Basics -- Extended Version}

\author{Hans-Gert Gr\"abe, Rainer Thiel}
\date{Version of June 2021} 

\begin{document}
\maketitle

\section{ProHEAL Basics -- a Short Overview}

\begin{center}
  \begin{tikzpicture}[scale=.95,transform shape,
    bbox/.style={draw,fill=white,rectangle,align=center},
    >={Triangle[length=0pt 6,width=0pt 5]},
    rounded corners=2pt,line width=.8pt]
  \draw[dashed,line width=.3pt] (4,11) -- (4.5,3) ;
  \draw[dashed,line width=.3pt] (12.5,11) -- (12,3) ;
  \node[bbox,text width=4cm] at (8,13) (A0)
       {Technical-economic problem situation};
  \node[text width=3cm,align=center] at (8,11) (B0)
       {Technical-economic field of operation};
  \node[bbox,text width=3cm] at (3,11) (B0a)
       {Societal need};
  \node[bbox,text width=3.5cm] at (13,11) (B0b)
       {State of the art. Basic variant};
  \node[bbox,text width=5cm] at (8,9) (A1)
       {Technical-economic (external) contradiction};
  \node[text width=3cm,align=center] at (8,7) (B1)
       {Critical functional area of the basic variant};
  \node[bbox,text width=3cm] at (3.3,7) (B1a)
       {IDEAL of the core variant};
  \node[bbox,text width=3.5cm] at (13,7) (B1b)
       {Harmful technical effect};
  \node[bbox,text width=5cm] at (8,5) (A2)
       {Technical-technological (inner) contradiction};
  \node[text width=3cm,align=center] at (8.5,3)  (B2)
       {Critical operational area of the core variant};
  \node[bbox,text width=4cm] at (3.6,3) (B2a)
       {Ideal scientific effect};  
  \node[bbox,text width=3.5cm] at (13,3) (B2b)
       {Harmful scientific effect};
  \node[bbox,text width=6cm] at (8,1) (A3)
       {Technical-scientific (internal) contradiction};
  \node[bbox,text width=3cm] at (8,-1) (A4)
       {Solution strategy};
  \node[rotate=90,text width=3.6cm] at (0,3) (D1)
       {Problem solution\\ imaginable at the level of natural sciences}; 
  \node[rotate=90,text width=3.5cm] at (0,7) (D2)
       {Problem solution conceivable at the state of art of technical
         science};   
  \node[rotate=90,text width=3.5cm] at (0,11) (D2)
       {Problem solution feasible at the state of the art};
  \draw[->] (A0) -- (B0a) ;
  \draw[->] (B0a) -- (A1) ;
  \draw[->] (A0) -- (B0b) ;
  \draw[->] (B0b) -- (A1) ;
  \draw[->] (A1) -- (B1a) ;
  \draw[->] (B1a) -- (A2) ;
  \draw[->] (A1) -- (B1b) ;
  \draw[->] (B1b) -- (A2) ;
  \draw[->] (A2) -- (B2a) ;
  \draw[->] (B2a) -- (A3) ;
  \draw[->] (A2) -- (B2b) ;
  \draw[->] (B2b) -- (A3) ;
  \draw[->] (A3) -- (A4) ;
  \draw[->,dashed] (B0) -- (B0a) ;
  \draw[->,dashed] (B0) -- (B0b) ;
  \draw[->,dashed] (B1) -- (B1a) ;
  \draw[->,dashed] (B1) -- (B1b) ;
  \draw[->,dashed] (B2) -- (B2a) ;
  \draw[->,dashed] (B2) -- (B2b) ;
\end{tikzpicture}
\end{center}

\section{The Problem Field Levels in the ProHEAL Path Model}

The ProHEAL path model provides for a graded structuring of the problems in
different problem field levels as manageable sections of a solution path and
thus creates for the first time the necessary prerequisites for systematic
application of various methodological instruments for determining the current
location of the real core problem.

\subsection{The Technical-Economic Problem Field Level}

At this first level, all problem-determining facts come into consideration
that the social need in the sense of a potential need for a solution and the
status of technology as a system of available technical products and processes
in the sense of a potential solution offer.

The consideration is person- and process-related and determined by the
product-goods-purpose relationship\footnote{German:
  Produkt-Ware-Zweck-Relation}.

Results in this problem field level are
\begin{itemize}
\item the \emph{technical-economic objectives} of an innovation project,
\item the \emph{basic variant} of a process or product innovation that is
  tailored to the technological requirements,
\item the \emph{critical functional area} in the multi-dimensional
  optimization behavior of this basic variant,
\item the \emph{technical-economic contradiction} (TEC) that prevents an
  optimal design and tayloring of the basic variant.
\end{itemize}
If there is no basic variant that can be optimized in terms of the
technical-economic objective, we are faced with an inventive problem that is
to be analyzed at the next level on which the solution of the TEC is the aim
of the invention.

\subsection{The Technical-Technological Problem Field Level}

At this next level, all the facts are considered that affect the technical
system of the basic variant, its structure, function, its behavior and its
immediate technological environment.

The consideration is object- and function-related and determined by the
technical means-action-counteraction relations\footnote{German: Technische
  Mittel-wirkung-Gegenwirkung-Relation}.

Results in this problem field level are
\begin{itemize}
\item the in the sense of solving the TEC \emph{ideal technical subsystem} in
  the critical functional area of the basic variant,
\item the \emph{undesired effects} as not inteded, technically disadvantageous
  influence of the ideal subsystems on the functional behavior of the basic
  variant,
\item the \emph{critical operational area} in the functional structure where
  the causal interdependency of the ideal subsystem and the undesired effects
  are located,
\item the \emph{technical-technological contradiction} (TTC), that prevents to
  eliminate or suppress the undesired effect by varying the functional
  principle of the ideal subsystem.
\end{itemize}
If a technical subsystem with an alternative functional principle for the
critical functional area of the basic variant can be found without causing a
significant undesirable side effect, then we obtained an invention as a
solution to the TEC. Due to the heuristic approach, this often turns out to be
located in the low-tech area, as "surprisingly simple solution (SSS)" that in
the best case only requires a technical trial run as application.

If the solution at this problem field level is not achieved, the problem
situation has to be formulated as invention task that contains the TTC as well
as a solution strategy tailored to this contradiction. This aims at defining
the harmful natural laws in the critical operational area of the functional
structure and to replace it with an alternative, known operating principle.
We move forward to the third problem field level.

\subsection{The Technical-Scientific Problem Field Level}

At this third level, all facts come into consideration that concern the
operating principle, the requirements for its technical use as well as its
theoretical and experimental basics.

The consideration is model- and event-related and determined by the
field-factor-effect relationships\footnote{German:
  Feld-Faktor-Effekt-Relationen}.

Results in this problem field level are
\begin{itemize}
\item the \emph{ideal operating principle} that solves the TTC in the critical
  operational area of the functional structure,
\item the \emph{harmful natural law} that prevents the technical deployment of
  the ideal operating principle,
\item \emph{new technical-constructive boundary conditions} in the critical
  operational area, which suppress the effects of the harmful natural law,
\item the \emph{technical-scientific contradiction} (TSC), which prevents the
  development of the ideal principle by varying the technical-constructive
  boundary conditions in the critical operational area.
\end{itemize}
If the new operating principle can be technically unfolded in the necessary
dimensions to ensure the fulfillment of the function in the critical range, we
are faced with an invention as a solution to the TTC.  Since this enters new
technical-scientific territory, the solution is usually in the high-tech
area. It requires for its verification application-oriented fundamental
research.

If a solution to the problem cannot be found in this way, we are faced with a
system-immanent TSC, that questions the development and viability of the
system as a whole. The solution strategy then requires the search for a
suitable, so far unknown operating principle or a fundamental process
innovation. Both solution strategies usually go beyond the scope of a timely
and financially definable innovation project. They were therefore not subject
to further methodological considerations in the inventor school, insofar as
they could not based on corrections of the existing process and a
corresponding new solution for the basic variant.

\section{The ABER Matrices as a Strategic Tool in the Invention Methodology}

For the path model through the problem field levels, the invention methodology
proposes a set of methodological instruments, which from its heuristic
application characteristics includes three categories of tools and techniques:
\begin{itemize}
\item \emph{Strategic tools} for goal and route planning, for working out the
  problem-determining contradiction at each level and to find solution
  strategies to overcome such a contradiction. These tools differ in the three
  problem field levels and have an inventive method specifics.
\item \emph{Tactical tools} for the procurement and processing of information,
  for the generation of solution variants and their evaluation according to
  given solution strategies.
\item \emph{Creativity techniques} to activate and strengthen intuition,
  imagination and fluency in thinking and the ability to abstract, associate
  and lateral thinking.
\end{itemize}
The technical tools and creativity techniques do not have inventive methodical
specific. They can be used in all three problem field levels. Your choice is
determined solely by the heuristic specifics of the respective work situation
and the activities related to the situation.

\subsection{The Evaluation Matrix (ABER(1) Matrix)}
This is used to systematically record the goal-determining
\begin{center}
  \begin{tabular}{lcl}
    Requirements (\textbf{A}nforderungen) && Functionality\\
    Conditions (\textbf{B}edingungen) & \emph{related to}&
    Profitablity\\
    Expectations (\textbf{E}rwartungen) && Controllability\\
    Restrictions (\textbf{R}estriktionen) && Usefulness
  \end{tabular}
\end{center}
of the technical system that is to be the subject of the innovation.

\begin{center}\renewcommand{\arraystretch}{1.5}
  \begin{tabular}{|l|c|c|c|c|}\hline
    & {Functionality} & {Profitablity} & {Controllability} &
    {Usefulness}\\\hline 
    \textbf{A:} Requirements & &&&\\\hline
    \textbf{B:} Conditions   & &&&\\\hline
    \textbf{E:} Expectations   & &&&\\\hline
    \textbf{R:} Restrictions & &&&\\\hline
  \end{tabular}
\end{center}

The \emph{need for innovation} which is expressly or implicitly expressed in a
technical-economic problem situation results, for example, from increased
requirements, changed conditions, new expectations and tightened restrictions
with regard to production, distribution, use, abrasion or removal of the
technical system.

The ABER(1) matrix has 16 fields and contains at least as many evaluation
parameters as elements. It thus serves to systematically question the actual
\emph{need for action}, the \emph{objective of action} and the \emph{project
  idea} on which the innovation project is based, and converts these into
technical-economic system properties of the technical product or service with
direct reference to the corresponding process parameters.

Working with the ABER(1) matrix therefore also includes a process analysis
going beyond the scope of the actual objective of the action. As result of
this analysis the technical system with its overall function is delimited in
the sense of a black box model and sufficiently defined with its interfaces in
the overall process. It is important that no process level is skipped to
capture also hidden facts, that not immediately trigger need for action and
therefore are not mentioned in the action goal, but may cause additional
problems.

This already may result in a \emph{more precise definition of the action
  objective} and in a \emph{modification of the project idea}, which can be
decisive for the later invention. In the end the intention of the ABER matrix
is to include all conceivable "yes, but" to anticipate, which would otherwise
be opposed against an invention when it comes to putting it into production
and introduce it to the market.

The heuristic goal of further work with the ABER(1) matrix is first to find
out the technical-economic parameter that serves as \emph{guiding variable}
for the objective of the action according to variation as an independent
variable, the variation behavior of the parameter system the evaluation
figure\footnote{The German "Zielgröße" is translated as "figure" since it is
  multidimensional by definition.} as a whole. In the further analysis of the
complex evaluation figure it is important to define the systemic,
technical-economic problem that results from it this variation of the
reference variable results.

The \emph{technical-economic problem} is basically seen in the fact that the
variation of the reference variable deteriorate other, high-ranking evaluation
parameters to an inadmissible degree or they cannot be complied with in terms
of limit values.

Whether this problem can be solved within the framework of a professional
design and dimensioning or whether the system-related limits of parameter
optimization are exhausted here, i.e.  whether there is a \emph{need for
  innovation} in the sense of solving a TEC, can of course only be determined
on the specific technical system.  This can be an existing technical system in
terms of the required overall function (reference variant) or one composed of
components of the known and commercially available state of the art (basic
variant). The advantage of a reference variant is that optimization algorithms
as well as manufacturing and operational experience are available. The
potential for error is therefore relatively small. But the potential for
contradiction is high as the system as a whole may be out of date. It is the
other way in the case with a basic variant. A decision goed usually for a
basic variant with a balanced ratio of potential for error and contradiction.

\subsection{The critical function matrix (ABER(2) matrix)}

It serves to systematically delimit the critical functional area and to define
the \emph{technical-technological innovation goal} in the form of the ideal
subsystem of the basic variant by defining
\begin{itemize}
\item the functional requirements,
\item the design and manufacturing conditions,
\item the technological influences as well
\item the natural law restrictions and their fulfillment
\end{itemize}
in relation to the elementary components of the subsystem:
\begin{itemize}
\item \emph{Operand} (object that is acted on),
\item \emph{Operation} (way of acting)
\item \emph{Operator} (means to act),
\item \emph{Counter-operation} (type and way of counter-action in the sense of
  creating a function-realizing equilibrium) and
\item \emph{Couter-operator} (means to stabilize the function).
\end{itemize}
This results in determining the \emph{requirements for a technical-scientific
  solution} in terms of new functional requirements, other design and
manufacturing conditions, changed technological influences or other types of
natural law restrictions in the functional realization that are to be
considered for which neither suitable means-effect relationships nor
function-fulfilling technical arrangements are known in the system-related
state of the art.

Working with the ABER(2) matrix is based on a \emph{function-related
  structural analysis} of the system considered as a whole, to delimit the
critical functional area and define the interface conditions for the
\emph{ideal subsystem} in both structural and functional direction. This makes
the interrelations transparent and manipulable, which cause the
\emph{undesired effect} in the functional behavior of the ideal subsystem.

The ABER(2) matrix has 20 fields and at least as many functional or structural
parameters as elements for the ideal subsystem. When it is created, the
\emph{need for innovation} and the \emph{technical-technological innovation
  goal} are questioned. At the same time the \emph{inventive innovation idea}
is taking shape in the new functional principle of the ideal subsystem. The
considerations extend beyond the ideal subsystem also to its
interrelationships with the technical system as a whole. This is fixed in the
definition of the design conditions and the technological influences in the
ABER(2) matrix.

Working with the ABER(2) matrix does not only pursue to find the
\emph{contradiction-free inventive solution idea} for the ideal subsystem.
The result also may be the formulation of a \emph{technical-technological
  contradiction} (TTC) that prevents such a solution based on known principles
of action. It is that the contradicting structural and functional parameters
in the critical operational area of the ideal subsystem have been found and
based on this, a solution strategy can be generated that is oriented on a new
operating principle.

\subsection{The matrix of the operational field (ABER(3) matrix)}

It is based on a scientific-mathematical model and a working hypothesis based
on that model concerning the processes in the critical operational area of the
ideal subsystem. It serves to systematically record
\begin{itemize}
\item Requirements (A),
\item Conditions (B),
\item Findings (E),
\item Restrictions (R)
\end{itemize}
in relation to
\begin{itemize}
\item technically usable \emph{effects},
\item technologically to be controlled \emph{side effects} and
  \emph{accompanying effects},
\item constructively required \emph{counter-effects} and \emph{guiding
  effects} in the functional structure of the ideal subsystem
\end{itemize}
as well as the elaboration of the causal relationships between those
operational field parameters.

The required \emph{application-oriented scientific research effort} results
from previously unrealized effectiveness and efficiency requirements,
completely new usage conditions, not yet available scientific knowledge or
ethical and ecological restrictions.

The operational field matrix has 12 fields and at least as many operational
field parameters as elements to transform the problem and the \emph{solution
  goal} from the technical to the natural science level of observation and
representation. The problem remains unchanged the TTC. The solution goal now
is the new function-fulfilling according to the solution strategy
\emph{operating principle}. The solution goal is therefore no longer
immediately oriented towards the invention, but primarily towards the
acquisition of scientific knowledge, which opens up new space for inventive
thinking.

However, the operational field matrix also serves to critically question
\emph{inventive innovation ideas} and needs for technical-scientific solutions
from natural science point of view. This can lead to a new view of the problem
and a \emph{new inventive innovation idea}, which no longer implies an
undesirable effect and therefore is free of contradictions in the
technical-technological meaning.

For the critical, solution-oriented questioning of the inventive innovation
idea from this scientific point of view, the substance-field analysis (Wepol
analysis) is suitable (ALTSCHULLER 1979).  Within ProHEAL it was further
developed from a more phenomenological to an analytical tool to create
effect-related solution modules.

For this purpose, a \emph{system of physical effects} in different forms was
developed, most recently as a knowledge store on an electronic data carrier,
that can be used to search for suitable solution variants or solution modules
starting with a problem- and contradiction-oriented menu. Also Manfred
v. Ardenne's monograph "Effects of Physics and their Application"
published in 1988 was used in the inventor schools.  

\subsection{ProHEAL -- tool for the engineer and inventor}

The ABER matrix gives the open-minded observer many suggestions to have. So it
is made by increasing parameters in one or more matrix fields Strived for and
perhaps found inventive solutions. The open-minded one The observer had to do
with the so-called basic variant, which he considered reason was. But soon the
open-minded viewer can also look at obstacles and had reached the limits that
prevented him from advancing. Then it got difficult.

The attempts to improve (increase) common parameter values of technical
objects lead mostly approaches the situation in which the engineer has to
pause to say: "Yes but what shouldn't happen?" That is a question of
undesirable consequences. One objective of inventions is based on the most
accurate knowledge of all possible "Yes but ...". It is to be striven to find
for every "Yes but ..."  not a "Either ... or", but an "As well ... as". The
observing engineer feels now caught in a vicious circle. The "yes but ..."
signals that there is a dialectical contradiction: Two tendencies are opposed
to each other -- battle of opposites -- each tendency inevitably produces the
other, often several others as well. The ABER matrix helps to perceive and
understand what is happening. This is the starting point to determine the
entries in the ABER matrix. What the designer refers to from the very
beginning are the societal needs, the manufacturer needs, the user needs. A
concrete objective must be derived from this.  If no solution has been found
for the contradiction after the first attempts, further questions about the
basic variant must be asked and answered.

ProHEAL shows the way to a solution. The \emph{mandatory evaluation figure}
(either completely specified or completed by the responsible engineer) is
characterized in [P:(1.3)] as an expression of the identified system of ABER.
The technical-economic parameters are to be derived from it. That some or many
of them should be improving a lot (but none should deteriorate) is initially
just a request or wish. They are rooted in the network of
technical-technological or technical-scientific properties of the \emph{basic
  variant} and are interlinked by these properties. From these connections in
the ABER system inevitably result the "Yes, but", which are to be inventively
transformed -- through changes in the basic variant -- in "as well as". In
this inevitability lies the difficulty of a purposeful, effective invention.

The "yes but" become all the more delicate and acute, 
\begin{itemize}
\item the more the effectiveness $E$ is to be increased, because then limits
  of improvements based only on optimization of the technical object are
  reached or have to be crossed;
\item the more consistently the evaluation figure -- the ABER system -- is
  respected in its complexity so that improvements regarding some parameters
  no longer can be achieved by "cheap" approaches at the expense of
  deterioration in other parameters that one improperly ignored.
\end{itemize}

In the technical-economic development, such contradictions can be ignored for
a while. It is possible
\begin{itemize}
\item if they are not serious in terms of their importance, because the
  parameters standing in contradictory relation are not serious in the overall
  system; in other words, if the complex target (the system of ABER) allows
  for a distinction between basic and less important parameters.
\item if the contradiction is in the initial stage of development, where
  improvements of the parameters affect each other in a not yet excluding way.
  In this initial stage, a compromise solution can be found by optimization
  based on the state of the art which leads to a sufficient increase of value
  for each of the two parameters.
\end{itemize}

The method of invention must now evoke awareness,
\begin{itemize}
\item how the technical-economic contradiction to be resolved by analyzing the
  technical object -- more precisely: the basic variant -- is determined.
  Which relationships in the technical object are "responsible" for the
  conflict situation in the system of ABER?  These primary, often complex
  relationships are usually not easily disentangled. More about the analysis
  up to the determination of the technical-economic contradiction see
  [P:sect. 1--4].
\item how to find the point or secondary context hidden deeper in the
  technical object, in its structure where parameters can be changed in such a
  way that the primary context, on which the technical-economic contradiction
  is based, can be rendered harmless. This leads to the question about the
  technical-technological and possibly technical-scientific contradiction that
  must be uncovered when the technical-economic contradiction should be
  brought to disappearance. See [P:sect. 5--9].
\end{itemize}

\subsection{The basic structure of ProHEAL}
  
At the beginning the term a \emph{guiding variable} GV has to be defined. The
guiding variable is a system-specific parameter of central importance, the
variation of which influences the development of the performance capability
and/or the effectiveness of a technical system may be increased. In different
words: Such increases are -- mostly based on many years of experience --
expected as a result of a variation in the guiding variable. The guiding
variable can be, for example: the unit of performance of a large transformer
or a large generator, the number of integrated curcuits of a microprocessor,
the load level of a transport system, the starting torque of an electric motor
(based on its nominal torque), the clock speed of a machine tool, the
equilibrium concentration of a chemical process, the specific number of
separation stages of an extraction process. More on this in [P:(2.3.4)].

A \emph{technical-economic contradiction} (TEC) arises if in the variation of
a technical parameter that is decisive for the achievement of this higher
economic effect -- the guiding variable GV -- at least two important
technical-economic parameters $E_1$ and $E_2$ of the technical object behave
mutually opposite.

For example, consider the development of a container. As guiding variable GV
we choose the thickness of the wall of the container in relation to its edge
length. Reducing the thickness of the wall two essential technical-economic
action parameters of the container are favourably influenced: the specific use
of material, related to the container volume, and the payload ratio, i.e. the
ratio of loadable mass to self mass of the container.  We can therefore
combine both parameters to the technical-economic action parameter $E_1$. The
opposing technical-economic action parameter $E_2$ is the specific load
capacity of the container, i.e. the loadable load in relation to the load
level of the transport system in question. The latter becomes to low when the
container wall thickness falls below a certain limit thus outweighing by far
the economic advantages of material savings and the low weight of the
container. In addition, slimming the wall thickness makes the container more
susceptible to corrosion and mechanical damage. A further reduction in the
specific wall thickness under a characteristic statistical limit value thus
gives rise to a critical technical-economic contradiction.

A \emph{technical-technological contradiction} (TTC) is given when during
variation of the technical parameter that primarily determines the expression
of the functional technical effect -- the \emph{structural value} SV -- at
least two decisive technical-technological action parameters $T_1$ and $T_2$
of the technical object become opposite to each other in their behavior.

In the example of the container with regard to overcoming the
technical-economic contradiction we initially consider the container function
as structural value SV.  A functional technical effect is decisively
influenced by it, namely the distribution of the load forces and the type of
their effect (in the form of compression, tension, shear and/or bending
stresses) in relation to the local strength distribution in the wall of the
container. One of the essential effectiveness parameter of the container is
based on this technical effect, its specific, i.e. related to its volume load
capacity. We denote this with $T_1$. By suitably shaping the container wall
the technical effect can be brought into effect in such a way that material
savings and higher payload ratio no longer are in a technical-economic
contradiction with the specific load capacity of the container.

If this does not succeed, we are faced with a TTC.  It arises due to the fact
that with the container shape as structural value SV not only the load-bearing
capacity is significantly determined, but also the specific usable volume,
i.e.  the portion of the volume of the cargo container that can be filled and
its accessibility, i.e. the way of loading and unloading the container. We can
combine these properties to technical-technological effectiveness parameter
$T_2$. The technical-technological contradiction can consist in the fact that
a container shape is required to achieve a higher load-bearing capacity, that
leads to a lower specific usable volume and/or less favorable way of loading
or unloadability the container. This is the case, for example, when the
container bottom should be given a curved shape to increase the load-bearing
capacity and thereby the usable container height (when transporting bulky
goods) or its unloadability (when transporting port of bulk goods) is
inadmissibly impaired.

A \emph{technical-scientific contradiction} (TSC) exists if, in the variation
of a technical-scientific parameter that is decisive for the occurrence of a
scientific effect, i.e. the \emph{impact value} IV, at least two
technical-scientific effectiveness parameters $S_1$ and $S_2$ of the technical
object take values in opposite directions instead of the same direction as
required.

Concerning the solution of the TTC in the case of the container we can, at
least locally, consider the elasticity of the material of the container wall
as effective variable IV. The impact of the natural law bound to this quantity
is the elastic deformation of the container wall. This impact causes two
essential technical-scientific effectiveness parameters to go in opposite
directions:
\begin{itemize}
\item on the one hand, the adaptability of the shape of the container to the
  shape of the cargo and the adaptability of its strength distribution to
  the distribution of the specific load ($S_1$), but also
\item on the other hand, the stability pf the form of the container ($S_2$).
\end{itemize}

Both effectiveness parameters will not contradict each other if the elasticity
is appropriately distributed in the container wall or if the shape durability
does not play a decisive role or is even undesirable. The latter is for
example the case with the waste container. That's why the garbage bag is made
of extremely thin, highly elastic shear and biodegradable plastic film.
Choosing the impact value "elasticity" not only the TSC is solved, but also
the TTC.  At the same time the susceptibility to corrosion due to the
extremely thin container wall is here even desirable because it results in
rapid biodegradation of the container.

A TSC can be given, for example, if the negative influence of the selected
impact value "elasticity" on the stability of the shape consists in that under
dynamic loading vibrations of the container wall occur, which lead to
resonance phenomena and as a result to an impairment of the load and/or to
premature destruction of the container wall.

\subsection{Surprisingly simple solutions (SSS)}

The SSS are solutions with a particularly favorable ratio of effort and
benefit.  In [P:(6.4),(9.3)] we pay particular attention to solutions whose
verbal description contains word like "by itself", "self-movement",
"self-fixation" etc.  Already [P:(2.14)] contains a question aimed at such
solutions: "Which secondary functions in the system are suitable, to make
other side effects usable or to avoid harmful side effects or to transform
them into useful ones?" Very often such a suitability is given. Then a
solution or partial solution of the type "by itself" can already be formulated
during the very beginng of the system analysis, in this case a
self-compensation.  Experience show that such simple and ideal solutions are
mostly out of the field of vision.  Therefore inventors rarely search for
them.

Such solutions are characterized by the fact that their material realization
can be reached predominantly with already existing functional units and energy
potentials, with little equipment effort and/or little operating energy. In
that sense they are simple, elegant, ideal. The technical world is full of
such solutions from ancient times on, which unfortunately we are carelessly
passing because we are using them from very childhood on. A typical example is
the ship's anchor, an extremely simple device with pointed shovels that "by
itself" all the more dig deeper into the ocean floor the stronger the wind or
the waves attack the ship.  The fishhook behaves analogously in the fish's
mouth.

A similar example is provided by Duncker's pendulum, which addresses the
problem of accuracy of a pendulum clock under temperature changes. But even in
physics classes in schools such SSS that can be found en masse throughout the
centuries of history of technology found are totally ignored. As a curious
child I was surprised that the simple toilet cistern regulates the water
supply automatically.  So I asked my father, and because he was a craftsman,
he explained it to me. Even G.S. Altshuller does not pay enough attention in
his books to such surprisingly simple solutions. If you have never seen the
opportunities, it becomes difficult.

\subsection{The general heuristic path model of ProHEAL}

\subsubsection*{W.1. The structure of the path model}

The path model is shown in the figure as a heuristic scheme. It shows how from
the technical-economic issues based on the necessary effectiveness and
functional properties of technical objects, their structural and functional
properties are derived and finally, it is abstracted to functional basics of
technical-scientific effects and how to advance -- progressing from
abstraction level to abstraction level -- on the search for solution ideas
further and further away from the own specialist domain to distant analogy
areas.  Already here inventions with a high economic benefit can arise.

\subsubsection*{W.2. The social need and the ABER}

From the more or less vaguely formulated technical-economic problem situation
as starting point -- according to the heuristic path model -- are the social
needs and the associated ABER to be determined. It is indispensable to detect
the causes for the emergence of the social need and the ABER derived from it
contrasting it with the current available state of technology and its past
development.  In this course it is always necessary also to check whether the
given task is oriented to overcome the causes or only to eliminate undesirable
economic, social, technical or ecological effects.  During such an analysis
the main technical-scientific problem to be solved can be delimited and a
\emph{reference variant} of the technical system can be determined that most
closely matches the ABER.

Now in order to identify and weight the defects and shortcomings of the
reference variant, to find the causes and to create an independent, "tailored"
definition and solution of the specific problem, within a conceptual process
of product planning first an \emph{evaluation figure} is determined. Along that
target, from the ideal state of the art a representative \emph{basic variant}
of the technical system is created, identifying suitable technical means
through patent research and analysis of world-class solutions and combining
them to form the overall system. As part of a system analysis the basic
variant is compared to the reference variant to find weak points and defects
that lead to TEC in their behavior and are inventively to fix. From the basic
variant a solution is to be developed that has clear technological and
economic advantages compared to the reference variant.

\subsubsection*{W.3. The Evaluation Figure and the State of the Art}

The ABER are initially available in a verbal-descriptive form and express
social, economic and technological issues that affect a certain social
situation of need and interests (see [P:sect. 1]). From this an evaluation
figure is to be derived, which essentially expresses the functional properties
of the technical system to be created and in what way its production and
application does meet the social need in the best way. This is done assigning
to the components of the evaluation figure the relevant ABER parts. This
defines and evaluates concrete characteristics of suitability and
effectiveness. On the one hand, these include the respective social, economic
and/or technological specifics of the social need and on the other hand the
objective specificity of the technical object or object area (see [P:(1.3)]).

These suitability and effectiveness characteristics must first be described
qualitatively, before parameter values can be specified. A premature and
uncritical commitment to functional or economic parameters that are familiar
or mentioned in the task or even limiting oneself to them must be avoided.

In order to be able to define these parameters correctly, it is necessary to
derive from the state of technology the most suitable
\emph{technical-technological principle} (TTP) for the technical object to be
developed. That is a characteristic principle of manufacturing and/or applying
technical objects in a specific technology domain. With this principle, a
class of methods and means is delimited in the state of the art which
represents the basis for further problem processing. Thus the choice of the
TTP has decisive importance for the further solution. It should be done in
such a way that such a TTP is preferred that fits the purpose of the technical
object to be created (evaluation component $Z_1$) in the best way and that
does not conflict with the ABER or -- in comparison to other principles --
violates as less as possible A and E from the ABER system. For this, the
following should first be considered all procedures and means
\begin{itemize}
\item that are \emph{available} on the material state of the art,
\item that \emph{appear feasible} based on the ideal state-of-the-art, and
  finally
\item that \emph{seem conceivable} on the given state of technological
  development and that \emph{seem imaginable} on the given state of sciences.
\end{itemize}
If a TTP is prescribed with the task, it has to be checked if it is feasible
for the evaluation figure and should be compared with other known principles.
If necessary, this must be discussed with the client.

On the basis of the TTP, basic variants of the technical system are designed
using the \emph{methods and means available from the state of the art}.  This
is done transforming the evaulation figure determined by the ABER in two
stages:

In the \emph{first transformation stage}, the types of technical objects are
determined, which are \emph{necessary} according to the TTP to ensure the
suitability of the technical system with respect to the evaulation figure.  To
every object type now such utility properties are attributed, which on the one
hand are typical for the respective object type and on the other hand
correspond to special suitability characteristics of the evaulation figure. In
doing so, it is appropriate first to determine the necessary contributions
specific for that object type to the expediency of the system (component $Z_1$
of the evaulation figure). Then those for the respective object type
characteristic functional properties are defined that guarantee the
suitability of the technical system with regard to its controllability and its
usability. For a sufficient suitability of the technical system -- especially
with regard to its controllability -- it can be necessary to take into account
additional object types, that match such specific suitability characteristics
with their functional or operational properties.

In this way, the evaulation figure is transformed from a system of socially
determined \emph{suitability characteristics} into a system of object-related
\emph{functional properties}. This evaulation figure is the basis for a systematic
patent research and world status analysis for the pre-selection of suitable
technical objects, which in their combination according to the TTP are
sufficiently suitable to form a technical system that meets the ABER
conditions.

In a \emph{second transformation stage}, the \emph{main function} of the
technical system according to the TTP is defined. It can be assumed that the
main function activates the functional properties of the individual objects
and links them in the process of their use in such a way that the suitability
characteristics of the technical system according to the ABER are produced.
This main function breaks down, related to the usage process, in its
\emph{necessary and sufficient subfunctions}. Here, a hierarchy level of the
technical system is selected that on the one hand is as high as possible, on
the other hand takes into account the extraction of object types from the
evaulation figure that was already completed.

To the individual sub-functions such objects are assigned, that are activated
by the respective sub-function in the sense of the main function of the
technical system. For each sub-function, those functional properties are
defined that are caused by these technical objects, which means that they
define \emph{specific technical means}. The sub-functions through which those
object-related functional properties are activated that produce the
suitability of the technical system concerning manageability and usability are
established in the same way, however define \emph{necessary supporting
  functions}.

In this way, the evaulation figure is transformed from a \emph{system of
  object-related functional properties} into a \emph{system of process-related
  functional properties of technical means}. This evaulation figuer is used for
the appropriate selection of technical means from the set of technical objects
take into account and their functional coupling constituting the basic variant
of the technical system. Additionally, the evaulation figure forms in this
transformation stage together with the non-transformed component $Z_2$
(economic efficiency) the basis for the definition of the main
technical-economic performance data of the technical system and for their
quantitative determination in terms of a nominal figure.

The \emph{system analysis} is based on this nominal figure. It is aimed at
determining the effectiveness properties of the technical system in their
combination, especially to uncover contradictory tendencies in their
developmental behavior and to reveal the relevant technical causes in the
context of a developmental weakness analysis. The function-related evaulation
figure can already provide an initial indication of the critical functional
area (critical system area). This is usually the area where the greatest
number of sub-functions meet in a technical object.

\subsubsection*{W.4. The basic variant}

The technical means selected from the available state of the art according to
the evaulation figure are divided into \emph{sub-systems in the form of separable
  structural units} based on their function.  Each structural unit embodies
one of the process-related sub-functions as part of the main function or a
necessary auxiliary function that is responsible for control, protection
and/or environmental compatibility of the system. With the functional
combination of the technical means to sub-systems and the sub-systems to the
overall system of the basic variant the ABER -- the functional requirements
(\textbf{A}nforderungen) and structural conditions (\textbf{B}edingungen) as
well as the external influences (\textbf{E}infl\"usse) and restrictions
(\textbf{R}estriktionen) -- it has to be taken into account how the individual
technical means or subsystems exert influence on each other when combined to
form the basic variant. To do this, for their coupling (by means of
morphological scheme) a \emph{ranking} has to be defined according to the
technical-technological importance of the subsystems in such a way that a
subsystem or technical means of higher rank defines the ABER for the
subsystems or technical means on the respective lower levels of hierarchy.

\subsubsection*{W.5. The decisive defect and the core variant}

The basic variant developed according to the state of the art or the technical
sciences still have decisive deficiencies. These deficiencies can be of
technical-economic nature, arising from the fact that the utility and economic
properties could not be brought into agreement with the evaulation figure, i.e.
requirements and/or restrictions had to be violated. The defects can also of
"heuristic" nature, e.g. if means are neither available nor feasible, but at
most conceivable or even only imaginable.

A \emph{technical-economic deficiency} is present if the technical means
required according to the evaulation figure are principally available or known,
but at least in a crucial functional property the required performance and/or
effectiveness parameters are not achievable or only at the expense of other
evaluation parameters.

A \emph{heuristic deficiency} is present if for at least one of the functional
properties required by the ABER no technical means are known which would be
suitable according to their functional properties to produce the required
means-effect relationships. To become aware of a heuristic deficiency requires
inventive instinct and courage to work since conventional and proven
technology has to be questioned.

For further problem processing, that basic variant is selected which has the
smallest deficiencies. An inventive approach is characterized by the property
that it does not allow any serious technical-economic deficiencies, but
deliberately accepts serious heuristic deficiencies when they challenge for
inventive solutions.  If there is a serious heuristic deficiency the subsystem
or system area where the deficiency appears is declared as the decisive
subsystem or the \emph{core variant} of the technical system. For the
problematic core of the basic variant in this subsystem or system area new,
conceivable solutions are generated by new modifications or previously unusual
combinations of known technical objects.  From these core variants, the one is
chosen that does best fit into the overall context of the technical system of
the basic variant. This may already be an inventive solution and is then the
result of a heuristic approach, which can be described as \emph{projecting
  invention}.

If the basic variant consists of an inventive core variant with only low
technological scope and in the remainder of verified and tested system
components according to the available state of the art, and does not show
significant deficiencies in relation to the evaulation figure, it can be
optimized, transferred into an overall operational project and tested in a
pilot series or trial production.

However, if there are still considerable deviations between the utility value
and the effectiveness of the basic variant on the one hand and the evaulation
figure on the other, and in particular the functional characteristics of the
core variant in the overall context of the technical system are still in
question, then the further procedure is aimed at identifying the causes of
these deficiencies more precisely and to investigate and fix them. For this
purpose, first of all, a \emph{technical-economic objective} is derived from
the evaulation figure as more precise specification, which is aimed at the
increase of those performance and/or economic parameters (main performance
data), the fulfillment of which is still in question.

\subsubsection*{W.6. The structurally prepared basic variant}

The cause of the deficiencies is initially searched for in the structure of
the technical system. For this purpose, the basic variant is structured
according to its structure abstracting from functional properties of
individual objects or groups of objects to structural properties of the
technical system. This is done in such a way that the technical objects
combined and functionally linked in the basic variant are considered with
regard to their necessary structural properties (mainly contained in the
evaulation components \emph{controllability} and \emph{usability}) and are
coordinated in such a way that they can be spatially and/or temporally
combined to form the structural units and the overall system of the basic
variant.

This creates the system-specific structural properties of the technical means.
Here, above all, the structural properties in the system area of the core
variant are emphasized that primarily influence the specific performance
and/or economic parameters of the technical-economic objective.  A variation
of the structural properties of the technical system in the sense of the
technical-economic objectives often results in a deterioration in specific
functional properties that already highlightes a technical-economic
contradiction.

In many cases this is a conflict between the requirements of
manufacturability, mountability and/or maintainability (or the continuous
process management, the monitorability and controllability of procedures) and
the requirements of functionality, insensitivity to external disturbances and
internal functional security.

Here the inventive processing is initially aimed to find the critical
structural unit or functional weak point that primarily prohibits an optimal
design and dimensioning of the basic variant. By a clever transformation or
redesign of one or more objects within this critical system area the
functional performance can be increased without changing the function itself.
If this succeeds, then an inventive solution of the contradiction between
structural and functional properties of the basic variant has been found in
the sense of the technical-economic objective.  Such an approach is called
\emph{constructive invention}. The inventive solution has initially to be
transferred into a functional model and to be test for its functionality.

If it turns out that the technical-economic objective cannot be met, if
functions are not changed, the basic variant has to be prepared concerning its
functional fulfillment and a corresponding system analysis is required.

\subsubsection*{W.7. The preparation of the basic variant cncerning its
  functional fulfillment}

In the functional preparation of the basic variant, the structural properties
of technical objects is abstracted to their functional properties. The aim is
to \emph{identify the essential functional relationships} of the basic variant
as technical system with its environment, and which internal functional
relationships (means-effect relationships) between its components are decisive
for that.

First of all, on has to determine the overall function of the basic variant
and its known or foreseeable side effects as well as the interface conditions
to its technical-technological environment. This is done by a \emph{black box
  analysis}.  The interface conditions (boundary conditions) of the black box
define the input variables of the technical system from the specified output
variables of a preceding system within a higher-level process, and its output
variables from the specified input variables of a following system at the same
higher level. Depending on the type of input and output variables from this
results the transfer or support function mainly to be implemented by the basic
variant as a technical system. This is therefore defined as the \emph{main
  function}.

However, this does \emph{not} happen process-related -- as with the
transformation of the evaulation figure -- process-related, but object-related.
That is, the function is not considered as necessary, process-related
activation of certain functional properties of technical objects, but as
\emph{structurally constrained effect of certain functional properties of
  technical means}. After that the necessary technical prerequisites for the
creation and maintenance of the main function, i.e. for the functioning of the
technical system, are determined. From this the required auxiliary functions
are defined, concerning the types \emph{interference suppression function} and
\emph{protection function}.

Defining the \emph{interference suppression function}, one can first refer to
the functional properties that are contained in the evaulation component $Z_3$
(controllability).  In addition, it is necessary to determine what side
effects are triggered from the specific objects of the basic variant during
operation or use. These side effects must be recorded as completely as
possible.  There are harmful as well as useful or usable side effects.
Necessary measures to suppress the harmful side effects caused by the overall
function to acceptable levels lead to the definition of the interference
suppression function.

Necessary measures to suppress harmful effects caused on the main function and
the interference suppression function of the technical system by the
environment, lead to the definition of the \emph{protection function} of the
system. Defining the protection function we can initially be conducted by the
usage properties and usage conditions, which are combined in the evaulation
component $Z_4$ (usability). Concerning the harmful effects from the
environment not only technical, technological and natural ones are to be
considered, but possibly also social (qualification, discipline) and
organizational (supply of means of transport, material, energy and/or
information) are to be taken into account.

An important for the inventor functional class are the \emph{auxiliary
  functions}. This are functions, which are generated "for free" by the
objects of the basic variant in addition to their main functional
destination. They have to be investigated whether and to what extent they can
used to support or even to replace functions of one or more other objects of
the basic variant. This can lead to a \emph{fusion of functions}, the effect
of which goes beyond the sum of the individual effects of the objects in
question. This is an important indicator for an inventive achievement.

Auxillary functions that cannot be used are considered as \emph{unnecessary
  functions}.  They should be eliminated as completely as possible by a more
suitable choice or design of the objects of the basic variant, at least when
they provoke a disruption of the functional flow of value or cause unnecessary
costs.

For a complete recording and advantageous design of all interrelationships
between the system and its environment, it is necessary to delimit an
\emph{operational field} for the inventor in relation to the technical system.
It includes all objects -- technical and natural -- as well as all factors --
social, organizational and technological -- with their harmful and beneficial
effects that the technical system with its function has to take into account
or that can be effectively included in its function. It depends on the correct
delimitation of the operational field and the technical system whether the
protection function is correctly determined and whether objectively available
options to simplify functions or to increase function value are recognized
and used for improvements. (See [P:sect. 3]).

Depending on the situation, this can be done in such a way that suitable
objects from the operational field are used to support or simplify functions
including them in the basic variant by structural as well as functional
integration assigning them an adaptation function.  Conversely, it can also
prove to be advantageous, and in some cases even necessary to relocate certain
objects from the basic variant to the outer part of the operational field. A
close functional and structural link across the system boundaries in the sense
of a mediation function can generate a positive influencing factor in the
outer operating field or reinforce an existing one. Possibly one gets thereby
at the same time a simplification of the function of the basic variant or an
increase in their functional value.

With the black box analysis of the basic variant, its function-related
processing is essentially completed. The knowledge gained about the functional
characteristics of the basic variant, their mutual dependencies and the
possibilities to optimally coordinate them with each other and with the system
environment are now used to attempt to resolve the contradiction between
structural and functional properties that occurred during the
structure-related preparation of the basic variant.  Here you can in an
inventive way, through an original distribution of the required functions to
the individual objects of the basic variant and the skillful use of so far
neglected structural and functional properties create the prerequisites for an
optimal overall solution.

\subsubsection*{W.8. The optimization of the basic variant compared to the
  reference variant and the TEC}

In order to be able to optimize the basic variant, a technical performance
parameters has to be determined as a \emph{guiding variable}, the variation of
which affects to a decisive extent on the one hand the effectivity parameters
of the technical-economic objective and, on the other hand, the necessary
structural and functional properties of the technical system. Choosing the
guiding variable, we decide the direction of the further development of the
technical system and the development trend of its utility and economic
efficiency properties.

The guiding variable must therefore be in agreement with the evaulation figure,
even if it turns out that its variation -- although professonally thought
ahead -- leads to changes in the structural and functional properties of the
technical system, which (at least partially) are still in contradiction to the
technical-economic objectives. As orientation for the correct determination of
the guiding variable can serve the reference variant which was derived from
the worldwide material state of the art.

As guiding variable serves the technical performance parameter in which the
reference variant still deviates at most from the evaulation figure.  That is, the
requested performance (in the evaulation component $Z_1$) is achieved either not
at all or under the given implementation conditions only with impermissibly
high technical ($Z_3$), technological ($Z_4$) and/or economic effort ($Z_2$).
This way, a follow-up strategy is prohibited from the beginning and a
progressive solution strategy is designed that meets real social needs.

In addition, the reference variant can be included in the black box analysis
as a suggestion for the functional and structural conceptualizsation of the
basic variant. Did we already succeed -- possibly in an inventive way -- to
develop a basic concept that can be optimized then we will find an optimal
overall solution for the basic variant that meets the technical-economic
objective through well-coordinated design and dimensioning of the individual
objects. If it is found, then the functionality and the functional value of
the basic variant has to be tested on a test sample.  For this, it is
sufficient to reconstruct that part of the functional area of the basic
variant, in which the decisive structural and functional changes are located
that were carried out compared to the tried and tested state of the art. As a
rule, it is the core variant and its closer system environment.

If an optimal overall solution has not yet been found or if the design of the
basic variant proves to be not functional, the decisive TEC is be to
determined.  In other words, the decisive technical-economic effectiveness
parameters have to be determined that relate to one another in such a way that
the increase in one parameter leads to an impermissible reduction of the other
parameter, if the guiding variable is varied according to the
technical-economic objective (see [P:sect. 4]).

The further procedure is now no longer possible through \emph{accompanying} or
tactical inventing, but characterized by \emph{forward-looking}, strategic
invention. It is actually inventing the true meaning of the method, the
\emph{invention itself}. This brings us to stage 2 of the organizational
model, at the beginning of which a \emph{renewal pass} and a
\emph{specification sheet} with a clear inventive task hav to be negotiated.
The subject of the invention is now a technical-economic contradiction, the
goal is to overcome it.

\subsubsection*{W.9. The inventive core variant (key variant)}

When overcoming the TEC, it is assumed that its cause is not distributed
around the whole technical system, but essentially focused in a specific
system area -- the area that is critical for the functioning of the technical
system, the \emph{critical functional area}. The inventive goal now is to
\emph{discover} this system area and to produce an inventive solution, that
creates in this critical area new technical conditions, new means-effect
relationships, opening new possibilities for the development of the basic
variant and the corresponding variation of the guiding variable in terms of
the technical-economic objective.

\subsubsection*{W.10. The critical functional area of the basic variant and
  the ABER}

Based on the result of the black box analysis, the findings during the
unsuccessful optimization of the basic variant and possibly from a functional
that was completed with testing negative result, the causes of the
technical-economic contradiction are to be explored. (See [P:sect. 5]).

For this purpose, continuing the black box analysis the basic variant is first
divided into the object-related sub-functions, which are essential for a
working chain of intended and/or to be prevented \emph{changes in state of one
  or more objects} that produces in the end a stable and effective main
function.  For the specification of the necessary functional features one can
exploit the usage properties that are summarized in the evaulation component $Z_1$
(Expediency) in terms of effects of their activation. Thus one can assign
system components contained in the basic variant as objects to individual
sub-functions and assess them with regard to their functional value.

It will always be possible to delimit an area of the technical system in which
one or several sub-functions are present, which in comparison to the
neighboring system areas have a significantly lower functional value. This
system area acts like a bottleneck in the function value flow of the main
function, which this and other sub-functions does not take full effect and
thus decisively limits the overall functionality of the technical system. It
is therefore called \emph{critical functional area} of the technical system.

For the inventor, it is not only the question of the technical-scientific
causes for the emergence of the functional bottleneck, but also the question
of the \emph{technical-constructive or operational-procedural reasons} that
prevent the elimination of these causes in the way to an optimal dimensioning.
These reasons are to be reduced to a \emph{harmful technical effect} (HTE),
which prevents the development of the technical system according to the evaulation
figure.

The answer to this question about the HTE, which is crucial for the
inventional task, can only be derived gradually. For this purpose, the
sub-functions created in the critical functional area are divided according to
their procedural principle into \emph{elementary functions} and corresponding,
objective \emph{functional units}. The individual functional units are
resolved into their operational components -- \emph{operation},
\emph{operand}, \emph{operator} and \emph{counter-operator} -- and according
to the functional principle of the respective functional unit technically
defined as functional parameter. In this way the critical functional area can
be clearly and transparently displayed in a morphological scheme.

The \emph{function value flow} is now examined from elementary function to
elementary function.  In doing so, following a suitably selected guide
variable (structural variable), an optimization of the functional units is
attempted to implement varying the functional determinants while keeping the
principle of operation.

Depending on the result of these optimization attempts, the root of the
harmful technical effect may be limited on certain functional units and their
structural and functional properties. This means that the critical functional
area is increasingly narrowed and more precisely defined.  At the same time,
the technical and natural law requirements, constraints, influences and
restrictions (the ABER) are determined in their specific for the technical
system form of interrelation that determines the technical-scientific core of
the problem.

That is, the further set of ABER defined at the beginning of [P:(2.4)] become
a system linking the sub-objects of the basic variant. This ABER on
technical-scientific level is the analogue of the ABER on the
technical-economic level. Note that at this level, we work as \textbf{E} with
\emph{influences} rather than \emph{expectations}.

These new ABER prevent to overcome the TEC. These ABER are to be changed in a
following stage of the inventive process in the sense of a technical ideal
(IDEAL) in such a way that the HTE disappears. In that process it is initially
not allowed to vary technical product requirements or restrictions by natural
laws.

\subsubsection*{W.11. The harmful technical effect and the IDEAL}

The (technical) IDEAL primarily refers to the behavior of the technical system
in its critical functional area. The rest of the technical system is initially
essentially set immutable. With the IDEAL, such ideal constructive conditions
and/or such ideal procedural arrangements are thought ahead the recognized
optimization limits, that all undesirable technical-scientific influencing
factors disappear or are at least reduced so far in their effect that a
decisive increase of the functional value in the critical functional area
appears. The functional principle or the operational principle are initially
not changed. (See [P:sect. 6]).

In contrast to the technical-economic ABER the technical-scientific ABER are
not directly derived from the social supersystem and the technological
environment of the technical system, but rather from its constructive or
procedural structure and the functional principle implemented there.  With
these ABER next to \emph{requirements}, \emph{conditions} and
\emph{restrictions} also \emph{influences} (in the sense of side effects) of
technical-constructive and technical-scientific type are recorded, which the
components of the technical system exert on each other, or which affect them
from the system environment.

Opposing new conditions and pushing back the influencing factors has to
respect technical requirements for structural and functional basic properties
of the basic variant and must not violate natural law restrictions, which are
set by the overall function of the basic variant in a principal way. Otherwise
the IDEAL will cause another harmful technical effect in another system area,
which usually also leads to a specific TEC.

Should it turn out that the elimination of a harmful technical effect is only
possible with the emergence of another one, so in every case it is to "scout"
whether there is one of these harmful consequential effects, against which a
supplementary IDEAL may be thought that meets all the requirements and
restrictions of the technical system.  As a rule, however, this requires a
detailed examination of the structural and functional interrelationships of
the technical system -- at least in the environment of the critical functional
area.

In order to avoid an odyssey through the technical system, this exploratory
procedure therefore only makes sense as long as it does not go too far beyond
the originally delimited system.  If such an IDEAL approach is found that can
be developed futher, a stable \emph{technical effect} (TE) and a mediating
\emph{functional principle} (FP) has to be searched which correspond to the
new conditions and influencing factors in the system of technical-scientific
ABER.  This is the starting point to develop the \emph{sub-function
  principles} and the \emph{technical principle} of the inventive solution for
the core variant (key variant) that can be tested in a test sample.

However, if no viable IDEAL approach has been found, so the further process
starts with the approach that is in greatest compliance with the requirements
and restrictions in the system of scientific ABER. The findings from the
exploration of the technical system are now summarized as the
technical-technological contradiction (TTC) summarized. (See [P:sect. 7]).

\subsubsection*{W.12. The technical-technological contradiction (TTC) and the
  new functionals principle for the key variant}

The TTC defines the specific technical issue that the removal of the initially
found harmful technical effect \emph{necessarily} yields another, just as
difficult to remove harmful effect. To resolve this contradiction now the
\emph{general problem-solving principles} are brought into consideration. (See
     [P:sect. 9]).

If a solution has been found to overcome the TTC, it is first to be checked
its \emph{technical effect} at IDEAL for its fundamental usability.  Then the
technical-scientific ABER are to be modify accordingly, and it is important to
ensure that this does not violate technical requirements or scientific
restrictions.  Finally, it must be checked whether the \emph{harmful technical
  effect} has actually been eliminated and no new TTC has been created.  Only
then it time to define -- with reference to the IDEAL -- a more detailed
specification of the \emph{new technical effect} and the system-compatible
expression of the \emph{new functional principle} for the key variant.

If a usable approach to solving the TTC is not found, the technical-scientific
fact is to be determined, which decisively opposes this solution.  For this
purpose, the system analysis is directed to the critical point of action of
the key variant, where those technical-scientific restrictions start that are
decisively involved in the creation of the TTC. Theis from the critical point
of action emanating technical-scientific restriction is called a \emph{harmful
  scientific effect} (HSE). It essentially consists in the fact that the
functional principle of a functional or utility value-determining technical
partial effect which should be evoked at the critical point of action
prohibits certain functional and/or structural changes in the vicinity of this
point of action. (See also [P:sect. 7]).

\subsubsection*{W.13. The technical-scientific contradiction (TSC) and the new
  operational principle for the key variant}

In a \emph{database of scientific effects and principles}, such approaches are
searched for that produce the necessary technical partial effect at the
critical point of action in at least the same strength but the original
scientific restriction is no more relevant.

Of course, it must always be checked whether only a problematic restriction
has been exchanged against another. This examination can initially be done on
a theoretical basis of a technical-scientific model of the point of action and
its immediate surrounding.  In this process the elementary (functional and
structural) conditions and relationships have to be investigated that are
required to create the necessary conditions for the appearance of the new
technical partial effect at the point of action. It always turns out that at
least one of these conditions must be met without restriction due to the
selected operating principle. That is, it is to be regarded as the \emph{new
  natural law restriction}.

To determine whether or not this new restriction prevents solving the problem
it can be compared with the IDEAL within the system context of the
technical-scientific ABER and examined whether the harmful technical effect is
now eliminated or the TTC can be solved.  If this is the case, starting from
the IDEAL a \emph{specification of the new technical effect} and the
system-compatible \emph{expression of the new functional principle} for the
key variant are to be developed.

However, before the partial function principles and the technical principle
are developed from this, the simplifying assumptions and the neglected
possible secondary effects and subordinate influencing factors of the
\emph{technical-scientific model} have experimentally to be checked on
validity and reliability.  A \emph{laboratory sample} is used for this, i.e. a
replica of the structure of the technical system in the area of the critical
point of action.

If, even after several approaches, a suitable technical-scientific principle
of action for the solution of the TTC is \emph{not} found, the knowledge
gained is expressed as \emph{technical-scientific contradiction} (TSC). This
justifies the problem-specific scientific fact that for the technical system
of the basic variant \emph{there is no operational principle}, that removes a
technical-scientific restriction without causing other equally serious ones.
The reason for this are \emph{restrictive functional and/or structural
  conditions and requirements} of the technical system that do not allow new
operating principles to develop either.

With the help of the \emph{general problem-solving principles} an attempt is
now made to meet to "weaken" these conditions and requirements in such a way
that one of the considered operational principles no longer leads to a TSC.
This means that also the TTC and the TEC are also solvable in principle. (See
[P:sect. 9]). 

\section{PROHeal -- the Algorithm} 

The information in brackets refers to the points in the handout ...

\subsection*{A. Technical-economic Part of the Program}

\textbf{Objective:} Critique of the state of the art from a
technical-economical point of view. Determine the relevant evaulation and
reference variables.

\begin{itemize}[leftmargin=30pt]
\item[(A1)] Specify the societal need (SN) according to operational tasks
  of the enterprise (OTE) including
  \begin{itemize}[leftmargin=20pt,noitemsep]
  \item[(A1a)] Determine the overall SN (1.3), (1.6) 
  \item[(A1b)] Determine the special SN (1.1), (1.2), (1.4)
  \end{itemize}
\item[(A2)] Find the ABER (1.4), (1.6), (1.7)
\item[(A3)] Determine the required use properties (1.4), (1.5) 
\item[(A4)] Define the components $Z_i$ of the evaulation variable $\zeta$
  (1.8), (1.9), (1.10)
\item[(A5)] Choose the technical-technological principle (2.1)
\item[(A6)] Determine the basic variant of the technical system starting from
  the state of the art (2.2), (2.3), (2.4)
\item[(A7)] Formulate the technical-economic objective (2.5), (4.3b)
\item[(A8)] Black box analysis of the technical system (2.6), (2.7), (2.8),
  (2.9), (2.10), (2.11), (2.14), (2.15), (3.4)
\item[(A9)] Delimit the technical-economic field of operation (2.12), (2.13),
  (3.1), (3.2), (3.3)
\item[(A10)] Determine the lead variable $G_F$ (2.5f), (4.1)
\item[(E1)] \textbf{Decide:} Is the technical system appropriately delimited?
  (4.2), (3.4)
  \begin{itemize}[leftmargin=20pt,noitemsep]
  \item \textbf{Yes:} Go to (E2)
  \item \textbf{No:} Back to (A8)
  \end{itemize}
\item[(E2)] \textbf{Decide:} Is an optimization solution possible?  (2.9),
  (2.14), (2.15)
  \begin{itemize}[leftmargin=20pt,noitemsep]
  \item \textbf{Yes:} Work out the optimization solution $\to$ \textbf{STOP}
  \item \textbf{No:} Go to (A11)
  \end{itemize}
\item[(A11)] Find and formulate the technical-economic contradiction that
  determines the problem (4.2), (4.3), (4.4)
\item[(E3)] \textbf{Decide:} Is there a case of "Business blindness"? 
  \begin{itemize}[leftmargin=20pt,noitemsep]
  \item \textbf{Yes:} Back to (E2)
  \item \textbf{No:} Proceed with part B
  \item \textbf{Unknown:} Back to (A5)
  \end{itemize}
\end{itemize}

\subsection*{B. Technical-technological Part of the Program}

\textbf{Objective:} Critique of the state of the art from a
technical-technological point of view. Determine the decisive operation
parameters.

\begin{itemize}[leftmargin=30pt]
\item[(B1)] Find and formulate the undesired effect (2.10), (2.11), (2.15c),
  (2.15e), (5.1), (5.4)
\item[(B2)] Delimit the critical functional area in the structure of the
  technical system (2.8), (2.15c), (2.15d) (3.4), (5.2), (5.3)
\item[(B3)] Draft the ideal vision for the core variant (in the critical
  functional area of the technical system) -- IDEAL -- (6.1)
\item[(B4)] Develop ideas about the necessary technical requirements (ABER)
  for the usefulness of the ideal image (ideal conceptions) (6.2)
\item[(B5)] Conceptual modification of the technical system with regard to
  required functional properties outside the critical functional area
  according to the ideal vision on the ABER (6.3), (6.4)
\item[(E4)] \textbf{Decide:} Will a harmful technical effect reappear?  (6.5)
  \begin{itemize}[leftmargin=20pt,noitemsep]
  \item \textbf{Yes:} Back to (B2)
  \item \textbf{No:} Go to (E5)
  \end{itemize}
\item[(E5)] \textbf{Decide:} Are the ABER sufficiently determined? (6.2a)
  \begin{itemize}[leftmargin=20pt,noitemsep]
  \item \textbf{Yes:} Go to  (B6)
  \item \textbf{No:} Back to (B4)
  \end{itemize}
\item[(B6)] Exgtract the ideal final result (IFR) (6.4)
\item[(E6)] \textbf{Decide:} Is the ideal vision in the ABER technically
  feasible?  (6.2a), (9.5)
  \begin{itemize}[leftmargin=20pt,noitemsep]
  \item \textbf{Yes:} An unexpected approach to a \textbf{surprisingly simple
    solution} (SSS) is found (6.5). Back to (E2). 
  \item \textbf{No:} Go to (B7)
  \end{itemize}
\item[(B7)] Find and formulate the technical contradiction (6.2d), (7)
\item[(E7)] \textbf{Decide:} Is it a prejudice of the professional world?
  (6.2a), (9.5)
  \begin{itemize}[leftmargin=20pt,noitemsep]
  \item \textbf{Yes:} Transition to the elimination of a technical
    contradiction with surprising impact (6.2a), (9.5). Back to (E2).
  \item \textbf{No:} Go to part C
  \item \textbf{Unknown:} Back to (B1)
  \end{itemize}
\end{itemize}

\subsection*{C. Technical-scientific Part of the Program}

\textbf{Objective:} Critique of the state of the art from a
technical-scientific point of view.  Determination of the decisive effective
variables.

\begin{itemize}[leftmargin=30pt]
\item[(C1)] Derive the technical-scientific cause of the harmful technical
  effect from the ideal vision of the ABER (8.1a)
\item[(C2)] Limit the area of the critical point of action in the technical
  system (2.8)
\item[(C3)] Model the critical point of action
\item[(C4)] Formulate a search query to the database of scientific effects to
  realize the ABER according to the ideal vision (ideal scientific effect)
  (8.3)
\item[(E8)] \textbf{Decide:} Is there an appripriate scientific effect?
  \begin{itemize}[leftmargin=20pt,noitemsep]
  \item \textbf{Yes:} Consider it as basis for a new technical approaches.
    Back to (E6).
  \item \textbf{No:} Go to (C5)
  \end{itemize}
\item[(C5)] Formulate the technical-scientific contradiction (8.1b), (10.1)
\item[(E9)] \textbf{Decide:} Is it a matter of blindness in the professional
  world?  (8.2), (10.1)
  \begin{itemize}[leftmargin=20pt,noitemsep]
  \item \textbf{Yes:} Consider these technical approaches from a foreign
    domain.  Back to (E2).
  \item \textbf{No:} Go to (C6)
  \item \textbf{Unknown:} Back to (C1)
  \end{itemize}
\item[(C6)] Find suitable solution strategies in the technical system to
  overcome the technical contradiction (8.2), (9.1), (9.2), (9.4a), (10.2)
\item[(C7)] Formulate the inventional task with the goal of a radical
  renewal of the structure of the technical system (9.4b)
\item[(C8)] Find suitable solution principles to solve the problem of the
  inventional task (9.3), (9.4a)
\item[(C9)] Find fundamentally new approaches to solutions (creation of a new
  generation of the technical system) (10.2) $\to$ \textbf{Back to (A5)}
\end{itemize}

\begin{center}
\begin{tikzpicture}[scale=1.4,transform shape,
    >={Triangle[length=0pt 6,width=0pt 5]},
    rounded corners=2pt,line width=.8pt]
  \node[draw] at (0,15) [rectangle] (A0) {Start};
  \node[draw] at (0,14) [rectangle] (A1) {A1--A5};
  \node[draw] at (0,13) [rectangle] (A6) {A6--A8};
  \node[draw] at (0,12) [rectangle] (A9) {A9--A10};
  \node[draw] at (2,14) [circle] (E1) {E1};
  \node[draw] at (4,14) [circle] (E2) {E2};
  \node[draw] at (6,14) [rectangle] (A11) {A11};
  \node[draw] at (4,12.7) [rectangle] (A4) {\emph{Optimization}};
  \node at (4,11.7) [rectangle] (A5) {STOP};
  \node[draw] at (8,14) [circle] (E3) {E3};
  \node[draw] at (0,10) [rectangle] (B1) {B1};
  \node[draw] at (0,9) [rectangle] (B2) {B2};
  \node[draw] at (0,8) [rectangle] (B3) {B3--B4};
  \node[draw] at (0,7) [rectangle] (B5) {B5};
  \node[draw] at (2,9) [circle] (E4) {E4};
  \node[draw] at (4,9) [circle] (E5) {E5};
  \node[draw] at (5,9.7) [rectangle] (B6) {B6};
  \node[draw] at (6,9) [circle] (E6) {E6};
  \node[draw] at (7,8.3) [rectangle] (B7) {B7};
  \node[draw] at (6,10.3) [rectangle] (Rel) {\emph{SSS}};
  \node[draw] at (8,10.3) [rectangle] (UW) {\emph{SI}};
  \coordinate (Z1) at (7,12) ;
  \node[draw] at (8,9) [circle] (E7) {E7};
  \node[draw] at (0,4.7) [rectangle] (C1) {C1--C4};
  \node[draw] at (0,3.7) [rectangle] (C6) {C6--C9};
  \node[draw] at (3,4.7) [circle] (E8) {E8};
  \node[draw] at (7,4.7) [circle] (E9) {E9};
  \node[draw] at (5,4.7) [rectangle] (C5) {C5};
  \node[draw] at (3,6) [rectangle] (NTL) {\emph{NTS}};
  \node[draw] at (7,6) [rectangle] (FTL) {\emph{TSOD}};
  \coordinate (Z2) at (7,7) ;
  
  \draw[->] (A0) -- (A1) ;
  \draw[->] (A1) -- (A6) ;
  \draw[->] (A6) -- (A9) ;
  \draw[->] (A9) -- (E1) ;
  \draw[->] (E1) -- (E2) ;
  \node at (2.7,14.3) {yes};
  \draw[->] (E1) -- (2,12) -- (A9) ;
  \node at (2.5,13.4) {no};
  \draw[->] (E2) -- (A11) ;
  \node at (4.8,14.3) {no};
  \draw[->] (E2) -- (A4) ;
  \node at (4.3,13.3) {yes};
  \draw[->] (A4) -- (A5) ;
  \draw[->] (A11) -- (E3) ;
  \draw[->] (E3) -- (7.5,15) -- (4,15) -- (E2) ;
  \node at (7.5,14.5) {yes};
  \draw[->,dashed] (E3) -- (8,15.5) -- (2,15.5) -- (A1) ;
  \node at (3,15.7) {unknown};
  \draw[->] (E3) -- (8,11.1) -- (0,11.1) -- (B1) ;
  \node at (7.5,13.5) {no};
  \draw[->] (B1) -- (B2) ;
  \draw[->] (B2) -- (B3) ;
  \draw[->] (B3) -- (B5) ;
  \draw[->] (B5) -| (E4) ;
  \draw[->] (E4) -- (B2) ;
  \node at (1.3,9.2) {yes};
  \draw[->] (E4) -- (E5) ;
  \node at (2.9,9.2) {no};
  \draw[->] (E5) -- (B6) ;
  \node at (4.3,9.7) {yes};
  \draw[->] (E5) |- (B3) ;
  \node at (3.6,8.3) {no};
  \draw[->] (B6) -- (E6) ;
  \draw[->] (E6) -- (Rel) ;
  \node at (6.3,9.7) {yes};
  \draw[->] (E6) -- (B7) ;
  \node at (6.2,8.4) {no};
  \draw[->] (B7) -- (E7) ;
  \draw[->] (E7) -- (UW) ;
  \node at (8.3,9.7) {yes};
  \draw[->,dashed] (E7) -- (7,10.7) -- (2,10.7) -- (B1) ;
  \node at (3.4,10.4) {unknown};
  \draw[-,dashed] (UW) -- (Z1) ;
  \draw[-,dashed] (Rel) -- (Z1) ;
  \draw[->,dashed] (Z1) -- (7,13) -- (E2) ;
  \draw[->] (E7) -- (8,6.5) -- (0,6.5) -- (C1) ;
  \node at (8.4,8.4) {no};
  \draw[->] (C1) -- (E8) ;
  \draw[->] (E8) -- (NTL) ;
  \node at (2.7,5.3) {yes};
  \draw[->] (E8) -- (C5) ;
  \node at (3.9,5) {no};
  \draw[->] (C5) -- (E9) ;
  \draw[->] (E9) -- (FTL) ;
  \node at (7.3,5.4) {yes};
  \draw[->] (E9) |- (C6) ;
  \node at (7.4,4) {no};
  \draw[-,dashed] (NTL) |- (Z2) ;
  \draw[-,dashed] (FTL) |- (Z2) ;
  \draw[->,dashed] (Z2) -- (E7) ;
  \draw[->,dashed] (C6) -- (-1.5,3.7) -- (-1.5,14) -- (A1) ;
  \draw[-,dotted] (-1,15.9) -- (9,15.9) -- (9,11.3) -- (-1,11.3) --
  (-1,15.9) ; 
  \draw[-,dotted] (-1,10.9) -- (9,10.9) -- (9,6.6) -- (-1,6.6) --
  (-1,10.9) ; 
  \draw[-,dotted] (-1,6.4) -- (9,6.4) -- (9,3.3) -- (-1,3.3) --
  (-1,6.4) ; 
\end{tikzpicture}
\end{center}

\subsubsection*{Legende}
\begin{tabular}{ll}
SSS & Surprisingly simple solution\\
SI & Surprising impact\\
NTS & Novel technical solution \\
TSOD & Technical solution from other domain
\end{tabular}

\section{The ABER(1)-Matrix. An Example}

Evaluation parameters and components. An example of the ABER matrix by
Hans-Jochen Rindfleisch and Rainer Thiel published in \cite[appendix
  III/2]{RT94}.
  
\begin{center}\renewcommand{\arraystretch}{1.5}
  \begin{tabular}{|l|c|c|c|c|}\hline
    & {Functionality} & {Profitablity} & {Controllability} &
    {Usefulness}\\\hline
    \textbf{A:} Requirements &  (A.1) & (A.2) & (A.3) & (A.4)\\\hline
    \textbf{B:} Conditions &  (B.1) & (B.2) & (B.3) & (B.4)\\\hline
    \textbf{E:} Expectations& (E.1) & (E.2) & (E.3) & (E.4)\\\hline
    \textbf{R:} Restrictions& (R.1) & (R.2) & (R.3) & (R.4)\\\hline
  \end{tabular}\par 
\end{center}
\begin{itemize}[noitemsep]
\item[(A.1)] Performance and fitness to drive up to a driving speed of $x$
  km/h
\item[(A.2)] 1. Fuel saving\par 2. Utilising exhaust gas heat
\item[(A.3)] 1. Easy to operate, wearing parts easily accessible\par
  2. Replacement parts available on board (or can be carried)
\item[(A.4)] 1. Adaptable to local traffic conditions\par 
  2. Can be used as truck unit, delivery van and touring van
\item[(B.1)] 1. Approved for road transport\par 2. Can be used as traction unit
\item[(B.2)] 1. Service-friendly\par 2. Well suited fo load transport
\item[(B.3)] 1. Temporarily overloadable to $x$ times normal load\par
  2. Driving behaviour (undelayed), follows steering
\item[(B.4)] 1. Insensitive to stone impact\par 2. Heat repellent\par
  3. Temperature regulating\par 4. Humidity balancing
\item[(E.1)] 1. High acceleration capacity\par 2. Delay-free acceleration
\item[(E.2)] 1. High transport yield\par 2. Low cost
\item[(E.3)] 1. Automatically compensating skidding movements\par
  2. Self-adjusting to changing road conditions\par 3. Self-monitoring 
\item[(E.4)] 1. Independent of service stations\par 2. Insensitive to low
  temperatures (e.g. when starting)
\item[(R.1)] 1. Traction and braking system true to track\par
  2. Lighting and signalling system in conformity with traffic regulations
\item[(R.2)] 1. Undemanding in terms of maintenance\par 2. Frugal in terms of
  fuel quality 
\item[(R.3)] 1. Traffic-safe\par 2. Vibration resistant\par 3. Shock and
  impact resistant\par 4. Theft-proof
\item[(R.4)] 1. Compatible with emission standards\par 2. Corrosion resistant
  to de-icing salt\par 3. Harmless for inner-city traffic
\end{itemize}

\section{German-English Translations of Terms}
\begin{center}
  \begin{tabular}{l|l}
    English & German \\\hline
    auxiliary function & Hilfsfunktion \\
    basic variant & Basisvariante\\
    core variant & Kernvariante\\
    critical functional area & kritischer Funktionsbereich\\
    critical funtion matrix & kritische Funktionsmatrix\\
    critical operational area & kritischer Wirkbereich\\
    evaluation matrix& Zielgrößenmatrix \\
    evaluation parameter & Zielparameter, Zielgröße\\
    functional structure & Funktionsstruktur \\
    guiding variable & Führungsgröße \\
    harmful natural laws & schädliche Naturgesetzmäßigkeit\\
    honorable inventors & Verdiente Erfinder\\
    objectives & Zielstellung \\
    operational area & Wirkbereich\\
    operating principle & Wirkprinzip\\
    operational field matrix & Wirkfeldmatrix\\
    problem field level & Problemfeldebene \\
    ProHEAL path model & ProHeal Wegemodell\\
    reference variant & Referenzvariante\\
    subfunction & Teilfunktion\\
    surprising impact & überraschende Wirkung \\
    surprisingly simple solution (SSS) & raffiniert einfache Lösung (REL)\\
    technical-constructive boundary conditions & technisch-konstruktive
    Randbedingungen \\
    technical-economic objectives & technisch-ökonomische Zielstellung\\
  \end{tabular}
\end{center}

\begin{thebibliography}{xxx}
\bibitem{Altshuller1973} Genrich S. Altshuller. Erfinden – (k)ein Problem?
  Verlag Tribüne, Berlin (1973).
\bibitem{Altshuller1983}  Genrich S. Altshuller, Alexander B. Seljuzki. Flügel
  für Ikarus. Urania-Verlag, Leipzig (1983).
\bibitem{Altshuller1986} Genrich S. Altshuller. Erfinden. Wege zur Lösung
  technischer Probleme. Verlag Technik, Berlin (1986). 
\bibitem{Cavallucci2000} Denis Cavallucci, P. Lutz, F. Thiebaud. Intuitive
  Design Method (IDM): A new framework for design method integration. Journal
  for Manufacturing Science and Production, 3(2-4), pp. 95-102 (2000).
  \url{https://doi.org/10.1515/IJMSP.2000.3.2-4.95}
\bibitem{Graebe2019a} Hans-Gert Gräbe. The Development of the GDR Inventor
  Schools and the Evolution of TRIZ (in Russian). In: Online material of the
  TRIZ Summit, Minsk (2019).
\bibitem{Graebe2019b} Hans-Gert Gräbe. The Contribution to TRIZ by the
  Inventor Schools in the GDR. Proceedings of the 15th MATRIZ TRIZfest, pp.
  346-352 (2019).
\bibitem{RM-23} Peter Koch, Klaus Stanke. 50 Jahre systematische Heuristik.
  Rohrbacher Manuskripte, Heft 23. LIFIS, Berlin (2021, in preparation).
\bibitem{Mueller1973} Johannes Müller, Peter Koch et al. Programmbibliothek
  zur systematischen Heuristik für Naturwissenschaftler und Ingenieure. In:
  Wissenschaftliche Abhandlungen des Zentralinstituts für Schweißtechnik
  Halle, Band 97-99, Halle (1973).
\bibitem{Mueller1990} Johannes Müller. Arbeitsmethoden der
  Technikwissenschaften.  Systematik, Heuristik, Kreativität. Springer, Berlin
  (1990).
\bibitem{Petrov2020} Vladimir P. Petrov. Laws and patterns of systems
  development. Book in 4 vol. (in Russian).  Ridero, Moscow (2020).
\bibitem{RT89} Hans-Jochen Rindfleisch, Rainer Thiel, Gerhard Zadek.
  KDT-Erfinderschule, Lehrbrief 2: Erfindungs­methodische Arbeitsmittel.
  Lehrmaterial zur Erfindungsmethode. Berlin (1989).
\bibitem{RT94} Hans-Jochen Rindfleisch, Rainer Thiel. Erfinderschulen in der
  DDR. Trafo Verlag, Berlin (1994).
 \bibitem{RM-21} Rainer Thiel.  Dialektik, TRIZ und ProHEAL. Rohrbacher
   Manuskripte, Heft 21. LIFIS, Berlin (2020).
\end{thebibliography}

\end{document}
