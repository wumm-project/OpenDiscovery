\documentclass[11pt,a4paper]{article}
\usepackage{od}
\usepackage[utf8]{inputenc}
\usepackage[ngerman]{babel}

\title{Mathematik und Gesellschaftswissenschaften}

\author{Thomas Kuczynski}

\date{Erschienen in Jb. f. Wirtschaftsgeschichte 1969/11}
\begin{document}
\maketitle

\begin{quote}
  Rainer Thiel, Quantität oder Begriff? Der heuristische Gebrauch
  mathematischer Begriffe in Analyse und Prognose gesellschaftlicher Prozesse

  VEB Deutscher Verlag der Wissenschaften, Berlin 1968, IX u. 611 Seiten,
  Preis: 19,60 M
\end{quote}
Die Mathematik wird häufig als „die Wissenschaft von den quantitativen
Verhältnissen und den räumlichen Beziehungen in der objektiven Realität“
bezeichnet\footnote{Meyers Neues Lexikon, Leipzig 1963, Stichwort
  Mathematik.}.

Dementsprechend befassen sich fast alle Veröffentlichungen über die Anwendung
mathematischer Methoden in den Gesellschaftswissenschaften, die in der DDR
erschienen sind, mit der quantitativen Seite der Angelegenheit.

Der Autor des hier zu rezensierenden Buches teilt diese Auffassung nicht, geht
im Gegenteil „stattdessen von den wesentlichen Bedürfnissen der
Gesellschaftswissenschaft aus“ (1) und setzt „deren \emph{Begriffe} zu den
mathematischen \emph{Begriffen} in Beziehung“ (S. V)\footnote{Alle nicht näher
  bezeichneten Seitenangaben im laufenden Text beziehen sich auf das hier
  besprochene Buch.}.

Diese Aufgabe hat der Autor in glänzender Weise gelöst. Das Buch ist im besten
Sinne des Wortes eine Sensation auf dem Büchermarkt der DDR. Da dem Autor die
Mathematik und die Gesellschaftswissenschaften nicht nur bekannt sind, er sie
vielmehr in ihrem Inhalt erkannt hat, hat das Buch den Vorzug, keine platten
Praktizismen zu liefern, sondern die Verhältnisse \emph{begrifflich} zu
fassen, sie \emph{begreifbar} zu machen. Das kommt zum Beispiel in seiner
Beurteilung von Waffenschmidts „Wirtschaftsmechanik“\footnote{Waffenschmidt,
  W. G., Wirtschaftsmechanik, Stuttgart 1957.} zum Ausdruck: „Die Untersuchung
ist auch ziemlich unergiebig, denn Waffenschmidt versteift sich darauf,
effektive Anwendungen zu finden. Die Frage nach dem heuristischen Nutzen wird
von W. nicht gestellt ...“ (S. 590, meine Hervorhebung -- Th. K.)

Genau das aber tut Thiel. Er sucht nicht nach praktischen
Anwendungsmöglichkeiten (jedenfalls nicht in erster Linie), sondern nach
praktikablen Theorien. Diese Unterscheidung ist sehr wichtig. Sie kommt ganz
klar zum Ausdruck bei der Betrachtung der Wissenschaftsentwicklung.

Ausgehend von Engels' Behauptung, „die Wissenschaft schreitet fort im
Verhältnis zu der Masse der Erkenntnis, die ihr von der vorhergehenden
Generation hinterlassen wurde ...“\footnote{Engels, Friedrich, Umrisse zu
  einer Kritik der Nationalökonomie, in: Marx/Engels, Werke, Bd. 1, Berlin
  1956, S. 521.}, schreibt er: „Das 'Fortschreiten' wird sicherlich durch die
Geschwindigkeit, mit der sich die 'Masse der Erkenntnis' verändert,
charakterisiert. Wir nehmen vereinfachend an: durch diese Geschwindigkeit und
nichts anderes. Dann kann die Formel von Engels -- 'die Wissenschaft schreitet
fort im Verhältnis der Masse der Erkenntnis' - durch das Modell $y' = py$
(wobei $y$ die Masse der Erkenntnis zur Zeit $t$ ist) interpretiert
werden. Diese Differentialgleichung liefert als Lösung eine
Exponentialfunktion ... Das Prinzip der Einfachheit in Rechnung stellend kann
man also mit völliger Klarheit sagen, was Fortschritt 'im Verhältnis zur Masse
der Erkenntnis' bedeutet. Man ist daher erstaunt über die Behauptung, diese
'Wissenschaftsentwicklung' sei 'kaum vorstellbar'\footnote{„Deutsche
  Zeitschrift für Philosophie“, Sonderheft 1956, S. 521.}. Erstens kann man,
wie wir sahen, den von Engels angenommenen Zusammenhang mathematisch
formulieren ... Zweitens kann man aus dieser Formulierung die wohlbekannte
Exponentialfunktion als Wachstumsgesetz ableiten\footnote{Die
  Exponentialfunktion definiert Thiel mit $y = y_0e^{a(t-t_0)}$ (S. 258).} ...
Drittens findet man dasselbe Wachstumsgesetz in allen Bereichen der
Wirklichkeit ... Viertens schließlich braucht das Wachstumsgesetz $y' = py$
keineswegs bedeuten, daß sich etwa die Wissenschaft mit atemberaubender
Geschwindigkeit entwickle. Man braucht nur anzunehmen, daß der Exponent $a =
0,006931$ Jahr$^{-1}$, so hat man den Fall, daß die Erkenntnismasse hundert
Jahre braucht, um sich zu verdoppeln. Die philosophische, in mathematischer
Form ausgesprochene Annahme über die Erkenntnisentwicklung -- wie wir sie bei
Engels finden -- ist \emph{unabhängig} davon, ob wir eine genaue Meßvorschrift
für den Fortschritt der Erkenntnis kennen ... Das Modephänomen der
Gnoseologie, die Behauptung nämlich, die Erkenntnis verdopple sich alle zehn
oder fünfzehn Jahre, beruht auf einer Illusion. Wäre die Messung derartiger
Prozesse so einfach, dann müßte man sich wundern, daß es bis jetzt noch keine
Theorie des Wachstums der Erkenntnis gibt ... Im Grunde genommen handelt es
sich darum, einen Weg einzuschlagen, der schon vom Schulunterricht über
Dreieckskonstruktionen bekannt ist: 'Angenommen, Dreieck ABC ist konstruierbar
...' Auch hier kam es darauf an, durch Betrachtungen über die Bedingungen der
Lösung die Lösung selbst zu finden ... In diesem Sinne betrachten wir auch die
Angabe der Bedingung $y'=ay$ für das Erkenntniswachstum als Ausgangspunkt zur
Suche nach einem etwaigen Maß der Erkenntnis“ (S. 285--88).

Wiewohl Thiel auch Probleme der Wirtschaftsgeschichte streift (vgl. zum
Beispiel S. 301 f. über die asiatische Produktionsweise, Kapitel 3.3 über die
internationale Arbeitsteilung usw.), liegt die Bedeutung des Buches auch für
den Wirtschaftshistoriker in den theoretischen Ableitungen mathematisch
formulierbarer Zusammenhänge in der Gesellschaft.

Die bedeutendste Leistung von Thiel liegt aber meines Erachtens in der streng
philosophisch begründeten Definition der Aufgaben der Mathematik: sie habe „je
nach den Anforderungen der Praxis oder der Theorie selber beliebige Strukturen
und Prozesse \emph{in ihrer Reinheit} zu untersuchen, und zwar mit einer
solchen Konsequenz \emph{in ihrer Reinheit}, daß sogar die Feststellung der
\emph{empirischen Existenz} dieser Objekte anderen Wissenschaften überlassen
bleibt“ (S. 254). Danach stehe die Mathematik auf der niedrigsten
Interpretations- und der höchsten Abstraktionsstufe, während diejenigen
Einzelwissenschaften, die den höchsten Grad an Interpretationsmöglichkeit
aufweisen, auf der niedrigsten Abstraktionsstufe stehen“ (S. 394). Diese
Unterscheidung ist für jeden Wissenschaftler wichtig, da man eben aus
einzelnen Beispielen, die sehr schön zu interpretieren sind, keine Begriffe,
geschweige denn Theorien ableiten kann (vgl. auch S. 204 f.) -- für den
Wirtschaftshistoriker insofern, als man aus historischen Einzelfakten weder
die Theorie des historischen Materialismus noch die der Politischen Ökonomie
entwickeln kann.

Die Lektüre des Buches ist ein Genuß -- nicht zuletzt deshalb, weil der
Verfasser seine Gedanken in ausgezeichneter Form darbietet und, was auch nicht
verbreitet ist, auf ihre Wurzeln in der Geschichte der Wissenschaft hinweist.
Allerdings ist es zum „Diagonallesen“ nicht gegeignet, man muß im Gegenteil
die Anstrengung des Begriffs auf sich nehmen, um es mit Nutzen zu lesen. Die
Anforderungen, die es an die mathematische Vorbildung des Lesers stellt, sind
nicht sehr hoch, um so mehr die, die es an das mathematisch-philosophische
Denken stellt. Zu Recht schreibt der Autor: „In den letzten Jahren sind auch
naive Vorstellungen über die Möglichkeiten des Gebrauchs der Mathematik
entstanden, insbesondere die Illusion, daß exakt, aber doch im Gewande des
herkömmlichen soziologischen Begriffssystems aufgeschriebenes Faktenmaterial
den Rechenmaschinen anheimgegeben wird, welche dann wunderbare Resultate
zeitigen. Wir müssen dazu bemerken: Sollen überhaupt Fortschritte in dieser
Richtung erzielt werden, so nur dadurch, daß die interessierten
Gesellschaftswissenschaften ihren Begriffsapparat weiter
entwickeln\footnote{Vgl. auch \emph{Gunther Kohlmey}, Zu einigen Fragen des
  Erkenntnisprozesses in der marxistischen politischen Ökonomie, in: „Probleme
  der politischen Ökonomie“. Berlin 1959, sowie in: „Mathematik und Kybernetik
  in der Ökonomie“, S. 9.}. Gerade hierzu ist die Integration von
Vorstellungen, die an den einschlägigen mathematischen und kybernetischen
\emph{Begriffen} orientiert sind, entscheidend“ (S. 141, meine Hervorhebung --
Th. K.). Und nichts ist schwieriger, als einen Begriffsapparat zu
entwickeln. Aber: Nichts ist auch notwendiger. Wenn Kurt Hager auf dem
9.~Plenum des ZK der SED zu den Ergebnissen der Arbeit unserer
Gesellschaftswissenschaftler sagte: „Die Ergebnisse reichen jedoch nicht aus,
wenn wir sie an den großen und komplizierten Aufgaben, die die Gestaltung der
sozialistischen Gesellschaft unter den Bedingungen der
technisch-wissenschaftlichen Revolution sowie die verschärfte ideologische
Auseinandersetzung zwischen Sozialismus und Imperialismus heute stellen,
messen“\footnote{ Hager, Kurt, Die Aufgaben der Gesellschaftswissenschaften in
  unserer Zeit, Berlin 1968, S. 49.}, so resultiert das nicht zuletzt aus dem
ungenügend entwickelten Begriffssystem der marxistisch-leninistischen
Gesellschaftswissenschaften.

Sicher wird man über viele Lösungen, die Thiel anbietet, streiten können. Das
erscheint mir nicht als das Wesentliche. Einem Wort von Georg Cantor folgend,
ist in der Mathematik die richtige Fragestellung wichtiger als die richtige
Lösung\footnote{So in seiner Dissertation: \emph{Cantor, Georg}, De
  aequationibus secundi gradus indeterminatis, Berlin 1867, These III: „In re
  mathematica ars propo‚endi quaestionem pluris facienda est quarr
  solvendi.“}.  Und hier hat Thiel Außerordentliches geleistet. Jeder
Gesellschaftswissenschaftler sollte dieses Buch studieren?

\end{document}
