\documentclass[12pt,a4paper]{article}
\usepackage{od}
\usepackage[utf8]{inputenc}

\title{Dialektik, TRIZ und ProHEAL}
\author{Rainer Thiel, Storkow}
\date{19.02.2019} 

\begin{document}
\maketitle
\begin{quote}
  Überarbeitetes Material des Autors zu ProHEAL, das vom Autor wesentlich
  zusammen mit Dr.-Ing. Hans-Jochen Rindfleisch, Verdienter Erfinder der DDR,
  erarbeitet wurde.
\end{quote}
\tableofcontents

Wie kam es zu dieser Konstellation?
Dazu Rainer Thiel: . . . . . . . . . Hier weggelassen

\section*{Motivation zu ProHEAL -- Eintritt gesellschaftlicher Bedürfnisse}

ProHEAL steht für \emph{Programm des Herausarbeitens von Erfindungsaufgaben und
  Lösungsansätzen}. Die Exposition von ProHEAL erfolgt in drei
Darstellungsweisen:
\begin{itemize}
\item [1.] ProHEAL – mehrfache Provokation! Kleine Einführung ins ProHEAL
\item [2.] ProHEAL – das Programm in algorithmischer Darstellung. Zum
  Warm-Laufen empfiehlt sich, mit Kapitel 3 zu beginnen:
\item [3.] ProHEAL – das Programm in erzählender Darstellung
\end{itemize}

\section{ProHEAL – mehrfache Provokation! Kleine Einführung in das ProHEAL}

Erstens wendet sich ProHEAL an alle Ingenieure, gleich, in welchem Bereich sie
tätig sind: Sie sollen nicht nur Aufträge erfüllen – sie sollen kreativ über
den Auftrag hinaus denken: Wir brauchen eine neue, friedliche,
menschenfreundliche und umweltverträgliche Welt. Ingenieure sollen Mitgestalter
werden.

Zweitens sollen sich Ingenieure mit den gesellschaftlichen Bedürfnissen
vertraut machen, denen ihr Auftrag entsprungen zu sein scheint.

Drittens sollen Ingenieure über den jeweils empfangenen speziellen Auftrag
hinausgehend zu denken beginnen:

Assoziations-anregend kann ein brainstorming sein, wenn es denn richtig
verstanden wird: Als Paar von direkter und inverser Ideenkonferenz. Durch
direkte Ideenkonferenz sollen Teilnehmer ihre Hemmungen überwinden, sie dürfen
auch spinnen, darüber lachen und sich in ihre Ausgeburten sogar verlieben. Aber
aber: In den nachfolgenden Stunden muss die Inverse Ideenkonferenz folgen, wie
es der Schöpfer des brainstorming – anno 1944 in Amerika - gewollt hat, nämlich
Gericht zu halten über Ausuferungen der Lust an ungezügelter Phantasie. Da
prallen die Meinungen aufeinander im brainstorm-team. Da kann es hoch hergehen
wie beim Sturm im Wasserglas. Dann aber müsste systematisches Denken einsetzen,
nämlich System-Denken. Und da setzt das ProHEAL ein. Zu Beginn werden in einer
matrixförmigen Tabelle Anforderungen A, Bedingungen B, Erwartungen E und
Restriktionen R als Zeileneingänge sowie sogenannte Zielgrößen als
Spalteneingänge gesammelt. Also fülle die Felder der folgenden Matrix aus mit
konkreten Angaben relevanter Parameter vorgefundener Objekte und zu
wünschenswerter Entwicklung ihrer Werte:
\begin{center}
  Erste ABER-Matrix einfügen
\end{center}

Es erscheint uns also entscheidend, die wesentlichen Parameter durch Analyse
von technisch-ökonomischen Belangen zu finden, sodann durch kräftige, extensive
Parameter-Variation über das Vorgefundene (damit auch über Altschuller)
hinausgehend – Widersprüche gedanklich vorwegzunehmen (zu antizipieren) und
analysierbar zu machen. Deshalb also die $4\times 4 = 16$ Felder-Matrix mit den
ABER und den Zielgrößenkomponenten. So findet der Ingenieur selbständig zur
Analyse von vorwegzunehmenden Widersprüchen im technisch-ökonomischen
Denkfeld. Der Ingenieur gewinnt an Zielklarheit und an Lust, kreativ zu werden,
hinausgehend über den Tag und den Auftrag des Chefs. So gewannen wir einen
Assoziations-Generator für den Ingenieur, um dessen Erfahrungen zu aktivieren
für gründliche Recherche der Bedürfnisse von Nutzer und Hersteller, Kunde und
Fabrikant, und um den Ingenieur zu motivieren, weiteres Material aus Literatur
und Nachbar-Abteilungen seines Betriebes zu beschaffen. Dass die Zeilen- und
die Spalten-Inhalte sich redundanz-artig überdecken können, stört nicht. Es
geht darum, die Assoziation möglichst stark anzuregen und Widersprüche sichtbar
zu machen.

Dabei kann man einen Spaß und zwei Beinahe-Redundanzen bemerken. Begonnen
hatten wir, dem gesellschaftlichen Bedürfnis entsprechend die Anforderungen,
die Bedingungen und die Restriktionen zu notieren, also die A, die B und die R.
Da meinte Hans-Jochen: Nehmen wir doch gleich noch die Erwartungen mit dem
Anfangsbuchstaben E hinzu, das duftet zwar nach Redundanz, denn Anforderungen,
Bedingungen und Restriktionen haben wir schon im Blick, doch wir haben statt
ABR jetzt ABER. Das ist nicht nur ein schönes Logo aus vier Buchstaben, das ist
auch ein Alarm-Signal beim Brainstorming, dem ein inverses Brainstorming folgen
muss. Da spielt in deutscher Sprache die Kopula „aber“, also der kritische
Einwand, eine Rolle.

Eine weitere Beinahe-Redundanz erlaubten wir uns, indem wir außer den ABER auch
noch vier Zielgrößen-Komponenten ins Blickfeld zogen. So entstand aus der
eindimensionalen ABER Liste eine zweidimensionale Matrix mit insgesamt $4^2=16$
Feldern. Und Redundanz wegen der Begriffsverwandtschaft der Zeilen- und der
Spalten-Eingänge? Das ist ein Glücksfall. Denn jetzt entsteht für den Techniker
ein (4-hoch-2)-Generator, das Denken anzukurbeln. Das ist Wind, um Widersprüche
herauszuarbeiten und lösbar zu machen. Wir brauchten nur noch zu sagen: „Und
jetzt, liebe Kolleginnen und Kollegen, treiben wir übliche Parameterwerte in
die Höhe.“ In unsren Erfinderschulen hat das Wirbel ausgelöst. Techniker waren
aufs höchste angeregt, bald auch aufgeregt, und bald kamen laute Zwischenrufe.
Verdutzte Techniker riefen: „Da kommen wir doch in Widersprüche!“ Ja, genau das
wollen wir, das hätte auch Altschuller begeistert. Und ich konnte den
Technikern sagen: „Da sind Sie von ihren Professoren getäuscht worden, von
wegen in einer Ingenieur-Aufgabe dürfe nie ein Widerspruch auftreten.“ Wir aber
haben gelernt von Hegel und von Marx. Dort lernten wir außerdem, dass zur
Dialektik gehört: Bäume wachsen nicht bis zum Himmel. Da muss etwas Neues
kommen! Wer darauf hinwirkt, provoziert dialektische Widersprüche.

Die ABER-Matrix ist (mit redaktionellen Wandlungen) auch von Hansjürgen Linde
in seiner Dissertation TUD 1988 und 1993 in der Druckausgabe seiner WOIS
vielfach verwendet worden, ebenso Schreibweisen in Textfassungen, 1980 von
Thiel eingeführt, um Texte Altschullers komprimieren zu können. Linde und Thiel
waren in herzlicher Freundschaft verbunden. Eines Tages rief er mich an:
„Morgen muss ich ins Krankenhaus.“ Drei Wochen später empfing ich eine
Nachricht aus seinem Institut: Sein reiches Leben hatte sich vollendet.

Seit Ende der achtziger Jahre haben wir – zumindest in Berlin – die ABER-Matrix
in Erfinde-Workshops angewandt. Das erste Ergebnis war schockierend. Beim
Ausfüllen der Felder, mit denen fixiert wird, was alles gleichzeitig erreicht
werden soll, bricht immer ein Ingenieur aus in den Ruf: „Da kommen wir ja in
Widersprüche.“ Unsre Antwort: Sie sollen auch mal kühn und frech sein! Mit
Ihrem Ausruf ´da kommen wir ja in Widersprüche´ zeigen Sie, dass etwas gefehlt
hat in Ihrer Ausbildung. Sie sind von ihren Professoren in die Irre geführt
worden.“

Die Denkarbeit, die zu dieser matrix-förmigen Tabelle führte, hatte – über
Altschuller hinausgehend - 1980 begonnen mit dem Vorschlag zu einer
Notierungsweise technisch-ökonomischer Widersprüche, die auch zitiert wurde in
dem ersten Erfinderschule-Lehrmaterial, das von Michael Herrlich verfasst
worden war. Die weitere Ausgestaltung hat fünf Jahre in Anspruch genommen.
Damit waren Rindfleisch und Thiel zum zweiten Mal zur Dialektik aller
Entwicklung vorgedrungen und zum ersten Mal über Altschuller hinausgegangen.

Freude hatte ursprünglich ausgelöst, dass Altschuller in seine Tabellenfelder –
von ihm als fest vorgegeben - sogleich auch Lösungsverfahren eingetragen hatte,
die von ihm als relevant angenommen wurden. Gewiss kann die
Kurzerhand-Zuordnung von Standard-Lösungsverfahren für manchen Nutzer Anregung
bieten. Wir meinen aber, dass manchem Nutzer Zweifel kommen, ob Lösungen immer
auf diese Weise gefunden werden können. Das war uns auch Grund, die
Lösungsverfahren aus der Altschuller-Tabelle herauszulösen, um sie zu einem
späteren Zeitpunkt des schöpferischen (kreativen) Prozesses effektiver ins
Spiel bringen zu können. Es nützt nicht viel, sie zu früh anwenden zu wollen,
wenn das Problem noch gar nicht hinreichend als dialektischer Widerspruch oder
als set von dialektischen Widersprüchen bestimmt ist.

Wir fanden aber einen dritten Anlass, die Dialektik aller Entwicklung in der
Methodik des Erfindens geltend zu machen: Die Bewertung der von Altschuller
1973 in „Erfinden (k)ein Problem“ vorgeschlagenen vierzig Lösungsprinzipe. Wir
verwerfen sie nicht. Doch in ihrer Relevanz für die Ausprägung einer
erfinderischen Denkweise unterscheiden sie sich. Grundlegend sind die Prinzipe
Nr. 23 „Umwandlung des Schädlichen in Nützliches“ und Nr. 22 „Überlagerung
einer schädlichen Erscheinung mit einer anderen“. Das kann auch geschehen durch
Spaltung des Einheitlichen in entgegengesetzte Komponenten, die sich – zum
Beispiel bei thermisch bewirkten Längenänderungen beim berühmten Duncker-Pendel
-- gegenseitig kompensieren. Leider hatte Altschuller den Psychologen Carl
Duncker missverstanden und obendrein dessen geniales Pendel-Syndrom völlig
übersehen. Wir kommen in Kapitel 3 darauf zurück.

Nun blicken wir nochmals auf den Kern von ProHEAL, die Matrix, des Beispiels
halber 1988 mit Eintragungen für Kraftfahrzeuge versehen:
\begin{center}
  Zweite Version der ABER-Matrix einfügen.
\end{center}
Nun suche Dir die Kästchen aus, wo Du gängige Parameterwerte erhöhen willst:
Ein oder zwei oder auch mehrere. Treibe gängige Parameterwerte (bzw. ihren
Kehrwert) in die Höhe. Provoziere Widersprüche. Was meinte Johann Wolfgang
Goethe? „Das Gleiche lässt uns in Ruhe, aber der Widerspruch ist es, der uns
produktiv macht.“ Mit ProHEAL werden daraus prägnante Erfindungsaufgaben
abgeleitet.

Und nun die vierte Provokation, von mir ausgesprochen beim Leibniz-Institut für
interdisziplinäre Studien LIFIS, auf dessen Konferenz im November 2016 in
Lichtenwalde bei Chemnitz auf der Abschluss-Beratung (jetzt redaktionell
präzisiert):

Gestatten Sie bitte zwei Anmerkungen zu den täglich sichtbaren Leuchtschriften
„Innovation“ und „Wachstum“:

\paragraph{1.}
Innovationen, unser Zeitalter. Da ist ja etwas dran. Aber nur etwas. Ich
zitiere aus einem Buch, das gerade in Frankfurt und New York erschienen ist und
sich auf umfangreiche Literatur stützt, u.a. auf eine Fraunhofer-Studie. In
diesem Buch heißt es – ich zitiere: …. “dass ein immer größerer Teil der
Patentanmeldungen nicht mehr dadurch motiviert ist, eigene Innovation vor
Imitation zu schützen. …. Stattdessen dominiere das Ziel, die Anwendung
bestimmter Technologien durch Konkurrenten zu blockieren …. Oder es werden
Verfahren patentiert, denen überhaupt keine Innovation zugrunde liegt. Immer
öfter würden Patente nicht \emph{deshalb} angemeldet, um sie zu nutzen, sondern
um die Nutzung einer den eigenen Produkten gefährlichen Innovation zu
verhindern.“ Auch deshalb meine ich, dass Altschuller der massenhaften
Auswertung von Patentschriften zu viel Bedeutung beigemessen hat.

Und noch ein Wort zum Worte „Innovation“: Einer der kreativsten Menschen aller
Zeiten, Albert Einstein, ein Humanist, den Kommunisten zugetan, forderte den
amerikanischen Präsidenten auf, die Entwicklung der Atombombe administrativ
einzuleiten, um damit Hitler zuvorzukommen. Doch die arroganten
US-Geheimdienstler hatten gar nicht bemerkt, dass Hitler noch lange vor seiner
Atombombe besiegt werden konnte. Eine Innovation! Doch diese Innovation
missachtend begann in Los Alamos die Entwicklung von Atombomben. Als der Krieg
schon entschieden war, wurde aus machtpolitischen Gründen auf Hiroshima und
Nagasaki je eine Bombe aus den USA geworfen. Hunderttausende Japaner starben.
Der kalte Krieg begann. Diese Innovation hätte verflucht werden müssen.

\paragraph{2.}
Meine zweite Anmerkung, nun zu den Leuchtbuchstaben „Wachstum“: Forciert wird
wirtschaftliches Wachstum, das die Bewohnbarkeit unsrer kosmischen Heimat,
unsrer Erde, untergräbt. In nördlichen Industrieländern wird Menge und Vielfalt
von Konsumgütern und Waffen hemmungslos vergrößert. Schon im 19. Jahrhundert
begannen Philosophen und Dichter davor zu warnen: Rousseau, Jean Paul, Karl
Marx. Der Dichter Jean Paul erzählt, wie er sich an einen Freund wandte: Kannst
Du denn nicht sehen, „dass die Menschen toll sind und schon Kaffee, Tee und
Schokolade aus besonderen Tassen, Früchte, Salate und Heringe aus eigenen
Tellern, und Hasen, Früchte und Vögel aus eigenen Schüsseln verspeisen. – Sie
werden aber künftig, sag´ ich Dir, noch toller werden und in den Fabriken so
viele Fruchtschalen herstellen, als in den Gärten Obstarten abfallen…., und
wär´ ich nur Kronprinz oder Hochmeister, ich müsste Lerchenschüsseln und
Lerchenmesser, Schnepfenschüsseln und Schnepfenmesser haben, ja eine
Hirschkeule von einem Sechsender würd´ ich auf keinem Teller anschneiden, auf
dem ich einen Achtender gehabt hätte.“ Ich füge hinzu: So leben wir. Die
Schränke voll und voller. Dicht und dichter gedrängt verdecken Sachen die Sicht
auf Sachen, die schon da sind: Verdeckt, vermisst und abermals gekauft. Man
tröstet sich, das Neue sei moderner…. Bis schließlich nur noch Röcheln ist: Wir
können nicht anders. Fahren wir zum Kaufhaus.“\footnote{Auszüge aus R. Thiel:
  „Marx und Moritz. Unbekannter Marx. Quer zum Ismus“. Trafo Verlag Berlin
  1998.  Dort auch ein Marx-Zitat aus MEW 25, Seite 784: „Selbst eine ganze
  Gesellschaft, eine Nation, ja alle gleichzeitigen Gesellschaften
  zusammengenommen, sind nicht Eigentümer der Erde. Sie sind nur ihre Besitzer,
  ihre Nutznießer, und haben sie als boni patres familias den nachfolgenden
  Generationen verbessert zu hinterlassen.“}

Ist das nicht unsre Wirtschaft seit Jahrzehnten? Nichts gegen Märkte, wir
brauchen sie. Sie werden durch mittelständische, genossenschaftliche,
gemeinnützige Unternehmen belebt\footnote{Sahra Wagenknecht 2016, Hans Küng
  2010.}.  Doch das Gerüst unbegrenzter Marktwirtschaft strebt Richtung Hölle,
und das hat längst neue Widersprüche hochgepuscht. Beunruhigt sind Mitbürger
christlichen Glaubens, Naturfreunde, Nichtregierungsorganisationen NGO sowie
einige Linke und Grüne. Bei ATTAC gibt es eine Arbeitsgruppe „Transformation
statt Wachstum“. Ich war Mitbegründer. Doch Techniker sind kaum dabei.
Techniker lassen sich vom herrschenden Kapital missbrauchen und helfen, unsre
kosmische Heimat „Erde“ unbelebbar zu machen.

Was tun wir nun mit den extensiven Texten zu TRIZ? Altschuller hatte in einem
Land gewirkt, in dem noch manches fehlte, was uns im Westen längst Gewohnheit
war.  Auch in Asien und Afrika fehlt es an vielem. Muss aber in
Entwicklungsländern alles wie in nördlichen Industrie-Ländern geraten? Deshalb
ist „Transformation statt Wachstum“ anzusagen, eine Kiste mit vielen Problemen,
mit Widersprüchen, vor denen wir alle stehen. Wir müssen sie erkunden. Mit
ProHEAL und seiner ABER-Matrix sind viele Probleme direkt ansprechbar.

Wenn wir trotzdem Freunde von TRIZ sein möchten, müssen wir auch diese
Widersprüche erkunden. Ingenieure – widersteht dem Wachstums-Wahn! TRIZ darf
nicht missbraucht werden. Wir wollen keine Sklaven des großen Kapitals sein.
Lasst uns überlegen: Wie muss TRIZ genutzt werden, um unsre kosmische Heimat zu
sichern? TRIZ im Rucksack und ProHEAL im Kopf. Für eine menschenfreundliche,
uns allen zuträgliche, friedliche Welt, und nicht für Wachstum, Waffen und
Kaufhäuser. Schon wieder steckte ein Riesen-Katalog in meinem Briefkasten: Was
es alles Neues gibt auf der Welt. Das meiste aber ist Unsinn, den man nicht
braucht. Und der Gipfel aller Angebote: Modelle der jüngeren und der neuesten
Kriegs-Panzer samt Übungsmunition.

Es lebe die Demonstration auf der Straße. Es lebe das Brot und es lebe der
Wein.

\section{ProHEAL – komplex und in algorithmischer Darbietung}

ProHEAL ist ein Akronym und steht für \emph{Programm des Herausarbeitens von
  Erfindungsaufgaben und Lösungsansätzen}.

\subsection{Das gesellschaftliche Bedürfnis. Vorläufige Systembenennung}

1.1 Welche Funktion soll das technische System erfüllen? In welchen
übergeordneten Nutzungsprozess soll diese Funktion eingebunden sein?

1.2 Welchem speziellen Bedürfnis der Gesellschaft (bzw. des Exportkunden) soll
dieser übergeordnete Nutzungsprozess dienen? Welche Gebrauchseigenschaften und
Eignungsmerkmale dieses Systems sind notwendig und hinreichend, damit es dem
übergeordneten Nutzungsprozess besser als bisher entspricht? Welches spezielle
Bedürfnis kommt darin zum Ausdruck? Welche Nutzungsprozesse sind im In- und
Ausland bekannt, die einem vergleichbaren Bedürfnis dienen?

1.3 Analysiere Literatur, Patente, Forschungsberichte, Marktinformationen,
Reiseberichte.

1.4 Wie lange gibt es das spezielle Bedürfnis schon? Wie hat es sich
entwickelt? Welche Bedingungen für die Verwendung und welche Anforderungen an
die Gebrauchseigenschaften und Eignungsmerkmale haben sich mit der Entwicklung
des Nutzungsprozesses verändert? Zeige mögliche Tendenzen der weiteren
Entwicklung auf. Lässt sich eine Tendenz finden, die bisher nicht gesehen
wurde?

1.5 Mit welcher Hauptfunktion erfüllt das zu erneuernde technische System
seinen spezifischen Zweck im übergeordneten Nutzungsprozess? Welche seiner
Gebrauchseigenschaften sind dafür kennzeichnend? Welchen Anforderungen und
Bedingungen müssen sie genügen?

1.6 Welche allgemeinen, übergreifenden gesellschaftlichen Bedürfnisse sind zu
beachten? Warum sind sie entstanden? Wie haben sie sich entwickelt? Wie werden
sie sich voraussichtlich entwickeln? Welche Restriktionen in bezug auf die
Nutzung von Ressourcen und welche Erwartungen in bezug auf den Nutzungseffekt
ergeben sich daraus?

1.7 Welche Anforderungen, Bedingungen, Erwartungen und Restriktionen (ABER)
bestimmen die erforderliche Entwicklung der gesellschaftlichen Effektivität des
technischen Systems? Nenne die ABER vollständig und begründe sie. Prüfe, ob sie
nicht aus subjektiven Auffassungen oder Vorurteilen resultieren. Welches
Entwicklungsziel folgt aus den ABER?

1.8  Welche spezifischen ABER bestimmen die Zielgrößenkomponenten 
\begin{itemize}
  \item Zweckmäßigkeit (Z1)
  \item Wirtschaftlichkeit (Z2)
  \item Beherrschbarkeit (Z3)
  \item Brauchbarkeit (Z4)
\end{itemize}
des zu schaffenden technischen Systems als Ganzes? (Zielgrößen-Matrix)

1.9 Welche Prioritäten ergeben sich aus den ABER für die einzelnen Merkmale des
auftragsgemäß zu erneuernden technischen Systems, die den vier
Zielgrößen-Komponenten zugeordnet sind?

1.10 Welche Zusammenhänge zwischen diesen Merkmalen bzw. Eigenschaften –
kooperative oder gegenläufige – lassen sich in der Zielgrößen-Matrix abheben?

\subsection{Stand der Technik, Vorauswahl und System-Analyse einer
  Start-Variante.  Bedürfnisgemäße Variation der Systemparameter.}

2.1 Welches ist das für die Realisierung der Zielgröße am besten geeignete
technisch-technologische Prinzip?

a) Untersuche die auf dem Stand der Technik bekannten Prinzipien der
Herstellung und/oder Nutzung technischer Objekte aus deinem Technologie-Bereich
auf Eignung in bezug auf die ABER. Wähle das technisch-technologische Prinzip,
\begin{itemize}
  \item das dem Zweck des zu schaffenden technischen Systems
    (Zielgrößen-Komponente Z1) am meisten entspricht, 
  \item mit dessen Anwendung voraussichtlich nicht oder vergleichsweise wenig
    gegen Anforderungen und Restriktionen verstoßen wird und
  \item das die Bedingungen und Erwartungen ohne wesentlichen Zusatzaufwand zu
    erfüllen vermag.
\end{itemize}
Hierbei sind die verfügbaren und alle machbar erscheinenden Mittel und
Verfahren auf dem Stande der Technik in Betracht zu ziehen.

b) Ist ein technisch-technologisches Prinzip mit der Aufgabenstellung
verbindlich vorgegeben, so überprüfe es auf seine Eignung und vergleiche es mit
anderen bekannten Prinzipien.

c) Ist ein geeignetes technisch-technologisches Prinzip im eigenen
Technologie-Bereich nicht auffindbar, ist die Suche auf weitere, auch fern
liegende Bereiche auszudehnen.

d) Formuliere technisch-ökonomische Parameter (Effektivitäts-Parameter) so,
dass sie dem technisch-technologischen Prinzip gemäße Messgrößen für die
Eigenschaften sind, welche durch die Zielgrößenkomponenten gefordert werden.

2.2 Welche Arten von Objekten müssen in Betracht gezogen werden, um das
technische System dem technisch-technologischen Prinzip entsprechend nutzbar zu
machen?

a) Welche Gebrauchseigenschaften müssen die Vertreter der einzelnen Objektarten
haben, damit sie entsprechend Z1 für die Verwendung im technischen System
geeignet sind?

b) Welche Objektart trägt am meisten zu den Effektivitäts- und
Eignungsmerkmalen des technischen Systems bei?

c) Sind weitere Objektarten mit spezifischen Gebrauchseigenschaften in Betracht
zu ziehen, um allen notwendigen Eignungs- und Effektivitätsmerkmalen des
technischen Systems hinreichend im Hinblick auf Z3 und Z4 Rechnung tragen zu
können?

2.3 Welche Hauptfunktion hat der Nutzungsprozess des technischen Systems?

a) Mit welcher notwendigen Teilfunktionen ist die Hauptfunktion gemäß
technisch-technologischem Prinzip zu verwirklichen?

b) Durch welche Teilfunktionen werden welche Objekte auf welche Weise in den
Nutzungsprozess einbezogen?

c) Wie werden dadurch ihre Gebrauchseigenschaften aktiviert?

d) Gibt es eine Teilfunktion, durch die besonders viele Objekte in den Prozess
einbezogen und aktiviert werden?

e) Gibt es Objekte, welche durch mehrere Teilfunktionen auf unterschiedliche
Weise in den Prozess einbezogen werden?

f) Durch welche notwendigen Funktionseigenschaften lässt sich die
prozessgerechte Wirkungsweise und durch welche Struktureigenschaften lässt sich
der erforderliche Aufbau und die zweckmäßige Anordnung der einzelnen Objekte
(technischen Mittel) kennzeichnen?

2.4 Welche für den Nutzungsprozess gemäß 2.3 geeigneten technischen Mittel sind
zu den einzelnen Objektarten gemäß 2.2 auf dem Stand der Technik verfügbar oder
bekannt?

a) Gibt es ein technisches System auf dem internationalen Stand der Technik,
welches die notwendigen Eignungsmerkmale gemäß Z1, Z3, Z4 prinzipiell besitzt?
Ist dieses System verfügbar? Wähle dieses System als Referenzvariante, auch
wenn es nicht auf dem gewählten technisch-technologischen Prinzip beruht.

b) Welche der einzelnen erforderlichen Mittel gemäß 2.3.f) gibt es auf dem
Stand der Technik? Welche sind verfügbar? Welche sind machbar?

c) Welche gemäß 2.3.f) notwendigen technischen Mittel sind auf dem Stand der
Technik weder verfügbar noch bekannt?

d) Wie wären technische Mittel gemäß 2.4.c) auf dem Stand der
Technikwissenschaften denkbar?

e) Welche der in Betracht gezogenen technischen Mittel lassen sich aufgrund
ihrer Mittel-Wirkungs-Beziehungen miteinander zu einer Basisvariante
verknüpfen? (Eventuell morphologisches Schema)

f) Welche funktionalen Anforderungen, strukturellen Bedingungen sowie
naturgesetzmäßigen Einflüsse und Restriktionen (\emph{aber}) sind dabei zu
berücksichtigen?
	
g) Bei welchen neuartigen technischen Mitteln treten demgemäß die meisten
Unvereinbarkeiten auf? Bei funktionsbestimmenden oder bei untergeordneten
technischen Mitteln?

h) Worin bestehen diese Unvereinbarkeiten? Lassen sie sich durch Verlagerung
auf untergeordnete technische Mittel und geeignete Variation ihrer Funktions-
und Struktureigenschaften beheben?

2.5 Welche technisch-ökonomischen Mängel bzw. technisch-ökonomischen Defekte
besitzt die mit bekannten technischen Mitteln bestenfalls erreichbare
Basisvariante?

a) In welchen technisch-technologischen Eignungsmerkmalen weicht die bevorzugte
Basisvariante von der Zielgröße voraussichtlich am stärksten ab?

b) In welchen technisch-ökonomischen Hauptleistungsdaten weicht sie
voraussichtlich von der Sollgröße am stärksten ab?

c) In welchen Eignungs- und Effektivitätsmerkmalen ist die Basisvariante der
Referenzvariante prinzipiell überlegen?

d) Welche neuen technischen Mittel sind notwendig und denkbar, um mit der
Basisvariante die Sollgröße zu erreichen und die Referenzvariante in allen
Hauptleistungsdaten zu übertreffen?

e) Wie lautet die technisch-ökonomische Zielstellung der notwendigen
technischen Entwicklung?

f) Welcher Hauptleistungsparameter liegt ihr als Führungsgröße zugrunde?

2.6 Fasse das technische System, das die Zielgröße realisieren soll, insgesamt
als Black Box auf. Mit welchen Eingängen und Ausgängen realisiert das
technische System in der gewählten bzw. vorgefundenen Ausführungsform das
spezielle gesellschaftliche Bedürfnis?  Beschreibe die Ein- und Ausgangsgrößen
in auftragsgemäßen Bestimmungen der Art, der Zusammensetzung, der Struktur und
des Zustandes von Stoff, Energie und Information.

2.7 Durch welches Verfahrensprinzip wird bei der gewählten Basisvariante die
zweckbestimmte Eingangs/Ausgangsrelation (Überführungsfunktion) realisiert?

a) Nenne die funktionellen Merkmale der wesentlichen Teilsysteme zur
Realisierung der Hauptfunktion!

b) Nenne die dabei zu erzielenden notwendigen Zwischenstadien der
Eingangs-Ausgangs-Transformation der Zustandsgrößen von Stoff, Energie und
Information.

2.8 Welche technischen Wirkprinzipe liegen bei der Basisvariante der
Hauptfunktion zugrunde?

a) Untersuche das technische Wirkprinzip jeder Elementarfunktion:
\begin{itemize}
\item Durch welchen Operator soll welche Einwirkung (welche Operation) auf
  welches Objekt (Operand) ausgeübt werden?
\item Welche Rückwirkung (Gegenoperation) ist dafür erforderlich und durch
  welchen Gegenoperator wird sie hervorgerufen?
\item Welche Auswirkungen ergeben sich aus dem Zusammenwirken von Operator und
  Gegenoperator und wie wird sie hervorgerufen?
\item Welche Auswirkungen ergeben sich aus dem Zusammenwirken von Operator und
  Gegenoperator in dem zu verändernden Objekt?
\end{itemize}

b) Kennzeichne Art und Weise der konstruktiven bzw. verfahrenstechnischen
Verknüpfung der (elementaren) Funktionseinheiten zur Struktureinheit der
Hauptfunktion.

c) Konfrontiere die technische Wirksamkeit – den Funktionswert – der einzelnen
Elementarfunktionen und der Hauptfunktion als Ganzes mit den Anforderungen an
die Gebrauchseigenschaften des technischen Systems.

2.9 Enthält das technische System für den vorgesehenen Verwendungszweck
überflüssige Elementarfunktionen?

2.10 Welche Nebenwirkungen der Hauptfunktion treten auf bzw. sind bei
vorgesehenen technisch-technologischen Maßnahmen zu erwarten?

a) Untersuche die einzelnen Elementarfunktionen in der Wirkungskette der
Hauptfunktion und die ihnen zugrunde liegenden Wirkprinzipe auf technisch
und/oder naturgesetzlich bedingte Nebenwirkungen.

b) Unterscheide dabei nützliche, verfügbare und schädliche, zu unterdrückende
Nebenwirkungen.

2.11  Wodurch sind die Nebenwirkungen verursacht?

a) Nenne die konstruktiv bzw. technologisch und die naturgesetzlich
determinierten Anforderungen, Bedingungen, Einflüsse und Restriktionen
(\emph{aber}), auf Grund derer die Nebenwirkungen entstehen bzw. nicht ohne
Weiteres unterdrückt werden können.

Diese \emph{aber} ergeben sich für ein technisches Gebilde aus den
technisch-konstruktiven Merkmalen seines Aufbaus und/oder den
technisch-technologischen Merkmalen seiner Herstellung, und für ein technisches
Verfahren aus den technisch technologischen Merkmalen seines Ablaufs und/oder
den technisch konstruktiven Merkmalen des Aufbaus des mit ihm herzustellenden
technischen Gebildes.

b) Konfrontiere die Nebenwirkungen und den Grad ihrer Nutzung
bzw. Unterdrückung mit den gesellschaftlich-ökonomischen Anforderungen,
Restriktionen, Erwartungen und Bedingungen (ABER), welche sich aus den
übergreifenden gesellschaftlichen Bedürfnissen ergeben.

c) Ermittle die nachteiligste Nebenwirkung.

2.12 Gibt es technische Mittel (Operatoren) zur Realisierung der Hauptfunktion,
für die international bereits andere Wirkprinzipe genutzt werden?

2.13 Welche Anforderungen, Bedingungen und Restriktionen
gesellschaftlich-ökonomischer, technisch-technologischer und/oder
schutzrechtlicher Art behindern die Einführung international bekannter Lösungen
in das technische System?

2.14 Untersuche die Funktionseinheiten des technischen Systems, ob sie
Nebenfunktionen enthalten, die geeignet sind, Nebenwirkungen besser nutzbar zu
machen oder schädliche Nebenwirkungen zu unterdrücken oder sogar in nützliche
zu verwandeln.

2.15 Welches Verhalten des technischen Systems ist zu erwarten, wenn Werte der
technisch ökonomischen Parameter erhöht werden?

a) Variiere die Werte jedes einzelnen technisch-ökonomischen Parameters gemäß
technisch-ökonomischer Zielstellung bis an die im Auftrag geforderten
Grenzwerte und darüber hinaus. Beachte dabei die Rangfolge in der
gesellschaftlich-ökonomischen Wichtung der Parameter der Zielgröße Z.

b) Untersuche, welche technischen Systemparameter-Leitgrößen (Führungsgröße,
Strukturgröße, Wirkgröße) dazu in welcher Richtung und in welchem Maße
verändert werden müssten.

c) Untersuche, ob dann die technisch-technologische Wirksamkeit der einzelnen
Elementarfunktionen in der Wirkungskette der Hauptfunktion den ABER gemäß
gewährleistet bleibt, oder ob schädliche Effekte und damit
technisch-ökonomische Widersprüche entstehen.

d) Untersuche, welche Elementarfunktion auf Grund ihres Wirkprinzips und/oder
auf Grund der vorliegenden \emph{aber} die Verbesserung der Parameterwerte
primär begrenzt.

e) Untersuche, wie sich das Verhältnis von Haupt- und Nebenwirkungen (bezogen
auf jede einzelne Elementarfunktion und auf das gesamte technische System) mit
der Variation der technisch ökonomischen Parameter verändert. Stelle fest, ob
die schädlichen Nebenwirkungen durch die Nutzung vorhandener Nebenfunktionen
besser beherrscht werden können

\subsection{Das Operationsfeld des Erfinders}

3.1 Welche Teilsysteme (Baugruppen, Bauteile, Verfahrensstufen,
Verfahrensschritte) des technischen Systems sind auf Grund von
gesellschaftlich-ökonomischen Restriktionen und technisch-technologischen
Bedingungen einer Veränderung nicht zugängig und daher der
technisch-technologischen Umgebung zuzuordnen?

3.2 Welche stofflichen, energetischen und/oder informellen Komponenten des
technisch-technologischen Umfeldes können bzw. müssen in die Systembetrachtung
mit einbezogen werden?

Untersuche, ob es bestimmte Komponenten der technisch-technologischen Umgebung
des Systems oder sogar des gesellschaftlichen Obersystems gibt, die als
Operatoren in der Wirkungskette der Hauptfunktion oder die im Sinne der
Nebenfunktion genutzt werden können.

3.3 Grenze das technische System bzw. die entsprechende Basisvariante auf die
Fragen 3.1 und 3.2 neu ab. Bestimme seine Aus- und Eingangsgrößen, seine
Hauptfunktion sowie die ihm zugehörigen technisch-technologischen Bedingungen
entsprechend neu.

3.4 Welches Teilsystem stellt für die Erhöhung der Werte der
technisch-ökonomischen Parameter im Sinne des gesellschaftlichen Bedürfnisses
eine primäre Barriere dar?

a) Stelle fest, zu welchem Teilsystem (Baugruppe, Bauteil), Verfahrensstufe,
Verfahrensschritt) das effektivitätsbegrenzende technische Mittel (Operator)
gehört bzw. in welchem Teilsystem sich das Verhältnis von Haupt- und
Nebenwirkung bei Erhöhung der Werte von Haupt- und Nebenwirkung bei Erhöhung
der Werte von technisch-ökonomischen Parametern am stärksten zu Ungunsten der
Hauptwirkung verändert.

b) Bestimme die Ein- und Ausgangsgrößen dieses Teilsystems und stelle die
Wirkungskette seiner Elementarfunktion dar.

\subsection{Der technisch-ökonomische Widerspruch}

4.1 Untersuche, wie die technisch-ökonomischen Parameter der Zielgröße bei dem
in Betracht gezogenen Stand der Technik (Basisvariante) durch das ihnen
zugrunde liegende System der technisch-technologischen Parameter der
Basisvariante miteinander verknüpft sind. Bestimme den
technisch-technologischen Parameter des technischen Systems, der den stärksten
Einfluss auf die technisch-ökonomische Effektivität gemäß der Zielgröße hat.
Wähle ihn als Führungsgröße.

4.2 Lässt sich durch Variation der Werte der Führungsgröße das erforderliche
Wachstum aller technisch-ökonomischen Parameter erzielen? Oder ist das
erforderliche Wachstum einzelner Parameter nur bei Abnahme anderer
technisch-ökonomischen Parameter erreichbar?

a) Stelle die Entwicklung der technisch-ökonomischen Effektivität des zu
betrachtenden technischen Systems als Funktion der Verbesserung seiner
technisch-ökonomischen Parameter dar. Gewährleiste, dass dabei die Interessen
der Volkswirtschaft insgesamt zum Ausdruck kommen.

b) Zeige, dass unter dem Gesichtspunkt der zu steigernden Effektivität die
Entwicklung technisch-ökonomischer Parameter widersprüchlich geworden ist
(Widersprüche zwischen Parametern hinsichtlich ihres Beitrags zur
Effektivitätssteigerung und Widersprüche zwischen den Konsequenzen der
Entwicklung des einen oder anderen Parameters):
\begin{itemize}
\item Nenne die Parameter, deren Einfluss auf das Effektivitätswachstum sich
  zunehmend spaltet in einander entgegengesetzte Einflüsse (Innerer Widerspruch
  zu der Entwicklung eines technisch-ökonomischen Parameters).
\item Nenne die Paare von technisch-ökonomischen Parametern, deren Entwicklung
  derart voneinander abhängig ist, dass Verbesserung der Werte des einen
  Parameters zwangsläufig zur Verschlechterung der Werte des anderen
  führt. (Äußere Widersprüche zwischen technisch-ökonomischen Parametern).
\end{itemize}

c) Untersuche, ob sich die widersprüchliche Entwicklung eines Parameters oder
Parameterpaares besonders ungünstig auf die auftragsgemäße
Effektivitätsentwicklung auswirkt.

4.3  Warum ist das technisch-ökonomische Problem besonders jetzt aktuell?

a) Versuche, Dir einen Überblick über die zurückliegende technisch-ökonomische
Entwicklung und ihre Ursachen zu verschaffen, soweit sie das zu betrachtende
technische System betrifft.

b) Versuche, von dorther die Notwendigkeit der technisch-ökonomischen
Zielstellung (die objektive technisch-ökonomische Problemlage) als Resultat
einer Zuspitzung zu verdeutlichen.  Zeige, dass diese Abflachung der
Entwicklungskurve der technisch–ökonomischen Effektivität des betrachteten
technischen Systems vorliegt oder zu erwarten ist.

Zeige, dass diese Abflachung auf die Wirkung eines maßgebend hervortretenden
technisch–ökonomischen Widerspruchs und auf Abnahme der Möglichkeiten zu einer
Abschwächung durch kompromissbildende, optimierende Maßnahmen (Auslegungen,
Dimensionierungen) zurückzuführen ist.

c) Prüfe, welche ABER in Zukunft noch an Gewicht gewinnen. Prüfe, ob auf Grund
des wissenschaftlich-technischen Fortschritts im technisch-technologischen
Umfeld hinsichtlich einiger Widersprüche in Kürze mit Entspannung statt
Zuspitzung zu rechnen ist.

4.4 Welcher der ermittelten Widersprüche hat eine Schlüsselstellung für die
Lösung bzw. Abschwächung aller anderen Widersprüche? Formuliere den
technisch-ökonomischen Hauptwiderspruch.

\subsection{Der schädliche technische Effekt (stE)}

5.1. Welche der Beziehungen zwischen Führungsgröße und dem System der
technisch-technologischen Parameter ist für die Entstehung der
technisch-ökonomischen Hauptwidersprüche die entscheidende?

a) Welcher technisch-ökonomische Parameter würde sich bei zielgemäßer Variation
der gewählten Führungsgröße rückläufig verhalten? Gibt es mehrere solcher
Parameter? Beschreibe diesen unerwünschten technisch-ökonomischen Effekt
einerseits nach der Art des rückläufigen technisch-ökonomischen Parameters.
Beschreibe ihn andererseits nach den kausalen technischen Zusammenhängen, die
zwischen dem Verhalten des rückläufigen technisch-ökonomischen Parameters und
der Führungsgröße bestehen.

b) Schließe hieraus auf den kritischen Bereich der Entwicklung des technischen
Systems („Entwicklungsschwachstelle“, kritischer Funktionsbereich), in dem
derjenige schädliche technische Effekt (stE) erzeugt wird, in dessen Folge der
unerwünschte technisch-ökonomische Effekt in Erscheinung tritt.

c) Wie würden sich die Verhältnisse bei Wahl einer anderen Führungsgröße
ändern?

5.2 Ist es im Wesentlichen ein Teilsystem, das die „Entwicklungsschwachstelle“
enthält?  Beschreibe die Kette derjenigen Wirkungen, die von strukturellen
und/oder funktionellen Eigenschaften dieses Teilsystems ausgehend den
unerwünschten technisch-ökonomischen Effekt hervorrufen. Gehe aus von den
durchgeführten Systemanalysen (besonders Abschnitt 2.8, bezogen auf den
kritischen Funktionsbereich).

5.3 Sind es zwei oder mehr Teilsysteme, auf deren Zusammenwirken der
technisch-ökonomische Widerspruch zurückgeführt werden kann? Beschreibe die
Kette derjenigen Wirkungen, die von Eigenschaften dieser Teilsysteme und deren
Kopplung ausgehen und einen unerwünschten technisch-ökonomischen Effekt
entstehen lassen. Gehe aus von den durchgeführten Systemanalysen (besonders
Abschnitt 2.8, bezogen auf den kritischen Funktionsbereich.

5.4 Ist der schädliche technische Effekt (stE) mit vorhandenen Mitteln
behebbar? Prüfe, ob er vielleicht durch Betriebsblindheit entstanden war.
Garantiere, dass die etwaige Inanspruchnahme von Mitteln zur Behebung der
Entwicklungsschwachstelle und damit des schädlichen technischen Effekts (stE)
nicht dazu führt, dass ein anderer ins Gewicht fallender Effekt entsteht, der
den ABER gemäß unzulässig ist. Setze dann oder im Zweifelsfall die Analyse
gemäß Abschnitt 6 fort.

\subsection{Das IDEAL, Anstoß und Orientierung zu vertiefter Systemanalyse}

6.1 Welches Verhalten oder welche Eigenschaften müsste das Teilsystem oder
müssten die Teilsysteme aufweisen, damit ein schädlicher technischer Effekt im
technischen System nicht auftritt?

a) Stelle Dir das Teilsystem oder die Teilsysteme, von denen der stE ausgeht,
in ihrem Verhalten und/oder in ihrem Zusammenwirken so vor, dass seine (ihre)
schädlichen Auswirkungen auf technisch-ökonomische Parameter des Systems nicht
mehr auftreten. Lass dabei Strukturen und Wirkprinzipien zunächst unverändert.

b) Nenne die Voraussetzungen, die bestehen müssten, damit das ideale Teilsystem
oder das ideale Zusammenwirken zustande kommen kann. Diese vorgestellten
Voraussetzungen können technischer, technologischer oder naturgesetzlicher Art
sein.  Beachte, dass jenes ideale Teilsystem oder das ideale Zusammenwirken
vorerst nur als fiktive Anordnung zur Verhinderung des stE gedacht ist.  Nimm
im Augenblick keinen Anstoß daran, dass unter den Idealvorstellungen die
Hauptfunktion und/oder die Herstellung des technischen Systems aus
technisch-naturwissenschaftlicher Sicht vorerst infrage gestellt sein wird.

6.2 Welche \emph{aber} stehen den Idealvorstellungen im Wege? Sind sie im
Hinblick auf die Hauptfunktion und/oder die Herstellung des technischen Systems
(Basisvariante) irreal? Warum?

a) Nenne die für das Funktionieren oder das Herstellen des technischen Systems
wichtigen Eigenschaften, welche die für das Funktionieren oder das Herstellen
des technischen Systems notwendigen technischen Anforderungen und Bedingungen
sowie naturgesetzlichen Einflüsse und Restriktionen (\emph{aber}) beschreiben,
die sich im Widerspruch zu den Idealvorstellungen befinden.

b) Versuche jetzt, die gemäß 6.1.b vorgestellten Voraussetzungen so zu denken,
als wären sie real, wiederum ohne dabei auch nur in Gedanken an den
Wirkprinzipien oder Strukturen des technischen Systems etwas zu ändern.
Schwäche nun in Gedanken die Idealvorstellungen schrittweise so weit ab, dass
der schädliche technische Effekt gerade noch nicht in Erscheinung tritt.

c) Stelle fest, ob und wieso auch die auf diese Weise modifizierten
Idealvorstellungen immer noch mit der Funktion und/oder Herstellung des realen
technischen Systems (der Basisvariante) unvereinbar sind.

d) Kennzeichne diese systemspezifische Unvereinbarkeit unter Hinweis auf
Zusammensetzung, Struktur und/oder Information sowie unter Hinweis auf
Naturgesetze, die für die Wirkung der systemspezifischen Stoffeigenschaften
und/oder Wirkprinzipe relevant sind.

6.3 Welche Eigenschaften des technischen Systems treten in erster Linie als
eine Störung des Ideals in Erscheinung?

a) Benenne die störenden Merkmale der Entwicklungsschwachstelle, die dem Ideal
des Systems entgegenstehen, oder zwei sich gegenseitig ausschließende ideale
Eigenschaften.

b) Bilde in sich widersprüchliche, paradoxe Begriffe, welche die störende reale
Eigenschaft und die erstrebte ideale Eigenschaft oder die sich gegenseitig
ausschließenden idealen Eigenschaften des technischen Systems in einer
semantischen Einheit zum Ausdruck bringen. (Z.B. schwingende Ruhe, schreiende
Stille, sprunghafte Beharrlichkeit, rasendes Rendezvous). Versuche hierzu das
technische System in seinen phänomenologischen Eigenschaften zu
personifizieren, ihm einen „Willen“ oder eine „Absicht“ zu unterstellen, als ob
es dem Idealzustand trotz seiner Unzulänglichkeit zustrebe.

6.4 Lässt sich das technische System so denken, dass es im idealen Endresultat
den schädlichen technischen Effekt in der „Entwicklungsschwachstelle“ von
selbst ohne Aufwand beseitigt bzw. das Ideal von selbst, ohne Aufwand,
erreicht?

a) Formuliere das ideale Endresultat.

b) Versuche zu erreichen, dass die Formulierung des idealen Endresultats die
Wörter „von selbst“ enthält. Versuche, hierzu verfügbare Nebenfunktionen
auszunutzen.

6.5 Entspricht die Formulierung des idealen Endresultats den zuvor
herausgearbeiteten \emph{aber} gemäß 6.2.b)?  Gewährleiste, dass die
Vorstellung des idealen Endresultats die (vorgestellte) Lösung des
technisch–ökonomischen Widerspruchs ergibt.

\subsection{Der technisch-technologische Widerspruch (ttW)}

7.1 Welche Struktureigenschaften und/oder Wirkprinzipe des technischen Systems
sind im Verlaufe seiner historischen Entwicklung zunehmend zur Grundlage des
Widerspruchs zwischen dem Ideal und den ihm entgegenstehenden Systemmerkmalen
(vgl. 6.3) geworden?

a) Gehe aus von den gemäß Abschnitten 2.1--6. festgestellten strukturellen,
funktionellen und phänomenologischen Eigenschaften des technischen Systems, die
im Hinblick auf die technisch-ökonomische Zielstellung schädlich sind. Suche
sie zu verstehen als Ergebnis eines historischen Prozesses, der durch die
Grundstruktur und das funktionstragende Wirkprinzip des technischen Systems
einerseits und die langzeitige Effektivitätsentwicklung andererseits bestimmt
war.

b) Unterscheide dabei Phasen der nur dimensionierenden Ausreifung und des
Eintretens in qualitativ neue Entwicklungsphasen.

c) Untersuche, in welchem Maße bzw. auf welche Weise die einzelnen Teilsysteme
an dieser Entwicklung beteiligt waren.

d) Stelle fest, ob hierbei Disproportionen im Zusammenwirken von Teilsystemen
oder im Verhältnis von Haupt- und Nebenwirkungen bzw. Haupt- und
Nebenfunktionen hervorgerufen und natürliche Grenzen des Wirkprinzips einer
oder mehrerer der bereits als kritisch erkannten Teilfunktionen (Teilsysteme)
erreicht worden sind.

e) Formuliere das technisch-wissenschaftliche Problem, das dem Auftrag zugrunde
liegt. Beantworte, warum es erst jetzt aktuell geworden ist und früher nicht in
Erscheinung trat.

7.2 Auf welchem technisch-technologischen Widerspruch beruht das technische
Problem?

a) Interpretiere den schädlichen technischen Effekt (stE) als die technische
Folge derjenigen technisch-naturgegebenen Anforderungen, Bedingungen, Einflüsse
und Restriktionen (\emph{aber}) im entwicklungsbedingt kritischen
Funktionsbereich (in der „Entwicklungsschwachstelle“) des technischen Systems,
die eine Realisierung des Ideals objektiv ausschließen. Berücksichtige dabei
6.3.

b) Entwickle eine technisch-naturwissenschaftliche Modellvorstellung oder
Arbeitshypothese zur Erklärung derjenigen \emph{aber}, die dem Ideal
entgegenstehen. Überprüfe zunächst im Gedankenexperiment Deine Vorstellung über
die kausalen Zusammenhänge, die dem schädlichen technischen Effekt zugrunde
liegen. Bestimme die Prämissen und Ungewissheiten Deiner Modellvorstellung und
leite daraus ein Versuchsprogramm zu ihrer experimentellen und/oder
theoretischen Prüfung ab. Achte vor allem auf Ergebnisse, die Deinen bisherigen
Vorstellungen oder der fachgerechten Erwartung widersprechen. Füge sie logisch
widerspruchsfrei in Deine Modellvorstellung ein. Überprüfe zuvor, dass sie
nicht auf einem experimentellen oder mathematischen Irrtum beruhen.

Suche zu erkennen, ob ein technisch-technologischer Widerspruch objektiv
vorhanden ist. Entsteht bei dem Versuch, einen schädlichen technischen Effekt
zu beheben, auf Grund der \emph{aber} ein anderer?

7.3 Welche einander ausschließenden, aber notwendigen Forderungen bilden den
technisch-technologischen Widerspruch? Formuliere den technisch-technologischen
Widerspruch (den Kern des technischen Problems) als Verhältnis der beim
Verfolgen des idealen Endresultats sich einander erfordernden und gleichzeitig
einander ausschließenden Komponenten des technischen Systems.

Diese im Verhältnis des dialektischen Widerspruchs stehenden Komponenten können
sein:
\begin{itemize}
  \item zwei technisch-technologische oder konstruktive (geometrische)
    Eigenschaften eines Bauelements, das in zwei unterschiedliche
    Teilfunktionen eingebunden ist;
  \item je eine technisch-technologische und/oder konstruktive Eigenschaften
    von zwei unterschiedlichen Bauelementen eines Teilsystems;
  \item je eine funktionelle und/oder strukturelle Eigenschaft zweier
    miteinander unmittelbar verketteter Teilsysteme;
  \item Haupt- und Nebenwirkung eines Teilsystems;
  \item Hauptwirkung und Nebenwirkung verschiedener Teilsysteme;
  \item Hauptfunktionen zweier Teilsysteme;
  \item eine Nebenfunktion und die Hauptfunktion des Systems.
\end{itemize}
Stelle den strukturellen und/oder funktionellen Erfordernissen der einen
Komponente die entsprechenden Erfordernisse der anderen Komponente gegenüber.
Nimm diese in die Formulierung des technischen Widerspruchs auf.

7.4 Stehen die Mittel zur Aufhebung des technischen Widerspruchs problemlos zur
Verfügung?

Prüfe, ob das Bestehen des technisch-technologischen Widerspruchs auf einem
Vorurteil der Fachwelt zurückführbar ist.

\subsection{Der technisch–naturgesetzmäßige Widerspruch (tnW)}

8.1 Welche naturgesetzlichen Unvereinbarkeiten werden sichtbar, wenn versucht
wird, den technisch-technologischen Widerspruch verschwinden zu lassen?

a) Formuliere das naturgemäße (physikalische, chemische biologische ) Verhalten
oder die naturgemäßen stofflichen Eigenschaften oder die geometrische Struktur
jeder der beiden Komponenten des technisch-technologischen Widerspruchs fiktiv
so,
\begin{itemize}
  \item dass ihre naturgesetzliche Unvereinbarkeit mit der jeweils anderen
    Komponente prägnant hervortritt
  \item und zugleich so, dass die sowohl-als-auch-Realisierung beider
    Komponenten das ideale Endresultat – zunächst fiktiv – bedeuten würde.
\end{itemize}
b) Beschreibe den technisch-naturgesetzlichen Widerspruch als ein Paar sich
ausschließender Forderungen an
\begin{itemize}
\item naturgemäße Eigenschaften eines Stoffes,
\item Naturvorgänge und/oder Naturzustände,
\item geometrische Strukturen.
\end{itemize}

8.2 Welche naturwissenschaftlich beschreibbare Paarung einander
entgegengesetzter Wirkungen wurde bisher ignoriert, die den
technisch-technologischen Widerspruch gesetzmäßig hervorruft und im jetzt
erreichten Entwicklungsstadium maßgebend in Erscheinung treten lässt?

Versuche, Dir retrospektiv die technische Problemsituation und die technischen
Möglichkeiten und Erkenntnisse zu ihrer Bewältigung, aber auch mögliche
Vorurteile der Fachwelt zum Zeitpunkt der Entstehung des Widerspruchs
vorzustellen und die allmähliche Zunahme seiner Wirkung bis zur jetzt
erreichten Grenze der Kompromissmöglichkeiten zu verfolgen.

Mache Dir klar,
\begin{itemize}
\item dass dieser Widerspruch historisch bedingt, d.h. auf dem jeweils
  erreichten Stand der Technik in das technische System „implantiert“ worden
  ist, ohne dass dadurch die zu jenem Zeitpunkt erforderliche Effektivität
  nachteilig beeinflusst wurde; 
\item dass erst jetzt im erreichten Entwicklungsstadium die Überwindung dieses
  Widerspruchs zwingend erforderlich ist, um die weitere
  Effektivitätsentwicklung zu ermöglichen.
\end{itemize}

8.3 Sind während der letzten Jahre naturgesetzmäßige Effekte bekannt geworden,
die einzeln oder in Verkettung eine direkte Realisierung des idealen
Endresultats ermöglichen?

a) Suche solche Effekte, aber garantiere, dass die etwaige Inanspruchnahme von
Mitteln zu ihrer Nutzung nicht gegen die gemäß 1.5 und 1.7 begründeten ABER
verstößt und der zu lösende Widerspruch nicht lediglich durch einen anderen
ausgetauscht wird.

b) Signalisiere (unter Wahrung der Vertraulichkeit) im Kombinat und möglichst
durch eine Patentanmeldung die aufgedeckten Möglichkeiten zur Nutzung
naturgemäßer Effekte, auch wenn deren Verwendung auf Grund von 1.5 und 1.7 im
Augenblick nicht möglich ist.

\subsection{Die Strategie zur Widerspruchslösung}

9.1 Wo in seiner Struktur ist die Möglichkeit zur Auflösung des
technisch-technologischen Widerspruchs enthalten?

Gehe davon aus, dass die Lösung des technisch-technologischen Widerspruchs in
einer Veränderung
\begin{itemize}
  \item des Verfahrensprinzips (Blockschemas) und des räumlichen und/oder
    zeitlichen Aufbaus eines komplexen Teilsystems
    („Entwicklungsschwachstelle“, entwicklungsbedingter kritischer
    Funktionsbereich)
  \item des Funktionsprinzips, der räumlichen und/oder zeitlichen Anordnung
    eines elementaren Teilsystems (kritische Wirkstelle innerhalb des
    kritischen Funktionsbereichs) zu suchen ist. Führe die weiteren
    Überlegungen ausgehend von dem Denkniveau, das nach dem Herausarbeiten des
    technisch–technologischen und des technisch-naturgesetzlichen Widerspruchs
    bestimmt ist durch
  \item Abstraktion von konstruktiven bzw. technologischen Details;
  \item Konkretisierung im Sinne der exakten, zugeschärften Kennzeichnung des
    zu lösenden Widerspruchs.
\end{itemize}

9.2  Untersuche den technisch-technologischen Widerspruch bezüglich
\begin{itemize}
  \item seiner gegensätzlichen Komponenten und der technisch-naturgesetzlichen
    Art und Weise, in der sie sich ausschließen;
  \item der technisch-technologischen Erfordernisse, auf Grund derer die
    gegensätzlichen Komponenten sich gegenseitig hervorbringen und eine Einheit
    bilden;
  \item der Sachverhalte im System, welche die Trennstelle bestimmen, an der
    sich die Komponenten des technisch-technologischen Widerspruchs
    gegenüberstehen.
\end{itemize}
„Trennstelle“ ist nicht unbedingt räumlich-geometrisch, sondern oft im
übertragenen Sinne (als Scheide zwischen gegensätzlichen Erscheinungen) zu
verstehen.

9.3  Versuche, einen Weg zur Überwindung des Widerspruchs zu finden durch

A) Auftrennen der Einheit der gegensätzlichen Komponenten, z.B. durch
\begin{itemize}
\item [(1)] technologisches Parallelschalten, räumliches Entkoppeln
  (Reißverschlussprinzip 1a);
\item [(2)] zeitliches oder zeitweiliges Entkoppeln (Reißverschlussprinzip 1b)
  bzw. Phasenverschieben der Wirkdauer der Komponenten;
\item [(3)] räumliches oder zeitliches Ineinanderschachteln von Bauelementen
  und/oder Wirkungen (Steckpuppenprinzip) oder Verzahnen (Reißverschlussprinzip
  2)
\item [(4)] Funktionstrennung bzw. Übertragen zweier Elementarfunktionen oder
  funktionswichtiger Elementareigenschaften von einem auf zwei
  Strukturelemente;
\item [(5)] Spaltung einer Struktureinheit (Baugruppe, Prozessstufe) in ein
  wechselwirkendes Paar von Komponenten (z.B. zur Selbstkompensation von
  Störungen);
\item [(6)] Durchlaufen oder Koexistenz verschiedener Gebrauchs- bzw.
  Betriebszustände
\item [(7)] Gewährleistung extremer Geschwindigkeiten (Schock, Abschreckung,
  Kriechen).
\end{itemize}

B) Überwinden der im Widerspruchsverhältnis stehenden Merkmale einer der beiden
Komponenten, z.B. durch
\begin{itemize}
\item [(1)] Auffinden eines anderen, geeigneteren Wirkprinzips (z.B. des Gegen- oder Komplementärprinzips);
\item [(2)] Auffinden oder Schaffen neuer Werkstoffe, evtl. Verbundwerkstoffe;
\item [(3)] Übertragen einer Elementarfunktion oder -eigenschaft auf ein noch
  zu findendes bzw. zu schaffendes (separates) technisches Teilsystem, das die
  nötigen Natureigenschaften, die bisher zum technisch-naturgesetzlichen
  Widerspruch führten, zu geeigneten Zeitpunkten, an geeigneten Orten und unter
  geeigneten Bedingungen hervorbringt;
 \item [(4)] Gezieltes Überschreiten der Werte eines ausgewählten Parameters
   bis zum Qualitätsumschlag (Nutzung einer Nichtlinearität);
 \item [(5)] Verstärken einer Elementarfunktion oder Abschwächen (Unterdrücken)
   ihrer Nebenwirkung durch positive bzw. negative Rückkopplung;
 \item [(6)] Hierarchische Aufteilung einer Funktion auf mehrere
   Funktionsebenen (Segelschiffstakelage).
\end{itemize}

C) Überwinden der gegenseitigen Ausschließung der polaren Komponenten,
z.B. durch
\begin{itemize}
\item [(1)] Strukturelle und funktionale Verschmelzung im Konflikt befindlicher
  Teilsysteme;
  \item [(2)] Vereinigung von zwei gegensätzlichen Elementarfunktionen in einem
    Strukturelement;
  \item [(3)] Überlagern einander entgegengesetzter schädlicher Wirkungen;
  \item [(4)] Einführung einer dritten Komponenten, besonders Erzeugung dieser
    Komponente aus systemeigenen Komponenten;
 \item [(5)] Simulation kennzeichnender Eigenschaften einer der beiden
   Komponenten und Einführung in die jeweils andere Komponente (adaptive
   Maskierung, Prinzip des Trojanischen Pferdes).
\end{itemize}

D)   Nutzung anderen Verfahren gemäß Listen nach Altschuller und anderen.

9.4 Welches Vorgehen ist der Natur des technisch-technologischen Widerspruchs
adäquat?

a) Entwickle eine Lösungsstrategie, indem Du von der Art derjenigen Lösung des
Widerspruchs ausgehst, die an der schwächsten Stelle seiner Struktur angreift.

b) Formuliere die Erfindungsaufgabe in einer möglichst detailfreien, aber
problemspezifischen Weise so, dass sie den technisch-technologischen
Widerspruch und den Auftrag zu dessen Überwindung zum Ausdruck bringt. Erstrebe
eine knappe, zugeschärfte Form.

9.5.  Formuliere die Erfindungsaufgabe so, dass die im System bereits
vorhandenen Eigenschaften weitestgehend selbst die Aufhebung des Widerspruchs
ermöglichen.

\subsection{Die eigene Erfindung – Schrittmacher internationaler Entwicklung}

10.1 Prüfe, ob die Lösung der Erfindungsaufgabe in der internationalen
Entwicklung schrittmachend wirken wird und dazu beiträgt, das Kombinat in
Spitzenposition zu bringen oder seinen technologischen Spielraum zu vergrößern.
Aktualisiere die Analyse von Fach- und Patentliteratur, Markt-, Forschungs- und
Reiseberichten (vgl. 1.3) entsprechend der fortgeschrittenen Zeit.

10.2 Könnte die Lösung der Erfindungsaufgabe auch für andere als die
vorgegebenen Applikationsbereiche relevant sein? (Multivalenz!)

10.3 Konzipiere die rasche Überleitung. Beachte: Ohne Kampf kein Fortschritt.
Hinterfrage hierzu vor allem die Formulierung des Ziels und der Aufgabe der
Erfindung in den Patentbeschreibungen.

10.4 Wie muss der erarbeitete und mit dem internationalen Stand konfrontierte
Lösungsansatz zu einer Innovationsstrategie ergänzt werden?

a) Disponiere die erforderlichen Experimente, Informationsbeschaffungen und
Abstimmungen.

b) Entwirf die Innovationsstrategie so, dass Dein Betrieb sie realisieren kann
und damit selber den internationalen Stand bestimmt oder durch Überwindung von
Engpässen sich hierzu den Spielraum verschafft.

\section{ProHEAL – das Programm in erzählender Darstellung (leicht gekürzt)}

Die ABER-Matrix wird dem aufgeschlossenen Betrachter schon viele Anregungen
vermittelt haben. Also wird er durch Erhöhung von Parametern in einem oder
mehreren Matrix-Feldern erfinderische Lösungen angestrebt und vielleicht auch
gefunden haben. Der aufgeschlossene Betrachter hatte es mit der sogenannten
„Basisvariante“ zu tun, die seinen Überlegungen zugrunde lag. Doch bald schon
kann der aufgeschlossene Betrachter auch auf Hindernisse und Grenzen gestoßen
sein, die ihn am Weiterkommen hinderten. Dann wurde es schwierig.

Genialen Erfindern kann dann trotzdem eine wünschenswerte Lösung gelingen. Doch
der geniale Erfinder arbeitet ja intuitiv. Es wird ihm schon schwerfallen,
seine Denkoperationen nachvollziehbar und jedermann verständlich zu
protokollieren.  Doch Dr.-Ing. Hans-Jochen Rindfleisch, der Hauptautor von
ProHEAL, war auch darin genial, ein Programm zum Herausarbeiten von
Erfindungsaufgaben und Lösungsansätzen zu entdecken und aufzuschreiben. Dazu
musste er mehrere Begriffe konzipieren, um die Schritte, Sprünge und Stationen
zu kennzeichnen, die das Herausarbeiten von Erfindungsaufgaben und
Lösungsansätzen bestimmen, auch wenn sie dem Betrachter nicht klar vor Augen
stehen. Hans-Jochen Rindfleisch war auch ein genialer Analytiker. Ihm zu folgen
ist nicht leicht. Das ProHEAL in seiner algorithmischen Darstellung ist ein
anspruchsvoller Text. Deshalb entwarf Hans-Jochen auch eine erzählende
Darstellung des ProHEAL. Diese sei jetzt (mit einigen Kürzungen) wiedergegeben,
wie schon als Kapitel 3 angekündigt:

Die Versuche, gängige Parameterwerte technischer Objekte zu verbessern (zu
erhöhen), führen zumeist an die Situation heran, in welcher der Ingenieur
innehalten muss, um zu sagen: „Ja, aber was darf dabei nicht geschehen?“. Das
ist eine Frage nach unerwünschten Konsequenzen. Eine Zielstellung des Erfindens
beruht auf möglichst genauer Kenntnis aller möglichen „Ja, aber….“ Zu erstreben
ist bezüglich eines jeden „ja, aber….“ nicht ein „Entweder oder“, sondern ein
„Sowohl als auch“. Der betrachtende Ingenieur fühlt sich nun in einem
Teufelskreis gefangen. Die „ja aber….“ signalisieren, dass ein dialektischer
Widerspruch besteht: Zwei Tendenzen sind einander entgegengerichtet – Kampf der
Gegensätze – jeweils eine Tendenz bringt eine andere zwangsläufig hervor, oft
auch mehrere andere. Die ABER-Matrix – siehe Kapitel 1 -- hilft wahrzunehmen
und zu verstehen, was geschieht. Eben das beginnt mit den oben (Kapitel 1)
signalisierten ABER in der ABER-Matrix. Worauf der Betrachter sich von Beginn
an bezieht ist das Bedürfnis der Gesellschaft, des Herstellers, des Anwenders.
Daraus muss nun eine konkrete Zielstellung entwickelt werden. Sind nach den
ersten Anläufen keine Widerspruchslösungen gefunden worden, müssen
weiterführende Fragen zur Basisvariante gestellt und beantwortet werden.
ProHEAL zeigt Möglichkeiten:

Die obligatorische Zielgröße (entweder vollständig vorgegeben oder vom
verantwortungsbewussten Ingenieur vervollständigt) war in Abschnitt 1.3 (des
ProHEAL) als Ausdruck des Systems der ABER gekennzeichnet worden. Aus der
Zielgröße werden die technisch-ökonomischen Parameter abgeleitet. Dass einige
oder viele von ihnen sich kräftig verbessern sollen (aber keiner sich
verschlechtern soll), ist zunächst nur ein Auftrag oder Wunsch. Sie haben ihr
Wurzelgeflecht im System der technisch-technologischen oder
technisch-naturgesetzlichen Eigenschaften der Basisvariante und sind durch
diese Eigenschaften miteinander verbunden, vernetzt. Aus diesen Verbindungen
ergeben sich im System der ABER zwangsläufig die „ja, aber“, die erfinderisch
-- durch Wandlungen in der Basisvariante -- in „sowohl als auch“ zu überführen
sind. In dieser Zwangsläufigkeit liegt die Schwierigkeit des zielstrebigen,
wirksamen Erfindens.

Die „ja aber“ werden umso heikler und umso akuter,
\begin{itemize}
\item je kräftiger die Effektivität E gesteigert werden soll, weil dann
  Grenzbereiche der nur optimierenden Verbesserung des technischen Objekts
  erreicht werden oder überschritten werden müssen; 
\item je konsequenter die Zielgröße – das System der ABER -- in ihrer
  Komplexität respektiert wird, sodass Verbesserungen hinsichtlich einiger
  Parameter nicht mehr „unter der Hand“ auf Kosten von Verschlechterungen
  anderer Parameter erzielt werden können, die man unzulässigerweise ignoriert.
\end{itemize}
In dieser Situation bringen die „ja, aber“ zum Ausdruck: Achtung! Wenn ich
einen Parameter sehr stark verbessern will, kann sich – zunächst einmal, beim
Stand der Technik – ein anderer verschlechtern, und das darf in der Regel nicht
sein. (Abb. 6) Wenn -- zunächst einmal, beim Stand der Technik – die
Verbesserung eines Parameters einen anderen zwangsläufig beeinträchtigt oder
ausschließt oder gar zur Verschlechterung eines anderen Parameters führt, ist
ein technisch-ökonomischer Widerspruch eingetreten.

In der technisch-ökonomischen Entwicklung können solche Widersprüche eine
zeitlang in Kauf genommen werden. Das ist dann möglich,
\begin{itemize}
\item wenn sie ihrer Bedeutung (ihrem Gewicht) nach nicht gravierend sind, weil
  die im Widerspruchsverhältnis stehenden Parameter (je ein Parameter des
  Widerspruchspaares oder beide) im Gesamtsystem nicht gravierend sind; anders
  gesagt, wenn die komplexe Zielgröße (das System der ABER) eine entsprechende
  Unterscheidung zwischen grundlegenden und weniger wichtigen Parametern
  zulässt.
\item wenn der Widerspruch im Anfangsstadium seiner Entwicklung steht, wo sich
  die Verbesserungen beider Parameter schon gegenseitig beeinträchtigen, aber
  noch nicht gegenseitig ausschließen. In diesem Anfangsstadium kann auf dem
  Stand der Technik durch Optimierung (Kompromissbildung) noch ein günstiger,
  mitunter ausreichend erhöhter Wert für jeden der beiden Parameter gefunden
  werden.
\end{itemize}
Die Erfindungsmethode muss nun das Bewusstsein hervorrufen,
\begin{itemize}
\item wie der zu lösende technisch-ökonomische Widerspruch durch Analyse des
  technischen Objekts – genauer: der Basisvariante – bestimmt wird. Welche
  Zusammenhänge im technischen Objekt sind „verantwortlich“ für die
  Konfliktsituation im System der ABER? (Diese primären, häufig komplexen
  Zusammenhänge sind meist nicht ohne Weiteres entflechtbar. Wie die Analyse
  bis zur Bestimmung des technisch-ökonomischen Widerspruchs getrieben werden
  kann, zeigen die Abschnitte 1 bis 4 des „Erfindungsprogramms“).
\item wie man den noch tiefer im technischen Objekt, in seiner Struktur
  verborgenen Punkt oder sekundären Zusammenhang findet, wo Parameter so
  geändert werden können, dass der primäre Zusammenhang (s.o.), welcher dem
  technisch-ökonomischen Widerspruch zugrunde liegt, unschädlich gemacht werden
  kann. Das führt auf die Frage nach dem technisch-technologischen und evtl.
  nach dem technisch-naturgesetzlichen Widerspruch, der aufgedeckt werden muss,
  wenn der technisch-ökonomische Widerspruch zum Verschwinden gebracht werden
  soll. (Vgl. hierzu die Abschnitte 5 bis 9 des „Erfindungsprogramms“).
\end{itemize}
Das zielstrebige Erfinden ist zu einem beträchtlichen Teil tiefgründige
Analyse. Aus dieser ergeben sich schrittweise die Lösungsansätze: Nicht nur der
Suchraum wird eingegrenzt. Durch die Herausarbeitung der genannten Widersprüche
entsteht vor allem das Bild von der Struktur des zu lösenden Problems. Es
zeichnet sich ab, welche Lösungsstrategie zu wählen ist (vor allem die
Abschnitte 8 und 9 des „Erfindungsprogramms“). Auf diese Weise geht die Analyse
allmählich in die Lösung über. Das beginnt – genau besehen – bereits in
Kapitel 2 des „Erfindungsprogramms“.

\subsection{Die Beziehungen des technisch-ökonomischen zum
  technisch-technologischen und zum technisch-naturgesetzlichen Widerspruch}

Verschaffen wir uns Einblick in den Prozess, der von der Zielgröße ausgehend
immer tiefer in die technisch-technologischen Zusammenhänge im Inneren des
technischen Objekts und in deren naturgesetzliche Bedingtheit hineinführt.
Knüpfen wir dabei an den schon umrissenen Begriff des technisch-ökonomischen
Widerspruchs an.

Zuvor ist jedoch der Begriff „Führungsgröße“ zu definieren. Die Führungsgröße
ist ein systemspezifischer Parameter von zentraler Bedeutung, durch dessen
Variation die Entwicklung der Leistungsfähigkeit und/oder der Effektivität
eines technischen Systems erhöht werden soll. Anders gesagt: Solche Erhöhungen
werden – meist auf Grund langjähriger Erfahrungen -- als Folge einer Variation
der Führungsgröße erwartet. Führungsgröße kann z.B. sein: die Einheitsleistung
eines Großtransformators oder eines großen Generators, die Anzahl der
integrierten Schaltkreise eines Mikroprozessors, die Laststufe eines
Transportsystems, das Anlaufdrehmoment eines Elektromotors (bezogen auf sein
Nennmoment), die Taktzahl einer Werkzeugmaschine, die
Gleichgewichtskonzentration eines chemischen Prozesses, die spezifische
Trennstufenzahl eines Extraktionsverfahrens. (Zum Begriff der Führungsgröße
siehe auch Abschnitt 2.3.4 dieses Lehrbriefs.)

Ein technisch-ökonomischer Widerspruch (töW) besteht darin, dass bei der
Variation eines für das Erreichen dieses höheren volkswirtschaftlichen Effekts
entscheidenden technischen Parameters – der Führungsgröße GF – zumindest zwei
wichtige technisch-ökonomische Effektivitätsparameter Etö1 und Etö2 des
technischen Objekts in ihrem Verhalten zueinander gegenläufig werden.

Betrachten wir zum Beispiel die Entwicklung eines Containers. Als Führungsgröße
GF wird die Wandstärke des Containers im Verhältnis zu seinen Kantenlängen
gewählt. Durch Verringerung der Wandstärke werden zwei wesentliche
technisch-ökonomische Effektivitätsparameter des Containers günstig
beeinflusst: der spezifische Materialeinsatz, bezogen auf das Containervolumen,
und das Nutzlastverhältnis, d.h. das Verhältnis von zuladbarer Masse zur
Eigenmasse des Containers. Wir können daher beide Parameter zu dem
technisch-ökonomischen Effektivitätsparameter Etö1 zusammenfassen. Der sich
dazu gegenläufig verhaltende technisch-ökonomische Effektivitätsparameter Etö2
ist die spezifische Belastbarkeit des Containers, d.h. die zuladbare Last im
Verhältnis zur Laststufe des infrage kommenden Transportsystems. Diese wird
sich bei Unterschreiten einer bestimmten Grenzwandstärke des Containers so
stark verringern, dass dadurch die ökonomischen Vorteile der Materialeinsparung
und der geringeren Eigenmasse des Containers bei weitem aufgewogen werden.
Hinzu kommt bei abnehmender Wandstärke eine höhere Anfälligkeit des Containers
gegen Korrosion und mechanische Beschädigung. Eine weitere Verminderung der
spezifischen Wandstärke unter den charakteristischen Grenzwert ruft somit einen
kritischen technisch-ökonomischen Widerspruch hervor.

Ein technisch-technologischer Widerspruch (TTW) besteht darin, dass bei der
Variation des für die Ausprägung eines funktionstragenden technischen Effekts
vorrangig bestimmenden technischen Parameters – der Strukturgröße GS –
mindestens zwei maßgebende technisch-technologische Effektivitätsparameter Ett1
und Ett2 des technischen Objekts in ihrem Verhalten zueinander gegenläufig
werden.

Im Beispiel des Containers ist mit Blick auf die Überwindung des
technisch-ökonomischen Widerspruchs zunächst die Behälterfunktion als
Strukturgröße GS in betracht zu ziehen. Durch sie wird ein funktionstragender
technischer Effekt entscheidend ausgeprägt, nämlich die Verteilung der
Lastkräfte und der Art ihrer Wirkung (in Form von Druck-, Zug-, Schub- und/oder
Biegespannungen) in Relation zur örtlichen Festigkeitsverteilung in der Wandung
des Containers. Auf der Wirkung dieses technischen Effekts beruht einer der
wesentlichen Effektivitätsparameter des Containers, seine spezifische, d.h. auf
sein Volumen bezogene Tragfähigkeit. Wir bezeichnen ihn mit Ett1. Durch
geeignete Formgebung der Behälterwandung kann der technische Effekt so zur
Wirkung gebracht werden, dass sich Materialeinsparung und höheres
Nutzlastverhältnis nicht mehr in einen technisch-ökonomischen Widerspruch mit
der spezifischen Belastbarkeit des Containers befinden müssen.

Gelingt dies nicht, so liegt ein technisch-technologischer Widerspruch vor.
Dieser kommt dadurch zustande, dass mit der Behälterform als Strukturgröße GS
nicht allein die Tragfähigkeit, sondern auch das spezifische Nutzvolumen, d.h.
der durch das Ladegut ausfüllbare Anteil des Containervolumens sowie seine
Zugänglichkeit, d.h. die Be- und Entladbarkeit des Containers, maßgeblich
bestimmt werden. Diese Eigenschaften können wir zum technisch-technologischen
Effektivitätsparameter Ett2 zusammenfassen. Der technisch-technologische
Widerspruch kann darin bestehen, dass zur Erzielung einer höheren Tragfähigkeit
eine Behälterform erforderlich wird, die ein geringeres spezifisches
Nutzvolumen und/oder eine ungünstigere Be- bzw. Entladbarkeit eines Containers
notwendig zur Folge hat. Dies ist z.B. der Fall, wenn der Behälterboden zur
Erhöhung der Tragfähigkeit eine gewölbte Form erhalten soll und dadurch die
nutzbare Behälterhöhe (bei Transport von Sperrgut) oder seine Entladbarkeit
(bei Transport von Schüttgut) unzulässig beeinträchtigt wird.

Ein technisch-naturgesetzlicher Widerspruch (TNW) besteht dann, wenn bei der
Variation eines für das Eintreten einer naturgesetzmäßigen Wirkung maßgebenden
technisch-naturgesetzmäßigen Parameters, d.h. der Wirkgröße GW, sich mindestens
zwei technisch-naturgesetzmäßige Effektivitätsparameter Etn1 des technischen
Objekts gegenläufig zueinander verhalten statt wunschgemäß in gleicher
Richtung.

Im Hinblick auf die Lösung des technisch-technologischen Widerspruchs kann im
Fall des Containers zumindest örtlich die Elastizität des Materials der
Behälterwandung als Wirkgröße GW in Betracht gezogen werden. Die mit dieser
Einwirkgröße eintretende naturgesetzmäßige Auswirkung ist die elastische
Verformung der Behälterwandung. Durch diese Auswirkung werden zwei wesentliche
technisch-naturgesetzmäßige Effektivitätsparameter gegenläufig beeinflusst:
\begin{itemize}
\item einerseits die Anpassungsfähigkeit der Form des Behälters an die Form des
  Ladegutes sowie die Anpassungsfähigkeit seiner Festigkeitsverteilung an die
  Verteilung der spezifischen Belastung (Etn1), aber auch
\item andererseits die Formbeständigkeit des Behälters (Etn2).
\end{itemize}
Beide Effektivitätsparameter werden dann nicht in Widerspruch zueinander
stehen, wenn die Elastizität als Wirkgröße in der Behälterwandung zweckmäßig
verteilt bzw. wenn die Formbeständigkeit keine maßgebende Rolle spielt oder
sogar unerwünscht ist. Letzteres ist z.B. beim Müllcontainer der Fall. Deswegen
gibt es den Müllsack aus extrem dünner, hochelastischer und biologisch
abbaubarer Plastfolie. Hier entsteht durch die Wahl der Wirkgröße „Elastizität“
kein technisch-naturgesetzmäßiger Widerspruch, sondern es wird auch der
technisch-technologische Widerspruch gelöst und mit ihm der
technisch-ökonomische Widerspruch. Dabei ist die durch extrem dünne
Behälterwand bedingte höhere „Korrosionsanfälligkeit“ hier sogar erwünscht,
weil sie einen schnellen biologischen Abbau des Behälters zur Folge hat.

Ein technisch-naturgesetzmäßiger Widerspruch kann z.B. dann gegeben sein, wenn
der negative Einfluss der gewählten Wirkgröße „Elastizität“ auf die
Formbeständigkeit darin besteht, dass bei dynamischer Belastung Schwingungen
der Behälterwand auftreten, die zu Resonanzerscheinungen und infolgedessen zu
einer Beeinträchtigung des Ladegutes und/oder zur vorzeitigen Zerstörung der
Behälterwand führen können.



3.2  Systemanalyse des technischen Objekts

„Das Widersprechende im Dinge selbst…., die widersprechenden Kräfte und Tendenzen in jedweder Erscheinung; …. das Ding (die Erscheinung etc.) als Summe und Einheit der Gegensätze, der innerlich widerstrebenden Tendenzen (und Seiten) in diesem Ding, wobei fast jedes Teil dieses Systems mit jedem verbunden ist. (W. I: Lenin in seinen Konspekten zur Dialektik von G. W. F. Hegel)

So wurde in vorstehenden Abschnitten zur Geltung gebracht: Die dialektische Widersprüchlichkeit der zur Geschichte gewordenen und der bevorstehenden Entwicklung des technischen Objekts reduziert sich nicht auf ein einzelnes Gegensatzpaar. Sie ist vielmehr die Form und der Trieb der Entwicklung des technischen Objekts als eines Systems, als einer Gesamtheit von vielen Elementen und vielen Beziehungen, die ihrerseits in Beziehungen zur gesellschaftlichen und zur technischen Umgebung des Objekts eingebunden sind. Das stofflich vorliegende Objekt ist ein System. Im technischen Entwicklungsprozess, den der Erfinder zu sehen hat, ist es eine Momentaufnahme, ein Ausschnitt, dazu veranlassend, das Konzept des Objekts in Entwicklung und Beziehung zu sehen. Das erfordert erst recht, das Objekt als System zu sehen.

In vorstehenden Abschnitten wurde begonnen, die Menge der Beziehungen in ihrer Systematik zu zeigen, die zunächst einmal durch den gesellschaftlichen und den technischen Entwicklungsprozess gegeben ist und grob durch die Betrachtungsebenen

- technisch-ökonomisch (mit den ABER und den Zielgrößen)

- technisch technologisch (mit den \emph{aber})

- technisch-naturgesetzlich

gekennzeichnet wurde. Die Beziehungen innerhalb jeder Ebene – z.B. zwischen den ABER – wurzeln in Beziehungen anderer Ebenen. Die Wurzeln sind selbst Beziehungen. Die widersprüchlichen Beziehungen wurden hervorgehoben, weil sie jeder Entwicklungsproblematik ihre Schärfe, ihre konkrete Struktur, und dem Erfinder seine Ansatzpunkte und die Fingerzeige zur schöpferischen Lösung geben.

Im Abschnitt 1.9 wird auch angedeutet werden, wie der Erfinder im technischen Objekt Paare technischer Elemente schaffen kann, deren entgegengesetztes Wirken gerade diejenige Resultante hervorbringt, die als technisches Ideal gewünscht wird.

(Die weiteren Ausführungen in diesem Abschnitt erlauben es, im Jahre 2017 weggelassen zu werden.) Hier passt Duncker-Pendel etc. s.F. besonders 3.4, auch REL 


3.3  Vertiefte Systemanalyse – schädlicher technischer Effekt stE und technisches Ideal, technisch technologischer und technisch-naturgesetzlicher Widerspruch

Im Erfindungsprogramm (Abschnitte 2, 3, 4) wurde gezeigt, wie Systemanalyse in einem ersten Anlauf zur Wirkung gebracht wird: Vom vollständig erfassten System der ABER und der Zielgröße wird auf systemanalytischem Weg zur Feststellung des gravierenden technisch-ökonomischen Widerspruchs vorgedrungen. Unter Umständen können schon auf dieser Wegstrecke erfinderische Ergebnisse gewonnen werden. (Dieser Effekt wurde in Berliner Erfinderschulen der achtziger Jahre häufig erzielt. Die weiteren Ausführungen in diesem Abschnitt 1.8 erlauben es, 2017 weggelassen zu werden)




3.4.  Die „raffiniert einfachen Lösungen“ REL

Die REL sind Lösungen mit besonders günstigem Verhältnis von Aufwand und Nutzen. In den Abschnitten 6.4 und 9.3 des „Erfindungsprogramms“ wird die Aufmerksamkeit unter anderem auf Lösungen gerichtet, deren verbale Beschreibung Wörter wie „von selbst“, „Selbstbewegung“, „Selbstfixierung“ enthält. Bereits der Programmabschnitt 2.14 enthält eine Frage, die auf solche Lösungen abzielt: „Welche Nebenfunktionen im System eignen sich, um andere Nebenwirkungen nutzbar zu machen oder schädliche Nebenwirkungen zu unterdrücken oder in nützliche zu verwandeln?“ Sehr oft ist eine solche Eignung gegeben. Dann kann eine Lösung oder Teillösung des Typs „von selbst“ schon während der gerade begonnenen Systemanalyse gefunden werden, in diesem Falle eine Selbstkompensation. Erfahrungen zeigen, dass an solche einfachen und idealen Lösungen zumeist gar nicht gedacht wird. Deshalb werden sie leider gar nicht gesucht. 

Solche Lösungen sind dadurch gekennzeichnet, dass ihre stoffliche Realisierung überwiegend mit schon vorhandenen Funktionseinheiten und Energiepotentialen, mit wenig apparativem Aufwand und/oder wenig Betriebsenergie auskommt. In diesem Sinne sind sie einfach, elegant, ideal. Die technische Welt ist seit alters her voller solcher Lösungen, an denen wir leider achtlos vorübergehen, weil wir uns schon im Kindesalter an sie gewöhnt haben. Typisch ist der Schiffsanker, ein äußerst einfaches Gerät, dessen Spitzschaufeln sich „von selbst“ umso tiefer in den Meeresgrund eingraben, je stärker Wind oder Strömung am Schiff angreifen (Analog verhält sich der Angelhaken im Fischmaul.)

Nähme man an, dass vor der Erfindung des Ankers vielleicht ein schwerer Körper vom Boot ins Wasser geworfen wurde, könnte man sich die Erfindung des Ankers folgendermaßen vorstellen: Dieser schwere Körper war eine Funktionseinheit. Diese wurde in zwei Komponenten gespalten: Eine Komponente, die sich unter gewissen Umständen eingraben kann, und eine Komponente in Form eines Querstabes am Ankerschaft, die dafür sorgt, dass am Meeresboden stets eine Spitzschaufel in Eingrabestellung ist. Kein Taucher, kein Roboter braucht am Meeresboden eine der Spitzschaufeln in Eingrabestellung zu postieren. Mit seiner Komponente „Querstab“ besorgt das der Anker „von selbst“. Oft haben Anker drei oder vier Spitzschaufeln, wobei eine Spitzschaufel die Position des Querstabes mit übernimmt.

Außerdem wird der Anker insgesamt als Teilsystem, als Komponente des übergeordneten Systems //Meeresboden, Anker, Schiff, Wind// aufgefasst. Und dieses System wird seinerseits in zwei Hauptkomponenten gespalten:

- die Komponente /Wind, Schiff/,
- die Komponente /Anker, der sich in Meeresgrund eingräbt, falls eine Zugkraft am Ankerschaft angreift/.

Damit entsteht eine verblüffend einfache Lösung: Der schädlichen Abtrift des Schiffes wird umso stärker entgegengewirkt, je stärker sie zu werden droht. Anders gesagt: Dem schädlichen Phänomen wird umso stärker entgegengewirkt, je stärker die Kraft ist, die das schädliche Phänomen hervorbringt: „Das Schädliche macht sich selbst unschädlich“. Es kompensiert sich selbst. Die beiden Komponenten /Wind, Schiff/ und /Anker…./ sind so gebildet und zusammengefügt, dass eine funktionelle Verschmelzung ihrer Solo-Wirkungen entsteht, und zwar eine Verschmelzung, welche die Selbstkompensation zum Effekt hat. Und diese erlaubt, alle denkbaren ABER zu erfüllen: Es wird nicht nur die Abtrift verhindert, sondern es geschieht mit äußerst geringem Aufwand an Mitteln und mit Hilfe der Windenergie, deren Wirkung gerade ausgeschlossen bzw. verhindert werden sollte. Die gratis zur Verfügung stehende Naturkraft „Wind“ wird in den Dienst der Sache gestellt, desgleichen das Schiff als Wandler der Windenergie zur Antriebsenergie, die mittels Kette auf den Anker übertragen wird. Das ist eine ideale Lösung.

Sind Ingenieure bereit und fähig, solche genialen Von-selbst-Lösungen zu finden? Das erprobte ich mit dem Objekt „Uhrenpendel mit temperatur-abhängiger Pendellänge“, das sich im 19. Jahrhundert in der Hochsee-Schifffahrt stellte und tatsächlich gelöst wurde. Das Problem wurde 1935 vom Psychologen Carl Duncker für ein psychologisches Experiment genutzt. Fest stand schon seit langem: Die Pendellänge einer Pendeluhr verändert sich bei wachsender oder fallender Temperatur der Umgebung. Dabei ändert sich die Umlauf-Geschwindigkeit der Uhrzeiger und damit die Zeitmessung der Pendel-Uhr. Wenn auf hoher See – zumal bei Atlantik-Überquerungen - sehr genaue Zeit-Ermittlungen nötig werden, um Bewegungsrichtung und Standort des Schiffes aus den Himmels-Koordinaten (Sonne oder Sterne) ableiten zu können, kann es peinlich werden, wenn das Uhrenpendel auch nur ein klein wenig länger oder kürzer wird und die Uhr auch schon ein klein wenig schneller oder langsamer geht. Senkt sich das Pendel-Schwerwicht nach unten, geht die Uhr langsamer. Steigt das Pendelschwergewicht nach oben, geht die Uhr schneller. Die Lösung wurde im 19. Jahrhundert gefunden. Etwa fünfzig Jahre später, nämlich 1935, trug der Psychologe Carl Duncker das Problem seinen Probanden vor: Wer ist so schlau, eine Problemlösung zu finden? Nochmals fünfzig Jahre später trug ich das Problem 150 ausgebildeten Ingenieuren vor, die per Postgradual-Studium Patentingenieure werden wollten. Ich gab 10 Minuten Zeit zum Überlegen. Doch was kamen da als Lösungsvorschläge? „Die Uhr in einer Kammer mit konstanter Temperatur einschließen. Die Kammer gut gegen Temperatur-Änderungen isolieren. Die Kammer innen beheizen bzw. kühlen.“ Als wäre das im 19. Jahrhundert möglich gewesen. Ein Einziger von 150 Ingenieuren besann sich darauf, im Schulunterricht mal etwas vom Dunckerschen Uhrenpendel gehört zu haben. Am Ende der vorgegebenen 10 Minuten rief er: „Ich hab´s“. Die Lösung? Die Uhr wird mit einem Doppel-Pendel ausgestattet. Ein Stab aus Metall mit niedrigem Wärmedehnungs-Koeffizienten hängt nach unten. An seinem unteren Ende ist eine kurze horizontale Traverse befestigt, und von dieser ragt ein Stab nach oben, aus Metall mit relativ hohem Wärmedehnungskoeffizienten. Er trägt das entscheidende Pendel-Schwergewicht an seinem oberen Ende. Die Koeffizienten sind so ausgewählt, dass sich ihre Wärme-Dehnungen gegenseitig kompensieren. Die Störquelle „Temperatur-Änderung“ wirkt zugleich als Energiespender der Kompensation. Die Regelstrecke „Pendel“ war in zwei entgegengesetzte und zugleich kooperativ wirkende Komponenten gespalten worden.

Selbst im Physik-Unterricht an den Schulen werden REL, die in der jahrhundelangen Geschichte der Technik massenhaft gefunden worden sind, total ignoriert. Als neugieriges Kind habe ich mich gewundert, dass der simple Toiletten-Spülkasten die Wasserzufuhr automatisch regelt. Da fragte ich meinen Vater, und weil er Handwerker war, hat er es mir erklärt. Sogar G. S. Altschuller hat den raffiniert einfachen Lösungen in seinen Büchern die gebotene Aufmerksamkeit nicht gewidmet. Erkennt man die Chancen nicht, kann es schwierig werden:

4.  Vorgehensweise bei der Arbeit mit dem Erfindungsprogramm (Allgemeines heuristisches Wegmodell der KDT-Erfinderschule)

4.1  Die Struktur des Wegmodells

Das Weg-Modell ist in der Abbildung als heuristisches Schema dargestellt. Es zeigt, wie von den technisch-ökonomischen Sachverhalten ausgehend auf die notwendigen Effektivitäts- und Gebrauchseigenschaften technischer Objekte, ihre Struktur- und Funktionseigenschaften und schließlich auf die funktionstragenden technisch-naturgesetzlichen Effekte abstrahiert wird und wie man dabei - von Abstraktionsstufe zu Abstraktionsstufe fortschreitend – auf der Suche nach Lösungsideen immer weiter vom eigenen Fachgebiet in entfernte Analogiebereiche vordringt. Bereits hierbei können Erfindungen mit hohem wirtschaftlichem Nutzen entstehen.



Legende zur Abbildung:


ABER       Anforderungen, Bedingungen, Erwartungen, Restriktionen

TTP          technisch-technologisches Prinzip

STE          schädlicher technischer Effekt

HTE          hemmender Traditionseffekt

TÖW         technisch-ökonomischer Widerspruch

TTW          technisch-technologischer Widerspruch

TNW          technisch-naturgesetzmäßiger Widerspruch

PLP           Problemlösungsprinzipe

PL              Prinziplösung

TE              technischer Effekt

FP              Funktionsprinzip

TFP            Teilfunktionsprinzip (Verfahrensfunktionsprinzip)

tP                technisches Prinzip

BEP            Speicher bionischer Effekte und Prinzipe

EGT            Speicher von Entwicklungsgesetzen der Technik

TNM            technisch-naturwissenschaftliches Modell


Für die in den einzelnen Erprobungsstufen der Realisierungsphase benötigten Versuchsobjekte sind die in der Erzeugnisentwicklung üblichen Begriffe verwendet worden. Für die Produkt- und die Verfahrensentwicklung sind analoge Begriffe einzusetzen. Hierbei gilt allgemein, dass das Versuchsobjekt die materialisierte Form der bei der Realisierung der erfinderischen Idee erreichten Entwicklungsstufe ist.


4.2  Das gesellschaftliche Bedürfnis und die ABER

Von der mehr oder weniger unscharf formulierten technisch-ökonomischen Problemsituation ausgehend werden - dem heuristischen Wegmodell gemäß - zunächst das gesellschaftliche Bedürfnis und die zugehörigen ABER ermittelt. Unerlässlich ist es, durch Gegenüberstellung mit dem materiellen Ist-Stand der Technik und seiner vergangenen Entwicklung die Ursachen für das Entstehen des gesellschaftlichen Bedürfnisses und der daraus abgeleiteten ABER zu erkennen. Dabei ist stets auch zu prüfen, ob die angenommene Aufgabe auf die Überwindung der Ursachen oder nur auf die Beseitigung unerwünschter ökonomischer, sozialer bzw. technologischer Auswirkungen orientiert ist. Dabei lässt sich das in technisch-wissenschaftlich technischer Hinsicht zu lösende Hauptproblem abgrenzen und eine Referenzvariante des technischen Systems bestimmen, die den ABER am ehesten entspricht oder nahekommt.

Um nun die Defekte und Mängel der Referenzvariante bestimmen und wichten, die Ursachen hierfür ermitteln und daraus eine eigenständige, „maßgeschneiderte“ Definition und Lösung des speziellen Problems ableiten zu können, wird im Rahmen einer konzeptionellen Produkt-Planung die Zielgröße bestimmt. Der Zielgröße folgend wird aus dem ideellen Stand der Technik eine repräsentative Basisvariante des technischen Systems geschaffen, indem durch Patentrecherche und Weltstands-Analyse geeignete technische Mittel herausgefunden und zum Gesamtsystem zusammengefasst werden. Im Rahmen einer Systemanalyse wird die Basisvariante im Vergleich zur Referenzvariante auf solche Schwachstellen und Defekte untersucht, die in ihrem Verhalten auf technisch-ökonomische Widersprüche führen und erfinderisch zu beheben sind. Dabei ist aus der Basisvariante eine Lösung zu entwickeln, die gegenüber der Referenzvariante deutliche technologische und ökonomische Vorteile aufweist.


4.3. Die Zielgröße und der Stand der Technik

Die ABER liegen zunächst in einer verbal-beschreibenden Form vor und bringen soziale, ökonomische und technologische Sachverhalte zum Ausdruck, die eine bestimmte gesellschaftliche Bedarfssituation und Interessenlage kennzeichnen (siehe hierzu Erfindungsprogramm, Abschnitt 1). Hieraus soll nun eine Zielgröße abgeleitet werden, die im Wesentlichen zum Ausdruck bringt, mit welchen Gebrauchseigenschaften des zu schaffenden technischen Systems und auf welche Art und Weise seiner Herstellung und Anwendung dem gesellschaftlichen Bedürfnis am besten entsprochen werden kann. Dies geschieht, indem den Komponenten der Zielgröße die auf sie zutreffenden ABER zugeordnet werden. Dadurch werden konkrete Eignungs- und Effektivitätsmerkmale definiert und gewertet. Diese beinhalten einerseits die ihnen jeweils zukommende soziale, ökonomische und/oder technisch-technologische Spezifik des gesellschaftlichen Bedürfnisses und andererseits die gegenständliche Spezifik des in Betracht gezogenen technischen Objekts oder Objektbereiches (Siehe Abschnitt 1.3).

Diese Eignungs- und Effektivitätsmerkmale müssen zuerst qualitativ beschrieben werden, bevor Parameterwerte angegeben werden können. Auf jeden Fall muss vermieden werden, sich vorzeitig und kritiklos auf gewohnte oder in der Aufgabenstellung genannte Gebrauchswert- und Wirtschaftlichkeitsparameter festzulegen oder sich auf sie zu beschränken.

Um diese Parameter treffend definieren zu können, ist es erforderlich, aus dem Stande der Technik das geeignetste technisch-technologische Prinzip (TTP) für das zu entwickelnde technische Objekt zu wählen. Das ist ein charakteristisches Prinzip der Herstellung und/oder der Anwendung technischer Objekte in einem bestimmten Technologiebereich. Mit diesem Prinzip wird eine Klasse von Verfahren und Mitteln im Stande der Technik abgegrenzt, welche die Grundlage der weiteren Problembearbeitung darstellt. Der Wahl des TTP kommt damit eine entscheidende Bedeutung für den weiteren Lösungsweg zu. Sie sollte so erfolgen, dass dasjenige technisch-technologische Prinzip bevorzugt wird, das dem Zweck des zu schaffenden technischen Objekts (Zielkomponente Z1) am meisten entspricht und mit dem gegen keine ABER oder - im Vergleich zu anderen Prinzipien - gegen die wenigsten A und E aus dem System der ABER verstoßen wird. Hierfür sind zunächst In Betracht zu ziehen:

- alle auf dem materiellen Stand der Technik verfügbaren,

- alle auf dem ideellen Stand der Technik machbar erscheinenden und schließlich

- die auf dem Stand der Technikwissenschaften denkbaren und die auf dem Stand der Naturwissenschaften vorstellbaren Verfahren und Mittel.

Sollte ein technisch-technologisches Prinzip mit der Aufgabenstellung verbindlich vorgegeben sein, so ist es auf Eignung in bezug auf die Zielgröße zu überprüfen und mit anderen bekannten Prinzipien zu vergleichen. Gegebenenfalls muss hier Rücksprache mit dem Auftraggeber genommen werden.

Auf der Grundlage des technisch-technologischen Prinzips werden nun zunächst mit den im Stand der Technik Vorgefundenen Verfahren und Mitteln Basisvarianten des technischen Systems konzipiert. Das geschieht durch Transformation der durch die ABER determinierte Zielgröße in zwei Stufen:

In der ersten 'Transformationsstufe werden die Arten von technischen Objekten benannt, die dem technisch-technologischen Prinzip gemäß notwendig sind, um die Eignung des technischen Systems gemäß Zielgröße gewährleisten zu können. Jeder Objektart werden nun solche Gebrauchseigenschaften zugeschrieben, welche einerseits typisch für die jeweilige Objektart sind und andererseits den besonderen Eignungsmerkmalen der Zielgröße entsprechen. Dabei wird zweckmäßig so vorgegangen, dass zunächst die notwendigen objektartspezifischen Beiträge, zur Zweckmäßigkeit (Zielgrößenkomponente Z1) des Systems bestimmt werden. Danach werden diejenigen für die jeweilige Objektart charakteristischen Gebrauchseigenschaften definiert, auf Grund derer die Eignung des technischen Systems hinsichtlich seiner Beherrschbarkeit und seiner Brauchbarkeit gewährleistet werden kann. Dabei kann es sich für eine hinreichende Eig­nung des technischen Systems - vor allem in Bezug auf seine Beherrschbarkeit - als notwendig erweisen, zusätzlich Objektarten in Betracht zu ziehen, die mit ihren Gebrauchs- bzw. Betriebseigenschaften solchen spezifischen Eignungsmerkmalen gerecht werden.

Auf diese Weise wird die Zielgröße von einem System gesellschaftlich determinierter Eignungsmerkmale in ein System objektbezogener Gebrauchseigenschaften transformiert. Diese Zielgröße bildet die Grundlage für eine systematische Patentrecherche und Weltstandsanalyse zur Vor-Auswahl geeigneter technischer Objekte, welche in ihrem Verbund gemäß technisch-technologischem Prinzip hinreichend geeignet sind, ein technisches System zu bilden, das den ABER gerecht wird.

In einer zweiten Transformationsstufe wird nun dem technisch-technologischen Prinzip entsprechend die Hauptfunktion des technischen Systems definiert. Dabei ist davon auszugehen, dass die Hauptfunktion die Gebrauchseigenschaften der einzelnen Objekte aktiviert und im Prozess ihrer Nutzung so miteinander verknüpft, dass die für die Zweckmäßigkeit bestimmenden Eignungsmerkmale des technischen Systems den ABER entsprechend hervorgebracht werden. Diese Hauptfunktion ist, auf den Nutzungsprozess bezogen, in ihre notwendigen und hinreichenden Teilfunktionen aufzugliedern. Dabei wird eine Hierarchie-Ebene des technischen Systems gewählt, die einerseits möglichst hoch ist, andererseits der bereits getroffenen Aufgliederung der Zielgröße auf Objektarten Rechnung trägt.

Den einzelnen Teilfunktionen werden diejenigen Objekte mit ihren Gebrauchseigenschaften zugeordnet, welche durch die jeweilige Teilfunktion im Sinne der Hauptfunktion des technischen Systems aktiviert werden. Für jede Teilfunktion werden die durch sie hervorgerufenen Funktionseigenschaften der technischen Objekte definiert, wodurch diese den Charakter spezifischer technischer Mittel bekommen. Die Teilfunktionen, durch welche diejenigen objektbezogenen Gebrauchseigenschaften aktiviert werden, die die Eignung des technischen Systems in Bezug auf Beherrschbarkeit und Brauchbarkeit herstellen, werden in gleicher Weise, jedoch im Sinne von notwendigen Hilfsfunktionen definiert.

Auf diese Weise wird die Zielgröße von einem System objektbezogener Gebrauchseigenschaften in ein System prozessbezogener Funktionseigenschaften technischer Mittel transformiert. Diese Zielgröße dient der zweckmäßigen Auswahl technischer Mittel aus der Menge der in Betracht gezogenen technischen Objekte und ihrer funktionsgerechten Koppelung zur Basisvariante des technischen Systems. Darüber hinaus bildet die Zielgröße in dieser Transformationsstufe zusammen mit der nicht transformierten Komponente Z2 (Wirtschaftlichkeit) die Grundlage für die Definition der technisch-ökonomischen Hauptleistungsdaten des technischen Systems und für ihre quantitative Bestimmung im Sinne einer Sollgröße.

Von dieser Sollgröße, geht die Systemanalyse aus. Sie verfolgt das Ziel, die .Effektivitätseigenschaften des technischen Systems in ihrem Zusammenhang zu bestimmen, insbesondere widersprüchliche Tendenzen in ihrem entwicklungsbedingten Verhalten aufzudecken und im Rahmen einer Entwicklungsschwachstellen-Analyse die hierfür maßgeblichen technischen Ursachen herauszufinden. Dabei kann die funktionsbezogene Zielgröße bereits einen ersten Hinweis auf den kritischen Funktionsbereich (kritischen Systembereich) geben. Dieser liegt in der Regel dort, wo die größte Anzahl von Teilfunktionen in einem technischen Objekt zusammentreffen.


4.4. Die Basisvariante

Die gemäß Zielgröße aus dem Stand der Technik ausgewählten technischen Mittel werden ihrer Funktion entsprechend zu Teilsystemen in Form voneinander abgrenzbarer Struktureinheiten zusammengefasst. Jede Struktureinheit verkörpert dabei jeweils eine der prozessbezogenen Teilfunktionen in der Hauptfunktion oder eine für deren Beherrschung, Schutz und/oder Umweltverträglichkeit notwendige Hilfsfunktion. Bei der funktionsgerechten Kombination der tech­nischen Mittel zu Teilsystemen und der Teilsysteme zum Gesamtsystem der Basisvariante müssen die funktionellen Anforderungen (a) und strukturellen Bedingungen (b) sowie die naturgesetzmäßigen Einflüsse (e) und Restriktionen (r) berücksichtigt werden, welche die einzelnen technischen Mittel bzw. Teilsysteme bei ihrer Vereinigung zur Basisvariante aneinander stellen bzw. aufeinander ausüben. Hierzu muss für ihre Koppelung (vermittels morphologischem Schema) eine Rangordnung nach der technisch-technologischen Bedeutung der Teilsysteme festgelegt werden, derart, dass ein Teilsystem bzw. technisches Mittel höheren Ranges die (a), (b), (e), (r) für die Teilsysteme bzw. technischen Mittel auf den jeweils darunter liegenden Stufen der Rangordnung setzt.

4.4.1. Der entscheidende Mangel und die Kernvariante

Die entsprechend der Zielgröße auf dem Stand der Technik bzw. der Technikwissenschaften entwickelten Basisvarianten haben in der Regel noch entscheidende Mängel. Diese Mängel können technisch ökonomischer Art sein, entstanden dadurch, dass die Gebrauchs-und Wirtschaftlichkeitseigenschaften nicht in Übereinstimmung mit der Zielgröße gebracht werden konnten, also gegen Anforderungen und/oder Restriktionen verstoßen werden musste. Die Mängel können aber auch "heuristischer" Art sein, d.h. darin bestehen, dass Mittel weder verfügbar sind noch machbar erscheinen, sondern höchstens denkbar oder gar nur vorstellbar sind.

Ein technisch-ökonomischer Mangel liegt vor, wenn die gemäß Zielgröße benötigten technischen Mittel zwar grundsätzlich zur Verfügung stehen oder bekannt sind, aber mindestens in einer entscheidenden Gebrauchseigenschaft die erforderlichen Werte eines kennzeichnenden Leistungs- und/oder Effektivitätsparame­ters nicht oder nur auf Kosten anderer gebrauchswertbestimmen­der Parameter erreichbar sind.

Ein heuristischer Mangel liegt vor, wenn zur Erzeugung mindestens einer auf Grund der ABER erforderlichen Gebrauchseigenschaft keine technischen Mittel bekannt sind, welche ihren Funktionseigenschaften nach geeignet wären, die hierfür gemäß Zielgröße erforderlichen Mittel-Wirkungs-Beziehungen hervorzubringen.

Sich einem heuristischen Mangel bewusst zu stellen erfordert erfinderischen Spürsinn und Mut, herkömmliche und bewährte Technik in Frage zu stellen.

Für die weitere Problembearbeitung wird diejenige Basisvariante ausgewählt, welche die geringsten Mängel aufweist. Dabei zeichnet sich erfinderisches Vorgehen dadurch aus, dass es keine gravierenden technisch-ökonomischen Mängel zulässt, dafür aber gravierende "heuristische Mängel" bewusst in Kauf nimmt, wenn sie zu erfinderischen Lösungen herausfordern. Liegt ein gravierender "heuristischer" Mangel vor, so wird das Teilsystem (TS) bzw. der Systembereich, in welchem dieser Mangel auftritt, zum entscheidenden Teilsystem bzw. zur „Kernvariante" des technischen Systems erklärt. Für den in diesem Teilsystem bzw. in diesem Systembereich liegenden problematischen Kern einer Basisvariante werden durch neuartige Abwandlungen oder bisher nicht übliche Kombinationen bekannter technischer Objekte neue, denkbare Lösungen generiert. Aus diesen „Kernvarianten" wird diejenige gewählt, welche sich am besten in den Gesamtzusammenhang des technischen Systems der Basisvariante einfügen lässt. Sie kann bereits eine erfinderische Lösung sein und ist dann das Ergebnis eines heuristischen Vorgehens, das als „projektierendes Erfinden“ bezeichnet werden kann.

Besteht die Basisvariante aus einer erfinderischen Kernvariante mit nur geringer technologischer Tragweite und im Übrigen aus betriebserprobten Systemkomponenten aus dem verfügbaren Stand der Technik und weist sie keine erheblichen Mängel in Bezug auf die Zielgröße auf, so kann sie auf dem Wege der Optimierung in ein betriebliches Gesamtprojekt überführt und in einer Nullserie bzw. Versuchsproduktion erprobt werden. Sind jedoch noch erhebliche Abweichungen zwischen Gebrauchswert und Effektivität der Basisvariante einerseits und der Zielgröße andererseits zu verzeichnen und sind insbesondere die Funktionseigenschaften der Kernvariante im Gesamtzusammenhang des technischen Systems noch in Frage gestellt, so ist das weitere Vorgehen darauf gerichtet, die Ursachen dieser Mängel genauer zu untersuchen und zu beheben. Hierzu wird zunächst als Präzisierung der aus der Zielgröße abgeleiteten eine technisch-ökonomische Zielstellung formuliert, welche auf die Erhöhung gerade derjenigen Leistungs- und/oder Wirtschaftlichkeitsparameter (Hauptleistungsdaten) abzielt, deren Erfüllung noch in Frage gestellt ist.


4.4.2  Die strukturell aufbereitete Basisvariante

Die Ursache für die festgestellten Mängel wird zunächst in der Struktur des technischen Systems gesucht. Hierzu wird die Basisvariante nach dem Gesichtspunkt ihrer Struktur aufbereitet, indem von Gebrauchseigenschaften einzelner Objekte bzw. Objektgruppen auf Struktureigenschaften des technischen Systems abstrahiert wird. Dies geschieht in der Weise, dass die in der Basisvariante zusammengefassten und funktionell verknüpften technischen Objekte bezüglich ihrer notwendigen strukturellen Gebrauchseigenschaften (vor allem enthalten in den Zielgrößenkomponenten "Beherrschbarkeit" und "Brauchbarkeit") betrachtet und so aufeinander abgestimmt werden, dass sie sich räumlich und/oder zeitlich zu den Struktureinheiten und zum Gesamtsystem der Basisvariante zusammenfügen lassen. Damit entstehen im Ansatz die systemspezifischen Struktureigenschaften der technischen Mittel. Hierbei werden vor allem die Struktureigenschaften in dem durch die Kernvariante bestimmten Systembereich hervorgehoben, durch welche die spezifischen Leistungs- und/oder Wirtschaftlichkeitsparameter der technisch-öko­nomischen Zielstellung primär beeinflusst werden. Eine Variation der Struktureigenschaften des technischen Systems im Sinne der technisch-ökonomischen Zielstellung ruft häufig eine Verschlechterung spezifischer Funktionseigenschaften hervor, worin sich bereits ein technisch-ökonomischer Widerspruch abzeichnet. Vielfach handelt es sich hierbei um einen Konflikt zwischen den Erfordernissen von Herstellbarkeit, Montagefähigkeit und/oder Instandhaltbarkeit (bzw. der kontinuierlichen Prozessführung, der Überwachbarkeit und Steuerbarkeit bei Verfahren) und den Erfordernissen der Funktionsfähigkeit, der Unempfindlichkeit gegen äußere Störungen und der inneren Funktionssicherheit. Das erfinderische Vorgehen ist hier zunächst darauf gerichtet, auf dem Wege der Optimierung diejenige kritische Struktureinheit oder Funktionsschwachstelle herauszufinden, die primär einer optimalen Gestaltung und Dimensionierung der Basisvariante im Wege steht. Durch eine geschickte Um- bzw. Neugestaltung eines oder mehrerer Objekte innerhalb dieses kritischen Systembereichs kann eine Erhöhung der Funktionstüchtigkeit bewirkt werden, ohne eine Veränderung der Funktion selbst vornehmen zu müssen. Gelingt dies, so ist eine erfinderische Lösung des Widerspruchs zwischen Struktur- und Funktionseigenschaften der Basisvariante im Sinne der technisch-ökonomischen Zielstellung gefunden worden. Ein solches Vorgehen wird als „konstruierendes Erfinden“ bezeichnet. Die erfinderische Lösung ist zunächst in ein Funktionsmuster zu überführen und hinsichtlich seiner Funktionstüchtigkeit zu erproben.

Stellt sich dabei heraus, dass die technisch-ökonomische Zielstellung sich nicht erfüllen lässt, wenn nicht auch Funktionen verändert werden, so ist eine Aufbereitung der Basisvariante unter dem Gesichtspunkt ihrer Funktionserfüllung und einer entsprechenden Systemanalyse erforderlich.


4.4.3  Die Aufbereitung der Basisvariante unter dem Gesichtspunkt der Funktionserfüllung

Bei der funktionellen Aufbereitung der Basisvariante wird von den Struktureigenschaften technischer Objekte auf ihre Funktionseigenschaften abstrahiert. Dabei wird das Ziel verfolgt, die wesentlichen funktionellen Zusammenhänge zu erkennen, in welchen die Basisvariante als technisches System mit ihrer Umgebung stehen soll, und welche inneren funktionellen Zusammenhänge (Mittel- Wirkungsbeziehungen.) zwischen ihren Bestandteilen dafür maßgeblich bestimmend sind.

Dabei kommt es zunächst darauf an, die Gesamtfunktion der Basisvariante und ihre bekannten bzw. voraussehbaren Nebenwirkungen sowie die Schnittstellenbedingungen zu ihrer technisch-technologischen Umgebung zu bestimmen. Hierzu bedient man sich der Black-Box-Analyse. Auf Grund der Schnittstellenbedingungen (Randbedingungen) der Blackbox ergeben sich die Eingangsgrößen des zu betrachtenden technischen Systems aus den vorgegebenen Ausgangs­größen eines in einem übergeordneten Nutzungsprozess jeweils vorgelagerten Systems, und seine Ausgangsgrößen aus den notwendigen Eingangsgrößen eines in diesen Nutzungsprozess jeweils nachgelagerten Systems. Je nach Art der Eingangs- und Ausgangsgrößen ergibt sich hieraus die von der Basisvariante als technischem System hauptsächlich zu realisierende Überführungs- bzw. Tragfunktion. Diese wird daher als Hauptfunktion definiert. Dies geschieht jedoch nicht - wie bei der Transformation der Zielgröße - prozessbezogen, sondern objektbezogen. Das heißt, die Funktion wird nicht als notwendige, prozessbedingte Aktivierung bestimmter Gebrauchs­eigenschaften technischer Objekte, sondern als strukturbedingte Auswirkung bestimmter Funktionseigenschaften technischer Mittel aufgefasst. Danach werden die notwendigen technischen Voraussetzungen für das Zustandekommen und die Aufrechterhaltung der Hauptfunktion, also für die Funktionsfähigkeit des technischen Systems, ermittelt. Daraus werden die hierfür erforderlichen Hilfsfunktionen definiert, wobei von den Arten „Entstörfunktion" und „Schutzfunktion" ausgegangen wird.

Bei der Definition der Entstörfunktion kann man sich zunächst an den Gebrauchseigenschaften orientieren, welche in der Zielgrößenkomponente Z3 (Beherrschbarkeit) enthalten sind. Darüber hinaus ist es erforderlich festzustellen, welche Nebenwirkungen von den konkreten Objekten der Basisvariante während ihres Betriebes bzw. ihres Gebrauchs ausgehen. Diese Nebenwirkungen müssen möglichst vollständig erfasst werden. Hier gibt es schädliche, aber auch nützliche bzw. nutzbare Nebenwirkungen. Notwendige Maßnahmen zur Unterdrückung der durch die Gesamtfunktion hervorgerufenen schädlichen Nebenwirkungen auf zulässige Werte führen zur Definition der Entstörfunktion.

Notwendige Maßnahmen zur Unterdrückung schädlicher Wirkungen, wel­che von der Umgebung auf die Hauptfunktion und die Entstörfunktion des technischen Systems ausgeübt werden, führen hingegen auf die Definition der Schutzfunktion des Systems. Bei der Definition der Schutzfunktion kann man sich zunächst von den Gebrauchseigenschaften und Gebrauchsbedingungen leiten lassen, welche in der Zielgrößenkomponente Z4 (Brauchbarkeit) zusammengefasst sind. Bei den schädlichen Wirkungen aus der Umgebung sind nicht nur technische, technologische und naturbedingte, sondern gegebenenfalls auch soziale (Qualifikation, Disziplin) und organisatorische (Versorgung mit Transportmitteln, Material, Energie und/oder Information) mit in Betracht zu ziehen.

Eine für den Erfinder wichtige Funktionsklasse bilden die Nebenfunktionen. Es sind Funktionen, welche von den Objekten der Basisvariante sozusagen „gratis" neben ihrer eigentlichen Funktionsbestimmung hervorgebracht werden bzw. hervorgebracht werden können. Sie sind daraufhin zu untersuchen, ob und inwieweit sie zur Unterstützung, gegebenenfalls sogar zum Ersatz der Funktion eines oder mehrerer anderer Objekte der Basisvariante, genutzt werden können. Dabei kann es zu einer Funktionsverschmelzung kommen, deren Wirkung über die Summe der Einzelwirkungen der betreffenden Objekte hinausgeht. Dies ist ein wichtiges Indiz für das Vorliegen einer erfinderischen Leistung.

Nebenfunktionen, welche nicht genutzt werden können, werden als unnötige Funktionen bezeichnet. Sie sind durch geeignetere Wahl bzw. Gestaltung der Objekte der Basisvariante möglichst vollständig zu eliminieren, zumindest dann, wann sie eine Störung des Funktionswertflusses hervorrufen oder unnötige Kosten verursachen.

Für eine vollständige Erfassung und vorteilhafte Gestaltung aller Wechselbeziehungen zwischen dem System und seiner Umgebung ist es erforderlich, ein Operationsfeld für den Erfinder in Bezug auf das technische System abzugrenzen. Es umfasst alle Objekte - technische wie natürliche - sowie alle Faktoren - soziale, organisatorische und technologische - mit ihren schädlichen und nützlichen Wirkungen, denen das technische System mit seiner Funktion Rechnung tragen muss oder die in seine Funktion wirksam einbezogen werden können. Von der richtigen Abgrenzung des Operationsfeldes und des technischen Systems hängt es also ab, ob die Schutzfunktion richtig bestimmt ist und ob objektiv vorhandene Möglichkeiten zur Vereinfachung von Funktionen bzw. zur Erhöhung ihres Funktionswertes erkannt und als Handlungsspielraum genutzt werden. (Siehe hierzu Erfindungsprogramm, Abschnitt 3).

Je nach Sachlage kann das so geschehen, dass geeignete Objekte aus dem Operationsfeld zur Unterstützung bzw. Vereinfachung der Funktionen in die Basisvariante einbezogen und strukturell wie funktionell integriert werden, indem ihnen eine Anpassfunktion übertragen wird. Umgekehrt kann es sich auch als vorteilhaft, in manchen Fällen sogar als notwendig erweisen, bestimmte Objekte aus der Basisvariante in den äußeren Teil des Operationsfeldes zu verlagern. Durch eine enge funktionelle und strukturelle Verknüpfung über die Systemgrenze hinweg kann dadurch im Sinne einer Vermittlungsfunktion ein positiver Einflussfaktor im äußeren Operationsfeld erzeugt bzw. ein vorhandener verstärkt werden. Möglicherweise erreicht man dadurch gleichzeitig eine Vereinfachung der Funktion der Basisvariante bzw. eine Erhöhung ihres Funktionswertes.

Mit der Black-Box-Analyse der Basisvariante ist deren funktionsbezogene Aufbereitung im Wesentlichen abgeschlossen. Die dabei gewonnenen Erkenntnisse über die Funktionseigenschaften der Basisvariante, ihre gegenseitigen Abhängigkeiten und die Möglichkeiten, sie optimal aufeinander und mit der Systemumgebung abzustimmen, werden nun genutzt, indem versucht wird, den bei der strukturbezogenen Aufbereitung der Basisvariante aufgetretenen Widerspruch zwischen Struktur- und Funktionseigenschaften zu beseitigen. Hierbei können auf erfinderische Weise, durch eine originelle Aufteilung der erforderlichen Funktionen auf die einzelnen Objekte der Basisvariante und die geschickte Nutzung bisher vernachlässigter Struktur- und Funktionseigenschaften die Voraussetzungen für eine optimale Gesamtlösung geschaffen werden.


4.4.4.  Die Optimierung der Basisvariante im Vergleich zur Referenzvariante und der technisch-ökonomische Widerspruch

Um die Optimierung der Basisvariante durchführen zu können, muss ein technischer Leistungsparameter als Führungsgröße bestimmt werden, dessen Variation einerseits die Effektivitätsparameter der technisch-ökonomischen Zielstellung und andererseits die erforder­lichen Struktur- und Funktionseigenschaften des technischen Systems in entscheidendem Maße beeinflusst. Mit der Wahl der Führungsgröße wird über die Richtung der Weiterentwicklung des technischen Systems und die Entwicklungstendenz seiner Gebrauchs- und Wirtschaftlichkeits-eigenschaften entschieden. Die Führungsgröße muss daher in Übereinstimmung mit der Zielgröße stehen, auch wenn sich herausstellt, dass ihre Variation - obwohl fachgemäß voraus­gedacht - zu Veränderungen der Struktur- und Funktionseigenschaften des technischen Systems führt, welche (zumindest teilweise) noch im Widerspruch zur technisch-ökonomischen Zielstellung stehen. Als eine Orientierung für die treffende Bestimmung der Führungsgröße kann die aus dem materiellen Weltstand der Technik gewählte Referenzvariante dienen. Dabei ergibt sich als Führungs­größe derjenige technische Leistungsparameter, in dem die Refe­renzvariante noch am stärksten von der Zielgröße abweicht. Das heißt, dass die geforderte Leistungsfähigkeit (in Zielgrößenkomponente Z1) entweder nicht oder unter den gegebenen Realisierungs-bedingungen nur mit unzulässig hohem technischen (Z3), technologischen (Z4) und/oder ökonomischen Aufwand (Z2) erzielt werden kann. Auf diese Weise wird von vornherein einer „Nachlauf"-Strategie vorgebeugt und die Grundlage für eine dem tatsächlichen ge­sellschaftlichen Bedürfnis gerecht werdende, progressive Lösungsstrategie gelegt. Darüber hinaus kann die Referenzvariante durch Einbeziehung in die Black-Box-Analyse als Anregung für die im Sinne der Zielgröße vorteilhafte funktionelle und strukturelle Konzeption der Basisvariante genutzt werden. Ist es dabei - gegebenenfalls auf erfinderische Weise - bereits gelungen, ein optimierungsfähiges Grundkonzept zu entwickeln, so wird durch eine gut aufeinander abgestimmte Gestaltung und Dimensionierung der einzelnen Objekte eine optimale, der technisch-ökonomischen Zielstellung entsprechende Gesamtlösung für die Basisvariante zu fin­den sein. Ist sie gefunden, so wird die Funktionsfähigkeit und der Funktionswert der Basisvariante an einem Versuchsmuster erprobt. Hierzu genügt die Nachbildung desjenigen Funktionsbereichs der Basisvariante, in dem die entscheidenden strukturellen und funktionellen Veränderungen gegenüber dem betriebserprobten Stand der Technik vorgenommen worden sind. In der Regel handelt es sich um die Kernvariante und ihre nähere System-Umgebung.

Wird eine optimale Gesamtlösung noch nicht gefunden oder erweist sich der Entwurf der Basisvariante als nicht funktionstüchtig, so ist der entscheidende technisch-ökonomische Widerspruch (TOW) zu bestimmen. Das heißt, es sind die entscheidenden technisch-ökono­mischen Effektivitätsparameter zu benennen, die sich so zueinander verhalten, dass die Erhöhung des einen systembedingt zu einer unzulässigen Verringerung des anderen Parameters führen muss, wenn die Führungsgröße entsprechend technisch-ökonomischer Zielstellung va­riiert wird. (Siehe hierzu Erfindungsprogramm, Abschnitt 4).

Das weitere Vorgehen ist nun nicht mehr durch „begleitendes" bzw. taktisches Erfinden, sondern durch „voranweisendes", strategisches Erfinden gekennzeichnet. Es ist das Erfinden im eigentlichen Sinne der Methode, das Erfinden „an sich". Damit kommen wir zugleich in die Etappe II des Organisationsmodells, an deren Anfang ein Erneuerungspass und ein Pflichtenheft mit klarem erfinderischen Auftrag stehen. Gegenstand der Erfindung ist jetzt ein technisch-­ökonomischer Widerspruch; Ziel ist dessen Überwindung.


4.5.  Die erfinderische Kernvariante ( Schlüsselvariante)

Bei der Überwindung des technisch-ökonomischen Widerspruchs (TÖW) wird angenommen, dass dessen Ursache nicht im ganzen technischen System „verstreut" ist, sondern sich im Wesentlichen auf einen bestimmten Systembereich - den für das Funktionieren des technischen Systems kritischen Bereich, den „kritischen Funktionsbereich" - eingrenzen lässt. Das erfinderische Ziel besteht nun darin, diesen Systembereich zu „entdecken" und eine erfinderische Lösung hervorzubringen, welche in diesem kritischen Bereich neue technische Verhältnisse, neue Mittel-Wirkungs-Beziehungen schafft, so dass neue Möglichkeiten für die Entwicklung der Basisvariante und die entsprechende Variation der Führungsgröße im Sinne der technisch- ökonomischen Zielstellung eröffnet werden.

4.5.1.  Der kritische Funktionsbereich der Basisvariante und die ABER

Ausgehend von dem Ergebnis der Black-Box-Analyse und von den Erkenntnissen, welche bei der erfolglosen Optimierung der Basisvariante und gegebenenfalls aus einer mit negativem Ergebnis abge­schlossenen Funktionserprobung gewonnen wurden, wird nun den Ursachen des technisch-ökonomischen Widerspruchs auf den Grund ge­gangen. (Siehe hierzu Erfindungsprogramm, Abschnitt 5).

Hierzu wird die Basisvariante in Fortführung der Black-Box-Analyse zunächst in die objektbezogenen Teilfunktionen zerlegt, welche unerlässlich sind, um in einer funktionierenden Verkettung von bewirkten und/oder verhinderten Zustandsänderungen eines oder mehrerer Objekte am Ende eine stabile und effektive Hauptfunktion hervorzubringen. Hierbei kann zur Präzisierung der notwendigen Funktionsmerkmale auf die in der Zielgrößenkomponente Z1 (Zweckmäßigkeit) zusammengefassten Gebrauchseigenschaften im Sinne von Auswirkungen ihrer Aktivierung zurückgegriffen werden.

Die in der Basisvariante als Systemkomponenten enthaltenen Objekte können demgemäß einzelnen Teilfunktionen zugeordnet und bezüglich ihres Funktionswertes beurteilt werden. Dabei wird sich immer ein Bereich des technischen Systems abgrenzen lassen, in dem eine oder auch mehrere Teilfunktionen angelegt sind, die einen im Vergleich zu den benachbarten Systembereichen deutlich niedrigeren Funktionswert besitzen. Dieser Systembereich wirkt wie ein „Flaschenhals" im Funktionswertfluss der Hauptfunktion, der diese und andere Teilfunktionen nicht voll zur Wirkung kommen lässt und damit die Funktionsfähigkeit des technischen Systems insgesamt entscheidend ein­schränkt. Er wird deshalb als der „kritische Funktionsbereich" des technischen Systems bezeichnet. Entscheidend für den Erfinder ist nun nicht allein die Frage nach den technisch-naturgesetzmäßigen Ursachen für die Entstehung des funktionellen „Flaschenhalses", sondern die Frage nach den technisch-konstruktiven bzw. verfahrens­technischen Gründen, die einer Beseitigung dieser Ursachen auf dem Wege der optimalen Dimensionierung entgegen stehen. Diese Gründe sind auf einen schädlichen technischen Effekt (STE) zurückzuführen, der die Entwicklung des technischen Systems im Sinne der Zielgröße nicht zulässt.

Die Beantwortung dieser für die Erfindungsaufgabe entscheidenden Frage nach dem schädlichen technischen Effekt (STE) kann nur schrittweise erfolgen. Dazu werden die im kritischen Funktionsbereich angelegten Teilfunktionen ihrem Verfahrensprinzips gemäß in Elementarfunktionen und in die entsprechenden, gegenständlichen Funktionseinheiten zerlegt. Die einzelnen Funktionseinheiten werden in ihre operationalen Bestandteile - Operation, Operand, Operator und Gegenoperator - aufgelöst und diese dem Funktionsprinzip der jeweiligen Funktionseinheit gemäß als funktionelle Bestimmungs­größen technisch definiert. So kann der kritische Funktionsbereich in einem morphologischen Schema übersichtlich und durchschaubar dargestellt werden.

Der Funktionswertfluss wird nun von Elementarfunktion zu Elementarfunktion untersucht. Dabei wird, einer geeignet gewählten Leit­größe (Strukturgroße) folgend, durch Variation der funktionellen Bestimmungsgrößen eine Optimierung der Funktionseinheiten unter Beibehaltung des Funktionsprinzips versucht (siehe Kapitel 4). Je nach Ergebnis der Optimierungsversuche kann die Wurzel des schädlichen technischen Effekts auf bestimmte Funktionseinheiten und deren strukturelle und funktionelle Eigenschaften begrenzt werden. Damit wird der kritische Funktionsbereich zunehmend eingeengt und präziser definiert. Gleichzeitig werden die in einer für das technische System kennzeichnenden Weise zusammenhängenden, technisch und naturgesetzmäßig gegebenen Anforderungen, Bedingungen, Einflüsse und Restriktionen (ABER) ermittelt, welche den technisch-wissenschaftlichen Problemkern bilden. Das heißt, die am Anfang von Abschnitt 2.4. definierte Menge der (a), (b), (e), (r) ist durch, die Verknüpfung der Teilobjekte der Basisvari­ante zu einem System geworden. Die *ABER* sind das Analogon der ABER. Letztere bestehen auf der technisch-ökonomischen Ebene, erstere auf der technisch-naturgesetzliehen. Man beachte, dass hier mit „Einflüssen" statt mit „Erwartungen" zu rechnen ist.

Die \emph{aber} stehen der Überwindung des technisch-ökonomischen Widerspruchs im Wege. Diese \emph{aber} werden in einer folgenden Stufe des erfinderischen Vorgehens im Sinne eines technischen Ideals (IDEAL) so verändert, dass der schädliche technische Effekt verschwindet. Technische Punktionsanforderungen und naturgesetzmäßige Restriktionen werden dabei von der Variation zunächst ausgeschlossen.


4.5.2.  Der schädliche technische Effekt und das IDEAL

Das (technische) IDEAL bezieht sich primär auf das Verhalten des technischen Systems in seinem kritischen Funktionsbereich. Das übrige technische System wird zunächst als im Wesentlichen unveränderlich gesetzt. Mit dem IDEAL werden nun solche idealen konstruktiven Bedingungen und/oder solche idealen verfahrensmäßigen Anordnungen über die erkannten Optimierungsgrenzen hinausgehend vorausgedacht, dass alle unerwünschten technisch-naturgesetzmäßigen Einflussfaktoren verschwinden oder zumindest in ihrer Wirkung so weit abgebaut werden, dass eine entscheidende Erhöhung des Funktionswertes im kritischen Funktionsbereich zustande kommt. Dabei wird das Funktionsprinzip bzw. das funktionstragende Wirk­prinzip zunächst nicht verändert. (Siehe hierzu Erfindungsprogramm Abschnitt 6).

Im Unterschied zu den ABER ergeben sich die \emph{aber} nicht unmittelbar aus dem gesellschaftlichen Obersystem und dem technologischen Umfeld des technischen Systems, sondern aus seinem konstruktiven bzw. verfahrenstechnischen Aufbau und den in ihm realisierten Funktionsprinzip. Mit den \emph{aber}  werden neben Anforderungen, Bedingungen und Restriktionen auch Einflüsse (im Sinne von Nebenwirkungen) technisch-konstruktiver und tech­nisch-naturgesetzmäßiger Art erfasst, welche die Bestandteile des technischen Systems aufeinander ausüben, oder welche auf sie aus der Systemumgebung einwirken.

Das Entgegensetzen neuer Bedingungen und das Zurückdrängen der Einflussfaktoren darf nicht gegen technische Anforderungen an strukturelle und funktionelle Grundeigenschaften der Basisvarian­te und naturgesetzmäßige Restriktionen verstoßen, welche durch die Gesamtfunktion der Basisvariante prinzipiell gesetzt sind. Andernfalls ruft das IDEAL einen anderen schädlichen technischen Effekt in einem anderen Systembereich hervor, der in der Regel ebenfalls einen spezifischen technisch-ökonomischen Widerspruch zur Folge hat. Sollte sich heraussteilen, dass die Beseitigung eines schädlichen technischen Effekts nur durch das Entstehen eines anderen möglich ist, so ist in jedem Falle „auszukundschaften", ob es einen dieser schädlichen Folge-Effekte gibt, gegen den ein ergänzendes IDEAL gedacht werden kann, das allen Anforde­rungen und Restriktionen des technischen Systems entspricht. In der Regel setzt dies aber eine eingehende Untersuchung der strukturellen und funktionellen Wechselbeziehungen des technischen Systems - zumindest im Umfeld des kritischen Funktionsbereichs - voraus. Um dabei eine Irrfahrt durch das technische System zu vermeiden, ist dieses erkundende Vorgehen daher nur sinnvoll, solange nicht zu weit über den ursprünglich abgegrenzten Systembereich hinausgegangen werden muss. Wird dabei ein entwicklungsfähiger IDEAL-Ansatz gefunden, so ist ein tragfähiger technischer Effekt (TE) und ein vermittelndes Funktionsprinzip (FP) zu suchen, die den neuen Bedingungen und Einflussfaktoren im System der \emph{aber} genügen. Hieraus werden die Teilfunktionsprinzipe ( TFP ) und das tech­nische Prinzip ( tP ) der erfinderischen Lösung für die Kernvariante (Schlüsselvariante) entwickelt und in einem Versuchsmuster erprobt.

Sollte jedoch kein entwicklungsfähiger IDEAL-Ansatz gefunden worden sein, so ist beim weiteren Vorgehen von dem Ansatz auszugehen, der am wenigsten gegen Anforderungen und Restriktionen im System der \emph{aber} verstößt. Die Erkenntnisse aus den Erkundungen des technischen Systems werden jetzt zum technisch-technologischen Widerspruch (TTW) zusammengefasst. (Siehe hierzu Erfindungspro­gramm, Abschnitt 7).


4.5.3. Der technisch-technologische Widerspruch und das neue Funktionsprinzip für die Schlüssel-Variante

Der technisch-technologische Widerspruch begründet den spezifisch technischen Sachverhalt, dass die Beseitigung des ursprünglich Vorgefundenen schädlichen technischen Effektes einen anderen, ebenso schwerwiegenden schädlichen Effekt notwendig hervorrufen muss. Zur Lösung dieses Widerspruchs werden nun die allgemeinen Problemlösungsprinzipe (PLP) in Ansatz gebracht. (Siehe hierzu Erfindungsprogramm Abschnitt 9).

Ist ein Lösungsansatz zur Überwindung des technisch-technologischen Widerspruchs (TTW) gefunden, so ist dieser zunächst am IDEAL bezüglich seines technischen Effekts (TE) auf prinzipielle Brauchbarkeit zu prüfen. Dann sind die \emph{aber} entsprechend zu modifizieren, und es ist darauf zu achten, dass dabei nicht gegen technische Anforderungen und naturgesetzliche Restriktionen ver­stoßen wird. Schließlich ist zu prüfen, ob der schädliche technische Effekt (STE) tatsächlich beseitigt bzw. kein neuer technisch-technologischer Widerspruch entstanden ist. Erst dann ist - bezugnehmend auf das IDEAL - eine Präzisierung des neuen tech­nischen Effekts (TE) und die systemgerechte Ausprägung des neuen Funktionsprinzips (FP) für die Schlüssel Variante vorzunehmen.

Wird ein brauchbarer Ansatz zur Lösung des technisch-technolo­gischen Widerspruchs nicht gefunden, so ist der technisch-naturgesetzmäßige Sachverhalt zu bestimmen, der dieser Lösung ent­scheidend entgegensteht. Hierzu wird die Systembetrachtung auf die kritische Wirkstelle der Schlüsselvariante gerichtet, von der diejenige technisch-naturgesetzliche Restriktion ausgeht, die maßgeblich am Zustandekommen des technisch-technologischen Widerspruchs beteiligt ist. Diese von der kritischen Wirkstelle aus­gehende technisch-naturgesetzliche Restriktion wird als schädlicher naturgesetzlicher Effekt (SNE) bezeichnet. Er besteht im Wesentlichen darin, dass ein an der kritischen Wirkstelle zu erbringender, funktionstragender bzw. gebrauchswertbestimmender technischer Teileffekt auf Grund des ihm zugrunde liegenden Wirkprinzips bestimmte funktionelle und/oder strukturelle Veränderungen des technischen Systems im Umfeld dieser Wirkstelle verbietet. (Siehe hierzu Erfindungsprogramm, Abschnitt 7).


4.5.4.  Der technisch-naturgesetzmäßige Widerspruch und das neue Wirkprinzip für die Schlüsselvariante

In einem Speicher naturgesetzlicher Effekte und Prinzipe wird nach solchen Lösungsansätzen gesucht, die den notwendigen technischen Teileffekt an der kriti­schen Wirkstelle in mindestens gleicher Höhe hervorbringen, ohne dass die ursprüngliche naturgesetzmäßige Restriktion aufrechterhalten werden muss. Natürlich ist dabei immer zu prüfen, ob nicht nur eine problematische Restriktion gegen eine andere eingetauscht worden ist. Diese Prüfung kann zunächst auf theoretischem Wege anhand eines technisch-naturwissenschaftlichen Modells der Wirkstelle und ihrer näheren Umgebung erfolgen. Dabei werden die ele­mentaren (funktionellen und strukturellen) Bedingungen und Zusam­menhänge untersucht, die für das Entstehen des notwendigen tech­nischen Teileffekts an der Wirkstelle zu schaffen sind. Hierbei stellt sich immer heraus, dass mindestens eine dieser Bedingungen aufgrund des gewählten Wirkprinzips uneingeschränkt erfüllt werden muss. Das heißt, sie ist als die neue naturgesetzmäßige Restriktion zu betrachten.

Ob diese neue Restriktion der Problemlösung entgegensteht oder nicht, kann festgestellt werden, indem sie im Systemzusammenhang der \emph{aber} dem IDEAL gegenübergestellt und untersucht wird, ob jetzt der schädliche technische Effekt beseitigt bzw. der tech­nisch-technologische Widerspruch gelöst werden kann. Ist dies der Fall, so erfolgt - ausgehend vom IDEAL - eine Präzisierung des neuen technischen Effekts und die systemgerechte Ausprägung des neuen Funktionsprinzips für die Schlüsselvariante. Bevor jedoch hieraus die Teilfunktionsprinzipe und das technische Prinzip ent­wickelt werden können, müssen die dem technisch-naturwissenschaft­lichen Modell (TNM) zugrunde liegenden vereinfachenden Annahmen und die dabei getroffenen Vernachlässigungen möglicher Nebeneffekte und untergeordneter Einflussfaktoren experimentell auf Gültigkeit und Zuverlässigkeit überprüft werden. Hierzu dient ein Labormuster, d.h. eine Nachbildung der Struktur des technischen Systems im Bereich der kritischen Wirkstelle.

Wird auch nach mehreren Ansätzen ein geeignetes technisch-naturgesetzliches Wirkprinzip zur Lösung des technisch-technologischen Widerspruchs nicht gefunden, so werden die dabei gewonnenen Erkenntnisse als technisch-naturgesetzmäßiger Widerspruch zum Ausdruck gebracht. Dieser begründet den problemspezifischen natur-wissenschaftlichen Sachverhalt, dass es für das technische System der Basisvariante kein Wirkprinzip gibt, welches eine technisch-naturgesetzliche Restriktion aufhebt, ohne andere, ebenso schwerwiegende hervorzurufen. Der Grund hierfür sind restriktive funktionelle und/oder strukturelle Bedingungen und Anforderungen des technischen Systems, welche auch neue Wirkprinzipe nicht zur Entfaltung kommen lassen.

Mit Hilfe der allgemeinen Problemlösungsprinzipien (PLP) wird nun versucht, diese Bedingungen und Anforderungen so „aufzuweichen", dass eines der in Betracht gezogenen Wirkprinzipien nicht mehr auf einen technisch-naturgesetzmäßigen Widerspruch führt. Damit ist dann auch der technisch-technologische und der technisch-ökonomische Widerspruch prinzipiell lösbar geworden. (Siehe hierzu Erfindungsprogramm, Abschnitt 9).

Hiermit beende ich – R. Thiel – im Juli 2017 die Wiedergabe des ProHEAL im Lehrmaterial zu den KDT-Erfinderschulen aus den Jahren 1988/89. Die originale Fassung musste mühsam digitalisiert werden, da half auch kein Scanning. Computer standen den ehrenamtlichen, außerdienstlichen Autoren 1988/89 nicht zur Verfügung. Im Lehrmaterial von 1988 waren darüber hinaus auf 34 Druckseiten „Spezielle Wegmodelle zum Erfindungsprogramm“ und auf 13 Druckseiten „Systemtheoretische Grundbegriffe der Erfindungsmethode“ sowie Literaturangaben abgedruckt, die vom Verdienten Erfinder Dr. Ing. Hans-Jochen Rindfleisch ausgearbeitet wurden. Vor allem ihm ist zu danken, dass mit ProHEAL eine sehr anspruchsvolle Anleitung zum erfolgreichen Erfinden entstanden ist. Diese Anleitung auch nach 1990 zu praktizieren scheiterte nur daran, dass im Gebiet der vormaligen DDR fünfundachtzig Prozent der Ingenieure plötzlich arbeitslos geworden und großenteils in die alten Bundesländer ausgewandert waren oder ihren Lebensunterhalt fortan berufsfremd als Versicherungsagenten verdienen mussten. Ihre vormaligen Industriebetriebe waren größtenteils stillgelegt oder in verlängerte Werkbänke westdeutscher Unternehmen verwandelt worden, von Forschung und Entwicklung befreit. Funktionsträger des VDI, denen das Projekt „Erfinderschule“ als Geschenk angeboten wurde, meinten, Erfinderschulen seien in der Bundesrepublik nicht möglich. Unser älterer Freund und Kollege, der geniale Verdiente Erfinder Ing. Karl Speicher, verzweifelte und schied aus dem Leben. Unser Freund und Kollege Dr. Ing. Michael Herrlich, Verdienter Erfinder und unser aller Bahnbrecher, versuchte sich jahrelang in technik-ferner Arbeit, bevor es ihm gelang, ein workshop-System „Erfinder-Akademie“ zu entwickeln. Unser jüngerer Freund und Kollege, der Verdiente Erfinder Dr. Ing. Hansjürgen Linde, der uns methodologisch am nächsten stand, nannte sein Potential „Widerspruchsorientierte Innovationsstrategie WOIS“. Er ging auf Wanderschaft, machte Erfindungen und workshops in einem Großbetrieb der westdeutschen Auto-Industrie und wurde bald von einer bayrischen Fachhochschule zum Professor berufen.

Seit Kurzem werden unser aller Erkenntnisse aufgegriffen von Professor Dr. Ing. Kai Hiltmann an der FHS Coburg sowie von dem jungen Master der Philosophie Justus Schollmeyer und vom Leibniz-Institut für interdisziplinäre Studien LIFIS in Berlin.

Rainer Thiel       info@rainer-thiel.de

\end{document}
