\documentclass[11pt,a4paper]{article}
\usepackage{od}
\usepackage[utf8]{inputenc}
\usepackage[ngerman]{babel}

\title{Erfindungen schaffen und sichern Arbeitsplätze -- Das Erfinden kann man
  bei der Deutschen Erfinder-Akademie lernen}

\author{Dr.-Ing. Michael Herrlich, Vorstand der Deutschen Erfinder-Akademie
  e.V.} 
\date{Etwa 2004} 

\begin{document}
\maketitle
 
Zum Bewältigen der Strukturkrise, zum Reduzieren der Massenarbeitslosigkeit
und der Insolvenzen benötigt die deutsche Wirtschaft wesentlich mehr
weltmarktfähige Produkte, die mit ökonomisch effizienten und ökologisch
optimalen Verfahren in großen Mengen produziert sowie gewinnbringend verkauft
werden.

Solche Produkte und Verfahren basieren auf Erfindungen, also weltneuen,
überraschend fortschrittlichen Problemlösungen, die, unter nationalen oder
internationalen Rechtsschutz gestellt, dem Inhaber ein Monopolrecht sichern,
was zum Zusatzgewinn führt.

Erfindungen können nicht allein mit Fleiß und Fehlervermeidung, wie bei vielen
anderen geistigen Leistungen erarbeitet werden, sondern es muss ein großer
Sprung in eine neue Informationsqualität, die Erfindungshöhe gelingen, die für
den Durchschnittsfachmann überraschend fortschrittlich, d.h. wesentliche
Mängel behebend, ist.

Einfach ausgedrückt: erst wenn die Lösung des scheinbar Unmöglichen gelingt,
kann eine Erfindung vermutet werden, die meist dann vorliegt, wenn die Lösung
mehr als eine Zehnerpotenz besser ist als das Weltbeste. Zumindest am Tage der
Erteilung eines auf alle Schutzvoraussetzungen geprüften Patentes ist folglich
der Erfinder absoluter Weltmeister auf seinem Gebiet. In Japan wird er deshalb
als kleiner Gott verehrt, weil er etwas Niveauvolles geschaffen hat, das
Menschen zur Befriedigung relevanter, notwendiger Bedürfnisse nicht von der
Natur direkt bekommen können.

Die Innovationsschwäche der deutschen Wirtschaft hat z.B. im Bauwesen mit dazu
beigetragen, dass durchschnittlich nur eine Umsatzrendite von 0{,}8\% erlöst
wird, Hochtief schafft weniger als 1\%, Holzmann und Walter sind pleite.

Die PIER-Hausbau GmbH Leipzig, ein Unternehmen zur Vermarktung eigener und
fremder Erfindungen, realisiert eine Umsatzrendite von über 8\%, weil
z.B. durch die erfinderischen, superleichten Porenbetonplatten auf die teure
und diffusionsbehindernde Wärmeschutzhaut sowie der wassereintragende
Innenputz trotz Überbietung der Energieeinsparverordnung verzichtet wird, was
bei höherer Qualität pro Quadratmeter Außenwand über 70 Euro spart.

Das Unternehmen kann dadurch massive, behindertengerechte
Öko-Energiespar-Solarhäuser mit Wärmepumpen, Regenwassernutzungsanlagen und
voll als Wohnergänzungsflächen nutzbaren Spitzböden binnen eines Quartals zum
Schlüsselfertigpreis von nur 1\,300 Euro/qm Nettowohnfläche an zufriedene
Bauherren übergeben, die bei nur 8\% Eigenkapitaleinsatz im Monat für Zins und
Tilgung weniger als die übliche Miete von 7 Euro/qm Nettowohnfläche bezahlen.

Der Vorteil der erfinderischen Energiespartechnik besteht darin, dass der
Bauherr Wärme für nur 2 Cent/kWh erhält, während er bei Gas-, Öl- oder
Elektroheizung bis zum Siebenfachem zahlen müsste und außerdem fast 40\% des
teuren Trinkwassers eingespart. Die Jahresbetriebskosten sinken dadurch
zusammen um über 1\,500 Euro.

Durch die Erfindungen gelang das sonst Unmögliche, nämlich höchste Qualität zu
niedrigen Preisen bei gleichzeitig optimaler Rentabilität zu realisieren.
Erfindungen sind folglich sowohl für den Produzenten als auch für den Kunden
hilfreich. Sie befriedigen heutige und zukünftige Bedürfnisse, sichern Gewinne
und Arbeitsplätze.

Deutschland war früher das Spitzenland der Erfinder.  Benz, Bosch, Diesel,
Otto, Siemens, um nur einige Erfinder des letzten Jahrhunderts zu nennen,
begründeten die moderne Welttechnik und sicherten damit Tausende von
Arbeitsplätzen und Wohlstand.

Auch bis zur Mitte unseres Jahrhunderts schufen deutsche Erfinder, wie von
Ardenne mit dem elektronischen Fernsehen, Zuse mit dem Computer oder
Mauersberger mit der Nähwirktech\-nik, mit der heute fast 30\% aller
Industrietextilien der Welt hergestellt werden, Bedeutendes, dann mussten sie
leider den Staffelstab der Weltbesten an Japan und die USA abgeben. Heute
schon gehören 2/3 aller Patente in Deutschland den Ausländern, die damit
Monopole besitzen.

Trotz der lobenswerten BMBF-Patentinitiative kamen 2003 nur 52\,425 deutsche
und 12\,093 europäische Patentanmeldungen aus unserem Lande (1930 waren es
80\,000), was wesentlich zu wenig ist, wenn man weiß, dass z.B. schon im Jahre
1993 mehr als 330\,000 japanische und rund 130\,000 amerikanische
Inlandspatente angemeldet wurden.

Noch schlimmer ist, dass weniger als ein Drittel der deutschen Anmeldungen
nach Prüfung auch erteilt werden, folglich echte und nicht nur vermutete
Erfindungen sind. Aus dem gesamten mit Milliarden gefördertem Bereich der
Hochschulforschung kommen nur knapp 5\% der Patentanmeldungen mit ebenfalls
nur 22-30\% Erteilungsquote.

Ein vergleichbar schlechtes Bild vermittelt die Statistik des betrieblichen
Vorschlagswesen.  Hundert Japaner bringen es auf rund 3\,000
Verbesserungsvorschläge im Jahr, 100 Amerikaner auf rund 35 und 100 Deutsche
auf nur 12.

Was sind die Gründe für diesen Missstand?  Es sollen aus der Vielzahl nur vier
genannt werden:
\begin{itemize}
\item Not macht bekanntlich erfinderisch; viele glauben, sie könnten durch
  mehr kosmetische Verbesserungen oder Kostensenkungen weiterexistieren. Sie
  werden aber schnell vom Markt bestraft, da die Bedürfnisse exponentiell
  wachsen und bereits nach kurzer Zeit keiner mehr Ladenhüter sogar als
  Geschenk annehmen will.

  Würden Sie z.B. heute noch in eine Rechenschieberfabrik investieren? xHier
  gilt die Weltweisheit „Wer nicht erfindet, verschwindet".
  
\item Das „Wurde nicht bei uns erfunden — kann daher nichts taugen — Syndrom“
  Der geniale Erfinder Karl Speicher hat z.B. kurz nach dem Krieg mit seinen
  Erfindungen zur wasserhydraulischen Dampfturbinenregelung und -lagerung das
  Problem heiße Maschine contra brennbares Öl auf raffiniert einfache Weise
  gelöst, so dass es in der DDR keinen Turbinenbrand mehr gab.

  Nach der Wende ignorierte die KWU diesen Fortschritt beim Kraftwerk Mitte in
  Leipzig und setzte wieder brennbares Öl zur Schmierung sowie Hydraulik
  ein. Ergebnis: ein Brand mit Toten und Millionenschaden.

\item Falsche Prioritätensetzung und zu geringe Erfinderförderung.
  
  Das BMBF-INSTI-Programm („Innovationsstimulierung der deutschen Wirtschaft
  durch wissenschaftlich-technische Informationen“) beachtet zu wenig, dass
  erst niveauvoll erfunden werden muss (der Teil Innovationstraining INTRA ist
  unterentwickelt), ehe man niveauvoll patentieren kann. Es ist außerdem
  unterkritisch finanziert, weshalb es auch nicht die Innovationsschwäche der
  deutschen Wirtschaft mit den daraus resultierenden vielen Insolvenzen und
  der Massenarbeitslosigkeit wirksam mildem konnte.

  Für die dringende notwendige Erfinderausbildung erhält die Deutsche
  Erfinder-Aka\-demie keine Förderung, die INSTI-Partner erhalten dafür
  maximal 5\,000 Euro.
  
  Damit ist der Innovationsrückstand der deutschen Wirtschaft gegenüber Japan
  und den USA nicht reduzierbar, wenn man bedenkt, dass alleine durch das seit
  1904 agierende Hatsumi Kyokai in Japan bereits im Jahre 1993 über 344 Mio.
  DM zur Erfinderförderung bereitgestellt wurden, folglich für eine
  erfolgreiche Aufholjagd wesentlich größere Anstrengungen notwendig sind.

  Wir haben daher vorgeschlagen, dass mindestens 0{,}1\% der geplanten
  F/E-Ausgaben vor Beginn der Themenbearbeitung in eine solide
  erfindermethodische Qualifizierung der F/E-Teams sowie in internationale
  Patentrecherchen investiert werden, was bei der Eröffnungsverteidigung des
  Pflichtenheftes kontrolliert werden muss.

  Nur so kann ein weiteres Zurückfallen Deutschlands von der Weltspitze sowie
  der jähr\-liche Verlust von fast 20 Mrd. Euro durch Themenabbrüche,
  Nachentwicklungen und Patentablehnungen vermieden werden.

  Man beachte bitte nochmals die Zahlen: weil ohnehin für F/E eingeplante 200
  Mio. Euro nicht für den richtigen Zweck der Erfinderausbildung ausgegeben
  werden, resultieren Verluste von fast dem Sechszigfachen! Das BMBF wäre
  daher gut beraten, wenn es schnell eine „Verteidigungsanordnung für
  F/E-Themen über 50\,000 Euro“ erlassen würde.

\item Die größte Katastrophe besteht darin, dass nur etwa 1\% der deutschen
  Ingenieure und Naturwissenschaftler heute noch niveauvoll erfinden können.

  Obwohl sich die Berufsgruppe der „Ingenieure“ (Lehnwort aus dem Lat./Franz.,
  voll eingedeutscht „Erfinder“) speziell zum professionellen Schaffen von
  Erfindungen aus dem Handwerk entwickelt hat, sind leider heute durch die
  dominante Theoriebezogenheit des Lehrkörpers weniger als 2\% der
  Technikprofessoren erfahrene Erfinder, so dass schnellstens durch Berufung
  von Erfinder-Unternehmern z.B. als Gastprofessoren, unterstützt durch
  industrienahe Stiftungen, entsprechendes Praxiskönnen als zukunftssichernde
  geistige Nährhefe den naturwiss.-technischen Hoch- und Fachschulen sowie
  Universitäten unseres Landes zugeführt werden muss.

  Das hätte außerdem den Vorteil, dass dadurch auch mehr
  Vorbildpersönlichkeiten der kreativen Menschenführung und des
  Innovationsmanagements vorhanden wären, denn niveauvolle Erfindungen
  entstehen zunehmend durch multidisziplinäre Teams in der Einheit von Wollen,
  Wissen, Können und marktbezogenem, aktivem Handeln.

  Durch angegliederte studentische Entwicklungsbüros könnten schon im Studium
  Existenzgründererfahrungen gesammelt und ausgeprägt werden.
\end{itemize}
Da im Osten die wirtschaftliche Not für Insider bereits ab den 1970er Jahren
deutlich zu erkennen war, haben wir parallel zum Aufbau eines wirksamen
Hochbegabtenfördersystems mit 16 math.-natwiss.-techn. Spezialschulen in den
Bezirkshauptstädten eine international beachtete Kreativitätsforschung
betrieben, Polytechniklehrer wie Ingenieure ausgebildet und ab 1980 die
postgradualen Erfinderschulen aufgebaut, aus denen nach der Wende die Deutsche
Erfinder-Akademie, zuständig für die Erfinderaus- und -weiterbildung im
gesamten deutschsprachigen europäischen Raum, entstand.

Fünf wesentliche Erkenntnisse der Kreativitätsforschung bilden die
wissenschaftliche Basis.

\paragraph{1.)}
Der Mensch denkt und verständigt sich im Sprachkreis mit allgemein bekannten
Begriffen und Bildern (kommunikativer Denkstil).

Da er diese vom Weltneuen nicht hat, entstehen meist nur wenige
Zufallserfindungen. Die „Barriere des Bekannten“ kann er leichter überwinden,
wenn er zusätzlich zum kommunikativen noch das funktionsbezogene Denken in
unseren Seminaren erlernt.

Die bewährten Rationalisierungsmethoden „Wertanalyse“ sowie die einfache
Erfindermethode „Systematische Aufwand-Nutzen-Optimierung (SANO)“ nutzen das.

\paragraph{2.)}
Alle niveauvollen Erfindungen sind „raffiniert einfache“ Problemlösungen, weil
sich bedingende und gleichzeitig widersprechende Forderungen, also
dialektische Widersprüche, wie z.B. heiß/kalt oder hart/weich durch bewusste
Nutzung naturgesetzmäßiger Effekte und Prinzipien (NEP) „fast von selbst“
auflösen und dadurch das scheinbar Unmögliche tatsächliche möglich machen.

Das Antrainieren des widerspruchsbezogenen Denkstils ist daher der Kern der
von uns durchgeführten Erfinder- und Innovationsmanagerseminare.

Ein Beispiel: Für Notevakuierungen von mehrgeschossigen Hotels, Kaufhäusern
oder Tribünen sind immer noch teure Nottreppen vorgeschrieben, obwohl sich bei
Panik dort durch Drängelei oder durch das Straucheln die meisten schweren
Unfälle ereignen. Werden hingegen die bei Mühlen seit Hunderten von Jahren
bewährten, preiswerten Sackrutschen eingesetzt, kann es nicht zu Verletzungen
kommen, da durch die Wirkung der Hangabtriebs- und Zentrifugalkraft
automatisch Dicke und Dünne mit gebotenem Sicherheitsabstand schnell nach
unten gelangen. Bei Flugzeugen sind Rutschen schon Norm, warum nicht eine
Übertragungserfindung bald auch im Hochbau?

Leider nutzen erfindermethodisch unqualifizierte Ingenieure und
Naturwissenschaftler maximal 10 NEP bei ihrer Arbeit, obwohl über 1\,000 NEP
bekannt sind.

Wir nutzen bei den Erfinderseminaren bis zu 250 NEP, so dass es kein Wunder
ist, dass über 23\% unserer bisher über 10\,000 Seminaristen bereits binnen
eines Jahres so niveauvolle Patente anmelden konnten, dass deren
Erteilungsquote mehr als das Dreifache des sonst Üblichen betrug.  Erfinden
kann man folglich, ja man muss es sogar erlernen!

\paragraph{3.)}
Der erfinderische Schaffensprozess endet nicht beim Patent (Edison sprach von
1\% Intuition und 99\% Transpiration) sondern besteht aus drei Phasen:
\begin{itemize}
\item dem rationellen Informieren über den Stand der Welttechnik und der
  zukünftigen, relevanten Bedürfnisse, der vorhandenen oder entstehenden,
  erfinderisch zu lösenden Widersprüche, der Planung der Zeiten und Kosten
  sowie dem Zusammenstellen und erfinder-methodischen Qualifizieren des zur
  Lösung erforderlichen, optimalen multidisziplinären Teams als ein quasi
  geistiges Schwungholen, was meist vernachlässigt wird,
\item dem methoden- und rechnerunterstützten Erfinden und Abfassen der
  Patentschrift,
\item dem optimalen Innovieren, d.h. Überleiten in die gewinnbringende
  Verwertung,
\end{itemize}
die alle sicher beherrscht werden müssen, soll sich der Erfolg einstellen.

\paragraph{4.)}
Erfinden ist eine naturwiss.-technische Kunst und kann daher nicht durch
Buchlektüre oder durch Massenvorlesungen, sondern nur durch Arbeit in
Kleingruppen unter Führung eines erfahrenen Erfinders als primus inter pares
an konkreten Problemstellungen in der Einheit von Wollen, Wissen, Können und
aktivem Handeln erfolgreich trainiert werden.

Zweckmäßig sind Zweitageseminare mit dazwischenliegenden Selbstarbeitsphasen
zum Ausprägen der Fähigkeiten.  Je nach Vorkenntnis und
Persönlichkeitsstruktur gelangte die Mehrzahl der Seminaristen nach fünf
Zweitagsseminaren mit dazwischenliegenden Selbstarbeitsphasen zum sicheren
Beherrschen des wirksamen, widerspruchsbezogenen, erfinderischen Denkstils.

\paragraph{5.)}
Da nur technische erfinderische Lösungen patentierbar sind, aber
Problemlösungen auf allen Gebieten notwendig sind, ist Erfinden wesentlich
mehr als Patentieren. Außerdem muss erst niveauvoll erfunden werden, ehe man
patentieren kann.  Die BMBF-Patentinitiative sollte daher schnell durch die
viel wichtigere Erfinderoffensive ergänzt werden.

Wenn man das erfinderische, widerspruchsbezogene Denken durch solide Anleitung
sowie genügend Üben so gut erlernt hat, dass es wie eine sicher beherrschte
Fremdsprache im Unterbewusstem verankert ist, sich folglich Können ausgeprägt
hat, verlernt man es lebenslang nicht mehr und hat gegenüber den
Unqualifizierten viele Vorteile, weil unser Leben oftmals problematisch ist.
Der Untrainierte erkennt meist Probleme zu spät oder gar nicht und hat keine
Waffen, um sie niveauvoll erfinderisch, d.h. raffiniert einfach zu lösen. Der
Erfolg ist dann nur Zufall.

Die Erfinder- und Innovationsmanagerseminare können nach schriftlicher
Bestellung und nach Erhalt der Bestätigung jeweils von Freitag, 11 Uhr bis
Sonnabend 16 Uhr am Monatsende in Leipzig absolviert werden und kosten pro
Teilnehmer 300 Euro.

Wenn die Seminare zu anderen Zeiten fremdorganisiert außerhalb Leipzigs, am
besten in Unternehmen parallel zum F/E-Prozess von uns realisiert werden,
berechnen wir pauschal für ein Zweitageseminar 2\,000 Euro plus Reise- und
Hotelspesen, so dass dem Organisator noch genügend Gewinn verbleibt.

Seminarinhalte:
\begin{itemize}[noitemsep]\enlargethispage{1em}
\item Gesetzmäßigkeiten der internationalen Bedürfnis- und Technikentwicklung
\item Das rationelle Informieren zum Stand der Technik mit dem Zuspitzen des
  Problems
\item Das methoden- und rechnerunterstützte Erfinden und Erarbeiten der
  Patenschrift
\item Das optimale Überleiten der erfinderischen Lösung in die gewinnbringende
  Verwertung
\item Das Bilden und Optimieren multidisziplinärer Erfinderteams
\item Mentale und finanzielle Unterstützungsmöglichkeiten für Erfinder und ihr
  Werk.
\end{itemize}
In jedem Seminar wird der gesamte erfinderische Schaffensprozess unter
verschiedenen Aspekten trainiert, so dass sich mosaiksteinartig ein
Könnensbild ausprägt.

Je zeitiger man mit dem erfindermethodischen Training beginnt, gute
Erfahrungen wurden z.B. mit Spezialschülern ab dem 16. Lebensjahr gewonnen,
desto leichter prägt sich das Erfolgskönnen aus.

Wir werden alles tun, damit bald das erfindermethodische Training an
aufgeschlossenen Technischen Hoch- und Fachschulen sowie Universitäten
durchgeführt wird, um die zukünftigen Ingenieure schnell zu befähigen, dass
sie wieder ihre berufsnamensgegebene Haupttätigkeit, nämlich das niveauvolle
Erfinden zum Wohle unserer Volkswirtschaft, möglichst zu 100\% realisieren
können.

In Brandenburg wurde mit dem INWORK-Projekt eine wirksame Maßnahme zur
kostenlosen Unterstützungen von KMU sowie Schaffung zukunftssicherer
Arbeitsplätze für arbeitslose Akademiker realisiert, die bundesweit
nachvollzogen werden sollte.

Mit Mitteln des Europäischen Sozialfonds und der BfA erhalten die Akademiker
nach dem Prinzip „Geld für Arbeit statt Arbeitslosengeld“ zwei Jahre lang ein
nicht üppiges, aber ausreichendes Einkommen sowie eine solide
erfindermethodische Betreuung, so dass sie erfinderisch weltmarktfähige
Produkte entwickeln und erproben können, die dann vom Praktikumsbetrieb
hergestellt und gewinnbringend verkauft werden.

Drei Vorteile ergeben sich:
\begin{itemize}[noitemsep]
\item Der Arbeitslose qualifiziert sich und verliert nicht den Kontakt zum
  Weltwissen.
\item Er schafft sich seinen, meist sogar mehrere zukunftssichere
  Arbeitsplätze und
\item Der Praktikumsbetrieb kommt zu gewinnträchtigen Produkten und
  stabilisiert sich dadurch. Seine Umsatzrendite wird sich mehr als
  verdreifachen, er entgeht der Insolvenz.
\end{itemize}
Wir bemühen uns, dass zukünftig INWORK in allen Bundesländern realisiert und
die bewähr\-te Erfindermethodik möglichst in allen Schulen ab Sekundarstufe II
sowie an den naturwiss.-techn. Höheren Lehranstalten durch erfahrene
Erfinder-Unternehmer den Schülern, Studenten, aber auch Dozenten antrainiert
wird.

Es wurde daher ein Förderantrag für eine intelligente Dienstleistung „Optimale
Nutzung des Humankapitals durch erfindermethodisches Training der Lehrer,
Dozenten, Schüler, Studenten, Ingenieure, Naturwissenschaftler,
Unternehmensberater, Manager und Meister“ gestellt, weil dadurch mehrere
positive Effekte erreichbar sind:
\begin{itemize}[noitemsep]
\item Die Lehrer und Schüler erhalten geistige Werk- oder Denkzeuge, die sie
  zum Problemerkennen und raffiniert einfachem Lösen sowie zum lebenslangem
  Lernen befähigen.
\item Es werden die zur erfolgreichen Lebensbewältigung notwendigen
  Kompetenzen (kreative, soziale, technisch-naturwissenschaftliche,
  informationelle, unternehmerische usw.) ausgeprägt, Kenntnisse der Technik
  vermittelt und zur Wahl entsprechender Berufe angeregt.
\item Ein zum Erreichen von Weltspitzenleistungen notwendiges
  Begabtenfördersystem Wissenschaft und Technik kann gegründet werden und
\item Studenten sowie Absolventen, aber auch Unternehmen lernen niveauvoll zu
  erfinden.
\end{itemize}
\enlargethispage{1em}
Unterstützen Sie uns bitte dabei wirksam, damit sich Deutschland wieder zu
einem Land mit innovationsstarken, gewinnbringenden, steuerzahlenden
Unternehmen und wenig Arbeitslosen entwickeln kann.

%\ccnotice
\end{document}
