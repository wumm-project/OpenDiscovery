\documentclass[11pt,a4paper]{article}
\usepackage{od}
\usepackage[english]{babel}
\usepackage[utf8]{inputenc}

\title{The \enquote{abstract} society. Systems Sciences as Salvation Message
  in Parsons', Dahrendorf's and Luhmann's Models of Society}

\author{Camilla Warnke}
\date{1974}
\begin{document}
\maketitle

\begin{quote}
  Camilla Warnke: Die „abstrakte“ Gesellschaft. Systemwissenschaften als
  Heilsbotschaft in den Gesellschaftsmodellen Parsons’, Dahrendorfs und
  Luhmanns. Akademie-Verlag Berlin 1974. Reihe: Zur Kritik der bürgerlichen
  Ideologie. Hrsg. v. Manfred Buhr, Nr. 46.
  
  Translated from the German version as available at the Max Stirner Archiv
  Leipzig\footnote{\url{http://www.max-stirner-archiv-leipzig.de/dokumente/WarnkeAbstrakteGesellschaft.pdf}}
  by Hans-Gert Gräbe
\end{quote}

\section*{Preface}

{\it 
  Me-ti often joked about the fairy tale that business leaders like to circulate
  about their indispensability.

  He said: You always display the economy extremely complicated to workers, and
  and indeed it is so, but only, as long as you are and complicate it. You
  yourself are the greatest complication. Your economy is completely without
  plan, one works against the other, one wins harming the other, and if you say,
  it is so difficult to make such plans, we must answer you that planning is
  completely unnecessary, even harmful.  Your complicateness is the
  complicateness of disorder, your job is to maintain and enlarge the disorder
  from which you make profit. These business leaders are just essential to the
  disorder, their thoughts are only valuable for exploitation.

  B. Brecht: Me-ti. Book of Turns}

Bourgeois sociology has taken a remarkable turn in the past few decades. It
turned to the construction of the \enquote{Great Theory}.

With this term, C.W. Mills \cite[64-92]{Mills1963} ironically describes
T. Parsons' attempt to overcome the narrow minded sociological empiricism by
means of a theory that ultimately explains the essence and functioning of
society.

“Large” is indeed the claim of this theory; “big” is the leap it takes: from
the level of the directly given empirical-sociological data to the functioning
of every possible society at all.

In the meantime, Parsons' example has caught on. Especially in the USA, but
recently also in the Federal Republic, one endeavors to develop a general
theory of society that expressis verbis or implicitly is to be understood as a
“modern” alternative to historical materialism.

All of these attempts have one thing in common: they ignore the dialectical
method, the “great method” -- as Brecht calls it in his “Me-ti. Book of Turns“
-- without which the Great Theory on society must necessarily miss its object.

This denotes the point of view from which the following Marxist criticism is
to be exercised: it wants to be fundamental in the sense that it reveals
hidden fundamental philosophical statements and brings to light the
implications of the “great theories” of bourgeois sociology. It wants to prove
that their declared \enquote{Forward, beyond Marx, to more modern social
  theories} in reality is a relapse on philosophical positions that have long
since been overcome.

This behaving as progressive \enquote{Forward to ...} becomes, because it is
philosophically reactionary, turn out to be politically reactionary, too, or
-- and that is of course the real determinative relationship -- the
politically reactionary goals of today's bourgeois society cannot be
philosophically founded differently than through reactionary philosophical
problem solutions.

If here, as a rule, the basic philosophical position and methodology of the
bourgeois sociology is assumed and the political consequences are derived from
it, so the sense of this mode of representation consists primarily in proving
that today any social theory that ignores or distorts the materialist
dialectic leads necessary to reactionary, or at least conservative, political
positions\footnote{From the standpoint of historical materialism, class
  theory, class struggle and theory of revolutions, B.P. Löwe \cite{Loewe1971}
  has dealt with the \enquote{great theory} of society (especially with
  Bentley, Parsons, Dahrendorf and Luhmann).  From the standpoint of Marxist
  sociology there is a disputation with T. Parsons by E. Hahn \cite{Hahn1965}.
  The mentioned works are well-founded contributions to a disputation of the
  so-called \enquote{great theory} of the bourgeois society, which I assume
  and to which I refer, so that my own contribution can be limited to mainly
  methodological questions.}.

\begin{thebibliography}{xxx}
\bibitem{Campbell1969} D.T. Campbell, Variation and Selective Retention in
  Socio-cultural Evolution, in: General Systems, Yearbook of the Society for
  General Systems Research, ed. by L. v. Bertalanffy, A. Rapoport, R.L. Meier,
  Vol. XIV., New York 1969, pp. 69-85.
\bibitem{Hahn1965} E. Hahn, Die Übersozialisierung des Menschen (The
  oversocialisation of the human being) -- T. Parsons, in: Soziale
  Wirklichkeit und soziologische Erkenntnis (Social reality and sociological
  insight), Berlin 1965, pp. 41-104.
\bibitem{Mills1963} C.W. Mills, Kritik der soziologischen Denkweise (Critique
  of Sociological Thinking), in: Soziologische Texte, Vol. 8, Neuwied/Berlin
  (West) 1963.
\bibitem{Loewe1971} B.P. Löwe, Zum Verhältnis von spätbürgerlicher politischer
  Soziologie und politischer Ideologie des Imperialismus -- dargestellt an den
  konflikt- und gleichgewichtstheoretischen Vorstellungen von A.F. Bentley,
  T. Parsons und R. Dahrendorf (On the relationship between late bourgeois
  political sociology and political Ideology of imperialism -- represented by
  the conflict and equilibrium theoretical ideas of A.F. Bentley, T. Parsons
  and R. Dahrendorf), PhD, Halle 1971.
\end{thebibliography}

\end{document}


positive

1 Cf.


2 From the standpoint of historical materialism, class theory, class struggle and revolutionary
theory has dealt with the "great theory" of society (especially with Bentley, Parsons, Dahrendorf and Luhmann) B.



Page 2
Camilla Warnke: The “abstract” society - 2
OCR text recognition Max Stirner Archive Leipzig - 11/27/2019
In other words: a social theory with progressive political implications can
can only be found in the materialistic dialectic. This claim - initially apodictic in the
Space provided - will have to be proven and should be proven through criticism of essentials and
common cornerstones of contemporary bourgeois "great theories" of society: an
their philosophically not or incorrectly reflected systems science [11] foundation. to
For this purpose, I consider it necessary to define your own point of view precisely. That
means 1. The position of the dialectic cannot simply be assumed to be known, but rather it
is to highlight and present those aspects of theirs that are of particular relevance to the
Confrontation with the bourgeois "great theory" of society are.
2. The attempt should be made to establish the relationship between dialectics and systems science
to investigate in order to use this as a basis for a fundamental methodological critique of each variant of the "large
Theory "that believes that the systems science through the philosophical foundation
to be able to replace it. I am fully aware that there is a lot in this chapter
is fragmentary and a number of statements are necessarily preliminary, hypothetical in nature
wear. The theses put forward in this context are intended as a contribution to the ongoing
sensitive discussion on the relationship between dialectical materialism and systems science
to understand. Despite numerous question marks, however, I am fairly certain that future problem solving
Solutions on this subject should be sought in the direction marked here.
Finally, the relationship between dialectical materialism and systems science is also
to be discussed because it matters to the theory and practice of our own society.
The political implications and consequences of incorrectly used cybernetics, game theory, sy-
stem theory etc., as negative instances, can very well contribute to our own location
to more clearly determine and relate to the viability and scope of these new sciences
their possible potencies for the general theory of society, historical materialism,
to investigate. In this respect, the following study aims to act as a constructive contribution to the discussion.
The bourgeois "great theory" is treated here in only one respect, in relation to the attempt
to establish a theory of society as a whole with the help of ideas and concepts that
are taken from the reservoir of systems science. The related elaboration
management, application and absolutization of the structural [12] functional (Parsons) or the functional
nal-structural method (Luhmann) must be largely excluded because its thorough
Investigation and criticism would be a company in itself. The same goes for the ones lately
constituent evolutionary theory of society, which - in connection with especially in the USA
developed approaches - Luhmann represents and those, in a modification of biological concepts
and in the social-Darwinian tradition, social evolution based on variation, selection and
Wants to reduce conservation mechanisms. 3 A Marxist Critique of These Concepts and of the
so-called functionalism is not yet available to my knowledge. But it is urgently needed
since their influence is growing.
The study is limited to criticizing the sociology of Parsons, Dahrendorf and Luhmann, because
these, even if they are contemporaries, three stages of development of bourgeois social theory
represent: Parsons the stage in which the imperialist bourgeoisie is still the illusory
Nary hoped to be able to achieve consensus in society through the compulsion to conform;

3 See DT Campbell, Variation and Selective Retention in Socio-cultural Evolution, in: General Systems, Yearbook of
the Society for General Systems Research, ed. by L. v. Bertalanffy, A. Rapoport, RL Meier, Vol. XIV., New York
1969, pp. 69-85.
Page 3
Camilla Warnke: The “abstract” society - 3
OCR text recognition Max Stirner Archive Leipzig - 11/27/2019
Dahrendorf the stage in which the internal conflicts of bourgeois society are so numerous
have become that this hope must be given up. The conflict becomes the subject
theoretical reflection; Dissensus and plurality are becoming basic features of society.
Finally, Luhmann represents the stage in which the survival of imperialism in the
Confrontation with the socialist social system on the cardinal problem of the civic
society: the relationship between the system and the environment takes center stage
social theoretical construction.
The selection was made at the same time from the point of view of demonstrating that social theory,
designed solely on the basis of the theoretical requirements of systems science, to
nem abstract, ignoring the qualitative specifics of the relations of production, class and power
leading social concept, which makes it possible to define structures and mechanisms of the state mono-
to explain political capitalism to the nature of society par excellence, with this
Process goes on regardless of whether by means of the concepts of cybernetics (Parsons),
vulgarized [13] game theory (Dahrendorf) or the general system theory according to L. v. Ber-
talanffy (Luhmann) is being worked on. The system terms used by Parsons and Luhmann are
vulgarized, blurred and out of focus, by the way.
Although they derive their origin from cybernetics and the theory of open systems in the sense of L.
v. Bertalanffy here, but - for ideological purposes immeasurably with regard to their scope
overstretched and transformed into the ontological - they experience a change of meaning underhand,
so that it is not at all those system terms of the system sciences that are used,
but their diluted derivative. I think this is necessary in order not to give opinions
To encourage those from ideologically denounced system terms such as scientific
The worthlessness of cybernetics, systems theory, etc. for the investigation of social
believing to be able to derive gears.
If the term "systems science" is used again and again in this work, it should
so that no defined term with a clearly defined subject is introduced. The term
acts as a stylistic means to avoid cumbersome enumeration and means regardless
Their differences from Chapter II onwards are cybernetics, general systems theory (in the sense of L. v.
Bertalanffy) and game theory, that is, those sciences whose ideas and concepts differ
suitable for the construction of bourgeois theories of society as a whole; in Chapter I - if from
the emergence of the systems sciences is mentioned - in addition, sciences such as
Operations research, queuing theory, etc., which are primarily of practical importance and
The task at hand is hardly suitable for the construction of social theories. In the
Discussion of the relationship between systems science and dialectics does not speak of the latter.
Their specifics are not included in the present work.
[15]
Page 4
Camilla Warnke: The “abstract” society - 4
OCR text recognition Max Stirner Archive Leipzig - 11/27/2019
I. On the relationship between dialectics and systems science
In the last few decades a wealth of new sciences has emerged that have one thing in common:
they do not resolve their objects of investigation into more and more specific detailed problems, but try to
to create objects in their holistic aspect, as systems, in terms of structure and functionality
grasp. To name just a few belong here: cybernetics, systems theory, systems research, information
mation theory, game theory, operational research, etc. with their respective terms and methods.
According to L. v. Bertalanffy, these sciences have their roots in technology, which
no longer in terms of the individual machine, but in terms of machine systems
thinks 4 in the complex organizations that the market, production, transport, etc.
sen have, in the sphere of the military, in tendencies towards the integration of the sciences, etc.
The systems sciences thus emerged primarily as a constructive answer to questions that
not the development of theory, but social practice and which
aimed directly or indirectly at mastering the scientific equipment
initially to provide relatively demarcated, later large systems. 5
This statement is important because without it it is impossible to understand why within decades
a new type of sciences emerge, develop and cover almost all object areas
could. "The concept of the system has penetrated all areas of science and everyday thinking.
gen “ 6 , says L. v. Bertalanffy and is right about it, even if the exuberant hopes
which many partisans of the new sciences linked with them, have not fulfilled.
Of course, in the development of the systems sciences, the inner one, which is often emphasized one-sidedly, also played a role
The logic of the development of science has its share in so far as it has prepared the view that
the objects to be examined are to be regarded as systems. To name just a few examples:
It was discovered that the atom is not the last indivisible element of matter, but a whole
zes, which has a specific internal structure and specific internal laws. It came into being
Gestalt psychology, which works against reducing perception to individual sensations.
suggested that perception is a holistic act. In biology the eye was
pay attention to the general structure and function of the organism. The brain physiology
In contrast to the one-sided localization theory, gical research developed an experimentally verifiable one
The point of view that the totality of the brain participates in every function of the brain, etc.
These impulses for the training of systems sciences only had from that moment on
Chance to produce a group of independent sciences than practical need
after mastering the social reality systems thinking in the form of independent, concrete
Applicable, practicable sciences made urgently necessary. Primarily under duress
Unavoidable extra-scientific needs made the leap from spontaneous approaches
for the constitution of a new group of sciences according to its type. Primarily the social
The systems sciences created practical practice by setting them up with concrete targets -
what was previously a secondary finding of the traditional individual sciences - to an independent research
direction of research. The practice thus had an organizing effect in that it
brought together; constituent by combining, mutual complementation and
fusion of insights already gained into the system nature of the research objects.
led, increased them to the power and thus created the prerequisite that independent system knowledge
could train people. And finally the practice had an institutionalizing effect, since [17] the
4 See L. v. Bertalanffy, General System Theory, New York 1968, pp. 3 f.
5 “Large (multivariable) systems include such material-energy systems or socio-economic systems
or organizational systems, which are characterized in their function by a highly meshed signal
fluß ... These large-scale systems are therefore in no way suitable for z. B. (spatially) large material-energy systems
bound. Rather, the term 'large' essentially refers to a complex flow of signals. ”- Cf. K. Szostak / W.
Kriesel, The philosophical problem of optimality and controllability in large systems using the example of the
rationsforschung, Diss., Berlin 1968, P. 90. Large systems are systems that have a high degree of complexity and
have great complexity.
6 See L. v. Bertalanffy, General System Theory, op. Cit., P. 3.
Page 5
Camilla Warnke: The “abstract” society - 5
OCR text recognition Max Stirner Archive Leipzig - 11/27/2019
Needs of technology, production organization, the military, etc. the ruling social
induced economic forces to finance systems science investigations.
This crude and rigorous assertion is of course not intended to ignore the fact that, for example,
wise the beginnings of game theory up to the twenties or that of the theory of open systems up to
go back to the thirties and as a consequence of certain mathematical or biological
problems arose. It should only make it clear that the real driving force, the last
Instance why preparatory work and approaches suddenly and within a short time lead to a whole
The addition of new sciences arose, was practically of a social nature.
It was not by chance that systems science emerged in the USA. Since in the USA the concentration
the process of production of all imperialist states had advanced furthest and the over-
The earliest entry into the state monopoly phase of imperialism were also here
For the first time, all those problems that arise with the control of large systems are acute
two levels: at the level of the productive forces that enter the scientific-technical stage
Revolution have occurred (problems of automation, business organization, traffic
etc.) and at the level of the relations of production.
Undoubtedly, it was primarily those with the scientific-technical revolution and the con
centering of the production needs arising, which the systems sciences as independent
Group of sciences organized and institutionalized: their practical branches and you
Basic knowledge of the nature and functionality of systems. The immediate occupation
their concepts and ideas on the part of bourgeois sociology, however, was of the effort
dictates the creation of a social theory that is adequate to state monopoly capitalism
on the one hand geared towards ideology that conforms to rule and on the other hand provides the tools,
to manipulate social processes in the sense of state monopoly capitalism. There
The latter is actually the most interesting level for our problem, something has to be said about it.
[18] With the transition to state monopoly capitalism, the economic pro
processes on the subject of government regulatory measures. Order in the fight for maximum profits
The most successful monopoly groups in each case subordinate the state organs of power to order
to increase their capital accumulation from state funds in order to get the orders of the state
To secure consumption in order to support the price level, manipulate wages, etc. The state intervenes
in favor of the financial oligarchy directly in the capitalist reproduction process
tries to direct him. This intervention of state power in the economic sphere is necessary.
It has become dig since modern reproduction on the basis of the level of development of the productive forces
Requires funds that exceed the capabilities of individual monopoly associations.
The state provides the means required for capitalist accumulation through taxes, price increases
struggles, etc., which are to the detriment of the people, that is, through redistribution of the social
Wealth in favor of the monopoly bourgeoisie, with the executives of the monopoly groups and the
of the state apparatus are so closely intertwined that they represent a collective exploiter.
The amalgamation of economic with political power increases the influence of the state
all areas of social life including the ideological sphere. Aimed for and
The integrated society is propagated, in which all members think and act in accordance with the rule.
because the class struggle has disappeared from it. But as the state monopoly capital
lism through the basic contradiction between social production and private appropriation
he reproduces the growing socialization of production at a higher level
at the same time the class struggle. The middle classes, and as a result the scientific-technical
Revolution the intelligentsia, with regard to their objective social position, they approach the
situation of the working class, since the problems of education that are insoluble under imperialist conditions
education, housing policy, transport, environmental pollution, etc. the non-
meet imperialist strata and classes alike. 7th
7 Cf. Der Staatsmonopoly Kapitalismus, Berlin 1972, pp. 65-67, 70-72.
Page 6
Camilla Warnke: The "abstract" society - 6
OCR text recognition Max Stirner Archive Leipzig - 11/27/2019
[19] State monopoly capitalism reacts to these conflicts by giving the state its own
Performs control and integration functions even more intensively; as long as possible, by means of the
Strategy of dampening, paralyzing and channeling conflicts that disrupt the system, ideally in
to convert system-sustaining forces. Of course, as H. Holzer rightly remarks, this only succeeds,
"If ... the conflicts that arise are not of such a fundamental nature that with them the
the institutional framework of conflict regulation itself is called into question ” 8 , or if so, with others
Words, "consensus on the binding validity of the institutions created" is guaranteed,
on the other hand, "conflict and dispute only within the framework of this consensus-secured
institutions ”occur. 9
Within this delineated framework, the system of state monopoly capitalism wandered
flexible, flexible in terms of his strategies. It has to be flexible when it comes to the dynamics of the
withstand the social-technical revolution with its social consequences and in the dispute
wants to survive with the socialist system. But this is also reduced in this context
Scope “on criticism and formulation of preventive measures within a principally not
touchable frame “ 10 . The competing variants in the politics of the state monopoly
On closer inspection, capitalism only turns out to be different solutions for one and
same problem (survival of capitalism), although there is agreement that the social
The quality of the system must be taken for granted, and disagreement only in
what priorities and interdependence between the tasks to be solved. Willy Brandt
said in 1961, shortly after the Godesberg program was passed, quite in this spirit: “In
In a healthy and evolving democracy it is not unusual, but there is
It is normal for the parties in a number of areas to have similar, even substantive, demands.
gen represented. The question of priorities, the ranking of the tasks to be solved, the methods
and accents, that is becoming more and more part of the content of the formation of political opinion. ” 11
[20] Their socially determined developmental context, their genetic connection with
systems science does not prevent scientific knowledge
nisse, objectively true statements regarding the structure, functioning and behavior of systems
to formulate that have socially independent validity. So little, for example, through
the advances in logic made by medieval thought are invalid because
they emerged from the apologetic motive to add contradictions within Christian dogma
Eliminate, the systems sciences are just as little reactionary, because “they are caused by interest
a reactionary class received their impulses and are taken into service by it. Between
Between their creation and application there is a space within which they keep themselves free
of apologetics, in which they unearth objective knowledge about objective reality.
if they are to be applicable. And this space is wide. It includes in relation to the
Systems science System issues of technology as well as the human psyche, the
Integration of science as well as environmental research. It ranges from special system
science to attempts to create a general systems theory. Once created
the systems sciences also show a relatively independent development; you will be
applied and further developed independently of their specific social origins
refer to and cover fundamentally in terms of the structure, functioning and behavior of systems
established legal relationships.
Even if the euphoria of the early days, in which one turned away from the systems sciences, is over
Miracle hoped for 12 , the conclusion that remains is that it represents a qualitative leap in development
8 H. Holzer, Failed Enlightenment? Politics, economy and communication in the Federal Republic of Germany, Munich
chen 1971, p. 31.
9 Ibid, p. 30.
10 Ibid.
11 W. Brandt, Plea for the Future, Frankfurt (Main) 1961, p. 20.
12 “The death of Norbert Wieners provides an opportunity to establish that this heroic dream has come to an end. cybernetics
is a fundamental as well as popular idea in the best sense of the word, but it has been shown to be less comprehensive and
Page 7
Camilla Warnke: The "abstract" society - 7
OCR text recognition Max Stirner Archive Leipzig - 11/27/2019
of the sciences mark the systemic aspect of reality, insofar as they have opened the way
to be able to grasp scientifically in its diverse moments. Let that be expressed at this point
lich emphasized against pseudo-revolutionary left-wing phrase, which with the concept of practice against the system
stembegriff intervenes and believes subject and humanism before the grasp of imperialist power
To be able to save politics if it denies the systemic character of society, the concept of the system
devalues ​​or belittles it in terms of its scientific importance.
In the discussion with N. Luhmann, J. Habermas takes the shorter one when he opposes him
Knowledge of the functional relationships of the state monopoly society based -
Theory in which the individual becomes a fully-
is constantly integrated into the system ("The subject must first be a contingent selective
activity ” 13 ), as a social path to salvation the“ repressive-free discourse ”
want. According to Habermas, this is “... subject to the claim of the cooperative search for truth, ie the
in principle unrestricted and informal communication, which is solely for the purpose of understanding
It serves ... The discourse is not an institution, it is simply a counter-institution. Therefore he leaves
do not see themselves as a 'system', because it only works under the condition of suspend-
the compulsion to obey functional imperatives ”. 14 Luhmann doesn't have a special one
Difficulty to prove that this "informal communication" does not exist, that discussion is also a
Organization represents, insofar as it develops certain hierarchies, constraints, such as, for example
that of sticking to the topic, etc., that is, works as a system that does not work from the social context
can be extracted. 15th
The administrator of counter-Enlightenment rule 16 always has the stronger troops than him
Subjectivist. because he takes note of reality, makes use of the sciences to
to preserve and defend the societal conditions. The subjectivist, on the other hand
deprives itself of power by giving reality the mere abstract idea of ​​a counter-reality
opposes the feasibility of which he neither measures nor measures against the social possibilities
socially organized. The sensible demand to make this social system revolutionary
change, is inverted here into the call to abolish the systemic character of society.
For Marxism-Leninism, both attitudes to the systems sciences are unacceptable.
As a revolutionary theory that strives for the social goal of changing the social [22]
If it is necessary to fundamentally revolutionize the situation and build a new social system, it must have structure
and functioning of both society that is destroying and society that is destroying
is strangely less interesting than we hoped when the idea came up 20 years ago ”. - J. Bronowski,
Review of "Brains, Machines and Mathematics" by MA Arbib, Scientific American, July 1964, No. 4, p. 130.
13 N. Luhmann, System Theoretical Argumentations, in: J. Habermas / N. Luhmann, Theory of Society or Social
technology - What does systems research do ?, Frankfurt (Main) 1971, p. 327.
14 J. Habermas, Theory of Society or Social Technology. A discussion with Niklas Luhmann, in: J.
Habermas / N. Luhmann, Theory of Society or Social Technology, op. Cit., P. 201.
15 Cf. N. Luhmann, Systems Theoretical Argumentations, loc. Cit., P. 327 ff.
KH Tjaden is right when he writes in a review of the controversy just quoted: “In times of
State monopoly regulation of capitalism is unfortunately the ideologue of bourgeois democracy - one has to
regret, especially as a Marxist - quite in retreat from the theory of self-regulating systems ", in: Argument,
Karlsruhe 1972, no. 1/2, p. 158.
16 In his manuscript, which unfortunately only became known to me after the completion of this manuscript, WR Beyer aptly describes it
Study: "On the sense or nonsense of a 'reformulation' of historical materialism" the political and ideological
Luhmann's position when he writes: “His (Luhmanns - CW) theses are currently considered to be the most dangerous in
apply to the FRG because it is supposedly neutral and modern observance under the guise of a sophisticated system
Throws away the old slow-moving reaction, with a bold gesture, but replaces them with future prospects that are in
in no way show the slightest consideration for the political demands of the working class. He pays homage
class optimism when it 'trusts' that' others (groups, people) have their particular expectations
which they cannot even have now, in accordance with the developments in a common future.
adjust according to and complementary to their own expectations. ' Because: this 'expectation' that the contrary
Letting class antagonisms 'expect' a common future is precisely the 'expectation' of the monopolistic, oligopo-
cunning or state monopoly capitalism , not socialism, which ... for the future the fulfillment of its
and only expected his 'expectations' ... "(Berlin 1974, p. 39)
Page 8
Camilla Warnke: The "abstract" society - 8
OCR text recognition Max Stirner Archive Leipzig - 11/27/2019
is to be built, know. To do this, it uses, among other things, systems science and
Further cultivates them for their own purposes, as has been the case for a long time, especially in the Soviet Union, and for many
Levels happening. Of course, this cannot be done under the auspices of an anti-enlightenment stance.
happen, according to which the system is everything, but the individual is nothing, not in abstraction from the goals
and the specifics of the socialist society and not independent of the Marxist-Leninist
philosophy and social theory.
K. Hager criticized certain tendencies to distort the relationship between systems science and
ten and Marxism-Leninism as follows: “The consequence of a schematic application
cybernetic methods on the analysis of social processes is that the dialectic than that
indispensable theoretical basis for the correct assessment of the extremely complicated interrelationships
between the material and the ideological relationships, between the base and
the superstructure, between society, class and personality, etc. is eliminated. The dialectical
Analysis of the living contradicting social processes is carried out by a functional
schema, whereby it is not uncommon for the specific, social, class-related nature of the societal
economic processes ... is disregarded ... To move forward to the cybernetics on the
Fundamentals of Marxist-Leninist social theory with all its potentials to full
develop and use them effectively in solving the mature tasks, we need inside-
gend creative discussions about the open development problems of cybernetics and the
other organizational sciences. " 17
These statements by K. Hager imply the following problems which are important for our subject:
What is the relationship between systems science and materialistic dialectics, system
science and historical materialism, whereby this question includes the specific
to grasp the essence of systems science.
[23] As a scientist. and philosophers of the socialist camp the great importance of sy-
When they recognized stem sciences, they initially saw them as brilliant confirmation of certain ones
Statements of the materialistic dialectic and the possibility of using terms such as "interaction",
"Determination" etc. to be recorded more precisely than was previously possible. In the first over-
the appropriation over the identities swung the differences between the materialistic di-
lectics and systems science are trivialized or completely overlooked. An attitude dominated
which, in relation to cybernetics, was articulated as follows: “This science brings everywhere
unconsciously and spontaneously expressing dialectical-materialistic trains of thought. But that means
cybernetics represents in its entirety, in its scientific core (and this core is so
massive and unshakable that the other, 'the waste', the reactionary philosophical failure
need, the epistemological errors of important Western cyberneticists, etc., next to it. as
can be considered irrelevant) a for the philosophical abstraction in the sense of the dialectical
Roman materialism is already largely mature material and must be considered one of the
most impressive individual scientific confirmations of dialectical materialism that it
has ever existed up to now. ” 18
The tendency to increase the differences between materialistic dialectics and systems science
blurring, is also marked in my opinion in O. Lange's well-known work “Wholeness and development in
cybernetic point of view ". Since Lange only identifies the identities but not the differences between Ky-
emphasizing bernetics and dialectics, he suggests that the cybernetic description of different
of the following system states of a system developing in the sense of ontogenesis directly
bar to equate with the Marxist theory of development. 19th
That the philosophical reflection of the concepts and methods of the systems science does not arise
could stop at this first stage is conditioned by the nature of the materialistic dialectic,
which does not allow identity to be thought without its inherent difference and opposition
17 K. Hager, Basic Questions of Spiritual Life in Socialism, Berlin 1969, p. 48 f.
18 G. Klaus, Cybernetics in a philosophical perspective, Berlin 1965, p. 22.
19 Cf. O. Lange, Wholeness and Development in a Cybernetic View, Berlin 1966.
Page 9
Camilla Warnke: The "abstract" society - 9
OCR text recognition Max Stirner Archive Leipzig - 11/27/2019
is. After the fundamental compatibility [24] of materialistic dialectics and systemic
Science had been established, it was about being philosophical at a higher level
Reflections on the one hand to seek out their differences, on the other hand to seek out their actual relationships.
This was all the more necessary as there were already critical voices in circles of bourgeois scholars
had become loud, which in the expansion of cybernetic and other systems science
the formation of concepts and methods for the worldview rightly represent an extension or return of the
saw chanicism at a higher level.
L. v. Bertalanffy, who proved that the absolutization
such cybernetic terms as "homeostasis", "adaptation" for the objective function of systems
at all in the field of biological 20 and social to a machine theory, a
Mechanicism of a new type of reading that has explosive political consequences if it gets into the psy-
chology and social theory intrudes. L. v. Bertalanffy shows this, among other things, with a
widespread current in American psychology, the psyche as homeostatic, equal
understands important system that asked to adapt to the benchmark environment. But is psyche like that
defined, then their functioning can only be seen in the inner equality
to restore the state of weight that has been disturbed by the conflict and tension-creating environment.
represent what will be achieved through improved customization. That is, according to this model, psychological
cal conflicts, the contradiction between humans and the social environment is in any case the same
solve that man does not rule his environment, but himself in the sense of the given rules.
changes the norms so that he can find his inner contradictions - including the productive and creative -
gives up in favor of its conformity with the circumstances. 21
From a Marxist point of view it should be added: In these conceptions the dialectic of the individual becomes
and society destroys and in its place the unidirectional linear control of the human being
set on the part of the objective conditions. Instead of the world-changing activity of man
in which he realizes his development, his mere adaptation to the given occurs, a pseudo
development, in [25] which his activity is only against himself - to eradicate his opposition
can judge sayings and he therefore remains pseudoactive. L. v. Bertalanffy is right when he says this
Conceptual design referred to as a “computer” or “robot model” 22 , in which the theoretical
the basis for the manipulation of the masses sees the possibility of using psychology to
to make long reactionary financial and political interests; 23 he is right when he thinks
that homeostasis is an explanatory principle of an economic-commercial philosophy that conforms
ity and opportunism are the ultimate values. 24 However, L. v. Bertalanffy in his constructive
tive, system-theoretical counter-draft the dialectic of man and social environment
also not to be grasped. If the "robot model" takes control of the individual on the part of the
world absolutized, so L. v. Bertalanffy - in the humanistic concern, the theoretical one
To reject the basis and practices of manipulation - to the psychic system
Attributing spontaneous activities completely independent of the environment. For the level of
20 The basic philosophical position of this conception of the organism is mechanicistic. Organism is seen as the same
weighty, harmonics considered, which cannot tolerate internal contradictions, which - if disturbances from the environment
genes occur that lead to internal system contradictions - these must be eliminated as quickly as possible. For this view
there are no internal contradictions of the organism, only the external contradictions of the organism and the environment that have been resolved
as the organism adapts.
L. v. Bertalanffy pointed out the weakness of this conception: “Life is not restoring
or maintaining balance, but essentially maintaining what results from imbalances
Determination of the organism as an open system follows. Reaching equilibrium is equivalent to death.
When, on the occasion of a disturbance from the outside, life has simply always returned to what is known as equilibrium
if it had, it could never have risen above the amoeba, which is, after all, the best-adapted creature. "General
System Theory, op. Cit., P. 191 f.
21 Cf. ibid., P. 187 ff .; L. v. Bertalanffy, ... but we don't know anything about humans, Düsseldorf / Vienna 1970, pp. 19-37.
22 Cf. L. v. Bertalanffy, General System Theory, op. Cit., P. 191; L. v. Bertalanffy, ... but we know about people
nothing, loc. cit., p. 24.
23 Cf. L. v. Bertalanffy, General System Theory, op. Cit., P. 191.
24 See ibid., P. 211.
Page 10
Camilla Warnke: The "abstract" society - 10
OCR text recognition Max Stirner Archive Leipzig - 11/27/2019
In neurophysiology he puts it this way: “... that the autonomic activity of the nervous system ... as that
Primary to look at is. ... The stimulus ... does not cause a process in an otherwise immobile
he only modifies processes in an automatically active system. ” 25
This view - if it is absolutized - amounts to the internal conditions
and to hypostatize contradictions, whereby against the mechanicism not the dialectic, but
irrationalism is brought up. Separate from the interaction of the environment is the
human activity is deprived of its objective function. It becomes linear as in the case of the single-minded
Control by the environment thrown back on itself what was thought through to the end in the periphery
existentialist thinking leads, in which "venture", "risk", "failure" and the like of their own
are dignified behavioral values ​​that are detached from the concrete goal.
At this point at the latest it becomes clear that the enthusiastic equation of philosophical categories
rien with system-scientific concept formations falls short that the stopping on this
Stage inevitably to dilute and [26] flatten the dialectic, to relapse into the
chanicism must lead. It makes a fundamental difference whether questions like
which, according to the relationship between part and whole, of causality and purposefulness, etc.
are embedded in the text of a philosophical system and have their place and value in it.
zen or whether they, taken separately and on their own, are included as independent objects of the investigation.
step. In the first case they are eo ipso mediated on all sides within the philosophical system.
If the thing and the how of its existence is presented as a system, then it is within the framework of the
relevant philosophy at the same time determine what things are according to their origin:
External expressions of transcendent beings, subjective ideas or through their collective
hang with other things materially conditioned objects. Then the systemic nature of things becomes
not reflected independently of the views expressed in relation to movement and development
exist, whereby the last-mentioned formulations must be seen in the closest connection.
sen, since the respective concept of thing and the respective concept of development determine each other mutually.
That changes when the systemic problem becomes independent in the systems sciences and
breaks away from the philosophical context. Not that it stopped, constituent part
to be of philosophy; just as little as quantity and measure cease to be categories of philosophy
to be, although they are special subjects of mathematics. But independently
the above-mentioned philosophical questions now arise in the systems sciences.
from the outside, as it were, as philosophical prerequisites to be clarified and as necessary
Advance understanding, for example, with regard to the question of what the systems science actually
form. Of course, this question can now - and that was not possible before -
socially meaningless and insignificant questions are explained that do not need to be answered. That
is the standpoint of positivism, which usually leads to the fact that the after the ratio of
Turning the image and the depicted not "questioned" terms into ontological statements.
[27] This danger lies, for example, with Mesarović's approach to a general systems theory
close, which is to be developed on the basis of formal logic and set theory. Mesarović
emphasizes that everything "what systems theory does (consists) in the formal relation (ie the sy-
stem), which is already implied in the statements of the theory. The benefit is self-evident
of course in the deductions that one can make starting from the formal system,
and which one cannot make so easily and clearly if one starts from the statements themselves ... That
The method of applying systems theory then consists in using the statements about the behavior
starting from a real system, deriving the underlying formal system and then the
To analyze the behavior of such systems in order to shed light on the observed real appearance ” 26 .
Here, of course, there is no polemic against a formalized concept of system and its application.
but rather asserted with R. Schwarz that “in this respect the actual measuring process and
25 Ibid, pp. 208, 209.
26 Systems Theory and Biology - View of a Theoretican. In: Proc. III. Systems Symp. At Case Institute of Technology,
ed. by MD Mesarović (West) Berlin / Heidelberg / New York 1968, pp. 78, 79.
Page 11
Camilla Warnke: The "abstract" society - 11
OCR text recognition Max Stirner Archive Leipzig - 11/27/2019
its prerequisites are not in the field of vision ... the relationship between the statements of the general
Systems theory to objective reality is greatly simplified ”. 27 The philosophical problem of
between the concept of the system and objective reality is not reflected, which leads to
to interpret the strict concept of system as an ontological fact. That is, the math
Relation appears to be immediately discoverable in objective reality, which as a consequence
amounts to viewing the world as a "multiplicity of things with fixed properties" 28 ,
whereby the dialectical concept of thing is suspended and the metaphysical substance accidental
zien model would be restored.
It is not the place here to investigate the often-voiced and accurate assertion that the sy-
stem thinking in its historical genesis original ideas mainly based on dialectical pro-
problem-solving orientated philosophies. To prove that would be an issue in itself. But be it
at least as much said that the way leading to the explanation of the objective reality from their own
Has led to contexts and which is the way to dialectical-materialistic explanation of the world,
at the same time brought about an understanding of the systematic nature of the world and things. The above
A long common history of dialectics and systems thinking can even be found in H.
Rombach's functionalistically recorded history of the terms "system", "structure" and "func-
tion "which, unintentionally by the author and without the dialectic appearing at all,
Contains remarkably rich material on its history. 29
Incidentally, this fact is the origin of the spontaneous and unreflective equation sketched above.
setting of systemic knowledge and dialectical-materialistic concept formation,
which is indeed wrong in its absolutization, but which for the reasons indicated still unites
contains rational core. In the context of our topic, this relationship must be neglected. It
can only be about the concrete general system concept of dialectical materialism
to be distinguished from the abstract, general system concept of the systems sciences.
divorce.
In my opinion, the system concept of dialectical materialism is primarily linked to this
Thing concept. It gives the answer to the question of the mode of existence, the nature of things.
For the materialistic dialectic, things are by definition systems of qualities. 30 This
mood implies that the thing conception of dialectical materialism is a qualitative one, that
means that, in contrast to mechanical materialism, the concept of thing is not
ability and space fulfillment is bound. 31 It implies - also in contrast to the mechanical
Materialism - furthermore the indifference of the concept of thing to material and ideal phenomena.
It also borders on the substance-accidental-thing model of Thomistic character
insofar as they do not separate the thing into a monolithic, unchangeable core (substance)
and allows a shell of external, changeable properties (accidents). 32 It contains further
ter that things are not as a sum of indifferently related determinations, but rather
be understood as structured (in the majority of cases hierarchically structured) wholes. she
includes the knowledge that the concept of thing is relative, that the “parts” of the thing (its
Qualities, insofar as they are also of qualities, in other contexts than things
27 R. Schwarz, Philosophical and Methodological Problems of General Systems Theory, Diss., Berlin 1970, p. 79.
28 See ibid.
29 Cf. H. Rombach, Substance; System, functionalism and the philosophical background of modern science,
Vol. I, II. Freiburg / Munich 1965-66; see the review by G. Kröber / C. Warnke, in: Deutsche Literaturzeitung, vol.
91/1970, H. 6, pp. 486-489.
30 Cf. Philosophical Dictionary, ed. by G. Klaus and M. Buhr, Leipzig 1971, vol. 1, p. 252; see also AI Ujomov,
Properties and Relations, Berlin 1965, p. 17.
31 Cf. AI Ujomov, things, properties and relations, op. Cit., P. 4 f.
32 Cf. J. Gredt, The Aristotelian-Thomistic Philosophy, Freiburg i. Br. 1935, Vol. I, p. 146. Here the Thomistic
Relationship between substance and accident is characterized as follows: "(The accidental form) is the
actually determining and realizing form. While the substantial form is the first being, the substantial being,
which gives being par excellence, the accidental form gives only a second-line being, a secondary being, a being in
some relationship. The accidental form is therefore the secondary reality (actus accidentialis), which is the
Realization ability or possibility (potentia) of the body substance is determined or realized as a secondary matter. "
Page 12
Camilla Warnke: The "abstract" society - 12
OCR text recognition Max Stirner Archive Leipzig - 11/27/2019
must be considered. 33 By using the concept of a thing in terms of qualities and not of one's own
that the identity of things with themselves is based on their essential properties
properties is due.
And if we finally include the concept of quality (= essential property)
Explicit the concept of property, we come to the insight that for the dialectical material
rialism that things are systems that are open to their environment. Hegel already realized this
formulates: “A thing has properties; they are first of all its determinate relations to other things ;
the property is only present as a mode of behavior towards one another, it is therefore the external
lich reflection and the posited side of the thing. But second , the thing in this genre is
being in oneself ; it is preserved in relation to other things, so it is only a surface
che, with which existence surrenders itself to the becoming of being and to change; the property
does not get lost in it. One thing has the property of effecting this or that in the other and
expressing himself in his relationship in a peculiar way. It only proves this quality
under the condition of a corresponding quality of the other thing, but it is to be
equally peculiar and its basis identical to itself. ” 34 That is, the system of qualities
it comes to the thing objectively, as its immanent determination, but not independently and
outside of interaction with other things.
We have the concept of the thing by means of the category of quality, i.e. by the concepts of the
sens and the property, determined. The thing thus proves itself to be a system, a range of dimensions
possesses the ability to be able to maintain identity in the face of changes, but also the
Ability to become another thing.
In a closer and more precise definition, things are therefore relatively constant material or ideal sy-
stems of qualities. The problem that arises in systems sciences (such as cybernetics, universal
common systems theory) is, in my opinion, in the area of ​​the philosophical
Thing- [30] term to be found. But it is already evident at this point, all of the things connected with it
Regulations in systems science are taken into account and processed. So will be in them
abstracted from it, whether it is the system or the structure, in relation to a certain thing
is determined in order to determine the essential connections for this thing, or to which it is actually constitutive
ende structure acts. The structure determined can be an essential or
essential, since from the abstract specific task of systems science the
Question about the essence of the subject is irrelevant. The object of the systems sciences are yes
not the things, but the systems, structures, the behavior of systems, etc. Not that
philosophical question about the nature of things is up for discussion, but rather the systemic
Scientific problem of the nature of systems. In other words: while in philosophy
things are taken in the multiplicity of their general determinations, their systemic
character represents only one aspect, the question of the mentioned systems sciences is:
what are systems like, how do they work, the answer not being the philosophical one
Thing concept concerns in the totality of its aspects, but leads to the explication of what for
Heard around the concept of system and deepened it. The systems sciences relate to
the qualities (the essential properties) of the thing with the help of which it is philosophically comprehensible
becomes indifferent. In the systems sciences, qualities only occur in their general determinati-
to be properties, but in their concrete determination they become essential properties
properties are not taken into account, since the selection of the structure of the system to be investigated is not
from the point of view of the object, but from the point of view of the respective system task.
33 Cf. AI Ujomov, Things, Properties and Relations, op. Cit., P. 18: “A qualitatively understood thing exists
as well as a thing in the traditional conception of parts. But these parts are not parts of the room, they are
Parts of a system of qualities. In this respect, these parts for their part again. Systems are of qualities, they represent
if there are special things. B. the magnetic and electrical components of an electromagnetic
View the field as special things. These two parts form a whole, but not in the spatial, but in the
qualitative sense as two subsystems of a unified system of qualities. "
34 GWF Hegel, Science of Logic, Part Two, ed. by G. Lasson, Leipzig 1934, p. 110.
Page 13
Camilla Warnke: The "abstract" society - 13
OCR text recognition Max Stirner Archive Leipzig - 11/27/2019
To determine the status of the identified systems, structures, functional mechanisms, etc.
from the standpoint of what is essential or insignificant for the thing itself, one can only
come when the systems sciences as instruments of those concerned with concrete things
Sciences occur. These have a concept of the object that defines it in the manifold
activity of its essential [31] properties and thus also specifies which hierarchy of the
System relationships and structures within the property are to be assumed.
So - to return to the social sciences - the systems sciences are for
deeper knowledge and better mastery of society useful and fruitful provided them
a concept of the respective society is given by the social sciences in which
the qualitative specifics of the thing to be examined and the hierarchy in relation to essential ones,
dominant structures and processes have already been identified. The application of systems science
ten on society (and that applies to other things as well, to organisms, etc.) must therefore, with
In other words, to be linked to the concrete, general concept of the thing which - how still to
will be shown - in relation to society, the concept of economic social
tion is, since only the concrete-general concept is the thing in the mediated totality of the
maps essential provisions.
VP Kuzmin and others rightly pointed out that Marx was on the occasion
the study of capitalism long before the systems science emerged,
Thought and system methods are used in abundance, but these never apply to him
to functionalism, structuralism or some other “ism”, but rather that
The aim is subordinate to the "thing" capitalism in the totality of its essential provisions
to grasp, 35 and this is achieved by means of the dialectical method, which involves the use of all others,
also of the system methods, regulated.
What has been said so far should already make it clear that the system representation of the dialectical
Materialism differs from the system concept of systems science. This assertion will
understandable when we talk about dialectical-materialistic philosophy and systems science
ten equally used terms for their meaning, if we pay attention to what is in
The content of the terms is taken into account or what is abstracted from.
Both the systems sciences and the materialistic dialectics of terminology make use of
minus "interaction". [32] While the systems science examines the objective facts of the
Subject interaction to abstraction by converting it into (mathematical) relation
change, philosophy concretizes this term by adding to the determination of the
saying enriched. Hegel is to be agreed when he says that the “interaction” only occurs “in the
Threshold of the (concrete - CW) concept “stands that it does not affect the movement of the object
can explain, because in this concept both sides are left as immediately given.
And Lenin notes at this point in the margin: “just 'interaction' = hollowness”. 36 About this
Threshold and into the space of the concrete philosophical concept, which the movement and development
The concept of interaction occurs when it is conceived as a contradiction
is, as a relationship "in which one of the opponents dominates the other " 37 .
35 VP Kuzmin, Voprosy kačestvennogo, količestvennogo i mernogo analiza, in: Istorija marksistskoj dialektiki, Moskva
1971, p. 150. Here the author writes: “However, it would be insufficient to merely subscribe to Marx's qualitative analysis
to characterize the point of view of the application of system principles. Because the system view is in spite of it
extraordinary importance not a panacea for all ills. Their unreflected application can lead to absolute
tion, lead to ossified systematic thinking. Many contemporary ones suffer from methodological bias of this kind
bourgeois philosophical schools and groups such as structuralism, functionalism and anthropologism
and other 'isms' fixed on a single method ... Marx strove to make human society so deep
and to explore as fully as possible and to discover the fullness of their qualities, properties, relationships and regularities.
to explore. Marx always applied the methods of scientific knowledge in an astonishingly adequate manner
on the object to be examined or the aspect of social reality to be explored. "
36 WI Lenin, vol. 38, p. 153 f.
37 P. Ruben, Strategic Game and Dialectical Contradiction, in: Deutsche Zeitschrift für Philosophie, Berlin 1970, H.
11, p. 1377.
Page 14
Camilla Warnke: The "abstract" society - 14
OCR text recognition Max Stirner Archive Leipzig - 11/27/2019
R. Schwarz characterizes the different viewpoints under which systems science
and materialistic dialectics treat the concept of interaction as follows: “After
In terms of reality, the connections between the objects are understood as reciprocal effects, not
but as existing mathematical relations. That in itself is the turn towards philosophy that
but mostly not carried out by looking at the interaction on the concept of contradiction
rather than contradicting the use of philosophical terms such as cause, we-
effect, causality principle, etc. appears. The individual sciences are not concerned with the essence
to understand the interaction as a conflict, which is a matter of philosophy, but above all
about getting certain interactions under control. The turn immediately follows
towards mathematics by paying attention to particular aspects of the interaction that
state of the science concerned. " 38
Or let's take the fact that systems science and dialectical-materialistic philo-
sophie can view the thing as an element in relation to a given system. The systems science
In their concept of element, sciences necessarily abstract from the fact that elements of their objective
ven nature of things with a structured internal structure [33]. Regarded as elements, interested
not this internal structure, they are by definition only identical to themselves, from the difference
within the identity is disregarded. For systems science, element is “part of a
Totality (a system) of objects that are not further broken down within this totality
can ... From a cybernetic point of view, elements are the last building blocks of a
cybernetic system, which - in relation to this system - cannot be further broken down or
and from which the system is built through certain circuits (or couplings) ”. and
Finally, the concept of elements in set theory demands “of all properties of those considered
Objects ... with the exception of those (to be abstracted) to be an element of a certain set ”. 39
The three definitions of the term element cited above show that this approach
It is expressly forbidden to understand elements as things. It is prescribed that and which
Abstractions are to be made.
If we compare this with the idea that in materialist dialectics with "element"
is connected, it is noticeable that this term, wherever it appears, as an "elementary form" of one
concrete, evolving system is understood. This is the first sentence of Marx's “Ka-
pitals ":" The wealth of the societies in which the capitalist mode of production prevails
appears as an 'immense collection of goods', the individual goods as their elementary form . ” 40 This
Marx's statement contains the following implications: "Element" is the term for an objectively existent
rendes concrete thing (the commodity); the thing nature of the element is not abstracted. The element
ment is not that which is identical with itself, but rather it contains the contradiction and thus the
movement in itself (contradiction between use value and exchange value). The element understood in this way
holds not any arbitrary, but the system-determining, essential contradiction. In-
which it contains, it is a qualitatively determined, constitutive element. And by making everyone
Reshaped and transformed according to its own nature, it is an elementary form.
A philosophical, dialectical-materialistic concept of the element [34] is not yet available
has been worked out, but I am sure that it is only in the direction taken here
grip training can be sought and developed. "Element" as a term in the conceptual structure of the dia-
lectic materialism can only appear as a concrete, general concept, that is, as a concept
which has its opposite in itself and merges into it. The concept of elementary form, like Marx
uses it, fulfills this requirement as a concept: it is only elementary in one of its determinations.
ment, in the other it is a system, a system of opposites moving within itself.
Using the example of the dialectical-materialistic concrete concept of the elementary form,
that behind what at first glance is a purely methodical difference in relation to the
In terms of abstraction and generalization, the difference between ideologically revolutionary
38 R. Schwarz, Philosophical and methodological problems of general systems theory, op. Cit., Pp. 93-94.
39 Dictionary of Cybernetics, ed. by G. Klaus, op. cit., p. 173.
40 MEW, vol. 23, p. 49.
Page 15
Camilla Warnke: The "abstract" society - 15
OCR text recognition Max Stirner Archive Leipzig - 11/27/2019
Philosophy and individually ideologically indifferent individual science hides the
can in principle be used by any ideology.
Let us make this clear with Lenin's concept of monopoly. For Lenin the monopoly is an element
or elementary form of imperialism. Following the example of Marx, it is countered as a unit.
formulated in the form of a tendency towards the concentration and unification of capital
fewer and fewer owners and as competition within the monopolies and between the monopoly
cunning associations. Monopoly is the movement caused by this opposition. These
consists in the fact that competition smashes the respective monopoly power groupings,
negated; in and through negation, however, at the same time sets new power groupings so that straight
by virtue of competition, the opposition to monopoly, the tendency towards monopoly formation through
puts. Because competition and the tendency towards union are not in separate terms, not abstract
juxtaposed and externally put together, since each is rather the concrete opposite
the other, the premise that is inextricably linked to it, is Lenin's concept of the
Crete concept of monopoly, which contains its movement and development in itself. He closes the
Realization that monopoly and competition, in that they mutually negate and set up each other, on
enlarged scale reproduce [35], whereby both the socialization of production
increases, as the competition intensifies and thus the basic contradiction of the capitalist
different mode of production is deepened. 41
“Monopoly” can also be understood as an abstract concept that excludes its concrete opposition
become, as a mere tendency towards ever larger monopoly associations. This loading
In terms of attitudes - which is typically bourgeois and revisionist 42 - competition is seen as one
external to the monopolies, not necessarily connected with them, not belonging to their essence
Appearance as its abstract opposite. If it still exists, it is the childhood disease of the
Valuing capitalism that has already been overcome by monopoly. That of his opposition
The clean-up monopoly is necessarily its “dead”, motionless concept, the
no longer allows transitions into other terms; which itself as the end of a conceptual movement
output and thus perpetuate.
The methodical decision, however, to construct monopoly as an abstract or as a concrete concept.
is based on an ideological and political decision. Who, from his contradiction
Purified concept of monopoly suggests one purged of its contradictions and class struggles
Capitalism. Its mere identity, indistinguishable in itself, implies identity,
Harmony close to society; and its inability to transition into other terms
the concern of the bourgeoisie not to allow any qualitative changes in society is theirs
to perpetuate imperialist rule.
Lenin's choice of the concrete concept of monopoly is no less political. As a prole-
a tarian revolutionary, he was interested in the germs of the downfall of bourgeois society.
to track down the tendencies inherent in it, the objective possibilities for the social
to determine the tic revolution. The scientific and political conclusion that emerges from Lenin's
grasp of the monopoly results is the following: "In his imperialist stage the capital
lism right up to the all-round socialization of production, it draws the capitalists
to a certain extent without their knowledge and against their will into a kind of new social order
which form the transition from completely free competition to complete socialization.
det. " 43
Let us generalize this procedure: Beyond the qualitative limit of your subject
Current knowledge, i.e. knowledge of development processes, requires as a prerequisite the
41 Cf. C. Warnke, On the Dialectic of the Concrete-General in Lenin's Investigation and Presentation of Imperialism
("Imperialism as the highest stage of capitalism"), in: Philosophers' Congress 1970, Die Leninsche Weiterent-
development of Marxist philosophy, Berlin 1970; P. 79.
42 See VI Lenin, Vol. 22, p. 278.
43 Ibid, p. 209.
Page 16
Camilla Warnke: The "abstract" society - 16
OCR text recognition Max Stirner Archive Leipzig - 11/27/2019
concrete term, since only this - in its positive determination encompasses its own negation
- is able to transition into its opposite, into its opposite. 44
This specific nature of the concrete term binds it inextricably to critical-revolution
tional thinking, which - to use Marx's expression - “in the positive understanding of the existing
at the same time also includes the understanding of its negation, its necessary downfall, every
form that has become in the flow of movement, thus also grasped according to its transitory side ”and therefore
"To the bourgeoisie and its doctrinal spokesmen a nuisance and an abomination". 45
The object of dialectical-materialistic philosophy is objective reality in totality
their general provisions. With respect to them, the system and structure aspect is only one among
many, an aspect that says something about the way things exist . The system character of the
In philosophical statements, objects appear as a logical predicate in
Sentences like: “things are systems of qualities” or “the world is as an ordered system
of material formations ”, whereby it is agreed that things, the world, etc. also
possess other properties which are found in other categories and laws of the materialistic dialects
tics can be expressed.
The recognition of the systemic character of the world and of things leads in the dialectical-materialistic
We do not have to look for a “universal structure”, a “structure in itself” according to our philosophy.
It is merely stated that "any classes of phenomena can have a corresponding structure
without specifying which exactly “ 46 . The dialectical-materialistic philosophical
Furthermore, it is a matter of being able to state in what way, by what means of mediation the system
The stem and structural aspect of things is related to their other general determinations.
VS Tyuchtin is right when he is aware of this relationship between philosophy and system problems.
against the [37] attempts to create a general systems theory as an ontological theory,
objected: “The thesis of the existence of a 'universal structure', of 'structure in general', of the
'Law in general' leads to the logical paradox: because of its universality, such a
Structure the complete indeterminacy with regard to the procedures (types) of orderliness of different
of classes of things from which it abstracts. In short, the recognition uni-
Verseller structures logically contradicts the definition of the structure. ” 47
In contrast to the philosophical way of thinking, there are “system”, “structure”, “function” etc.
in the systems sciences as the logical subject of statements. It is asked: which types
of systems, structures and functions and what are they like? Real exi-
constant material or ideal structures made variable in relation to the empirical objects,
from which they are originally borrowed, and depicted in an abstract (mathematical) way, i.e. with
recorded by means of abstract, general concept formation. This results in the following situation: “On the one hand, are
the mathematical structures universal in the sense that they do not have rigid limits of their possible
can be used in various areas of reality. On the other hand (and that is
essential) there are no universal methods that can be used to solve all classes of tasks
could effectively solve which relate to all qualitatively different areas of reality
hen. In other words, the specifics of the system objects and consequently their organization, structure,
require special mathematical structures (methods) and (or) for their adequate expression
their special link. And that means that the mathematical terms and methods are not
have universal generality like the philosophical categories, but rather limited ones
Have generality by uncovering the kinds of harmony, order that is necessary for this or
that class of appearances are most characteristic. ” 48
44 See VI Lenin, vol. 38, p. 213.
45 MEW, vol. 23, p. 28.
46 VS Tyuchtin, Sistemno-strukturny podchod i specifika filosofskogo znanija, in: Voprosy filosofii, Moscow 1968, H.
11, p. 54.
47 Ibid.
48 Ibid, p. 56.
Page 17
Camilla Warnke: The "abstract" society - 17
OCR text recognition Max Stirner Archive Leipzig - 11/27/2019
The abstract generality of the structures depicted in systems science means that
ner that an objectively real structure is assigned a “set of mathematical structures and
that the same formal system ... two [38] not only different, but even incompatible "
can objectively represent real structures. 49 These findings apply regardless of whether
the systems science concepts are at the level of mathematization
or in non-mathematical form. Because of their abstract, general character
also the systems sciences and the general systems theories, which are in the process of becoming, for themselves
not taken that universal science sought after, say, since Descartes, which demonstrates the unity of science
sens could produce. 50 hopes of this kind, for example, formulated by L. v. Bertalanffy in relation
on general systems theory. For their main object he considers the elaboration of principles
pien that apply to systems in general, whatever the nature of the constituent elements
and relations may be constituted. Naturally, such a theory is at the highest level
Abstraction, which is why L. v. Bertalanffy thinks that in its elaborated form it is a
logical-mathematical, purely formal theory, whose objectively real correspondence is the iso-
morphism of the laws of different areas that are structural identities of reality. These
Identities are to be found and formulated, since on their basis the unity of knowledge
properties can be produced. 51
Consistently thought through to the end, this approach should lead to the search for the structure of all structures.
which is why the cited criticism by VS Tyuchtin applies to the Bertalanffyian approach that the
grasped the universal structure of the definition of the structure logically contradicts. The concept of a uni-
versellen structure must necessarily lead to logical contradictions because it is the property of the
Identity and isomorphism of things fixed on one side, the sub-
differences and heteronomies but excluded from the term. The objective contradiction that the
Things are identical and not identical at the same time, have isomorphic and different structures
wise does not disappear if it is not thought. It just shifts into the sphere
subjective external reflection and occurs here as a logical contradiction, as an antinomy
appearance.
In relation to these ambitions, however, the fundamentally correct objections also apply
Hegel against a [39] mathesis universalis [universal mathematics]. Hegel's criticism is criticism of the
Dialectic's standpoint on the absoluteness of every abstract-general determination, the hy-
postas every method in general and today meets the system theory elevated to ontology, the
Structuralism, functionalism and other possible ones based on systems science
educated “isms”, as they met the idea of ​​a possible mathesis universalis in Hegel's time.
Hegel does not deny that mathematics is universal. But he thinks it's a kind of verall-
resentment, in which no distinction is made between the essential and the insignificant,
hence for a formalism that is applied externally to the object: “It has no concrete one
Object, which has inner relationships in itself ", there is in it" indifference of
linked against the connection ”, through them“ that which is incapable of any necessity ”can be achieved with one another
be linked ". 52
Apart from the fact that Hegel misunderstood the powers that mathematics and today systems science
possess properties precisely because they depend on the distinction between
Abstract from the likeness and the inessential, the organically linked and the non-linked in the thing
can and must, he is right when he calls the activity of the mathematical type “one for the
Matter of external action "denotes 53 , and if he therefore demands that the application of math-
matik “an awareness of its worth as of its meaning (must) precede; such a
49 R. Schwarz, Philosophical and methodological problems of general systems theory, op. Cit., P. 98.
50 Cf. R. Descartes, The rules for the guidance of the spirit, Leipzig 1920, p. 21.
51 Cf. L. v. Bertalanffy, General System Theory, op. Cit., Pp. 37, 48, 86 f.
52 GWF Hegel, Science and Logic, Part One, op. Cit., P. 208.
53 GWF Hegel, Phenomenology of Spirit, ed. by J. Hoffmeister, Berlin 1964, p. 36.
Page 18
Camilla Warnke: The "abstract" society - 18
OCR text recognition Max Stirner Archive Leipzig - 11/27/2019
But consciousness is only given by thinking contemplation, not its authority from mathematical
tik. Such consciousness about them is logic (ie dialectic - CW) itself ”. 54
Such a consciousness is guaranteed by the dialectic that governs the abstract-general determinations.
genes mediated with each other, which does not make a determination independent, but the structure of the whole in
of his pure being ”has in front of him. 55
I. W Blauberg and EG Judin come up with the question of which task philosophy
has to meet the following requirements in relation to the systems sciences: "The philosophical
On the one hand, answering the questions of
basically of a methodological character that appear in this [40] research. on the other hand
does it have the heuristic possibilities and the limits of the application of the new knowledge
methods to determine their position and role in the general development of modern knowledge
Establish the status quo as well as cases of unjustified absolutization or underestimation of this
Criticizing methods ... The philosophical methodology appropriate to these tasks is this
materialistic dialectic ... " 56
Probably the most explosive and most spectacular case of unjustified absolutization at the moment.
In the following, from the standpoint of material science, concepts and methods of stem science
investigate the dialectic.
[41]
54 GWF Hegel, Science of Logic, loc. Cit., P. 212.
55 GWF Hegel, Phenomenology of Spirit, op. Cit., P. 40.
56 IV Blauberg / EG Judin, Philosophical Problems of Systems and Structural Research, in: Soviet Science, Ge
social science contributions, Berlin 1970, no. 10, pp. 1063-1064; see VP Kuzmin, Voprosy kačestvennogo,
količestvennogo i mernogo analiza, op. cit., p. 149. Kuzmin sees the tasks of Marxist philosophy in relation
on the system sciences therein: "The intensive elaboration of concrete scientific methods (the structural
len, functional, genetic and others) that are connected in one way or another with the systematic view,
to be supplemented by an exact philosophical analysis that gives each of them their real place in the general method
theology of the sciences. "
Page 19
Camilla Warnke: The "abstract" society - 19
OCR text recognition Max Stirner Archive Leipzig - 11/27/2019
II. Abstract-general or concrete-general concept of society
Let us draw the conclusion from our previous considerations. Because the systems science for itself
represent knowledge of the type of abstract generality taken because they are based on concrete content
are empty, variable with respect to empirical objects, they can be associated with different content.
that are, also with different terms of society, provided that society is only ir-
how is understood as a holistic phenomenon. They allow in the place of their conceived
to set the model of a structure and function in a concrete totality, which society as
depicts an abstract whole.
The abstract whole should mean: It is true that a structure and function structure becomes more social
Appearances demonstrated (and no question that the depicted structures and functions too
exist), but the specified value and hierarchy of the structures and functional mechanisms
men, i.e. the determination of the overall quality, cannot be achieved by means of systems science.
will be. Rather, this task requires a scientific concept of society, the
lies outside the competencies of systems science; and they have one
bourgeois social sciences, although the bourgeoisie in the progressive revolution
It was on the verge of discovering it in the tional phase of its development, although the metropolitan
methodical prerequisites - dialectics - with the help of which he could be won.
As the bourgeoisie, oriented towards practical and theoretical world domination, it was independent
Worldview began to develop, it was faced with the task of
bild God- given provisions and [42] categories to bring home step by step and to the
Binding subject world . For the medieval two-world theory, in which God = general
nes, being, substance, necessity, ultimate cause, etc., and world = individual, nothing, accident,
Accidental, effective cause, etc. as separate, only externally related series of determinants
the concept of contradiction was not a necessity for thought.
That changes, as general and individual, to be and not to be, substance and accidental, emergency
nimble and accidental etc., in short, as opposing determinations to one and the same opposing
stood to be bound to the world and naturally caused things. Now it becomes a problem
like something general and individual at the same time, to be and not to be, to be necessary and accidental
can. Not the connection within the rows, but the mediation of opposing facts
tegorien becomes the task to be solved. Around the beginning of this development is the great idea
of Nikolaus von Kues from the coincidentia oppositorum, in which still very directly and
uninterruptedly expresses the experience that nature, the world, is now home to all
has become.
The “Copernican turn” in this way occurs with Kant. Kant subjects in his
famous antinomies which continued as the legacy of the unresolved medieval two-world theory.
sluggish metaphysical style of thinking of a fundamental, devastating methodological criticism.
It provides the evidence that the absolutization of a single philosophical category for
The status of a universal statement, the tearing apart of polar determinations into indissoluble logical ones
Contradiction leads. But Kant is resigned to the task of thinking something general, that
is not the abstract-general, but in which the opposing determinations with one another
are conveyed. He considers this problem unsolvable for the human mind.
Not so Hegel, whose entire philosophical endeavor is directed towards the
to cope with the task left: to combine the contradicting abstract determinations in such a way
convey that they form a universal which is not in antinomies, not as a regressive force into the bad
Infinite [43] leads. The way to do this is the dialectical method, which for Hegel is under this to
The aspect at issue is the ascent from the abstract to the concrete-general; the result
act of this path is the concrete concept, the concrete-general, which distinguishes and contradicts
sentence contains the provisions in which the provisions contained in it alternate
presuppose, condition, determine and limit on each side. General and individual, being and
Not-being, necessary and accidental etc. are for Hegel taken in isolation and abstractly
Page 20
Camilla Warnke: The "abstract" society - 20
OCR text recognition Max Stirner Archive Leipzig - 11/27/2019
irreconcilable opposites, but precisely because they are opposites, they are inextricably linked
chains. The concrete-general is thus a general that contains the contradiction in itself.
Hegel's train of thought is the methodologically most mature solution to the self-imposed task,
to understand the world as functioning according to independent laws. But this colossal method
dic achievement comes into the world as a miscarriage, because the mediation of the determinations in the idea,
takes place in the absolute, in the realm of concepts. Ultimately, Hegel, too, misses the point of the
progressive bourgeoisie: in order to rule the world, without it from their own contexts
Recourse to transcendent forces and causes to explain what - if you take them seriously -
leads to consistent materialism. Consistent materialism only exists when everyone
general provisions are tied to objective reality, which in turn implies
to put the concrete - containing the contradiction - in place of the abstract concept.
Consistent materialism is therefore necessarily dialectical materialism.
The bourgeoisie has neither materialism nor dialectic throughout its existence
thought through to the end. To be consistently materialistic and dialectical requires materialism
also to be applied to society, that is, to the really existing society from its own
to explain their own context. But here lies the crux of bourgeois philosophy, theirs
insurmountable barrier to knowledge.
When the proletariat came on the scene as a political opponent and the bourgeoisie
function [44] began, the most important source from which the dialectic had flowed dried up:
the interest in relentlessly uncovering social contradictions. A social
Bild began to establish itself, from which the contradiction as a constitutive and changing principle
Strength was banished. Society is now defined in one layer as something which is identical in nature,
because their objective function is geared towards the preservation of the existing.
A. Comte's sociology, in which consensus and harmony are the principles
social elements and everything that disturbs the harmony and identity of the system,
is considered pathological. 57 With this Comte leads the return to the bourgeois social theory
metaphysical thinking style. One now holds back to one-sided abstractions, e.g. B. to consensus
sus, harmony, balance, order, etc., which are declared to be the essence of society, while-
if dissensus, disharmony, imbalance, disorder, etc.
pushes.
This separation of related opposites into different spheres creates the methodological
The possibility of treating society in the abstract at all levels of consideration. The of her-
Regulations that have been purified from their respective contradictions are not
conventional counter-provisions, restricted in their validity and thus made more concrete.
It is indisputable that consensus, harmony, equilibrium, etc.
economic systems are; just as indisputable is the opposite, that dissensus, disharmony,
Imbalance etc. occur in any social system. Whether you are this or that group
absolutized by terms or whether they can be converted into an external
brings external reflective context, methodologically amounts to the same thing: on the
refrain from starting from concrete terms, that is, from terms that contain their opposites in themselves.
sen and in which one of the determinations has dominance over the other. The latter is important
because only through the determination of the dominance within the contradiction does the term
social phenomenon is grasped in relation to the objective tendency of its development.
[45] The metaphysical style of thought prevails today at all levels of bourgeois society.
contention: on the level of general social theory - as we shall demonstrate - ‚and
- as we have shown with the concept of monopoly - at the level of the investigation of special geographic
social phenomena.
57 Cf. P. Kellermann, Critique of a Sociology of Order, Organism and System in Comte, Spencer and Parsons,
Freiburg 1967, p. 43.
Page 21
Camilla Warnke: The "abstract" society - 21
OCR text recognition Max Stirner Archive Leipzig - 11/27/2019
The relapse of the bourgeois class on methodological positions criticized by itself can
Sufficient to be understood only when viewed as a reflection of their objective social role
and understands function. After its political victory, the bourgeoisie is the dominant in every respect.
power within a class society, the dominant in a contradicting
relation of opposing classes, which their goals and ideas only in constant struggle against
can enforce the class it oppresses and its political goals. This position of the bourgeois
geoisie requires that it absolutizes its economic and political goals and ideas,
that is, as the aims of the whole society pass out; the goals and ideas of the oppressed
Classes, on the other hand, are not system-determining and therefore only disruptive phenomena that need to be eliminated.
considerations.
Indeed, capitalism is now dominated by economic, political and ideological supremacy.
position of the bourgeoisie so that the intellectual absolutization of their goals and
Concepts of society have a semblance of justification because they are objectively real
the practice of bourgeois politics reflects a dominant relationship that has been confirmed a thousand times over. A domi-
nance relationship, as I said, that is, the fact that in the contradicting class relationship
of the bourgeoisie and the proletariat, the bourgeoisie is itself, its antithesis, and thus the bourgeoisie
Is the overall contradiction-determining relationship. For the bourgeois ideology this is reversed
But underhand dominance in a claim to absoluteness that they derive from practical experience
deduces that it would be able to achieve its goals much better if these were not constantly
aim and countermeasures of the oppressed classes would be thwarted.
From these sources the metaphysical one typical of antagonistic class societies becomes
This style of thinking is fed which cannot be eradicated by any progress in cognition in the individual and [46] the
necessarily entails an abstract concept of society. He can only get one
Class to be overcome, which on the basis of their class position the contradiction with all his
Able to think about consequences.
The contemporary bourgeois "great theories" of society, the imaginary
lungs of what society is, how it works, what sets it in motion, etc.,
a general social term that describes every past, present and future society
should be valid.
The “great theory” wants its “considerations” - according to Luhmann - “consistently in a higher
straktionslage as the Marxian theory of evolution "settle, so" that the theory of evolution does not
to be formulated more analogously to the laws of nature as the law of the development process itself
needs, but as a theory of the system structures and processes that produce evolution, but
are not evolution themselves ” 58 .
This declaration, in which Luhmann formulated a program that he collaborated with Parsons, Dahrendorf, Deutsch 59
and others, is directed against the general Marxist concept of society, against the
Concept of economic social formation, the content of which Marx determined as follows:
“In the social production of their life, people go through certain ... conditions
a, relations of production , which a certain stage of development of their material productive
forces correspond. The mode of production of material life determines the social, political
and spiritual life process in general. ... At a certain stage of their development they get
material productive forces of society in contradiction with the existing production
relationships or, what is just a legal expression for it, with the property relationships, internal
half of which they had moved so far. These strike from the forms of development of the productive forces
58 N. Luhmann, Systems Theoretical Argumentations, op. Cit., P. 362.
59 Cf. K. Deutsch, Politische Cybernetik, Modelle und Perspektiven, Freiburg 1969. For the purpose of this study, in
which is about investigating the results to which the philosophically unreflected use of system knowledge
social concept formation leads to the construction of a general concept of society, would have for the cybernetics
German approach can be used instead of Parsons. If here the preference was given to Parsons, so lies
this is because the already existing Marxist Parsons criticism from the Marxist side (see Note 2) from general
my methodological aspects should be supplemented.
Page 22
Camilla Warnke: The "abstract" society - 22
OCR text recognition Max Stirner Archive Leipzig - 11/27/2019
Relationships in fetters of the same. Then comes an epoch of social revolution. With the
change in the economic basis, the whole enormous superstructure is rolling more slowly or rapidly.
shear. " 60
Luhmann is mistaken if he believes that the difference between what he
strived and the Marxist concept of society a difference with regard to the abstract
tion height is. Rather, the difference lies in the fact that bourgeois sociology is based on an abstract
general, the Marxist social theory, however, builds. This means in detail:
1. By using the concept of economic social formation as the core of the definition, the objective
the contradiction between productive forces and production relations (including dominance).
forms and captures the general hierarchy of the dependencies (base - superstructure), it is in the above
Meaning concrete term. But it is at the same time a general term, since the statement contained in it,
that every society is affected by the relationship between productive forces and
is true and receives its direction of development is a general statement, a general statement that
society in general as a historical change in modes of production, as in
understands itself to be qualitatively different. The term "society in general" thus contains the different
rentia specifica, by means of which he is determined, in himself, so that he is out of himself, out of his imma-
nent provisions (differences) can be determined. But this methodical approach is
the consequence of a materialistic view of society, a necessary result of the demand that
To understand society from its own and not its external contexts. It's going ok
in other words about the "immanent consideration" of the object, "... it is taken for itself-
men, without prerequisites, idea, ought, not according to external circumstances, laws, reasons.
You sit down completely in the matter, look at the object in itself and copy it
the provisions that he has “ 61 .
For the type of bourgeois social theory treated here, the process is in contrast to this
the transfer of the structural and functional model from other areas of reality or the
Use of very abstract structural and functional models constitutive. Because society is not as
Diversity of social formations is thought, can be used for the determination of the
the term "society" required differentia specifica only outside of the societal [48] term.
be found, either in the concept of another form of movement of matter (definition by
Comparison and delimitation) or in a higher-level, more extensive term (definition
through subordination).
The first case is with T. Parsons, who compares society with organism and retrospectively
wants to differentiate from it. Parsons is expressly committed to its structural-functional observation
manner and the concept based on it of how society as a whole is constituted,
to have borrowed from physiology, if he, relying on Cannon's "The Wisdom of the Body",
writes: “A relatively complete and explicit general system of this kind is for the
Physiology has been developed ... The fixed reference point for all physiological function analyzes
sen forms the anatomical structure of the organism. The criteria for the importance of processes
such as breathing, nutrition, etc. and their dynamic interdependence result from their function
in relation to the maintenance of this structure in a given environment. ” 62
The second case is represented by N. Luhmann, who, bearing in mind the criticism of the organism,
seeks to win over the concept of society in relation to and demarcation from the world. This fine
However, units make no difference with regard to the principle methodological process: whether
Society as self-sustaining and stability in adapting to the respective circumstances
60 MEW, vol. 13, p. 8 f.
61 VI Lenin, vol. 38, p. 241.
62 T. Parsons, Contributions to Sociological Theory, Neuwied / (West-) Berlin 1964, p. 39; see P. Kellermann, Critique of one
Sociology of Order, loc. Cit. Kellermann describes the tradition of explaining society as a whole
nism model to use, very material and insightful after. But he doesn't ask himself why this one
methodological process a necessary undertaking for bourgeois social theory is why it is applied to others
Wise cannot develop a concept of society.
Page 23
Camilla Warnke: The "abstract" society - 23
OCR text recognition Max Stirner Archive Leipzig - 11/27/2019
is understood as a preserving system or as a symbolic system for recording and reducing
of world complexity; in both cases it is determined by borrowing from “outside”: from the
cybernetically interpreted physiology or, as with Luhmann, in general systems theory
in the reading L. v. Bertalanffys. In this regard, Luhmann's method does not represent a break
with the organic tradition of social theory, as J. Habermas have us believe
wants when he writes: “Luhmann's system theory of society is, according to its claim, more and more
other than social cybernetics. This is what makes them so effective. Luhmann knows the price reduc-
ionistic processes ... Organisms are integrated on the basis of 'life', social systems on
the basis of 'sense'. " 63
[49] In both cases there is the external one which is not obtained from their internal specific difference
Concept of society only abstract, meager. Regulations, just so much understanding
for the real social processes when model ideas were put into them. the
Concept of society as an organism or as a system in the sense of L. v. Bertalanffy will last
not exceeded because the specifics of society in relation to the wealth of its determinants
mere external analogy or subsumption cannot come into view at all.
Society is seen as a system of the nature and nature (as a cybernetic system, as a general
ne abstract system, etc.) and as nothing else; and one can look at the real social
always find the appropriate evidence for it, since society actually does the mentioned
Has system aspect.
W.-D. Fool correctly identifies this procedure when he writes: One defines tautologically "...
for example the system as self-sustaining and adapting to the respective circumstances
That preserves stability and then concludes for every social system that self-preservation and
gration are the most necessary and supreme goals and characteristics, just the goals ... 'that you had previously
used to define the social system to be examined first “ 64 .
But this procedure necessarily leads to reductionism, to the reduction of society
Determinations of the essence and target functions of the biological or (and) for the reduction to abstract
general determinations of system-theoretical origin. The systems science terms
can not be specified and in the "great theory", since the concrete concept of society is missing
be concretized, that is, are not assigned in such a way that they are appropriate to the object. she
do not play the role that is due to them by their nature: from a meaningful social
concept of being directed instrumental knowledge in order to grasp the systemic aspect of society more precisely.
to be able to. Rather, they fill the gap that the concrete concept of society actually contains.
should have, and take his place.
The concept of society is thus reduced in two senses: Society only becomes
conceived under the aspect of being a system, but even then not in its specificity as a social
system, but as a biological, cybernetic, abstract system-theoretical, etc. system.
The real contradiction in terms of methodology and theorizing in contemporary
social outlook is not to be found where Habermas seeks it, not within the framework
mens of the "great theories" of bourgeois sociology, but rather it runs between "external"
bourgeois and “immanent” Marxist social theory.
The organicistic or systems science concept of society thus obtained, which includes the fac-
processes, etc., as a formalism that is only done externally, is interchangeable and changeable. He
can be borrowed from cybernetics, game theory or general systems theory, depending on your taste
are and will be borrowed, primarily in accordance with the strategic goals and. Regulatory problems
of state monopoly capitalism.
2. The concrete, general concept of society in Marxist philosophy resolves in my opinion
one of the most difficult methodological problems of social theory: the problem of how that
63 J. Habermas, system theory of society or social cybernetics, in: J. Habermas / N. Luhmann, Theory of Society
Science or social technology, op. cit., p. 146.
64 P. Kellermann, Critique of a Sociology of Order, op. Cit., P. 11.
Page 24
Camilla Warnke: The "abstract" society - 24
OCR text recognition Max Stirner Archive Leipzig - 11/27/2019
empirical individual statements are conveyed with general statements about society
can. By using the concept of economic social formation the concepts of feudalism,
Including capitalism, socialism and communism, etc., it rightly fulfills R. Merton's
demand made, general social theory through the creation of theories “middle
Reach "to constitute. Merton writes: “Throughout, I try to draw attention to that
focus on what might be called mid-range theories: mid-range theories
between the smaller but necessary working hypotheses that abound during everyday life
routine of research, and the all-encompassing speculation that is a fundamental
enclose tales conceptual scheme from which a very large number of empirically observed
Hopes to derive uniformities of social behavior. " 65 That is, what is sought is a" medium "
Level of generalization that defines the relationship between the individual empirical statements and the
should make very general statements in relation to society in general (which the individual with
the general [51]). In other words, the way out of the dilemma is sought
bourgeois sociology, which has been so drastically portrayed in von Mills' since its inception
Antinomy of "abstract" empiricism and equally abstract "great theory" floating around. 66
Merton's proposal is an attempt to show this way out. But because Merton made his claim
according to theories of “medium range” not to the establishment of objective social laws
(not to a content), it remains of a purely formal nature. You can through the investigation and
generalization of any (any) regularity within a field of social phenomena
requirements are met, which is chosen broad enough (spatially and temporally). The one on this way
The concept of society that has been gained cannot therefore guarantee whether it will be
explanatory phenomena (empirical data) with regard to their essence.
Generality, uniformity of a property, a relationship is a necessary “indicator of the
Essentially “, but it does not guarantee that this property, the relationship to the
Being heard. 67 “In general” - writes Lenin - “is a poor determination, everyone knows about the universe
mean; but knows nothing of him as a being. ” 68
Marx has it that the essence cannot be ascertained by means of ever thinner abstractions
shown often enough, especially on the occasion of his capitalism analyzes: This is how capital can, for example
the determination to be abstracted, that it is accumulated, objectified labor that is used as a means
serves for new work, whereby it is only grasped in its material mode of existence (but not as
Relationship, in its contradiction and as a process) and on the poor, timeless, general determinati-
tion “means of production at all” is reduced. 69 This abstraction misses the essence of the
pitals, misses what makes its specific difference to "means of production" in general-
that can only be determined if the abstraction obtained is in the concrete term-
at least other terms are limited, specified, etc.
Marx therefore defines the essence under the structural aspect of society with terms such as "internal
rer context " 70 , [52]" inner organization " 71 ," hidden structure " 72 and under the movement
and development aspect with terms such as "inner gear" 73 , "inner real movement" 74 to the
Difference from the appearance of things, “their outwardly appearing forms of life” 75 .
65 K. Merton, Social Theory and Social Structure, Glencoe 1968, p. 39.
66 Cf. CW Mills, Critique of Sociological Thinking, loc. Cit., Chap. II and III.
67 Cf. F. Kumpf, Problems of Dialectics in Lenin's Analysis of Imperialism. A study on dialectical logic, Berlin
1968, p. 55.
68 WI Lenin, Werke, 38, p. 256.
69 Cf. K. Marx, Grundrisse der Critique der Politischen Ökonomie, Berlin 1953, p. 169. [MEW Vol. 42, p. 182]
70 MEW, vol. 25, p. 825.
71 Ibid, p. 839.
72 MEW, Vol. 26.2, p. 162.
73 MEW, Vol. 24, p. 218.
74 MEW, vol. 25, p. 324.
75 MEW, vol. 26.2, p. 162.
Page 25
Camilla Warnke: The "abstract" society - 25
OCR text recognition Max Stirner Archive Leipzig - 11/27/2019
These determinations make it clear: the essence is neither a single general property
or relationship still a set of such properties or relationships, but a self-contained
structured and mediated whole.
With his call for theories of "medium range", Merton remains in the vicinity of the abstract
general social term. He does not understand that the "average" degree of generality, the special
cifically-general, which he strives for, cannot be determined purely formally, but rather the transition from
General to essence makes necessary. This specific-general, which exists between the empirical
ric individual data and the statements conveyed at the level of the highest generality, but are in
in relation to society, those included in the concept of economic social formation
statements about the "inner structure" and "the real inner movement" of the various
other replacing social formations.
That the antinomy, here “abstract” empiricism - there “great theory”, only by means of the concrete
The concept of economic social formation has to be overcome by the bourgeois society
ziologists most clearly understood Mills' outsider when he demands that the concept of
historically determined social structure to form the basis of a general social theory
have. 76
[53]
76 Cf. CW Mill, Critique of Sociological Thinking, loc. Cit., Pp. 91, 199.
Page 26
Camilla Warnke: The "abstract" society - 26
OCR text recognition Max Stirner Archive Leipzig - 11/27/2019
III. T. Parsons' homeostatic model of society
Structural and functional models for wholes, borrowed from systems science, play a role in
the contemporary bourgeois "great theory" of society played the same role that the
sation model (the model of physiology and biological ontogenesis) in the older sociological
gie played. T. Parsons marks the transition.
Parsons 77 formulates his conception of making society "in itself" the object of his theory.
as follows: “After anthropologists and sociologists spent about a generation
have paid attention to those phenomena that distinguish one society from the other and
The different structures within the same society differ from one another is in the
In the last few years there has been renewed interest in the question of whether there are traits that all human
societies are common and what forces bring about the maintenance of these common traits. ” 78
These common features, as Merton observes, are borrowed from physiology by the following
- Method determined: "First, certain functional conditions of the organism are established,
which must be fulfilled if the organism is to survive or function reasonably effectively
target. Secondly, it is described specifically and in great detail how the arrangements (structures
and processes) by which those conditions are typical and in 'normal'
Cases are met. If it turns out that some of the mechanisms necessary for the fulfillment of those
Conditions are typical, have been destroyed or do not function properly, thirdly, the
observers have to look for replacement mechanisms that (if [54] at all) the
fulfill the necessary function. Fourthly, and this is evident from the previous steps, it follows
a detailed description of the structure to which the functional conditions apply, and a
precise description of the precautions by which the function is fulfilled. ” 79
In Parsons' self-image, however, this model, borrowed from physiology, is not intended to make any statement about
the nature of the structure and the functioning of society, but merely methodological instru-
ment, a system of terms and orientation theses for the formation of hypotheses. 80 So
draw structure “not any ontological stability in the appearances, but only
a relative stability - uniformities in the results of certain underlying processes,
which are sufficiently stable to be considered constant for pragmatic purposes within certain limits
to be accepted. " 81
In contrast to Marx, for example, Parsons does not make the comprehensive claim that
to formulate pirical general statements (i.e. to discover social laws). Theo-
According to Parsons, the general physiology that he uses contains “absolutely none
empirical general statement. It is just a tool that can be used to make certain em-
pirical solutions and empirical general statements can be obtained if one
Applies appropriate data “ 82 .
Parsons is fundamentally mistaken when he thinks that theory and method are so strictly separated.
to be able to. The above-cited model of physiology is in no way intended as a mere analytical work-
to understand stuff; it also implies theoretical statements about the nature of the organism.
The terms that Parsons uses in instrumental use are not empty of content, and that is with them
It is therefore not a model constructed with their help either. It makes statements about a certain
Way of existing. "Survival", "balance", "integration" etc. form a conceptual structure
77 Here T. Parson's teaching is not presented in a comprehensive sense, but only so far as to make it visible, as in
Parsons the relationship between the concept of society and the system is based. A more comprehensive account and criticism of Parsons
can be found in E. Hahn, Social Reality and Sociological Knowledge, loc. cit., pp. 41-104; and BP Löwe, Zum Ver
relationship between late bourgeois political sociology and the political ideology of imperialism, op. cit., pp. 50-114.
78 T. Parsons, Contributions to Sociological Theory, op. Cit., P. 109.
79 RK Merton, Social Theory and Social Structure, loc. Cit., P. 103.
80 Cf. T. Parsons, Contributions to Sociological Theory, op. Cit., P. 15.
81 See ibid, p. 37.
82 Ibid, p. 41.
Page 27
Camilla Warnke: The "abstract" society - 27
OCR text recognition Max Stirner Archive Leipzig - 11/27/2019
in which the individual concepts condition and presuppose one another. Is survival the goal
function of the system, then at the same time a field of associated determinations is
finishes. The conceptual structure is relatively closed: [55] both as a theory and in relation to the
methodological possibilities that it implies.
Parsons transfers this model to another area of ​​reality, to society, and
sees his task in finding the corresponding analog structures, functions, etc. here
to make that meet the model. For Parsons, for example, the social pair of terms “doer
- Situation ”analogous to the pair of terms“ organism - environment ”in biology. 83 This way
the concepts of physiology, held by Parsons for empty formulas, are combined with social
Content fulfills, and its totality and conceptual structure is then used as a structure and mode of operation
issued by the company.
It is undisputed that in this way certain structural and functional relationships of the
society can be revealed. But the scope of this approach is only so
how objective identities exist between organism and society. The specifics of
Society that lies beyond these identities cannot be identified in this way. That
the physiological model as a whole does not change its nature - if it is transferred -
it retains its structural and functional determinations shaped by physiology. Parsons
does not check whether the model he has chosen in view of the specifics of another reality
Reichs grasps the essential, whether it contradicts the immanent specific regulations of society.
reflects, but only asks which mechanisms in society create stability and equilibrium.
and in relation to what there is stability, equilibrium. This transforms the supposed-
Lich heuristic concepts but in ontological determinations. Society is now objective on that
Objective survival set. For Parsons, stability, equilibrium, etc. are the ultimate social
real values ​​that must be preserved and established because they define society, that is, because they
appear as concepts which are not themselves determined but presupposed and in relation to the
other terms can be defined.
Parsons can therefore not carry out the intended positivist separation of theory and method.
because it cannot be objectively realized. By using Parsons' method [56] underhand
transformed into a system of theoretical statements, the unity, as it were, places itself behind his
Back again, as a bad, unreflected unity, as a unity in which the differences
are extinguished. The result brought about by the application of the physiological conceptual system
is not richer than the original model. Parsons doesn’t get anything back other than what he’s in
the analysis has put into it: the bare abstract terms “survival”, “stability”, “equality”
weight ”etc., under which the society-specific terms are subsumed. Whether Parsons wants that
or not, he subsumes the society-specific terms under one belonging to physiology
Conceptual structure that assigns them their place, their status and their meaning, so that societies
as a whole is understood as a special case of organism. How much this procedure affects the
A. Rapoport has shown in the context of the conflict that the interpretation of social phenomena has an influence.
So - if society, as with Parsons, is viewed as a homeostatic system - a
empirically established social conflict are viewed as a dysfunctional deviation,
the inadequate functioning of the equilibrium forces of society
originates, or - if society is conceived of as a developing system - as a symptom
apply to the next development step. 84
The “external” declaration of society, the mere subsumption of social issues under
abstract-general models, leads, if they are not based on the concrete concept of society as its basic
sis is worn to absolutize one aspect of the appearance to be interpreted
and thus to arbitrariness.
83 See ibid., P. 52.
84 Cf. A. Rapoport, Methodology in the Physical, Biological and Social Sciences, in: General Systems, Vol. XIV, aa
Cit., P. 182.
Page 28
Camilla Warnke: The "abstract" society - 28
OCR text recognition Max Stirner Archive Leipzig - 11/27/2019
At this point, a note on the position and function of the specific term in the cognitive
to add to the process of social phenomena: The path from the unrecognized
tion (the unknown empirical data) to the recognized phenomenon is not a logical act of
Subsumption of empirical data under abstract general terms. Abstract-general in relation
on society are also coherent conceptual structures that relate to another movement
belong to the form of matter (whose essence they [57] represent!) or to the abstractions of the sy-
are borrowed from stem sciences, since they only record those properties in society.
that they can do with the other realm of reality or with those in the systems sciences
generalized structures. The path from the unknown to the known social
Appearances must therefore be about the determination of the essence, the at the same time universal and for the
Society-specific lead: as has been shown, about the concrete-general concept of eco-
nomic social formation, which has a very different structure from the physiological
Represents provisions. The path from the immediately given, unrecognized concrete to the recognized
th concrete, from the unrecognized to the recognized phenomenon could idealized as follows
can be described: The process begins with the immediate, the unrecognized appearance, with
the abstract-concrete, so to speak, which is abstract insofar as there is no knowledge of the connections
exist who rule the appearance. At the level of the abstract-general, too, the
inner relationships of the whole are not yet visible, but properties, individual additions
connections etc. abstracted and generalized. At the level of the concrete-general
the abstract-general becomes the object of operations, in that the relations between the
Looking for abstractions, determining their hierarchy, i.e. determining the essence. The transition
for the recognized appearance, in the concrete-particular, is repeated concretization of what is in one
its provisions are already concrete.
The "mediator" between the stages of this mediation is the one of the two determinations
genes which - to tie in with Hegel's inference - appear in the statement as a predicate. she
changes its place and becomes the subject of the statement in the next premise. The whole
“Conclusion”, which converts the unrecognized immediate into the recognized mediated appearance, is therefore
to be presented as follows: the concrete is abstract; the abstract is general; the general
is concrete; the concrete is concrete.
Parsons omits the level of the concrete-general. He's too short. He is content
in view of the analysis of society as a whole with their (and also on a certain [58] type of
System aspect. The theoretical consequences of the central error of the Parsonsian
Methodology meets W.-D. Fool when he writes: “The quintessence of a carried out and
tested analysis: Society as a system, is no longer open to discussion at the beginning of the analysis.
sion. The heuristically necessary purpose of systems thinking turns into a heuristic of the
System, ie the system is no longer examined as a problem, but only the problem of the
Systems. The heuristic-constructive sense of the system analysis is at best established at the beginning.
but there is no heuristic feedback, since all of the investigation elements are from the system
and the system itself, its survival, is the yardstick. ” 85
Only problems of the system are investigated. With this the abstract concept of system comes to the
Place of the concept of society and becomes the subject whose predicates are to be determined. An egg
The property of the concrete whole is hypostatized to the whole.
This whole, the aim of which is survival as effective functioning, must be definite
meet functional conditions that Parsons described in the form of four "functional imperatives"
ben has. These are:
"1. Maintaining the institutionalized cultural pattern around which a social system is based
organized (maintenance pattern = pattern of behavior Grade); 2. the formation of an order of
inter-individual and inter-constitutional relations by fitting the functionally differentiated ones
Role complexes and sub-areas in the entire system process (integration = integration); 3. the
85 W.-D. Narr, theoretical terms and system theory, Stuttgart / (West) Berlin / Cologne / Mainz 1969, p. 171.
Page 29
Camilla Warnke: The "abstract" society - 29
OCR text recognition Max Stirner Archive Leipzig - 11/27/2019
Realization of collective goals to be achieved (goal-attainment = achievement) and 4 training
dealing with the system with its environmental conditions, in particular the satisfaction of the
teriellen needs of a social system and its members (adaption = adjustment). " 86
Four systems correspond to the four functional imperatives: adaptation, economic sy-
system (including technology and science), the achievement of goals the system of social
power, integration the system of legal norms, the preservation of behavioral patterns
initialization instances.
[59] Everything that contributes to the maintenance of the system - to the maintenance of behavior patterns, to the
Integration, goal attainment, adaptation, everything that is balance and stability
of the system is considered "functional", and everything that disturbs the equilibrium, the stability
and effectiveness is impaired, is considered "dysfunctional". By definition, the function is inherent to the system.
effect of an element. “A process” - writes Parsons - “or a series of conditions
can either “contribute” to the maintenance (or development) of the system or they are “dys-
functional ', that is, they affect the integration, effectiveness, etc. of the system. ” 87
This definition of society as a system is - because it is vague - one-sided. System on the target
function of self-preservation, stability through integration, to switch off, not only absolutizes the system
aspect of society, but also a certain system concept, that of homeostatic
system, the system par excellence. That Parsons considers homeostasis and system to be identical,
emerges from the following statement: “In theory, the concept of equilibrium is a simple one
Correlate to that of the system, the interdependence of the components as interconnected. Of the
The concept of the system is in turn so fundamental to science that it can be used at higher levels
theoretical universality no science can exist without it. ” 88 The homeostatic sy-
stembegriff excludes the historical perspective. It tempts the temporary to
historical conditions of the relative stability of society as the "north
state of painting ”of society (an assumption which, according to its terminology, is completely in
belongs to the realm of physiology, and which is meaningless in relation to historical objects) and
to understand the state of instability and change as the "abnormal".
This turns the real situation on its head. In relation to socio-historical
ric objects created by the contradiction between productive forces and relations of production
are advanced, the development is that of the identical and expanded reproduction of the
Structures ultimately overarching moment. Everything stable, identically reproducing itself exists
here only temporarily.
[60] In Parsons' work, the constellation within the contradiction “between relative stability
and absolute instability reinterpreted in social theory as a relative instability (generated by
Conflict or dysfunction) and absolute stability (embodied by equilibrium) " 89 .
From this point of view it has been rightly emphasized that Parsons' sociological theory
within bourgeois sociology to embody the "anti-Marx". In this context, however
Marx's question was always: How is society developing? - while Parsons on the problem
the stability of society is over. 90
Parson's social theory marks both general features of late bourgeois social
theory as well as a specific variant, a specific stage in the development of the imperial
lism.
86 Quoted in: H. Holzer, Failed Enlightenment? Politics, economy and communication in the Federal Republic, loc. Cit.,
P. 436; See T. Parsons, An Outline of the Social System, in: T. Parsons, KD Naegele, JR Pitts (eds.), Theory of
Society, Glencoe 1961, p. 38; T. Parsons, Societies, Englewood Cliffs 1966, p. 7.
87 T. Parson, Contributions to Sociological Theory, op. Cit., P. 38.
88 T. Parsons, The Point of View of the Author, in: The Social Theories of Talcott Parsons, ed. by M. Black, Englewood
Cliffs 1961, p. 337.
89 BP Löwe, On the relationship between late bourgeois and political sociology and the political ideology of the imperial
mus, loc. cit., p. 109.
90 See E. Hahn, Social Reality and Sociological Knowledge, p. 85.
Page 30
Camilla Warnke: The "abstract" society - 30
OCR text recognition Max Stirner Archive Leipzig - 11/27/2019
The general characteristics it shares with its bourgeois critics are more fundamental
methodological and theoretical nature: 1. The theories of society as a whole are not based on
a concrete concept of society, but rather an abstract system concept; 2. Subject
of the "great theory" are not the type of structure and functioning of the given societies, but
that of society par excellence; 3. The empty formula society = system is used with the concrete
th content of the structures and functioning of state monopoly capitalism and
issued as a model for any society at all. That is, a real social status
quo is declared to be the absolute order of society; 91 4. This on a universal structure
and a universal working society that is practical and ideological
The needs of state-monopoly capitalism represented in a stylized form, applies because for
universal, also for qualitatively unchangeable. She doesn't know a story in the true sense of the word, but rather
only changes of state, at most "social change" within the framework of a given eternal,
general system; 5. The late bourgeois social theories are due to this common
men methodological and theoretical basis conservative in their essence, [61] as much as theirs
Changing manifestations too. Within the framework defined by this basic structure
there is wide scope within which different variants of the "great theory" can develop.
can be developed, more precisely: must be developed - because imperialism
ready to hand reservoir of different theories needed in order to adapt to the changing domestic
to be able to adapt to the natural conditions of the system and to the changing world situation.
L. I Brezhnev expressed himself at the international consultation of the communist and labor par-
said in Moscow in 1969: “The internal processes and the politics of imperialism are
through the growth of the power of socialism, through the liquidation of the colonial regime,
increasingly influenced by the onslaught of the labor movement. Many important features of the
modern imperialism can be explained by the fact that it is forced to accept the new conditions,
to adapt to the conditions of the struggle of the two systems. ” 92 And he marked purpose and
Result of this adjustment: “There is no doubt that imperialism will continue to strive in the future
will be to find new ways to extend its existence ... The further the Imperia-
lism with its attempts to adapt to the situation, the deeper its inner social
economic antagonisms. ” 93 E. Honecker characterized this adaptation under the conditions
changes in the changed balance of power between the capitalist and socialist camps
so: "Under the pressure of the changes in the international balance of power in favor of the
After all, imperialism tries to adapt to the new conditions of the class struggle
and still achieve his old goals with other methods. ” 94
Imperialism is forced to adapt. But that means nothing else than that he is the
lost historical initiative, 95 has lost the ability to develop its strategy largely independently
to determine. It is forced upon him through his relationship to the socialist camp, it arises
predominantly on the way of a passive, reactive process, "as a forced self-change
tion for the purpose of establishing correspondence or correspondence with [62] something else
... Adaptation to something only takes place when the object in question is unable to do so
To change others in such a way that correspondence, correspondence, can be established ” 96 .
Because, in the face of the growing power of the socialist camp, imperialism is no longer in
is able to have a decisive influence on its strategy and development because it is contradicting itself
of the two camps the dominance has passed to the socialist system, which is considered the dominant
91 Cf. H. Holzer, Failed Enlightenment? Politics, economy and communication in the Federal Republic, loc. Cit., P.
241 ff.
92 International Consultation of the Communist and Workers' Parties, Moscow 1969, p. 176.
93 Ibid, pp. 177-178.
94 E. Honecker, Report of the Central Committee of the Socialist Unity Party of Germany to the Eighth Party Congress of
SED, in: Minutes of the Negotiations of the Eighth Party Congress of the Socialist Unity Party of Germany, Vol. 1, Berlin
1971, p. 39.
95 Cf. O. Reinhold, Der Imperialismus in der BRD, in: Einheit, Berlin 1971, no. 6, p. 763.
96 G. Pawelzig, adaptation, in: Deutsche Zeitschrift für Philosophie, Berlin 1972, no. 2, p. 219.
Page 31
Camilla Warnke: The "abstract" society - 31
OCR text recognition Max Stirner Archive Leipzig - 11/27/2019
Page to a large extent determined its own development and strategy as well as that
The direction of the overall social development is determined, its behavior today is primarily based on
Survival geared towards adaptation. The adaptation of imperialism to a development
which he can no longer determine autocratically, requires a high degree of flexibility and variability.
lity in relation to its strategies, requires for the purpose of preserving the given social
basic quality that the system-internal mobility and variability is increased. That
this conversion to adaptation to the changed balance of power is not easy for imperialism
JK Galbraith testified with regard to foreign policy when he wrote: “The sixties
Years are too long in the absence of sustained and successful efforts to rethink
be counted on the darkest periods of American foreign policy. Only exceeded
of the state of the cities, foreign policy is considered the first-rate disaster area of ​​the American
life, and you will bear a large part of the blame for the misuse of energy.
resources and resources are attributed, which in turn cause unrest in the urban ghettos as well as
alienation and unrest in the universities. " 97 " Today, ten years later, we look on-
back to an apparently unbroken chain of disasters. ” 98
The increasingly history-defining role of socialism is expressed - like Galbraith in the
The lines just quoted suggests - not just directly, adding the socialist to the imperialist camp
for example, was able to impose the state of peaceful coexistence that left the space for
further undisturbed growth of the socialist system and [63] the possibility of consolidation
its dominance creates. It also expresses itself indirectly: by activating the class
struggle in the imperialist states, whose progressive forces are beginning to understand that the
socialist social order, the far greater potential for solving general societal
has economic problems, which in view of the high degree of socialization of the productive
forces appear en masse. Under the pressure of these forces, the ruling classes are forced
gen, non-system models - albeit adapted to and modified by the own system -
to allow. The self-determined own line begins, the adapting imperialism certainly
to get lost.
The implications associated with the adaptive character of social events
are reflected in the development of the “great theory” of Parsons and Luhmann. they are
rather a set of interchangeable variants of imperialist domination, depending on the situation
sens as the independent line of development of a theory, its truth content as it progresses
is understood. If they have something like a line of development, this follows those practical-
political, economic and ideological problems associated with the internal and external imperial
cunning regulatory requirements, although in the last ten years there have been
the strategic tasks, which result from the con-
frontation of the systems. But as in political practice for the reasons mentioned above
Pluralism of the parties, the pluralism of interchangeable solutions for one and the same
Stabilization task has the last word, so is the field of social concepts
Plurality of theories is the order of the day. When a reservoir to different situations and
Tasks of adapted theoretical models exist which, with regard to their theoretical content, are called
In principle, equal competitors can apply, only the expediency decides.
about which of the models is to be used, but not a concept of truth, of knowledge as itself
comprehends developing knowledge and in which earlier stages are dialectically canceled. Of course it can
on such a theory of truth only a society can insist that the historical development
actively determine the development that can give it a direction. A society, on the other hand, that
is dependent on adaptation, transforms the relationship between theory - truth - expediency
into the too short-circuited relationship between theory and expediency.
97 JK Galbraith, Lessons from a Bad Decade of American Foreign Policy, in: Europa-Archiv, No.
5/1971, p. 159.
98 Ibid.
Page 32
Camilla Warnke: The "abstract" society - 32
OCR text recognition Max Stirner Archive Leipzig - 11/27/2019
Its most mature and conscious theoretical articulation finds this short-circuited relationship of
Theory and expediency at Luhmann. On the methodological basis of its equivalence
functionalism (according to which means interchangeable in relation to every end and interchangeable in relation to every means
interchangeable goals exist), which for Luhmann precedes the causal explanation
the problem of truth is seen as belonging to the metaphysical ontology and outdated. 99 ideological
technologies are therefore not in relation to their veracity 100 and power relations, not in
with regard to their legitimacy 101 to “question”, but only with regard to their functional needs.
to evaluate the measure of the given optimization task: “Opportunism will
essential to the inventory. " 102
Uni to come back to Parsons once again: The model of society outlined here represents
offers a certain variant with regard to the solution of imperialist regulatory tasks. Of the
The focus is not on the relationship between the system and the environment, but on the target-compliant
set the direction of the internal system processes. The environment remains na-
too indefinite boundary condition. They are only interested insofar as they are the inner environment for them
Subsystems exist. The identification of conflict and dysfunction shows that Parsons does
knows very well about the existence of conflicts; but that he sees them as system-disruptive moments that
can be eliminated, can still be demoned, makes it clear that the conflicts have become a problem.
they are, but not to the extent that one could no longer hope to find them through alignment
of the members of society on system-compliant behavior, through control mechanisms, under the
to keep the stability endangering limit. Parsons' attitude reveals, writes E. Hahn, “the socio-
logical claim of a society that must endeavor to achieve the best possible
to hold on tight to [65] carefully shield oneself against all developments and everything new. The So-
ziology becomes the control mechanism that prescribes the actions of individuals in the
Lanes hold. " 103
Since Parsons went public with his equilibrium model, the social
social situation of capitalism changed. The abundance and depth of its internal system conflicts
no longer allows it to be trivialized theoretically and equilibrium as a given normal
state to be assumed. Under the pressure of realities and its conflict theory critics
Parsons has made a “turn” in recent years that has led him to reevaluate the conflict
and leads to certain evolutionist ideas. 104
[66]
99 Cf. N. Luhmann, Sociological Enlightenment, Opladen 1971, p. 53.
100 See ibid, pp. 54-65.
101 Cf. N. Luhmann, Sociological Enlightenment, loc. Cit., P. 162.
102 Ibid, p. 168.
103 E. Hahn, Social Reality and Sociological Knowledge, op. Cit., P. 77.
104 Cf. T. Parsons, Evolutionäre Universalien der Gesellschaft, in: Theorien des Sozialwandels, ed. by W. Zapf,
Cologne / (West) Berlin 1970.
Page 33
Camilla Warnke: The “abstract” society - 33
OCR text recognition Max Stirner Archive Leipzig - 11/27/2019
IV. R. Dahrendorf's game theory impoverished concept of conflict
If, due to the escalation of the fundamental contradiction in imperialist society, a
sharpening and multiplication of social conflicts, they demand a different kind
the treatment. It then proves to be useful to transfer them from system-destroying to system-maintaining
to repurpose tending forces and not the "social change" within the framework of the given system
only to allow, but to place oneself at its head so that it does not go in uncontrolled ways.
runs. The conflict theo-
retic critic Parsons' on the scene. G. Koch grasps the social background of the
entire conflict-theoretical direction, if in relation to R. Dahrendorf's model of the "open",
"Pluralistic society" formulated: Dahrendorf describe in this model "the matter
only what the leading groups of the monopoly bourgeoisie under the conditions of the state monopoly
already practicing cunning capitalism and striving for it by means of increased political reaction:
forms of movement for the fundamental contradiction of capitalism and the resulting contradictions
to find sayings, the struggle of the working class and the democratic movements of the popular
to paralyze the demands of the technical revolution and the socialization of the
to realize production and life processes in such a way that the new ones that grow out of these processes
productive forces, functions and organs are subordinated to monopoly capital ”. 105
With regard to its theoretical approach, the conflict-theoretical conception is the abstract counter-
part of the equilibrium-theoretical concept of society by Parsonsian [67] coining. Of choice
Of "consensus", "integration", "functionality" and "stability" as the central sociological
tegories are opposed to an alternative conceptual structure, which deals with concepts such as "compulsion",
"Conflict", "Disintegration" and "Change" grouped. D. Rüschemeyer grasps the undialectical
Character of this opposition when he writes: “The proposed alternative program
is largely in a simple reversal of the criticized: instead of value consensus
coercion and power, instead of integration, inconsistency and conflict, instead of functional
nality Dysfunctionality is set as the central concept. ” 106
At first glance, it might appear that bourgeois sociology, by
flikt "and" social "change makes the basis of social theory, the dialectical contradiction
saying discovered for itself and took the standpoint of dialectical development theory. This one
pressure is to be increased by assurances such as the one that "in conflicts the fruitful
to recognize bare and creative principle ”. 107
Indeed, in the concept of conflict, bourgeois sociology has shown the contradiction in terms of its
social order noted because it shows the “philistine 'tenderness' for nature
and history ... striving to free them from contradictions and from struggle ” 108 , at
Can no longer afford the punishment of their downfall. But the contradiction comes on the conflict
brought down as a stillbirth, as a term that, in its abstractness, is
There is a lack of motivation and power.
Let us test this assertion against Dahrendorf's conception of conflict: “The concept of conflict
is initially intended to denote any relationship of elements that are characterized by objective ('latent') or
subjective ('manifest') opposites. " 109 And" socially, a conflict should be
if it can be derived from the structure of social units, that is, it is supra-individual. ” 110 Dem-
there are objective (latent) and subjective (manifest) opposites between social
social groups (social units).
105 G. Koch, "Social action" versus practice, on the sociological conception of Ralf Dahrendorf, in: DZfPh, Berlin 1965,
H. 7, p. 804.
106 T. Parsons, Contributions to Sociological Theory, loc. Cit., Introduction by D. Rüschemeyer, p. 23.
107 R. Dahrendorf, Society and Freedom, Munich 1965, p. 227.
108 WI Lenin, vol. 38, p. 125.
109 R. Dahrendorf, Society and Freedom, op. Cit., P. 201.
110 Ibid, p. 202.
Page 34
Camilla Warnke: The "abstract" society - 34
OCR text recognition Max Stirner Archive Leipzig - 11/27/2019
If I understand Dahrendorf correctly, then the differentiation set in the definition means from
objective (latent) [68] and subjective (manifest) opposites that the first hidden
genes, essential, the second appearing opposites that the groups are aware of are supposed to be.
With regard to their relationship to one another, Dahrendorf claims: “The causes of conflict
ten - in contrast to their manifest individual objects - cannot be eliminated; therefore
When resolving conflicts, it can only be a question of the visible appearances
application forms ... " 111 This statement contains the following implication: The objective (latent)
essential conflicts determine the subjective (manifest) conflicts, the conflicts on
the level of appearance, but there is no way to influence the objective conflicts.
They are, if perhaps not unrecognizable, then practically indissoluble. With this philo-
sophical opinion - and it means to take a philosophical position when one is in
makes such a fundamental decision with regard to the relationship between essence and appearance
- Dahrendorf presents himself as a student of Kant.
As is well known, Kant claims that we understand the essence of things, the "things in themselves"
cannot recognize, but are inescapably prisoners of the world of appearances. These
However, Kant restricts the statement by the condition: Provided that we generated
the "things in themselves", so we could also recognize them. But that is not possible for man
but only to the divine Creator.
Dahrendorf proves to be Kant's successor when he ascertains two types of conflict: that
Manifest conflicts obviously generated by the socially active groups, which -
because self-generated - human activity is also accessible, and conflicts to which the societal
socially acting actors have no influence, which - according to the compelling logic of Kant - then however
also cannot have been brought about by human actions. Because provided that
Even the objective (latent) conflicts are the product of social action, so they are
of course changeable. Assuming they weren't, Dahrendorf would have with his assertion
right that they are irrevocable. Dahrendorf's postulate that objects cannot be influenced
It identifies tive conflicts as not man-made [69]. The objectivity of the social
union relations (conflicts) is a fetish. Separate from social behavior , except
and over it, a realm of conflict is established in itself, which - because it cannot be influenced - as eternal
is applicable. Dahrendorf expressly states this when he writes that it is necessary “that the conflicts
general as well as given individual opposites by all those involved as inevitable, yes as
be recognized justifiably and sensibly ... This means, however, that every intervention in conflicts
limited to regulating its forms and trying in vain to eliminate them
Causes renounced ". 112
This independence of the objective conflicts expresses itself, admittedly in a distorted and
luted form, the insight that in class societies in general and to an extreme extent
in late capitalism made social relations independent in two senses:
"Firstly, in the sense that these relationships take an objective and, to that extent, independent form.
to the individuals and their real behavior, assume that they are in things and the
Embody, materialize relationship, the movement of things, of urgent objects.
And second, in the sense that individuals have power and control over their own, over the
arising from their individual empirical behavior and determining this in turn
Losing relationships. ” 113
When the fundamental social conflicts are no longer under social control
the following statement by Dahrendorf is conclusive: “Antagonisms and conflicts appear
Then no longer act as forces that push for their own abolition and solution. ”In a Marxist way
This statement is correct as long as the accent is placed on "appear" and it is supplemented by the
111 Ibid, p. 228.
112 Ibid.
113 E. Hahn, Historical Materialism and Marxist Sociology, Berlin 1968, p. 79.
Page 35
Camilla Warnke: The “abstract” society - 35
OCR text recognition Max Stirner Archive Leipzig - 11/27/2019
Determination to whom these forces appear so that the statement must be exactly: “... appear
bourgeois consciousness then no longer acts as forces acting on their own abolition and dissolution
push. ”But Dahrendorf does not mean“ appear ”in the philosophical sense, but makes one
Actual statement. He ontologizes the antagonisms and conflicts when he continues: “... you do it yourself
the human meaning of history: societies remain human societies,
[70] as far as they unite the incompatible and keep the contradiction alive. ” 114 Der
The philosophical sense of this statement consists in the production of the following relationships
hangs: Because conflicts on the level of objectivity are not self-generated, they do not live
the principle of self-movement, which is why they cannot cancel themselves either. So are
but they are in principle unsolvable. But if antagonisms and conflicts are fundamentally unsolvable,
do they rule any society in general: “Societies do not differ in that
there is conflict in some and not in others. Societies differ in their violence
the intensity and intensity of conflicts. ” 115 This is a distorted reception of the Marxist antagonism.
verdict principle on the part of bourgeois social theory as well as the following. Anged
In view of the irrevocability of the objective contradictions, there is actually no sufficient reason
to go beyond their resigned recognition and enthusiastically find the fruitful and in them
Celebrating the creative principle as Dahrendorf does. The emphasis on their productivity is just there
in the place where contradictions as the self-moving and moving principle of social pro-
processes are considered where they are considered forces that have exactly those properties that Dahrendorf
denies them to press for their own abolition and solution and to resolve the old
to have already produced the new contradiction. Only where the contradiction by definition
able to produce something qualitatively new, it can usefully be considered fruitful and creative
are designated.
But the conflicts at Dahrendorf also create something really new on the level of appearance
not. Because even with regard to the subjective (manifest) conflicts it is not a question of dealing with them
solve, in qualitatively new contradictions to resolve, but only to "regulate" them.
Dahrendorf expressly emphasizes that the conflicts must be "managed" and "preserved" at the same time.
ten 116 that "every intervention in conflicts is limited to the regulation of its forms", that it applies
"To exploit their variability", to "channel" them in a binding way 117 , to "tame" 118 etc.
Neither should anything new be created by influencing the conflicts. Rather, it goes
it is about the [71] through the basic social contradiction, the essence of imperialism
to treat the multiple conflicts caused at the level of appearance in such a way that they do not arise
come to a head and develop system-destroying potencies. Sense of "channeling", "fractionation" (R.
Fisher), "fanning out" (A. Etzioni) of conflicts is to use forms of movement for the fundamental contradiction
to create the verdict of capitalist society, and not to abolish it. Hence stand
only the quantitative determinations of the conflicts for debate, whose "intensity", "violent
speed ”,“ radicalism ”,“ rapidity ”etc.
New methods of handling conflicts are thus being used instead of the simple, crude ones
Suppression proclaimed, smarter methods, which at the same time indicate that the. Bourgeoisie
in view of the multiplication of conflicts can no longer rule in the traditional way. Dahren-
dorf expresses the perceived danger as follows: “Whoever they (the conflicts - CW) through recognition and
Regulation tames, so has the rhythm of the story under his control. Who this taming
spurned, has the same rhythm as his opponent. Where conflicts are suppressed because they are
appear as an annoying resistance or should be eliminated once and for all, this hal-
treatment in the unexpected setback of the oppressed forces. ” 119
114 R. Dahrendorf, Society and Freedom, loc. Cit., P. 130.
115 R. Dahrendorf, Society and Democracy in Germany, Munich 1965, p. 171.
116 Cf. R. Dahrendorf, Society and Freedom, loc. Cit., P. 130.
117 See ibid, p. 228.
118 R. Dahrendorf, Society and Democracy in Germany, op. Cit., P. 173.
119 Ibid.
Page 36
Camilla Warnke: The “abstract” society - 36
OCR text recognition Max Stirner Archive Leipzig - 11/27/2019
In the relationship between conflict and consensus, consensus also applies in the conflict-theoretical approach
than the overarching moment. Conflict integration for the purpose of maintaining stability has the last
Word.
On the occasion of the comparison of the equilibrium-theoretical and conflict-theoretical
I came to the correct conclusion that “the dysfunction through out-regulation
the conflicts about adhering to given behavioral 'rules of the game' have been repurposed
on the ultimate goal of a match ... This impression is reinforced by the fact that there is
yes, it is not about the uncovering of the causes of conflict and their possible subsequent elimination,
which, on the basis of conflict theory, is also beyond its possibilities, but that “it
The point is to direct the 'destructive' tendencies of dysfunctionality in such a way that they
become 'productive' [72] forces. In this way, the 'conflict' is understood functionally, without
to have clarified the cause. ” 120
The fact that the conflict is only functional, but not causal, primarily in relation to his
quantitative, but not with regard to its qualitative aspects, it is considered that the fundamental
additional community of interests of the conflicting groups, parties, etc. is assumed that
maintain themselves in the conflict, so that the conflict is predominantly taken from the abstract side,
has to strongly tie the conflict-theoretical conception of society to game-theoretical
positions led.
This connection also exists in conflict concepts such as Dahrendorf's, which the game theory
certainly not in that of v. Neumann and Morgenstern developed a mathematical version in
Spruch takes (like certain branches of conflict research), but only verbally with certain
bypasses their terms. Even if Dahrendorf is certainly not a game theorist in the true sense of the word
is, he still skilfully handles game theory ideas because these - like us still
will see - are suitable in their formality to support his abstract conception of conflict.
For Dahrendorf, conflicts are carried out according to the rules of the game: “So it is correct that the thought
of the conflict presupposes a common context for the contestants ... it means two things, namely-
Lich certain rules of the game of argument and a power structure within which these
takes place. " 121 And:" The manifestation of conflicts, such as the organization of conflict groups
pen, is a condition of the possibility of regulation ... If all these conditions are met, then
the next step is that the participants agree on certain 'rules of the game' according to which
they want to resolve their conflicts. This is certainly the decisive step in any social regulation
Conflicts; ... 'rules of the game', framework agreements, constitutions, statutes, etc. can only be
become effective if they do not favor or disadvantage anyone involved from the outset,
limit yourself to formal aspects of the dispute and the binding channeling
of opposites. ” 122
D. Senghaas pointed out the formality of Dahrendorf's concept of conflict by writing
criticized: "Instead of asking about [73] the causes of conflicts, Dahrendorf sets the task
to determine their forms of expression ”; and yet the hypostatization of social conflict is as
Element that promotes progress is analytically inadequate if it keeps the conflicts awake
Forces and constellations of interests are assumed to be simply given ... Now, however
precisely the explanation of the necessity of conflict, the question of the cause of the degree of it
Inevitability, one of the most intriguing, if difficult, problems in conflict resolution
schung. " 123
It is doubtful that conflict research insofar as it is further than a highly formal theory
understands who can meet the demands made by Senghaas. The question of the cause of
120 BP Löwe, On the relationship between late bourgeois political sociology and the political ideology of imperialism,
loc. cit., p. 367, note 63.
121 R. Dahrendorf, Society and Democracy in Germany, op. Cit., P. 239.
122 R. Dahrendorf, Society and Freedom, op. Cit., P. 228 f.
123 D. Senghaas, conflict and conflict research, in: "Kölner Zeitschrift für Soziologie und Sozialpsychologie", Cologne
1969, no. 1, pp. 35, 36, 40.
Page 37
Camilla Warnke: The “abstract” society - 37
OCR text recognition Max Stirner Archive Leipzig - 11/27/2019
social conflicts, according to their necessity and inevitability, sets - should they
be answered - not a formal conflict concept, but the qualitatively determined concept
of social contradiction with all its implications. And this one is miles away
removed from the concept of conflict. When Dahrendorf speaks of conflicts in all societies
exist that societies are alive only insofar as they exhibit conflicts, so is it
something fundamentally different from what was meant by the Marxist philosophy that every
Social formation is structured and moved by its immanent contradictions.
In order to clarify this difference, let us return once more to Dahrendorf's concept of conflict
return. We have seen that Dahrendorf faces the unchangeable, objective, and therefore eternal conflict
juxtaposes the influenceable, manifest, changeable conflicts. We saw that
this operation brings about the separation of the conflict “in itself” from the empirical conflicts.
Dahrendorf thus has the separation of the universal, typical of the bourgeois philosophical style of thinking.
common and empirical in relation to the concept of conflict and a gap torn open.
sen that can no longer be bridged. Because the two conceptual levels cannot exist without the
To convey a concrete, general term is all that remains for him - in order to bring them under one roof - only
the way of subordination under a general [74] term: "The concept of conflict should ...
denote the drawing of elements that are characterized by objective (latent) or subjective (manifest)
Can denote opposites. ”Undoubtedly, this provision is general enough to em-
pirical conflicts of every type, but only to the extent that they
Have the property required by definition. All other characteristics - their specific sub-
differences and their respective causes, the wealth of types of conflict, the quality of the
social contradictions, the degree of materiality of the conflicts in their respective
level hierarchy - are leveled in this dead abstraction.
That Dahrendorf's reception (and the conflict-theoretical reception of the contradiction
main) is aimed at leveling the qualitative, is already evident from the choice of the term
flikt ”instead of“ contradiction ”. "Conflict" is a phenomenon and, instead of at the level of
appearance. In diverse conflicts in all spheres of social life express themselves
sen deeper contradictions, so that the relationship between contradiction and conflict is that of
Essence and appearance is. The conflicts brought about by the contradictions alternate
the empirical conditions under which the contradictions are realized. They are in their nature
as a phenomenon not only of the contradictions expressed in them, but of the
the entirety of the circumstances.
The antagonistic class antagonism between the bourgeoisie and the proletariat is thus expressed in abundance
of changing and changing constellations of conflicts at all levels of the social
social life (in the form of economic, political, ideological etc. conflicts),
while the underlying contradiction remains relatively invariant with respect to this.
One can refer to this abundance of social conflicts and their diversity theo-
behave retically in different ways: in a Marxist way, that is, one understands them as
Appearances of underlying contradictions (which includes identity and differences
to grasp the possibility of conflicts and contradictions), and in a civil way, for which
Dahrendorf decides [75]. Dahrendorf initially takes the conflicts for himself, as mere occurrences
nings. He separates them from their essence by identifying their concrete causes (the specific social
social contradictions) and instead binds the conflicts to an abstract, not
relative, but absolutely invariant being that should be the basis of every conflict. With that embarks
Dahrendorf to that night when all cows are black. In relation to this abstract, invariant
The essence of the conflict is that all conflicts are actually the same - the same in a formal and
external sense.
That would be completely irrelevant if Dahrendorf were not an ontological operation of his thought
There would be a turn. But abstraction, all possible social conflicts
identified, then appears in Dahrendorf as the real way of being of the conflicts. After that everyone is
Page 38
Camilla Warnke: The “abstract” society - 38
OCR text recognition Max Stirner Archive Leipzig - 11/27/2019
social conflicts in terms of their social status, the result of an unchanging
invariant, eternal essence: the conflict of "below" and "above" that occurs in every society
Shank form, permeated in every group of people, under all historical circumstances.
It is only logical that the abstract concept of conflict should also be an extremely meager abstract one
Content has nothing but the opposites of "below" and "above". With regard to this, Dahren-
dorf: The division into an “above” and “below” , this “inequality”, which is the cause of all conflicts
be, the division of society, “into those who benefit from the existing conditions because
they 'sit on the trigger', and those who are dependent and 'but cannot change anything' ”must be called
"A basic social fact" must be accepted. 124 Now at the latest reveals the ontification of the
abstract-general concept of conflict and the thought operation that produces it make sense:
If all conflicts are in principle the same, then the class conflict is between proletariat and
The bourgeoisie is not socially more essential than the conflict between CDU / CSU and SPD / FDP
around - shall we say - questions of financial policy. Then there is no difference between the conflict
that occur in socialist states and the class antagonisms in the imperialist one
Society, unless there are differences in intensity.
[76] The class antagonism between the bourgeoisie and the proletariat, von Dahrendorf "industrial conflict"
called, is thus the same conflict among the same, nothing but a subsumed special case of the eternal
Conflict structure “below” and “above”, a principally accidental phenomenon of industrializing
Companies " 125 . Through this turn of being an equal among equals, the class
in contrast, pushed off into the sphere of appearance, relativized, equal to all other conflicts -
posed and thus comes as a general determining cause for social movement
out of the question in class societies. Like all other conflicts, the "industrial contro-
flikt "to" tame ", to" cope with ", to" channel "etc. and to" maintain "!
The transformation of the class struggle into an "industrial" conflict would not be complete,
if Dahrendorf did not have to propose a new - of course - abstract class concept,
who is tied to “rule”, to the terms “below” and “above”: “Classes are conflicting
greedy social groups whose determining factor ... in the share in or exclusion of dominant
society lies within any ruling associations. ” 126 With that, the concept of social
Class so emptied by overstretching and expansion that it is in relation to almost all social
applies to economic groups in all epochs, but in relation to these almost nothing
able to testify. The ideological consequences of the class concept thus emptied become clear.
if you read the following passage in Dahrendorf's “Homo sociologicus”: “Arbeiter und Un-
Entrepreneurs are carriers of two roles, which are caused (among other things) by contradicting role expectations
are defined ... The conflict between workers and employers only exists insofar as the gentlemen
A, B, C Holders of the position 'Entrepreneur' and Messrs. X, Y, Z Holders of the position 'Workers'
are. In other positions - e.g. B. as members of a football club - can A, B, C and X, Y, Z
be good friends. " 127 If you take this passage together with Dahrendorf's class term.
men, one comes to the following conclusion: If the men X, Y, Z in the football club because of their
sporting qualities and experience holder of the position "board member" and the gentlemen A,
B, C would be carriers of the position "simple member" due to a lack of [77] sporting qualities,
then we would not only have a new class in front of us, which in all respects the conditions of the Dahrendorf
class definition, but even the ideal case of a counter-class in which the master
have reversed relationships. And we would have the recipe in our pocket, the "industrial"
To “tame” conflict: by forming new “classes”, ideally by forming counter-classes.
The stylization of the contradiction to conflict that Dahrendorf and other conflict theorists suggest
take, makes it possible to follow the social conflicts (including the class struggle)
124 Cf. R. Dahrendorf, Society and Freedom, op. Cit., P. 154.
125 Cf. R. Dahrendorf, Social classes and class conflict in industrial society, Stuttgart 1957, p. 138.
126 Ibid, p. 139.
127 R. Dahrendorf, Homo sociologicus, An attempt on the history, meaning and criticism of the social role, Cologne / Op-
laden 1964, p. 51.
Page 39
Camilla Warnke: The "abstract" society - 39
OCR text recognition Max Stirner Archive Leipzig - 11/27/2019
the model of a two-person zero-sum game. Let us remember: the concept of
According to Dahrendorf, conflict should denote any relationship between elements that are
tive or subjective opposites can be identified.
Such an abstraction can be reconciled with the ideas of the game theoretic mo-
dells can be brought together easily and without intermediation stages. It is not difficult to
opposing elements, which as such have no qualitative determinacy with respect to one
to fixate the only quality of being opponents in a game, to be "players", that is, to be the decision-makers.
to act as agents of “which of the possible actions they can take against each other
want to realize ". 128 The abstraction from the concrete nature of the opposites in conflict is
at the same time abstraction from the specific nature of the relationship between them that they cause,
so that these can be understood from the sole point of view of being a player. A game
but is defined by the entirety of the rules of the game which describe it 129 and these rules of the game
determine the possible action options (strategy set) from which
the players choose their strategy, decide on a certain strategy. After that will be
Conflicts are therefore carried out by the players according to common rules of the game and through selection
optimal strategies strive to achieve the highest possible profit, with the game solution - provided it is
"Reasonable" players act - the [78] players a maximum possible mean profit or
minimum possible mean loss guaranteed. The aim is to achieve a state of equilibrium.
“This maxim”, writes Fr Ruben, “is the necessary result of the premise, a conflict
to study under precisely fixed conditions, that is, to disregard its historical determinacy .
hen , that is, that the physical execution either changes the conditions of the conflict or
but completely eliminates him from the world by disappearing one of the opponents. ” 130
It cannot apply here to the two-person zero-sum game or to other models of game theory
will be discussed in more detail. For our topic it is only important to understand what the game theory is about
understood conflict is abstracted in relation to the real conflict: from the objective, qua-
litatively determined relationship of the opponents to one another, from the fact that real conflicts do not
In any case and under all circumstances, strive to achieve a balance that the objective and
tional “acting people when choosing their strategy do not necessarily have to adhere to it
of "keeping the rules of the game, etc.
P. Ruben makes this distance between the abstract concept of conflict and the real social
conflicts in the wage conflict between capitalists and workers are clear: according to the game theory
The opposite assumption of the two-person zero-sum game would have to be the opponents in the wage conflict
Strive for a position of equilibrium that is given when the price and value of labor are identical.
The workers then acted in accordance with the rationality requirement of this variant of game theory
“Reasonable” if they are satisfied with the maximum possible mean profit. These
But assumption already presupposes what needs to be clarified first: whether the essence of the wage conflict
underlying contradiction in general, the relationship workers and capitalists as
to interpret a conflict in which both opponents are interested in each other as partners
and whether both opponents are willing to adhere to these "rules of the game". But that can
game theory does not fundamentally decide.
P. Ruben is right when he writes: “... game theory as such does not deal with the question of whether it is
generally applicable [79] is that the worker is interested in fixing the price of labor
to assume. For them, as a mathematical theory, it is only a matter of determining: If the
game theoretic conditions for handling conflicts are met somewhere, so one can click on
they use the game theory tools. Whether these conditions exist, and completely
how they came about are questions that cannot be answered in game theory ... ” 131
128 P. Ruben, Strategic Game and Dialectical Contradiction, op. Cit., P. 1371.
129 See ibid.
130 Ibid, p. 1374.
131 Ibid, p. 1387 f.
Page 40
Camilla Warnke: The "abstract" society - 40
OCR text recognition Max Stirner Archive Leipzig - 11/27/2019
The contradiction underlying the wage conflict is the antagonistic class antagonism of
Capitalists and wage workers. This implies “that the opponents don't even think about themselves
to behave according to the game-theoretic rationality recommendation. Instead, the
In this sense, 'irrational' behavior is the order of the day, i.e. the endeavor to lower the price of labor
force to be fixed as far as possible from the value ". 132
It also implies that the aim of the workers as a class is not "to sell their labor
to the capitalists as a 'normal human condition', but vice versa the conditions
abolish genes under which they have to sell their labor. ” 133
And that means with regard to the “rules of the game” that these “are not a result of the agreement with the
Capitalist class, but simply the objective conditions of struggle given to it historically
(are). These conditions of struggle are not an object of preservation for the workers, but they
are to be eliminated the other way round! With the socialist revolution, the rules of the game also become
of resolving conflicts between workers and capitalists. It is exactly the histo-
ric task of the working class to abolish it! ” 134
The assumptions associated with the two-person zero-sum game, which in game theory make sense
are full abstractions, appear in bourgeois sociology converted into statements of the actual state in
in relation to the subject of society in general. We now recognize in the assertion that social
an economic equilibrium of the conflicts is to be striven for in every case and under all circumstances,
because equilibrium is the "reasonable", the "normal" state of society, the rationality
demand for the two-person zero-sum game again. The assumption of the fundamental in-
Conservative interests [80] of conflicting social forces who, according to binding "game rules"
apply their controversies, the postulate that conflicts must always be dealt with in such a way that the
The framework of the given society is not broken, also proves to be ontologized
Assumption of game theory concepts. The abstraction "rules of the game" that applies to the
Two-person zero-sum game the assumption of a fundamental community of interests
which includes opponents, namely the requirement that the players in the game each other
received occurs as a social doctrine. P. Ruben exposes the ideological meaning of these unre-
inflected transference when he remarks that these rules of the game are “considered by the bourgeois ideologues
'General-human' rules for dealing with 'conflicts in general' are (are) proclaimed with
the only sense to ideologically adhere the workers to the capitalist systemic conditions
bind: This ideological function is realizing the legal opportunities today with particular devotion.
social democracy by adopting the bourgeois rules of the game under the name of 'democracy'
and suggest their acceptability for workers with the adjective 'social'. One
'Conflict research', which is based on the 'rules of the game', is within the framework of bourgeois condi-
A research was carried out to explore the possibilities of precisely applying the bourgeois rules of the game
conserve, that is, to preserve the capitalist system ”. 135
We have the relationship just discussed between Dahrendorf's concept of conflict and the model
of the two-person zero-sum game the typical mode of interweaving system knowledge
societies and social theory under the auspices of the bourgeois style of thinking before us.
Game theory in its various approaches constituted itself as abstract mathematical
Theory. She is capable of knowing conflicts of various kinds with regard to certain formal aspects.
socially, regardless of whether it is an individual or a social one
social, economic, military or other conflicts.
Because game theory has its philosophical problems: its methodology, the character of its theory,
thus their relation to objective social reality is not or only insufficient [81]
reflected, it was inevitable that their ideas and scientific results
to the essence of conflicts in general were absolutized.
132 Ibid, p. 1388 f.
133 Ibid, p. 1389.
134 Ibid, p. 1385 f.
135 Ibid.
Page 41
Camilla Warnke: The "abstract" society - 41
OCR text recognition Max Stirner Archive Leipzig - 11/27/2019
This hypostatization can primarily be described as the variant of the "great theory" of the bourgeoisie
Society whose central project is the theoretical processing of the
social conflicts in the sense of maintaining the imperialist social order
tion is. Significantly, the arsenal of possible types of games always turns into that
Model of the two-person zero-sum game is received because of its specific abstractions
because of the conflict-theoretical conception of society and vice versa: the
The ideological concept of society in conflict theory has none because of its ideological concept of society
Difficulty in absorbing these results of game theory directly and without reflection.
It transforms the results of game theory into ontological statements about the nature of the real ge
social conflicts.
This is a mechanism of mutual support between game theory and conflict theory
Set in motion a conception of society that made game theory under imperialist conditions
gen almost impossible to break free from the shackles of the ruling ideology. Because the
Conflict-theoretical social concepts match theoretical concepts and ideas
based on societal models, it appears as if game theory is based on its-
on the other hand, had its philosophical background in these comprehensive social theories and as if
what she understands by conflict is actually identical with social conflict. There
game theory sees itself confirmed in the concepts mentioned, finds its findings again,
identification with their political implications is only too obvious.
If - as P. Ruben remarks - the “possibility of abuse (of game theory) in your own
Framework is impossible to prevent ”, 136 this statement is even more true when game theory
is embedded in the broader framework of bourgeois social concepts. Too apologetic
Assigned for specific purposes, it is only misused.
[82]
136 Ibid, p. 1387.
Page 42
Camilla Warnke: The “abstract” society - 42
OCR text recognition Max Stirner Archive Leipzig - 11/27/2019
V. The nightmare of N. Luhmann of the "overly complex", "ruthless"
environment
1. Transformation of the theory of open systems into "frame of reference relativism"
As we have seen, one of the roots of conflict theory is the urgent need for imperial
cistical system to deal with its intrinsic conflicts. No less significant
However, conflict research received impulses from dealing with international conflict
ten 137 : mainly from the confrontation with the socialist system and with problems that arise
arising from the relationship of the imperialist system to the developing countries.
Since the 1950s, since the Cold War era, the study of international con
conflicts in certain aspects of it are the subject of sociological research. Their most important
The aim is to optimize the imperialist political strategy in relation to the growing social
socialist world system: the development of strategic variants with regard to the ratio of
“Profit” and “costs” for different and changing political constellations.
In this context, deterrent strategies were developed and threat systems were investigated (T.
Schelling), devised methods to defuse conflicts and to confront them with the danger of a
to steer nuclear war into nonviolent channels, etc., whereby - as D. Senghaas remarked critically
- these questions are treated completely outside of their socio-economic context. 138 Also in
In this branch of conflict research, conflicts are not
investigations. The handling of international conflicts has undoubtedly made the interaction great
of systems and their surroundings have become the focus of interest, even if the di-
vergence and contradiction [83] in the behavior of the subsystems to one another (subsystem and internal
environment) may have contributed to the relationship between system and environment as
Make the problem clear.
As long as the focus was mainly on the regulation of system-internal processes,
the system could be examined as a closed model. This came about the environment
with the assumptions: the environment restricts the system's scope for decision-making,
but it is otherwise indifferent to the system. The investigation of international
Conflict has made this limiting assumption impossible. The closed one had to go through
open model, which is governed by the following assumptions:
1. There is a more or less pronounced conflict of interests between the system and the environment.
That is, “the environment is viewed as a consciously acting, active decision-making unit that
opposes at least some of the system's interests. It consists of one or more
players who are in a variable alternative or cumulative relationship of interests to the system
stem. " 139
2. Since the environment is viewed as an opponent or teammate that is only partially predictable and controllable.
can be grasped, it must be assumed to be largely unstructured and extremely complex.
3. The system not only reacts to environmental influences, it not only behaves adaptively, but
also has a certain autonomy towards the environment. It processes information from the
Environment in its own way and behaves purposefully in this conflict of interests. 140
4. The degree of purposefulness of a system is variable. That is, to what extent a system is
Retain goals unchanged and enforce them against the environment or to what extent it is
what is necessary to change this in reactive adaptation depends on the system-environment relationship.
From assumption 4, two principally possible constraints follow for the theory of open decision models
behavior of systems:
137 Cf. D. Senghaas, Conflict and Conflict Research, loc. Cit.
138 See ibid., P. 51.
139 F. Naschold, System Control, Stuttgart / (West) Berlin / Cologne / Mainz 1971, p. 40.
140 See ibid., P. 48 f.
Page 43
Camilla Warnke: The "abstract" society - 43
OCR text recognition Max Stirner Archive Leipzig - 11/27/2019
5a. The adaptation to the environment prevails: then "the internal structure of the decision-making
unity to the problem ", [84] then means" decision ... above all system-internal complexity
reduction in constant confrontation with a complex environment ... which by itself is not
more does the necessary reduction of complexity ". 141
5b. The purposefulness of the system prevails over the environment: then it comes to
qualitative change of the environment, for "system-external restructuring", which according to F. Naschold
"Be seen as a functional equivalent to the reduction of complexity within the system"
can. 142
From this systems science specification, which is based on the theory of open systems, like her
from L. v. Bertalanffy and in which the com-
conception of complexity 143 has been received, N. Luhmann has a conception for society as a whole
which, in contrast to Parsons' and Dahrendorf's philosophical abstinence,
expressly registers philosophical claims.
Let us now turn to Luhmann's theory. Its stated aim is to create a “functional
Systems Theory ”to overcome and replace the traditional social ontology
target. For Luhmann, this also includes Parsons' social theory, which applies to everyone
Advantages could be blamed by her critics, she ultimately sees the constitution of society
Licher systems claim to be unchangeable, they secretly serve to justify the status quo, they look at
social reality as always already structurally integrated. The reasons for this narrow-mindedness
Luhmann sees Parsons methodology, in which the functions are always based on given structures
can be obtained. You have to reverse Parsons' methodology, from your head to your feet, so to speak
put the "structural-functional system theory" through a "functional-structural system theory
rie ”.
The shortcoming of the first is that “it puts the concept of structure in front of the concept of function.
As a result, the structural-functional theory increases the possibility of structures per se
problematize and ask about the meaning of system formation in general. Such a possibility
However, if one reverses the relationship between these basic concepts, i.e. the functional
concept precedes the structure [85] concept. According to the
Asking the function of system structures without using a comprehensive system structure as a reference point
the question presuppose having to 144 .
The antinomic consequences that the assumption of a comprehensive "or" universal "structural
which we have already discussed 145 , Luhmann is evidently aware that
when he rejects it as the ultimate point of reference for the function of system structures. So it has to be a
another reference is sought that "no longer implies any system structural prerequisites" 146 ,
a point of reference that is not a system. Luhmann believes that he has found this in the term "world",
whereby he wants to see the world defined as the radically other of the system, as everything else that exists,
that is beyond the limits and reach of the system.
Luhmann's line of thought is: “The term social system is supposed to mean a context of social
Actions are understood that refer to each other and do not differ from an environment.
delineation of subservient acts. If one starts from this concept of system, which is divided into
inside and outside has its constituent principle, and one tries to transcend it,
141 Ibid, p. 50.
142 Ibid, p. 77.
143 Cf. HA Simon, The Architecture of Complexity, in: General Systems, loc. Cit., Vol. X, 1965, pp. 63-76; see also F.
Naschold, System Control, loc. Cit., Pp. 138-139, which contains the views that are of interest in our context
mons reported on the complexity as follows: “The problem arises for every organization, given
to be able to act sensibly in the face of overwhelming environmental complexity. The internal organizational problem-solving
processes require a prior reduction in complexity ”.
144 N. Luhmann, Sociological Enlightenment, Cologne / Opladen 1971, p. 114.
145 See this brochure, p. 36 ff.
146 N. Luhmann, Sociological Enlightenment, loc. Cit., P. 115.
Page 44
Camilla Warnke: The "abstract" society - 44
OCR text recognition Max Stirner Archive Leipzig - 11/27/2019
then one asks for a reference unit that no longer has any limits. One asks about the world. the
The world cannot be understood as a system because it has no 'outside' against which it can be differentiated.
If one wanted to think of the world as a system, one would immediately have to think about an environment of the world, and that
the thinking, guiding world concept, shifts to this environment. ” 147
In the context of systems science questions, it is of course permissible and in certain cases
It is even essential to abstract from the systematic nature of the environment. "Environment" in
However, as Luhmann does, it cannot - as Luhmann does - with the phi-
philosophical concept of the world can be identified. This is a "unit of reference" to which it applies that it
has both limits and "no limits" and with respect to which any system as it does system in.
certain contexts, at the same time also the environment in other contexts. Luhmann re
reduces “world” to just one of these determinations, to limitlessness, unstructuredness, etc. and
thus transcends into the [86] bad infinite, in view of which logical paradoxes cannot be excluded.
can stay. The separation of the determinations that belong together in relation to the object world
mings: "inside" and "outside", "system" and "environment", "structured" and "unstructured"
and the assignment of the respective second term of the term pairs to the "world" sets precisely that
what Luhmann does not want, an “outside” of the world so determined, its limitation. System can namely
Lich, if the world is defined as the limitless, unstructured and only as that within the
World does not occur, but only outside of the world and thus proves to be its own
contrary: as environment, border and outside, with which the world itself must be thought of as a system and
contrary to Luhmann's assumption, it does imply system-structural prerequisites. the
The radicalization of the functionalist approach is no less problematic than the
Parsonsian structuralist; Luhmann's concept of the world is no less contestable than the
position of a universal structure.
It is not to be assumed that the astute philosophically trained social theorist
Luhmann could not come up with this mistake of reasoning himself, which behind Kant led him into the metaphysical
way of thinking, leads back to the two-world theory, which does not correspond to opposing categories.
able to mediate others.
If Luhmann is nevertheless not afraid to separate system and world and they rigidly
to be compared, if he is behind long-established positions of philosophical knowledge.
if it relapses, the cause is ideological. Luhmann's concern is to
term of general systems theory in v. Bertalanffyian coinage as a social term
to be able to use. In order to evade this procedure from the accusation of arbitrariness and arbitrariness,
the model of the open system is ontologically secured by placing everything that exists in the two-
division of system and world is exhausted. But "world" is not a philosophically admissible denial
remission of "environment". Because the juxtaposition of system and world results in concrete
Contrast between system and environment disappeared, their dialectical contradiction resolved. "World"
is for Luhmann by definition [87] "non-system" (and nothing else or everything that exists except
System), while the environment in the correctly understood systems science model of the open
System is perceived as a concrete world, i.e. as that area of ​​reality
which is in real interaction with the system. Only in this juxtaposition
development, so to speak from the standpoint of the system to be examined, from the systemic character of the environment
abstracted, but at the same time presupposed that the environment in other contexts itself than
System occurs, so that the interaction between the system and the environment in reality
effect of systems is. And this statement is open to further concretization: the question of what
certain determinations of being outside of a system are still possessed by those who interact with one another;
it can therefore be conveyed with the concrete philosophical concept of the thing.
The hypostatization of the environment to the world (= non-system) obstructs this path. It prevents sy-
system and environment from the aspect of their identity (as an interaction of systems).
It transforms the relative opposition (difference) of system and environment into the absolute of
147 Ibid.
Page 45
Camilla Warnke: The "abstract" society - 45
OCR text recognition Max Stirner Archive Leipzig - 11/27/2019
System and world. In other words, Luhmann makes the transition from
Monism to the dualism of system and world.
The meaning and consequences of this company, with which Luhmann typically adheres closely
Husserl leans on, is the following: “It was the pulling apart of the concept of the world and the concept of system
not possible as long as the system is defined in the classical way as a whole consisting of parts,
thus without reference to an environment. This concept of the system corresponded to a world concept, the
Tried to understand the world as the totality of beings. A radicalization of the functional question
position presupposes that this ontological conceptuality is broken open; it must use the concepts of the world
and separate systems in order to be able to relate them to each other ... are preliminary work for this
especially in the phenomenological philosophy of E. Husserl, especially through the
Differentiation between meaningful, intended identity and the horizon of all experiences that make it possible
clear to define the world as a universal horizon. ” 148
[88] The philosophical operation that Luhmann is undertaking here boils down to this, and not just that
objective structure of the world, but also to dissolve the objective structure of things.
Instead of the thing concept, according to which - from a Marxist point of view - things are relatively stable, independent
dige, hierarchically ordered systems (wholes) of qualities, a system concept occurs that
The system is principally linked to instability and qualitative indeterminacy. System is not something
that owns its own existence and essence, but merely a relationship that is constantly changing
changing relationship between a “complex identity” and an overly complex environment.
According to Luhmann, "the theory is about to ignore system theories that only
consider the system, refer to system / environment theories ”. From this indisputable fact
But Luhmann draws the conclusion: “The ontological system conception, the systems as a whole
defined, which consist of parts, and thus draw attention inward, becomes more
and replaced more by a functional systems theory that considers systems to be complex identities.
takes hold of itself in an overly complex, confusing, fluctuating environment as
can get higher order. Only when this transition has been consistently carried out can
systems theory is based on the presupposition of something that has already been determined and structurally mapped out
Solve internal order and recognize the function of system formation at all: it consists in the creation
comprehension and reduction of world complexity. ” 149
But Luhmann defines complexity as follows: “The term complexity always denotes
a relation between system and world, never a state of being. ” 150
If we take the last two statements cited together, then Luhmann's view can be reduced
characterize the relationship between system and environment as follows: Outside of
A move to the system, taken in and of itself, according to Luhmann, cannot speak of the world at all
will. Their objective nature is not up for discussion, any more than anything about sy-
stems can be predicted outside of their relationship to the environment. Being, the objective
Creativity of the systems and the environment is outside of their [89] relationship to one another
determine because system and environment are only distinguished quantitatively: namely in relation to the
Degree of their complexity, being at the same time in relation to this property (to be complex) as
identical to each other, therefore considered to be eo ipso related to each other.
This system-scientific abstraction, which is meaningful in certain contexts, is supported by
Luhmann ontifies. It is turned into a statement by means of which he describes the relationship between things
and characterizes the world. If Luhmann asserts, on the one hand, that the world is only sub-
jectively, in the perspective of systems there and that the real existing systems on the other hand only
are parts of the world, but cannot be understood as concrete totalities, then is
the consequence of the fact that he is only quantitative. but does not allow qualitative differences.
148 Ibid., Note 5, p. 131.
149 N. Luhmann, Sociological Enlightenment, loc. Cit., P. 75.
150 Ibid, p. 115.
Page 46
Camilla Warnke: The "abstract" society - 46
OCR text recognition Max Stirner Archive Leipzig - 11/27/2019
With this systemic narrowing of the concept of thing, through which things are one-sided
defined by their relation to the environment, Luhmann establishes a new variety of relativism
mus, a "frame of reference relativism", as it aptly describes his point of view
Has. 151
Although Luhmann does not directly say that this ultimately subjectivist point of view is agnostic
mus includes, although it is optimistic, lies in the definition of the world as complexity
Resignation regarding the possibility of being able to control their structures and processes.
This is shown by the attributes Luhmann uses for the environment. Environment moves without
Consideration for the system 152 , it is - as quoted above - excessively complex, confusing fluctuations
ending, it is indefinite 153 etc. It is therefore not changeable. Luhmann receives that variant of the
Decision model of open systems that was outlined at the beginning of this chapter and that of
is based on the assumption: if adaptation to the environment prevails, then the internal structure
structure of the decision-making unit on the problem, then decision means primarily system-internal
Complexity reduction in constant engagement with a complex environment. So that
Luhmann knows himself as a [90] representative of those theories that support and stabilize the
Late capitalism by means of increased adaptation to the changes in favor of socialism
want to provide a social environment.
Luhmann's program reads: Stabilization of the social system through internal complex
activity reduction: "All functionalist analyzes are ultimately related to stabilization pro-
bleme as guidelines. ” 154
BP Löwe grasps the political-ideological position and function of Luhmann's theory, if
he writes: “One resigns itself to the environment of the system in that one sees it as' on the whole not
rulable 'factor classified. And at the same time it is said that 'some important inventory
conditions' as the 'internal performance conditions' of the system related to its purpose and objectives
enabled a useful engagement with the 'environment'. With the conception of a step
program for reacting the system to influences that contradict its inventory conditions.
running is an indirect admission that the power of one's own system is losing substance
and one is therefore forced to differentiate adaptations. ” 155
L. v. With his so-called "perspectival worldview", Bertalanffy represents a Luhmann-
position very similar to the relativism of the frame of reference. For v. Bertalanffy "closes the system
also intervened a new epistemology ... the replacement of an absolutist by a personal
tivist philosophy ”. 156 This is based on the knowledge that, like every little animal in its Uex-
küll's environment, which makes up only a tiny part of the universe, 157 , including every view of the world
is only “a certain perspective of a reality that has recently been unknown, seen through glasses
general human, cultural and linguistic categories ” 158 .
Here the cautious agnosticism of Luhmann, the expressis verbis only in relation to the environment
is admitted, generalized to general philosophical agnosticism!
The rational core that lies in this perspectivism or frame of reference relativism,
consists in bourgeois ideology taking note of the fact [91]:
the environment affecting complex systems is not reflected unchanged in the system,
but broken, processed and transformed in a way peculiar to the system; or that
151 Cf. N. Luhmann, Systems Theoretical Argumentations, op. Cit., P. 385.
152 Cf. N. Luhmann, Sociological Enlightenment, loc. Cit., P. 41.
153 Ibid., P. 75 f.
154 N. Luhmann, Sociological Enlightenment, loc. Cit., P. 27
155 BP Löwe, On the relationship between late bourgeois political sociology and the political ideology of imperialism,
loc. cit., p. 216
156 L. v. Bertalanffy, ... but we don't know anything about humans, loc. Cit., P. 158.
157 See ibid, p. 197.
158 Ibid, p. 162.
Page 47
Camilla Warnke: The "abstract" society - 47
OCR text recognition Max Stirner Archive Leipzig - 11/27/2019
Appearance not only depends on who appears, but also maintains its character
from whom it appears. Of course, the dialectician Hegel already knew that without
knowledge would have inflated it to an absolute statement.
Perspectivism and frame of reference relativism, which take up positions of philosophy that have been overcome.
warm, pay a high price for this insight. The objective moment in the appearance
disappears and evaporates; the objectivity of the world is considered either as unknowable or as a
problem not to be discussed or it will be eliminated altogether.
The latter is the case with H. Rombach, who proposed an "ontology of functionalism" for the future.
prophesies, in which the central category will not be "substance" as in the old ontology -
“Substance is that which is able to exist for itself” - but “function”, whereby “function” as
"Dependence, concern for other, being in the other" is defined. 159
For Rombach, function is an abstract, undialectical negation of substance: being-in-yourself
is excluded from being in the other. This attack on the concept of substance is insofar
justified when they opposed the absolutization of the implications posed by the concept of substance.
turns against one-sidedness: calm, essence, unity, being, stability, etc. to the real
to explain certain determinations of things. But it means to cast out the devil with Beelzebub,
if one also unilaterally considers the implications Dy-
namik, appearance, multiplicity, non-being, instability, etc. ontified. Like the Luhmannsche
The concept of function also differs from that of Rombach, as we meet at Luhmann - albeit
less transparent - the same constellation.
At the center of the polemics is the old concept of substance; at the center of his own theory the
category of function. Here as there, the result can only be a restoration of metaphysical thinking
be, but not reception of the dialectic, as Luhmann would have us believe, even if he is to
Purposes [92] of the critique of the concept of substance use original ideas of the dialectic like that
the following.
He polemicizes against the idea of ​​perceiving identity as a self-sufficient substance. she
must be understood as a system, the existence of which is not based on an unchangeable essence
based on 160 , because functionalism shares “with dialectics the ontological premise: that a
the end cannot be true and permanent if it contains a contradiction to itself ” 161 . So
criticizes Luhmann that for the old substance ontology stability as the real essence of a system
Stems held 162 that they excluded non-being from being 163 that system and process
were mediated with each other, etc. 164
Insofar as the dialectic includes in itself, with the metaphysical concept of the thing the old substance
concept of having overcome, to define things - as we have seen - as systems, namely as the
Systems that are open to the environment (which include as a moment "concern for other"
ßen), it also provides good and useful arguments for “radicalized functionalism”.
But the usefulness of dialectics for functionalism ends where the horizon of the system
system and function terms would have to be exceeded where these terms - for philosophical
to be able to be griffe - would have to be concretized in the problem area of ​​philosophy, that is, in
first and foremost: where the concept of system with the concept of thing, the concept of function with the categories
Essence and appearance and these would have to be determined with the basic question of philosophy.
Functionalism refuses to tread this path. You mean the donkey and hit the sack:
The criticism of the concept of substance (of what “is able to exist for itself”) is intended at the same time
159 See H. Rombach, Substance, System, Structure, Vol. I, op. Cit., Pp. 11-13.
160 Cf. N. Luhmann, Sociological Enlightenment, loc. Cit., P. 26.
161 Ibid, p. 34.
162 See ibid, p. 39.
163 See ibid., P. 55.
164 See ibid, p. 125.
Page 48
Camilla Warnke: The "abstract" society - 48
OCR text recognition Max Stirner Archive Leipzig - 11/27/2019
Meet the concept of matter of dialectical materialism. This implies, insofar as it is not simply
a straight epistemological answer to the basic question of philosophy is taken (matter
is that which exists outside and independently of human consciousness), but as more concrete
Concept, the knowledge that nothing exists but moving matter. Taken as more concrete
The term “matter” includes its opposite. Consciousness is one more defined
Form of movement of matter, [93] "... the real unity of the world consists in its materiality" 165 ,
or: matter is that which can exist for itself. She is the self-determining
Self-moving, self-developing. For its existence it does not need any external forces.
Insofar as empirical things, relationships and processes are material, they too come as a mo-
ment in the entirety of its properties the property of being able to exist for itself (self-
moved to be self-determined etc.). Of course, they only show this property to a limited extent, only
relative, because "matter", insofar as it is a term used to describe the unity of things in more detail,
their mutual determination, including their relative lack of independence.
By totally eliminating the concept of substance, functionalism denies that things, relationships
and processes are material, and in and with this process eliminates the concept of matter. At
Its place is taken by the concept of function, which marks the transition to that frame of reference relativism
makes possible who is attached to something else, who has detached the relationship from what is
relates what is in the relationship of the interaction. But this gives interaction,
Relationship, etc., underhand the determinations for the sake of which the concept of substance fights
became: they are made into that something “that can stand for itself”. The new one, with
“Substance” filled with “modern” content is the relationship. AJ Bahm was right
made fun of this process when he called it the hocus-pocus of systems theory
and indicates that logical contradictions are inevitable, provided that the relations are
ontologically independent from things. 166
So that no misunderstandings arise: We are not talking about what is in the context of the
Systems science is permissible, but only of what is legitimate in philosophy. It will not
polemicized against system science abstractions that were proposed with the express goal
be taken, certain system relationships from different empirical objects
to detach the and to identify objects with regard to these connections.
[94] L. v. Bertalanffy, for example, is right when he considers it a virtue of systems theory
holds that "... their concepts and models refer to both material and non-material appearances.
Applying the “ 167 . He is certainly also right when he is based on systems thinking
a “ unified theory ” hopes, “in which 'body' and 'soul', in their formal and structural
respect, to be grasped in a uniform conceptual system ”if he thinks that in this way
a science could arise "in which the physical and the psychological, the unconscious and the conscious,
Physiology and psychology would be grasped by similar, very abstract constructions or models.
the. Whatever the nature of these constructions, we can be sure that the terms
of the system and the organization in it will play a central role. " 168 But this system-
Scientific body-soul theory can of course "give no answer to what reality ultimately
literally 'is' ”. 169
2. The "radicalized functionalism"
Luhmann eliminates the basic question of philosophy by replacing it with another, his opinion
after a fundamental ontological problem, replaced by the question: What is the function (the meaning)
of system formation at all? And he answers: The function of system formation is the reduction of
165 MEW, vol. 20, p. 41.
166 Cf. AJ Bahm, Systems Theory: Hocus Pocus or Holistic Science ?, in: General Systems, op. Cit., Vol. XIV, 1969,
P. 174 f.
167 L. v. Bertalanffy, ... but we don't know anything about humans, loc. Cit., P. 170.
168 Ibid, p. 170 f.
169 Ibid.
Page 49
Camilla Warnke: The "abstract" society - 49
OCR text recognition Max Stirner Archive Leipzig - 11/27/2019
World complexity. Luhmann wins this formula by using the functional-structural method
absolutized, or, as he calls it, "radicalized".
The functional-structural method works with statements like: Structure a has the function α in relation
to system A. Or: The function α for system A is fulfilled by the structure a, where
either the structure or the function is kept constant or variable. The knowloedge-
The economic concern of this method is to make functions and structures comparable.
Chen, that is, to research to what extent and by what means in relation to the task, functions or
To keep structures constant, the variable size can be replaced. It works by means of the
functional-structural method to discover the following kind of knowledge: The [95] structures a, b, c are
equivalent with respect to the function α; the functions α, β, γ are equivalent with respect to the structure
tur a. Luhmann calls this process "equivalence functionalism", the goal of which is "no longer the
Establishing a legal connection between certain causes and certain effects,
but rather the determination of the equivalence of several co-ordinated causal factors (is). The question
is not: A always causes B (or with a definable probability), but: are A, C, D, E
equivalent in their ability to effect B ”. 170
The functional-structural method was developed as early as the 19th century for the purpose of
to be able to determine internal interrelationships of a specific society. She is already at
Durkheim existed, was later mainly used by the ethnologists Malinowski and Radcliffe-Brown
further developed and given the name "functionalism". 171 The greatest importance to the
Elaboration of the functionalism had however the systematization of its theses by Merton. 172
By using functionalism as a method, the internal functional relationships of a particular
Makes the social system the subject of his investigation, he can do an essential one
To make a contribution to research into these relationships. His self-imposed limit is
that the functional relationships - as Merton emphasizes - are only ever within the framework of a given
benign system can be determined 173 , that is, it is used to record development processes
is unsuitable. This already follows from the concept of function: Function requires the specification of a
Reference system, function is function in relation to something, is - as Rombach rightly remarks - "an
reliance on other ”, whereby this other is the constant in relation to the function.
Functionalism, in the unanimous opinion of its most qualified proponents, is not
Social theory that makes substantive statements about society, the social
find, but merely a research program, "a number of methodological
fonts such as B. This: To find the explanation for a certain behavior, pay attention
all about what goals it might serve. That does not mean, however, that the [96] on this
Explanation gained by the way is necessarily an explanation by the goal ”. 174
In contrast to this measured attitude, Luhmann expands functionalism to the greatest possible extent
general theory, the statements about the relationship between the system in general and the world in general
want to do. This theory, radicalized functionalism, has "according to the function of sy-
to ask questions about stems and structures " 175 and for this purpose is related to the non-system, to the world,
more precisely: on their complexity. "The function of system formation in general ... consists in the recording
Solution and Reduction of World Complexity ” 176 .
System formation in all areas of reality (in the biological, social) solves according to Luh-
man thus a problem, namely the problem of the world, which in its excessive complexity
stands. In my opinion it is completely nonsensical to claim that the world - taken as a subject
the statement - has problems to be solved, or that the system formation has a function in relation to it
170 N. Luhmann, Sociological Enlightenment, loc. Cit., P. 23.
171 Cf. IS Kon, The Positivism in der Sociology, Berlin 1968, p. 298.
172 See ibid., P. 303.
173 See ibid.
174 Ibid., P. 296.
175 N. Luhmann, Sociological Enlightenment, loc. Cit., P. 75.
176 Ibid.
Page 50
Camilla Warnke: The "abstract" society - 50
OCR text recognition Max Stirner Archive Leipzig - 11/27/2019
the world have to meet. Habermas argues conclusively when he accuses Luhmann with the
"Radicalized functionalism" to carry out the regress into the bad infinite. Habermas
writes: “When reducing world complexity, this 'last' structure-independent reference point
If the analysis is to be, then world complexity must be something that is objectively divided before all structure formation
Problem to be introduced. 'World' must then be thought of as a 'problem in itself', so that education
of structures (namely 'first structures') can appear as a solution to that original problem. ” 177
If you replace the term "function" with "sense" in the above statement, then it becomes completely clear
That is where absolutized functionalism leads: in the assertion of an objectively given
Sense of the world. For if the formation of a system has a meaning in relation to the world, then you must
this must be given by the world, because system formation has a meaning for the world .
Thought through to the end, the “radicalized functionalism” leads to a modernized new edition of the
teleological world view, in which there are final causes in the constitutive sense, according to which everything in
the world existing for the existence of the world has meaning and function and everything according to this
receives meaning and function from this world.
Luhmann twisted the actual relationships in his radicalized functionalism. the
Processing an environment of high complexity poses systems of problems: how and with which
internal system resources that can cope with complex environmental influences. The environment
confronts the systems with problems, not with problems of the world "per se", but with problems of the
own behavior towards the environment. To put it in Luhmann's terms: Not the problem
The reduction of world complexity is through systems, but system problems are through the
To solve reduction of world complexity.
“Radicalized functionalism” has little to do with functionalism as a method
to do. Only the form of the statement indicates its origin. If in the functionalist
Theorem: The function α for the system A is fulfilled by the structure a, the values ​​are inserted:
“The function of reducing world complexity in relation to the world is performed by systems and
Structures ", the scientific value of the functional-structural method is shown.
lifted. According to Luhmann's own words - and this has to be agreed - there is the advantage of the radio
tionalism precisely in this, for the purpose of comparability, to find equivalences,
To make structures or functions variable. But from Luhmann's "radicalized functionalism
mus ”, the variability, comparability and equivalence of functions have completely disappeared.
In relation to the complexity of the world, there is only one single function that the systems can fulfill
have to reduce this complexity. But this lifts functionalism,
insofar as it is radicalized, it appears as a scientific method itself.
3. System as an actor in society and history
The question must be asked: why is this system theory stylized as a philosophy, what is it for?
soluted functionalism instead of the functional-structural method? In my [98] opinion
there can be no doubt that neither Parsons nor Dahrendorf, nor any of the representatives of the
"Great theory" with the same awareness, determination and consistency as Luhmann ans
Work has gone to a philosophically founded alternative to historical materialism
mus to work out.
The transcendental-philosophically expanded theory of open systems is intended to provide a general
social concept, which applies the Marxist concept of society to generality and modern
outperforms. It has to take on the role that philosophical materialism opposes
plays to historical materialism.
Let us proceed to the discussion of Luhmann's concept of society. Society is a system -
namely an open system - belonging to the type of social system. What society is can do
According to Luhmann, when we ask about the function, we recognize society in relation to the
177 J. Habermas, Theory of Society or Social Technology? A discussion with N. Luhmann, op. Cit., P. 153.
Page 51
Camilla Warnke: The “abstract” society - 51
OCR text recognition Max Stirner Archive Leipzig - 11/27/2019
Environment has to meet. This function consists - like the function of every system formation - in the
Reduction of environmental complexity.
Since the environment is always much more complex than any possible system, there is a "selection
compulsion ”, a compulsion to choose in relation to experiences and actions. This selection is made by
social systems by means of the formation of meaning, whereby meaning as “a certain strategy of the
selective behavior under conditions of high complexity ”. 178 Sense is thus se-
lesson from other possibilities. 179 Under this aspect, systems of action can be “functional
define as meaning relationships between actions that reduce complexity through stabilization
an inside-outside difference ” 180 and social systems as“ meaningfully identified systems ” 181 . "Under
social system is here to understand a context of social actions that
refer to each other and demarcate themselves from an environment of no longer associated actions
let. ” 182 Since social systems reduce the overly complex environment through selective formation of meaning,
make a choice, so to speak, in comparison to which other possibilities for meaningful ideas
tifications exist, systems are only “excerpts” from an overly complex world. 183
[99] Social systems, to which society is to be counted, are with systems of other areas
(biological, psychological) in that they reduce overly complex environments, that they
Lesson achievements that are excerpts of the world. They differ from the latter only in theirs
Selection mode , in that they are integrated on the basis of meaning . It is the mode of selection
which distinguishes the systems in different areas from one another: “The identity of the system becomes
constituted by his mode of selection; it is depending on the physical system, organic system,
psychological system, social system. ” 184 It is sufficient to grasp the specifics of society
therefore - according to Luhmann - "" for systems of meaning a new type, with organic processes in concrete terms
to adopt an incomparable selection style ” 185 . If it is the systems that create meaning,
or if the formation of a social system is the formation of meaning, then - so Luhmann concludes - the concept of meaning can
cannot be derived from the concept of the subject. It coincides, so to speak, with the concept of system.
“The concept of meaning is to be defined primarily, i.e. without reference to the concept of the subject, because it is defined as
meaningfully constituted identity already presupposes the concept of meaning. " 186 The subject is not
but a meaningful system. 187
For Luhmann, society is (in terms of its function) "that social system
that in the absence of presuppositions structured by physical and organic system formations
Environment regulates social complexity and lasts for fundamental reductions ” 188 ,“ ... of which
Structures decide how high a level of complexity people can endure, i.e. in meaningful
can live and act ". 189
For Luhmann, society is, according to its general nature, a social system (symbolic system).
Insofar as it establishes “final, fundamental reductions”, it is the individual social systems
somehow superior. She regulates their relationship to one another, she takes care of it - if I am Luhmann
I understand correctly - that the social systems have a cohesion towards the environment. Luh-
man describes society “as a social system par excellence, or as a social system
178 N. Luhmann, Modern system theories as a form of analysis of society as a whole, in: J. Habermas / N. Luhmann,
Theory of society or social technology, op. Cit., P. 12.
179 See ibid.
180 Ibid, p. 11.
181 Ibid.
182 N. Luhmann, Sociological Enlightenment, loc. Cit., P. 115.
183 Cf. N. Luhmann, Systems Theoretical Argumentations, loc. Cit., P. 307.
184 N. Luhmann, Sociological Enlightenment, loc. Cit., P. 143.
185 N. Luhmann, Systems Theoretical Argumentations, op. Cit., P. 300.
186 N. Luhmann, Sense as a Basic Concept of Sociology, in: J. Habermas / N. Luhmann, Theory of Society or Social
al technology, op. cit., p. 28.
187 Cf. N. Luhmann, Modern System Theories as a Form of Overall Society Analysis, op. Cit., P. 12.
188 N. Luhmann, Sociological Enlightenment, loc. Cit., P. 145.
189 N. Luhmann, Modern systems theories as a form of analysis of society as a whole, op. Cit., P. 16.
Page 52
Camilla Warnke: The "abstract" society - 52
OCR text recognition Max Stirner Archive Leipzig - 11/27/2019
of the social systems, or as a social system that is a condition of the possibility of other social systems
steme works ". 190 Society ensures that the social systems function in a coordinated manner.
kidneys, but it is not a whole consisting of subsystems to which the social systems are subordinated
would be.
In this definition of the social system and society, the attack on the takes place in two steps
Marxist concept of society: first on its materialism, then on its dialectic.
From Luhmann's social theory, the ideological intention of his - apparently po-
litically and ideologically neutral - philosophical foundations are made transparent.
Social systems are integrated on the basis of meaning. This statement is obtained by the fact that
The subject's ability to create meaning is withdrawn and tied to the social system. This Opera-
tion can now be achieved on a philosophically demanding level if at the same time the
Materiality and objectivity of things, relationships and processes are canceled, that is, if one
A void is created in which something else can enter. This meant other, for that
Luhmann, in the tradition of phenomenology, had “sense” in mind from the start, but it has to
Apply the appearance of objectivity if it is to take the place of the concept of matter. to
for this purpose the “radicalized functionalism” is organized, by means of its “function” and
"System" are played up to ontological determinations; for this purpose the reference
system relativism established.
The latter not only has the general goal of eliminating materialism, but is also
moreover, the intention is based on the specifics of dialectical materialism. Of the
As we have seen, frame-of-reference relativism is directed against the view that things are
chisch - are articulated and coherent wholes, and thus creates the
Possibility to separate the "thing" of society in an independent way, its relation to the environment
to be able to dissolve individual realizing systems. If there are no things, then society cannot either
concrete totality, Marx is wrong when he introduced social theory to the concept of ecological
based on nomic social formation.
[101] Luhmann wants to surpass the concept of the social formation by the following conception.
ten that differ from other conceptions of the “great theory” historically. In
Succession to the archaic society (which is structured through segmented differentiation
wunde) the following types of society have appeared in history:
a) Societies that “have been determined by the primacy of one of their subsystems, namely
of that sub-system which, due to its own complexity, led the way in evolution
of the political, then of the economic subsystem ”. 191 In this type of society "it made sense
fully to understand society itself from its respective leading subsystem ... yes, it with
to identify him ”. 192
b) But this type of society, which is shaped by a leading subsystem, is
historically overcome. Today social theory has to reckon with a society in which
several functionally differentiated subsystems exist, "between which the functional pri-
mat can vary (or remain undecided) ”. 193 “The unity of society is, at least
today, no longer to be understood as a unity of a purpose or a final instance. She is ultimately
nothing more than the adjustment of a relationship of corresponding complexity between a
A multitude of social systems that are mutually beneficial to one another's social environment. ” 194
It could easily be demonstrated at this point that Luhmann provided the conclusive evidence of the
Historical research has not taken note, according to which also in the Greek and Roman
In ancient times, not the political, but the economic sub-system of all other sub-systems
190 N. Luhmann, Sociological Enlightenment, loc. Cit., P. 143.
191 Ibid, p. 142.
192 Ibid.
193 Ibid.
194 Ibid., Pp. 149 f.
Page 53
Camilla Warnke: The "abstract" society - 53
OCR text recognition Max Stirner Archive Leipzig - 11/27/2019
The fact that the rise, prosperity and decline of ancient society gave it a specific character
Have causes within the potencies and limits of the slavery-based economy.
But in connection with our topic, the Luhmanns model is of primary interest
Construction of the course of history is based. Luhmann obviously obtained this from L.
v. Bertalanffy, who for his part used the type of embryological development of the organism for general
my- [102] holds valid. Bertalanffy's model is based on the thesis that differentiation and loss
the holistic character of a system go hand in hand. Biological, psychological and
social systems move from an equipotential holistic state in which the
The performance or function of the part does not depend on the part itself, but on its position in the whole,
to differentiate into parts continued (progressive segregation), with differentiation increasing
Complexity, but also progressive mechanization, that is, independent, self-
constant functioning of the parts that lose their interaction. You're just hanging
of itself, which leads to a loss of controllability of the system. The system is approaching
a mere sum of independent parts. 195 In this context, L. v. Bertalanffy
of elements that act as “triggers”, that is, of “leading parts” that, when they themselves
change even slightly, bring about a considerable change in the whole system.
nen. 196 It is unmistakable that Luhmann accepted the model just described. Even with him
tet the development towards differentiation and progressive mechanization continues, also with him there is
Stages of development in which “triggers” determine the state of the overall system. This to
leihe leads Luhmann back into the organism, which was asserted against KH Tjaden 197
who is of the opinion that Luhmann has the organism of the bourgeois society
theory overcome.
The ontogenetic-embryological development model has implications that include a concept of
Induce history that makes it impossible to grasp the specifics of its laws of development.
The embryonic development of the organism is objectively goal-oriented insofar as its end
duct, the developed organism, is already established in the starting cell, so that
the development in this case essentially as a strictly regular unfolding of something
must be understood. This fact is called "equifinality" in biology and
describes the fact that organic systems are made up of widely variable starting points
states always arrive at one and the same final state.
[103] If history is subsumed under this type of development, it is assumed that it is one
takes a steadfast course and strives towards an unchangeable target state: differentiated
to be, to succumb to progressive mechanization, etc. The historical present indicates
Luhmann as entering this stage. But with this he puts the dogma ahead of history,
that the nature of the present (western!) societies in the bud already in the original
society and that the intervening story does not concern itself with anything else
was than gradually bringing about this state.
On the basis of the embryological model, Luhmann's classification of the
Story in three stages. He distinguishes between the prehistoric and archaic as successive ones
Stage that is structured by segmenting differentiation, an epoch that is characterized by the "trig-
ger ”principle is determined (by that subsystem“ which is characterized by its own complexity in the
Evolution was leading " 198 , and today's society, which is differentiated and not led by any
of the subsystem (or by the change in primacy).
195 See L. v. Bertalanffy, General System Theory, op. Cit., Pp. 67-70.
196 See ibid, p. 71
197 Cf. KH Tjaden, review on: J. Habermas / N. Luhmann, Theory of Society or Social Technology, op. Cit. Tjaden
writes: “Habermas tends to oppose Luhmann's theory - wrongly - the model of the organic
to subordinate itself to a hostile environment asserting a system, thus to deny the specific achievement of Luhmann
know the constitution, function and evolution of complexity-reducing systems on a deadly problematic
to relate to the world. "(p. 155)
198 N. Luhmann, Sociological Enlightenment, loc. Cit., P. 142.
Page 54
Camilla Warnke: The "abstract" society - 54
OCR text recognition Max Stirner Archive Leipzig - 11/27/2019
Luhmann admittedly does not relate the concept of history to embryological processes
conceals, by exaggerating the biological starting model in terms of system theory, evolution more social
Systems is measured by the changes that occur in terms of reproducing complexity
appear. From the point of view of complexity, the embryological model is thus
subjected to straction and purified from the concrete biological conceptions. Because this process
does not affect the structure of the underlying development scheme, it remains as an abstract empty
formulas which, to a certain extent, are used subsequently and externally
Structured content.
It is not without the comic that Luhmann of all people accuses Marxism, biologism
never to have overcome. He writes: “The terms production and production relations
with Marx ... are still under the presuppositions of the old European tradition; they relate
still focus on the satisfaction of the needs of organic and social life . ‚Le-
ben 'is and remains [104] however also as a good, social, culturally interpreted life
Category from the sphere of organic systems; it denotes those characteristic of this sphere
System / environmental processes. " 199
If Luhmann, on the other hand, takes the view that "the problem of evolution in sense systems
do not stem in changes in the 'reproduction of life' ... that is, in a reproduction of the
Productive forces and relations of production, but rather in changes in the 'reproduction of
Complexity '" if there is 200 , the blindness to one's own organism can only be explained.
if one takes class-related, ideological narrow-mindedness into account.
Once again we come across the fact that the bourgeoisie does not understand the dialectic
able. That means here that Luhmann is unable to understand that only by means of the concrete concept of
human-social mode of existence, the organism in the view of society over-
is windable. “Overcoming” does not mean to negate abstractly, does not mean such a thing
To put term in place of the old one, which is in no way conveyed by the old term.
Development is a process in which a concrete object of a given quality is transformed into a
transformed concrete object of a different quality. It is a unity of continuous and dis-
continuous moments. In order to make the specifics of a new quality visible,
probably the specific concrete identity as well as the specific concrete difference between the
old and new quality can be determined.
Luhmann blames Marxism as a mistake in not imposing the social mode of existence
the bleak, abstract, external difference to the animal mode of existence is reduced by the fact that he
Historical not banished from the logical. He tries to do that himself by following the
classical logical relationship of subordination biological and societal-human data
as special cases of the reduction of
Comprehends environmental complexity by replacing the concrete identity with the abstract one
Systems overriding property. It cannot fail that [105] Luhmann but-
once caught in the snares of an antinomy: the excluded concrete identity creeps
recovers itself as an abstract, superordinate, absolutized identity. The generally accepted
bloated embryonic development model eliminates the concrete differences between the (ontoge-
genetic) history of organisms and society.
And it also eliminates the hierarchy of interdependent relationships within society.
gen. Luhmann thinks logically from his assumptions when he considers the central meaning of
Can't see Marx's terms “productive forces” and “production relations”. Under the
only society-specific point of view that he accepts, namely that in society
Reduction of environmental complexity takes place through the creation of meaning, can production and reproduction
of productive forces and production relations only as one of equivalent possibilities
can be seen as one of the ways to reduce environmental complexity.
199 N. Luhmann, System Theoretical Argumentations, loc. Cit., P. 372.
200 Ibid, pp. 362-363.
Page 55
Camilla Warnke: The "abstract" society - 55
OCR text recognition Max Stirner Archive Leipzig - 11/27/2019
Marxism gains its social concepts on all levels of generalization (den
Concept of economic social formation, that of a specific social formation)
not by specifying the system-environment relationship, not even by the simple and external
the question of what distinguishes the biological from the social mode of existence or the
capitalism from socialism, etc., but by means of historical-logical analysis. The "Ka
pital "and Engels' study" Part of the work involved in the development of the ape "
taken the evidence that history is ultimately of the development of the productive forces
and production relations is advanced: as an overall process and within the individual
corporate formations. By doing the crucial part of the work on anthropogenesis, the leading
role of the production of tools within a system of qualitatively changing
The animal characteristics have been demonstrated by making the distinguishing characteristic
took what the objective, actual development from the animal to the human form of existence
primarily caused by the question of the general validity of the relationship between productive
forces and relations of production [106], so to speak, to their “elementary form”, one could
also be sure that the "range" of human-social existence in its basic
structure to have understood his Basic Law, but at the same time also the identities of human
cher and animal mode of existence, the fact that in modified, specifically social
Form ineradicable needs are satisfied, which correspond to the biological existence of humans.
come. The concrete concept of the social mode of existence preserves that of the biological
Form of movement of matter as its limit, as a suspended concept in itself.
The ideological meaning of Luhmann's concept of history is obvious. In defense and as a con
the Marxist concept of history is intended to give the course of history a kind of objective tendency.
tendency, however, a tendency that is independent of interests and needs.
ness of the socially acting individuals and classes permeated. Luhmann's inau-
gured concept of history is based on the idea that the relationships are independent of the
t work that the systems and not the people and classes are the actual actors of the
History is because it is not the social classes but the systems that relate to the subjects
on the formation of meaning are: “The concept of meaning is primary, that is, without reference to the concept of the subject
Define. ” 201 Luhmann's concept of history suggests an ideology of powerlessness,
of settling down, agreeing to be managed, opportunism. He is excellent
suitable to provide the theoretical hinterland for manipulative practices.
4. The dissolution of society into autonomous subsystems
Luhmann's social theory is - as already mentioned - a reactionary counter-enlightenment
tion constructed counter-draft to historical materialism. Logically, Luh-
man, all the questions raised by Marxist social theory by means of their own position
to answer, consequently his theory excludes implicit and explicit attempts at destruction
the Marxist [107] social theory, with Luhmann never attacking minor issues
opportunities, but always focused on the fundamental statements. And he tries to be the effect
Heightened a theory by turning Marxism backwards, with respect to the past
partly right, but not for today and for the future.
Luhmann accepts the statement of Marx's analysis of capitalism that the economic part
system in early capitalism (which Luhmann would certainly call differently!) that all other sub-systems
was dominant, but he firmly denies that Marx was thereby a general domestic
ner system legality discovered in relation to society. Has. By denying the latter,
he wins a whole chain of "refutations".
If the economic subsystem is not generally the leading one in the history of society
Is a subsystem, social development is not determined by the relationship between productive
and production relationships, then contradictions play out within these relationships.
If there is no historically decisive role, the class struggle is only a historically narrow one
201 N. Luhmann, Sense as a Basic Concept of Sociology, op. Cit., P. 28.
Page 56
Camilla Warnke: The "abstract" society - 56
OCR text recognition Max Stirner Archive Leipzig - 11/27/2019
Phenomenon (which only applies to the epoch that Marx describes in “Capital”) then exists
also no legal dominance of the economic base over the social superstructure,
then it is possible to claim that economics is not an area for human satisfaction
Needs.
And it is precisely on this claim that Luhmann depends. It is necessary indirectly to support the thesis of the
to support the functionally differentiated society of the present; straight to the Marxist
Humanism for which society exists for the purpose of satisfying and generating
increasingly humanized needs produced, to be replaced by the inhumane idea,
that what really matters is the needs and interests of the systems.
The Marxist (and, by the way, also early bourgeois) determination of the objective function of economic
Luhmann countered with the following suggestion: “Better approaches can be found in more recent
laydowns of economic theory, either the rational structure of action (the end / means
Scheme) or the money mechanism or the scarcity [108] problem as a starting point for a
Definition of economy to choose. ” 202
Of course, this proposal also reflects a part of the social reality of late capitalism.
mus, insofar as its economic subsystem is in fact not aimed at satisfying human
Needs, but is preoccupied with itself: with its "monetary mechanisms", the
Fact that “money is chronically scarce”, etc. 203
The concern that the production and reproduction of the productive forces and production
conditions actually have to serve the production and satisfaction of human needs
not be dismissed, even or especially when production or "economy" is subject to social
conditions that distort their purpose beyond recognition.
Luhmann builds his economic theory on what Marx “commodity fetishism” or also
"Money fetishism" calls when he claims: "... with the help of the money mechanism, the economic
own values, own purposes, norms, rationality criteria ... to which the behavior
orientate the elections in their area. ” 204
In that Luhmann Economy as a differentiated sub-system is indispensable to the money mechanism
he leads back, hypostatizes and asserts the general validity of economic relations
of capitalism, which it also only grasps on the level of their appearance. He gives the money f
schism brought about by the economic practice of capitalism, the theoretical
Consecration, it absolutizes the appearance that there is a "social relationship" in the economy
of things ”. The possibility of asking whether there is a "social-
ches relationship between people "can hide 205 , he cuts from the outset by the assertion
that the economic subsystem leads an independent existence. Luhmann's economic
theory is theoretically and methodologically based on the capitalist character of those
ignoring economic conditions that he describes. Marx's criticism of the
Goods and money fetishism not only in a general sense, but literally to: “But it is precisely this
form - the money form - of the world of goods, which [109] reflects the social character of the pri-
Father work and therefore the social conditions of private workers, objectively veiled instead
they reveal ” 206 and:“ It is only the specific social relationship of the people themselves,
which here for them assumes the phantasmagoric form of a relation of things. ” 207
Luhmann is realistic enough to admit that work in the economic sense is a commodity
202 N. Luhmann, Sociological Enlightenment, loc. Cit., P. 206; see p. 208: here Luhmann asserts, “that the economy
not an immanent logic of need, but the need follows an immanent logic of the economy ”.
203 See ibid., P. 207.
204 Ibid., P. 210.
205 See MEW, Vol. 23, 5.87.
206 Ibid, p. 90.
207 Ibid., P. 86.
Page 57
Camilla Warnke: The "abstract" society - 57
OCR text recognition Max Stirner Archive Leipzig - 11/27/2019
is treated. But from this admission - under the auspices of abstract society
economic theory - immediately a proposition that applies to all highly developed contemporary societies
claims validity: for capitalism as well as for socialism. Also allowed
the independence of the subsystems in differentiated societies claimed by Luhmann, the
To evaluate human exploitation as a purely economic phenomenon. Luhmann designs in
This context has a strange exclusivity relationship with which he defines the character of a commodity
wants to take its societal function from work: work can be done in the economic
can only be treated as a commodity because they are used in religious, political, family, educational
medical system is not treated in this way. 208
This assertion presupposes that what has to be proven first has already been proven. It would have to be proven
that the sub-systems that Luhmann lists actually function strictly autonomously and with a division of labor.
so that the properties of the subsystems cannot be transferred to one another. This proof
Luhmann remains guilty with good reason. Unless you accept the - outlined above
system-theoretical model - abstract assertion as evidence: social development
development just happens this way and not differently. The strict separation that Luhmann has between the
social subsystems, and the emphasis on their independence as differentiated
adorned systems thus allow exploitation to be seen as only a partial, not societal phenomenon
to mark nouns.
But it also allows the question of what is behind the monetary mechanism (behind the social
relationships of things) existing social relationships of people to
veil, since, according to Luhmann, in other [110] subsystems of society (e.g. the family
lie) can be realized.
In an economy understood in this way, “productive and“ production relations ”do not come about at all
before; not even the concept of production, insofar as it is socially determined production
is meant. Wherever these terms are used, they do not mean the same thing as in
Marxism. At Luhmann, production is increasing under the aspect of reducing complexity
a special case of technology, whereby technology also includes social technologies, e.g. B. manipulation includes
like making material objects. “The 'production' of objects is ... a
Application, admittedly a particularly spectacular application of technology. ” 209
Luhmann can understand facts such as the unity of productive forces and production relations,
Do not even think about production, etc. in terms of its theoretical and methodological specifications, because
whose concepts are based on the concrete concept of society and by means of dialectical me-
method: This is required in the abstract juxtaposition of productive forces
and production relations not to stand still. Only then do the terms find their content
Fulfillment when they are conveyed together, that is, when each of the two concepts is about
enriches its opposite, its counter-concept. The one about the determination of the relations of production
The enriched concept of the productive forces is only - to speak with Hegel - its "truth" because
only in this case the productive forces are determined as concrete by the relations of production
Productive forces are understood, and vice versa. Luhmann expressis verbis borders on the
Method of dialectical concept formation and its result, the concrete concept. He is himself
aware that he wants to practice social theory with abstract, general concepts and that
his system-theoretical approach cannot produce anything else.
Here is his programmatic explanation: "With universality it is only asserted that all facts
stand, in the case of sociology, allow all social facts to be interpreted in terms of system theory ...
The universality of theory does not mean that its objects total, that is, in all possible things
Grasp [111] things ... A consequently developed, functionalist system / environment theory ...
Totalizes systems only in one specific respect: in their relation to the world. Just the world itself
208 Cf. N. Luhmann, Sociological Enlightenment, loc. Cit., P. 211.
209 N. Luhmann, System Theoretical Argumentations, op. Cit., P. 359.
Page 58
Camilla Warnke: The "abstract" society - 58
OCR text recognition Max Stirner Archive Leipzig - 11/27/2019
no system is seen as a concrete totality ... systems themselves are always selective conductors
stungen, aspects of world conditions, the selectivity of which science with its analytical
tegories only traces, not justified ... totality is not asserted for systems, only
for what their selectivity comes from, systems relate to the total precisely because they do not
are , but reduce , and the same applies, on a different level, to the categories of sy-
stem theory ... The analytics of science therefore comes in handy, and that is for its conceptual
to take advantage of education that even factual systems of the physical, organic, or psychological
social reality are not concrete totalities. These clarifications should provide access to the
Facilitate thesis that science (and within its framework also: systems theory) become reflexive,
that means being able to see oneself as a system; what is meant is not a total comprehension of dead
rather a selective assessment of selectivities. Especially with a system-theoretical analysis
In the analysis of systems theory, their characteristic weaknesses emerge in a profiled manner: ... especially you
Frame of reference relativism. " 210
When Luhmann asserts that systems are not concrete totalities, he is undoubtedly right,
since systems are not things, but merely “perspectives” of things. Luhmann also has
right if he considers the limitations of systems-theoretical analysis, among other things, in its reference system.
stemrelativism sees. And he is finally and finally right when he understands this frame of reference relation-
vism claims this abstract universality for its own systems-environment theory.
But it is no longer legitimate to believe that by means of this perspectivism (or by means of a
rality of perspectives) the whole process of cognition can be achieved in relation to that whose
Perspectives are to be determined.
Luhmann holds the just cited insights into the limitations of the systems science possibilities
opportunities for social theory. Because Luhmann systems, even if only with regard to
He [112] consequently “totalizes” and “totalizes” them in relation to the world in general. Under the-
hand he transforms “perspectives” into essence, aspects of society into their essence. Otherwise
would not understand why he used the social concept of Marxism, which applied to totality
is so fiercely opposed (instead of enriching it in relation to certain of its statements) why
he liquidates the concept of the thing and with it the concept of wholeness, why he uses the “factual
Systems “of physical, organic, psychological or social reality - those for things
should stand - denies being totalities. And when Luhmann finally claims that the
present society is a functionally differentiated system in which the sub-systems
function independently of each other, then that is more than the specification of one of their "perspectives",
then that is a determination with regard to the nature of today's society, that of von Luhmann
is not relativized or restricted at any point in his writings. Why should he? In the end
if the “totalization has already taken place”, after all, an “ontology” already exists: that of the radika-
lized functionalism.
That old Indian fable, in that of five blind men, goes well with Luhmann's system of reference relativism
who wanted to find out what an elephant was. The first got his trunk closed
the second his tail, the third his tusk, the fourth his leg, the fifth his
Belly. Then they quarreled. The first claimed that an elephant was a big, fat snake, the
the second, however, that the elephant is a kind of thin rope with a tassel. The third was again
willing to swear that an elephant can only be a particularly fine kind of ebony. For the
fourth, the animal was clearly columnar in shape. And the fifth firmly represented the supreme
evidence that it is a matter of the sagging ceiling of a hut.
For Luhmann, too, only the snakes, ropes with tassels, etc. would be of importance in this dispute.
He would either not ask the question about the elephant at all, or he would act like one
of the five blind. If he asked what goes beyond “perspectives”, so
he would have to come to the conclusion that the concept of system, because of its abstract universality, with
the concrete, philosophically mediated concept of thing must be conveyed.
210 Ibid., Pp. 378 f., 384 f.
Page 59
Camilla Warnke: The "abstract" society - 59
OCR text recognition Max Stirner Archive Leipzig - 11/27/2019
[113] For Luhmann, society is not a "thing", not a concrete totality; it is just a system
(System of systems). The function of social systems is to reduce complexity and totality
to reduce the environment, with regard to this task on the basis of the selection type
are strictly specialized.
For example, Luhmann differentiates between the social systems economy, science, family and
Politics based on appropriate mechanisms (communication media): Economy on the
Money Mechanism, Science on Mechanism. Truth, Family on Mechanism
Love and politics on working power. In contrast to truth and love, the se-
Lessons of equal experience, the peculiarity of money as well as power lies in the fact that they
Selection of actions arranges. 211 In modern society, these social systems are
referenced, that is, independent and autonomous; with their own values, norms, demands and de-
winding directions equipped. They each have more possibilities than the overall system
stem allowed to develop. So that the social systems do not overstretch their claims, theirs must
Opportunities can still be processed selectively. "The unity of society is, at least today ...
ultimately nothing more than the adjustment of a ratio of corresponding complexity between
between a multitude of social systems that are mutually beneficial for each other's social environment
are. ” 212 A society understood in this way, consisting of a multitude of autonomous, independent sub-systems
stems exists, cannot be understood as a whole, as a "thing" and thus as a concrete totality.
The sub-systems in this model have no connection to one another. You determine
not reciprocally within the whole, but rather find a common point of reference .
Outside the respective system, in the environment, in relation to the social systems, just as many do not
mutually mediated perspectives, each with its own reduction processes of environmental com-
represent complexity. In this view, the subsystems do not determine each other, but rather
- if they interact at all - they limit one another, they only restrict one another
alternately one.
[114] Luhmann not only does not pose the question of the elephant - to stay in the picture above
in relation to the overall social system, but also not in relation to the subsystems.
What Luhmann calls “economy”, for example, is not identical with what the Marxist one
political economy referred to as the “economic subsystem of society”. This can
do not reduce to the monetary mechanism, even if money is the communication medium of the time
co-operative economics is. To base the economy on the monetary mechanism means, namely, it
can be traced back to the secondary, derived, the exchange process, each of which is specific
However, to eliminate character-determining mode of production from the economy. With the statement,
that economy is based on the money mechanism, Luhmann brings the property relations into
move to the means of production to disappear from the economy; the opposition of capital and
Work is "canceled". “Economy” in Luhmann's sense is in relation to the concept of
economic subsystem an abstraction, the mere "perspective" of a whole hierarchically
ordered relationships (of a "thing").
The same applies to the other subsystems mentioned by Luhmann, which are also under the
single point of view of the communication media on which they are based. the
The contexts declared by Luhmann as subsystems only capture their subject matter in the
Sphere of appearance. The money mechanism is not - a favorite word of bourgeois society
to use ziologists - "questioned" about the underlying essence, its fetish character
is not revealed.
At the level of appearance, in relation to the so-called communication media truth,
Love and Power, indeed it looks as if the social systems are science, family and
Politics neither with each other nor with the social system based on the money mechanism
Economy as if the subsystems are based on independent values,
211 Cf. N. Luhmann, Sociological Enlightenment, loc. Cit., P. 213.
212 Ibid, p. 149 f.
Page 60
Camilla Warnke: The "abstract" society - 60
OCR text recognition Max Stirner Archive Leipzig - 11/27/2019
Norms etc. functioned as if they followed autonomous directions of development. The extreme my-
stification and obfuscation that express the essence of civil society in its manifestations
experiences, still makes [115] judgments possible like this: that love and truth cannot be bought
be, or - to quote Luhmann's variant on this: “Wealth does not convey
without further ado political power, love, truth ... education, although the statistics have certain corrections
lations in the sense of higher average participation rates. ” 213
To this one would have to reply with Marx and Engels: It is not at all to speak of "the" family,
but from the concrete bourgeois, proletarian, etc. family, in relation to the established
can be that their character is by no means independent of the economic conditions. So
gives the bourgeoisie “the family the character of the bourgeois family, in which the boredom and
money is the binding factor and to which also the civil dissolution of the family belongs, in which
the family itself always continues. Their filthy existence corresponds to the sacred term in offi-
cial idioms and general hypocrisy ”. 214 But this remark is not all
laid bare the economic nature of the bourgeois family, but also the way
how this expresses itself, its appearance, which is hypocrisy. As a conservative ideologist of the bourgeoisie
Luhmann takes - in agreement with the prevailing conditions - the illusionary appearance for
reality, apparent autonomy for real.
The wholeness or totality character of societies is of course not at the level of appearance
to investigate. At best, like Luhmann, one arrives at registering “correlations”.
In order to track down the objective relationships of determination hidden in the correlations,
In the dialectical sense, concrete, profound economic analysis is allowed to do so, with their help
the inner essential connection of the forms independent as appearances becomes visible.
The economic analysis of society, however, led Marx to the following realization: “In all areas
social forms, it is a certain production, the rest of them, and therefore their relationships
also instructs all others, rank and influence. It is a general lighting in which all the rest
Colors are immersed and (which) modifies them in their peculiarity. It is a special ether that
the specific gravity determines all existence protruding from it. ” 215
[116] Concretely and related to capitalism: That in the contradiction of wage labor and capital
dominant capital (this particular production) modifies, shapes and transforms all found
to which social conditions, processes and sub-systems according to its own nature
so that the social systems of politics, science and the family mentioned by Luhmann also
cifically adopt capitalist characteristics and thus appear as institutions that sustain the existence
secure the system, stabilize its quality.
So where Luhmann believes he can establish autonomy, independence is not there
to find. The independence of relationships, processes and sub-systems of the capitalist
Society takes place in completely different dimensions than those assumed by Luhmann.
The contradiction between wage labor and capital, that is, the fact that the working conditions
are separated from the laborers and become independent in relation to labor in capital, 216 the
Fact that the conditions of production and products become an independent reified power
towards the producers, 217 that is, who in capitalism are full on the level of its being.
Luhmann does not take note of the pulling independence processes. Not even the consequences.
One does not find a word in him that the state is seen as one in relation to the non-possessing classes
"Foreign, external violence" and "separate from the real individual and collective interests
sen "exists, 218 not a word that there is such a thing as" objectification of the relations of production "
213 Ibid, p. 211.
214 MEW, Vol. 3, p. 164.
215 MEW, Vol. 13, 5, 637.
216 See MEW, Vol. 25, p. 879.
217 See ibid, p. 832.
218 MEW, Vol. 3, pp. 33, 34.
Page 61
Camilla Warnke: The "abstract" society - 61
OCR text recognition Max Stirner Archive Leipzig - 11/27/2019
or "reification of social relations", 219 and not a word that the ar-
working masses in the society he reflects on from an autonomous one towards them
Machinery to be ruled by institutions.
Since Luhmann identifies himself with the ruling class of the capitalist system, since he -
like the vulgar economists criticized by Marx - "in these alienated and irrational forms ...
feels completely at home ”, 220 these independence due to class antagonisms are for
it is not an object of reflection, but is accepted without question. Luhmann's subsystems are
Defined as class indifferent. They are based on the illusory notion of [117] the existence of one
The general interest of all members of society in the survival and smooth functioning of the
civil society. This goal function of society is clear for Luhmann, it becomes in
in no way problematized his remarks on social theory. Has been problematized
in a preliminary agreement only the concept of system in general, whereby the reference to system
men in the world system received a single uniform objective function to reduce world complexity.
Luhmann's concern is exclusively the optimization of the interaction of the subsystems, their pro-
portability in terms of their complexity. In this context, Luhmann is aware of the following
deeply worried about the tendencies emerging in late capitalism:
Increasing complexity and intricacy of the subsystems under the control of the ruling class
Class threatens to slip away, and the system resolution that goes hand in hand with this process
send, disintegrative tendencies.
And here at last the hidden and rational meaning of Luhmann's assertion that the
today's society is not a qualitatively determined whole that it differentiates into sub-systems.
it is adorable that she - to refer once more to that of L. v. Bertalanffy borrowed the development model
to come back - in the course of progressive mechanization, the character of summativity was assumed.
men have.
Behind this concern is the knowledge that the machine of rule functions and works based on division of labor
must function so that the ruled classes in relation to all areas of life on systemic
conform behavior, and the insight into the threats to the system, if
the sub-systems of the machine of domination are independent, deviating from the character of the system
Make developments.
And in fact it is a characteristic of descending stages of development that in them as a rule
disintegrative processes outweigh the integrative processes and that movement and development in them
of the elements and sub-systems follows the in-house development rather than the system requirements that
are no longer able to assert themselves. 221 These disintegrative tendencies develop all the more
It doesn't matter the more ideological and institutional effort the ruling class drives
must and drive in order to suppress the aggravating and intensifying main antagonisms,
to repress, redirect and fragment the crises and conflicts that break out of them,
to petrify, etc., since this effort has an almost unmanageable degree of complexity in terms of
social problems, the integration of which is more and more difficult to cope with.
To this extent, Luhmann's development model of society confirms it, albeit in a more mystified manner
Form that reflects the real state of the overall social system of late capitalism -
unwanted the Marxist realization that imperialism is the last stage of development
of capitalism is.
5. "Negation of negation" in the service of system stabilization
The increase in complexity and tendency towards disintegration, observed with concern by Luhmann
is a consequence of the acute economic that dominates the capitalist system
Basic contradiction. Luhmann's social theory is not about problems of complexity in
219 MEW, vol. 25, p. 838.
220 Ibid.
221 Cf. G. Pawelzig, Dialectics of the Development of Objective Systems, Berlin 1970, p. 144.
Page 62
Camilla Warnke: The "abstract" society - 62
OCR text recognition Max Stirner Archive Leipzig - 11/27/2019
with reference to any social parameter, but to those of the social systems. Otherwise
he would hardly have combined complexity and disintegration. As the social systems however
are grasped not in relation to their essence, but only in relation to certain of their aspects, remain
Luhmann's suggestions on how their complexity can be made manageable on the manipulation
lation of sequelae is limited. With little system-stabilizing effect, since every solution
of a complexity problem which is likely to result in a further increase in complexity.
The only possible real way out of the oversized, uncontrollable complexity of the late night
capitalism can only be radical in nature. Only when the evil that is causing the com-
the basic economic contradiction that causes complexity and disintegration is resolved if socially
tables take the place of the capitalist relations of production [119] also disappear
those follow-up problems that make Luhmann so difficult. The socialist revolution leads
namely - because and by the contradiction between social production and private
aptitude overrides - for simplicity in relation to the class structure of society.
Such a process also took place during the transition from the feudal to the capitalist
talist society. The development culminating in the bourgeois revolution simplified
through polarization the complexity of the feudal class relations to the class antagonism
Bourgeoisie and proletariat and created in this way, among other things, the integration basis for the
Productive forces developing in a qualitatively new way and with a leaps and bounds in intensity
(first industrial revolution) and its consequences. It was spoken - cybernetically
- the degree of complexity of the system with regard to its class structure is reduced by
Complexity turned into increased complexity.
Here is a note on Luhmann's concept of complexity. This one is pretty unclear and
blurred. Complexity "should, as a first approximation, include the totality of possible events
nit understood " 222 and in terms of the behavior of systems:" systems are complex,
if they can assume more than one state. ” 223 The definitions presented are at issue
is obviously an attempt to use a concept of complexity at the highest level of abstraction
form, which necessarily has the consequence that this is almost empty of content.
This is what Luhmann does, if I have understood him correctly, between complexity and intricacy
no difference, but uses the term “complexity” for both facts as well
for their common occurrence, so that he does not enter the dialectic of complexity and intricacy
gets the look.
Since Luhmann's unreflective identification of these terms has ideological consequences,
their difference must be emphasized. A system is complex by the type and number of between
relations existing between the elements; it is complicated by the number of its different ones
Elements [120] elements. Complexity is therefore always linked to complexity, but not necessarily linked to it
Complexity, namely not when the system consists of universal elements. 224 off
this relationship follows: the complexity of a system can grow without its complications
ness increases; yes, the complexity of a system can increase and decrease at the same time
its complexity. But the complexity of a system cannot reach a higher degree
without its complexity also increasing. Simplification of systems can therefore be achieved through the
Transformation of intricacy into complexity can be achieved, the process undertaken in the process
Reducing the degree of complexity to increase the current (and possible) complexity
degree of efficiency.
Without having these terms available, Marx has to do with social developments
this relationship of complexity - complexity - simplification of the matter seen when
he speaks of the above-mentioned polarization of civil society in two basic classes,
or when he notices that the objective “abstract” value-creating work and its concept are only just emerging
222 N. Luhmann, Sociological Enlightenment, loc. Cit., P. 115.
223 Ibid, p. 116; see N. Luhmann, Systems Theoretical Argumentations, loc. cit., p. 312.
224 Dictionary of Cybernetics, loc. Cit., Pp. 307, 308.
Page 63
Camilla Warnke: The "abstract" society - 63
OCR text recognition Max Stirner Archive Leipzig - 11/27/2019
can, if a developed totality (i.e. complexity) of really different kinds
ways of doing things exist that lose the quality of being complicated when they become part of the social
reality can be reduced to a means of creating wealth in the first place. As abstract values
End work becomes work as a universal element within a complex system. In relation
The transition from complexity to complexity has the advantage over the process mentioned
that the system of labor is now open to new growth, insofar as a social
state arises “in which the individuals easily pass from one work to the other and
the particular kind of work they do accidentally and is therefore indifferent ”. 225
Likewise, the socialist revolution leads - as already indicated - on the path of transformation
from intricacy to complexity to simplification of certain basic social
Structures. The solution to the fundamental contradiction between social production and private supply
aptitude which, through a transitional epoch, mediates the [121] abolition of the classes in the wake
transforms individuals into
similar, universal elements and the relevant system into a complex result of the
change is the opening of a new range of measurement for the growth of the productive forces and a
leaps and bounds for the individuals to develop, due to the multiplication of the species
and number of their possible social relationships is guaranteed.
These fragmentary remarks must suffice to make it clear that Luhmanns
Scheme related to the constant increase in social intricacy and complexity
on all social systems with a simultaneous increase in disintegrative tendencies the complicated
The dialectic of the development of social systems is a failure. But also the relationship outlined above
of complexity - complexity - simplification does not claim to be more than on those
to draw attention to a special case of system development that Luhmann stubbornly overlooks.
In order not to become one-sided myself, I would like to add that the relationship between complexity
- Complexity - simplification was dealt with abstractly, i.e. without specifying the conditions,
under which this relationship holds, regardless of the fact that complexity reduction
with regard to certain system parameters, as a rule, increase in the degree of complexity with regard to
on other parameters causes simplifications to be followed by new differentiations, etc.
The roots of Luhmann's arbitrary restriction to one possibility of system development
ment are ideological in nature. Luhmann's construction is particularly well suited to
theoretically blocking a tional way out of uncontrollable social complexity,
because it only allows solutions in which the economic foundations of the late bourgeoisie
Society that produce the registered complexity, not touched, radical simplification
can be withdrawn. With Luhmann, complexity becomes a fact and a fetish, one of the last
unchangeable social reality with which one can come to terms and on whose permanent one
Increase one has to adjust [122]. Likewise with the constantly growing disintegration and
The social subsystems become independent, which, according to Luhmann, evidently emerge as the
the tendency towards integration prevailing, dominant tendency into the future of society
continued ad infinitum.
But “simplifications” are also provided for in Luhmann's social model. The discovery
"Complexity-reducing processes", from "Mechanisms of simplification and relief
stung " 226 is even one of the general concerns of Luhmann's social theory,
digerweise. It is the most difficult task to be mastered, one with great complexity
to cope without affecting the system, i.e. the fundamental quality of society,
is changed.
Luhmann sees the most important mechanisms that reduce social complexity in “re-
flexive mechanisms ” 227 as well as in single and double negation. Though philosophical
225 Karl Marx, Grundrisse der Critique of Political Economy, op. Cit., P. 25. [MEW Vol. 42, p. 38]
226 N. Luhmann, Sociological Enlightenment, loc. Cit., P. 104.
227 See ibid.
Page 64
Camilla Warnke: The "abstract" society - 64
OCR text recognition Max Stirner Archive Leipzig - 11/27/2019
It would be extremely worthwhile to discuss Luhmann's concept of reflexive mechanisms
here a restriction to the dialectical philosophy borrowed terms "negation" and "ne-
gation of negation ", since it is precisely on these that it can be demonstrated what is involved with Luhmann's recipe
tion of dialectic is about.
Luhmann introduces the terms “negation” and “negation of negation”, but uses them in
a meaning that is in affront to its Marxist sense - its content precisely that
Deprived of moments by which they are suitable, social leaps in quality, historical-revolutionary
to record tional developments.
For Luhmann, "negation" is one of experience and action, the meaningful social systems
peculiar activity by means of which complexity is preserved and reduced at the same time. Reduced in-
if, as a symbolic system from a field of possibility, from a multitude of alternatives, one
certain possibility (perspective), choose alternative, while all other possibilities
(Alternatives) are negated, that is, cannot be realized. At the same time, for Luhmann "Nega-
tion "but also preservation of complexity, insofar as the alternative solution options not chosen
opportunities still exist. This means that the field of possibility [123] remains unchanged against
about realizations. Luhmann expresses this as follows: “Experience and action are incessant
selection, but must not eradicate the unselected alternatives and make them disappear
bring ..., but may only neutralize them. Complexity must therefore not, as it is in the computer
jargon means ... to be 'destroyed', but is just as it were excluded from moment to moment
Moment is reduced in different ways and remains preserved as a generally constituted se-
lesson area, as 'what from' always new and different choices - as world. ” 228
But if “negation” is nothing more than the selection of an alternative and the neutralization of other al-
alternatives that remain available, the “negation of negation” can consist of nothing more than
that the choice made is reversed, so to speak, another alternative to the place
the previous one: "I determined my yes and leave the necessary negations un-
true ... I reserve the right to negate such negations if necessary. " 229
According to this view, the terms "negation" and "negation of negation" include revoked-
bare, interchangeable, reversible processes. They no longer remain what they came from
according to are: categories of historical thought, “abstract, logical, speculative expression for the
Movement of History ” 230 , as Marx said in relation to the Hegelian concept of the negation of ne-
gation noticed.
Of reversible social processes (more precisely: of processes in which the property
reversibility that dominates irreversibility) can only be discussed wherever there is
courses are up for debate that take place within the framework of one and the same quality. But also in
In relation to this, the reversibility is never absolute. Absolute reversibility would be assumed
stipulate that, within the respective quality, only the identical reproduction of relationships takes place.
comes what is only possible under constant conditions. Only with identical reproduction
of relationships, however, negation and double negation are reversible processes, with negation
the negation in this connection is an expression poor in determinations, since it is nothing more
indicates that a cycle has taken place, a movement which in its qualitative and quantitative
has returned relatively unchanged [124] the starting point. According to Marx, the cycle would be a commodity
- Money - commodity (C - M - C) such a double negation process.
Luhmann is indeed interested in the qualitatively identical reproduction of the contemporary imperial
cunning society. Negation and negation of negation are therefore considered to be reversible processes
viewed.
Marx is quite different. For this, the identical reproduction was always only a borderline and special case
extended reproduction of social conditions, which in turn is the leap in quality that
228 N. Luhmann, Sense as a Basic Concept of Sociology, op. Cit., Pp. 33-34.
229 Ibid, p. 36.
230 MEW, Erg. 1, p. 570; see J. Zelený, Die Wissenschaftslogik bei Marx and “Das Kapital”, Berlin 1968, pp. 187-197.
Page 65
Camilla Warnke: The "abstract" society - 65
OCR text recognition Max Stirner Archive Leipzig - 11/27/2019
revolutionary upheaval prepares and merges into it. Thieves' view closes knowledge
a) that social formations go through a directed movement that despite and under
conclusion of reversible sub-processes is totally irreversible, with increasing progression
the degree of irreversibility increases from the initial state. 231 Already the process of the extended
Reproduction is no longer easily reversible, since the double negation leads to a return
leads to the starting point, but only apparently , since it is on a higher step ladder. This is how to behave
do not play the actual production and reproduction of capital according to the money-commodity cycle
- money, but according to the cycle money - commodity - money ' ; b) the direction of movement of the company
formations does not primarily result from their interaction with the natural and social
the environment, but rather from the inherent laws peculiar to them. The in-house
Ultimately, environmental influences dominate.
The general "logical" figure that demonstrates this independent direction of social development
expression, for Marx the negation of the negation, which describes a movement, is
which, figuratively speaking, does not run linearly, but rather as a spiral consisting of circles.
To be sure, the classics of Marxism-Leninism, in contrast to Hegel, have the negation
of negation is neither misused for arbitrary constructions nor a "dialectic of historical
in comfort ”, in which the entire content of the story is dragged along seamlessly
is, as Althusser aptly remarked. 232 Only after Marx worked on the concrete material of the economy
had shown that, for example, the simple reproduction of the commodity is circular
The process is that under capitalism money “goes through a series of processes in which it becomes
receives, proceeds from itself, returns to itself on a larger scale ” 233 , he illustrated the
extended reproduction of capital by means of the figure of the “spiral”. 234
The connection between the independent legality of the development of the social
formations and the process of negation and the negation of negation had to be emphasized,
because here is the point at which Luhmann repurposed the terms. There are none for Luhmann
Self-development of social systems or no dialectic of self-development and development
to be dealt with, but ultimately only the
averaged adaptation to the environment. In the relationship between social systems and the environment, the
Environment as a benchmark.
If Luhmann had meant to say that late capitalism was in the dispute with the
World systems is increasingly fixed on the behavior of adaptation - and doubtful-
off his theory is the most consistent reflex of this situation so far - 'he would be right. But since he is in
his theory gradually and consequently eliminated the economic basis of capitalism and in
transforms the social system of the economy, since it declares it to be autonomous and unconnected to a
A series of equal social systems, each of which is specific based on the formation of meaning
Have references to the environment, as it ultimately expressly dissolves society as a whole - with it
their independent regularities - 'he creates a theory that adapts to the environment
declared to be the behavior of social systems par excellence and under all conditions.
The if ... so assumption of the “open” model of decision theory: if adaptation to the
Environment prevails, then the internal decision-making unit becomes a problem, then development means
divorce, especially internal ones. Complexity reduction in constant engagement with
of the environment, which only applies when the condition of an adaptive system is met
in Luhmann's [126] general as-is statement that should apply to any society.
Luhmann receives the terms "negation" and "negation of negation" on the assumption that
that society shows no self-development and that in the relationship between society and
Environment the environment is the reference variable, furthermore provided that the system works with everything
231 Cf. G. Pawelzig, Dialectics of the Development of Objective Systems, loc. Cit., P. 88 ff.
232 L. Althusser, For Marx, Frankfurt (Main) 1968, p. 84.
233 MEW Vol. 26.3, p. 134.
234 Cf. MEW Vol. 23, p. 656.
Page 66
Camilla Warnke: The "abstract" society - 66
OCR text recognition Max Stirner Archive Leipzig - 11/27/2019
Change in terms of its fundamental quality (its existence) in dealing with
the environment preserves its identity.
Under the condition, which is incompatible with the Marxist development concept, that
social change through adaptation to the dominant environment forced development
Luhmann's concepts of negation and negation of negation include the following
Determinations a: negation and negation of negation are due to the change in the environment
forced operations. They get their respective direction from the accidental in relation to the system.
changes in the environment. The real social negativity follows as reactive processes.
NEN (changes) not an independent, societal law, but them
are - in relation to the system in which they occur - arbitrary. To stay in the picture: During
Marx and Lenin illustrate the negation of negation by means of a spiral of circles
were looking for, the figure of a linear, unregulated
teten zigzag movement apply, with which it has lost its characteristic features: an in
to be a closed and directed process.
What remains is the commonality of word usage and the presumably unacknowledged intention
Luhmanns, the exchange of variants of the imperialist political strategy in adaptation
to the internal and external system disputes as development processes
give.
It should of course not be denied that for the purpose of stabilizing the imperialist
System a plurality of interchangeable behaviors is required and established,
just as little as the fact that - generally speaking - within the framework of the identical reproduction
tion of systems in a changing environment, different strategies [127] functionally equi
valent in terms of their adaptive value. But even in this case
history cannot be entirely abstracted, as a strategy is implemented with respect to changes
the field of possibility so that the following choice of strategy is not entirely independent of
the preceding can be done.
So can - even if every future imperialist strategy is of course within the framework of preservation
imperialist interests - the current Ostpolitik of the SPD / FDP government in
of the FRG cannot simply be exchanged in the Luhmannian sense, since both the
inner social balance of power as well as that between the social systems
changes and determines the choice of future political strategies. Their field of possibility
but - if the balance of power continues to favor the socialist world and the revolutionary
shifts lutionary forces - one day exhausted for new variants of imperialist strategy
so that the imperialist system will perish.
Luhmann did not envisage this revolutionary negation. He therefore moves in a radius of
Concepts such as "reversibility of decisions", "unchangeable field of possibility", "preservation
management of complexity by reducing it "and" equivalence of functions ". Quite consequently
right by the way. Because as Luhmann for the conservative Para-University "Science Center Ber-
lin GmbH ”co-designed the“ Institute for Management and Administration ” 235 , he answered in
praxi with regard to his own theory the question that he repeatedly posed: what achieves
the systems theory for society? Knowledge of power to stabilize the state monopoly
capitalism and a new ideology of the status quo.
235 Cf. C. Grossner, Fall of Philosophy, Politics of German Philosophers, Hamburg 1971, p. 296.

