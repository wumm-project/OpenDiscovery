\documentclass[11pt,a4paper]{article}
\usepackage{a4wide,url}
\usepackage[german]{babel}
\usepackage[utf8]{inputenc}

\parindent0cm
\parskip3pt

\title{Beware, Contradiction Matrix!} 
\author{Leonid Shub}
\date{Peissenberg, 20 October 2006}
\begin{document}
\maketitle

\begin{quote}
  The German version was published in an anthology of the Hanserverlag in
  2007. The recording of the German version and translation to English by
  Hans-Gert Gräbe, University Leipzig. No longer functioning links were
  largely removed.
\end{quote}

In 1965--71 it emerged as a fantasy product and ``died out'' in 1975 --
Altshuller's Matrix together with the forty innovation principles.  Nowadays
it experiences an artificial renaissance in Europe. More than 90\% of the TRIZ
user prefers this ``tool'', which does not actually exist. But without support
from producers of modern ``Invention machines'' this comeback would be hardly
conceivable.

Not many questions were posed after my talk on ``Predeveloping innovations''
at the 4th European TRIZ Congress (Frankfurt am Main, 30.06.2005). Two of that
sounded similar: ``When working on a project, you analyze the problem,
formulate the technical contradiction, choose a solution method after the
Matrix, and then ...?''

After my remark that the Matrix is not a solution tool, and I therefore I do
not use it, and that Altshuller announced its loss of value as early as 1975
and attested 20 years before that it had only a historical significance, it
became quiet around me.  I sensed that this response shocked many congress
participants. No further questions.  Obviously nothing could be settled with a
TRIZ expert who did not work with the Matrix.

This was reason enough to intensify my interest in the phenomenon of the
``process table to overcome technical contradictions''. And at the same time
to explain for myself the meaning of their existence once and for all. Maybe I
missed some groundbreaking developments this question in the 20 years since my
last TRIZ course?

\section*{The most important tool}
In another lecture [2] at the same congress was given a comparative analysis
of frequency of use and effectiveness of the most important TRIZ tools
according to the results in numerous projects. These projects have been
performed by representatives of a TRIZ consulting in the years 1998 to 2005 in
thirty-eight industrial enterprises.

According to the presentation, the contradiction table together with the
principles of innovation reaps the unequal limited sympathy of the clients.
It was used in 96\% of all cases and is thus the clear leader among the TRIZ
applications by project consultants in the solution process. In doing so, it
significantly overrides the functional and contradiction analysis (80\%
frequency of use) and seems to be unattainable for ``second-rate'' tools such
as the separation principles (36\%) or ``standard solutions and
substance-field analysis'' (12\%). The old Matrix actually made it to
outperform the last five listed tools together (including forecasting and AFE
-- Anticipating Error Analysis) in their frequency of use to a large extent.

Almost half (47\%) of the Matrix applications by the company specialists led
-- according to the diagram presented -- to ``strong'' solutions. The total
number of Matrix solutions exceeds five times the results of the separation
principles. The successes of the substance field analysis appears rather
ridiculous on this background.

According to their representatives, this company is a consulting company and
software developer in the areas of TRIZ and innovations and leading in the
German-speaking market. Logically one assumes that it makes every effort to
popularize the method in different industries and thereby to expand their
stock of tools.

A comparison of the information mentioned in the presentation with those from
2004 [4] clearly shows that the demand for almost all shown in the diagram
``classical'' TRIZ tools dropped by 30\% to 50\% in just one year.  After all
-- except the Matrix.

Particularly explosive seems the fact that the separation principles still a
year ago had so much demand (63\%) from the clients that one wished the
current indication (36\%) would rather be a typo. And only the application of
home-made ARIZ was in 2004 even more sporadic in nature than today and counted
at 16\%.

Only at first glance the negative dynamics of the frequency of use of
different TRIZ tool in practice seems to be absurd. Do you know the most
important functional principles of common invention software programs, then
this dynamic is quite understandable.  The basis of these programs are again
the ``pioneering nothing-solver'' of TRIZ: The contradiction Matrix and the 40
innovation principles [3], [23].

Thus, the displacement of intelligent and effective tools, which, however,
require a longer time to be acquainted (and already for that reason are less
attractive), through the ``universal'' Matrix attain true industrial
magnitude. This tendency gets increasingly clearer contours, both among the
individual consultants as well as the users of TRIZ in companies. And their
avalanche-like acceleration infects the neighbors on the consultants market
literally with their eagerness.

\section*{Tell me, Matrix on the wall ...}
Another lecture at the congress was dedicated to dealing with the ``top
tools'' [22]. The core of a ``problem solving with TRIZ'', which leads to
revolutionary innovations, is in a consequent overcoming of the discovered
conflicts using the Matrix. In the shortly afterwards appearing media report
the secret of success of the mechanical engineering company was revealed:
Their research director has only ``\emph{the 40 principles always before eyes.
  In his office ... they hang on the wall, right next to a table that helps
  him to find out which rule fits your particular dilemma}'' [5].

The Wikipedia (the web-based free encyclopedia) dedicates to the ``success
potential of the table'' 90\% of a TRIZ article [24]. It is added as a
precaution that ``\emph{on average, the contradiction table can be
  successfully used in less than 10\% of all tasks.  It is therefore
  recommended a direct time-saving application of inventive principles in the
  order of their statistical frequency of use}''. The reader gets, however,
not rid of the feeling, that the table is also 90\% of TRIZ.

This kind of ``unobtrusive advertising'' appears more and more often in the
technical and popular scientific press as well as in the important information
magazines. The European still know that another not less important table (the
Periodic Table of the elements, originally proposed by the Russian chemist
D. Mendeleev) came from the frosty east and that it also had free cells. This
argument convinces many skeptics.

\section*{TRIZ Heraldry}
So what is that for an universal tool of the inventor, that found in 1998 to
2005 an absolute preference of 38 industrial companies? A powerful tool of
inventive thinking -- a monument to the genius Altshuller?  A fascinating
kaleidoscope that generates with every turn new exciting patterns?  A funny
toy of the project makers, that enables them to generate both crazy and
insanely realistic ideas?

The more accurate German version of Altshuller's table published in the
Internet in its ``classical'' form [6] is copied by many European supporters
and enthusiasts of this TRIZ section. The stylized Matrix served also as
picture on the cover page of one of the first German TRIZ/TIPS issues thus
becoming the TRIZ coat of arms [7].

This is exactly what many European consultants sell as the ``almost complete''
TRIZ, or at least as their basic, exact and error-free universal mechanism. At
least statistically, their claims can be well justified. In the constantly
increasing number of people of different professions the ``Russian'' method of
invention is associated with the Matrix.

Would you do a survey, and ask all Europeans, who know something about TRIZ,
what they imagine about that, you would get in 99\% of all cases this standard
answer: the Matrix and 40 general rules of innovation. Compared with this the
picture of a glowing Lenin pear, often used in Russia as a symbol for creative
and innovative thinking, would trigger rather painful for a European reader
associations.

Rodin's ``Thinker'', who had given his last shirt for an idea, impresses
neither.  The theses of Altshuller's \v{Z}STL (Life Strategy of a Creative
Personality) are not mentioned at the introductory and pilot seminars -- this
materials can not be adequately translated into any European language.
Compared with that the Matrix is understandable without translation.

The freedom to formulate numerous contradictions, and thereby to obtain the
most strange conflicting pairs, indeed generates the impression of a
recognizable, albeit superficial idea generation. This phenomenon presents the
Matrix application as a ``philosophically breathed'' and somewhat systematized
mishmash between Variant Choice and Brainstorming, making it at the same time
a big target for independent and technically experienced critics.

\section*{``Matrixed'' Inventing}
The once-extinct Matrix opened a new era in development and dissemination of
TRIZ in the West. The method it represented began to mutate into a mass
article of the invention branch. A profound and thoughtful study of
philosophical theses became no longer compulsory and as a rule barely
possible. The importance of ``cumnersome'' variants of the ARIZ algorithm is
increasingly taking a second ranked character. If you master the Matrix and
the 40 principles, you add one dozen ``delicious'' examples, the potential
customers are easily excited from an overwhelming idea: ``\emph{The
  contradiction Matrix allows in each case the targeted distillation of
  conflicting influencing variables with changes in the system}'' [8].

Hardly any TRIZ specialist of the new wave would dare to appear at a business
meeting without a well-groomed and accustomed curate folded Matrix table in
his jacket pocket. Because without the Matrix one is -- for everyone -- not a
specialist.

In Russia in the 1960s, the Matrix not only embodied the scientific
(statistical) conception of the invention activity, but played an important
political role in the process of official recognition of TRIZ. The idea of
their creation was for Altshuller a stroke of luck that made possible a
declared, hypothetically exact algorithmizability of the inventive process
with the possibility to combine it with valuable self-resources of the problem
solver. At times, it prevented the need to fall to his knees before the
sluggish invention theorists of the pre-TRIZ era.

Largely not all were ready to give up the cherished principle of inventing
through ``trial and error'' and its romantic and gold-plated lights of
enlightenment that went up though only sporadically, but certainly ``earned''.
Others were careful enough not to stick out of the crowd. It was not for
nothing that this science stood still for decades in the USSR.

Hardly in the world, the Matrix fell on a very fertile soil, as if this has
extra been prepared for it. At that time the Soviet country needed thousands
of new technical solutions. Above all, it was about improving old mechanisms
and their further development in virtually all industrial sectors.

Since the level of such innovations did not necessarily have to be too high --
preferred were not too difficult, but original solutions -- a quick and
relatively inexpensive realization was welcomed. The ``heating mechanism''
enclosed in the exact tabular form was at that time for the engineers a direct
hit. In a loose atmosphere it allowed some to stimulate the growth of advancing
ideas, and the other -- to pretend this.  
\begin{quote}
  ``\emph{Innovation procedures were still known before Altshuller. But he
    classified it (the Matrix and the 40 methods of resolving technical
    contradictions) and radically changed it in its structure}'' [11].
\end{quote}
With the help of some appealing fresh ``general'' procedures (innovation
principles) the Matrix seduced many technically enthusiasts by their unusual
external intensification of the variant selection. In a very convincing way
uneven distributed in the Matrix, the procedures had to be picked up like the
giant carrots in a computer game. For better or worse one was forced to strain
his gray cells. Dependent on the number of specified technical contradictions
and the level of general education of the ``collector'', you could get from
two or three to a dozen such ``carrots'' picked up.

\section*{Meaningless, not useless}
Today, only a small portion of TRIZ commercial users admits the need to retain
from the fraudulent practice of ``divination after the Matrix'' completely and
finally.  ``With what else?'' asks the significant amount of TRIZniks
reasonably back.

The reverent meticulousness with which the inventors of different countries
and nations guard the untaintedness of the Altshuller's Matrix, model 1971,
accurately assigning the sacred numbers from 1 to 40 to the right cells, you
can probably only compared with the sacred zeal of the Kabbalists. Any change
of the letter sequence in the old Pentateuch leads, in her opinion, not only
to a misrepresentation of meaning but also to irrevocable professional loss of
the prophecies coded in the gigantic letter puzzle and regulations.

The adoration of TRIZ scriptures including sacred masses before the image of
the Matrix in compliance with prescribed ceremonies, the West has every chance
to degenerate into an established inventor religion. If you read here and
there about the incredibly purposeful and fruitful selection of fateful
innovation principles, an old joke comes at mind, after which the old
paramedic divides a painkiller into two and admonished the sick soldiers: ``So
boy, that's for your head and that for the belly!  But beware, do not swap
it!'' Funny? Sales of similar processes in the industrial practice soared in
recent years so astronomical high that the laughter goes by.

Many diehard TRIZniks learned about the meaninglessness of the Matrix even
back in the Soviet era -- mind you: the futility, not uselessness. Somehow it
was not proper for a specialist to poke around blindly in the list of
innovation principles. And to try them out in turn (with sub-principles in
TRIZ over 100). It had a pseudoscientific smell. It is already another matter,
to grab in the Matrix with four thousand lucky numbers that are ``subject to
strict statistical laws''. This is then considered scientific.

In view of the recurring danger of revisionists and reformators which extend,
complement, grind and embarrass the beloved Matrix of 1971 especially the
fighters present themselves that stand for the ``genuine, indivisible and
nonshaky''. Anyway, it will probably not come to crusades. The situation,
rightly, is subject to the dialectical law -- ``Unity and struggle of the
opposites''. The struggle for the true TRIZ faith hides the unity of
principal questions: Whether the Matrix had ever been a functional and
well-founded tool?

\section*{A galaxy called TRIZ}
Every well educated TRIZnik knows that all utterances of the ``Classics'' has
to be shown respect, whether it is over, say, hundreds of thousands or a few
millions ``considered'' patent documents. But to distinguish fantastic things
from the real that was also taught in the Soviet Union. For this, even a
corresponding tool was developed: the Register of Fantastic Ideas by
G. Altshuller.

However, the register is not known to TRIZ's European missionaries. The power
of the Matrix is credible and traditionally recognized in Europe, and
documented with the dizzying number of ``strong'' solutions. Three-quarters of
all TRIZ activists demonstrate this power as agreed, on one and the same
example. The problem solution of pizza packaging for the US company Pizza Hut
is attributed to the Matrix application. Well, half of his interviews for a
business magazine [12] a seminar leader devotes to an enthusiastic analysis of
this long-worn solution.  ``\emph{Based on the analysis of 2.5 million patents
  identified Altshuller 39 technical parameters and 40 innovation principles
  ...}'' and ``\emph{the good ideas do not fall from the sky}'', he preaches
to the participants of his seminar.

Many active supporters of Altshuller's method are ``chewing'' at the story
with Pizza Hut [18], [19], [20], [5]. Sometimes this happens too ``under a
Dutch sauce'' [13].  This generates more or less the feeling that the few
other examples, translated from Russian and English, are inedible. Because own
application examples one might not have.

More and more often one reads from the European sources about the
\emph{gigantic} and \emph{huge} [21] patent fonds, worked through by
Altshuller during the Matrix composition between 1946--71, or that he
``\emph{examined and classified more than one million patents}'' [9] as well
as about the rest ``\emph{over 3 million writings}'' [10], that were analyzed
later by his students (with the goal of confirming the principles of
innovation already cited therein). To settle the question of the volume of
processed information once and for all, lets exchange the terms with local
proportions like ``huge'' and ``gigantic'' against a modest but tasty ``the
entire worldwide patent fund''.

\section*{The Matrix is dead! Long live the Matrix!}
At the beginning of the 1970th the position of TRIZ in the USSR was so stable
that G. Altshuller did not continue to develop his Matrix in the nearly 30
years until his death in 1998. She had already fulfilled her purpose. In
addition, their existence gradually endangered the new TRIZ policy. Altshuller
even tried to ``bury'' the results of his many years of work by proclaiming in
1975 that ``\emph{the emergence of ARIZ-71 led practically to depreciation of
  the Matrix and the 40 principles of innovation}'' [14].

But the process of the Matrix retreat from seminar programs and courses
threatened to become very worrying. Wait until it's naturally out of the
day-to-day operation, was also not a solution. The contradiction was solved on
the (for the TRIZ) usual way: by the separation principle. The one (developers
and close students) was -- in private letters or talks -- clearly indicated
that the Matrix has now no longer to be used. The other (simple readers) -- in
books and methodological materials -- it was presented in unchanged form. Of
course this was done exclusively ``for the purpose of popularizing the TRIZ''.

``\emph{At the beginning of the 1970s Altshuller himself pointed to the
  obsolescence of the Matrix and the 40 Principles. In his letter of 1976 he
  asked me not to put the Matrix in my lectures about TRIZ, because 'today
  everything seems different'}'' -- remembers TRIZ-master J. Murashkovski from
Latvia. For any average TRIZnik similar doubts ``in the board room'' would
have been very confusing. Fortunately, the sad news did not arrive at the
``simple people''.

These confessions had not prevented Altshuller, to dedicate to these
``worthless and different appearing'' tools over 30 pages of his next book
``Creativity as exact science'' [15] -- twice as much as the new ARIZ-77
algorithm. And the software developers of the 1990th, who were familiar with
TRIZ, understood that the death rumors about the Matrix are extremely
exaggerated.

After all, hundreds of business-savvy problem solvers -- not only among the
beginners -- choose with fascinating consistency the ``Worthless'' for the
main role in the most important art works. And a few dozen ``inaugurated''
insiders watch this circus and laugh softly in the fist.

\section*{The Empress's new clothes}
Entering the European arena the Matrix and the invention principles are
offered together with the nowadays already common, but still expensive
``Invention Machines'', also as posters, video clips, comics, slides and
playing cards.

``\emph{Triz poker workshop with a few beers, a problem, innovation cards, big
  fun and lots of brilliant ideas}'' [16] -- the whole training lasts just
three minutes!  ``\emph{The brainstorming will be an unforgettable experience
  for you}'', assure the game authors and advise to consolidate the gained
skills in a four-hour fun with the expert for only 590 euros. Your mission is
expressed with a simple wisdom: ``\emph{People should be able to do what they
  can not do, but would like to do, if they knew that they could do it}''.

What people can not do, for example, would be ... what a terrifying thought!

Among the card buyers and poker customers are in particular well-known
companies. The key business partners in the game in the European market are
leading European TRIZ management consultancies and magazines.

A shining criticism for the TRIZ poker game came from the Department of
Strategic Sales of a German industrial giant and the group of new technology
developments at Germany's largest scientific institute. The producer himself
prizes the incredible savings with the help of the cards of a three-day TRIZ
seminar: ``\emph{In 3 minutes you can start. You save yourself a 3-day TRIZ
training, as the Fraunhofer Institute confirmed}.'' 

One can only imagine what level of education is reached by the seminars
praised by the adorated scientists.

The Matrix new clothes are sewn on the fly but with momentum. Drinks,
children's games and radio shows for trainees are just a matter of time. But
to direct the attention of buyer's to the new, inexpensive TRIZ textbook as
conventional (printed) medium, a solid academic offerer emphasizes in its
advertisement the enclosure of the contradiction Matrix in DIN A2 format [17].

In one of his published letters from 1985, Altshuller's voice sounds very safe
and almost forgiving: ``\emph{The 40 principles of innovation have today only
  a historical significance. We work mainly with the standards}'' [1].

Did Genrich S. Altshuller joke or was he really wrong? The standards and
substance-field-analysis increasingly take on the historical significance.
The forty principles of innovation and the Matrix remain eternally alive.

\section*{Literature and Sources}

\begin{itemize}
\item[{[1]}] Altshuller's letter of 31.01.1985 (in Russian),\\
  \url{http://www.altshuller.ru/corr/correspondence1.asp} ;
\item[{[2]}] P. Livotov, D. Murnikov: „Innovation als Prozess“ (Innovation as
  process. Lecture at the 4th European TRIZ Congress, Frankfurt a. Main,
  30.06-01.07.2005).\\
  \url{https://www.researchgate.net/profile/Pavel_Livotov}; 
\item[{[3]}] TriSolver4.net Software: „TRIZ-Werkzeuge für Innovation und
  erfinderische Problem\-lösung“ (TRIZ tools for innovation and inventive
  problem solution);
\item[{[4]}] P. Livotov: „TRIZ im Innovationsprozess“ (TRIZ in the innovation
  process). Konstruktion \& Engineering, 03'2004.\\
  \url{https://www.researchgate.net/profile/Pavel_Livotov};
\item[{[5]}] M. Dworschak: „Zwergenarmeen im Kopf“ (Dwarv armies in the head).
  Der Spiegel, 30’2005, S. 114;
\item[{[6]}] \url{www.triz-online.de} WiderspruchsMatrix;
\item[{[7]}] Teufelsdorfer A., Conrad A.: „Kreatives Entwickeln und
  innovatives Problemlösen mit TRIZ/TIPS“ (Creative development and innovative
  problem solving with TRIZ/TIPS), Publicis MCD Verlag, Erlangen und München
  1998;
\item[{[8]}] R. Sietmann: „Erfinden nach Plan“ (Inventing according to plan).
  c't -- Magazin für Computertechnik, 23/01, Seite 96;
\item[{[9]}] Forschungsbuero.de: „TRIZ -- Lösen von technischen Problemen“
  (TRIZ -- solving technical problems).
  \url{http://www.forschungsbuero.de/html/triz.htm};
\item[{[10]}] H.-J. Günther: „Rechnergestützte Bearbeitung von innovativen
  Lösungen mittels der Software TechOptimizer“ (Computer-aided processing of
  innovative solutions by means of the software TechOptimizer);
\item[{[11]}] J.S. Murashkovski: “Scientific Innovation Principles“.  (in
  Russian)\\
  \url{http://subscribe.ru/archive/science.natural.triz/200310/03201643.html};
\item[{[12]}] D. Dürand: „Kind im Manne“ (Child in man). Wirtschaftswoche
  Nr.19/4.5.200; 
\item[{[13]}] C. Gundlach: „Mit Kreativität und Strategie zur Nachhaltigkeit“
  (With creativity and strategy to sustainability). 
  \texttt{triz-online-magazin.de}, volume 2005.03;
\item[{[14]}] G.S. Altshuller, G.L. Filkovski, „Modern state of the theory for
  solving the invention tasks“, 1975.
  \url{http://www.altshuller.ru/triz2.asp}, (in Russian);
\item[{[15]}] G.S. Altshuller, „Creativity as an Exact Science: Theory for
  Solving the Invention tasks“, Moscow: Soviet. Radio, 1979, (in Russian);
\item[{[16]}] Instant Innovation TRIZ: „Triz-Solution – Karten“ (Triz-Solution
  -- Cards).  \texttt{etzold.biz}, (no more online -- HGG);
\item[{[17]}] J.-P. Bläsing, W. Brunner: „TRIZ, Von der Theorie zur Praxis.
  Für TRIZ Moderatoren und TRIZ Teams“ (TRIZ, from theory to practice. For
  TRIZ moderators and TRIZ teams). (With the conflict Matrix in DIN A2),
  Edition 2001;
\item[{[18]}] B. Gimpel: „Widersprüche auflösen – Produktideen finden“
  (Dissolve contradictions -- find product ideas).  Wirtschaftliche
  Nachrichten 1/2006, S. 12;
\item[{[19]}] Interquality.de: „Wenn das der Einstein wüßte“ (If Einstein knew
  that);
\item[{[20]}] S. Nessler: „TRIZ -- Erfinden nach Plan“ (TRIZ -- inventing
  according to plan).  Deutschlandradio;
\item[{[21]}] N. Bromberger, K. Brandenburg, B. Dausien u. a.: „Neue
  Lerndienstleistungen, Vision und Wirklichkeiten“ (New learning services,
  vision and realities), Editor: Arbeitsgemeinschaft Betriebliche
  Weiterbildungsforschung e. V., printed as manuscript, September 2005, S. 48;
\item[{[22]}] Th. Bayer: „TRIZ in der Wittenstein AG“ (TRIZ in Wittenstein AG
  -- Lecture at the 4th European TRIZ-Congress, Frankfurt a. Main,
  30.06-01.07.2005);
\item[{[23]}] CREAX Innovation (2. Generate Solution);
\item[{[24]}] „TRIZ“ (Wikipedia), \url{http://de.wikipedia.org/wiki/TRIZ}.
\end{itemize}

\section*{The Author}

Dipl.-Ing. Leonid Shub, 44, is a founding member of INNOLOGICS e.V. and
responsible for the application and training in the theory of inventive
problem solving (TRIZ).

As a water engineer he started in 1984 at the Norilsk metallurgical combine,
Norilsk, Russia.  There he attended his first TRIZ seminars at B. Zlotin,
G. Ivanov and I. Bukhman.

From 1988--1990 he was head and instructor of the TRIZ department at the
Technical School of teenagers and students in Norilsk.

From 1995--2001 he was a TRIZ consultant at the Think-Tech GmbH, Kfar-Saba,
Israel.

From 2001--2003 he was a business consultant and TRIZ expert at Agamus Consult
Unternehmensberatung GmbH, Starnberg, where he conductedx innovation projects.

\end{document}
