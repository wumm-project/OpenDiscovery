\documentclass[11pt,a4paper]{article}
\usepackage{a4wide,url}
\usepackage[german]{babel}
\usepackage[utf8]{inputenc}

\parindent0cm
\parskip3pt

\title{Wie viele Väter hat die TRIZ? } 
\author{Leonid Shub}
\date{Peissenberg, 23. Mai 2006}
\begin{document}
\maketitle

\begin{quote}
  Der Text gehört zu einer 11-teiligen Serie von Texten des Autors, die in
  Russisch unter \url{http://metodolog.ru/conference.html} zu finden ist. Der
  vorliegende Text ist eine Übersetzung des Autors.

  Textaufnahme durch Hans-Gert Gräbe, Universität Leipzig. 
\end{quote}

\begin{flushright}
  „Wer die Wahrheit sagt, muss bis zum Ende gehen“\\ 
  Jean-Paul Sartre 
\end{flushright}

Als Gründer der TRIZ\footnote{TRIZ – Theorie des erfinderischen
  Problemlösens.} wird traditionell Genrich Saulovitsch Altshuller angesehen.
Sowjetisch nach seiner Staatsangehörigkeit, usbekisch – nach dem Geburtsort,
russisch – nach der Sprache in der er gedacht, gesprochen und geschrieben
hatte, aserbaidschanisch – nach seinem Wohnort, jüdisch – nach der
Nationalität der Eltern und seiner Denkweise, karelisch – im nordischen
Petrozawodsk verbrachte er seine letzten Lebensjahre – war er ein
Fantasy-Schriftsteller, Ingenieur und Patentpionier.

Die TRIZ war, wie Altshuller immer wieder betonte, sein Werk, so dass er oft
andere Spezialisten ermahnte, nicht zu versuchen, seine Methodik
weiterzuentwickeln, zu modernisieren oder sie mit anderen Methoden zu
kreuzen. Diese anfangs nur vereinzelten Versuche, an Altshullers Werk zu
basteln, wurden allmählich zu einem eigenständigen Phänomen, das mit dem
natürlichen Leistungsrückgang des Patriarchen und der „totalen
Demokratisierung“ in der UdSSR der Perestroika-Epoche zeitgleich kam.

Nach dem Tod des Lehrers 1998 erreichte der Vorgang des Umdenkens ein
lawinenartiges Ausmaß, das durch die Mehrsprachigkeit der TRIZ-Gemeinde
erheblich verstärkt wurde. Denn vor 20 Jahren war eine der Haupthürden der
globalen TRIZ-Expansion das Problem der originalgetreuen Übersetzung in
landessprachliche Ausgaben in den jeweiligen Ländern.  Wohingegen heute die
Systematisierung neuer Entwicklungen, die in etwa zehn Hauptsprachen
ausgearbeitet und nur zum Teil ins Englische übersetzt werden, immer weniger
realisierbar erscheint – geschweige denn, dass viele Koryphäen im Stande sind,
zu den neu erscheinenden nicht russischsprachigen Materialien ein
fachmännisches Gutachten abzugeben.

Es ist verständlich, dass die Proteststimmen meist aus den Reihen der treuen
und aufrichtigen Nachfolger von Altshuller kommen, die glauben -- nicht ohne
Grund -- dass die klassische TRIZ-Methode sowohl hinsichtlich des Volumens
ausgearbeiteter Materialien als auch des Entwicklungsgrades ihrer Instrumente
autark sei, und zumindest für die Lösung verschiedenster technischer Probleme
völlig hinreichend. Auf keinen Fall habe sie radikale Änderungen oder
Ergänzungen nötig.

Zwar sind diese Anhänger Altshullers höchst erfahrene Spezialisten, doch sind
sie oft territorial abgeschieden und haben von Russland aus nur wenig Einfluss
auf die Tätigkeiten der russischsprachigen Gemeinden anderer Länder. Dies ist
zweifelsohne ein vorübergehendes Phänomen und wird sich in 20 bis 30 Jahren
abschwächen. Denkt man dabei nur an die Technologieentwicklungen der nächsten
Generationen, dann freut einen diese Perspektive sicherlich.

Zukunft hin oder her, Notwendigkeit und Grund zum Umdenken des TRIZ-Leitfadens
kamen im Laufe der Zeit regelmäßig auf. Heutzutage, auf dem Hintergrund der
nicht einfachen Entstehung und Entwicklung der Methodik und der umfassenden
Datenbank aus öffentlichen und privaten Archiven, darf man sagen und sogar
behaupten, dass die TRIZ \emph{auch} Altshullers Werk gewesen ist. (So wie man
die Oktoberrevolution kaum eine Leninsche nennen kann, ohne die Namen von
Trotzki, Stalin und vieler anderer Theoretiker und Praktiker der
revolutionären Bewegung zu erwähnen.)

Die Gründerväter der Methode sind aus ihr nie mehr wegzudenken, seien es die
großartigsten Erfolge oder die ärgerlichsten Mängel. So schrieb Altshuller
darüber:
\begin{quote}
  Die Erfindungsmethodik vereinte und verallgemeinerte die Erfahrungen der
  Erfinder, und ich musste selbstverständlich mit sehr vielen Innovatoren
  Gespräche führen und mich, noch während der Ausarbeitung der Methode, von
  ihnen beraten lassen. Es waren Menschen, die sich in ihrer erfinderischen
  Laufbahn, ihrem technischen Horizont, dem Beruf, den Fähigkeiten und
  Neigungen sehr unterschieden. Und eines hatten sie gemein: Die Bestrebung,
  Neues zu erschaffen. Kein Wunder, dass sie versucht hatten – kaum die noch
  nicht vollständig ausgereifte Methode kennengelernt – diese auch gleich
  anzuwenden und später ihre eigenen Korrekturen und Ergänzungen einzubringen.
  Eine besonders wichtige Bereicherung brachten in die Entwicklung der Methode
  die Ingenieure P. Shapiro und D. Kabanov ein“ [1].
\end{quote}
Allein deswegen sind die Anfänge der TRIZ in den Persönlichkeiten ihrer
Gründerväter und deren Schüler und Anhänger nachzuvollziehen. Die Geschichte
der TRIZ zeigt sich nicht zuletzt in den chronologisch geordneten Etappen
ihrer eigenen Vorstellungen über die Wege und Möglichkeiten der Entwicklung
verschiedener Denkmechanismen. Diese Menschen – der anerkannte Autor Genrich
Altshuller und der weniger bekannte Co-Autor Rafael Shapiro (Foto 1 aus der
G.S. Altshuller-Stiftung), der völlig unbekannte Kollege D. Kabanov,
Altshullers Ehefrau Valentina Zhuravleva, die ihm lebenslang zur Seite stand,
und viele andere Denker.

Foto: V.N. Zhuravleva, G.S. Altshuller, R.B. Shapiro, Baku, 1959 (Quelle:
Altshuller - Stiftung)

Dies waren ihre Leidenschaften und Enttäuschungen, ihre genialen Ideen und
nicht weniger genialen Irrtümer. Trotz der Tatsache, dass diese
Persönlichkeiten, aus verschiedenen Gründen, noch vor der breiten Annerkennung
der Methodik aus dem Prozess ihrer Ausfeilung ausgeschieden waren, blieben
sowohl ihre brillanten Ideen, die vorgeschlagenen Konzepte und ausgearbeiteten
Modelle als auch Fehltritte und nicht erreichten Ziele der TRIZ selbst
erhalten.

Genrich Altshuller wird zwar als allgemein anerkannter Autor des ARIZ – des
komplexen Hauptinstrumentariums der TRIZ – angesehen, doch nicht als der
einzige.
\begin{quote}
  Man kann nicht die Erschaffung und schon gar nicht die Perfektionierung der
  ARIZ einer einzigen Person zuschreiben. Die Idee, ein rationales System für
  die Lösung der Erfindungsprobleme zu entwickeln, wurde in den Jahren
  1946--1949 vom leitenden Ingenieur der Erfindungsinspektion der Kaspischen
  Flotte Dimitri D. Kabanov unterstützt. Dabei begutachtete er neue
  Erfindungen nach dem Prinzip des maximalen Nutzens beim minimalen Aufwand.
  Diesem Kriterium konnten nur die Erfindungen entsprechen, die zur
  Überwindung technischer Widersprüche verhalfen.  Somit wurde einer der
  wichtigsten Grundsätze der algorithmischen Theorie für die erfinderische
  Problemlösung unter dem starken Einfluss von D.D. Kabanov ausgearbeitet [2].
\end{quote}
Es wäre schwer, Altshullers Biographen vorzuwerfen, sie hätten die
Unterrichtspläne des ersten Kurses am Aserbaidschanischen öffentlichen
Institut der erfinderischen Kreativität von 1973--74 nicht gründlich genug
studiert und den Namen einer Person übersehen, die der Methode den Begriff des
technischen Widerspruches so gut wie geschenkt hatte. Es wäre ebenfalls kaum
gerechtfertigt, Kabanov selbst die Schuld für eine mangelnde Mitwirkung bei
der Erschaffung und Weiterentwicklung der neuen Methodik zu geben, wobei man
seine Beteiligung auch als eine dienstliche Verpflichtung interpretieren
könnte.

Eine besondere Aufmerksamkeit verdienen die Person von Rafael Shapiro (siehe
Foto) sowie sein Beitrag zu der Entwicklung der TRIZ. Nicht nur, weil er
Altersgenosse und Klassenkamerad von Altshuller war, auch nicht, weil die
beiden ihre ersten Erfindungen gemeinsam gemacht und patentiert sowie die
ersten wissenschaftlichen Beiträge gemeinsam veröffentlicht hatten. Und auch
nicht, weil Shapiro jahrelang Altshullers engster Schriftstellerkollege
gewesen war und für die junge Methode aktive Öffentlichkeitsarbeit betrieben
hatte. Sondern in erster Linie, weil er einer der Mitbegründer, der
Ideengenerator und Katalysator des kreativen Prozesses war. Darf man die
Erschaffung der TRIZ manchmal mit einer Revolution vergleichen, so kann man
Shapiro zu einem ihrer leidenschaftlichsten Revolutionäre zählen, der zu
Unrecht aus den Geschichtsbüchern verbannt wurde.
\begin{quote}
  In den Jahren 1948-49 wirkte Rafael Borisovitsch Shapiro (literarischer
  Pseudonym R. Bachtamov) bei der Ausarbeitung der ersten Varianten von ARIZ
  aktiv mit. Er nahm außerdem von 1956--59 an der Arbeit teil [2].
\end{quote}
Mit anderen Worten, Shapiro war ein vollberechtigter und kompetenter Co-Autor
der effektiven, kompakten und genauen Varianten der Algorithmen ARIZ-56 und
ARIZ-59. Damals wurde er allerdings noch als der Co-Autor erwähnt [3,4].

Aus den Erinnerungen von Genrich Altshuller selbst kann man erschließen, dass
die feste Altshuller-Shapiro-Bindung auf der großen Polarität der beiden
Persönlichkeiten, ihrer Weltanschauung und Ziele begründet war. Diese
Eigenschaften erzeugten eine starke Anziehungskraft zwischen den beiden jungen
Erfindern und vereinten sie auf ihrem Weg zu den gemeinsamen Horizonten.
Kriegsjahre, Fehltritte in den ersten Jahren gemeinsamer Arbeit, Arreste,
Verhöre und Internierungen im GULAG – nichts konnte sie auseinanderbringen
und ihre spirituelle Bindung zerstören.

Erst als sich die ersten Erfolge angekündigt hatten und die Methodik
allmählich von der Öffentlichkeit anerkannt wurde, begannen ernste
Schwierigkeiten. Ein Augenzeuge dieses unermüdlichen Kampfes der Gegensätze
erinnert sich: „Rafael war gescheiter als Altshuller. Genrich war
eindrucksvoller, beeindruckender, interessanter, aber Rafa – gescheiter. Er
war überhaupt ein sehr kluger, sogar weiser Mensch.  Als er bei der Münchener
Zeitschrift „Das Land und die Welt“ gearbeitet hatte\footnote{Ein Münchener
  Blatt russischer Emigranten, in dem Shapiro, der damals in Jerusalem lebte,
  ein führender wirtschaftlicher und politischer Kolumnist war. -- L.S.},
witzelte man, dass Gorbatschow erst seine Entscheidungen trifft, nachdem er
die Artikel von Rafael gelesen hat. Wenn in einer Ausgabe zwei seiner Beiträge
erschienen waren, dann ging der eine unter dem Namen Bachtamov und der andere
unter Shapiro durch.“

Altshullers Denkweise war konkreter. Sehr arbeitsfähig und energiegeladen,
liebte er zu agieren, zu produzieren und Ergebnisse zu erzielen.  
\begin{quote}
  Mein erstes Patent für eine Erfindung bekam ich (LS: zusammen mit Shapiro
  und Talianski) noch in der Schule, Ende der zehnten Klasse. Nach der Schule
  wurde ich Student des industriellen Instituts in Aserbaidschan.  Man würde
  meinen, dass das im Studium gewonnene Wissen einem helfen sollte, bald
  wieder neue Erfindungen zu machen. Doch erst nach vielen Jahren habe ich
  mein zweites Patent bekommen. In diesen Jahren stellte ich 103 Anträge für
  eine Patentanmeldung und bekam 103 Ablehnungen [1].
\end{quote}
Altshuller war bereit, wieder und wieder zigtausend Erfindungsbeschreibungen
durchzulesen, um die Liste der Innovationsprinzipien (zuerst auf 50 und dann
noch weiter) und Standards, die Anzahl der Parameter in der Widerspruchsmatrix
etc. zu erweitern.

Shapiro hingegen widmete sich der Tätigkeit, die man eher als grandios, genial
und sensationell bezeichnen kann. Kein anderer als er initiierte dieses
verhängnisvolle Schreiben an den „Genossen Stalin“ (bzw. an die UNO), das
Altshullers Leben beinah zerstörte. Dieser Brief, der bis heute leider nicht
gefunden und veröffentlich worden ist, wurde damals an mindestens 40 Adressen
verschickt und war, nach der Legende, eine der Ursachen langjähriger
Internierungen beider Freunde in Lagern und Gefängnissen.  „Shapiro hatte die
Idee, einen Brief an Stalin zu schreiben, es war eine typische Reaktion für
ihn. Sobald er von der Erhabenheit einer Sache überwältigt wurde, wollte er
sie allen nahe legen und Resultate erzielen... Shapiro war erschütternd
optimistisch“ [5].

Es ist auch nicht verwunderlich, dass viele Informationen über Altshuller mit
Elementen der Phantasterei oder sogar des Heldentums gespickt sind, denn es
handelt sich hierbei um die Biographie eines professionellen
Fantasy-Schriftstellers. „Mit 14 hatte er bereits einige Patente auf eigene
Erfindungen. Mit 20 wurde er ein professioneller Patentrechtler und Experte
der Kaspischen Flotte in Baku. Später schuf er eine neue Wissenschaft.“ [6]

Für die so genannten Popularisatoren spielt es kaum eine Rolle, dass die erste
erfolgreiche Patentanmeldung erst mit 17 Jahren und mit zwei anderen
Co-Autoren erfolgte, und dass deren Zustimmung zwei Jahre später ankam, dass
über 100 folgende Anmeldungen abgelehnt worden waren, dass der „professionelle
Patentrechtler und Experte“ mit 20 sein Studium geschmissen hatte und „guten
Gewissens eine große Autorität auf dem Gebiet schlechter Erfindungen genannt
werden könnte“ [5]. Dass es noch viel zu früh war, über die Entstehung einer
neuen Lehre zu sprechen, war nicht wichtig. Hauptsache, es ist schön.

Zwar wird es in den meisten Archiven über die TRIZ-Geschichte gesagt, dass es
Altshuller war, der den Brief an Stalin geschrieben hatte. Manchmal heißt es
„zusammen mit einem Freund“.  Und äußerst selten wird der Name des Freundes
erwähnt. Beim Lesen von Altshullers Erinnerungen jedoch kann man das Gefühl
nicht loswerden, dass er es letztendlich nicht schaffte, die Verantwortung für
diese Tat zur Hälfte zu übernehmen. „Ich habe mehrmals versucht, ihn (LS:
Shapiro) davon abzubringen, wenn es um die Erfindung ging, doch hier ...“. An
dieser Stelle würden die Menschen, die Altshuller gut gekannt und auch seinen
schwierigen Charakter kennen gelernt hatten, wohl nur misstrauisch mit den
Schultern zucken. So wie auch bei der Erwähnung anderer Fälle, wenn Altshuller
halb freiwillig, halb gezwungen den Wünschen Rafael Shapiros Folge leistete.
Bis heute kann sich keiner an ein Präzedenzfall entsinnen, bei dem Altshuller
mit einer fremden Meinung einverstanden wäre, ohne sie auch zu teilen. Sich
Altshuller vorzustellen, der sich dem Willen anderer fügte, scheint noch
weniger möglich.

Über das Leid, das er im Gefängnis und dem GULAG zu ertragen hatte, über die
Verhöre, bei denen ihm geraten wurde, seine Schuld einzugestehen und
Altshuller zu verleumden, schrieb Rafael Shapiro in seinen Erinnerungen, als
er schon in Israel lebte. Er kam aber nie dazu, sie zu veröffentlichen. Erst
zwei Jahre nach seinem Tod wurden diese Aufzeichnungen von seiner Ehefrau Nora
in einem kleinen Buch „Fünfundzwanzig plus fünfundzwanzig“ herausgegeben. In
der Summe 50 – so alt hätte Rafael werden sollen am Tage seiner Entlassung,
entsprechend dem Gerichtsbeschluss, das – sei es Ironie des Schicksals oder
eine „hinterlistige“ Überlegung – an seinem Geburtstag am 13. Januar 1951
verlesen wurde. Das Buch, das sein engster Freund Wladimir Portnov betitelte
und dafür das Vorwort verfasste, wurde nie beendet. Shapiro, der sein Ende
kommen sah, vernichtete viele Entwürfe und Archivmaterialien.

Die Haltung Shapiros gegenüber Altshuller und deren gemeinsamen Arbeit könnte
man am ehesten als eine grenzenlose Anhimmelung und totale Selbstaufopferung
beschreiben. Als charakteristisches Beispiel lässt sich Folgendes anführen: Im
Jahre 1961 veröffentlichte R. Bachtamov sein Erzählungsband „Die Verbannung
des sechsflügeligen Seraphim“, in dem er die Verdienste Altshullers bei der
Erschaffung der neuen Methodik rühmte. Im selben Jahr räumte Altshuller in
seinem Buch „Wie lernt man zu erfinden“ Shapiros Rolle während der knapp
20-jährigen gemeinsamen Arbeit eine einzige Zeile ein und bewertet sie als
einen „besonders wichtigen Beitrag“ zu der Entwicklung seiner, der
Altshullerschen, Methode. Des Weiteren teilte er -- „von Rechts wegen“ --
diesen Beitrag auf die beiden: Shapiro und Kabanov.

Ob das nur ein Zufall war, dass Altshuller die unmittelbare Zusammensetzung
der Verfahrenstabelle zur Überwindung technischer Widersprüche (der
„Widerspruchsmatrix“) -- eine der wahrhaft mühsamsten, aber auch die am
wenigsten kreative Tätigkeit auf dem Wege zur komplexen Methode -- erst nach
Shapiros Abgang angefangen hatte? Und davor...

„Shapiro hatte unsere Perspektiven sehr schnell erfasst. Er sagte damals:
„Marx entdeckte die Gesetze der Gesellschaftsentwicklung, Darwin – die
Evolutionstheorie und wir werden eine Theorie ausarbeiten, die der Welt die
Entwicklungsgesetze technischer Systeme offenbart.“ ...  Somit hatte er die
Sache als erster richtig eingeschätzt, das ist wichtig. Er hatte den Ausmaß
gesehen“ [5]. Diese Worte Altshullers waren nicht in einem seiner Bücher oder
Artikel zu lesen und auch nicht in den 60ern. Erst 1986 bei einem privaten
Interview zwischen Altshuller und einem seiner Schüler erkannte er gebührend
Shapiros Verdienste an.

Zeitlich fällt der Abgang Shapiros mit dem starkem Interesse Altshullers an
der Konstruktion der „Evrotron“, der ersten mechanischen „Erfindungsmaschine“
zusammen, wie sie von der Altshuller-Stiftung präsentiert wird. Darüber ist
bis heute nur allzu wenig bekannt. Dafür ist die Rolle der „Evrotron“ als
Katalysator der Entstehung der Altshullerschen Widerspruchsmatrix ziemlich
offensichtlich. Der mechanische „Erfinder“ benötigte eine leicht verwertbare
„mechanische“ Kost.

Sollten wir jedoch den revolutionären Gedanken beiseite lassen und Altshuller
als den alleinigen „Vater“ der TRIZ wahrnehmen, der sie erschaffen und
aufgezogen hat, dann wäre Shapiro ihr Pate, der sie zur Zeit ihrer Reifung
verlassen hat und später aller Anerkennung und Rechte beraubt wurde. Erst zwei
Jahre nach dem Tod Shapiros, am Vorabend seines 70. Geburtstages, nennt
Altshuller zum ersten Mal einige Daten über seinen Freund und Kollegen:
„Shapiro, Rafael Borisovitsch (alias Bachtamov) hat sich 1942 mit mir zusammen
getan (LS: Genrich war damals 16), zusammen beendeten wir die Schule und
studierten an der Fakultät für Erdöltechnik des Aserbaidschanischen
Industrieinstituts. Gemeinsam haben wir unsere erste Erfindung angemeldet. Und
die Zeit der Internierung im GULAG war auch die gleiche, 25 Jahre.  Entlassen
wurden wir ebenfalls am selben Tag, am 23. Oktober 1954. Nach seiner
Entlassung arbeitete Shapiro bis 1961 für TRIZ. 1994 schied er aus dem Leben“
[7].

Ihre gemeinsame Arbeit war nun Vergangenheit, doch die ehemaligen Kollegen
verloren sich nicht aus dem Auge. Sie setzten sich weiter über die
Entwicklungswege der TRIZ unter dem Schleier ihrer literarischen Pseudonyme
auseinander. Tatsächlich geben die Erzählungen und Geschichten von Altow
(Altshuller) und Bachtamov (Shapiro) die Entfaltung der Persönlich"|keiten und
Ideologien dieser beiden bedeutendsten TRIZ-Gründer sehr anschaulich wieder.
\begin{quote}
  „Man kann nicht die Wissenschaft überholen“, sagt Bachtamov. „Ich möchte
  keinen meiner Schriftstellerkollegen beleidigen, glaube jedoch, dass der
  Gegenstand eines wissenschaftlich-phantastischen Buches kein technisches
  Problem sein darf, sondern einzig und allein menschliche Ideen, menschliche
  Probleme, kurzum, der Mensch der Zukunft. Dann ist es echte Literatur.“
\end{quote}
Altow widerspricht Bachtamov: 
\begin{quote}
  „Um über den Menschen der Zukunft sprechen zu können, muss man darüber
  sprechen, wo er jetzt lebt, in welcher Welt. Und davon, wie wir uns
  entwickeln und auf diese Zukunft vorbereiten, wie der technische Fortschritt
  sein wird – davon hängt es ab, wie dieser Mensch zu beschreiben ist.“
\end{quote}
R. Bachtamov schrieb das neue Buch „Herr der Oxywelt“. Den
Fantasy-Schriftsteller beeindruckte die Frage, was mit solch einer
menschlichen Tugend wie dem Heldentum wohl geschehe, sollte sie nicht mehr
gebraucht werden? [8]

Es ist denkbar, dass die grundlegenden Unstimmigkeiten zwischen den beiden
Schriftstellern bezüglich der Gestaltung der wissenschaftlich-phantastischen
Literatur eine Rolle dabei gespielt haben, dass der Name Shapiros für viele
Jahre aus sämtlichen TRIZ-Materialien und Veröffentlichungen verbannt wurde.

Wie dem auch sei, war Rafael Shapiro, so wie auch viele andere, ausgestiegen.
Altshuller blieb und setzte die Arbeit fort. Letztendlich war er 50 Jahre lang
Organisator und Koordinator, Verbindungselement und Informationsvehikel, kurz
gesagt, der Motor des ganzen Prozesses gewesen, der heutzutage traditionell
als die „Erschaffung der TRIZ“ verstanden wird. Allein schon deswegen ist der
Pool der Ideen, Erfahrungen, aber auch der Stereotypen, den wir als die
„klassische TRIZ“ bezeichnen, in vielerlei Hinsicht der Spiegel von
Altshullers Leidenschaften, genialer Ideen und verzweifelter Illusionen.

Das sind einerseits Altshullers Siege über die eigenen Denkblockaden, die
einem Genie im gleichen Maße inhärent sind wie seinen Vorgängern und
Zeitgenossen, die aber erst auf einem unvergleichbar höheren Niveau entlarvt
werden können. Andererseits sind es auch seine Fehler, die er auch sich selbst
nur schwer eingestehen und teilweise nie mehr ausbügeln konnte. Fehler, die
umso verletzender waren, je mehr sie ihm an Lebensjahren abverlangten, um sie
zuerst zu machen und dann zu beheben und zu „vergessen“. Wenn wir dem Genie
Altshullers Tribut zollen, dürfen wir seinen noch so genialen Fehlern nicht
völlig blind folgen und sie beschönigen, sei es in einem verzweifelten
Versuch, das Gesicht des Lehrers zu wahren. Fehler nicht zuzugeben heißt neue,
noch teurere und schwerer zu beseitigende Schäden anzurichten.

An dieser Stelle sei gesagt, dass die TRIZ ein bisher sehr erfolgreicher, aber
nicht beendeter Versuch ist, die ganze Vielfalt der Denkmechanismen mehrerer
Erfindergenerationen zu systematisieren, die Genrich Altshuller sein ganzes
Leben lang sorgfältig sammelte, entwickelte, prüfte und anwendete. Das ist der
Grund, warum die Methodik in ihrer ganzen Mannigfaltigkeit, vor allem in den
1980ern, die Dynamik der Persönlichkeitsentfaltung, Denkprozesse und der
wechselnden Ansichten Altshullers widerspiegelte. Lediglich vereinzelte Fälle
sind der Öffentlichkeit bekannt, bei denen Altshuller seine Einwilligung für
bedeutende Änderungen oder Ergänzungen seiner Nachfolger gegeben hatte.
Insbesondere bestand er auf seinem uneingeschränkten Vorrecht für die
Herstellung und Weiterentwicklung der ARIZ.

Die TRIZ ist keine in sich geschlossene „Methode zur Findung neuer Ideen“ für
jedermann, sondern eher ein Muster, eine Schablone, mit deren Hilfe man sein
eigenes System zur Denkentfaltung auf der Basis eigener Gegebenheiten,
Erfahrungen und Stereotype, angepasst an die zu lösenden Aufgaben,
zusammenstellen kann. Der theoretische Teil hätte für ihre Autoren
„abgeschlossen“ werden können. Doch mit dem Tod Altshullers starb auch die
Hoffnung darauf.

Die Allgemeine Theorie des Mächtigen Denkens und die Theorie der kreativen
Persönlichkeits"|entfaltung bleiben wohl unvollendet. Der Vorgang der
Erschaffung der Theorie des mächtigen Denkens verwandelte sich in ein das
System selbst überragendes hohes Ziel, das genauso ideell wie unerreichbar im
Rahmen eines Lebens oder des Lebens einer Generation ist.  Nicht umsonst ist
eine ernsthafte Auseinandersetzung mit der TRIZ eng mit dem Verständnis ihrer
Entstehungsgeschichte verbunden.

Die TRIZ-Methode, begründet auf einer Synthese der Innovationsprinzipien,
-verfahren und -algorithmen, spiegelt die Denkmodelle wider, die verschiedenen
kreativen Persönlichkeiten in ihren unterschiedlichsten Lebensabschnitten
inhärent waren. Dieser Umstand erklärt, warum sich einige erfahrene und
verdiente Spezialisten weigern, einzelne Elemente des Instrumentariums
anzuwenden oder sie sogar aktiv ablehnen. Nicht selten wurden einzelne
Elemente „umgedacht“, unwissend vereinfacht oder sogar auf das Niveau von vor
30 bis 40 Jahren zurückgebildet.  Andererseits werden einige Operatoren und
Verfahren, sogar die ARIZ selbst, zeitweise von Enthusiasten aus der Reihe
neuer „TRIZniks“ ergänzt und weiter ausgefeilt. Diese Vertreter der
marktwirtschaftlich orientierten Generation können sich oft nicht vorstellen,
dass ihre Vorgänger diesen Weg bereits einmal (oder mehrfach) gegangen waren.

\section*{Literatur und Quellen:}
\begin{itemize}
\item[{[1]}] G.S. Altshuller. «Wie lernt man zu erfinden», Bücherverlag, 1961,
  Tambov, in Russisch.
\item[{[2]}] G.S. Altshuller. Unterrichtspläne im ersten Jahr am
  Aserbaidschanischen Industrie-Instituts, 1973--74,
  \url{http://www.altshuller.ru/engineering6.asp}, in Russisch.
\item[{[3]}] G.S. Altshuller, R.B. Shapiro. «Zur Psychologie der
  Erfindungskreativität», Probleme der Psychologie, 1956, No 6,
  \url{http://www.altshuller.ru/triz0.asp}, in Russisch.
\item[{[4]}] G.S. Altshuller, R.B. Shapiro. «Verbannung des sechsflügeligen
  Seraphim», Erfinder und Innovator, 1959.  No. 10,
  \url{http://www.altshuller.ru/triz12.asp}, in Russisch.
\item[{[5]}] G.S. Altshuller. Das Leben eines Menschen 1-C-502. 1985--1986.\\
  \url{http://www.altshuller.ru/interview5.asp}, in Russisch.
\item[{[6]}] «G.S. Altshuller». Zentrale Jüdische Ressource.\\
  \url{http://www.sem40.ru/famous2/e428.shtml}, in Russisch.
\item[{[7]}] G.S. Altshuller. Antworten auf die Fragen von James Kovalik,
  16.06.1996.\\ \url{http://www.altshuller.ru/interview5.asp}, in Russisch.
\item[{[8]}] P. Amnuel. «Der Alte Jules Verne und die kosmische Ära». Die
  Jugend von Aserbaidschan, Baku, 29. März 1964. \url{http://fandom.rusf.ru},
  in Russisch.
\end{itemize}

\section*{Der Autor}

Dipl.-Ing. Leonid Shub, 44, ist Gründungsmitglied des INNOLOGICS e.V. und
verantwortlich für die Anwendung und Schulung in der Theorie des
erfinderischen Problemlösens (TRIZ).

Als Wasseringenieur begann er 1984 bei dem Norilsker Hüttenkombinat, Norilsk,
Russland. Dort hat er an seinen ersten TRIZ-Seminaren bei B. Zlotin, G. Ivanov
und I. Bukhman teilgenommen.

1988--1990 war er Leiter und Referent der TRIZ-Abteilung an der Technischen
Schule für Jugendliche und Studenten in Norilsk.

1995--2001 war er TRIZ-Berater bei Think-Tech GmbH, Kfar-Saba, Israel.

Von 2001 bis 2003 war Leonid Shub Unternehmensberater und TRIZ-Experte bei
Agamus Consult Unternehmensberatung GmbH, Starnberg, tätig und führte dort
Innovationsprojekte.

\end{document}
