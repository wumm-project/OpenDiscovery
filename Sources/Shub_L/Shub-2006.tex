\documentclass[11pt,a4paper]{article}
\usepackage{a4wide,url}
\usepackage[german]{babel}
\usepackage[utf8]{inputenc}

\parindent0cm
\parskip3pt

\title{Vorsicht, Widerspruchsmatrix!} 
\author{Leonid Shub}
\date{Peissenberg, 20. Oktober 2006}
\begin{document}
\maketitle

\begin{quote}
  Der Text gehört zu einer 11-teiligen Serie von Texten des Autors, die in
  Russisch unter \url{http://metodolog.ru/conference.html} zu finden ist. Der
  vorliegende Text ist eine Übersetzung des Autors.

  Textaufnahme durch Hans-Gert Gräbe, Universität Leipzig. Nicht mehr
  funktionierende Links wurden weitgehend entfernt.
\end{quote}
  
Die 1965-71 als ein Phantasiewerk entstandene und 1975 „ausgestorbene“
Altshullersche Matrix samt vierzig Innovationsprinzipien erlebt in Europa eine
künstliche Renaissance.  Gut 90\% der TRIZ-Anwender bevorzugen dieses
„Instrument“, das eigentlich nicht existiert. Doch ohne Unterstützung von
Produzenten moderner „Erfindungsmaschinen“ wäre dieses Comeback kaum denkbar.

Nicht viele Fragen wurden nach meinem Vortrag zum Thema „Innovationen
vorentwickeln“ auf dem 4. Europäischen TRIZ-Kongress (Frankfurt a. Main,
30.06.2005) gestellt. Zwei davon klangen ähnlich: „Bei Ihrer Arbeit an einem
Projekt analysieren Sie das Problem, formulieren den technischen Widerspruch,
wählen ein Lösungsverfahren nach der Matrix, und dann...?“

Nach meiner Bemerkung, dass die Matrix kein Lösungsinstrumentarium sei, und
ich sie darum nicht verwende, und dass Altshuller ihren Wertverlust schon 1975
verkündet und ihr vor 20 Jahren lediglich eine geschichtliche Bedeutung
eingeräumt hätte, wurde es still um mich. Ich spürte, dass diese Antwort viele
Kongressteilnehmer schockierte. Keine weiteren Fragen. Offenkundig ließ sich
mit einem TRIZ-Experten, der in seiner Arbeit nicht auf die Matrix
zurückgreift, nichts weiter bereden.

Dies war Anlass genug, mich nochmals intensiver mit dem Phänomen der
„Verfahrenstabelle zur Überwindung technischer Widersprüche“ auseinander zu
setzen. Und gleichzeitig mir selbst den Sinn von deren Existenz ein für
allemal zu erklären. Habe ich vielleicht in den 20 Jahren seit meinem letzten
TRIZ-Lehrgang irgendwelche bahnbrechenden Entwicklungen in dieser Frage
verpasst?

\section*{Das wichtigste Instrument}
In einem anderen Vortrag [2] auf demselben Kongress wurde eine vergleichende
Analyse der Nutzungshäufigkeit und Effektivität wichtigster TRIZ-Instrumente
nach den Ergebnissen zahlreicher Projekte dargeboten. Diese Projekte führten
Vertreter einer TRIZ-Consulting in den Jahren 1998 bis 2005 in achtunddreißig
Industrieunternehmen durch.

Die Widerspruchstabelle mitsamt Innovationsprinzipien erntet laut dem Vortrag
die uneingeschränkte Sympathie der Kundschaft. Sie wurde in 96\% aller Fälle
verwendet und ist somit der eindeutige Führer unter den durch Berater in
Projekten angewendeten TRIZ-Lösungsverfahren. Dabei überholt sie bedeutend
die Funktions- und Widerspruchsanalyse (80\% Anwendungshäufigkeit) und scheint
gar unerreichbar zu sein für solch „zweitrangigen“ Instrumente wie die
Separationsprinzipien (36\%) oder „Standardlösungen und Stoff-Feld-Analyse“
(12\%). Die alte Matrix schaffte es tatsächlich, die fünf zuletzt
aufgelisteten Instrumente zusammen (einschließlich der Prognostizierung und
AFE – Antizipierende Fehleranalyse) in ihrer Nutzungshäufigkeit weitgehend zu
übertreffen.

Knapp die Hälfte (47\%) der Matrixanwendungen durch die Firmenspezialisten
führte – laut dem vorgeführten Diagramm – zu „starken“ Lösungen. Die
Gesamtzahl der Matrix-Lösungen übersteigt fünffach die Resultate der
Separationsprinzipien. Die Erfolge der Stoff-Feld-Analyse erscheinen auf
diesem Hintergrund eher lächerlich.

Laut ihren Vertretern ist diese Firma als Consulting-Unternehmen und
Softwareentwickler im Bereich der TRIZ und Innovationen auf dem
deutschsprachigen Markt führend. Logischerweise geht man davon aus, dass sie
sich alle Mühe gibt, die Methode in verschiedenen Industrien zu popularisieren
und dabei ihren Instrumentenbestand zu erweitern.

Ein Vergleich der im Vortrag erwähnten Angaben mit denen aus dem Jahr 2004 [4]
macht deutlich, dass die Nachfrage nach beinahe allen in dem Diagramm
dargestellten „klassischen“ TRIZ-Instrumenten innerhalb nur eines Jahres um
30\% bis 50\% gesunken ist.  Nach allen – außer der Matrix.

Besonders brisant scheint die Tatsache, dass die Separationsprinzipien noch
vor einem Jahr eine so hohe Nachfrage (63\%) seitens der Kundschaft
erwirtschaftete, dass man sich wünschte, die aktuelle Angabe (36\%) wäre
lieber ein Schreibfehler. Und nur die Anwendung des hausgemachten ARIZ wies
2004 einen noch sporadischeren Charakter als heute, nämlich von nur 16\%.

Nur auf den ersten Blick scheint die negative Dynamik der Anwendungshäufigkeit
verschiedener TRIZ-Elemente in der Praxis absurd. Kennt man die wichtigsten
Funktionsprinzipien der gängigen Erfindungs-Softwareprogramme, dann ist diese
Dynamik durchaus nachvollziehbar. Die Basis dieser Programme stellen wiederum
die „vorreitenden Nichtslöser“ der TRIZ: Die Widerspruchmatrix und 40
Innovationsprinzipien [3], [23].

Somit erreicht die Verdrängung intelligenter und effektiver Instrumente, die
jedoch bei der Ausbildung einen höheren Zeiteinsatz verlangen (und allein
deswegen weniger reizvoll erscheinen), durch die „universelle“ Matrix echte
industrielle Ausmaße. Diese Tendenz nimmt immer deutlichere Konturen an,
sowohl bei den einzelnen Beratern als auch bei den Anwendern der TRIZ in
Unternehmen. Und ihre lawinenartige Beschleunigung steckt die Nachbarn auf dem
Beratermarkt wortwörtlich mit ihrem Eifer an.

\section*{Sag mir, Matrix an der Wand...}
Ein weiterer Vortrag auf dem Kongress widmete sich mit Begeisterung dem Umgang
mit den „Spitzenwerkzeugen“ [22]. Der Kern einer „Problemlösung mit TRIZ“, die
zur revolutionären Innovationen führt, befinde sich in einer konsequenten
Überwindung der entdeckten Konflikte nach der Matrix. Im kurz darauf
erschienenen Medienbericht wurde das Erfolgsgeheimnis der Maschinenbaufirma
geöffnet: Deren Forschungsleiter hat bloß „\emph{die 40 Prinzipien stets vor
  Augen. In seinem Büro ... hängen sie an der Wand, gleich neben einer
  Tabelle, die ihm herauszufinden hilft, welche Regel zu seinem jeweiligen
  Dilemma passt}“ [5].

Auch die Wikipedia (Web-basierte freie Enzyklopädie) widmet dem
„Erfolgspotenzial der Tabelle“ 90\% eines TRIZ-Artikels [24]. Es wird zwar
vorsichtshalber hinzugefügt, dass „\emph{Die Widerspruchstabelle kann
  durchschnittlich in weniger als 10\% aller Aufgaben mit Erfolg verwendet
  werden. Zu empfehlen ist deshalb eine direkte zeitsparende Anwendung von
  Innovationsprinzipien in der Reihenfolge ihrer statistischen
  Anwendungshäufigkeit.}“ Der Leser wird jedoch das Gefühl nicht los, die
Tabelle ist auch 90\% der TRIZ.

Diese Art „unaufdringlicher Werbung“ erscheint immer häufiger in der
technischen und populärwissenschaftlichen Presse sowie in den wichtigen
Informationszeitschriften. Die Europäer wissen es noch, dass eine andere,
nicht weniger wichtige Tabelle (das Periodensystem der Elemente, ursprünglich
vom russischen Chemiker D. Mendeleev vorgeschlagen) einst auch aus dem
frostigen Osten kam und dass auch sie noch freie Zellen hatte. Dieses Argument
überzeugt viele Skeptiker.

\section*{TRIZ-Heraldik}
Also was ist das für ein universelles Instrument des Erfinders, das 1998 bis
2005 eine absolute Vorliebe von 38 industriellen Unternehmen genoss? Ein
mächtiges Instrument des erfinderischen Denkens – Denkmal für das Genie
Altshuller? Ein faszinierendes Kaleidoskop, das bei jeder Drehung neue
spannende Muster erzeugt? Ein lustiges Spielzeug der Projektmacher, das ihnen
ermöglicht, sowohl irrsinnige als auch wahnsinnig realitätsnahe Ideen zu
generieren?

Die im Internet veröffentlichte exaktere deutsche Version der Altshullerschen
Tabelle in ihrer „klassischen“ Form [6] wird von zahlreichen europäischen
Anhängern und Liebhabern dieses TRIZ-Abschnittes kopiert. Und auch als
Abbildung auf dem Titelblatt einer der ersten deutschen TRIZ/TIPS-Ausgaben
diente die zum TRIZ-Wappen gewordene stilisierte Matrix [7].

Genau diese verkaufen viele europäische Berater als die „beinah komplette“
TRIZ, oder wenigstens deren grundlegenden, exakten und fehlerfreien
Universalmechanismus.  Zumindest statistisch können ihre Behauptungen durchaus
begründet werden. Bei der ständig zunehmenden Anzahl von Menschen
verschiedener Berufe wird die „russische“ Erfindungsmethode mit der Matrix
assoziiert.

Würde man eine Umfrage durchführen und alle Europäer, die irgendwas über die
TRIZ mitbekommen haben, danach fragen, was sie sich darunter vorstellen,
bekäme man in 99\% aller Fälle diese Standardantwort: die Matrix und 40
allgemeine Regeln der Innovation.  Dagegen ruft die Vorstellung einer
leuchtenden Lenin-Birne im Kopf, die in Russland oft als Symbol für das
kreative und erfinderische Denken dient, bei einem europäischen Leser eher
schmerzliche Assoziationen hervor.

Der Rodinsche „Denker“, der für eine Idee sein letztes Hemd hergegeben hatte,
beeindruckt ebenso wenig. Die Thesen der Altshullerschen ŽSTL (Lebensstrategie
der kreativen Persön\-lich\-keit) werden auf den Einführungs- und
Schnupperseminaren nicht erwähnt – diese Materialien lassen sich partout in
keine europäische Sprache adäquat übersetzen. Dafür ist die Matrix auch ohne
Übersetzung verständlich.

Die Freiheit, zahlreiche Widersprüche zu formulieren und dabei die
eigenartigsten Konfliktpaare zu erhalten, erzeugt tatsächlich den Anschein
einer erkennbaren, wenn auch oberflächlichen Ideengenerierung. Dieses Phänomen
lässt die Matrixanwendung als ein „philosophisch angehauchtes“ und
einigermaßen systematisiertes Mischmasch aus der Variantenauswahl und dem
Brainstorming erscheinen und macht sie gleichzeitig zu einer großen
Zielscheibe für selbstständig denkende und technisch erfahrene Kritiker.

\section*{„Gematrixtes“ Erfinden}
Die einst ausgestorbene Matrix eröffnete eine neue Ära in der Entwicklung und
der Verbreitung der TRIZ im Westen. Die durch sie repräsentierte Methode fing
an, in einen Massenbedarfsartikel der Erfindungsbranche zu mutieren. Ein
tiefgründiges und denkerisches Studieren philosophischer Thesen wurde nicht
mehr obligatorisch und in der Regel kaum möglich. Die Bedeutung
„schwerfälliger“ Varianten der ARIZ-Algorithmen nimmt immer mehr einen
zweitrangigen Charakter an. Beherrscht man die Matrix und die 40 Prinzipien
und fügt noch ein Dutzend „köstlicher“ Beispiele dazu, so lassen sich die
potenziellen Kunden leicht von einer überwältigenden Idee begeistern:
„\emph{Die Widerspruchsmatrix erlaubt in jedem Fall das gezielte Destillieren
  konfligierender Einflussgrößen bei Veränderungen am System}“ [8].

Kaum ein TRIZ-Spezialist der neuen Welle würde es wagen, ohne eine gepflegte
und akkurat zusammengefaltete Matrixtabelle in seiner Jackettasche zu einem
Geschäftstreffen zu erscheinen. Denn ohne die Matrix ist man – für jedermann –
kein Spezialist.

Im Russland der 1960er Jahre verkörperte die Matrix nicht nur die
wissenschaftliche (statistische) Auffassung der Erfindungstätigkeit, sondern
spielte eine wichtige politische Rolle im Prozess der offiziellen Anerkennung
der TRIZ. Die Idee ihrer Erschaffung war für Altshuller ein Glücksfall, der es
ermöglichte, eine deklarierte, hypothetisch exakte Algorithmisierbarkeit des
erfinderischen Prozesses mit der Möglichkeit zu verbinden, wertvolle
Eigenressourcen des Problemlösers anzuwenden. Zeitweise entschuldigte sie von
der Notwendigkeit, sich vor die trägen Erfindungstheoretiker der vor-TRIZschen
Epoche auf die Knie zu werfen.

Lange nicht alle waren bereit, das lieb gewonnene Prinzip des Erfindens durch
„Versuch und Irrtum“ aufzugeben, dessen romantische und goldumwobene Lichter
der Erleuchtung zwar nur sporadisch, dafür aber durchaus „verdient“ aufgingen.
Andere hüteten sich davor, aus der Menge herauszuragen. Denn nicht umsonst
hatte diese Wissenschaft in UdSSR Jahrzehnte lang still gestanden.

Kaum auf der Welt, fiel die Matrix auf einen sehr fruchtbaren Boden, als wäre
dieser extra für sie vorbereitet gewesen. Das sowjetische Land brauchte damals
tausende neuer technischer Lösungen. Vor allem ging es um die Verbesserung
alter Mechanismen und deren Weiterentwicklungen in praktisch allen
Industriebranchen.

Da das Niveau solcher Innovationen nicht unbedingt allzu hoch sein musste –
bevorzugt waren eben nicht zu schwierige, aber originelle Lösungen – begrüßte
man eine schnelle und relativ preiswerte Realisierung. Der in die exakte
tabellarische Form eingeschlossene „Aufheizungsmechanismus“ für die Ingenieure
war zu jener Zeit ein Volltreffer. In einer lockeren Atmosphäre ermöglichte es
den einen, das Wachstum weiter bringender Ideen zu stimulieren, und den
anderen – dieses vorzutäuschen.
\begin{quote}\em
  „Innovationsverfahren waren noch vor Altshuller bekannt. Doch er hat sie
  (die Matrix und die 40 Verfahren zur Lösung technischer Widersprüche)
  klassifiziert und in ihrer Struktur radikal geändert“ [11].
\end{quote}
Mit Hilfe einiger ansprechend frischer „allgemeiner“ Verfahren
(Innovationsprinzipien) konnte die Matrix viele technisch Begeisterten durch
ihre ungewöhnliche äußerliche Intensivierung der Variantenauswahl verführen.
Auf eine sehr überzeugende Weise ungleichmäßig in der Matrix verteilt, mussten
die Verfahren wie die Riesenmöhren im Computerspiel aufgesammelt werden. Wohl
oder übel sah man sich gezwungen, seine grauen Zellen anzustrengen. Abhängig
von der Anzahl festgelegter technischer Widersprüche und dem Grad der
Allgemeinbildung der „Sammler“, konnte man von zwei oder drei bis zu einem
Dutzend solcher „Möhren“ aufsammeln.

\section*{Sinnlos, nicht nutzlos}
Heutzutage gibt nur ein kleiner Teil kaufmännischer TRIZ-Anwender die
Notwendigkeit zu, von der betrügerischen Praxis der „Weissagung nach der
Matrix“ vollständig und endgültig abzusehen. „Womit denn sonst?“ fragt die
bedeutende Menge der TRIZniks vernünftigerweise zurück.

Die ehrfurchtsvolle Sorgfältigkeit, mit der die Erfinder verschiedener Länder
und Nationen die Unberührtheit der Altshullerschen Matrix, Modell 1971, hüten,
indem sie die heiligen Zahlen von 1 bis 40 akkurat den richtigen Zellen
zuordnen, kann man wohl nur mit dem heiligen Eifer der Kabbalisten
vergleichen. Jegliche Änderung der Buchstabensequenz im alten Pentateuch führt
ihrer Meinung nach nicht nur zu einer Sinnentstellung, sondern auch zum
unwiderruflichen Verlust der im gigantischen Buchstabenrätsel verschlüsselten
Prophezeiungen und Vorschriften.

Die Anbetung TRIZscher Schriften einschließlich heiliger Messen vor dem Bilde
der Matrix unter Einhaltung vorgeschriebener Zeremonien hat im Westen alle
Chancen, in eine etablierte Erfinderreligion auszuarten. Liest man hier und da
über die unglaublich zielgerichtete und ergebnisvolle Auswahl von
schicksalsträchtigen Innovationsprinzipien, so fällt einem unwillkürlich ein
alter Witz ein, wonach der alte Sanitäter eine Schmerztablette in zwei
zerteilt und den kranken Soldaten ermahnt: „So Junge, das ist für deinen Kopf
und das für den Bauch! Aber Vorsicht, nicht vertauschen!“ Lustig? Der Umsatz
von den ähnlichen Prozessen in der industriellen Praxis schnellte in den
letzten Jahren derart astronomische Höhe, dass einem das Lachen vergeht.

Viele eingefleischte TRIZniks hatten die Sinnlosigkeit der Matrix noch zur
Sowjetzeit kennen gelernt – wohlbemerkt: die Sinnlosigkeit, nicht
Nutzlosigkeit. Irgendwie gehört es sich nicht für einen Spezialisten, blind in
der Liste der Innovationsprinzipien herumzustochern. Und sie der Reihe nach
auszuprobieren (mit den Unterprinzipien sind es in TRIZ über 100), hat den
Ruch des Pseudowissenschaftlichen. Da ist es schon eine andere Sache, in die
Matrix mit viertausend glücklichen Zahlen, die „den strikten statistischen
Gesetzen unterliegen“, zu greifen. Das wird dann als wissenschaftlich
angesehen.

Angesichts der immer wieder aufkommenden Gefahr seitens Revisionisten und
Reformatoren, die die geliebte Matrix von 1971 erweitern, ergänzen, schleifen
und peinlich schmücken wollen, machen sich vor allem die Kämpfer präsent, die
für das „Echte, Unteilbare und Unerschütterliche“ stehen. Wie dem auch sei, zu
Kreuzzügen wird es wohl nicht kommen. Die Lage unterliegt, richtigerweise, dem
dialektischen Gesetz – „Einigkeit und Kampf der Gegensätze“. Der Kampf um den
echten TRIZ-Glauben kaschiert die Einigkeit prinzipieller Frage: Ob die Matrix
je ein funktionalfähiges und begründetes Instrument gewesen war?

\section*{Eine Galaxie namens TRIZ}
Jeder gut erzogene TRIZnik weiß, dass allen Äußerungen der „Klassiker“
gebührender Respekt zu erweisen ist, ob es sich nun über, sagen wir,
Hunderttausende oder paar Millionen „berücksichtigter“ Patentdokumente
handelt. Doch Fantastisches vom Realen zu unterscheiden, wurde in der
Sowjetunion ebenfalls gelehrt. Dafür wurde sogar ein entsprechendes Instrument
entwickelt: das Register fantastischer Ideen von G. Altshuller.

Europäischen Missionaren der TRIZ ist das Register jedoch kein Begriff. Die
Macht der Matrix wird in Europa gutgläubig und traditionell anerkannt und mit
den Schwindel erregenden Zahlen „starker“ Lösungen belegt. Drei Viertel aller
TRIZ-Aktivisten demonstrieren diese Macht, wie ausgemacht, an ein und
demselben Beispiel. Die Problemlösung der Pizzaverpackung für die US-Firma
Pizza-Hut wird der Matrixanwendung zugeschrieben. Gut die Hälfte seines
Interviews für ein Wirtschaftsmagazin [12] widmet ein Seminarleiter einer
begeisterten Analyse dieser längst abgedroschenen Lösung. „\emph{Auf der Basis
  der Analyse von 2{,}5 Millionen Patenten identifizierte Altshuller 39
  technische Parameter und 40 Innovationsprinzipien...“ und „die guten Ideen
  fallen nicht vom Himmel}“, predigt er den Teilnehmern seines Seminars.

Viele aktive Befürworter der Altshullerschen Methode „kauen“ an der Geschichte
mit Pizza-Hut [18], [19], [20], [5]. Manchmal geschieht dies auch „unter einer
holländischen Soße“ [13]. Da hat man wohl oder übel das Gefühl, dass die
wenigen anderen Beispiele, übersetzt aus dem Russischen und Englischen,
ungenießbar sind. Denn eigene Anwendungsbeispiele hat man möglicherweise gar
nicht.

Immer häufiger liest man aus den europäischen Quellen über den gigantischen
und riesigen [21], von Altshuller während der Matrixzusammensetzung zwischen
1946--71 bearbeiteten Patentfond, oder über von ihm „untersuchten und
klassifizierten mehr als eine Million Patente“ [9] sowie die restlichen „über
3 Millionen Schriften“ [10], die später durch seine Schüler (mit dem Ziel, die
darin bereits angeführten Innovationsprinzipien zu bestätigen) nachgearbeitet
wurden. Um die Frage nach dem Volumen verarbeiteter Informationen ein für
allemal abzuschließen, tauschen wir die Termini mit lokalem Ausmaße wie
„riesig“ und „gigantisch“ gegen ein bescheidenes aber geschmackvolles „der
gesamte weltweite Patentfond“.

\section*{Die Matrix ist tot! Es lebe die Matrix!}
Zu Beginn der 70er Jahren war die Position der TRIZ in UdSSR dermaßen stabil,
dass G. Altshuller seine Matrix in den knapp 30 Jahren bis zu seinem Tode 1998
nicht weiter entwickelte. Ihren Zweck hatte sie bereits erfüllt. Außerdem
gefährdete ihre Existenz allmählich die neue TRIZsche Politik. Altshuller
versuchte sogar die Ergebnisse seiner langjährigen Arbeit zu „begraben“, indem
er 1975 verkündete, dass „\emph{die Entstehung des ARIZ-71 praktisch zum
  Wertverlust der Matrix und der 40 Innovationsprinzipien führt}“ [14].

Doch drohte der Prozess des Matrix-Rückzuges aus den Seminarprogrammen und
Lehrgängen sehr beunruhigend zu werden. Abzuwarten, bis sie auf natürliche
Weise aus dem tagtäglichen Betrieb ausscheidet, war auch keine Lösung. Der
Widerspruch wurde auf die (für die TRIZ) übliche Art gelöst: durch das
Separationsprinzip. Den einen (Entwicklern und engen Schülern) wurde – in
Privatbriefen oder -gesprächen – eindeutig zu verstehen gegeben, dass die
Matrix nun nicht mehr zu verwenden sei. Den anderen (einfachen Lesern) wurde
sie – in Büchern und methodischen Materialien – in unveränderter Weise
dargestellt.  Selbstverständlich wurde dies ausschließlich „zum Zwecke der
Popularisierung der TRIZ“ getan.

„\emph{Schon Anfang der 70er Jahre wies Altshuller selbst auf die Veralterung
  der Matrix und der 40 Prinzipien hin. In seinem Brief von 1976 bat er mich,
  die Matrix nicht in meine Vorlesungen zur TRIZ einzubeziehen, denn ‚heute
  erscheint alles anders’}“ – erinnert sich TRIZ-Meister J. Murashkovski aus
Lettland. Für jeden Durchschnitts-TRIZnik wären ähnliche Zweifel „in der
Chefetage“ sehr verwirrend gewesen. Glücklicherweise erreichte die traurige
Neuigkeit nicht das „einfache Volk“.

Diese Geständnisse hatten Altshuller auch nicht davon abgehalten, den
„wertlosen und anders erscheinenden“ Instrumenten über 30 Seiten seines
nächsten Buches „Kreativität als exakte Wissenschaft“ [15] zu widmen – doppelt
so viel wie dem neuen Algorithmus ARIZ-77. Und die Softwareentwickler der
1990er Jahre, denen die TRIZ ein Begriff war, verstanden die Todesgerüchte um
die Matrix als äußerst übertrieben.

Denn trotz allem wählen Hunderte geschäftstüchtiger Problemlöser – nicht nur
unter den Anfängern – mit faszinierender Beständigkeit die „Wertlose“ für die
Ausführung der Hauptrollen in den wichtigsten Kunststücken. Und ein paar
Dutzend „eingeweihter“ Insider beobachten diesen Zirkus und lachen sich leise
ins Fäustchen.

\section*{Der Kaiserin neue Kleider}
Mit dem Einzug in die europäische Arena werden die Matrix und die
Innovationsprinzipien zusammen mit den heute schon gewöhnlichen, doch immer
noch teuren „Erfindungsmaschinen“, auch als Poster, Videoclips, Comics, Dias
und Spielkarten angeboten.

„\emph{Triz-Poker-Workshop mit ein paar Bier, einem Problem, Innovation
  Karten, Riesenspaß und jeder Menge genialer Ideen}“ [16] – die ganze
Ausbildung dauert gerade mal drei Minuten! „\emph{Das Brainstorming wird für
  Sie ein unvergessliches Erlebnis}“, versichern die Spielautoren und raten,
die gewonnenen Fertigkeiten in einem vierstündigen Spaß mit dem Profi für nur
590 Euro zu festigen. Ihre Mission drücken Sie mit einer einfachen Weisheit
aus: „\emph{Menschen sollen das tun können, was sie nicht können, aber gerne
  tun würden, wenn sie wüssten, dass sie es könnten}“.

Was die Menschen nicht können, wäre zum Beispiel ... welch Furcht einflößender
Gedanke!

Unter den Karten-Käufern und Poker-Kunden befinden sich u.a. namhafte
Konzerne.  Die wichtigsten Geschäftspartner im Spiel auf dem europäischen
Markt sind führende europäische TRIZ-Unternehmensberatungen und -Magazine.

Eine glänzende Kritik für das TRIZ-Pokerspiel wurde von der Abteilung des
strategischen Verkaufs eines deutschen Industrieriesen und der Gruppe neuer
Technologieentwicklungen bei Deutschlands größtem wissenschaftlichen Institut
zur Verfügung gestellt. Der Produzent selbst preist die unglaublichen
Einsparungsmöglichkeiten mit Hilfe der Karten eines dreitätigen TRIZ-Seminars:
„\emph{In 3 Minuten können Sie loslegen. Sie sparen sich eine 3-tägige
  TRIZ-Schulung, wie uns das Fraunhofer Institut bestätigte.}“

Man kann sich nur denken, um was für ein Ausbildungsniveau es sich bei den
durch die verehrten Wissenschaftler gelobten Seminaren handelt.

Der Matrix neue Kleider werden auf die Schnelle doch mit Schwung genäht.
Ausbildende Getränke, Kinderspiele und Radioshows sind nur eine Frage der
Zeit. Um aber die Aufmerksamkeit der Käufer auf das neue, preiswerte
TRIZ-Lehrbuch als konventionelles (gedrucktes) Medium zu lenken, unterstreicht
ein solider akademischer Anbieter in seiner Werbung die Beifügung der
Widerspruchsmatrix im DIN A2-Format [17].

In einem seiner veröffentlichten Briefe von 1985 klingt Altshullers Stimme
sehr sicher und fast schon nachsichtig: „\emph{Die 40 Innovationsprinzipien
  haben heute lediglich eine geschichtliche Bedeutung. Wir arbeiten
  hauptsächlich mit den Standards}“ [1].

Ob Genrich S. Altshuller dabei gescherzt oder sich wirklich geirrt hatte? Bloß
nehmen ausgerechnet die Standards und Stoff-Feld-Analyse immer mehr die
geschichtliche Bedeutung an. Ewig lebendig bleiben die vierzig
Innovationsprinzipien und die Matrix.

\section*{Literatur und Quellen}
\begin{itemize}
\item[{[1]}] G.S. Altshuller. Brief vom 31.01.1985.\\
  \url{http://www.altshuller.ru/corr/correspondence1.asp}, (in Russisch).
\item[{[2]}] P. Livotov, D. Murnikov. „Innovation als Prozess“ (Vortrag auf
  dem 4. Europäischen TRIZ-Kongress, Frankfurt a. Main, 30.06-01.07.2005).\\ 
  \url{https://www.researchgate.net/profile/Pavel_Livotov}.
\item[{[3]}] TriSolver4.net Software. „TRIZ-Werkzeuge für Innovation und
  erfinderische Problem\-lösung“.
\item[{[4]}] P. Livotov. „TRIZ im Innovationsprozess“. Konstruktion \&
  Engineering, 03'2004.\\
  \url{https://www.researchgate.net/profile/Pavel_Livotov}.
\item[{[5]}] M. Dworschak. „Zwergenarmeen im Kopf“. Der Spiegel, 30’2005,
  S. 114.
\item[{[6]}] \url{www.triz-online.de} Widerspruchsmatrix.
\item[{[7]}] A. Teufelsdorfer, A. Conrad. „Kreatives Entwickeln und
  innovatives Problemlösen mit TRIZ/TIPS“. Publicis MCD Verlag, Erlangen und
  München 1998.
\item[{[8]}] R. Sietmann. „Erfinden nach Plan“. c't -- Magazin für
  Computertechnik, 23/01, Seite 96.
\item[{[9]}] Forschungsbuero.de. „TRIZ - Lösen von technischen Problemen“.\\
  \url{http://www.forschungsbuero.de/html/triz.htm}.
\item[{[10]}] H.-J. Günther. „Rechnergestützte Bearbeitung von innovativen
  Lösungen mittels der Software ‚TechOptimizer’“.
\item[{[11]}] J.S. Murashkovski. “Wissenschaftliche Innovationsprinzipien“.
  (in Russisch)\\
  \url{http://subscribe.ru/archive/science.natural.triz/200310/03201643.html}.
\item[{[12]}] D. Dürand. „Kind im Manne“. Wirtschaftswoche Nr.19/4.5.200.
\item[{[13]}] C. Gundlach. „Mit Kreativität und Strategie zur
  Nachhaltigkeit“. \\ \texttt{triz-online-magazin.de}, Ausgabe 2005.03.
\item[{[14]}] G.S. Altshuller, G.L. Filkovski. „Moderner Stand der Theorie zur
  Lösung der Erfindungsaufgaben“, 1975.\\
  \url{http://www.altshuller.ru/triz2.asp}, (in Russisch).
\item[{[15]}] G.S. Altshuller. „Kreativität als exakte Wissenschaft: Theorie
  zur Lösung der Erfindungsaufgaben“, Moskau: Sowjet. Radio, 1979, (in
  Russisch).
\item[{[16]}] Instant Innovation TRIZ. „Triz-Solution – Karten“.
  \texttt{etzold.biz}, (nicht mehr online -- HGG).
\item[{[17]}] J.-P. Bläsing, W. Brunner. „TRIZ, Von der Theorie zur Praxis.
  Für TRIZ Moderatoren und TRIZ Teams“. (Mit der Konflikt-Matrix in DIN A2),
  Ausgabe 2001.
\item[{[18]}] B. Gimpel. „Widersprüche auflösen – Produktideen finden“.
  Wirtschaftliche Nachrichten 1/2006, S. 12.
\item[{[19]}] Interquality.de. „Wenn das der Einstein wüßte“.
\item[{[20]}] S. Nessler. „TRIZ -- Erfinden nach Plan“.  Deutschlandradio.
\item[{[21]}] N. Bromberger, K. Brandenburg, B. Dausien u.a. „Neue
  Lerndienstleistungen, Vision und Wirklichkeiten“, Herausgeber:
  Arbeitsgemeinschaft Betriebliche Weiterbildungsforschung e. V.,
  Manuskriptdruck, September 2005, S.48.
\item[{[22]}] Th. Bayer. „TRIZ in der Wittenstein AG“ (Vortrag auf dem
  4. Europäischen TRIZ-Kongress, Frankfurt a. Main, 30.06-01.07.2005).
\item[{[23]}] CREAX Innovation (2. Generate Solution).
\item[{[24]}] „TRIZ“ (Wikipedia). \url{http://de.wikipedia.org/wiki/TRIZ}.
\end{itemize}

    
\section*{Der Autor}

Dipl.-Ing. Leonid Shub, 44, ist Gründungsmitglied des INNOLOGICS e.V. und
verantwortlich für die Anwendung und Schulung in der Theorie des
erfinderischen Problemlösens (TRIZ).  Als Wasseringenieur begann er 1984 bei
dem Norilsker Hüttenkombinat, Norilsk, Russland. Dort hat er an seinen ersten
TRIZ-Seminaren bei B. Zlotin, G. Ivanov und I. Bukhman teilgenommen.

1988--1990 war er Leiter und Referent der TRIZ-Abteilung an der Technischen
Schule für Jugendliche und Studenten in Norilsk.

1995--2001 war er TRIZ-Berater bei Think-Tech GmbH, Kfar-Saba, Israel.

Von 2001 bis 2003 war Leonid Shub Unternehmensberater und TRIZ-Experte bei
Agamus Consult Unternehmensberatung GmbH, Starnberg, tätig und führte dort
Innovationsprojekte.

\end{document}
