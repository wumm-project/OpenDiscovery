\begin{thebibliography}{xxx}
\bibitem{Altshuller1956} Г.С. Альтшуллер, Р.Б. Шапиро. О психологии
  изобретательского творчества.  Вопросы психологии, 1956, № 6, стр. 37-49.
  \url{http://www.altshuller.ru/triz0.asp}
\bibitem{Altshuller1959} Г.С. Альтшуллер, Р.Б. Шапиро. Изгнание шестикрылого
  серафима. Изобретатель и рационализатор, 1959. № 10.
  \url{http://www.altshuller.ru/triz12.asp}
\bibitem{Altshuller1961} Г.С. Альтшуллер. Как научиться изобретать. Тамбов:
  Книжное издательство, 1961.  \url{http://www.altshuller.ru/triz/triz49.asp} 
\bibitem{Altshuller1963} Г.С. Альтшуллер. Как работать над изобретением.
  \emph{Азбука рационализатора}, Сост. Б. Зубков, Ю. Медведев, С. Муслин. --
  Тамбов: Кн. изд-во, 1963.
\bibitem{Altshuller1964} Г.С. Альтшуллер. Основы изобретательства. – Воронеж:
  Центрально-черноземное книжное издательство, Воронеж, 1964.
  \url{http://www.altshuller.ru/triz/ariz64.asp}
\bibitem{Altshuller1965} Г.С. Альтшуллер. Внимание: алгоритм изобретения!,
  Технико-экономические знания: приложение к «Экономической газете»,
  01.09.1965, -- Вып. 27(41).  \url{http://www.altshuller.ru/triz/triz022.asp}
\bibitem{Altshuller1971} Г.С. Альтшуллер.  Алгоритм решения изобретательских
  задач АРИЗ-71. – Баку: ОИИТ при ЦК ЛКСМ Азербайджана и Азербайджанском РС
  ВОИР, 1971. \url{http://www.altshuller.ru/triz/ariz71.asp}
\bibitem{Altshuller1973} Г.С. Альтшуллер.  Алгоритм изобретения, М.,
  «Московский рабочий», 2-е издание, 1973.
  \url{http://www.altshuller.ru/triz/technique2.asp} 
\bibitem{Altshuller1973a} Г.С. Альтшуллер.  Материалы к теме «Типовые приемы
  устранения технических противоречий». (точная дата написания работы
  неизвестна).  \url{http://www.altshuller.ru/triz/technique1a.asp}
\bibitem{Altshuller1974} Г.С. Альтшуллер. Планы занятий на первом курсе
  АзОИИТ.  1973--74. \url{http://www.altshuller.ru/engineering6.asp}.
\bibitem{Altshuller1975} Г.С. Альтшуллер, Г.Л. Фильковский. Современное
  состояние Теории Решения Изобретательских Задач. 1975.
  \url{http://www.altshuller.ru/triz2.asp}.
\bibitem{Altshuller1979} Г.С. Альтшуллер. Творчество как точная наука: Теория
  решения изобретательских задач. М.: Сов. радио, 1979.
\bibitem{Altshuller1980} Г.С. Альтшуллер. Душа обязана …  учиться.
  Литературная газета, 1980, 4 июня (№ 23).
  \url{https://www.altshuller.ru/interview6.asp}
\bibitem{Altshuller1984} Г. Альтов. И тут появился изобретатель.  - М.:
  Дет. лит-ра. - 1984 (1-е изд.); 1987 (2-е изд,); 1989 (3-е изд., перераб. и
  доп.); 2000 (4-е изд.).  \url{http://www.altshuller.ru/biography/)}
\bibitem{Altshuller1985} Г.С. Альтшуллер. Письмо 19.  31.01.1985.
  \url{https://www.altshuller.ru/corr/correspondence1.asp#19}
\bibitem{Altshuller1986} Г.С. Альтшуллер. Жизнь Человека 1-Ч-502, рассказанная
  Игорю Верткину, 1985-1986.
  \url{https://www.altshuller.ru/interview/interview5.asp}.
\bibitem{Altshuller1986a} Г.С. Альтшуллер.  История Развития АРИЗ (конспект).
  1986.  \url{http://www.altshuller.ru/triz/ariz-about1.asp}
\bibitem{Altshuller1994} Г.С. Альтшуллер, И.М. Верткин. Как стать гением:
  Жизненная стратегия творческой личности. -- Минск, Беларусь, 1994.
  \url{http://www.altshuller.ru/trtl/heretic2.asp}
\bibitem{Altshuller1996} Г.С. Альтшуллер. Ответы на вопросы Джеймса Ковалика.
  16.06.1996.  \url{https://www.altshuller.ru/interview/interview4.asp}.
\bibitem{Amnuel964} П. Амнуэль. Старик Жюль Верн и космическая эра. Молодежь
  Азербайджана, Баку 1964.  \url{http://www.fandom.ru/}.
\bibitem{Bachmatov1961} Р. Бахтамов. Изгнание шестикрылого серафима, М.:
  Детгиз, 1961.  
\bibitem{Blaesing2001} Jürgen P. Bläsing, Walter Brunner. TRIZ. Von der
  Theorie zur Praxis. Für TRIZ Moderatoren und TRIZ Teams. Mit der
  Konflikt-Matrix in DIN A2, Ausgabe 2001.
\bibitem{Dworschak2005} Manfred Dworschak. Zwergenarmeen im Kopf. Der
  Spiegel, 30/2005, S. 114.
\bibitem{Filkovsky2006} Г. Фильковский (2006). «Горин -- автор идеи
  физического противоречия ???!!!».
  \url{http://www3.sympatico.ca/karasik/GF_re_gorin_claim.html}
\bibitem{Gimpel2000} Bernd Gimpel, Rolf Herb, Thilo Herb. Ideen finden,
  Produkte entwickeln mit TRIZ. Hanser Verlag, 2000.
\bibitem{Herb2000} Rolf Herb, Thilo Herb, Veit Kohnhauser. TRIZ, der
  systematische Weg zur Innovation. Verlag Moderne Industrie, 2000.deen finden,
  Produkte entwickeln mit TRIZ. Hanser Verlag, 2000.
\bibitem{Jacobson1934}  П.М. Якобсон. Процесс творческой работы изобретателя.
  1934. 
\bibitem{Korneev1962} С.Г. Корнеев, Тайны творчества, Тамбов, 1962,
  Библиотечка новатора.  \url{https://www.metodolog.ru/00696/00696.html} 
\bibitem{Korneev1964} С.Г. Корнеев, Алгебра и гармония, Тамбов, 1964.
  (Библиотечка новатора; Вып. 2),
  \url{http://www.metodolog.ru/00630/00630.html}
\bibitem{Korolyev1998} В.А. Королёв. Современные тенденции развития АРИЗ.
  25.01.1998. \url{http://www.triz.org.ua/data/w55.html}
\bibitem{Kudryavtsev} А.В. Кудрявцев. Как выбирать приемы для решения.
  Учебник по ТРИЗ. гл. 8.  \url{http://metodolog.ru/00088/00088.html}
\bibitem{Leon2005} Noel Leon, Jose Jesus Martinez, Carlos Castillo.
  Methodology for the Evaluation of the Innovation Level of Products and
  Processes. TRIZ Journal 10/2005.
\bibitem{Livotov2004} P. Livotov. TRIZ im Innovationsprozess. Konstruktion \&
  Engineering, 03/2004.
\bibitem{Livotov2005} P. Livotov, D. Murnikov. Innovation als Prozess. 
  4. TRIZ-Kongress, Frankfurt, 29. Juni 2005. 
\bibitem{Mann2005a} Darrell Mann, Conall Ó Catháin.  Using TRIZ in
  Architecture: First Steps. TRIZ Journal 11/2005.
\bibitem{Mann2005b} Darrell Mann, Chris Bradshaw. Design for Wow 2 – Music.
  TRIZ Journal 10/2005.
\bibitem{MATRIZ2003} МА ТРИЗ. Проект Положения о многоступенчатой аттестации
  пользователей и сертификации специалистов Международной Ассоциации ТРИЗ.
  Принято на заседании Президиума МА ТРИЗ, 26 июня 2003 г.
  \url{https://triz-summit.ru/triz/history/300029/matriz-2003/300314/300315/}.
\bibitem{Murashkovsky2003} Письмо от Ю.С. Мурашковского. 03.10.2003.
  \url{https://subscribe.ru/archive/science.natural.triz/200310/03201643.html}.
\bibitem{Murashkovsky2006} Ю.С. Мурашковский, Детская болезнь
  «ниспровержизма». 22.03.2006, не опубликован.
  \url{https://subscribe.ru/archive/science.natural.triz/200310/03201643.html}.
\bibitem{OrlovNN} В. Орлов, Трактат о вдохновении, рождающем великие
  изобретения.  \url{https://www.metodolog.ru/00193/00193.html}
\bibitem{Povileiko1977} Р.П. Повилейко. Инженерное творчество, М.: Знание,
  1977.
\bibitem{Salamatov1992} Ю.П.Саламатов. Современное состояние и проблемы
  развития ИМ-ТРИЗ-технологии. Минск, 19-21 мая 1992 г.
  \url{http://www.triz-guide.com/publicat/articles/article7.html}
\bibitem{Shub2004} Л. Шуб. Особенности распространения и видоизменения ТРИЗ в
  центральной Европе (1998--2004).
  \url{http://metodolog.ru/00452/00452.html}.
\bibitem{Shub2006} Л. Шуб. Сравнительная таблица изменений и перемещений в
  ранних версиях АРИЗ. \url{https://metodolog.ru/00908/00908.pdf}
\bibitem{Sietmann2001} Richard Sietmann. Erfinden nach Plan. c't -- Magazin
  fur Computertechnik 23/01, Seite 96.
\bibitem{Stuart2005} Jack Stuart. Transactional TRIZ, Theory, Application, and
  Execution, Part II: The Contradiction Matrix. TRIZ Journal 10/2005.
\bibitem{Teufelsdorfer1998} A. Teufelsdorfer, A.  Conrad. Kreatives Entwickeln
  und innovatives Problemlösen mit TRIZ/TIPS, Publicis MCD Verlag, Erlangen
  und München 1998.
\bibitem{Terninko1998} John Terninko, Alla Zusman, Boris Zlotin. TRIZ -- Der
  Weg zum konkurrenzlosen Erfolgsprodukt. Verlag Moderne Industrie, 1998.
\bibitem{ZlotinZusmanNN} Б. Злотин, А. Зусман, «О попытке документирования
  истории ТРИЗ»,
  \url{http://www.trizscientific.com/TRIZ_sci/history/history_main_r.htm}
\end{thebibliography}

