\documentclass[11pt,a4paper]{article}
\usepackage{od}
\usepackage[utf8]{inputenc}
\usepackage[russian,ngerman]{babel}

\title{System der Gesetzmäßigkeiten des Aufbaus und der Entwicklung
  technischer Systeme }

\author{B.I.Goldovsky}
\date{1983}

\begin{document}
\maketitle
\begin{quote}
  Original: \foreignlanguage{russian}{Система закономерностей построения и
    развития технических
    систем}\footnote{\url{https://triz-summit.ru/file.php/id/f303253-file-original.pdf}.}.
  Übersetzt von Hans-Gert Gräbe, Leipzig.
\end{quote}

\section*{Vorwort}

Die folgende Übersicht der Gesetze, welche den Aufbau und die Entwicklung
technischer Systeme bestimmen, wurde im Juni 1981 entwickelt und im Mai 1983
überarbeitet. Der Inhalt dieses Dokuments wurde auf der Regionalkonferenz
„Probleme der Entwicklung wissenschaftlich-technischer Kreativität
ingenieur-technischer Arbeit (ITR)“ in Gorki im Jahr 1983 vorgestellt und
wurde auch bei der Erstellung der Kapitel 2 und 4 des Buches „Rational
Creativity“\footnote{\foreignlanguage{russian}{Б.И.Голдовский. «Рациональное
    творчество», М.: Речной транспорт, 1990}}. verwendet.

\begin{flushright}
  B. I. Goldovsky
\end{flushright}

\section*{I. „Grundlegende“ Muster}
(aus denen sich die Konstruktions- und Entwicklungsgesetze technischer Systeme
ableiten lassen) 
\subsection*{A. Gesetze der Dialektik}
\begin{itemize}\itemsep0pt
\item[1.1.] Das Gesetz von Einheit und Kampf der Gegensätze
\item[1.2.] Das Gesetz der Negation-Negation
\item[1.3.] Das Gesetz des Umschlags von Quantität in Qualität
\end{itemize}
\subsection*{B. Systemweite Gesetze, Naturgesetze}
\begin{itemize}\itemsep0pt
\item[1.4.] Redundanz in Systemen -- jedes System hat versteckte
  Eigenschaften, die in der gegebenen Struktur des Systems und des
  Obersystems nicht realisiert (nicht verwendet) werden
\item[1.5.] Wachstum der Systemintegrität während seiner Entwicklung
\item[1.6.] Jedes selbstorganisierende System ist bestrebt, das Gleichgewicht,
  die Stabilität aufrechtzuerhalten (Prinzip der \emph{Homöostasis})
\item[1.7.] Offene Systeme (die mit der Umgebung interagieren) können nur
  existieren, wenn sie Materie, Energie und Information mit der Umwelt [Dies
    kann als Naturgesetz betrachtet werden]
\item[1.8.] Spezialisierung von Systemen im Entwicklungsprozess als Weg zur
  Steigerung der funktionalen Effizienz
\item[1.9.] Mehrdeutigkeit von Zusammensetzung (Bewegungsform der Materie),
  Struktur und Parametern der internen Funktionsweise für die Funktionsweise
  und Entwicklung des Systems
\item[1.10.] Entsprechung zwischen Komplexität und Integrität des Systems
  und den Kosten für Änderungen des Systems oder seiner Teile
\item[1.11.] Künstlich geschaffene Systeme müssen kompatibel mit anderen
  Systemen und mit der Umwelt sein
\item[1.12.] Entsprechung zwischen der Komplexität der Funktionen und der
  Komplexität der Struktur des Systems
\item[1.13.] Die Entsprechung zwischen der Komplexität des Systems (Formen der
  Bewegung der Materie und seiner Struktur) und den Möglichkeiten seiner
  Entwicklung
\item[1.14.] Die funktionelle Breite des Systems (die Anzahl und Vielfalt der
  Funktionen) muss der Vielfalt der Anforderungen der Umwelt entsprechen
  (Vielfalt der Funktionsbedingungen)
\item[1.15.] Die bestimmende Rolle der Bewegungsform der Materie in Funktion
  und Entwicklung des System. Die Kompliziertheit der Bewegungsform der
  Materie wächst im Verlauf der Systementwicklung [Naturgesetz]
\end{itemize}
\subsection*{C. Soziale (und ökonomische) Gesetze}
\begin{itemize}\itemsep0pt
\item[1.16.] Die bestimmende Rolle der Erkenntnis, wenn es um die technische
  Verwendungsmöglichkeit der einen oder anderen Form der Bewegung der Materie
  geht
\item[1.17.] Soziale Bedürfnisse wachsen schneller als die Möglichkeiten, sie
  zu realisieren
\item[1.18.] Endlichkeit (Begrenztheit) der Ressourcen und Möglicchkeiten der
  Gesellschaft, der Natur und des Menschen (prinzipiell, für einen bestimmten
  Zeitpunkt)
\item[1.19.] Prinzipielle Unendlichkeit (Unbegrenztheit) der Bedürfnisse der
  Gesellschaft
\item[1.20.] Die bestimmende Rolle der Ökonomie bei der Realisierung dieses
  oder jenes technischen Systems
\item[1.21.] Mehrdeutigkeit der qualitativ unterschiedlichen Bedürfnisse und
  Aufwendungen für die Gesellschaft
\item[1.22.] Einfluss des Entwicklungsstands der Wissenschaft auf Wege und
  Formen der Entwicklung technischer Systeme (je weniger Wissen, desto mehr
  Fehler)
\item[1.23.] Kontinuierliches Wachstum (qualitativ und quantitativ) der
  gesellschaftlichen Bedürfnisse [Ableitung von 1.19]
\item[1.24.] Kontinuierliches Wachstum der allgemeinen Effizienz der
  gesellschaftlichen Produktion [Folge des Widerspruchs zwischen 1.18 und
    1.19]
\item[1.25.] Zusammenhang zwischen dem Wert für die Verbraucher (der
  Werthaltigkeit eines Bedürfnisses für Gesellschaft) und den daruf
  verwendeten Aufwendungen [Folgerung von 1.21]
\item[1.26.] Jegliche Ökonomie reduziert sich am Ende auf eine Ökonomie der
  Zeit.
\item[1.27.] Vorrang der Befriedigung gesellschaftlicher Bedürfnisse vor der
  Erhöhung der allgemeinen Effizienz der Produktion [Folgerung 1.21]
\item[1.28.] Die entscheidende Rolle der allgemeinen Entwicklung des Standes
  der Technik in der Möglichkeit der Realisierung technischer Systeme
\end{itemize}
\section*{II. „Methodologische“ Muster der Entwicklung technischer Systeme}
(die allgemeinsten)
\begin{itemize}\itemsep0pt
\item[2.1.] Die treibende Kraft der Entwicklung der Technologie ist der
  Widerspruch zwischen dem öffentlichen Bedürfnis und der Möglichkeit, diese
  mit den verfügbaren technischen Mitteln zu befriedigen
  (gesellschaftlich-technischer Widerspruch) [Manifestation von 1.1, Folgerung
    aus 1.17 und dem Widerspruch zwischen 1.18 und 1.19]
\item[2.2.] Die Entwicklung von technischen Systemen erfolgt durch
  Verschmelzung von Gegensätzen vom Typ „Quelle der Wirkung -- Target der
  Wirkung“ (Werkzeug -- Produkt; Energiequelle -- Energieempfänger; steuernd
  -- gesteuert, etc.) in einem Element (Universalisierung) und die Aufspaltung
  eines Elements in diese Gegensätze (Spezialisierung) [Manifestation von 1.1]
\item[2.3.] Jedem technischen System ist ein bestimmter Satz technischer
  Widersprüche eigen, deren Zuspitzung oder Auflösung die Entwicklung dieses
  technischen Systems bestimmt [Manifestation von 1.1, einer der Gründe für
    das Auftreten der Widersprüche 2.1]
\item[2.4.] Die Änderung des technischen Systems bei der Auflösung von
  technischen Widersprüchen im Entwicklungsprozess erfolgt durch Inversion
  eines der wesentlichen Merkmale des technischen Systems, welches für die
  Existenz und Zuspitzung des technischen Widerspruchs verantwortlich ist
  [Manifestation von 1.2, Form der Lösung des Widerspruchs 2.3]
\item[2.5.] Im Prozess der Entwicklung eines technischen Systems erfolgt eine
  Rückkehr zu früheren Formen auf einer neuen Ebene (Entwicklung „in einer
  Spirale“) in dem Umfang, wie neue Mittel zur Lösung des technischen
  Widerspruchs erscheinen, deren Zuspitzung zur Ablehnung dieser Formen
  geführt hat [Manifestation von 1.2, Folgerung 2.4 und 2.3, ist assoziiert
    mit 2.13]
\item[2.6.] Mit der Entwicklung von technischen Systemen steigt der Wert der
  Seiten, welche die Verbesserung der Wechselwirkungen mit der natürlichen
  Umwelt befördern [Folgerung 1.5]
\item[2.7.] Das Obersystem ist konservativer als das technische System (in der
  Regel fast immer, insbesondere auf der Ebene von Selbstorganisation --
  Unternehmen, Industriezweig usw.) [Folgerung 1.6 und 1.10; je komplizierter
    die natürliche Umgebung, umso gerechter]
\item[2.8.] Jeder primären nützlichen Funktion (Bedarf) entspricht eine
  bestimmte Gesamtheit funktionaler Parameter, die eine
  „funktional-parametrische“ Nische bilden (bestimmen) [Manifestation von 1.3,
    Folgerung 2.20]
\item[2.9.] Der stabile Zustand eines technischen Systems entspricht dem
  Ausfüllen einer bestimmten „funktional-parametrischen Nische“ (fehlende
  „Konkurrenz“) [Analogie zu Biologie, Folgerung 2.20, entfernte Folgerung von
    1.6 und 2.8] 
\item[2.10.] Quantitative Veränderungen im technischen System führen
  zwangsläufig zu einer solchen qualitativen Änderung wie etwa der Zuspitzung
  von technischen Widersprüchen [Manifestation von 1.3, Wirkung auf 2.2]
\item[2.11.] Die sprunghafte Änderung der Parameter der äußeren Funktionalität
  des technischen Systems erfordert den qualitativen Umbau des Systems
  [Manifestation von 1.3]
\item[2.12.] Die bestimmende Rolle des physikalischen Wirkprinzips bei der
  Funktionsweise und Entwicklung technischer Systeme [Manifestation von 1.15]
\item[2.13.] Die steigende Kompliziertheit der Bewegungsformen der Materie im
  technischen System im Zuge seiner Entwicklung [Form der Manifestation von
    1.15, Folgerung von 2.12, Zusammenhang mit 2.5; Form der Lösung des
    Konflikts zwischen 5.6 und 2.23, Verbindung mit 5.7; ist zur gleichen Zeit
    eine Gesetzmäßigkeit der Änderungen der Zusammensetzung technischer
    Systeme]
\item[2.14.] Es gibt eine Hierarchie des Optimums:
  \begin{itemize}\itemsep0pt
  \item für die „funktional-parametrische Nische“ (Summe der Parameter der
    äußeren Funktionalität) -- das optimale physikalische Prinzip; 
  \item für das physikalische Prinzip -- die optimale Struktur des Systems;
  \item für die Struktur -- die optimalen Parameter des inneren Funktionierens
    des Systems
  \end{itemize}
  [heuristische Setzung, Folgerung von 1.9, 1.15 und 2.12]
\item[2.15.] Vorhandene entwickelte Systeme absorbieren für die Entwicklung
  mehr Ressourcen als neue, im Entstehen begriffene (bei ersteren sind die NS
  weiter entwickelt)

  (HGG: Das Folgende ist unverständlich)  
  \begin{itemize}
  \item Die dominierende Absorption von Ressourcen durch diese technischen
    Systeme
  \item Die Realisierung einer beliebigen der Möglichkeiten der Entwicklung
    eines Systems begrenzt die Wahrscheinlichkeit der Realisierung anderer
    Möglichkeiten
  \end{itemize}
  [Folgerung 1.6, 1.18 und 2.7]
\item[2.16.] Das Prinzip minimaler Änderungen des Systems im
  Entwicklungsprozess:
  \begin{center}
  Interne Funktionsweise $\iff$ Struktur $\iff$ physikalisches Wirkprinzip

  Teil $\iff$ Summe aller Teile $\iff$ das Ganze
  \end{center}
  [Folgerung von 2.15 sowie 1.6, 1.18 und 2.7; Form der Konfliktlösung
    zwischen 1.6 und 2.22-2.23]
\item[2.17.] Das Prinzip der schrittweisen Optimierung des Systems (tritt nach
  jeder Änderung des Systems auf der entsprechenden Ebene auf):
  \begin{center}
  Suche nach dem optimalen physikalischen Prinzip $\Longrightarrow$
  Suche nach der optimalen Struktur $\Longrightarrow$
  Suche nach optimalen Parametern für die interne Funktion
  \end{center}
      [Folgerung 2.14 und 2.16]
\item[2.18.] Einfluss der Entwicklung der einen Systeme (der einen
  Technologiebereiche) auf andere Systeme (Bereiche):
  \begin{itemize}
  \item  Übertragung von Lösungen
  \item „Ausbreitung“ des physikalischen Wirkprinzips in verschiedene
    Industriezweige
  \end{itemize}
[Folgerung von 2.15 und 2.12 sowie der Invarianz von Funktionen in Bezug auf
  verschiedene technische Bereiche]
\item[2.19.] Der nützliche Output (Fähigkeiten, primäre nützliche Funktion)
  des Systems soll den (sozialen oder technischen) Bedürfnissen der Umgebung
  entsprechen [Folgerung von 1.11; Erste Akzeptanzbedingung des Systems]
\item[2.20.] Die erforderlichen Inputs (Bedürfnisse) des Systems dürfen die
  Fähigkeiten (Ressourcen) der Umgebung nicht überschreiten [Folgerung von
    1.11; Zweite Akzeptanzbedingung des Systems]
\item[2.21.] Unnützer (schädlicher) Output des Systems muss für die Umgebung
  zulässig (zumutbar) sein [Folgerung 1.11; Dritte Akzeptanzbedingung des
    Systems]
\item[2.22.] Kontinuität des Wachstums der funktionalen Effizienz des Systems
  (in erster Linie der Parameter der Output-Funktionen) [Folgerung 1.23 und
  2.19]
\item[2.23.] Kontinuierliches Wachstum der allgemeinen Effizienz des Systems
  (Streben nach dem absolut idealen System, Erhöhung des Grades der Idealität,
  Reduzierung der Redundanz des Systems) [Folgerung von 1.24, 2.19 und 2.20,
    teilweise Folgerung von 2.21; Form der Konfliktlösung zwischen 2.22 und
    1.18]
\item[2.24.] Verdrängung des Menschen aus technischen Systemen. „Die
  menschliche Arbeit wird zunehmend ersetzt durch die Arbeit von Maschinen“ in
  der Ausführung von Funktionen in den Bereichen
  \begin{itemize}
  \item Transport
  \item Energie
  \item Technologie
  \item Kontrolle und Regulation
  \item Entscheidungsfindung
  \end{itemize}
[Korollar 2.22 und 1.18 -- Endlichkeit, begrenzte Fähigkeit des Menschen;
  entspricht der sozialen Bedeutung der Technik; gleichzeitig Gesetzmäßigkeit
  der Änderung der Zusammensetzung des Systems, eine der Manifestationen von
  6.2]
\item[2.25.] Übereinstimmung zwischen der Möglichkeit der Realisierung des
  komplexen Teils des Systems und der Komplexität des Systems selbst (ein
  komplexes Teilsystem gehört normalerweise zu einem komplexen System; eine
  komplexe Technik erfordert eine komplexe Organisation.)  [Folgerung aus
    1.28, teilweise Folgerung aus 1.12]
\item[2.26.] Die Entsprechung zwischen der Wichtigkeit einer Funktion und dem
  Aufwand dafür (je wichtiger, desto mehr Nebenkosten) [Folgerung 1.25,
    teilweise 2.19]
\item[2.27.] Vorrang der funktionalen Effektivität des System vor der
  allgemeinen Effektivität (Idealität): 
  \begin{itemize}
  \item Eine Funktion zu erfüllen ist wichtiger als die Intensität zu erhöhen.
  \item Die Steigerung der Intensität ist wichtiger als die Steigerung der
    Idealität.
  \item Die Steigerung der Idealität des Prozesses ist wichtiger als die
    Steigerung der Idealität des technischen Mittels.
  \end{itemize}
  [Folgerung 1.27 und 1.26]
\item[2.28.] Änderung der Verwendung eines physikalischen Prinzips während der
  Entwicklung des technischen Systems:
  \begin{itemize}
  \item Überwindung einer parametrischen Schwelle (Maximum eines funktionalen
    Parameters)
  \item Steigerung der allgemeinen Effizienz (Wirkungsgrad, Qualität)
  \item Beseitigung von unerwünschten Wirkungen (schädlicher Output)
  \end{itemize}
[Folgerung 2.27 und 1.21 zur Gewährleistung der Bedingungen 2.19, 2.20 und
  2.21]
\item[2.29.] Zulässigkeit einer Verschlechterung im System bis zu einer
  bestimmten Schwelle [Folgerung 2.3, 2.10 und 1.21, Form der Manifestation
    1.3]
\end{itemize}
\section*{III. Gesetzmäßigkeiten des Baus arbeitsfähiger technischer Systeme} 
(für alle technischen Systeme)
\begin{itemize}
\item[3.1.] Die funktionale Vollständigkeit des Systems muss gewährleistet
  sein (die Summe des ESF muss die Erfüllung des GPF sicherstellen -- HGG: was
  die Abkürzungen auch immer bdeutuen mögen) [Folgerung 2.19 sowie die
    Konsistenz unserer Welt]
\item[3.2.] Die Energiedurchlässigkeit des Systems muss gewährleistet und
  dementsprechend die Vollständigkeit der energetischen Ketten, die das
  Funktionieren sichern [Folgerung 1.7, teilweise 3.1; Einfluss von 3.3 in
    Bezug auf die Intensität des Energieaustauschs und von 3.4 in Bezug auf
    Dynamik und Steuerbarkeit von Energieketten]
\item[3.3.] Es muss die Überwindung der für ein System charakteristischen
  Parameterschwelle gesichert sein (dementsprechend muss eine gewisse
  Intensität des Energieaustauschs gesichert sein) [Folgerung 1.3 und 2.8;
    teilweise 2.22 und 3.1]
\item[3.4.] Ein minimal erforderliche Maß an Steuerbarkeit (Variabilität,
  Dynamik) des Systems und seiner Teile muss gewährleistet sein (entsprechend
  muss ein gewisses Maß an Dynamik und Steuerbarkeit der Energieketten
  gewährleistet sein) [Folgerung 1.14, teilweise 3.1, eine der
    Umsetzungsformen 2.23]
\end{itemize}
\section*{IV. Gesetzmäßigkeiten von Änderungen im Funktionieren des Systems}
\begin{itemize}
\item[4.1.] Das Bestreben des Wirkprinzips des Systems zum Eindringen in
  benachbarte funktionale Nischen [Folgerung 1.23, 2.22 und 2.15 auf Grund von
    1.4]
\item[4.2.] Wachsende Spezialisierung des Systems im Entwicklungsprozess
  (Einschränkung der Funktion und Sicherstellung der Konstanz der
  Funktionsbedingungen auf Kosten des Obersystems) [Folgerung 1.14, Formen der
    Manifestation von 1.8 und 2.2; Form der Lösung des technischen
    Widerspruchs zwischen nützlichen Outputs; Zusammenhang zu 4.4]
\item[4.3.] Erhöhung der Vielseitigkeit des Systems und seiner Elemente (bei
  Stabilität des Obersystems) [Form der Manifestation 2.2; Folge-Mittel zur
    Bereitstellung von 2.23; Form der Auflösung technischer Widersprüche,
    Folgerung 4.4]
\item[4.4.] Begrenzte Änderungen des Obersystems, um die Rentabilität
  (Effektivität) einer engen Spezialisierung des Systems sicherzustellen.
  [Folgerung 1.18, 1.6 und 2.7; Verbindung mit 4.2]
\end{itemize}
\section*{V. Gesetzmäßigkeiten der Änderung der Struktur technischer Systeme}
\begin{itemize}
\item[5.1.] Ungleichmäßige Entwicklung von Systemteilen -- Die Änderungsraten
  verschiedener Teile sind verschieden [Folgerung 1.21, 1.25 und 2.26 sowie --
    hauptsächlich -- die Inhomogenität realer Systeme; führt zu einer realen
    (tatsächlichen, beobachtbaren) Zuspitzung technischer Widersprüche --
    Verbindung mit 2.3]
\item[5.2.] Inharmonische Entwicklung der Systemteile -- die einen Teile
  überholen andere wesentlich in ihrer Entwicklung [Folgerung 1.21, 1.25, 2.26
    und 2.29; Form der Lösung des Widerspruchs zwischen 1.18 und 1.19;
    Verbindung mit 2.3 -- wie mit 5.1]
\item[5.3.] Vergrößerung der Anzahl und Heterogenität der Wechselbeziehungen
  zwischen den Elementen des Systems (Obersystems) [Folgerung 4.3 und Mittel
    zur Umsetzung von 2.23]
\item[5.4.] Wachstum des Dynamismus im System (bei gegebener funktionaler
  Breite).

  Stufen des Wachstums des Dynamismus:
  \begin{itemize}
  \item[1)] Mindestniveau entsprechend dem Änderungsbereich der
    Betriebsbedingungen (Leitwerk am Flugzeug)
  \item[2)] Veränderung der Interaktion im Laufe der Zeit, um eindeutig
    schädliche Outputs zu beseitigen, die einer Intensivierung im Weg stehen
    [2.22] (einziehbares Fahrgestell)
  \item[3)] Änderung von Eigenschaften und Wechselwirkungen von Elementen, um
    die Erhöhung der allgemeinen Effizienz, Qualität, Funktionsweise, Senkung
    des Ressourcenverbrauchs im Obersystem etc. zu sichern.  [2.23]
    (Mechanisierung der Flügel)
  \item[A)] Einfacher Dynamismus -- Verlagerung einelner Systemteile im Raum 
  \item[B)] komplexer Dynamismus:
    \begin{itemize}
    \item[a)] Änderung der Form, der Konfiguration
    \item[b)] Änderung der inneren Eigenschaften (z.B. Aggregatzustand)
    \item[c)] Austausch mit der Umwelt (Müll + Regeneration)
    \end{itemize}
  \end{itemize}
[Folgerung 4.3 und 2.28; Mittel zur Realisierung von 2.23]
\item[5.5.] Erhöhung der Variabilität von Elementen und Beziehungen im System:
  \begin{itemize}
    \item Zerkleinern + Kombinieren
    \item Flexibilität der Verbindungen (mechanisch -- hydraulisch,
      pneumatisch -- Felder)
  \end{itemize}
  [Folgerung 5.3 und 5.4, eine der Folgerungen von 2.13]
\item[5.6.] Wachsende Kompliziertheit des Systems im Prozess seiner
  Entwicklung (Wachsende Kompliziertheit der Struktur) [Folgerung 5.3, 5.5,
    4.3 und 5.17]
\item[5.7.] Die Beschränkung des Wachstums der Kompliziertheit der Struktur
  des Systems bei gegebenem physikalischem Prinzip (Existenz einer Schwelle
  für die Kompliziertheit der Systemstruktur) [Folge der Zuspitzung der
    technischen Widersprüche im Bereich der Zuverlässigkeit, eine der
    Manifestationen von 2.12, entfernt Folgerungen von 5.6 und 2.3]
\item[5.8.] Wachstum der Integrität des Systems (Obersystems) im Prozess
  seiner Entwicklung:
  \begin{itemize}
    \item Erhöhung der Starrheit der Beziehungen zwischen System und
      Obersystem (Ersetzung organisatorischer Wechselbeziehungen durch
      physische); die Steifheit von Bindungen „schleicht sich“ entlang der
      Systemhierarchie nach oben.
    \item Zunahme der Abhängigkeit des Systems vom Obersystem und des
      Obersystems vom System, Abnahme des Grad der Unabhängigkeit von Teilen
      (einschließlich der energetischen Unabhängigkeit)
  \end{itemize}
[Form der Manifestation 1.5, Form der Auflösung des Widerspruchs zwischen 4.2
  und 4.3]
\item[5.9.] Übergang der Entwicklung ins Obersystem, wenn die Entwicklung des
  Systems beendet ist (anwendbar bei allen Verbot des weiteren Entwicklung des
  Systems) [Folge und eine der Formen von 5.8, Form der Auflösung des
    Widerspruchs zwischen der Notwendigkeit, 2.22 und 2.23 zu sichern, und dem
    Systemänderungs-Verbot aufgrund der Zuspitzung des technischen
    Widerspruchs -- 2.3]
\item[5.10.] Erhöhung der Nutzung der Umwelt (zur Steigerung der Effizienz des
  Systems wird die \emph{Offenheit} des Systems erhöht):
\begin{itemize}
  \item Nutzung von Stoffen ist eine der Bedingungen der Existenz der
    menschlichen Zivilisation
  \item Nutzung von Energie 
    \begin{itemize}
    \item in der Anfangsphase der Entwicklung Transformation nach Programm
      (Nutzung natürlicher Energiequellen -- Unregelmäßigkeit, Unmöglichkeit
      der Konzentration, parametrische Grenzen)
    \item in folgenden Stufen Transformation nach der Energieart (feine
      Prozesse auf der Mikroebene, Fotozellen, Kernfusion usw.)
    \end{itemize}
\end{itemize}
[eine der Folgerungen 5.8 und 2.23 in Bezug auf die Umwelt; Form der Auflösung
  technischer Widersprüche]
\item[5.11.] Reduzierung der Verwendung von Ressourcen aus der Umgebung
  (Zunahme der \emph{Geschlossenheit} des Systems):
  \begin{itemize}
  \item zur Überwindung natürlicher Grenzen 
  \item zur Beseitigung ökologischer schädlicher Effekte
  \item zur Steigerung der Effizienz des Systems bei der Nutzung von Abfällen
    (insbesondere wenn die natürlichen Ressourcen erschöpft sind)
  \end{itemize}
[eine der Folgerungen 5.8 und 2.21, Folgerung 2.23. Form der Konfliktlösung
  zwischen 1.18 und 2.22, Form der Lösung technischer Widersprüche]
\item[5.12.] Elimination von Zwischenketten in Systemem und Prozessen
  [Folgerung 2.2 und 2.23]
\item[5.13.] Einführung von Zwischenketten in Systemen und Prozessen
  [Folgerung 2.2, Form der Lösung technischer Widersprüche; um Widersprüche
    mit 5.12 zu beseitigen, Form 2.23 verwenden]
\item[5.14.] Beseitigung von Funktionsstörungen des Systems sowie von
  Zwischenstufen (Erhöhung der Prozesskontinuität) [eine der Formen 5.12,
    Folgerung 2.23]
\item[5.15.] Übergang zur direkten Interaktion (Einwirkung), Reduktion von
  Energiekettenlängen [eine der Formen 5.12. Folgerung 2.23 und 4.3]
\item[5.16.] Übergang zur indirekten Interaktion, zur Diskontinuität [eine der
  Formen 5.13, um den Widerspruch 5.14 und 5.15 zu beseitigen, werden die
  Formen 2.23 verwendet]
\item[5.17.] Erhöhung der relativen Eigenständigkeit von Systemteilen
  (Verschiebung von „Primärenergieketten“ in Systemteile -- Motoren,
  Formwandler; wird meist von einer wachsenden Kompliziertheit der
  Bewegungsformen der Materie begleitet - Energiearten verzweigen sich in
  lokalen Energieketten) [Folgerung 5.15, Form der Realisierung der
    Anforderungen von 5.4 sowie 2.13; Form der Lösung technischer
    Widersprüche]
\item[5.18.] Raum durch einen nützlichen Prozess ausfüllen (Punkt -- Linie --
  Fläche -- Volumen) [Form der Umsetzung der Anforderungen 2.22 und 2.23]
\item[5.19.] Reduzierung des Platzbedarfs von Hilfselementen [Folgerung 2.23,
  Form der Lösung technischer Widersprüche]
\item[5.20.] Reduzierung des Platzbedarfs des Hauptprozesses (Konzentration
  des Prozesses im Raum) [Form der Konfliktlösung zwischen 1.18 und 2.22]
\end{itemize}
\section*{VI. Muster von Änderungen in der Zusammensetzung des Systems}
\begin{itemize}
\item[6.1*.] Wachsende Kompliziertheit der Bewegungsform der Materie im System
  im Zuge seiner Entwicklung [wie 2.13]
\item[6.2*.] Die Verdrängung des Menschen aus technischen Systemen im Prozess
  der Entwicklung des Systems [wie 2.24]
\item[6.1.] Die Einbeziehung des Menschen in neu geschaffene technische
  Systeme [Folgerung 1.20 und 1.28]
\item[6.2.] Erhöhung des Grads der Artifizialität von System-Elementen im
  Prozess der Entwicklung [eine der Konsequenzen von 5.11, Form der
    Manifestation 6.2*]
\item[6.3.] Einheit von Heterogenität und Homogenität der System-Elemente
  (Additive, Komposite) [Form der Manifestation 2.2, eine der Formen der
  Realisierung von 6.2, entfernte Konsequenz 2.23]
\item[6.4.] „Hybridisierung“ von Systemen an den Grenzen einer funktionalen
  Nische.  Entstehen von Übergangsformen [Konsequenz und eine Form der
    Realisierung von 4.1]
\item[6.5.] Komplexe Nutzung der Bewegungsformen der Materie. Einbeziehung
  niederer Formen der Materiebewegung in Systeme mit höheren Formen der
  Materiebewegung.  [eine Form der Lösung von Widersprüchen zwischen 1.18 und
    2.22 im Gegensatz zu 6.1]
\end{itemize}

\end{document}
