\documentclass[11pt,a4paper]{article}
\usepackage{od}
\usepackage[utf8]{inputenc}
\usepackage[main=russian,ngerman]{babel}

\title{Zwei Methodologie-Projekte:\\ G.P. Shchedrovitsky und M.A. Rozov}

\author{N.I. Kuznecova, Moskau}
\date{2015}

\begin{document}
\maketitle
\begin{quote}
  Original: \foreignlanguage{russian}{ Два проекта методологии:
    Г.П. Щедровицкий и М.А. Розов // Х Сократические Чтения. Реальность как
    социальные эстафеты (памяти М.А. Розова). Сб. докл. /
    Отв. ред. В.А. Шупер. М.: ЭСЛАН, 2015. С. 12–32.}
  
  Quelle: \url{https://rozova.org/images/documents/KuznecovaDvaProekta.pdf}

  Mehr zur Autorin (in Russisch)
  \url{https://vovr.elpub.ru/jour/pages/view/Kuznetsova}

  Übersetzt von Hans-Gert Gräbe, Leipzig. 
\end{quote}

Словосочетание, которое сегодня привычно выделяет особую, самостоятельную
область философских исследований — «методология и философия науки» — отнюдь не
так давно стало общеупотребительным. В отечественной литературе про
методологические проблемы науки, про методологию в целом в 1960–1970-е
гг. говорили столь часто и столь широко, что эта тематика казалась
исключительно навязчивой и откровенно навязанной партийными идеологическими
предписаниями. Методология должна быть правильной, следовательно, —
«марксистско-ленинской». Казалось, что участники бесконечных конференций,
семинаров и кулуарных дискуссий, а также авторы соответствующих сборников и
монографий, никогда не сумеют найти общего языка и никогда ясно не
сформулируют своих исходных позиций. Кроме того, вплоть до конца 1980-х
годов часто утверждали, что «эпистемология» и «философия науки» — это вообще
лишние термины, без нужды «умножающие сущности», так как в философии издревле
существовала сфера гносеологических (теоретико-познавательных) исследований.

История философско-методологических поисков ХХ столетия — как в западной,
так и в отечественной традициях — представляет весьма интересный и яркий
материал для анализа, осмысления и попыток представить следующие шаги и
проекты в этой сфере.

ПРИКЛЮчЕНИЯ «МЕТОДОЛОГИИ»: ОТ HARD К SOFT METHODOLOGY

«Методология» — слово крайне ответственное и, к сожалению, очень неудачное.

Конечно, это слово ответственное, оно стало таким еще со времен Фр. Бэкона,
который провозгласил необходимость «Нового Органона», и «Рассуждения о
методе» Декарта. Юный Герцен в своей выпускной университетской работе,
анализируя Коперниканскую астрономическую революцию, напоминает, что великий
Декарт говорил, когда превозносили его математические открытия: «Хвалите не
открытия, а методу»1. В чем же суть правильного метода? — вот вопрос
вопросов. Какая дисциплина может дать ответ на такой вопрос? — естественно,
методология.

Но слово это крайне неудачное, потому что энергично подталкивает к очень
простому решению: методология — это наука о методе, о правильном
методе. Сегодня, когда мы миновали весь ХХ век с его грандиозными научными
открытиями, можно смело констатировать: нет такой науки! Нет и быть ее не
может!

Именно необходимость такого вывода демонстрирует вся история сферы
профессиональных философско-методологических исследований прошлого века.

Следует при этом подчеркнуть, что представление Бэкона или Декарта о
методологии — отнюдь не совпадает с тем, что мы называем «методологией (и
философией) науки» в ХХ веке.  Это довольно важное, принципиальное
уточнение. Известный итальянский философ и логик Эвандро Агацци весьма
справедливо заметил: «Традиционные [философские] размышления не посвящались,
так сказать, “тематически” исследованиям науки или наук, а были скорее
приложением к некоторым наукам общих рассуждений, обычно в связи с теорией
познания или с онтологией, поскольку внимание, уделяемое науке, было лишь
частью значительно более широкой концепции, или “системы”, в которой
интерпретация науки находила подобающее ей место.

Современная же философия науки есть философское исследование, ограниченное
тематически почти исключительно единственным предметом — наукой (или какой-то
конкретной отраслью науки) — и использующее интеллектуальные средства,
заимствуемые из других разделов философии, но применяемые как орудия для
понимания науки и лишь в той мере, в какой они так применяются»2.

1 Герцен А.И. Собр. соч. в 30 т. Т. I. — М., 1956. — С. 36.

2 Агацци Э. Переосмысление философии науки сегодня // Вопросы
философии. 2009. №1. — С. 40.

То, что произошло в данной сфере в течение ХХ столетия, можно представить в
виде трех периодов, которые отчасти совпадают хронологически и логически:
период 20–30-х годов (вплоть до эмиграции участников Венского кружка в разные
страны после аншлюса Австрии); период 50–70-х годов (прежде всего в
англоязычной традиции с «интеллектуальным центром» в Лондоне); наконец, то,
что называют «современностью», — где представлены весьма разнообразные
программы исследований социальной обусловленности науки. Отправной точкой
последнего периода можно считать 1979 г., когда появилась книга Бруно Латура и
Стива Вулгара «Лабораторная жизнь», а также сформировалась так называемая
«сильная программа» социологии знания Дэвида Блура. По оценке Э. Агацци, про-
изошел очевидный «социологический поворот» в философии науки, причем в
какой-то момент широко распространилось представление, «согласно которому
наука есть “социальный продукт” в буквальном смысле слова, т.е. деятельность,
полностью обусловленная динамикой власти, управляющей обществом и
производящая то содержание знания и те его приложения, которые нужны
различным властям независимо от какого-либо критерия объективной
значимости»3. Безусловно, последний период уже камня на камне не оставил от
исходных постулатов логических позитивистов, впрочем, безо всяких попыток
ответить на те вопросы, которые они поставили. Между первым и вторым периодом,
напротив, существовала, несмотря на смену ключевых моделей, очень тесная и
существенная преемственная связь.

Первый период развития анализа феномена науки, выявления специфики научного
познания связан с работой Венского кружка во главе с Морисом Шликом. Это
направление имеет, по сути, 4 названия: «позитивизм третьей волны»,
«неопозитивизм» (нетрудно видеть, что это просто историко-философские
«маркеры»), Венский кружок (географическое название), и — собственное имя,
выражающее смысл предложенной программы исследований, — «логический
позитивизм».

3 Там же. — С. 45.

Трудно переоценить все сделанное логическими позитивистами. Резко
отказавшись от традиционной философской постановки вопроса о том, что
представляет собой «чистый разум», способный достигать истин высшего
порядка, они поставили вопросы на «земную почву». В центре их модели
научного познания — теория, которая понималась (в конечном счете) как
совокупность высказываний, связанных отношениями вывода.  Они, бесспорно,
развили весьма тонкую логическую программу «прояснения» научного языка для
того, чтобы ясно указать те нормы, которые необходимо соблюдать ученому,
занятому производством научного (позитивного) знания.

Модель науки, которую они предложили, проста и убедительна: теория возникает
путем индуктивного обобщения и должна быть подвергнута проверке эмпирическим
опытом. Как это сделать? Теория должна работать, т.е. делать предсказания,
разрешимые в наблюдении. Если выводы из теоретических расчетов совпадают с
данными экспериментальной проверки, то теория «истинна», т.е. верна. Таков
знаменитый тезис о верификации научных высказываний. То, что не
верифицируется в принципе, — ненаучно. Поэтому, отбросив всякие сомнения, та-
кие высказывания (гипотезы или утверждения) необходимо удалить из системы
научных знаний. Критерий верификации, как говорится, строг, но справедлив! Он
полностью соответствует практике научных исследований, с чем фактически
согласны все, кто работает в науке.

И то, что верификация — это суть, вершина научного поиска, — своеобразно
излагается в знаменитом романе «Эроусмит» Синклера Льюиса (Нобелевская
премия 1930 г.). Прототипом главного героя, как известно, был знаменитый
микробиолог Поль де Крюи (он же, вероятно, был и «консультантом» писате-
ля). Поэтому к словам главного героя Мартина Эроусмита стоит прислушаться.

Вот наставник Мартина немецкий бактериолог Макс Готлиб дает наставления своему
ученику и говорит о религии ученого: «Быть ученым — это не просто особый вид
работы, не так, что человек просто может выбирать: быть ли ему ученым, или
стать путешественником, коммивояжером, врачом, королем, фермером. Это —
сплетение очень смутных эмоций, как мистицизм или потребность писать стихи;
оно делает свою жертву резко отличной от нормального порядочного
человека. Нормальный человек мало беспокоится о том, что он делает, лишь бы
работа позволяла есть, спать и любить. Ученый же глубоко религиозен — так
религиозен, что не желает принимать полуистины, потому что они оскорбительны
для его веры»4. И Мартин разделяет это религиозное чувство. Он буквально
молится перед началом собственного экспериментального исследования: «Боже,
дай мне незамутненное зрение и избавь от поспешности. Боже, дай мне покой и
нещадную злобу ко всему показному, к показной работе, к работе расхлябанной
и незаконченной. Боже, дай мне неугомонность, чтобы я не спал и не слушал
похвалы, пока не увижу, что выводы из моих наблюдений сходятся с
результатами моих расчетов, или пока в смиренной радости не открою и ра-
зоблачу свою ошибку. Боже, дай мне сил не верить в Бога!»5. Вот и формулировка
принципа верификации: выводы из моих на2 блюдений сходятся с результатами моих
расчетов. Разве это не главная норма, которая должна быть реализована в
научной практике? Только это позволяет избегать «полуистин», которые
оскорбительны для веры ученого, стремящегося к позитивному, подлинному,
надежному знанию о мире!

Можно сказать, что простенькая модель логических позитивистов была
одновременно глубоко романтичной, о чем нередко забывают, погружаясь в
развитую ими богатейшую «инструментальную часть» логической проверки научных
суждений.

Можно также утверждать, что концепт науки логических позитивистов заложил
фундамент всего последующего движения и навсегда сохранит свое значение в
качестве «стартового периода». Идея верификация была, бесспорно, нормативной
и в этом смысле — методологической. Причем, речь не шла о том, чтобы решить
какие-то конкретные научные проблемы, где ученый нуждался бы в «подсказке»
философа. Речь шла, скорее, о методологии науки в целом. Иначе говоря, каким
бы путем ни шел ученый в своем поиске, в конечном счете он должен добиться
того, чтобы выводы из его теоретических расчетов совпадали бы с данными
экспериментального наблюдения.

4 Льюис Синклер. Эроусмит. — М., 1998. — С. 308.

5 Там же. — С. 310.

Тем не менее, почти сразу идея верификации и соответствующий ей образ науки
был жестко раскритикован Карлом Поппером еще в 1934 году. Основная его идея,
как известно, (можно сказать — контридея) состояла в том, что основным
признаком научности следует считать не подтверждение теории, а саму
возможность ее опровержения (фальсификацию). Именно в беспощадной проверке
теории экспериментальными данными возможно ее опровержение и выдвижение новой
теории, которая стремится избежать обнаруженной «ошибки». Это был со-
вершенно новый концепт науки, ее принципиально новый образ.

Надо отдать должное, что именно логические позитивисты опубликовали монографию
малоизвестного автора в своей престижной серии «Wissenschaftlichen
Weltauffssung»6. По этой причине Поппера часто неоправданно причисляли к
позитивистам (в частности, франкфуртская школа), что вызывало его подлинный
гнев. Он писал по этому поводу: «Это старое недоразумение создали и
поддерживали люди, знавшие о моих работах из вторых рук. Благодаря терпимому
отношению некоторых участников Венского кружка, моя книга Logik der
Forschung, в которой я критиковал этот позитивистский кружок с реалистической
и антипозитивистской точек зрения, была опубликована в серии книг, выходивших
под редакцией Мориса Шлика и Филиппа Франка, двух ведущих членов этого кружка,
и те, кто привык судить о книгах по обложкам (или по редакторам), создали
миф о том, что я якобы входил в Венский кружок и что я — позитивист.  Никто из
читавших эту книгу (или любую другую из моих книг) не согласится с этим, разве
только поверит в этот миф с самого начала; в этом случае он, конечно, найдет
какие-нибудь подтверждения своей веры»7. Поппер полагал, что именно он
является «могильщиком» концепции верификации и других постулатов логического
позитивизма, хотя высоко ценил сам Венский кружок и его манеру работать в
философии. «Венский кружок, — уточнял свою позицию Поппер, — состоял из людей,
отличающихся оригинальностью и высоким интеллектуальным и нравственным
уровнем. Не все они были “позитивистами”, если подразумевать под этим
термином осуждение спекулятивного мышления, хотя таких было большинство. Я
же всегда стоял за открытое для критики спекулятивное мышление и, конечно,
за его критику»8.

6 Popper Karl. Logik der Forschung. Wien. Verlag von Julius Springer.  1935. —
250 s. Поппер позднее указывал, что год издания книги — 1934, что
издательством была допущена ошибка при публикации.

7 Поппер К. Разум или революция? // Эволюционная эпистемология и логика
социальных наук. — М., 2000. — С. 316.

8 Там же. — С. 325.

Именно по этой причине можно считать, что именно Поппер открыл тот период,
который получил общее название — «постпозитивизм». Но этот термин опять-таки
— просто историкофилософский «маркер», а не какая-то новая программа фило-
софско-методологических исследований.

Второй период — «постпозитивизм» — неоднороден, в нем быстро обозначились два
«крыла». Главным действующим лицом был, конечно, Поппер, который переехал в
Великобританию после Второй мировой войны, получив должность для работы в
Лондонской школе экономики и социальных наук. Он создал свое направление —
«критический рационализм». Это была уже программа действий, согласно с которой
сторонники «критического рационализма» успешно и с энтузиазмом принялась за
разработку новых проблем. Самыми видными фигурами этого направления были Имре
Лакатос, Пол Фейерабенд, Дж. Агасси.  Исходная идея Поппера —
фальсификационизм как точка роста научного знания — была разработана на
материале истории математики («Доказательства и опровержения» Лакатоса),
истории физики и астрономии (П. Фейерабенд), в ряде работ Дж. Агасси,
который подчеркнул, что основной предмет изучения новой методологии науки,
отличный от концепции логического позитивизма,— «наука в движении» («Science
in Flux»). Подчеркивая значение попперианской точки зрения, Агасси писал,
что вовсе не стабильность и устойчивость научного знания является главной
интеллектуальной ценностью, как это считалось веками. «Одним из тех немногих
философов, которые выступили против этой общепризнанной точки зрения,
является К. Поппер. Согласно его мнению, наука ценна своей восприимчивос-
тью, открытым характером — тем, что любые ее достижения в любое время могут
оказаться отброшенными и новые результаты могут заменить устаревшие. Наука,
говорит Поппер, есть постоянная борьба с собой, и она движется вперед
благодаря революциям и внутренним конфликтам»9.

9 Агасси Дж. Наука в движении //Структура и развитие науки. — М., 1978. —
С. 121.

Самым драматичным моментом в развитии постпозитивизма была, конечно,
полемика попперианцев с концепцией «нормальной науки» Томаса Куна. Симпозиум
1965 г. в Лондоне был проведен благодаря активной инициативе Лакатоса и привел
к ряду существенных результатов. Это свидетельствовало, что правильно
организованная критическая (по Попперу) дискуссия всегда приводит к «росту
знания», даже если речь идет о довольно разнородной совокупности
философско-методологических исследований10. Результатами можно считать в
первую очередь появление «методологии научно-исследовательских программ»
Лакатоса, который в пику Куну выдвинул новую концепцию, существенно
модифицируя исходный «наивный фальсификационизм» Поппера, а также выдвижение
нового методологического «принципа пролиферации» Фейерабенда. Критический
анализ слишком многозначного понятия парадигмы со стороны Маргарет Мастерман
привела Куна к отказу от этого понятия. Взамен «парадигмы» появилась
«дисциплинарная матрица». Кун объяснил мотивы своего «отступления» в весьма
содержательном «Дополнении 1969 года». Отныне «Дополнение» обязательно
сопровождает любые публикации «Структуры научных революций».

10 Материалы этого симпозиума были опубликованы спустя 5 лет — вероятно, по
причине того, что Лакатос, будучи ответственным редактором данной книги,
срочно дорабатывал новую (собственную) концепцию науки.  См.: Criticism and
the Growth of Knowledge. Ed. by Imre Lakatos, Alan Musgrave. Cambridge, 1970.

Таким образом, состоявшуюся дискуссию можно в полной мере назвать
плодотворной.

Вероятно, с того же момента возникла осознанная необходимость различать так
называемые Hard и Soft Methodology.  Строгость методологических предписаний —
исходящая от программы логических позитивистов или от попперианцев — нача-
ла смягчаться. Лакатос — один из самых ревностных последователей
критического рационализма — в своей «методологии научно-исследовательских
программ» утверждает, что даже если можно обоснованно зафиксировать, что
исследовательская программа находится в состоянии стагнации, что в ней не про-
исходит прогрессивного сдвига проблем, ведущего к увеличению эмпирического
роста, методолог все же не должен быть жестким в своих рекомендациях. Он
писал, что методолог может только честно зафиксировать «счета» конкурирующих
программ, но не указывать, какая из конкурентов одержит бесспорную
победу. «Никогда не исключается, — пишет он, — возможность того, что
какая-то часть регрессирующей программы будет реабилитирована»11. Он
неоднократно упоминает необходимость «методологической терпимости», которая
отрицает догматическую строгость как приверженца «верификации», так и
«фальсификациониста»12.

Довершил разрушение строгости методологических правил как пути к успеху,
конечно, Фейерабенд с его знаменитой книгой «Против методологического
принуждения». Он провозгласил, что в конечном итоге для решения творческой
задачи «подходит всё» («anything goes!»), а мечты о неуклонном следовании
универсальному научному методу — дело пустое. В свойственной ему манере
Фейерабенд задиристо утверждал: «…Становится очевидным, что идея жесткого
метода или жесткой теории рациональности покоится на слишком наивном
представлении о человеке и его социальном окружении. Если иметь в виду
обширный исторический материал и не стремиться “очистить” его в угоду своим
низшим инстинктам или в силу стремления к интеллектуальной безопасности до
степени ясности, точности, “объективности”, “истинности”, то выясняется, что
существует лишь один принцип, который можно защищать при всех обстоятельствах
и на всех этапах человеческого развития, – допустимо все»13. Как бы
провокационно это ни звучало, история науки фактически подтверждает такую
радикальную оценку. Чем большее количество историко-научного материала
подвергалось анализу, тем быстрее развеивалось представление о том, что
многообразные пути научного поиска можно свести к единому методологическому
«знаменателю».

Все эти модификации происходили тем не менее в рамках попперианства, которое,
как говорилось выше, представляет только один полюс, одно «крыло» данного
периода.

11 Лакатос И. Избранные произведения по философии и методологии науки. — М.,
2008. — С. 414.

12 Там же. — С. 410.

13 Фейерабенд П. Избранные труды по методологии науки. — М., 1986. —
С. 158–159.

Другое «крыло» постпозитивизма, которое за неимением лучшего назвали
«историко-социологическим» направлением и крупнейшими представителями которого
были Т. Кун и М. Полани, не соглашались с «критическим рационализмом» даже в
таком «ослабленном», более тонком варианте. Для них главный вопрос стоял
совсем иначе. Если говорить кратко, то суть дела в том, что необходимо
развивать не нормативный, а дескриптивный подход к анализу самой науки и ее
истории. И это — совершенно другая позиция, иная исходная установка. Оба они
пытались повернуть русло исследований, указывая на те аспекты научной
практики, которые просто не могли быть ассимилированы концепциями
попперианцев.

НОРМАТИВНОЕ И ДЕСКРИПТИВНОЕ: «КОПЕРНИКАНСКАЯ РЕВОЛЮЦИЯ» ТОМАСА КУНА

Необычный и до сих пор не полностью освоенный удар по «критической»
методологии науки нанес Майкл Полани с его концепцией «неявного знания» (tacit
knowledge). Эта концепция возникала вне всяких влияний со стороны Куна и
независимо от него. Полани — авторитетный химик с мировой известностью, занял
пост профессора кафедры социальных наук в Манчестерском университете в 1946
г. Его знаменитая «взрывная» книга «Personal Knowledge» (с подзаголовком «На
пути к посткритиче2 ской философии») увидит свет в 1958 г. Кун —
физик-теоретик по образованию, выпускник Гарвардского университета — увлечен
историей науки и публикует свою первую книгу «Коперниканская революция»
(1957). Его «Структура научных революций» (1962) является во многом обобщением
того материала, который проанализирован в первой работе. Именно история
коперниканства сама по себе прекрасно иллюстрирует особенности научных
революций как смены «парадигм». Позднее в «Структуре» Кун несколько раз
будет ссылаться на «неявное знание» как на важный фактор единства членов того
или иного научного сообщества.

Оба они знают науку не понаслышке, обоих отличает искренний и глубокий
интерес к историко-научным изысканиям.  По духу своих исканий Кун и Полани,
несомненно, единомышленники.

Главное, что удалось показать Полани с мощной убедительностью, — то, что
знание, в том числе научное, отнюдь не сводится к системе высказываний, не
может считаться чисто семиотическим объектом. Подлинная тайна всякого
профессионального мастерства, в том числе научного познания, — «неявное
знание», то, которое невозможно выразить в словах, формулировках, системе
строгих «предложений». Нет ничего удивительного в том, что медик и химик,
каким был Полани, смог зафиксировать такие особенности своей профессии. Он
писал: «Однако то большое количество учебного времени, которое студенты
химики, биологи и медики посвящают практическим занятиям14, свидетельствует
о важной роли, которую в этих дисциплинах играет передача практических
знаний и умений от учителя к ученику. Из сказанного можно сделать вывод, что
в самом сердце науки существуют области практического знания, которые через
формулировки передать невозможно»15. Любые формулировки и определения лишь
сдвигают область скрытого «молчания», но никогда не отменяют ее.

Итак, знание обладает экстралингвистическими характеристиками. А это
фактически убивало претензии всей логико-аналитической традиции на то, чтобы
ответить на основной вопрос — что есть знание? «Высказывание» предстает как бы
вершиной айсберга, большая, подводная часть которого скрыта в сфере «tacit
knowledge».

Что касается работ Томаса Куна, то модель «смены парадигм» оказалась в
центре такой оглушительной критики, что достаточно долго современники
фактически даже не приступали к анализу тех результатов, которые были им
достигнуты. Понадобилось немало времени, прежде чем остыли страсти и можно
было относительно беспристрастно оценить им сделанное.

14 Список этот явно можно продолжить: и геологи, и почвоведы, и географы, и
физики, и археологи… да есть ли вообще исключения из этого правила?!

15 Полани М. Личностное знание. — М., 1985. — С. 89.

В 1997 г. в Институте философии РАН состоялось заседание «круглого стола»,
посвященного памяти Т. Куна, обсуждению его философского наследия. Хотелось бы
обратить внимание на выступление М.А. Розова с оценкой достижений автора
новой, «посткритической» модели науки. По его словам, Томас Кун, как это
хорошо видно с дистанции времени, прошедшей после 1962 года, действительно
может считаться человеком, совершившим «коперниканский переворот» в изучении
науки. До Куна философия и методология науки были, по сути,
неразличимы. Задача исследования (описания) механизмов развития науки не отли-
чалась от задач формулировки тех методологических правил, которые могли бы
способствовать развитию науки. «Переворот» Куна состоял в том, что от
модальности долженствования, которая характерна для формулировки норм и
правил, философы науки перешли к модальности существования, т.е. сменили по-
нимание своей исследовательской позиции и комплекса задач, вытекающих из
такого понимания. Кун продемонстрировал, что в своей деятельности ученые (как
члены научного сообщества) определены некоторыми традициями (программами), и
задача состоит в том, чтобы реконструировать эти программы и выявить
механизмы их изменений. Он показал, что программы (парадигмы) заданы
примерно таким же образом, как и язык. Это позволило, в частности, осознать,
что философия науки включена в тот круг социо-гуманитарных дисциплин,
которые изучают такие классы явлений, как язык, пословицы, культурные тради-
ции, социальные эстафеты и тому подобное16.

На различие «нормативного» и «дескриптивного» в эпистемологии обратил
внимание Э. Агацци: «Переходя теперь к эпистемологии (понимаемой как общая
теория познания), надо отметить, что она всегда включала два аспекта,
которые можно назвать дескриптивным и нормативным. Нормативный аспект яв-
ляется предварительным, поскольку он состоит, во-первых, в каком-либо
определении понятия знания, т.е. достаточным уточнении того, что такое знание,
а это, во-вторых, определяет также, каким требованиям должно удовлетворять
то, что мы хотели бы квалифицировать как знание.

Дескриптивный аспект состоит в выяснении того, как добывается знание, какими
шагами, при каких условиях и согласно каким критериям можно убедиться в его
получении, и это применительно к различным предметам, которые мы хотели бы
знать. Конечно, эти два аспекта различимы только аналитически, а конкретно
они взаимозависимы»17.

16 Изложение выступления М.А. Розова см.: Философия естествознания ХХ века:
итоги и перспективы. (Материалы к Первому Всероссийскому Философскому
Конгрессу «Человек–Философия–Гуманизм»). — М., ИФРАН, 1997. — С. 41–42.

17 Агацци Э. Эпистемология и социальное: петля обратной связи // Вопросы
философии. 2010, №7. — С. 64.

Полностью разделяя высказанное мнение, все же не могу согласиться со следующим
тезисом Агацци (вернее, считаю следующее его утверждение существенно
неточным): «Историю эпистемологии можно рассматривать как непрерывное углубле-
ние и расширение ее дескриптивного аспекта, в ходе которых функции,
инструменты, критерии, возможности приобретения знания обнаруживались и
оценивались с точки зрения их здравости (нормативного аспекта)»18.

Почему приведенная выше оценка соотнесения дескриптивного и нормативного
кажется мне некорректной? Модель научных парадигм Куна попперианцы расценили
как защиту нетворчески работающего ученого. «Утешением для специалиста»
назвал куновскую «парадигму» П. Фейерабенд; «нормальная наука», конечно,
существует, но она опасна — мнение Поппера; иррациональной назвал концепцию
Куна Лакатос19…

Кун, честно, искренне и откровенно, пытаясь разъяснить свое принципиальное
несогласие с Поппером, выразительно писал: «Я называю то, что нас разделяет,
скорее гештальт-переключением, чем несогласием, и поэтому же я одновременно
и сбит с толку, и заинтригован тем, как лучше объяснить эти наши
расхождения. Как мне убедить сэра Карла, знающего все то, что знаю я о
развитии науки, и так или иначе уже сказавшего нечто об этом, в том, что
предмет, который он называет уткой, я называю кроликом? Как мне показать ему
то, что видно сквозь мои очки, когда он уже научился смотреть на все, что я
мог ему показать, через свои собственные?»20

18 Там же.

19 См.: Criticism and the Growth of Knowledge; перевод статьи Поппера
«Нормальная наука и опасности, связанные с ней» // Кун Т. Структура научных
революций. — М., 2001. — С. 525–538; частичный перевод статьи «Утешение для
специалиста» // Фейерабенд П. Избранные труды по методологии науки. — М.,
1986. — С.109–124.

20 Кун Т. Логика открытия или психология исследования // Кун Т.  Структура
научных революций. — М., 2001. — С, 543.

Нет, открытие дескриптивной позиции в философии науки вызывало искренне
недоумение у методологов (нормативистов), и этому соответствует подлинное
принципиальное расхождение позиций. Исходные позиции порождают различные
«очки», сквозь которые рассматривается историко-научный материал,
призванный доказать правоту выбранного подхода. Важно было осознать само
различие «нормативного» и «дескриптивного», не пытаясь совместить одно с
другим.

ТРАЕКТОРИИ ПОИСКОВ В СОВЕТСКОЙ ФИЛОСОФИИ НАУКИ ПОСЛЕВОЕННОГО ПЕРИОДА

Следует высоко оценить попытки советской философской молодежи того времени,
которая попыталась, с одной стороны, учесть опыт обсуждения методологической
тематики на Западе, усвоить эту традицию, а с другой, — смело обратиться к
анализу научной практики, чтобы не быть голословными и осознать, какие
реальные методологические проблемы волнуют ученых, ведущих работу на передовом
крае своих дисциплин.

В Москве под руководством Г.П. Щедровицкого возник Московский методологический
кружок, в Минске — семинар, где лидером был В.С. Степин, в Новосибирске —
семинар М.А. Розова. Все они примерно одинаково понимали свои задачи —
работать надо на конкретном материале истории науки, а также по возможности
обсуждать методологические проблемы, которые фиксируются самим научным
сообществом. В то же время в Институте философии АН СССР активно работала
группа молодых логиков, объединившихся вокруг четы В.А. Смирнова и
Е.Д. Смирновой. Эта группа, не скрывая своих намерений, ориентировалась на
современную западную логику (собственно, как и математики, они считали, что
формальная логика не может иметь каких-то демаркационных особенностей). Сего-
дня, по прошествии времени можно утверждать, что именно эта группа восприняла
всерьез исследовательскую программу Венского кружка и по мере сил вносила
свои результаты в эту «международную копилку».

Действительно, все вышеперечисленные герои нашей отечественной философии
были достаточно хорошо знакомы с работами Венского кружка (насколько это
возможно при том, что основные труды логического позитивизма находились в
«спецхране»). Критика позитивистской философии была одной из ключевых тем
методологических штудий того времени. И если Степин, Щедровицкий и Розов
критически относились к самим возможностям этой исследовательской программы,
то группа Смирнова, по сути дела, восприняли задачу логического анализа
языка науки вполне одобрительно.

Зато труды постпозитивизма — начиная с работ Карла Поппера, а также
участников «критического рационализма», тем более их соперников Томаса Куна
и Майкла Полани — становились известны относительно скоро, по мере
постепенного проникновения философской англоязычной литературы в наше
сообщество. Все-таки 1960-е гг. — времена хрущевской «оттепели». Эти труды
переводились, так сказать, «частным образом», внимательно изучались рефераты
их работ, которые систематически выполнялись в ИНИОНе. Заметим, что
Г.П. Щедровицкий даже имел в конце 60-х годов небольшую личную переписку с
И. Лакатосом, а на «Доказательства и опровержения» написал для журнала
«Вопросы философии» весьма интересную рецензию21.

21 Щедровицкий Г.П. Модели новых фактов для логики // Вопросы
философии. 1968. №4.

Картина поисков, идущая в рамках постпозитивизма, складывалась более или
менее адекватная. Тем не менее наши герои, идейные лидеры того времени,
предпочитали «идти своим путем». И связано это с тем, что они никогда не
ставили своей задачей быть просто историками (или критиками) современной
зарубежной философии, а к возможностям средствами формальной логики решать
новые задачи в сфере изучения феномена науки и ее актуальных проблем
относились весьма скептически.

В ретроспекции хорошо также видно, что «идеи витают в воздухе», что
интеллектуальные поиски успешно преодолевали «железный занавес». Велась в
нашем сообществе работа в духе Венского кружка; появились «критические
рационалисты» с их активным неприятием существующих практик научного мышле-
ния (сюда, конечно, относятся в первую очередь щедровитяне, которые «пришли в
мир, чтобы не соглашаться»); были и представители «дескриптивной установки»
в духе Томаса Куна, желающие не преобразовывать науку, а только изучать ее
(таким были М.А. Розов и круг его учеников). В.С. Степин, как нам пред-
ставляется, начинал в духе дескриптивной установки, в русле исторических
реконструкций (прослеживал историю электродинамики Максвелла), следуя в этом
плане путем Куна и Полани, но в конечном счете попытался совместить два
подхода, о чем свидетельствует его итоговая книга22.

СИСТЕМО/ДЕЯТЕЛЬНОСТНАЯ МЕТОДОЛОГИЯ Г.П. ЩЕДРОВИЦКОГО И ИССЛЕДОВАНИЕ ФЕНОМЕНА
«МЕТОДОЛОГИчЕСКОГО МЫШЛЕНИЯ» М.А. РОЗОВА

Весьма интересно проследить, каким образом в рядах отечественного
философского сообщества сформировались два проекта построения методологии как
особой деятельности, два способа понимания ее ключевых задач. Здесь
Г.П. Щедровицкий и М.А. Розов разошлись принципиально, хотя в исходе имели
фактически общую «рамочную» картину трансляции и воспроизводства социального
целого.

Оба они полагают, что никакие семиотические явления (включая знание)
невозможно изучать и понять вне объемлющего контекста социального
целого. Такая онтология была невозможна для Венского кружка, такой широты
онтологической картины не хватало и постпозитивизму. А самый общий
социальный процесс (подобно тому, как живое — прежде всего воспроизводство
самого себя) — это процессы трансляции различных компонентов деятельности и
дальнейшее воспроизводство деятельности по имеющимся прототипам. Но простое
«перетекание» в новые условия прежней деятельности не интересует
Г.П. Щедровицкого. Он считает это слишком элементарным случаем и фокусирует
свое внимание на том, что деятельность реализуется по «нормам», которые
выступают как эталоны. Именно по ним строится следующая деятельность, даже
если условия ее реализации сильнейшим образом варьируются. Все сказанное
четко выражено в статье 1967 г. «“Естественное” и “искусственное” в
семиотических системах»23. Авторы — В.А. Лефевр, Э.Г. Юдин,
Г.П. Щедровицкий. Эта статья-манифест восхищала и М.А. Розова.

22 Степин В.С. Теоретическое знание. — М., 2000. — 744 c.

23 Лефевр В.А., Юдин Э.Г., Щедровицкий Г.П. «Естественное» и «искусственное»
в семиотических системах // Семиотика и восточные языки.  — М., 1967. —
С. 48–56.

Вот основное рассуждение: «Рассмотрим простейший случай, когда
восстановление составляющих какой-то социальнопроизводственной структуры
(обозначим ее знаком А) происходит без введения каких-либо специальных
средств трансляции и образцом, или «нормой», для составляющих каждой последу-
ющей единицы являются составляющие предшествующей»24.  Естественно, что если
условия (обозначенные знаком B) постоянно меняются, то происходит и
некоторая эволюция от А1 до Аi. Это случай «естественного» процесса трансляции
социума.  Но представим другой случай, — продолжают рассуждать авторы, —
когда составляющие социально-производственной структуры А1 зафиксированы в
некоторых эталонах (А), которые как «норма» транслируются от одной единицы к
другой. Тогда А1 может варьировать в зависимости от условий реализации, од-
нако норма (А) остается вне изменений. Удержание «норм» (эталонов) — это уже
искусственный процесс, хотя и порождаемый естественным путем. Далее возникает
процесс обучения, т.е.  подготовка людей для воспроизводства той или иной
социально-производственной структуры. Именно он закрепляет искусственный
характер трансляции норм. Простейший случай «естественной» социальной
трансляции зафиксирован как элементарный и отставлен в сторону как не
требующий дальнейшего анализа. Особенно интересовали Щедровицкого случаи
проектирования норм (из чего и вырастает методология в его понимании).

24 Там же. — С. 49.

Но именно указанный базовый механизм социального воспроизводства (социальной
памяти) лег в основу концепции М.А. Розова. В дальнейшем он назовет
воспроизведение деятельности по непосредственным образцам социальными эста-
фетами. Вербализация образцов, а тем более обучение — это очень серьезная
эволюция. Важно то, что в основе социальной жизни лежит этот простейший
случай, когда еще не существует фиксированных «норм», а существует только
непосредственное подражание предшествующей деятельности. И это пребывание
актов предшествующей деятельности в качестве образцов для последующей Розов
называл «нормативами» деятельности.  Следует подчеркнуть также, что концепция
«неявного знания» М. Полани, с которой наше сообщество познакомилось значи-
тельно позже, без труда объяснялась именно идеей социальных эстафет.

Итак, можно указать на различие терминов — «нормы» у Щедровицкого, «нормативы»
у Розова. Эти различия направляли и дальнейшую разработку самой методологии
как особой области интеллектуальной работы.

У Щедровицкого методология приобрела черты глобальной индустрии разработки
норм для всех видов человеческой активности — как практической, так и
теоретической. Рефлексия для него — универсальный способ оптимизации всех
типов деятельности, поэтому рефлексия описывает то, что делается, фикси-
рует нарушение норм и «исправляет» то, что делается. Особо интересен случай,
когда нормы специально проектируются для новаций, а человеческий опыт не имеет
в своем арсенале аналогов того, что требуется в конкретных ситуациях. В
случаях «радикальной новизны» методолог составляет проекты, подобные
проектам «бумажной архитектуры». Советы, которые может здесь дать методолог,
направлены на «расстановку сил», на создание плодотворной кооперации
участников совокупного деятельностного процесса. В конечном итоге проект
Щедровицкого нашел свое воплощение в построении так называемой систе-
мо-деятельностной методологии (СД-методологии). Очевидно, что речь идет о
проектах системной реализации сложной деятельности, требующей большого
количества участников (акторов). Его проект методологии, по сути, —
организационный. К сожалению, методологические идеи Георгия Петровича во мно-
гом остались на уровне «бумажной архитектуры», ибо ни к какой управленческой
работе сам он никогда не допускался. Однако в 1980-е гг. Щедровицкий стал
проводить масштабные «орг-деятельностные игры», которые готовили по всей
стране кадры будущих топ-менеджеров и управленцев высшего государственного
слоя.

Ход рассуждений Розова совсем иной. В 1974 г. опубликована совсем небольшая
работа25, выразительно иллюстрирующая собственную постановку вопроса. Здесь
был предложен шутливый, но очень выразительный, мысленный эксперимент,
демонстрирующий стиль и особенности методологического мышления.

25 Розов М.А., Розова С.С. К вопросу о природе методологической деятельности
// Методологические проблемы науки. Вып. 2. — Новосибирск, 1974. — С. 25–35.

Представим себе, что в некоей условной древней цивилизации существуют только
два ученых человека: один умеет определять площади простых фигур (геометр),
а другой умеет взвешивать (например, зерно). И вот первого измерителя (гео-
метра) вызывает ужасный тиран и требует, чтобы он определил площадь листа
какого-то заморского растения, весьма сложной формы. Озадаченный геометр
уходит домой и начинает размышлять. Совета ему спросить не у кого. Но он
начинает думать о том, какая связь существует между процедурами измерения
площади и веса. На первый взгляд, ничего общего. Однако же, думает
незадачливый геометр, на большей площади земли вырастает больше урожая, что
выражается в большем количестве зерна. А что если засыпать зерном лист
заморского растения, а потом взвесить его? В конечном счете можно засыпать
этим количеством зерна простейшую фигуру, площадь которой он умеет
вычислять! И вот — задача решена, тирану сообщается ответ, а геометр получает
премию. Догадливый читатель может вспомнить, что этот мысленный эксперимент,
по сути дела, воспроизводит историю открытия Архимедом основного закона
гидростатики. Получив задание от тирана Сиракуз Гиерона определить, какая
из подаренных корон не является чисто золотой, Архимед погрузился в ванну…
откуда выскочил голышом с криком «Эврика».

Что можно здесь фиксировать? Во-первых, для того, чтобы человек начал думать,
необходимо, чтобы возникла «нештатная ситуация». Во всех других случаях
измерение идет совершенно стандартным путем. (Заметим, что Щедровицкий
неоднократно подчеркивал, что нетривиальное мышление начинается в ситуациях
«разрыва деятельности».) Во-вторых, если задача нетривиальная, то следует
привлечь самый неожиданный опыт из других сфер практики. Такую работу можно
назвать методом «отдаленных сопоставлений». В-третьих, надо перенести опыт
решения задачи, взятой из другой сферы жизни, на решение внезапно возникшей
задачи. Все перечисленное — и есть характеристики того типа мышления, которое
можно назвать «методологическим».

Как видим, Розов, если говорить кратко, предлагает для решения новой,
нестандартной задачи воспользоваться образцами работы из весьма отдаленных
сфер практики или мышления.  И в этом он видит суть методологических ходов
мысли. Это совершенно иное представление о методологии и ее задачах.

Действительно, в научной практике дискуссии о методологии возникают только
тогда, когда нет «нормальных» (Кун сказал бы — парадигмальных) способов
решения новых задач. Иначе говоря, люди, будь они теоретики или практики,
обсуждают методологические проблемы только в тех случаях, когда специали-
зированных методов работы просто нет.

Предлагаемое Розовым представление о сути методологии базируется на общей
посылке, что если нормативы решения задачи в каких-то ситуациях отсутствуют,
то нормативы другой практики могут оказаться весьма эффективными. Социальные
эстафеты, о которых говорилось выше, потому и действенны, что обладают
способностью «перескакивать» (точнее, их надо целенаправленно «перетаскивать»)
из одних сфер познавательного опыта в другие.

Как ни странно, проект Розова оказался успешно реализованным на
практике. Уже в 1990-е годы отечественные географы приняли именно такой
способ работы в своих систематически проводимых методологических
конференциях. С подачи Михаила Александровича эти заседания стали называться
«Сократическими Чтениями» (в 2012 году прошли X Чтения, посвященные памяти
М.А. Розова). Суть этих мероприятий — обмен непосредственным опытом решения
нетривиальных задач. Каждый участник — носитель этого опыта, и он
рассказывает о собственных ходах мысли с целью обмена живыми образцами мы-
шления, а также в попытках найти «подсказку» у других вынужденных
«вольнодумцев».

***

Думаю, что оба проекта методологии, созданные в рамках отечественного
философского сообщества, глубоко своеобразны и достойны самого пристального
анализа. Они были направлены на решение методологических задач, постоянно
возникающих в ходе научного поиска.

СД-методология Г.П. Щедровицкого, как уже говорилось, решает прежде всего
организационно-управленческие вопросы, связанные с кооперацией людей при
исполнении сложных, комплексных (системных) задач. Она учит рефлексии и опти-
мальному взаимодействию в ходе совместной работы.

Методологические ходы, которые предлагает фиксировать и анализировать
М.А. Розов, учат прежде всего искать «подсказки» на уровне познавательных
метафор, самого широкого обмена опытом решения нетривиальных задач. Такая
soft методоло2 гия не пафосна, но работоспособна.

Концепцию науки, с его точки зрения, надо строить совсем в другой плоскости —
в духе дескриптивной установки. Это — иная профессиональная задача.
\end{document}
