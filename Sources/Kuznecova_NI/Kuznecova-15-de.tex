\documentclass[11pt,a4paper]{article}
\usepackage{od}
\usepackage[utf8]{inputenc}
\usepackage[russian,main=ngerman]{babel}

\title{Zwei Methodologie-Projekte:\\ G.P. Shchedrovitsky und M.A. Rozov}

\author{N.I. Kuznecova, Moskau}
\date{10. Sokratische Lesungen, Moskau 2015}

\begin{document}
\maketitle
\begin{quote}
  Original: \foreignlanguage{russian}{ Два проекта методологии:
    Г.П. Щедровицкий и М.А. Розов // Х Сократические Чтения. Реальность как
    социальные эстафеты (памяти М.А. Розова). Сб. докл. /
    Отв. ред. В.А. Шупер. М.: ЭСЛАН, 2015. С. 12–32.}
  
  Quelle: \url{https://rozova.org/images/documents/KuznecovaDvaProekta.pdf}

  Mehr zur Autorin (in Russisch)\\
  \url{https://vovr.elpub.ru/jour/pages/view/Kuznetsova}

  Übersetzt von Hans-Gert Gräbe, Leipzig. 
\end{quote}

Die Wortkombination, die heute gewohnt ein besonderes, eigenständiges Gebiet
der philosophischen Forschung -- „Methodologie und Wissenschaftsphilosophie“
-- bezeichnet, ist keineswegs schon lange im Gebrauch.  In der vaterländischen
(HGG: also sowjetisch-russischen) Literatur über methodologische Probleme der
Wissenschaft, über Methodologie als Ganzes wurde in den 1960-70er Jahren so
oft und so umfassend gesprochen, dass das Thema außerordentlich aufdringlich
und von den ideologischen Vorgaben der Partei aufgezwungen schien.  Die
Methodologie muss richtig, also „marxistisch-leninistisch“ sein.

Es schien, dass die Teilnehmer der endlosen Konferenzen, Workshops und
Pausen-Diskussio\-nen wie auch die Autoren einschlägiger Sammelwerke und
Monographien, nie eine gemeinsame Sprache finden und nie ihre Ausgangspunkte
klar formulieren können. Darüber hinaus wurde bis in die späten 1980er Jahre
oft behauptet, dass „Epistemologie“ und „Wissenschaftsphilosophie“ überhaupt
überflüssige Begriffe seien, die ohne Not „Wesenheiten vervielfachen“
(\foreignlanguage{russian}{умножающие сущности}), da es in der Philosophie von
alters her die Sphäre der gnoseologischen (theoretisch-erkennenden --
\foreignlanguage{russian}{теоретико-познавательных}) Forschungen gab.

Die Geschichte der philosophisch-methodologischen Suche des zwanzigsten
Jahrhunderts -- wie im Westen so auch in den vaterländischen Traditionen --
liefert ein sehr interessant und deutliches Material für Analyse, Reflexion
und den Versuch, im Folgenden die Schritte und Projekte in diesem Bereich
vorzustellen.
\newpage

\section*{Die Abenteuer der „Methodologie“:\\ Von der Hard zur Soft
  Methodology} 

„Methodologie“ ist ein sehr verantwortungsvolles und leider sehr unglückliches
Wort.

Natürlich ist das Wort \emph{verantwortungsvoll}, es ist dies bereits seit den
Zeiten von Fr. Bacon, der die Notwendigkeit eines „Neuen Organons“ verkündete,
und den „Diskussion über die Methode“ Descartes'. Der junge Herzen erinnert in
seiner Abschlussarbeit an der Universität, in welcher er die kopernikanische
astronomische Revolution analysiert, daran, dass der große Descartes zu sagen
pflegte, wenn er ob seiner mathematischen Entdeckungen gelobt wurde: „Loben
Sie nicht die Entdeckung, sondern die
Methode“\footnote{\foreignlanguage{russian}{Герцен А.И. Собр. соч. в 30
    т. Т. I. — М., 1956. — С. 36.}.\\ Herzen A.I. Gesammelte Werke in 30
  Bänden. Moskau, 1956. Band 1, S. 36.}. Worin aber besteht das Wesen der
richtigen Methode? -- das ist die Frage aller Fragen. Kann eine Disziplin
Antwort auf diese Frage geben? -- natürlich, die \emph{Methodologie}.

Aber das Wort ist äußerst \emph{unglücklich}, da es energisch auf sehr eine
einfache Lösung drängt: Methodologie ist die Wissenschaft von der Methode, von
der \emph{richtigen} Methode. Heute, angesichts des gesamten 20. Jahrhundert
mit seinen grandiosen wissenschaftlichen Entdeckungen, können wir mutig
konstatieren: Eine solche Wissenschaft gibt es nicht! \emph{Es gibt sie nicht
  und es kann sie nicht geben!}

Genau die Notwendigkeit einer solchen Schlussfolgerung demonstriert die
gesamte Geschichte der Sphäre der professionellen
philosophisch-methodologischen Forschungen des vergangenen Jahrhunderts.

Dabei ist jedoch zu unterstreichen, dass die Vorstellungen von Bacon oder
Descartes über Methodologie nicht mit dem zusammenfallen, was wir im
zwanzigsten Jahrhundert als „Methodologie (und Philosophie) der Wissenschaft“
bezeichnen.  Das ist eine ziemlich wichtige, grundlegende Präzisierung. Der
berühmte italienische Philosoph und Logiker Evandro Agazzi hat sehr zu Recht
festgestellt: „Traditionelle [philosophische] Betrachtungen waren nicht --
sozusagen -- „thematischen“ Studien einer Wissenschaft oder der Wissenschaften
gewidmet, sondern waren eher die Anwendung allgemeiner Argumentationen auf
einige Wissenschaften, gewöhnlich in Verbindung mit einer Erkenntnistheorie
oder Ontologie, da die Aufmerksamkeit, die der Wissenschaft gewidmet wurde,
nur Teil eines viel umfassenderen Konzepts oder „Systems“ war, in dem die
Interpretation der Wissenschaft ihren angemessenen Platz fand.

Aber die moderne Wissenschaftsphilosophie ist eine philosophische Forschung,
die thematisch fast ausschließlich auf einen einzigen Gegenstand fokussiert
ist -- auf die Wissenschaft (oder irgendeinen konkreten Wissenschaftszweiges)
-- und intellektuelle Mittel anwendet, die aus anderen Teilen der Philosophie
entlehnt sind, aber nur als \emph{Werkzeuge} für das Verstehen der
Wissenschaft und nur in dem Maße, in dem sie so angewendet werden
können“\footnote{Agazzi E. Überdenken der Wissenschaftsphilosophie von heute.
  Fragen der Philosophie. 2009 Nr. 1, S. 40. \foreignlanguage{russian}{Агацци
    Э.  Переосмысление философии науки сегодня // Вопросы философии. 2009.
    №1. — С. 40.}}.

Was auf diesem Gebiet im zwanzigsten Jahrhundert geschah, kann in Form von
drei Perioden dargestellt werden, die sich teilweise chronologisch und logisch
überlappen: der Zeitraum der 20er bis 30er Jahre (bis zur Emigration der
Mitglieder des Wiener Kreises in verschiedene Länder nach dem Anschluss
Österreichs); der Zeitraum der 50er bis 70er Jahre (vor allem in der
englischsprachigen Tradition mit einem „intellektuellen Zentrum“ in London);
schließlich das, was man "Modernität"
(\foreignlanguage{russian}{современность}) nennt, mit einer großen Vielfalt
von Forschungsprogrammen über die soziale Konditionierung von Wissenschaft.
Als Ausgangspunkt der letzten Periode kann 1979 betrachtet werden, als das
Buch „Das Laboratorium des Lebens“ von Bruno Latour und Steve Woolgar erschien
und sich auch die sogenannte „starke Programm“ der Wissenssoziologie von David
Bloor formierte. E. Agazzi schätzt ein, dass sich eine offensichtliche
„soziologische Wende“ in der Wissenschaftsphilosophie vollzog, wobei sich in
einem bestimmten Moment die Vorstellung weit verbreitete, „wonach Wissenschaft
ein soziales Produkt“ ist im wörtlichen Sinne des Wortes, d.h. eine Aktivität,
die voll und ganz von der Dynamik der Macht bedingt ist, welche die
Gesellschaft steuert, und die Wissensinhalte und Anwendungen produziert, die
von den verschiedenen Mächten benötigt werden, unabhängig von jedem Kriterium
objektiver Bedeutung“\footnote{Ebenda, S. 45.}. Natürlich hat die letzte
Periode keinen Stein auf dem anderen gelassen von den ursprünglichen
Postulaten der logischen Positivisten, übrigens ohne jeden Versuch, die Fragen
zu beantworten, die jene gestellt haben.  Umgekehrt gab es zwischen der ersten
und zweiten Periode, trotz des Wechsels der Schlüsselmodelle, eine sehr enge
und wesentliche Verbindung der Kontinuität.

Die erste Periode der Entwicklung der Analyse des Phänomens Wissenschaft, die
Herausarbeitung der Spezifik wissenschaftlicher Erkenntnis steht im
Zusammenhang mit der Arbeit des Wiener Clubs unter der Leitung von Moritz
Schlick.  Diese Richtung hat im Wesentlichen vier Namen: „Positivismus der
dritten Welle“, „Neopositivismus“ (es ist nicht schwer zu erkennen, dass dies
historisch-philosophische „Marker“ sind), Wiener Kreis (eine geographische
Bezeichnung), und -- die Eigenbezeichnung, welche die Bedeutung des
vorgeschlagenen Forschungsprogramms zum Ausdruck bringt, -- „logischer
Positivismus“.

Die Resultate der logischen Positivisten sind schwer zu überschätzen. Indem
sie sich scharf von der traditionellen philosophischen Fragestellung
abgrenzten, was ein „reiner Geist“ sei, fähig, die Wahrheiten der höchsten
Ordnung zu erreichen, stellen sie ihre Fragen auf eine „irdische Grundlage“.
Im Zentrum ihres Modells der wissenschaftlichen Erkenntnis steht die Theorie,
die (letztlich) als eine Gesamtheit von Aussagen verstanden wurde, die durch
Ableitungsrelationen miteinander verbunden sind.  Sie haben zweifellos ein
sehr subtiles logisches Programm der „Klärung“ der Wissenschaftssprache
entwickelt, um jene Normen klar anzugeben, die ein Wissenschaftler einhalten
muss, der sich mit der Produktion von wissenschaftlichem (positivem) Wissen
beschäftigt.

Das von ihnen vorgeschlagene Wissenschaftsmodell ist einfach und überzeugend:
Theorie entsteht durch induktive Verallgemeinerung und muss der Überprüfung
durch empirische Erfahrung unterzogen werden. Wie ist das zu machen? Die
Theorie muss arbeiten, d.h. Vorhersagen machen, die in Beobachtungen auflösbar
sind. Wenn die Schlussfolgerungen der theoretischen Berechnungen mit den Daten
der experimentellen Überprüfung übereinstimmen, ist die Theorie „wahr“, d.h.
korrekt. Dies ist die berühmte These über die Verifizierung wissenschaftlicher
Aussagen. Was sich prinzipiell nicht verifiziert lässt, ist
unwissenschaftlich.  Also müssen ohne jeden Zweifel derartige Aussagen
(Hypothesen oder Behauptungen) aus dem System wissenshaftlichen Wissens zu
entfernt werden. Das Verifikationskriterium ist, wie man sagt, streng, aber
gerecht! Es entspricht vollständig der Praxis wissenschaftlicher Forschung,
faktisch jeder, der in der Wissenschaft arbeitet, stimmt dem zu.

Und dass die Verifizierung das Wesen, der Höhepunkt der wissenschaftlichen
Suche ist, wird auf eigene Weise in Sinclair Lewis' berühmtem Roman
„Arrowsmith“ (Nobelpreis 1930) ausgedrückt. Der Prototyp der Hauptfigut war
bekanntlich der berühmte Mikrobiologe Paul de Kruif (er war wahrscheinlich
auch ein „Berater“ des Schriftstellers). Daher sollten die Worte der
Hauptfigur Martin Arrowsmith beherzigt werden.

Martins Mentor, der deutsche Bakteriologe Max Gottlieb, gibt seinem Mentee
Ratschläge über die \emph{Religion des Wissenschaftlers}: „Wissenschaftler zu
sein ist nicht nur eine besondere Art von Arbeit.  Es ist nicht so, dass eine
Person einfach wählen kann, Wissenschaftler oder Reisender, Händler, Arzt,
König, Landwirt zu werden. Das ist ein Geflecht von sehr vagen Emotionen, wie
Mystizismus oder das Bedürfnis, Gedichte zu schreiben; es macht sein Opfer
vollkommen verschieden von einem normalen, ordentlichen Menschen. Ein normaler
Mensch kümmert sich wenig um das, was er tut. Hauptsache, die Arbeit
ermöglicht es ihm, zu essen, zu schlafen und zu lieben. Der Wissenschaftler
aber ist tief religiös, so religiös, dass er Halbwahrheiten nicht akzeptieren
mag, weil sie seinen Glauben beleidigen."\footnote{Lewis Sinclair:
  Arrowsmith. Moskau, 1998. S. 308.} Und Martin teilt dieses religiöse Gefühl.
Er betet buchstäblich, bevor er seine eigenen experimentellen Studien beginnt:
„Gott, gib mir einen unverschwommenen Blick und bewahre mich vor Eile. Gott,
gib mir Ruhe und gnadenlose Verachtung gegenüber allem Schein, gegenüber
Arbeit zum Schein, gegenüber wenig ernster oder unvollendeter Arbeit. Gott,
gib mir die Unruhe, nicht zu schlafen und kein Lob zu hören, bis ich sehe,
dass die Schlussfolgerungen aus meinen Beobachtungen und die Ergebnisse meiner
Berechnungen zusammenpassen, oder bis ich mich in demütiger Freude öffne, um
meine Fehler zu entlarven. Gott, gib mir die Kraft, nicht an Gott zu
glauben!“footnote{Ebenda, S. 325.} Dies ist die Formulierung des
Verifikationsprinzips: \emph{Die Schlussfolgerungen aus meinen Beobachtungen
  und die Ergebnisse meiner Berechnungen passen zusammen}. Ist das nicht die
Hauptnorm, die in der wissenschaftlichen Praxis umgesetzt werden sollte?  Nur
so lassen sich „Halbwahrheiten“ vermeiden, die beleidigend sind für den
Glauben eines Wissenschaftlers, der nach positivem, authentischem,
zuverlässigem Wissen über die Welt strebt!

Man kann sagen, dass das einfache Modell der logischen Positivisten
gleichzeitig zutiefst romantisch war, was oft vergessen wird, wenn man sich in
deren reichen „instrumentellen Teil“ der logischen Verifikation
wissenschaftlicher Urteile vertieft, der von ihnen entwickelt wurde.

Man kann auch behaupten, dass das Wissenschaftskonzept der logischen
Positivisten die Grundlage aller nachfolgenden Bewegungen gelegt hat und für
immer seine Bedeutung behalten wird als „Startperiode“. Die Idee der
Verifizierung war zweifellos normativ und in diesem Sinne auch methodologisch.
Wobei es nicht darum ging, irgendwelche konkreten wissenschaftlichen Probleme
zu lösen, bei denen der Wissenschaftler eine „Hilfestellung“ des Philosophen
benötigte. Es ging mehr um die \emph{Methodologie der Wissenschaft} im Ganzen.
Mit anderen Worten, welchen Weg der Wissenschaftler bei seiner Suche auch
einschlägt, er muss letztendlich erreichen, dass die Schlussfolgerungen seiner
theoretischen Berechnungen mit den Daten der experimentellen Beobachtung
übereinstimmen.

Dennoch wurde die Idee der Verifikation und das entsprechende Bild der
Wissenschaft fast sofort, bereits 1934, von Karl Popper heftig kritisiert.
Seine Grundidee (man kann sagen: Gegen-Idee), bestand bekanntlich darin, dass
als Hauptkriterium von Wissenschaftlichkeit nicht eine Bestätigung der Theorie
betrachtet werden sollte, sondern die Möglichkeit ihrer Widerlegung
(Falsifikation).  Gerade im rücksichtslosen Test einer Theorie auf
experimentellen Daten ist die Widerlegbarkeit und damit die Entwicklung einer
neuen Theorie begründet, die danach strebt, den entdeckten „Fehler“ zu
vermeiden. Das war ein völlig neues Konzept von Wissenschaft, ein grundlegend
neues Bild.

Man muss den logischen Positivisten Respekt zollen, dass sie die Monografie
dieses damals wenig bekannter Autors in ihrer angesehenen Reihe
„Wissenschaftlichen Weltauffassung“\footnote{Popper Karl. Logik der
  Forschung. Wien, 1935. -- Popper wies später darauf hin, dass das Buch 1934
  veröffentlicht wurde, der Verlag habe bei der Veröffentlichung einen Fehler
  gemacht.} veröffentlichten. Aus diesem Grund wurde Popper oft unangemessen
als Positivist bezeichnet (insbesondere durch die Frankfurter Schule), was ihn
stets wütend werden ließ. Er schrieb darüber: „Dieses alte Missverständnis
wurde von Menschen geschaffen und perpetuiert, die meine Arbeit nur aus
zweiter Hand kennen. Dank der Toleranz einiger Mitglieder des Wiener Kreises
wurde mein Buch „Logik der Forschung“, in dem ich diesen positivistischen
Kreis von einem realistischen und antipositivistischen Standpunkt aus
kritisiere, in einer Buchreihe veröffentlicht, die unter der Redaktion von
Moritz Schlick und Philip Frank herausgegeben wurde, den beiden führenden
Mitgliedern dieses Kreises, und diejenigen, die es gewohnt sind, Bücher nach
Einbänden (oder Herausgebern) zu beurteilen, haben den Mythos geschaffen, dass
ich angeblich zum Wiener Kreis gehörte und ein Positivist war.  Keiner, der
dieses Buch (oder eines meiner anderen Bücher) gelesen hat, wird dem
zustimmen, außer er glaubt von Anfang an an diesen Mythos; in diesem Fall wird
er sicherlich Beweise seines Glaubens finden.“\footnote{Popper K. Geist oder
  Revolution? Evolutionäre Erkenntnistheorie und Logik der
  Sozialwissenschaften.  Moskau, 2000. S. 316.} Popper ging davon aus, dass
gerade er der „Totengräber“ des Konzepts der Verifikation und anderer
Postulate der logischen Positivismus war, obwohl er den Wiener Club selbst und
seine philosophische Arbeitsweise hoch schätzte. „Der Wiener Kreis“,
präzisierte Popper seine Position, „bestand aus Menschen, die sich durch
Originalität und eine hohe intellektuelle und moralische Ebene auszeichneten.
Sie waren nicht alle 'Positivisten', wenn man unter dem Begriff die
Verurteilung spekulativen Denkens versteht, obwohl es die meisten von ihnen
waren.  Ich stand immer für spekulatives Denken, das für Kritik offen ist, und
natürlich für Kritik an ihm“.\footnote{Ebenda, S. 325.}

Aus diesem Grund kann man davon ausgehen, dass gerade Popper die Periode
eröffnet, die gemeinhin als „Postpositivismus“ bezeichnet wird. Aber auch
dieser Terminus ist wieder nur ein historisch-philosophischer „Marker“ und
nicht irgendein neues Programm philosophisch-methodologischer Forschung.

Die zweite Periode -- der „Postpositivismus“ -- ist heterogen, in ihm
zeichneten sich schnell zwei „Flügel“ ab. Die Hauptfigur war natürlich Popper,
der nach dem Zweiten Weltkrieg nach Großbritannien übersiedelte, wo er eine
Stelle an der London School of Economics and Social Sciences erhielt. Er schuf
seine eigene Richtung, den „kritischen Rationalismus“. Das war bereits ein
Aktionsprogramm, nach dem Befürworter des „kritischen Rationalismus“ sich
erfolgreich und enthusiastisch der Ausarbeitung neuer Probleme widmeten. Die
prominentesten Figuren in dieser Richtung waren Imre Lakatos, Paul Feyerabend,
Joseph Agassi.  Poppers ursprüngliche Idee -- Falsifikationismus als
Wachstumspunkt wissenschaftlichen Wissens -- wurde auf Material zur Geschichte
der Mathematik („Evidenz und Widerlegung“ von Lakatos), zur Geschichte der
Physik und Astronomie (P. Feyerabend), in mehreren Werken von J. Agassi
entwickelt, der betonte, dass das Hauptthema der neuen Methodologie der
Wissenschaft, anders als im Konzept des logischen Positivismus, „Wissenschaft
in Bewegung“ („science in flux“) sei. Die Bedeutung des Popperschen
Standpunktes betonend, schrieb Agassi, dass der primäre intellektuelle Wert
keineswegs in der Stabilität und Nachhaltigkeit der wissenschaftlichen
Erkenntnisse liege, wie seit Jahrhunderten angenommen. „Einer der wenigen
Philosophen, die sich dieser allgemein akzeptierten Ansicht widersetzten, ist
K. Popper. Seiner Meinung nach liegt der primäre Wert der Wissenschaft in
ihrer Aufnahmefähigkeit, ihrem offenen Charakter, in der Tatsache, dass jede
ihrer Leistungen zu jeder Zeit verworfen werden kann und neue Ergebnisse
veraltete ersetzen können. Wissenschaft, sagt Popper, ist ständiger Kampf mit
sich selbst, und sie bewegt sich vorwärts durch Revolutionen und innere
Konflikte.“\footnote{Agassi J. Wissenschaft in Bewegung. Struktur und
  Entwicklung der Wissenschaft. Moskau, 1978. S. 121.}

Der dramatischste Moment in der Entwicklung des Postpositivismus war natürlich
die Polemik der Popperianer mit dem Konzept der „normalen Wissenschaft“ von
Thomas Kuhn.  Das Symposium im Jahr 1965 in London kam dank der aktiven
Initiative von Lakatos in London zustande und führte zu einer Reihe von
signifikanten Ergebnissen. Es zeigte, dass eine richtig (im Sinne von Popper)
organisierte kritische Diskussion immer zum „Wachstum von Wissen“ führt, auch
wenn es um eine recht heterogene Gesamtheit philosophischee und
methodologischer Forschung geht\footnote{Die Materialien dieses Symposiums
  wurden nach 5 Jahren veröffentlicht -- wahrscheinlich aus dem Grund, dass
  Lakatos als verantwortlicher Herausgeber dieses Buches zunächst das neue
  (eigene) Konzept der Wissenschaft fertigstellen wollte.  Siehe: Criticism
  and the Growth of Knowledge. Ed. by Imre Lakatos, Alan Musgrave. Cambridge,
  1970.}. Als Ergebnisse können in erster Linie genannt werden die Entstehung
einer „Methodologie wissenschaftlcher Forschungsprogramme“ von Lakatos, der in
einer Spitze gegen Kuhn eine neue Konzeption vorlegte, die Poppers
ursprüngliche „naive Falsifizierung“ wesentlich modifizierte, sowie die
Entwicklung des neuen methodologischen „Proliferationsprinzips“ von
Feyerabend.  Die kritische Analyse des zu vieldeutigen Paradigmenkonzepts
durch Margaret Masterman brachte Kuhn dazu, diese Idee aufzugeben und durch
dem Begriff der „disziplinären Matrix“ zu ersetzen. Kuhn erläuterte die Motive
seines „Rückzugs“ in einem sehr substantiellen „Postscriptum von 1969“.  Das
„Postscriptum“ ist seither obligatorischer Teil jeder Veröffentlichung seiner
„Struktur wissenschaftlicher Revolutionen“.

Somit kann man die stattgefundene Diskussion als vollständig fruchtbar
bezeichnen.

Wahrscheinlich gab es von diesem Moment an die bewusst erkannte Notwendigkeit,
die sogenannten \emph{Hard} und \emph{Soft} Methodology zu unterscheiden.  Die
Strenge der methodischen Vorschriften -- ausgehend vom Programm der logischen
Positivisten wie auch der Popperianer -- wurde begonnen aufzuweichen. LaKatos
-- einer der eifrigsten Anhänger des kritischen Rationalismus -- behauptet in
seiner „Methodologie wissenschaftlicher Forschungsprogramme“, dass selbst wenn
begründet festgestellt werden kann, dass sich das Forschungsprogramm in einem
Zustand der Stagnation befindet, dass in ihm keine progressiven
Problembewegungen stattfinden, die zu einem Wachstum empirischer Evidenz
führen, der Methodologe dennoch nicht strikt in seinen Empfehlungen sein soll.
Er schrieb, dass der Methodologe nur ehrlich die „Konten“ der konkurrierenden
Programme führen kann, aber nicht, um anzugeben, welcher Konkurrent den
unumstrittenen Sieg davontragen wird. „Es ist niemals die Möglichkeit
ausgeschlossen“, schreibt er, „dass sich irgendein Teil eines regressierenden
Programms rehabilitieren wird“\footnote{Lakatos I. Ausgewählte Werke zur
  Philosophie und Methodologie der Wissenschaft. Moskau, 2008. S. 414.}. Er
betont wiederholt die Notwendigkeit einer „methodologischen Geduld“, welche
die dogmatische Strenge der „Verifizierer“ ebenso wie der „Falsifizierer“
verneint\footnote{Ebenda, S. 410.}. 

Die Zerstörung der Strenge der methodologischen Regeln als Weg zum Erfolg,
vollendete natürlich Feyerabend mit seinem berühmten Buch „Wider den
Methodenzwang“. Er verkün\-dete, dass letztlich zur Lösung eines kreativen
Problems „alles möglich ist“ („anything goes!“) und Träume vom
unerschütterlichen Festhalten an einer universellen wissenschaftlichen Methode
eine leere Sache ist. In der ihm eigenen Art behauptet Feyerabend rauflustig:
„...Es wird deutlich, dass die Idee einer starren Methode oder starren Theorie
der Rationalität auf einer allzu naiven Vorstellung vom Menschen und seinem
sozialen Umfeld beruht. Wenn man umfangreiches historisches Material
analysiert und nicht versucht, es um seiner eigenen niederen Instinkte willen
oder wegen des Strebens nach intellektueller Sicherheit durch den Grad der
Klarheit, Genauigkeit, 'Objektivität', 'Wahrhaftigkeit' zu 'säubern', dann
stellt sich heraus, dass nur \emph{ein} Prinzip gibt, das unter allen
Umständen und auf \emph{allen} Stufen der menschlichen Entwicklung verteidigt
werden kann -- \emph{alles ist erlaubt}“\footnote{Feyerabend P. Ausgewählte
  Arbeiten zur Methodik der Wissenschaft. Moskau, 1986.  S. 158–159.}.  Wie
provokativ das auch klingen mag, die Geschichte der Wissenschaft bestätigt
faktisch eine solche radikale Einschätzung. Je größer die Menge des
historischen und wissenschaftlichen Materials war, die analysiert wurde, desto
schneller verflüchtigte sich die Vorstellung, dass sich die vielfältigen Wege
der wissenschaftlichen Suche auf einen einzigen methodischen „Nenner“
reduzieren lassen.

All diese Modifikationen fanden jedoch im Rahmen des Popperismus statt, der,
wie oben erwähnt, nur einen Pol, einen „Flügel“ der gegebenen Periode
darstellt.

Der andere „Flügel“ des Postpositivismus, der mangels eines besseren Namens
als „historisch-soziologische“ Richtung bezeichnet wird und deren gewichtigste
Vertreter T. Kuhn und M. Polanyi sind, die mit dem „kritischen Rationalismus“
selbst in einer solchen „abgeschwächten“, subtileren Version nicht
einverstanden waren. Für sie stand die Hauptfrage ganz anders.  Um es kurz zu
sagen, ist das Wesen der Sache, dass es notwendig ist, nicht einen normativen,
sondern einen deskriptiven Zugang zur Analyse der Wissenschaft selbst und
ihrer Geschichte zu entwickeln. Und dies ist eine völlig andere Position, ein
anderer Ausgangspunkt. Beide versuchten, das Blatt der Forschung zu wenden,
indem sie auf diejenigen Aspekte der wissenschaftlichen Praxis verwiesen, die
in die Konzepte der Popperianer einfach nicht assimiliert werden konnten.

\section*{Normativ und Deskriptiv:\\ Die „Kopernikanische Revolution“ von
  Thomas Kuhn}

Einen ungewöhnlichen und immer noch nicht vollständig gemeisterten Schlag
gegen die „kritische“ Methodologie der Wissenschaft führte Michael Polanyi mit
seinem Konzept des „impliziten Wissens“ (tacit knowledge) aus. Diese
Konzeption entstand außerhalb jeglichen Einflusses von Kuhn und unabhängig von
ihm. Polanyi ist ein weltberühmter Chemiker, der die Professor für
Sozialwissenschaften an der Universität von Manchester seir 1946 inne hat.
Sein berühmtes „explosives“ Buch „Personal Knowledge“ (mit dem Untertitel „Auf
dem Weg zu einer post-kritischen Philosophie“) erscheint 1958. Kuhn -- ein
theoretischer Physiker der Ausbildungnach, Harvard-Absolvent -- befasst sich
mit Wissenschaftsgeschichte und veröffentlicht sein erstes Buch „Die
Kopernikanische Revolution“ (1957). Seine „Struktur wissenschaftlicher
Revolutionen“ (1962) ist in vielerlei Hinsicht eine Verallgemeinerung des in
der ersten Arbeit analysierten Materials. Gerade die Geschichte des
Kopernikanismus an sich illustriert perfekt die Besonderheiten
wissenschaftlicher Revolutionen als Wechsel von „Paradigmen“. Später verweist
Kuhn in seiner „Struktur“ mehrmals auf „implizites Wissen“ als wichtigen
Faktor für die Einheit der Mitglieder dieses oder jener
Wissenschaftsgemeinschaft.

Beide kennen die Wissenschaft nicht nur vom Hören-Sagen, beide zeichnet ein
aufrichtiges und tiefes Interesse an wissenschafts-historischer Forschung aus.
Im Geiste ihres Strebens sind Kuhn und Polani zweifellos Gleichgesinnte.

Das Wichtigste, was Polanyi mit großer Überzeugungskraft zeigen konnte, ist,
dass Wissen, wissenschaftliches eingeschlossen, sich nicht auf ein System von
Aussagen reduzieren lässt, nicht als rein semiotisches Objekt betrachtet
werden kann. Das wahre Geheimnis aller professionellen Meisterschaft,
wissenschaftliches Erkennen eingeschlossen, ist „implizites Wissen“, etwas,
das sich nicht in Worten, Formulierungen, in einem System strenger „Sätzen“
ausdrücken lässt. Es ist nicht überraschend, dass ein Arzt und Chemiker wie
Polanyi in der Lage war, solche Merkmale seines Berufs zu fixieren. Er
schrieb: „Allerdings ist die große Menge an Studienzeit, die die Studenten
Chemiker, Biologen und Mediziner praktischen Übungen widmen\footnote{Und diese
  Liste lässt sich klar fortsetzen: Geologen, Bodenkundler, Geographen,
  Physiker und Archäologen... ja, gibt es überhaupt Ausnahmen von dieser
  Regel?!}, bezeugt die wichtige Rolle, die der Transfer praktischer
Fertigkeiten und Fähigkeiten vom Lehrer zum Schüler in diesen Disziplinen
spielt.  Aus dem Gesagten kann man folgern, dass im Herzen der Wissenschaft es
Bereiche des praktischen Wissens gibt, die durch Formulierungen unmöglich
vermittelt werden können.“\footnote{Polanyi M. Persönliches Wissen.  Moskau,
  1985. S. 89.} Beliebige Formulierungen und Definitionen verschieben den
Bereich des versteckten „Schweigens“, aber sie heben ihn nie auf.

Wissen hat also außersprachliche Charakteristika. Und das tötet die Ansprüche
jeder logisch-analytischen Tradition darauf, die grundlegende Frage --
\emph{Was ist Wissen?} beantworten zu können.  „Aussagen“ erscheinen als
Spitze eines Eisbergs, dessen großer, unter Wasser liegender Teil in der
Sphäre des „tacit knowledge“ verborgen ist.

Was die Arbeit von Thomas Kuhn betrifft, so erwies sich das
„Paradigmenwechsel-Modell“ im Zentrum einer solch ohrenbetäubender Kritik,
dass Zeitgenossen relativ lange faktisch nicht einmal mit der Analyse
derjenigen Ergebnisse begannen, die ihnen zugänglich waren.  Es hat lange
gedauert, bis die Leidenschaften abgekühlt waren und man relativ
leidenschaftslos seine Ergebnisse beurteilen konnte.

1997 gab es am Institut für Philosophie der Russischen Akademie der
Wissenschaften eine Sitzung eines „runden Tischs“, dem Gedenken an T. Kuhn und
sein philosophisches Erbe gewidnet. Ich möchte auf den Auftritt von M.A. Rozov
mit der Bewertung der Leistungen dieses Autors eines neuen, „post-kritischen“
Modells der Wissenschaft aufmerksam machen. Nach seinen Worten kann Thomas
Kuhn, was aus der Distanz der seit 1962 vergangenen Zeit deutlich sichbar sei,
durchaus als eine Person betrachten, die eine „kopernikanische Wende“ im
Studium der Wissenschaft vollzogen hat. Vor Kuhn waren Philosophie und
Methodologie der Wissenschaft faktisch nicht unterscheidbar. Die Aufgabe der
Erforschung (Beschreibung) der Mechanismen der Entwicklung der Wissenschaft
unterschied sich nicht von der Aufgabe, jene methodischen Regeln zu
formulieren, die in der Lage sind, die Wissenschaft zu befördern. Kuhns
„Wende“ bestand darin, dass die Wissenschaftsphilosophen von der
\emph{Modalität des Sollens} (\foreignlanguage{russian}{модальность
  долженствования}), die charakteristisch ist für die Formulierung von Normen
und Regeln, zur \emph{Modalität der Existenz}
(\foreignlanguage{russian}{модальность существования}) übergegangen sind,
d.h. sie haben ihr Verständnis der eigenen Forschungsposition und der
Aufgaben, die sich aus diesem Verständnis ergeben, geändert. Kuhn hat gezeigt,
dass Wissenschaftler in ihrer Tätigkeit (als Mitglieder der wissenschaftlichen
Gemeinschaft) durch bestimmte Traditionen (Programme) bestimmt sind, und die
Aufgabe besteht darin, diese Programme zu rekonstruieren und Mechanismen für
ihre Veränderung herauszuarbeiten. Er zeigte, dass Programme (Paradigmen) in
etwa der gleichen Weise gegeben sind wie die Sprache. Das ermöglichte es
insbesondere zu erkennen, dass die Wissenschaftsphilosophie in jenes Spektrum
der sozial- und geisteswissenschaftlichen Disziplinen einbezogen ist, die
solche Klassen von Phänomenen wie Sprache, Sprichwörter, kulturelle
Traditionen, soziale Stafetten und dergleichen untersuchen\footnote{Die
  Ausführungen von M.A. Rozov sind dokumentiert in: Philosophie der
  Naturgeschichte des 20. Jahrhunderts: Ergebnisse und Perspektiven
  (\foreignlanguage{russian}{Философия естествознания ХХ века: итоги и
    перспективы. (Материалы к Первому Всероссийскому Философскому Конгрессу
    «Человек–Философия–Гуманизм»). — М., ИФРАН, 1997. — С. 41–42.  }).}.

Auf die Unterscheidung vom „Normativem“ und „Deskriptivem“ in der
Epistemologie hat bereits E. Agazzi aufmerksam gemacht: „Wenden wir uns nun
der Epistemologie zu (verstanden als allgemeine Theorie der Erkenntnis), ist
zu beachten, dass sie immer zwei Aspekte umfasst, die man als deskriptiv und
normativ bezeichnen kann. Der normative Aspekt ist vorläufig, denn er besteht
erstens aus einer irgendwie gearteten Definition des Begriffs Wissen,
d.h. einer ausreichenden Präzisierung dessen, \emph{was Wissen ist}, was
zweitens auch bestimmt, welche Anforderungen von dem erfüllt sein \emph{muss},
was wir gerne als Wissen qualifizieren würden.

Der deskriptive Aspekt besteht darin, herauszufinden, \emph{wie} Wissen
erworben wird, in welchen Schritten, unter welchen Bedingungen und nach
welchen Kriterien man sich von seinem Erhalt überzeugen kann, und dies
begüglich der verschiedenen Gegenstände, die wir uns wünschen zu wissen.
Natürlich sind diese beiden Aspekte nur analytisch unterscheidbar, im
Konkreten bedingen sie sich gegenseitig.“\footnote{Agazzi E. Erkenntnistheorie
  und Soziales: Eine Feedback-Schleife.  Fragen der Philosophie. 2010,
  №7. S. 64.}

Obwohl ich die geäußerten Meinung voll und ganz teile, kann ich dennoch
folgender These von Agazzi nicht zustimmen (genauer gesagt, halte die folgende
Aussage für wesentlich unpräzise): „Die Geschichte der Epistemologie kann als
kontinuierliche Vertiefung und Erweiterung ihres deskriptiven Aspekts gesehen
werden, in dessen Verlauf Funktionen, Instrumente, Kriterien, Möglichkeiten
des Wissenserwerbs entdeckt und unter dem Gesichtspunkt ihrer Gesundheit
(normativer Aspekt) beurteilt worden sind“\footnote{Ebenda}.

Warum erscheint mir die oben gegebene Bewertung der Beziehung zwischen
Deskriptivem und Normativem nicht korrekt? Kuhns Modell der
Wissenschaftsparadigmen wurde von den Popperianern als Verteidigung eines
nicht-kreativen Wissenschaftlers gewertet. „Beruhigung für einen Spezialisten“
bezeichnete P. Feyerabend das Kuhnsche „Paradigma“; „normale Wissenschaft“
existiert natürlich, aber sie ist nach Poppers Meinung gefährlich; Lakatos
nannte das Konzept Kuhns irrational\footnote{Siehe „Criticism and the Growth
  of Knowledge“; Popper K. „Normale Wissenschaft und die mit ihr verbundenen
  Gefahren“; Kuhn T. „Struktur wissenschaftlicher Revolutionen. Moskau, 2001.
  S. 525-538; Feyerabend P. „Comfort for Specialists“. Ausgewählte Arbeiten
  zur Methodik der Wissenschaft. Moskau, 1986. S. 109–124.}...

Kuhn, versuchte ehrlich, aufrichtig und freimütig, seine prinzipiellen
Differenzen mit Popper zu erläutern und schrieb ausdrucksvoll: „Ich bezeichne
das, was uns trennt, eher als Gestaltwandel denn als Meinungsverschiedenheit,
weshalb ich gleichzeitig verwirrt und fasziniert bin, wie man diese unsere
Differenzen besser erklären kann.  Wie überzeuge ich Sir Karl, der alles das
weiß, was ich über die Entwicklung von Wissenschaft weiß, und irgendwie schon
etwas dazu gesagt hat, davon, dass der Gegenstand, den er Ente nennt, von mir
Kaninchen genannt wird? Wie zeige ich ihm das, was ich durch meine Brille
sehe, wenn er bereits gelernt hat, auf alles durch seine Brille zu schauen,
was ich ihm zeigen könnte?“\footnote{ Kuhn T. Logik der Entdeckung oder
  Psychologie der Forschung; Kuhn T. Struktur wissenschaftlicher Revolutionen.
  Moskau, 2001. S. 543.}

Nein, die Entdeckung einer deskriptiven Position in der Philosophie der
Wissenschaft rief aufrichtig Unverständnis unter den Methodologen
(Normativisten) hervor, und dem entspricht das wirklich prinzipielle
Auseianderfallen der Positionen. Die Ausgangspositionen erzeugen
unterschiedliche „Brillen“, durch die das wissenschafts-historische Material
betrachtet wird, das die Richtigkeit des Ansatzes beweisen soll. Es war
wichtig, für sich selbst die Unterscheidung zwischen „Normativem“ und
„Deskriptivem“ zu erkennen, ohne zu versuchen, das eine mit dem anderen zu
vermengen.

\section*{Trajektorien der Suche in der\\ Sowjetischen
  Wissenschaftsphilosophie der Nachkriegszeit}

Die Versuche der sowjetischen philosophischen Jugend jener Zeit sind hoch zu
würdigen, die einerseits versuchte, die Erfahrungen der Erörterung der
methodologischen Thematik im Westen zu berücksichtigen, sich diese Tradition
anzueignen, und sich andererseits mutig der Analyse wissenschaftlicher Praxis
zuzuwenden, um nicht unsubstanziiert zu sein, und zu erkennen, welche realen
methodologischen Probleme Wissenschaftler bewegen, die an der vordersten Front
ihrer Disziplinen arbeiteten.

In Moskau wurde unter der Leitung von G.P. Shchedrovitsky der Moskauer
Methodologische Kreis gegründet, in Minsk das Seminar unter Leitung von
V.S. Stepin und in Novosibirsk das Seminar von M.A. Rozov. Alle haben ihre
Aufgaben ungefähr gleich verstanden; es gilt, an konkretem
wissenschaftsgeschichtlichem Material zu arbeiten, und auch, wenn möglich,
methodologische Probleme zu diskutieren, die von der wissenschaftlichen
Gemeinschaft selbst fixiert werden. Zur gleichen Zeit arbeitete am Institut
für Philosophie der Akademie der Wissenschaften der UdSSR aktiv eine Gruppe
junger Logiker, die sich um W.A. Smirnow und E.D. Smirnowa scharte. Diese
Gruppe konzentrierte sich, ohne ihre Absichten zu verbergen, auf die moderne
westliche Logik (in der Tat glaubten sie, wie die Mathematiker, dass die
formale Logik keinerlei Abgrenzungsmerkmale haben kann). Heute, mit dem
Abstand der Zeit, kann man behaupten, dass gerade diese Gruppe das
Forschungsprogramm des Wiener Kreises seriös aufnahm und nach ihren Kräften
einen Beitrag für dieses „internationale Sammelbüchse“ leistete.

In der Tat, all die oben genannten Helden unserer vaterländischen Philosophie
waren hinreichen vertraut mit den Werken des Wiener Kreises (soweit dies unter
den Bedingungen möglich war, dass sich die Hauptwerke des logischen
Positivismus im „Giftschrank“ der Bibliothek befanden). Die Kritik der
positivistischen Philosophie war eines der Schlüsselthemen der
methodologischen Studien der damaligen Zeit. Und während sich Stepin,
Shchedrovitsky und Rozov kritisch zu den Möglichkeiten dieses
Forschungsprogramms selbst äußerten, ging Smirnows Gruppe im Kern der Sache
zustimmend an die Aufgabe der logischen Analyse der Sprache der Wissenschaft.

Dafür wurden die Werke des Postpositivismus, angefangen mit den Werken Karl
Poppers, aber auch der Anhänger des „kritischen Rationalismus“ und ihrer
Rivalen Thomas Kuhn und Michael Polanyi relativ schnell bekannt, im Maße der
allmählichen Infiltration der englischsprachigen philosophischen Literatur in
unsere Gesellschaft.  Schließlich waren die 1960er Jahre die Zeit von
Chruschtschows „Tauwetter“. Diese Werke wurden -- sozusagen „privat“ --
übersetzt, die Referate der Arbeiten wurden aufmerksam studiert, die
systematisch von INION\footnote{\foreignlanguage{russian}{Институт научной
    информации по общественным наукам} -- Institut für wissenschaftliche
  Information in den Gesellschaftswissenschaften} herausgegeben wurden. Es sei
darauf hingewiesen, dass G.P. Shchedrovitsky in den späten 60er Jahren sogar
eine kleine persönliche Korrespondenz mit I. Lakatos hatte, und über „Beweise
und Widerlegungen“ für die Zeitschrift „Fragen der Philosophie“ eine sehr
interessante Rezension schrieb\footnote{\foreignlanguage{russian}{Щедровицкий
    Г.П. Модели новых фактов для логики. Вопросы философии. 1968. №4}. --
  Shchedrovitskiy, G.P. Modelle neuer Fakten in der Logik.}.

Das Bild der Suche im Rahmen des Postpositivismus gestaltete sich mehr oder
weniger angemessen.  Trotzdem bevorzugten unsere Helden, die ideellen
Vordenker jener Zeit, „ihren eigenen Weg“. Und es hat damit zu tun, dass sie
sich nie zur Aufgabe stellten, nur einfach Historiker (oder Kritiker) der
modernen ausländischen Philosophie zu sein, und bezüglich der Möglichkeiten,
mittels formaler Logik neue Aufgaben bei der Untersuchung von Phänomenen der
Wissenschaft und ihrer aktuellen Probleme lösen zu können, waren sie sehr
skeptisch.

Rückblickend ist auch deutlich sichtbar, dass „Ideen in der Luft hingen“, dass
intellektuelle Suchen den Eisernen Vorhang erfolgreich überwunden haben. Es
wurde in unserer Gemeinschaft im Geiste des Wiener Kreises gearbeitet; es gab
„kritische Rationalisten“ mit ihrer aktiven Ablehnung bestehender Praktiken
des wissenschaftlichen Denkens (dazu gehören natürlich in erster Linie
Shchedrovitskys Schüler, die „in die Welt traten, um zu widersprechen“); es
gab auch Vertreter der „deskriptiven Position“ im Sinne von Thomas Kuhn, die
die Wissenschaft nicht transformieren, sondern nur studieren wollten (das
waren M.A. Rozov und sein Schülerkreis). V.S. Stepin, so scheint uns, begann
im Geiste einer deskriptiven Position, im Fahrwasser historischer
Rekonstruktionen (er untersuchte die Geschichte von Maxwells Elektrodynamik),
er folgte dabei den Wegen von Kuhn und Polanyi, versuchte aber schließlich,
die beiden Ansätze zu vereinen, wie aus seinem
Resultate-Buch\footnote{\foreignlanguage{russian}{Степин В.С. Теоретическое
    знание. — М., 2000.}.} hervorgeht.

\section{Die System-Aktivitäts-Methodologie von G.P. Shchedrovitsky und die
  Untersuchung des Phänomens des „Methodologischen Denkens“ von M.A. Rozow}

Es ist sehr interessant zu sehen, wie in den Reihen der nationalen der
philosophischen Gemeinschaft haben zwei Projekte gebildet, um die Methodologie
als besondere Aktivität, zwei Möglichkeiten zum Verständnis ihrer
Hauptziele. Dies ist .  G.P. Shchedrovitsky und M.A. Rozov trennten sich im
Prinzip, obwohl sie am Ende tatsächlich ein allgemeines "Rahmen"-Bild der
Ausstrahlung und Wiedergabe der sozialen des Ganzen.

Beide glauben, dass keine semiotischen Phänomene (einschließlich Wissen)
kann nicht außerhalb des umfassenden Kontextes der sozialen
des Ganzen. Eine solche Ontologie war für den Wiener Kreis unmöglich, solche Breitengrade...
das ontologische Bild fehlte im Postpositivismus. Und die häufigste...
sozialen Prozess (so wie es beim Leben vor allem um Reproduktion geht).
des Selbst) sind die Prozesse der Übersetzung verschiedener Komponenten der Aktivität und
weitere Replikation bestehender Prototypen. Aber die einfache
Das "Einfließen" in das neue Umfeld früherer Operationen ist nicht von Interesse.
G.P. Shchedrovitsky. Er hält es für einen zu elementaren Fall und konzentriert
seine Aufmerksamkeit auf die Tatsache, dass die Aktivitäten nach "Normen" durchgeführt werden, die
sich wie Standards verhalten. Sie sind die Grundlage für die folgenden Aktivitäten, auch...
wenn die Bedingungen für seine Umsetzung sehr unterschiedlich sind. Das ist das, was ich gesagt habe.
wird im Artikel "Natürlich" aus dem Jahr 1967 klar zum Ausdruck gebracht und "Künstlich" im Artikel "Natürlich".
von semiotischen Systemen" 23. Die Autoren sind V.A. Lefebvre, E.G. Yudin,
G.P. Shchedrovitsky. Dieser Manifestartikel erfreute auch M.A. Rozov.


\end{document}

Словосочетание, которое сегодня привычно выделяет особую, самостоятельную
область философских исследований — «методология и философия науки» — отнюдь не
так давно стало общеупотребительным. В отечественной литературе про
методологические проблемы науки, про методологию в целом в 1960–1970-е
гг. говорили столь часто и столь широко, что эта тематика казалась
исключительно навязчивой и откровенно навязанной партийными идеологическими
предписаниями. Методология должна быть правильной, следовательно, —
«марксистско-ленинской». Казалось, что участники бесконечных конференций,
семинаров и кулуарных дискуссий, а также авторы соответствующих сборников и
монографий, никогда не сумеют найти общего языка и никогда ясно не
сформулируют своих исходных позиций. Кроме того, вплоть до конца 1980-х
годов часто утверждали, что «эпистемология» и «философия науки» — это вообще
лишние термины, без нужды «умножающие сущности», так как в философии издревле
существовала сфера гносеологических (теоретико-познавательных) исследований.

История философско-методологических поисков ХХ столетия — как в западной,
так и в отечественной традициях — представляет весьма интересный и яркий
материал для анализа, осмысления и попыток представить следующие шаги и
проекты в этой сфере.

ПРИКЛЮчЕНИЯ «МЕТОДОЛОГИИ»: ОТ HARD К SOFT METHODOLOGY

«Методология» — слово крайне ответственное и, к сожалению, очень неудачное.

Конечно, это слово ответственное, оно стало таким еще со времен Фр. Бэкона,
который провозгласил необходимость «Нового Органона», и «Рассуждения о
методе» Декарта. Юный Герцен в своей выпускной университетской работе,
анализируя Коперниканскую астрономическую революцию, напоминает, что великий
Декарт говорил, когда превозносили его математические открытия: «Хвалите не
открытия, а методу»1. В чем же суть правильного метода? — вот вопрос
вопросов. Какая дисциплина может дать ответ на такой вопрос? — естественно,
методология.

Но слово это крайне неудачное, потому что энергично подталкивает к очень
простому решению: методология — это наука о методе, о правильном
методе. Сегодня, когда мы миновали весь ХХ век с его грандиозными научными
открытиями, можно смело констатировать: нет такой науки! Нет и быть ее не
может!

Именно необходимость такого вывода демонстрирует вся история сферы
профессиональных философско-методологических исследований прошлого века.

Следует при этом подчеркнуть, что представление Бэкона или Декарта о
методологии — отнюдь не совпадает с тем, что мы называем «методологией (и
философией) науки» в ХХ веке.  Это довольно важное, принципиальное
уточнение. Известный итальянский философ и логик Эвандро Агацци весьма
справедливо заметил: «Традиционные [философские] размышления не посвящались,
так сказать, “тематически” исследованиям науки или наук, а были скорее
приложением к некоторым наукам общих рассуждений, обычно в связи с теорией
познания или с онтологией, поскольку внимание, уделяемое науке, было лишь
частью значительно более широкой концепции, или “системы”, в которой
интерпретация науки находила подобающее ей место.

Современная же философия науки есть философское исследование, ограниченное
тематически почти исключительно единственным предметом — наукой (или какой-то
конкретной отраслью науки) — и использующее интеллектуальные средства,
заимствуемые из других разделов философии, но применяемые как орудия для
понимания науки и лишь в той мере, в какой они так применяются»2.

1 Герцен А.И. Собр. соч. в 30 т. Т. I. — М., 1956. — С. 36.

2 Агацци Э. Переосмысление философии науки сегодня // Вопросы
философии. 2009. №1. — С. 40.

То, что произошло в данной сфере в течение ХХ столетия, можно представить в
виде трех периодов, которые отчасти совпадают хронологически и логически:
период 20–30-х годов (вплоть до эмиграции участников Венского кружка в разные
страны после аншлюса Австрии); период 50–70-х годов (прежде всего в
англоязычной традиции с «интеллектуальным центром» в Лондоне); наконец, то,
что называют «современностью», — где представлены весьма разнообразные
программы исследований социальной обусловленности науки. Отправной точкой
последнего периода можно считать 1979 г., когда появилась книга Бруно Латура и
Стива Вулгара «Лабораторная жизнь», а также сформировалась так называемая
«сильная программа» социологии знания Дэвида Блура. По оценке Э. Агацци, про-
изошел очевидный «социологический поворот» в философии науки, причем в
какой-то момент широко распространилось представление, «согласно которому
наука есть “социальный продукт” в буквальном смысле слова, т.е. деятельность,
полностью обусловленная динамикой власти, управляющей обществом и
производящая то содержание знания и те его приложения, которые нужны
различным властям независимо от какого-либо критерия объективной
значимости»3. Безусловно, последний период уже камня на камне не оставил от
исходных постулатов логических позитивистов, впрочем, безо всяких попыток
ответить на те вопросы, которые они поставили. Между первым и вторым периодом,
напротив, существовала, несмотря на смену ключевых моделей, очень тесная и
существенная преемственная связь.

Первый период развития анализа феномена науки, выявления специфики научного
познания связан с работой Венского кружка во главе с Морисом Шликом. Это
направление имеет, по сути, 4 названия: «позитивизм третьей волны»,
«неопозитивизм» (нетрудно видеть, что это просто историко-философские
«маркеры»), Венский кружок (географическое название), и — собственное имя,
выражающее смысл предложенной программы исследований, — «логический
позитивизм».

3 Там же. — С. 45.

Трудно переоценить все сделанное логическими позитивистами. Резко
отказавшись от традиционной философской постановки вопроса о том, что
представляет собой «чистый разум», способный достигать истин высшего
порядка, они поставили вопросы на «земную почву». В центре их модели
научного познания — теория, которая понималась (в конечном счете) как
совокупность высказываний, связанных отношениями вывода.  Они, бесспорно,
развили весьма тонкую логическую программу «прояснения» научного языка для
того, чтобы ясно указать те нормы, которые необходимо соблюдать ученому,
занятому производством научного (позитивного) знания.

Модель науки, которую они предложили, проста и убедительна: теория возникает
путем индуктивного обобщения и должна быть подвергнута проверке эмпирическим
опытом. Как это сделать? Теория должна работать, т.е. делать предсказания,
разрешимые в наблюдении. Если выводы из теоретических расчетов совпадают с
данными экспериментальной проверки, то теория «истинна», т.е. верна. Таков
знаменитый тезис о верификации научных высказываний. То, что не
верифицируется в принципе, — ненаучно. Поэтому, отбросив всякие сомнения, та-
кие высказывания (гипотезы или утверждения) необходимо удалить из системы
научных знаний. Критерий верификации, как говорится, строг, но справедлив! Он
полностью соответствует практике научных исследований, с чем фактически
согласны все, кто работает в науке.

И то, что верификация — это суть, вершина научного поиска, — своеобразно
излагается в знаменитом романе «Эроусмит» Синклера Льюиса (Нобелевская
премия 1930 г.). Прототипом главного героя, как известно, был знаменитый
микробиолог Поль де Крюи (он же, вероятно, был и «консультантом» писате-
ля). Поэтому к словам главного героя Мартина Эроусмита стоит прислушаться.

Вот наставник Мартина немецкий бактериолог Макс Готлиб дает наставления своему
ученику и говорит о религии ученого: «Быть ученым — это не просто особый вид
работы, не так, что человек просто может выбирать: быть ли ему ученым, или
стать путешественником, коммивояжером, врачом, королем, фермером. Это —
сплетение очень смутных эмоций, как мистицизм или потребность писать стихи;
оно делает свою жертву резко отличной от нормального порядочного
человека. Нормальный человек мало беспокоится о том, что он делает, лишь бы
работа позволяла есть, спать и любить. Ученый же глубоко религиозен — так
религиозен, что не желает принимать полуистины, потому что они оскорбительны
для его веры»4. И Мартин разделяет это религиозное чувство. Он буквально
молится перед началом собственного экспериментального исследования: «Боже,
дай мне незамутненное зрение и избавь от поспешности. Боже, дай мне покой и
нещадную злобу ко всему показному, к показной работе, к работе расхлябанной
и незаконченной. Боже, дай мне неугомонность, чтобы я не спал и не слушал
похвалы, пока не увижу, что выводы из моих наблюдений сходятся с
результатами моих расчетов, или пока в смиренной радости не открою и ра-
зоблачу свою ошибку. Боже, дай мне сил не верить в Бога!»5. Вот и формулировка
принципа верификации: выводы из моих на2 блюдений сходятся с результатами моих
расчетов. Разве это не главная норма, которая должна быть реализована в
научной практике? Только это позволяет избегать «полуистин», которые
оскорбительны для веры ученого, стремящегося к позитивному, подлинному,
надежному знанию о мире!

Можно сказать, что простенькая модель логических позитивистов была
одновременно глубоко романтичной, о чем нередко забывают, погружаясь в
развитую ими богатейшую «инструментальную часть» логической проверки научных
суждений.

Можно также утверждать, что концепт науки логических позитивистов заложил
фундамент всего последующего движения и навсегда сохранит свое значение в
качестве «стартового периода». Идея верификация была, бесспорно, нормативной
и в этом смысле — методологической. Причем, речь не шла о том, чтобы решить
какие-то конкретные научные проблемы, где ученый нуждался бы в «подсказке»
философа. Речь шла, скорее, о методологии науки в целом. Иначе говоря, каким
бы путем ни шел ученый в своем поиске, в конечном счете он должен добиться
того, чтобы выводы из его теоретических расчетов совпадали бы с данными
экспериментального наблюдения.

4 Льюис Синклер. Эроусмит. — М., 1998. — С. 308.

5 Там же. — С. 310.

Тем не менее, почти сразу идея верификации и соответствующий ей образ науки
был жестко раскритикован Карлом Поппером еще в 1934 году. Основная его идея,
как известно, (можно сказать — контридея) состояла в том, что основным
признаком научности следует считать не подтверждение теории, а саму
возможность ее опровержения (фальсификацию). Именно в беспощадной проверке
теории экспериментальными данными возможно ее опровержение и выдвижение новой
теории, которая стремится избежать обнаруженной «ошибки». Это был со-
вершенно новый концепт науки, ее принципиально новый образ.

Надо отдать должное, что именно логические позитивисты опубликовали монографию
малоизвестного автора в своей престижной серии «Wissenschaftlichen
Weltauffssung»6. По этой причине Поппера часто неоправданно причисляли к
позитивистам (в частности, франкфуртская школа), что вызывало его подлинный
гнев. Он писал по этому поводу: «Это старое недоразумение создали и
поддерживали люди, знавшие о моих работах из вторых рук. Благодаря терпимому
отношению некоторых участников Венского кружка, моя книга Logik der
Forschung, в которой я критиковал этот позитивистский кружок с реалистической
и антипозитивистской точек зрения, была опубликована в серии книг, выходивших
под редакцией Мориса Шлика и Филиппа Франка, двух ведущих членов этого кружка,
и те, кто привык судить о книгах по обложкам (или по редакторам), создали
миф о том, что я якобы входил в Венский кружок и что я — позитивист.  Никто из
читавших эту книгу (или любую другую из моих книг) не согласится с этим, разве
только поверит в этот миф с самого начала; в этом случае он, конечно, найдет
какие-нибудь подтверждения своей веры»7. Поппер полагал, что именно он
является «могильщиком» концепции верификации и других постулатов логического
позитивизма, хотя высоко ценил сам Венский кружок и его манеру работать в
философии. «Венский кружок, — уточнял свою позицию Поппер, — состоял из людей,
отличающихся оригинальностью и высоким интеллектуальным и нравственным
уровнем. Не все они были “позитивистами”, если подразумевать под этим
термином осуждение спекулятивного мышления, хотя таких было большинство. Я
же всегда стоял за открытое для критики спекулятивное мышление и, конечно,
за его критику»8.

6 Popper Karl. Logik der Forschung. Wien. Verlag von Julius Springer.  1935. —
250 s. Поппер позднее указывал, что год издания книги — 1934, что
издательством была допущена ошибка при публикации.

7 Поппер К. Разум или революция? // Эволюционная эпистемология и логика
социальных наук. — М., 2000. — С. 316.

8 Там же. — С. 325.

Именно по этой причине можно считать, что именно Поппер открыл тот период,
который получил общее название — «постпозитивизм». Но этот термин опять-таки
— просто историкофилософский «маркер», а не какая-то новая программа фило-
софско-методологических исследований.

Второй период — «постпозитивизм» — неоднороден, в нем быстро обозначились два
«крыла». Главным действующим лицом был, конечно, Поппер, который переехал в
Великобританию после Второй мировой войны, получив должность для работы в
Лондонской школе экономики и социальных наук. Он создал свое направление —
«критический рационализм». Это была уже программа действий, согласно с которой
сторонники «критического рационализма» успешно и с энтузиазмом принялась за
разработку новых проблем. Самыми видными фигурами этого направления были Имре
Лакатос, Пол Фейерабенд, Дж. Агасси.  Исходная идея Поппера —
фальсификационизм как точка роста научного знания — была разработана на
материале истории математики («Доказательства и опровержения» Лакатоса),
истории физики и астрономии (П. Фейерабенд), в ряде работ Дж. Агасси,
который подчеркнул, что основной предмет изучения новой методологии науки,
отличный от концепции логического позитивизма,— «наука в движении» («Science
in Flux»). Подчеркивая значение попперианской точки зрения, Агасси писал,
что вовсе не стабильность и устойчивость научного знания является главной
интеллектуальной ценностью, как это считалось веками. «Одним из тех немногих
философов, которые выступили против этой общепризнанной точки зрения,
является К. Поппер. Согласно его мнению, наука ценна своей восприимчивос-
тью, открытым характером — тем, что любые ее достижения в любое время могут
оказаться отброшенными и новые результаты могут заменить устаревшие. Наука,
говорит Поппер, есть постоянная борьба с собой, и она движется вперед
благодаря революциям и внутренним конфликтам»9.

9 Агасси Дж. Наука в движении //Структура и развитие науки. — М., 1978. —
С. 121.

Самым драматичным моментом в развитии постпозитивизма была, конечно,
полемика попперианцев с концепцией «нормальной науки» Томаса Куна. Симпозиум
1965 г. в Лондоне был проведен благодаря активной инициативе Лакатоса и привел
к ряду существенных результатов. Это свидетельствовало, что правильно
организованная критическая (по Попперу) дискуссия всегда приводит к «росту
знания», даже если речь идет о довольно разнородной совокупности
философско-методологических исследований10. Результатами можно считать в
первую очередь появление «методологии научно-исследовательских программ»
Лакатоса, который в пику Куну выдвинул новую концепцию, существенно
модифицируя исходный «наивный фальсификационизм» Поппера, а также выдвижение
нового методологического «принципа пролиферации» Фейерабенда. Критический
анализ слишком многозначного понятия парадигмы со стороны Маргарет Мастерман
привела Куна к отказу от этого понятия. Взамен «парадигмы» появилась
«дисциплинарная матрица». Кун объяснил мотивы своего «отступления» в весьма
содержательном «Дополнении 1969 года». Отныне «Дополнение» обязательно
сопровождает любые публикации «Структуры научных революций».

10 Материалы этого симпозиума были опубликованы спустя 5 лет — вероятно, по
причине того, что Лакатос, будучи ответственным редактором данной книги,
срочно дорабатывал новую (собственную) концепцию науки.  См.: Criticism and
the Growth of Knowledge. Ed. by Imre Lakatos, Alan Musgrave. Cambridge, 1970.

Таким образом, состоявшуюся дискуссию можно в полной мере назвать
плодотворной.

Вероятно, с того же момента возникла осознанная необходимость различать так
называемые Hard и Soft Methodology.  Строгость методологических предписаний —
исходящая от программы логических позитивистов или от попперианцев — нача-
ла смягчаться. Лакатос — один из самых ревностных последователей
критического рационализма — в своей «методологии научно-исследовательских
программ» утверждает, что даже если можно обоснованно зафиксировать, что
исследовательская программа находится в состоянии стагнации, что в ней не про-
исходит прогрессивного сдвига проблем, ведущего к увеличению эмпирического
роста, методолог все же не должен быть жестким в своих рекомендациях. Он
писал, что методолог может только честно зафиксировать «счета» конкурирующих
программ, но не указывать, какая из конкурентов одержит бесспорную
победу. «Никогда не исключается, — пишет он, — возможность того, что
какая-то часть регрессирующей программы будет реабилитирована»11. Он
неоднократно упоминает необходимость «методологической терпимости», которая
отрицает догматическую строгость как приверженца «верификации», так и
«фальсификациониста»12.

Довершил разрушение строгости методологических правил как пути к успеху,
конечно, Фейерабенд с его знаменитой книгой «Против методологического
принуждения». Он провозгласил, что в конечном итоге для решения творческой
задачи «подходит всё» («anything goes!»), а мечты о неуклонном следовании
универсальному научному методу — дело пустое. В свойственной ему манере
Фейерабенд задиристо утверждал: «…Становится очевидным, что идея жесткого
метода или жесткой теории рациональности покоится на слишком наивном
представлении о человеке и его социальном окружении. Если иметь в виду
обширный исторический материал и не стремиться “очистить” его в угоду своим
низшим инстинктам или в силу стремления к интеллектуальной безопасности до
степени ясности, точности, “объективности”, “истинности”, то выясняется, что
существует лишь один принцип, который можно защищать при всех обстоятельствах
и на всех этапах человеческого развития, – допустимо все»13. Как бы
провокационно это ни звучало, история науки фактически подтверждает такую
радикальную оценку. Чем большее количество историко-научного материала
подвергалось анализу, тем быстрее развеивалось представление о том, что
многообразные пути научного поиска можно свести к единому методологическому
«знаменателю».

Все эти модификации происходили тем не менее в рамках попперианства, которое,
как говорилось выше, представляет только один полюс, одно «крыло» данного
периода.

11 Лакатос И. Избранные произведения по философии и методологии науки. — М.,
2008. — С. 414.

12 Там же. — С. 410.

13 Фейерабенд П. Избранные труды по методологии науки. — М., 1986. —
С. 158–159.

Другое «крыло» постпозитивизма, которое за неимением лучшего назвали
«историко-социологическим» направлением и крупнейшими представителями которого
были Т. Кун и М. Полани, не соглашались с «критическим рационализмом» даже в
таком «ослабленном», более тонком варианте. Для них главный вопрос стоял
совсем иначе. Если говорить кратко, то суть дела в том, что необходимо
развивать не нормативный, а дескриптивный подход к анализу самой науки и ее
истории. И это — совершенно другая позиция, иная исходная установка. Оба они
пытались повернуть русло исследований, указывая на те аспекты научной
практики, которые просто не могли быть ассимилированы концепциями
попперианцев.

НОРМАТИВНОЕ И ДЕСКРИПТИВНОЕ: «КОПЕРНИКАНСКАЯ РЕВОЛЮЦИЯ» ТОМАСА КУНА

Необычный и до сих пор не полностью освоенный удар по «критической»
методологии науки нанес Майкл Полани с его концепцией «неявного знания» (tacit
knowledge). Эта концепция возникала вне всяких влияний со стороны Куна и
независимо от него. Полани — авторитетный химик с мировой известностью, занял
пост профессора кафедры социальных наук в Манчестерском университете в 1946
г. Его знаменитая «взрывная» книга «Personal Knowledge» (с подзаголовком «На
пути к посткритиче2 ской философии») увидит свет в 1958 г. Кун —
физик-теоретик по образованию, выпускник Гарвардского университета — увлечен
историей науки и публикует свою первую книгу «Коперниканская революция»
(1957). Его «Структура научных революций» (1962) является во многом обобщением
того материала, который проанализирован в первой работе. Именно история
коперниканства сама по себе прекрасно иллюстрирует особенности научных
революций как смены «парадигм». Позднее в «Структуре» Кун несколько раз
будет ссылаться на «неявное знание» как на важный фактор единства членов того
или иного научного сообщества.

Оба они знают науку не понаслышке, обоих отличает искренний и глубокий
интерес к историко-научным изысканиям.  По духу своих исканий Кун и Полани,
несомненно, единомышленники.

Главное, что удалось показать Полани с мощной убедительностью, — то, что
знание, в том числе научное, отнюдь не сводится к системе высказываний, не
может считаться чисто семиотическим объектом. Подлинная тайна всякого
профессионального мастерства, в том числе научного познания, — «неявное
знание», то, которое невозможно выразить в словах, формулировках, системе
строгих «предложений». Нет ничего удивительного в том, что медик и химик,
каким был Полани, смог зафиксировать такие особенности своей профессии. Он
писал: «Однако то большое количество учебного времени, которое студенты
химики, биологи и медики посвящают практическим занятиям14, свидетельствует
о важной роли, которую в этих дисциплинах играет передача практических
знаний и умений от учителя к ученику. Из сказанного можно сделать вывод, что
в самом сердце науки существуют области практического знания, которые через
формулировки передать невозможно»15. Любые формулировки и определения лишь
сдвигают область скрытого «молчания», но никогда не отменяют ее.

Итак, знание обладает экстралингвистическими характеристиками. А это
фактически убивало претензии всей логико-аналитической традиции на то, чтобы
ответить на основной вопрос — что есть знание? «Высказывание» предстает как бы
вершиной айсберга, большая, подводная часть которого скрыта в сфере «tacit
knowledge».

Что касается работ Томаса Куна, то модель «смены парадигм» оказалась в
центре такой оглушительной критики, что достаточно долго современники
фактически даже не приступали к анализу тех результатов, которые были им
достигнуты. Понадобилось немало времени, прежде чем остыли страсти и можно
было относительно беспристрастно оценить им сделанное.

14 Список этот явно можно продолжить: и геологи, и почвоведы, и географы, и
физики, и археологи… да есть ли вообще исключения из этого правила?!

15 Полани М. Личностное знание. — М., 1985. — С. 89.

В 1997 г. в Институте философии РАН состоялось заседание «круглого стола»,
посвященного памяти Т. Куна, обсуждению его философского наследия. Хотелось бы
обратить внимание на выступление М.А. Розова с оценкой достижений автора
новой, «посткритической» модели науки. По его словам, Томас Кун, как это
хорошо видно с дистанции времени, прошедшей после 1962 года, действительно
может считаться человеком, совершившим «коперниканский переворот» в изучении
науки. До Куна философия и методология науки были, по сути, неразличимы.
Задача исследования (описания) механизмов развития науки не отли- чалась от
задач формулировки тех методологических правил, которые могли бы
способствовать развитию науки. «Переворот» Куна состоял в том, что от
модальности долженствования, которая характерна для формулировки норм и
правил, философы науки перешли к модальности существования, т.е. сменили по-
нимание своей исследовательской позиции и комплекса задач, вытекающих из
такого понимания. Кун продемонстрировал, что в своей деятельности ученые (как
члены научного сообщества) определены некоторыми традициями (программами), и
задача состоит в том, чтобы реконструировать эти программы и выявить механизмы
их изменений. Он показал, что программы (парадигмы) заданы примерно таким же
образом, как и язык. Это позволило, в частности, осознать, что философия науки
включена в тот круг социо-гуманитарных дисциплин, которые изучают такие классы
явлений, как язык, пословицы, культурные тради- ции, социальные эстафеты и
тому подобное16.

На различие «нормативного» и «дескриптивного» в эпистемологии обратил
внимание Э. Агацци: «Переходя теперь к эпистемологии (понимаемой как общая
теория познания), надо отметить, что она всегда включала два аспекта,
которые можно назвать дескриптивным и нормативным. Нормативный аспект яв-
ляется предварительным, поскольку он состоит, во-первых, в каком-либо
определении понятия знания, т.е. достаточным уточнении того, что такое знание,
а это, во-вторых, определяет также, каким требованиям должно удовлетворять
то, что мы хотели бы квалифицировать как знание.

Дескриптивный аспект состоит в выяснении того, как добывается знание, какими
шагами, при каких условиях и согласно каким критериям можно убедиться в его
получении, и это применительно к различным предметам, которые мы хотели бы
знать. Конечно, эти два аспекта различимы только аналитически, а конкретно
они взаимозависимы»17.

16 Изложение выступления М.А. Розова см.: Философия естествознания ХХ века:
итоги и перспективы. (Материалы к Первому Всероссийскому Философскому
Конгрессу «Человек–Философия–Гуманизм»). — М., ИФРАН, 1997. — С. 41–42.

17 Агацци Э. Эпистемология и социальное: петля обратной связи // Вопросы
философии. 2010, №7. — С. 64.

Полностью разделяя высказанное мнение, все же не могу согласиться со следующим
тезисом Агацци (вернее, считаю следующее его утверждение существенно
неточным): «Историю эпистемологии можно рассматривать как непрерывное углубле-
ние и расширение ее дескриптивного аспекта, в ходе которых функции,
инструменты, критерии, возможности приобретения знания обнаруживались и
оценивались с точки зрения их здравости (нормативного аспекта)»18.

Почему приведенная выше оценка соотнесения дескриптивного и нормативного
кажется мне некорректной? Модель научных парадигм Куна попперианцы расценили
как защиту нетворчески работающего ученого. «Утешением для специалиста»
назвал куновскую «парадигму» П. Фейерабенд; «нормальная наука», конечно,
существует, но она опасна — мнение Поппера; иррациональной назвал концепцию
Куна Лакатос19…

Кун, честно, искренне и откровенно, пытаясь разъяснить свое принципиальное
несогласие с Поппером, выразительно писал: «Я называю то, что нас разделяет,
скорее гештальт-переключением, чем несогласием, и поэтому же я одновременно
и сбит с толку, и заинтригован тем, как лучше объяснить эти наши
расхождения. Как мне убедить сэра Карла, знающего все то, что знаю я о
развитии науки, и так или иначе уже сказавшего нечто об этом, в том, что
предмет, который он называет уткой, я называю кроликом? Как мне показать ему
то, что видно сквозь мои очки, когда он уже научился смотреть на все, что я
мог ему показать, через свои собственные?»20

18 Там же.

19 См.: Criticism and the Growth of Knowledge; перевод статьи Поппера
«Нормальная наука и опасности, связанные с ней» // Кун Т. Структура научных
революций. — М., 2001. — С. 525–538; частичный перевод статьи «Утешение для
специалиста» // Фейерабенд П. Избранные труды по методологии науки. — М.,
1986. — С.109–124.

20 Кун Т. Логика открытия или психология исследования // Кун Т.  Структура
научных революций. — М., 2001. — С, 543.

Нет, открытие дескриптивной позиции в философии науки вызывало искренне
недоумение у методологов (нормативистов), и этому соответствует подлинное
принципиальное расхождение позиций. Исходные позиции порождают различные
«очки», сквозь которые рассматривается историко-научный материал,
призванный доказать правоту выбранного подхода. Важно было осознать само
различие «нормативного» и «дескриптивного», не пытаясь совместить одно с
другим.

ТРАЕКТОРИИ ПОИСКОВ В СОВЕТСКОЙ ФИЛОСОФИИ НАУКИ ПОСЛЕВОЕННОГО ПЕРИОДА

Следует высоко оценить попытки советской философской молодежи того времени,
которая попыталась, с одной стороны, учесть опыт обсуждения методологической
тематики на Западе, усвоить эту традицию, а с другой, — смело обратиться к
анализу научной практики, чтобы не быть голословными и осознать, какие
реальные методологические проблемы волнуют ученых, ведущих работу на передовом
крае своих дисциплин.

В Москве под руководством Г.П. Щедровицкого возник Московский методологический
кружок, в Минске — семинар, где лидером был В.С. Степин, в Новосибирске —
семинар М.А. Розова. Все они примерно одинаково понимали свои задачи —
работать надо на конкретном материале истории науки, а также по возможности
обсуждать методологические проблемы, которые фиксируются самим научным
сообществом. В то же время в Институте философии АН СССР активно работала
группа молодых логиков, объединившихся вокруг четы В.А. Смирнова и
Е.Д. Смирновой. Эта группа, не скрывая своих намерений, ориентировалась на
современную западную логику (собственно, как и математики, они считали, что
формальная логика не может иметь каких-то демаркационных особенностей). Сего-
дня, по прошествии времени можно утверждать, что именно эта группа восприняла
всерьез исследовательскую программу Венского кружка и по мере сил вносила
свои результаты в эту «международную копилку».

Действительно, все вышеперечисленные герои нашей отечественной философии
были достаточно хорошо знакомы с работами Венского кружка (насколько это
возможно при том, что основные труды логического позитивизма находились в
«спецхране»). Критика позитивистской философии была одной из ключевых тем
методологических штудий того времени. И если Степин, Щедровицкий и Розов
критически относились к самим возможностям этой исследовательской программы,
то группа Смирнова, по сути дела, восприняли задачу логического анализа
языка науки вполне одобрительно.

Зато труды постпозитивизма — начиная с работ Карла Поппера, а также
участников «критического рационализма», тем более их соперников Томаса Куна
и Майкла Полани — становились известны относительно скоро, по мере
постепенного проникновения философской англоязычной литературы в наше
сообщество. Все-таки 1960-е гг. — времена хрущевской «оттепели». Эти труды
переводились, так сказать, «частным образом», внимательно изучались рефераты
их работ, которые систематически выполнялись в ИНИОНе. Заметим, что
Г.П. Щедровицкий даже имел в конце 60-х годов небольшую личную переписку с
И. Лакатосом, а на «Доказательства и опровержения» написал для журнала
«Вопросы философии» весьма интересную рецензию21.

21 Щедровицкий Г.П. Модели новых фактов для логики // Вопросы
философии. 1968. №4.

Картина поисков, идущая в рамках постпозитивизма, складывалась более или
менее адекватная. Тем не менее наши герои, идейные лидеры того времени,
предпочитали «идти своим путем». И связано это с тем, что они никогда не
ставили своей задачей быть просто историками (или критиками) современной
зарубежной философии, а к возможностям средствами формальной логики решать
новые задачи в сфере изучения феномена науки и ее актуальных проблем
относились весьма скептически.

В ретроспекции хорошо также видно, что «идеи витают в воздухе», что
интеллектуальные поиски успешно преодолевали «железный занавес». Велась в
нашем сообществе работа в духе Венского кружка; появились «критические
рационалисты» с их активным неприятием существующих практик научного мышле-
ния (сюда, конечно, относятся в первую очередь щедровитяне, которые «пришли в
мир, чтобы не соглашаться»); были и представители «дескриптивной установки»
в духе Томаса Куна, желающие не преобразовывать науку, а только изучать ее
(таким были М.А. Розов и круг его учеников). В.С. Степин, как нам пред-
ставляется, начинал в духе дескриптивной установки, в русле исторических
реконструкций (прослеживал историю электродинамики Максвелла), следуя в этом
плане путем Куна и Полани, но в конечном счете попытался совместить два
подхода, о чем свидетельствует его итоговая книга22.

СИСТЕМО/ДЕЯТЕЛЬНОСТНАЯ МЕТОДОЛОГИЯ Г.П. ЩЕДРОВИЦКОГО И ИССЛЕДОВАНИЕ ФЕНОМЕНА
«МЕТОДОЛОГИчЕСКОГО МЫШЛЕНИЯ» М.А. РОЗОВА

Весьма интересно проследить, каким образом в рядах отечественного
философского сообщества сформировались два проекта построения методологии как
особой деятельности, два способа понимания ее ключевых задач. Здесь
Г.П. Щедровицкий и М.А. Розов разошлись принципиально, хотя в исходе имели
фактически общую «рамочную» картину трансляции и воспроизводства социального
целого.

Оба они полагают, что никакие семиотические явления (включая знание)
невозможно изучать и понять вне объемлющего контекста социального
целого. Такая онтология была невозможна для Венского кружка, такой широты
онтологической картины не хватало и постпозитивизму. А самый общий
социальный процесс (подобно тому, как живое — прежде всего воспроизводство
самого себя) — это процессы трансляции различных компонентов деятельности и
дальнейшее воспроизводство деятельности по имеющимся прототипам. Но простое
«перетекание» в новые условия прежней деятельности не интересует
Г.П. Щедровицкого. Он считает это слишком элементарным случаем и фокусирует
свое внимание на том, что деятельность реализуется по «нормам», которые
выступают как эталоны. Именно по ним строится следующая деятельность, даже
если условия ее реализации сильнейшим образом варьируются. Все сказанное
четко выражено в статье 1967 г. «“Естественное” и “искусственное” в
семиотических системах»23. Авторы — В.А. Лефевр, Э.Г. Юдин,
Г.П. Щедровицкий. Эта статья-манифест восхищала и М.А. Розова.

22 Степин В.С. Теоретическое знание. — М., 2000. — 744 c.

23 Лефевр В.А., Юдин Э.Г., Щедровицкий Г.П. «Естественное» и «искусственное»
в семиотических системах // Семиотика и восточные языки.  — М., 1967. —
С. 48–56.

Вот основное рассуждение: «Рассмотрим простейший случай, когда
восстановление составляющих какой-то социальнопроизводственной структуры
(обозначим ее знаком А) происходит без введения каких-либо специальных
средств трансляции и образцом, или «нормой», для составляющих каждой последу-
ющей единицы являются составляющие предшествующей»24.  Естественно, что если
условия (обозначенные знаком B) постоянно меняются, то происходит и
некоторая эволюция от А1 до Аi. Это случай «естественного» процесса трансляции
социума.  Но представим другой случай, — продолжают рассуждать авторы, —
когда составляющие социально-производственной структуры А1 зафиксированы в
некоторых эталонах (А), которые как «норма» транслируются от одной единицы к
другой. Тогда А1 может варьировать в зависимости от условий реализации, од-
нако норма (А) остается вне изменений. Удержание «норм» (эталонов) — это уже
искусственный процесс, хотя и порождаемый естественным путем. Далее возникает
процесс обучения, т.е.  подготовка людей для воспроизводства той или иной
социально-производственной структуры. Именно он закрепляет искусственный
характер трансляции норм. Простейший случай «естественной» социальной
трансляции зафиксирован как элементарный и отставлен в сторону как не
требующий дальнейшего анализа. Особенно интересовали Щедровицкого случаи
проектирования норм (из чего и вырастает методология в его понимании).

24 Там же. — С. 49.

Но именно указанный базовый механизм социального воспроизводства (социальной
памяти) лег в основу концепции М.А. Розова. В дальнейшем он назовет
воспроизведение деятельности по непосредственным образцам социальными эста-
фетами. Вербализация образцов, а тем более обучение — это очень серьезная
эволюция. Важно то, что в основе социальной жизни лежит этот простейший
случай, когда еще не существует фиксированных «норм», а существует только
непосредственное подражание предшествующей деятельности. И это пребывание
актов предшествующей деятельности в качестве образцов для последующей Розов
называл «нормативами» деятельности.  Следует подчеркнуть также, что концепция
«неявного знания» М. Полани, с которой наше сообщество познакомилось значи-
тельно позже, без труда объяснялась именно идеей социальных эстафет.

Итак, можно указать на различие терминов — «нормы» у Щедровицкого, «нормативы»
у Розова. Эти различия направляли и дальнейшую разработку самой методологии
как особой области интеллектуальной работы.

У Щедровицкого методология приобрела черты глобальной индустрии разработки
норм для всех видов человеческой активности — как практической, так и
теоретической. Рефлексия для него — универсальный способ оптимизации всех
типов деятельности, поэтому рефлексия описывает то, что делается, фикси-
рует нарушение норм и «исправляет» то, что делается. Особо интересен случай,
когда нормы специально проектируются для новаций, а человеческий опыт не имеет
в своем арсенале аналогов того, что требуется в конкретных ситуациях. В
случаях «радикальной новизны» методолог составляет проекты, подобные
проектам «бумажной архитектуры». Советы, которые может здесь дать методолог,
направлены на «расстановку сил», на создание плодотворной кооперации
участников совокупного деятельностного процесса. В конечном итоге проект
Щедровицкого нашел свое воплощение в построении так называемой систе-
мо-деятельностной методологии (СД-методологии). Очевидно, что речь идет о
проектах системной реализации сложной деятельности, требующей большого
количества участников (акторов). Его проект методологии, по сути, —
организационный. К сожалению, методологические идеи Георгия Петровича во мно-
гом остались на уровне «бумажной архитектуры», ибо ни к какой управленческой
работе сам он никогда не допускался. Однако в 1980-е гг. Щедровицкий стал
проводить масштабные «орг-деятельностные игры», которые готовили по всей
стране кадры будущих топ-менеджеров и управленцев высшего государственного
слоя.

Ход рассуждений Розова совсем иной. В 1974 г. опубликована совсем небольшая
работа25, выразительно иллюстрирующая собственную постановку вопроса. Здесь
был предложен шутливый, но очень выразительный, мысленный эксперимент,
демонстрирующий стиль и особенности методологического мышления.

25 Розов М.А., Розова С.С. К вопросу о природе методологической деятельности
// Методологические проблемы науки. Вып. 2. — Новосибирск, 1974. — С. 25–35.

Представим себе, что в некоей условной древней цивилизации существуют только
два ученых человека: один умеет определять площади простых фигур (геометр),
а другой умеет взвешивать (например, зерно). И вот первого измерителя (гео-
метра) вызывает ужасный тиран и требует, чтобы он определил площадь листа
какого-то заморского растения, весьма сложной формы. Озадаченный геометр
уходит домой и начинает размышлять. Совета ему спросить не у кого. Но он
начинает думать о том, какая связь существует между процедурами измерения
площади и веса. На первый взгляд, ничего общего. Однако же, думает
незадачливый геометр, на большей площади земли вырастает больше урожая, что
выражается в большем количестве зерна. А что если засыпать зерном лист
заморского растения, а потом взвесить его? В конечном счете можно засыпать
этим количеством зерна простейшую фигуру, площадь которой он умеет
вычислять! И вот — задача решена, тирану сообщается ответ, а геометр получает
премию. Догадливый читатель может вспомнить, что этот мысленный эксперимент,
по сути дела, воспроизводит историю открытия Архимедом основного закона
гидростатики. Получив задание от тирана Сиракуз Гиерона определить, какая
из подаренных корон не является чисто золотой, Архимед погрузился в ванну…
откуда выскочил голышом с криком «Эврика».

Что можно здесь фиксировать? Во-первых, для того, чтобы человек начал думать,
необходимо, чтобы возникла «нештатная ситуация». Во всех других случаях
измерение идет совершенно стандартным путем. (Заметим, что Щедровицкий
неоднократно подчеркивал, что нетривиальное мышление начинается в ситуациях
«разрыва деятельности».) Во-вторых, если задача нетривиальная, то следует
привлечь самый неожиданный опыт из других сфер практики. Такую работу можно
назвать методом «отдаленных сопоставлений». В-третьих, надо перенести опыт
решения задачи, взятой из другой сферы жизни, на решение внезапно возникшей
задачи. Все перечисленное — и есть характеристики того типа мышления, которое
можно назвать «методологическим».

Как видим, Розов, если говорить кратко, предлагает для решения новой,
нестандартной задачи воспользоваться образцами работы из весьма отдаленных
сфер практики или мышления.  И в этом он видит суть методологических ходов
мысли. Это совершенно иное представление о методологии и ее задачах.

Действительно, в научной практике дискуссии о методологии возникают только
тогда, когда нет «нормальных» (Кун сказал бы — парадигмальных) способов
решения новых задач. Иначе говоря, люди, будь они теоретики или практики,
обсуждают методологические проблемы только в тех случаях, когда специали-
зированных методов работы просто нет.

Предлагаемое Розовым представление о сути методологии базируется на общей
посылке, что если нормативы решения задачи в каких-то ситуациях отсутствуют,
то нормативы другой практики могут оказаться весьма эффективными. Социальные
эстафеты, о которых говорилось выше, потому и действенны, что обладают
способностью «перескакивать» (точнее, их надо целенаправленно «перетаскивать»)
из одних сфер познавательного опыта в другие.

Как ни странно, проект Розова оказался успешно реализованным на
практике. Уже в 1990-е годы отечественные географы приняли именно такой
способ работы в своих систематически проводимых методологических
конференциях. С подачи Михаила Александровича эти заседания стали называться
«Сократическими Чтениями» (в 2012 году прошли X Чтения, посвященные памяти
М.А. Розова). Суть этих мероприятий — обмен непосредственным опытом решения
нетривиальных задач. Каждый участник — носитель этого опыта, и он
рассказывает о собственных ходах мысли с целью обмена живыми образцами мы-
шления, а также в попытках найти «подсказку» у других вынужденных
«вольнодумцев».

***

Думаю, что оба проекта методологии, созданные в рамках отечественного
философского сообщества, глубоко своеобразны и достойны самого пристального
анализа. Они были направлены на решение методологических задач, постоянно
возникающих в ходе научного поиска.

СД-методология Г.П. Щедровицкого, как уже говорилось, решает прежде всего
организационно-управленческие вопросы, связанные с кооперацией людей при
исполнении сложных, комплексных (системных) задач. Она учит рефлексии и опти-
мальному взаимодействию в ходе совместной работы.

Методологические ходы, которые предлагает фиксировать и анализировать
М.А. Розов, учат прежде всего искать «подсказки» на уровне познавательных
метафор, самого широкого обмена опытом решения нетривиальных задач. Такая
soft методоло2 гия не пафосна, но работоспособна.

Концепцию науки, с его точки зрения, надо строить совсем в другой плоскости —
в духе дескриптивной установки. Это — иная профессиональная задача.
\end{document}
