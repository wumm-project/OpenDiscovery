\documentclass[11pt]{beamer}
\usepackage{url}

\usepackage[utf8]{inputenc}
\parindent0pt
\parskip3pt

\newcommand{\Rahmen}[2]{
\setlength{\fboxsep}{12pt}\begin{center}
\shadowbox{\parbox{#1\textwidth}{\em #2}}\end{center}}

% Designelemente
\usetheme{Hannover}
\beamertemplatenavigationsymbolsempty

\title[The Inventor School Movement in the GDR]{The Contribution to TRIZ by\\ 
  the Inventor Schools in the GDR}

\subtitle{Presentation at the TRIZ-Fest 2019 in Heilbronn}

\author[Hans-Gert Gr\"abe]{Prof. Hans-Gert Gräbe}
\institute{Institut f\"ur Informatik, Universit\"at Leipzig,\\
  \url{http://bis.informatik.uni-leipzig.de/HansGertGraebe/}}
\date{September 13, 2019}
\begin{document}
\begin{frame}
\maketitle
\end{frame}

\section{ISM -- The Facts}
\begin{frame}{Inventor School Movement in the GDR (ISM) -- The Facts}
  \small
  
  \textbf{Size:} Between 1981 and 1990 in the GDR there were about 300
  inventor schools with about 7\,000 participants. The model for the
  implementation of inventor schools was realized along a standard methodology
  based on TRIZ ideas constantly evolving since 1982.

  \textbf{Outcome:} There is no precise statistics but it can be estimated
  that nevertheless 600 patent applications and 1\,000 practical problem
  solutions were achieved.
  
  \textbf{Materials:} From 1982, the participants were provided with a
  specially developed methodical hand material -- a small book. Authors:
  Michael Herrlich and others. In 1988/89, the material was significantly
  enhanced and was now available in two small books, both very demanding.
  Authors: Hans-Jochen Rindfleisch and Rainer Thiel.

\end{frame}

\begin{frame}{ISM -- Methodology}
  \small

  One to two dozen engineers from an industrial plant gather two courses in a
  rural place, each for a week, to learn innovative problem-solving methods
  and to inventively solve one to three business problems into one to three
  groups of community work.

  In the first week about 12 hours lectures are offered. In about 40 hours of
  teamwork, a problem is exposed and a solution is created. The moderation of
  each group -- ideally 7 participants -- is done by an experienced inventor,
  he acts as a methodologist and trainer.

  In the following weeks, the patent study is deepened in the company,
  calculations and hand tests, also laboratory tests are made.

  Finally, a second week follows in the rural place to complete patent
  applications and initiate the start of the pilot series.

\end{frame}

\section{Periodization and Components}
\begin{frame}{ARIZ and Analysis of Social Processes}

  How methodologically analyze that development, highly driven by
  social and societal contradictions on several levels? \medskip

  My ARIZ based approach cannot be explained here due to time restrictions. I
  refer to my papers and present only the application to the analysis of the
  history of the Inventor School Movement in the GDR.

\end{frame}

\begin{frame}{ISM -- Periodization}

  The Inventor School Movement (IS) grew up in the highly contradictorily
  developing socio-political conditions within the socio-political system
  (SPS) of ``real socialism''.  \vfill

  \small
   
  \textbf{Phase A: 1962 -- 1970}

  IS: Incubation phase of ideas that later lead to inventor schools.  Leeway
  for protagonists to propagate these ideas.
   
  SPS: Encouraging socio-political conditions on the background of ongoing
  ideological narrowness: half-hearted political experiments on economic
  mechanisms using modern scientific ways of thinking (cybernetics,
  prognostics, operations research, mathematical modeling, computer use)
  within the ``New Economic System of Planning and Management'' (NÖSPL).
\end{frame}

\begin{frame}{ISM -- Periodization}
\small
  \textbf{Phase B: 1971 -- 1978}

  IS: Formation of the plans of the inventor school concept and formation of a
  network of enthusiasts.

  SPS: Restoration of a rigid centralism under Honecker. Degradation of the
  timid turn to modern ways of thinking. The concept of the ``Unity of
  Economic and Social Policy'' replaces NÖSPL.
  
  \textbf{Phase C: 1979 -- 1982}

  IS: Creation of organizational structures for inventor schools within the
  Engineering Association (KDT). First practical tests and business contacts
  via trusts (``Kombinate''). First teaching material, mainly due to
  Michael Herrlich as author.

  SPS: Increasing oppressive feelings in parts of the political establishment
  given by the low growth rates of the economy. Within the intelligence grows
  the feeling of the need for profound reforms, but this is abandoned by the
  establishment.

\end{frame}

\begin{frame}{ISM -- Periodization}
\small
  \textbf{Phase D: 1983 -- 1989}

  IS: Inventor school movement enters the industry. The number of trainers and
  participants grow. Efforts, the breadth and depth effect increase rapidly.
  The second generation of teaching material and also coaching material is
  written.

  SPS: Increasing self-deception and political fraud by the political
  leadership.  Hectic attempts, by concentration of all reserves in the
  high-tech sector despite the tightening of the trade embargo (COCOM lists)
  and ongoing currency shortages to achieve breakthroughs to the world class
  level.

\end{frame}

\begin{frame}{ISM -- Components}

  During an \textbf{ARIZ-like component analysis} of the inventive system we
  identified
  \begin{itemize}
  \item the trainers, many of them from the group of \emph{Honored Inventors},
  \item inventive practices in the industry,
  \item dialectical traditions of thought in contradictions coming from
    cybernetics,
  \item structural relicts from the ``Systematic Heuristics'', an innovation
    theory developed by Johannes Müller and strongly pushed by the
    establishment until the early 1970th.
  \end{itemize}
\end{frame}

\begin{frame}{The System of Honored Inventors}
  
  \emph{Honored inventor} was a state honorary title of the GDR, which
  was awarded from 1950 in conjunction with a badge of honor and a monetary
  bonus.

  There existed a system of strong social ties between them that worked
  independently of all political changes, mainly inspired by Michael Herrlich.

\end{frame}

\section{Theoretical Contributions}
\begin{frame}{Theoretical Contributions of the Inventor School Movement}

\textbf{Three Theoretical Frameworks}
\begin{itemize}
\item WOIS -- Contradiction Oriented Innovation Strategies (Linde, TU Dresden
  1988)
\item PROHEAL -- Program for the Development of Inventive Approaches and
  Solutions (Rindfleisch, Thiel 1988)
\item Inventing as process of information processing and generation, presented
  at the own innovative work and at the approaches within the KDT inventor
  schools. (Herrlich, TU Ilmenau 1988)
\end{itemize}
\end{frame}

\begin{frame}{Theoretical Contributions of the Inventor School Movement}

\textbf{Main Contributions}
\begin{itemize}
\item The PROHEAL Path Model
\item Closer analysis of administrative contradictions on a technical-economic
  level (TÖW), thus already close to today's challenges of the inventor's
  everyday life in \emph{TRIZ for Business}.
\item Differentiation of contradictions on the three levels TÖW, TTW, TNW
\item The ABER\footnote{\textbf{A}nforderungen, \textbf{B}edingungen,
  \textbf{E}rwartungen, \textbf{R}estriktionen. $=$ Requirements, Conditions,
  Expectations, Restrictions.} matrix as a unified analysis approach on these
  three levels in three different versions adapted to the needs of the
  respective level.
\end{itemize}
\end{frame}

\end{document}


Beobachtungen

Kleine, oft evolutionäre Veränderungen im SPS lösen oft disruptive
Veränderungen im IS aus.

Kybernetik und MLO (ML Organisatonswissenschaft) waren noch stark von Vorgaben
aus dem SPS getrieben („planmäßige Entwicklung“)

Es gibt einen Hauptwiderspruch zwischen realsozialistischen
Entwicklungskonzeptionen (SPS) und der praktischen Dynamik der Wirtschaft,
welcher die Möglichkeiten des SPS begrenzt.

In den 1980er Jahren war das SPS kaum noch durchsetzungsfähig, die praktische
Führung der Prozesse lag bei Kräften aus der Wirtschaft, also im
sozio-ökonomischen System (SES).

All das ist für die Ideengeschichte der TRIZ hoch relevant, in dieser
Komplexität aber nur in einem größeren ideengeschichtlichen Projekt
aufzuarbeiten.

Beobachtung

Alle Komponenten des IS als Teile des SPS unterlagen stark wechselnden
Restritionen -- allein das System der HI als Teil des SES bleibt eine
sichtbare Konstante über die gesamten 30 Jahre.

Ansatz: Betrachte TRIZ Theorie und TRIZ Praxis als zwei Pole, zwischen denen
sich das IS entfaltet.

Betrachte IS als Vermittlungsstruktur (Feld) zwischen diesen beiden Polen.
Die Pole selbst sind eingebettet in die Widerspruchsstrukturen des SPS (TRIZ
Theorie) und der Praxen des SES (TRIZ Praxen).

Substanz-Feld-Swap

Widersprüche 2. Art = Widerspruch zwischen den Widersprüchen des SPS und den
Widersprüchen des SES.

Erst aus einem solchen Ansatz heraus werden die Widersprüche der
Erfinderschulbewegung sprechbar.


WOIS - Widerspruchsorientierte Innovationsstrategien (Linde, TU Dresden 1988)

PROHEAL - Programm zur Herausarbeitung von Erfindungsansätzen und
Lösungsansätzen (Rindfleisch, Thiel 1988)

Michael Herrlich: Erfinden als Informationsverarbeitungs- und
-generierungsprozess, dargestellt am eigenen erfinderischen Schaffen und am
Vorgehen in KDT-Erfinderschulen. (TU Ilmenau 1988)

PROHEAL Wegemodell

Genauere Analyse administrativer Widersprüche auf einem technisch-ökonomischen
Level, damit nahe an heutigen Herausforderungen des inventiven Alltags.

Unterscheidung von Widersprüchen auf drei Ebenen TÖW, TTW, TNW

ABER-Matrix in drei Versionen als einheitliches Analyse-Instrument auf diesen
drei Ebenen.

