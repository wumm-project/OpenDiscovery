\documentclass[11pt,a4paper]{article}
\usepackage{a4wide,url}
\usepackage[T1,T2A]{fontenc}
\usepackage[utf8]{inputenc}
\usepackage[main=ngerman,russian]{babel}

\parindent0cm
\parskip3pt

\title{Der Mensch und seine Technischen Systeme} 
\author{Hans-Gert Gräbe, Leipzig}
\date{Version vom 10. April 2020}
\begin{document}
\maketitle

\begin{flushright}
  Die Philosophen haben die Welt\\ nur verschieden interpretiert;\\ es kömmt
  darauf an sie zu verändern.\\ Karl Marx. 11. Feuerbachthese
\end{flushright}
\section{Vorbemerkungen}

Ausgangspunkt dieser Überlegungen war eine Debatte mit den Organisatoren des
TRIZ-Cups 2019/20 über die Gültigkeit eines „Gesetzes der Verdrängung des
Menschen aus technischen Systemen“, das in einer späteren Fassung der
Ausschreibung als „Trend“ bezeichnet wurde. Unter den acht Gesetzen der
Entwicklung technischer Systeme, die Altshuller 1979 selbst formulierte
\cite[S. 122--127]{Altshuller1979}, kommt ein solcher Ansatz nicht vor, auch
nicht in der Auf"|listung von fünf Gesetzen und zehn Tendenzen in
\cite[S. 148\,ff.]{KoltzeSouchkov2017}.  Er ist aber in der Literatur zur
Evolution technischer (bzw. in neuerer Formulierung
\emph{ingenieur-technischer} \cite{TESE2018}) Systeme weit verbreitet, siehe
etwa \cite[2.24]{Goldovsky1983}, \cite{TESE2018} oder die Literaturangaben
ebenda.  Insbesondere \cite{TESE2018} spielt als Referenz eine zentrale Rolle,
da hier von einflussreichen TRIZ-Theoretikern mit der Autorität der MATRIZ im
Rücken ein Zusammenschnitt der Debatten um „Trends of Engineering Systems
Evolution“ gegeben wird und insgesamt 10 grundlegende „Trends“ formuliert
werden, von denen einer der „Trend of Decreasing Human Involvement“
\cite[4.8]{TESE2018} ist.

Auch in der marxistischen Literatur wird ein solcher Herauslösungsprozess des
Menschen aus produktiven Prozessen thematisiert und an vielen Stellen als
unausweichlich charakterisiert.  So entwickelt Marx selbst im
„Maschinenfragment“ \cite[S. 570 ff.]{MEW42} -- einem frühen Rohentwurf der
eigenen ökonomischen Theorie -- die Vision einer Gesellschaft, in welcher der
„gesellschaftliche Stoffwechsel“ \cite[S. 37]{MEW23} auf eine Weise
organisiert ist, dass
\begin{quote}
  es nicht mehr der Arbeiter [ist], der modifizierten Naturgegenstand als
  Mittelglied zwischen das Objekt und sich einschiebt; sondern den
  Naturprozess, den er in einen industriellen umwandelt, er als Mittel
  zwischen sich und die unorganische Natur [schiebt], deren er sich
  bemeistert \cite[S. 572]{MEW42},
\end{quote}
und stellt weiter dar, dass die Entwicklung der Produktivkräfte
\emph{notwendig} auf eine solche Weise der Organisation des gesellschaftlichen
Stoffwechsels zusteuert.
\begin{quote}
  In den Produktionsprozess des Kapitals aufgenommen, durchläuft das
  Arbeitsmittel aber verschiedene Metamorphosen, deren letzte die
  \emph{Maschine} ist oder vielmehr ein \emph{automatisches System der
    Maschinerie} (System der Maschinerie; das \emph{automatische} ist nur die
  vollendetste adäquateste Form derselben und verwandelt die Maschinerie erst
  in ein System), in Bewegung gesetzt durch einen Automaten, bewegende Kraft,
  die sich selbst bewegt; dieser Automat bestehend aus zahlreichen mechanischen
  und intellektuellen Organen, sodass die Arbeiter selbst nur als bewusste
  Glieder desselben bestimmt sind. \cite[S. 584]{MEW42}
\end{quote}
Dieser Gedanke ist allerdings weitgehend singulär und im übrigen Marxschen
Werk nirgends ausgearbeitet. So wenigstens \cite{Goldberg2016}.

Derartigen technikoptimistischen Sichten steht die Position aus dem
Kybernetikdiskurs der 1960er bis 1980er Jahre entgegen \cite[S. 10]{KFK2000}:
\begin{quote}
  Welche Stellung hat der Mensch im hochkomplexen informations-technologischen
  System? Unsere Antwort auf die Frage war immer: Der Mensch ist die einzig
  kreative Produktivkraft, er muss Subjekt der Entwicklung sein und bleiben.
  Daher ist das Konzept der Vollautomatisierung, nach dem der Mensch
  schrittweise aus dem Prozess eliminiert werden soll, verfehlt!
\end{quote}
Die Probleme eines solchen „Konzepts der Vollautomatisierung“, einer Welt der
„in Bewegung gesetzten Automaten“ werden mittlerweile in einer ökologischen
Krise planetaren Ausmaßes sichtbar. Die Verdrängungsthese selbst wird dabei
als direkte Gefährung wahrgenommen, was hier als \emph{Gegenthese} explizit
formuliert werden soll:
\begin{quote}
  Die (scheinbare) Verdrängung des Menschen aus technischen Systemen weist auf
  eine existenziell gefährliche, unterkomplexe Fehlwahrnahme dieser
  technischen Systeme hin.
\end{quote}
Das Altschullersche „Prinzip 11 der Prävention“ (verschämt auch als „Prinzip
des untergelegten Kissens“ bezeichnet) weist auf Handlungsbedarf in dieser
Richtung hin, der Anwendungskontext dieses Prinzips wird in \cite{TT} wie
folgt umrissen: „Der Einsatz des Prinzips ist besonders in solchen Fällen
wichtig, in denen das System nicht über ein ausreichendes Maß an
Zuverlässigkeit verfügt“, um dann festzustellen, dass dem eigentlich
strukturell abgeholfen werden könne -- „Notfallsituationen können vermieden
werden, indem der Prozess zuverlässig gemacht wird.“ Dem stehen allerdings
gewichtige Gründe entgegen -- „was die technischen Systeme, die ihn
durchführen, erheblich verkompliziert oder verteuert. Dies ist kostspielig und
oft prinzipiell unmöglich. Mit anderen Worten: Notfälle sind unvermeidlich“.
Der Optimismus der Autoren, dass „zusätzliche Rettungs- und Notfallsysteme
[\ldots] in das Hauptsystem“ eingebaut werden, die allerdings „nicht am
Hauptsystem teilnehmen, sondern erst in einer Gefahrensituation zu arbeiten
beginnen“, erscheint mit Blick auf Kosten reduzierende Designpraxen und die
allgemeine Ausrichtung der anderen TRIZ-Prinzipien auf Effizienzgewinne als
Widerspruch in der TRIZ-Methodik selbst. Die Aussage, „das Prinzip kann dort
angewendet werden, wo die Zuverlässigkeit des Systems offensichtlich
unzureichend ist und ein Weg zur Erhöhung der Zuverlässigkeit auf das
notwendige Niveau nicht möglich ist“ (ebenda) zeigt, dass der Einbau solcher
Defizite in technische Systeme -- in Kenntnis derselben -- auf breiter Front
billigend in Kauf genommen wird.

Dies ist allerdings in keiner Weise mehr ein technisches Problem.  Harrisburg,
Tschernobyl, Fukushima, das Bienensterben \cite{Jacobasch2019} oder der
Klimawandel sind genü"|gend Fingerzeige, um sich mit diesen Positionen genauer
zu befassen.

\section{Technik und Welt verändernde Praxen}

Betrieb und Nutzung technischer Systeme ist heute sicher ein zentrales Element
Welt ver"|ändernder menschlicher Praxen. Dafür ist planmäßiges und abgestimmtes
arbeitsteiliges Handeln erforderlich, denn das Nutzen eines Systems setzt
dessen Betrieb voraus.  Umgekehrt ist es wenig sinnvoll ein System zu
betreiben, das nicht genutzt wird. In der Informatik ist dieser Zusammenhang
zwischen Definition und Aufruf einer Funktion gut bekannt -- der Aufruf einer
Funktion, die noch nicht definiert wurde, führt zu einem Laufzeitfehler; die
Definition einer Funktion, die nie aufgerufen wird, weist auf einen
Designfehler hin.

Eng verbunden mit der informatischen Unterscheidung von Definition und Aufruf
einer Funktion ist die Unterscheidung von Designzeit und Laufzeit.  Eine
solche Unterscheidung hat im realweltlichen arbeitsteiligen Einsatz
technischer Systeme noch größere Bedeutung -- während der Designzeit wird das
prinzipielle kooperative Zusammenwirken \emph{geplant}, während der Laufzeit
\emph{der Plan ausgeführt}. Für technische Systeme sind also zusätzlich deren
interpersonal als \emph{begründete Erwartungen} kommunizierten
\emph{Beschreibungsformen} und die in \emph{erfahrenen Ergebnissen}
resultierenden \emph{Vollzugsformen} zu unterscheiden.

Marx \cite[S. 193]{MEW23} merkt dazu an:
\begin{quote}
  Eine Spinne verrichtet Operationen, die denen des Webers ähneln, und eine
  Biene beschämt durch den Bau ihrer Wachszellen manchen menschlichen
  Baumeister. Was aber von vornherein den schlechtesten Baumeister vor der
  besten Biene auszeichnet, ist, daß er die Zelle in seinem Kopf gebaut hat,
  bevor er sie in Wachs baut. Am Ende des Arbeitsprozesses kommt ein Resultat
  heraus, das beim Beginn desselben schon in der Vorstellung des Arbeiters,
  also schon ideell vorhanden war.
\end{quote}
So einfach ist es allerdings nicht, wie das folgende Beispiel einer
Konzertauf"|führung zeigt. Dieser die Zuhörer erfreuenden Vollzugsform geht
die Erarbeitung der Beschreibungsform, die Verständigung über die genaue
Interpretation des aufzuführenden Werks, voraus. Diese Verständigung auf einen
\emph{gemeinsamen Plan} ist selbst ein voraussetzungsreicher praktischer
Prozess.  Die Voraussetzungen resultieren aus vorgängigen Praxen -- etwa dem
\emph{privaten Verfahrenskönnen} der einzelnen Musiker in der Beherrschung
ihrer Instrumente sowie dem Vorliegen der Partitur als etablierte
Beschreibungsform des aufzuführenden Konzertstücks.  Wenn Alexander Shelley am
14. Oktober 2018 im Leipziger Gewandhaus ohne diese Partitur von Mozarts
Klavierkonzert KV 491 ans Dirigentenpult tritt, so wird deutlich, dass jene
Beschreibungsform allenfalls das Rohmaterial liefert, auf dessen Basis sich
Dirigent und Orchester in den vorausgegangenen Proben auf eine situativ
konkrete Beschreibungsform als Basis der nun zur Auf"|führung gelangenden
Vollzugsform geeinigt haben. Mehr noch weisen die opulenten Gesten des
Dirigenten in Richtung Orchester darauf hin, dass in diesen Proben auch
\emph{Sprache} generiert wurde, um die Ergebnisse längerer
Verständigungsprozesse in eine kompakte Form zu fassen, die den zeitkritischen
Tempi der Vollzugsform gewachsen ist.  Den Rahmen einfacher
ingenieur-technischer „Baumeisterarbeit“ sprengt Gabriela Montero, die
Solistin dieses Abends, mit ihrer Zugabe: Das Publikum wird aufgefordert, eine
Melodie vorzugeben, woraus die Virtuosin eine Improvisation als Vollzugsform
entwickelt, zu der es keine interpersonal kommunizierbare Beschreibungsform
gibt, wenn man einmal von den Tonaufzeichungen jenes Gewandhausabends und den
Berichten der begeisterten Hörerschaft absieht.  Dass auch hierfür technische
Meisterschaft erforderlich war, steht außer Frage.

Das Verhältnis der Menschen zu ihren technischen Systemen ist also komplex und
nur in einer dialektischen Perspektive der Weiterentwicklung bereits
vorgefundener technischer Systeme zu fassen, wenn man sich nicht unentrinnbar
in unfruchtbare Henne-Ei-Debatten verstricken will.  Das relativiert aber auch
die Marxsche Forderung an die Philosophen, denn deren Interpretationen sind
die Differenzen zwischen den begründeten Erwartungen und den erfahrenen
Ergebnissen früherer Praxen vorgängig. Ob es ausreicht, diese Differenzen auf
der Ebene der Techniker, der Ingenieure oder der Fachwissenschaftler zu
besprechen oder eine Intervention der „interpretierenden Philosophen“ als
eigenständige Reflexionsdimension von Bedeutung ist, mag an dieser Stelle
offen bleiben. 

\section{Systeme und Komponenten}

Neben der Beschreibungs- und Vollzugsdimension spielt für technische Systeme
auch der \emph{Aspekt der Wiederverwendung} eine große Rolle.  Dies gilt,
zumindest auf der artefaktischen Ebene, allerdings \emph{nicht} für die
meisten technischen Großsysteme -- diese sind \emph{Unikate}, auch wenn bei
deren Montage standardisierte Komponenten verbaut werden. Auch die Mehrzahl
der Informatiker ist mit der Erstellung solcher Unikate befasst, denn die
IT-Systeme, die derartige Anlagen steuern, sind ebenfalls Unikate.  Dasselbe
gilt auch für die Ämter, Behörden und öffentlichen Einrichtungen. So ist zum
Beispiel die Leipziger Stadtverwaltung aktuell damit befasst, ihre
Verwaltungsprozesse zu „digitalisieren“, was unter Führung des Dezernats
Allgemeine Verwaltung und zusammen mit dem städtischen IT-Dienstleister Lecos
erfolgt. Im Industriesektor wird deshalb deutlich zwischen
Werkzeugmaschinenbau und Industrieanlagenbau als Ausrüster sowie Planern und
„Baumeistern“ entsprechender Unikate unterschieden, auch wenn diese in
einschlägigen Statistiken \cite{VDMA2019} zum \emph{Maschinen- und Anlagenbau}
zusammengefasst werden.

Die Besonderheiten eines technischen Systems liegen damit vor allem im Bereich
des \emph{Zusammenspiels der Komponenten}. So unterscheiden sich
beispielsweise die Produktionsleitsysteme verschiedener BMW-Werke deutlich
voneinander \cite{Kropik2009}. Die Werke wurden zu verschiedenen Zeiten nach
dem jeweiligen Stand der Technik und dem sich ebenfalls verändernden
Geschäftsmodell des Unternehmens konzipiert. Einmal in die Welt gesetzt, sind
derartige technischen Großsysteme nur noch bedingt modifizierbar und werden
deshalb nach Ablauf entsprechender Amortisationsfristen auch konsequent außer
Betrieb gestellt. Gleichwohl spielt der Aspekt der Wiederverwendung auch bei
solch unterschiedlichen technischen Systemen eine Rolle, verschiebt sich aber
von der unmittelbaren Ebene der technischen Artefakte auf höhere Ebenen der
Abstraktion in der Beschreibungsdimension.

Damit sind wesentliche Elemente zusammengetragen, die eine erste Annäherung an
den \emph{Begriff eines technischen Systems} erlauben.  Der Begriff ist in
einem planerisch-realweltlichen Kontext vierfach überladen
\begin{itemize}
\item [1.] als realweltliches Unikat,
\item [2.] als Beschreibung dieses realweltlichen Unikats
\end{itemize}
und für in größerer Stückzahl hergestellte Komponenten auch noch
\begin{itemize}
\item [3.] als Beschreibung des Designs des System-Templates sowie
\item [4.] als Beschreibung und Betrieb der Auslieferungs- und
  Betriebsstrukturen der nach diesem Template gefertigten realweltlichen
  Unikate.
\end{itemize}
Als Grundlage für einen derart abgrenzenden Systembegriff soll im Weiteren der
submersiv gefasste Begriff offener Systeme der Theorie dynamischer Systeme
\cite{Bertalanffy1950} verwendet werden, der
\begin{itemize}
\item [1.] eine innere Abgrenzung gegen vorgefundene Systeme (Komponenten), 
\item [2.] eine äußere Abgrenzung und funktional determinierte Einbettung in
  eine Umwelt sowie
\item [3.] einen externen Durchsatz postuliert, der zu innerer Strukturbildung
  führt und damit die Leistungsfähigkeit des Systems bestimmt,  
\end{itemize}
und seine Fruchtbarkeit für eine Behandlung mit mathematischen Instrumenten
seither vielfach unter Beweis gestellt hat.  

\emph{Technische Systeme} sind in einem solchen Kontext Systeme, auf deren
Gestaltung kooperativ und arbeitsteilig agierende Menschen Einfluss nehmen,
wobei vorgefundene technische Systeme auf Beschreibungsebene durch eine klare
\emph{Spezifikation} ihrer Schnittstellen und auf Vollzugsebene durch die
\emph{Gewähr spezifikationskonformen Betriebs} normativ charakterisiert sind. 

Wir bewegen uns dabei klar im Bereich der Standard-TRIZ-Terminologie eines
\emph{Systems von Systemen} -- ein technisches System besteht aus Komponenten,
die ihrerseits technische Systeme sind, deren \emph{Funktionieren} (sowohl im
funktionalen als auch im operativen Sinn) für die aktuell betrachtete
Systemebene vorausgesetzt wird. Der Begriff eines technischen Systems hat
damit eine klar epistemische Funktion der „Reduktion auf das Wesentliche“.
Einstein wird der Ausspruch zugeschrieben „make it as simple as possible but
not simpler“. Das \emph{Gesetz der Vollständigkeit eines Systems} bringt genau
diesen Gedanken zum Ausdruck, allerdings tritt dieser dabei nicht als
\emph{Gesetz}, sondern als \emph{Modellierungsdirektive} in Erscheinung.  Die
scheinbare „Naturgesetzlichkeit“ der beobachteten Dynamik ist also wesentlich
an \emph{vernünftiges} (im Sinne von \cite{Vernadsky1997}) \emph{menschliches
  Agieren} gebunden.

Mit einem Ansatz der „Reduktion auf das Wesentliche“ sowie der „Gewähr
spezifikationskonformen Betriebs“ sind in diese Begriffsbildung inhärent
menschliche Praxen eingebaut, aus denen heraus die Begriffe „wesentlich“,
„Gewähr“ und „Betrieb“ überhaupt erst sinnvoll gefüllt werden können.  Eine
Unterscheidung zwischen technischen und sozio-technischen Systemen, die für
M. Rubin „offensichtlich und wesentlich“ (Quelle?) ist, wird damit
problematisch. Wesentliche Begriffe aus dem sozial determinierten
Praxisverhältnis von Menschen wie Ziel, Nutzen, Gewährleistung und
Verantwortung sind fest in die Begriffsgenerierungsprozesse der Beschreibung
konkreter technischer Systeme eingebaut und finden in den konkreten
gesellschaftlichen Setzungen eines primär rechtsförmig konstituierten
bürgerlichen Systems ihre „natürliche“ Fortsetzung.

In der TRIZ-Literatur spielen solche begriff"|lichen Fundierungen kaum eine
Rolle.  Einschlägige Lehrbücher wie etwa \cite{KoltzeSouchkov2017} betrachten
den Begriff des \emph{technischen Systems} als intuitiv gegeben, der sich aus
einer „industriellen Praxis“ heraus \cite[S. 2]{KoltzeSouchkov2017} von selbst
versteht, während andere Begriffe wie „Prozess“, „Produkt“, „Dienstleistung“,
„Ressourcen“ und „Effekte“ \cite[S. 6--10]{KoltzeSouchkov2017} genauer
eingeführt werden. Selbst die ausführliche Beschreibung der „Evolution
technischer Systeme“ in 5 Gesetzen und 11 Trends
\cite[Kap. 4.8]{KoltzeSouchkov2017} basiert allein auf der lapidaren
Feststellung „Die Existenz technischer Evolution ist eine zentrale Erkenntnis
der TRIZ“.  Auch \cite{TESE2018} bleibt in dieser Frage vage; im Vorwort von
B. Zlotin heißt es allein zum \emph{Zweck} von Betrachtungen der Evolution
ingenieur-technischer Systeme „humanity can achieve practically any realistic
goal, but certain priorities must be set to ensure the greatest possible
impact on the economy and human life. [\ldots] The powers of contemporary
science and technology as well as financial investment should be applied to
carefully selected and formulated objectives.“

Es ist natürlich möglich, in einem diskursiven Rahmen die verbale Fassung
eines Begriffs offen zu lassen und auf andere Weise -- etwa durch den Bezug
auf gemeinsame Praxen oder durch den „gewöhnlichen Gebrauch“ -- die Konvergenz
der Begriffsverwendung zu erreichen.  Ein solches Grundmuster wird im
TRIZ-Kontext für den Begriff \emph{technisches System} besonders auch in
\cite{TESE2018} angewendet, indem der Begriff durch eine Vielzahl von
Beispielen in Kombination mit den Begriffen „Muster“ und „Evolution“
illustriert, die genaue Fassung aber dem geneigten Leser überlassen wird.  Der
dort mittlerweile erfolgte Rückzug auf Begriffe wie „Muster“ oder „Trend“
gegenüber dem schärferen und wissenschaftspraktisch vorbelegten Begriff
„Gesetz“ unterstützt das Anliegen der Autoren von \cite{TESE2018}, empirische
Erfahrung zu systematisieren, verweist aber zugleich auf das schwache
theoretische Fundament eines solchen Systematisierungsanliegens.  Das weite
Spektrum praktisch kursierender Präzisierungen eines derart im Ungewissen
gelassenen Begriffs wurde in einer Facebook-Diskussion \cite{Graebe2019} im
August 2019 deutlich. Für ein genaueres Abwägen der Argumente zu oben
formulierter These und Gegenthese ist ein solches Fundament allerdings nicht
ausreichend.

Wie kann der Begriff eines \emph{technischen Systems} also weiter geschärft
werden?  In unserem Seminar \cite{Graebe2020} haben wir „den Systembegriff als
Beschreibungsfokussierung identifiziert, mit der konkrete Phänomene durch
\emph{Reduktion auf das Wesentliche} [\ldots] einer Beschreibung zugänglich
werden.“  Die Reduktion richtet sich auf folgende drei Dimensionen
\cite[S. 18]{Graebe2020} 
\begin{itemize}
\item [(1)] Abgrenzung des Systems nach außen gegen eine \emph{Umwelt},
  Reduktion dieser Beziehungen auf Input/Output-Beziehungen und garantierten
  Durchsatz.
\item [(2)] Abgrenzung des Systems nach innen durch Zusammenfassen von
  Teilbereichen als \emph{Komponenten}, deren Funktionieren auf eine
  „Verhaltenssteuerung“ über Input/Output-Bezie"|hungen reduziert wird.
\item [(3)] Reduktion der Beziehungen im System selbst auf „kausal
  wesentliche“ Beziehungen.
\end{itemize}
Weiter wird ebenda festgestellt, dass -- ähnlich wie im Konzertbeispiel --
einer solchen reduktiven Beschreibungsleistung vorgefundene (explizite oder
implizite) Beschreibungsleistungen vorgängig sind:
\begin{enumerate}
\item[(1)] Eine wenigstens vage Vorstellung über die Input/Output-Leistungen
  der Umgebung.
\item[(2)] Eine deutliche Vorstellung über das innere Funktionieren der
  Komponenten.
\item[(3)] Eine wenigstens vage Vorstellung über Kausalitäten im System
  selbst, also eine der detaillierten Modellierung vorgängige, bereits
  vorgefundene Vorstellung von Kausalität im gegebenen Kontext.
\end{enumerate}
Die Punkte (1) und (2) können ihrerseits in systemtheoretischen Ansätzen für
die Beschreibung der „Umwelt“ (hierfür ist allerdings die Abgrenzung eines
oder mehrerer Obersysteme in einer noch umfassenderen „Umwelt“ erforderlich)
sowie der Komponenten (als Untersysteme) entwickelt werden, womit die
Beschreibung von \emph{Koevolutionsszenarien} wichtig wird, die ihrerseits für
die Vertiefung des Verständnisses von Punkt (3) relevant sind.

Dabei ist der Fokus zunächst auf ein genaueres Verständnis des Begriffs
\emph{System} gerichtet, der als Reduktion von Komplexität in den drei oben
angeführten Richtungen betrachtet wird. Da in diesem Verständnis Komponenten
eines Systems selbst wieder Systeme sind, liegt auch im allgemeinen Fall die
Betrachtung eines Systems als „System von Systemen“ nahe, wie es in
\cite{Holling2000} thematisiert ist.  Wesentliches Reduktionskriterium für
Beziehungen zwischen Komponenten sind in solchen Systemen spezifische
Eigenzeiten und Eigenräume wie in den Abbildungen 1--3 in \cite{Holling2000}
dargestellt ist.

Die Beschreibung von Planung, Entwurf und Verbesserung \emph{technischer
  Systeme} geht in einem solchen Ansatz von der Leistungsfähig"|keit bereits
vorhandener technischer Systeme aus, die sowohl in (2) als Komponenten als
auch -- aus der Sicht eines Systems im Obersystem -- in (3) als benachbarte
Systeme zu berücksichtigen sind.

Ingenieur-technische Praxen bewegen sich damit in einer \emph{Welt von
  technischen Systemen}. Aus der konkreten Beschreibungsperspektive eines
Systems sind andere Systeme als Komponenten oder Nachbarsysteme allein in
ihrer \emph{Spezifikation} wichtig. Eine solche Reduktion auf das Wesentliche
erscheint praktisch als verkürzte Sprechweise über eine gesellschaftliche
Normalität, was ich kurz als \emph{Fiktion} bezeichne.  Diese Fiktion kann und
wird im täglichen Sprachgebrauch so lange aufrecht erhalten, so lange die
gesellschaftlichen Umstände die Aufrechterhaltung der daran gebundenen
gesellschaftlichen Normalität garantieren können, so lange also der
\emph{Betrieb der entsprechenden Infrastrukturen} gewährleistet ist.
Technische Systeme sind damit wenigstens in ihrer Vollzugsdimension
\emph{immer} sozio-technische Systeme.

Das Ausblenden dieser sozialen Zusammenhänge wie von M. Rubin vorgeschlagen
kann sich also allenfalls auch die \emph{Planung} derartiger Systeme sowie
deren artefaktische Daseinsdimension beziehen, die den Betrieb der
erforderlichen Infrastruktur ausblendet oder in ein Obersystem verschiebt.
Letzteres ist aber unzweckmäßig, da das Beheben von Problemen im Betrieb eines
Systems Kenntnisse über dessen Funktionieren nicht nur auf der Ebene der
Spezifikation, sondern auch auf der Ebene der Implementierung erfordern.

Mehr noch lehrt die Theorie Dynamischer Systeme, dass die Kopplung zwischen
Systemen nicht so sehr allein von Durchsatzraten bestimmt wird, sondern in
ihrer Wirkung stark von zeitlichen Regimes in Form von Resonanzen und
Dissonanzen bestimmt sein können.  Damit kann das Zusammenspiel von System und
Systemkomponenten stark von der Verschränkung von Mikro- und Makrodynamik auf
kurzwelligen (Komponenten) und langwelligen (System) Skalen abhängen.  Ein
wesentliches viertes Charakteristikum \emph{autonom funktionierender
  technischer Systeme} ist eine Entkopplung dieser Systemdynamiken, im
Kontrast etwa zum TRIZ-Prinzip 19 der periodischen Wirkung, das auf die
Ausnutzung entsprechender Kopplungsphänomene gerichtet ist.





\section{Component Software. Funktion und Verhalten}
\section{Komponenten und Objekte. Zustände}
\section{Betrieb technischer Systeme}
\section{Evolution technischer Systeme}
\section{Bisheriger Text}

HGG: Die Ersetzung des Menschen als Gesetz der technischen Entwicklung wurzelt
in einem sehr merkwürdigen Verständnis des Begriffs \emph{Technik}, welches
das Offensichtliche vergisst -- es gibt keine \emph{technischen Systeme},
sondern nur \emph{technosoziale Systeme}.

Michail S. Rubin präzisierte in einer PM vom 10.11.2019 seine Position wie
folgt:
\begin{quote}
  Dies erfordert eine gesonderte Diskussion. Wir verweisen auf die Arbeit von
  Lubomirsky und Litvin, die sich auf die Verdrängung des Menschen aus
  technischen System bezieht.  Wir sind uns einig, dass dieses Phänomen kein
  Gesetz ist, sondern ein Trend, der einem anderen Gesetz folgt: dem Gesetz
  der Erhöhung der Autonomie von Systemen.  Wir haben die Liste von Gesetzen
  und Trends in der Ausschreibung entsprechend modifiziert. Sie haben absolut
  Recht, dass technische Systeme nicht unabhängig sind in ihrer Entwicklung
  und allgemeiner sozio-technische Systeme betrachtet werden müssen. Gesetze
  der Entwicklung sozio-technischer Systeme unterscheiden sich aber von den
  Gesetzen der Entwicklung technischer Systeme. Für rein technische Systeme
  kann wirklich der Trend der schrittweisen Herauslösung menschlicher
  Beteiligung beobachtet werden. Statt eines Ruderbootes erscheint ein Boot
  mit einem Motor. Die ganze industrielle Revolution des 17. Jahrhunderts ist
  auf der Verdrängung des Menschen durch Motoren und Maschinen aufgebaut. Die
  nächste technologische Revolution ist mit der Verdrängung des Menschen aus
  dem Bereich der Kontrolle durch Automatisierung und Computer verbunden. Das
  heißt aber nicht, dass aus technologischer Sicht der Mensch aus dem
  sozio-technischen System verdrängt wird. Im Gegenteil, der Mensch bleibt die
  Hauptanforderungsquelle für technische Systeme. Aber diese Anforderungen
  werden zunehmend ohne menschliches Eingreifen erfüllt. Dieser Trend ist auch
  charakteristisch für das Kino als technisches System\footnote{Das Thema der
    Aufgaben des TRIZ-Cups.}. Es ist klar, dass der Mensch weder aus dem
  Prozess der Schaffung von Filmwerken, noch von Kunstwerken, noch aus dem
  Prozess des Konsums von Kinoprodukten herausgedrängt wird, er bleibt das
  Zentrum all dieser Prozesse.
\end{quote}

In diesem Zusammenhang ergeben sich für mich eine Reihe von Fragen, die auch
in einer ersten Diskussion auf
Facebook\footnote{\url{https://www.facebook.com/groups/111602085556371}} nicht
ausgeräumt werden konnten.
\begin{enumerate}
\item Was ist ein \emph{technisches System} im Gegensatz zu einem
  \emph{sozio-technischen System}?
\item Wie ist der Ansatz \emph{Entwicklung technischer Systeme} zu verstehen?
  Gibt es eine Entwicklung einzelner technischer Systeme oder kann deren
  Entwicklung nur in der Gesamtheit technischer Systeme oder nur in noch
  umfassenderen gesellschaftlichen Strukturen sinnvoll besprochen werden?
\item In welchem Verhältnis steht \emph{der Mensch} zu einzelnen technischen
  Systemen und zur Gesamtheit seiner technischen Schöpfungen? In welchem
  Umfang ist bei dieser Frage zwischen dem \emph{Menschen als Gattungssubjekt}
  (dem verfügbaren Verfahrenswissen), einzelnen Menschen als handelnden
  \emph{Akteuren in Mittel-Zweck-Verhältnissen} (dem privaten
  Verfahrenskönnen) und kooperativen Akteuren als \emph{Betreiber der
    einzelnen technischen Systeme} (den institutionalisierten
  Verfahrensweisen) zu differenzieren?
\end{enumerate}

Derartige Fragen ergeben sich insbesondere beim Studium von sozio-ökologischen
Systemen, in welche die Wirkungen technischer Systeme ja offensichtlich
eingebettet sind.  Siehe hierzu etwa die Ansätze von Elinor Ostrom in
\cite{Anderies2004} sowie \cite{Ostrom2007}, die in unserem Leipziger
Seminar\footnote{\url{https://github.com/wumm-project/Leipzig-Seminar}.}
gerade besprochen werden.

\section{Was sind technische Systeme?}

\subsection{Einige vorbereitende Überlegungen}

Insbesondere der letzte Punkt, der Zusammenhang zwischen einer Komponenten als
Konzept und den realweltlich verbauten Komponenteninstanzen, ist komplex, da
die produktiven Strukturen der Herstellung und des Einsatzes dieser
Komponenteninstanzen gewöhnlich auseinanderfallen, die Komponenteninstanzen
nach der Herstellung also verschickt und an ihrem Einsatzort für den konkreten
Gebrauch vorbereitet und verbaut werden müssen. In der Theorie einer
\emph{Software aus Komponenten} werden dabei die drei Phasen \emph{deploy,
  install, configure} deutlich unterschieden. 

\subsection{Kommentar von Nikolay Shpakovski, 8.12.2019}

Gesetze und Entwicklungslinien werden aktiv bei der Lösung von situativen und
prognostischen Aufgaben eingesetzt. Es geht um das System, aber sehr wenig,
und das habe ich schon lange verstanden.

In letzter Zeit denke ich oft an das Konzept des „technischen Systems“. Dieses
Konzept ist ein wichtiger Teil des Prozesses zur Lösung von Problemen nach
unserem Ansatz. Ich finde nichts Falsches am Ansatz des VDI\footnote{Auf
  Facebook schrieb ich dazu: Als zentrale Frage steht für mich, was überhaupt
  ein „Technisches System“ ist. Ist dieser Begriff in der Mehrzahl, wie im
  TRIZ-Kontext wie selbstverständlich gebraucht, überhaupt sinnvoll
  verwendbar? Der VDI -- Verein Deutscher Ingenieure -- als
  Standesorganisation, der in der VDI-Richtlinie 3780 den Technikbegriff
  normiert, ist in dieser Frage uneins, indem er von einer „Menge von
  Systemen“ spricht und Technik in folgenden drei Dimensionen betrachtet: 
  \begin{itemize}
  \item Menge der nutzenorientierten, künstlichen, gegenständlichen Gebilde
    (Artefakte oder Sachsysteme);
  \item Menge menschlicher Handlungen und Einrichtungen, in denen Sachsysteme
    entstehen und
  \item Menge menschlicher Handlungen, in denen Sachsysteme verwendet werden.
  \end{itemize}}, alles stimmt, alles auf der Welt kann als System betrachtet
werden.  Jedes System kann als „System von Systemen“ dargestellt werden, wir
wählen einfach irgendeine Ebene aus und sagen -- das ist ein System.  Dann
ergibt sich sofort die Möglichkeit zu sagen, dass es Obersysteme und
Subsysteme gibt.

Du hast eine konkrete Frage gestellt - was ist der Unterschied zwischen den
Konzepten „System -- Subsysteme“ und „System -- Komponenten“. Es ist einfach
-- die Komponente ist ein noch nicht systematisierter Teil des Systems, ein
potenzielles Teilsystem.
\begin{quote}
  Anmerkung HGG: Das widerspricht aber dem Verständnis der
  Komponententechnologie, nach dem die Komponenten zur Bauzeit des Systems,
  also \emph{vor} dessen Betrieb vorhanden sein müssen.
\end{quote}

Das Konzept des „technischen Systems“ ist in der TRIZ schrecklich verstrickt.
Als technisches System wird eine Reihe von Mechanismen betrachtet, die eine
neue Qualität ergeben, zum Beispiel ein Auto, ein Stift, eine Uhr. Als
technisches System wird ein System zur Durchführung einiger Funktionen
bezeichnet, beispielsweise zum Transport von Gütern, wozu außer dem Auto noch
viel mehr gehört. Das ist nicht schlimm, das Problem ist, dass diese
Definitionen kühn vermischt werden, was zu schrecklicher Verwirrung führt.
Den Fahrer in das System Auto einbeziehen oder nicht? Was ist mit Benzin? Ist
Luft ein Teil des Autos oder nicht? Menschen leben mit diesen Verwirrungen
gut, bauen ganze Theorien und führen Seminare durch, was diese Verwirrungen
nur noch verstärkt.

Für mich unterscheide ich
\begin{enumerate}
\item ein technisches System (systematisiertes technisches Objekt, eine
  Maschine auf dem Lager),
\item ein funktionierendes System (was im Patent als „Maschine in Arbeit“
  bezeichnet wird),
\item ein nützliches technisches System (das, was ein nützliches Produkt
  herstellt).
\end{enumerate}

Natürlich verwirrt das Wort „technisch“ hier viel, aber in dieser Situation
ist das so zu verstehen, dass ein technisches System ein System ist, das Bezug
zur Technik (Ingenieurwesen) hat oder zur Technik des Durchführens irgendeiner
nützlichen Handlung. Wirf besser dieses Wort komplett weg. Das Wichtigste, das
Nützlichste zur Lösung des Problems ist ein nützliches System. Auf dieser
Ebene verliert das Wort „technisch“ seine Bedeutung, weil es ein Elektriker
sein kann, der eine Glühbirne einsetzt oder ein Raumschiff geht in die
Umlaufbahn oder ein Anwalt oder ein Computerprogramm. Das Hauptkriterium ist,
ob dies ein nützliches Ergebnis ergibt oder es sich um „Mozhaiskis
nicht-fliegendes Flugzeug“ handelt\footnote{Ein im russischen Kontext
  berühmtes Beispiel ähnlich dem „Schneider von Ulm“ im Deutschen, siehe
  \url{https://de.wikipedia.org/wiki/Geschichte_der_Luftfahrt}.}?

\subsection{Kommentar von Michail S. Rubin, 31.12.2019}

\paragraph{1.}
Was ist ein technisches System im Gegensatz zu einem sozio-technischen?

Die Antwort ist ganz einfach: Bei der Betrachtung eines technischen Systems
berücksichtigen wir keine anderen bestehenden Beziehungen (soziale,
wirtschaftliche, politische, wirtschaftliche, Marketing usw.) im System, mit
Ausnahme von Objekten und Beziehungen technischer Natur. Diese externen
(menschlichen, kulturellen) Beziehungen können durch zusätzliche Anforderungen
oder Einschränkungen an technische Objekte ersetzt werden.

Bei der Betrachtung von Systemen als sozio-technisch werden eine Reihe
technischer Objekte und Zusammenhänge berücksichtigt, beispielsweise wenn die
TRIZ-Analyse von Produktionsunternehmen nicht nur als technisches System
(Maschinen, Geräte), sondern die Fabrik als sozio-technisches Objekt
betrachtet wird: Bestellsystem und Marketing, Personalpolitik, Finanzen und
die wirtschaftliche Lage des Unternehmens, Systeme der Entscheidungsfindung
usw. Offensichtlich verändert dies den Gegenstand der Überlegungen und die
Forschungsinstrumente grundlegend.

Ich verweise dazu zum Beispiel auf meine Aufsätze \cite{Rubin2007} und
\cite{Rubin2010}.

\paragraph{2.}
Ist es möglich, die Entwicklung technischer Systeme isoliert von sozialen
Strukturen zu betrachten?

Unter dem Gesichtspunkt der Entwicklung der Materie in der Natur sollten
technische Systeme als soziokulturelle Systeme auf der Ebene wirtschaftlicher,
finanzieller, politischer und anderer Systeme klassifiziert werden, die als
Teil der menschlichen Kultur in einem zivilisierten Umfeld entstanden sind
und nur dort existieren können.

Gleichzeitig werden Geschäftssysteme, politische Systeme, ethische Systeme
usw. als eigen\-ständige Objekte betrachtet. Trotz der direkten Verbindung
dieser Systeme mit der Zivilisation, mit soziokulturellen Systemen, haben sie
ihre eigenen Gesetze und gelten als eigenständige Forschungsobjekte. Es ist
nicht überraschend, dass technische Systeme unabhängig voneinander betrachtet
werden können, aber auch als sozial und technisch betrachtet werden können
(d.h. das soziale Umfeld einschließen).

\paragraph{3.}
Zwischen Menschen und technischen Systemen gibt es viele sehr unterschiedliche
Zusammenhänge. Welche Aspekte dieser Verbindungen interessiert Sie?  Erzeugung
und Entwicklung, Betrieb, ästhetische Bindungen, ethische, wirtschaftliche,
politische, künstlerische, biologische -- es gibt eine Vielzahl von Schichten
des Verhältnisses von technischen Systemen und Menschen.

\subsection{Kommentar von Hans-Gert Gräbe, 18.01.2020}

Mir ist bewusst, dass ich mit meinen Fragen einen großen diskursiven Korpus
berühre, so dass sich kurzschlüssige Antworten oder gar Urteile verbieten.

Das wird in \cite{Rubin2010} besonders deutlich, mit dem Sie die von mir
aufgeworfenen Fragen in einen sehr weiten Kontext stellen. Dies ist zweifellos
möglich und wahrscheinlich auch sinnvoll. Sie beziehen sich dabei besonders
auf die Ergebnisse der sowjetischen Ethnografie, die im Westen wenig bekannt
sind.  Ich selbst bin mit wesentlichen Ansätzen von L.N.Gumilev wie
Passionarität und seinem Ansatz einer etwa 1500-jährigen „S-Kurven“-artigen
Entwicklung von Zivilisationen aus dem Sammelband \cite{Gumilev1994} etwas
vertraut.  Ins Deutsche übersetzt wurde bisher nur sein Text
\cite{Gumilev2017}, in dem er seinen Passionaritätsansatz auf den Übergang
von der Kiewer zur Moskauer Rus als zivilisatorischen „S-Kurven-Wechsel“
entwickelt.  Andere Autoren wie \cite{Adji2011} weisen darauf hin, dass die
Kiewer Rus zu einer Zeit entstanden ist, wo das Gebiet faktisch eine
normannische Kolonie war und das Establishment von Warägern gebildet wurde.
Nach diesem Ansatz ist die Quelle der Entwicklung dort dem Aufeinandertreffen
der Kulturen der Eroberer und der Eroberten geschuldet.  Die Migrationsströme
auf der Karte \cite[Karte 5]{Rubin2010} müssen in dem Kontext seit wenigstens
10\,000 Jahren auch als Ströme kultureller und technologischer
Entwicklungspotenziale und Vermischungen gesehen werden. So lange währt in
diesem Verständnis bereits die „Globalisierung“.

Über diese Zusammenhänge gibt es wichtige Arbeiten von V.I. Vernadsky,
insbesondere dessen Noosphärenansatz \cite{Vernadsky1997}. Welche Rolle
spielen in Ihrem Systemansatz diese Überlegungen?

Auch scheint nach \cite{KlixLanius1999} klar zu sein, dass diese
Migrationsbewegungen oft durch lokalen Klimastress in Folge globaler
Klimaveränderungen ausgelöst wurden, ja dieser Klimastress viel ursächlicher
für technologische Weiterentwicklungen war als Migration und
kulturell-technische Vermischung, die ja nur das global bereits Verfügbare
fusioniert, nicht aber zu neuen Ufern aufbricht.

Trotz Sesshaftigkeit seit der Neolithischen Revolution spielen derartige
Migrationsbewegungen noch bis in die Neuzeit eine wichtige Rolle.  Wir haben
uns in letzter Zeit -- durch die recht ungenauen Angaben in \cite{Adji2011}
angeregt -- etwas genauer für die Migrationsbewegungen in Europa im Zeitraum
1000 v.Chr. bis 400 n.Chr. interessiert und dabei für die von Gumilev in
diesem Territorium beschriebenen Entwicklungen, die in (Rubin 2010:Karte 17)
referenziert werden, aber ja weitgehend spekulativen Charakter haben, wenig
Bestätigung gefunden.

Ich komme zu meiner Ausgangsfrage über das Verhältnis technischer und
sozio-kultureller Systeme zurück. Meine Studentengruppe löst gerade die
Aufgabe „Schiffsmast“\footnote{Siehe
  \url{https://triztrainer.ru/tasks/section-1/ship-mast/}.} aus dem Minsker
TRIZ-Trainer.  Betrachtet man das Boot als Teil eines Wassertransportsystems,
dann ist die Lösung einfach: Gib in dein Navi ein, dass die Brückendurchfahrt
gesperrt ist, und lasse dieses einen anderen Weg suchen. Im Kontext eines
autonom fahrenden Boots (der Mensch ist -- in Ihrem Verständnis -- vollkommen
aus dem System „Boot“ verdrängt) gibt es wohl auch keine andere Lösung
(\textbf{These:} Wo kein Mensch, da auch kein TRIZ).

Natürlich ist das keine „ordentliche“ Lösung, man soll das System „Boot“
analysieren.  Für dessen Hauptfunktion (Wassertransport) ist der Mast komplett
entbehrlich, dessen beide Funktionen (Träger von Antenne und Oberlicht)
gehören zur Komponente „Steuerung“ des Systems „Boot“. Also muss man diese
Komponente als Teilsystem weiter analysieren -- wozu Antenne und wozu
Oberlicht?

Oberlicht: Im Ober-Ober-System „Wassertransport“ gibt es die Vorschrift
(technisch? sozio-kulturell?), einen hohen Mast zu beleuchten.  Diese vom
Menschen (in einem komplizierten sozio-kulturellen Prozess der
Institutionalisierung) vereinbarte Vorschrift entfaltet also ihre konkrete
Wirkung als technische Vorgabe (so jedenfalls ist das nach der von Ihnen
angegebenen Vorgehensweise in die Anforderungen an das System „Radio“ zu
importieren, dessen Komponente die Antenne ist, das wiederum ein Teil des
Systems „Steuerung“ ist). Dieses Problem kann also nach Studium der Regeln aus
dem Ober-Ober-System einfach gelöst werden -- kein Mast, auch kein Oberlicht
erforderlich.

Antenne: Gehört offensichtlich zur Komponente Funkverbindung als Teil des
Systems „Steuerung“. Auch wenn der Mensch in einem komplett autonom fahrenden
Boot eliminiert wäre -- auch jenes autonom fahrende Boot benötigte die von
Menschen betreute und betriebene Navigationsinfrastruktur, auf die der
Bootsführer im gegebenen Kontext über die Funkverbindung zugreift.

Zusammenfassend: Ihre Argumentation greift nach meinem Verständnis zu kurz. 

\begin{thebibliography}{xxx}
\bibitem{Altshuller1979} G.S. Altshuller. \foreignlanguage{russian}{Творчество
  как точная наука. — М.: «Советское радио»}, 1979. — С. 122 – 127.
\bibitem{Adji2011} Murad Adji (2011). \foreignlanguage{russian}{Азиатская
  Европа}. (Das asiatische Europa).  Astrel, Moskau. ISBN 978-5-271-36400-6.
\bibitem{Anderies2004} J.M. Anderies, M.A. Janssen, E. Ostrom (2004).
  Framework to Analyze the Robustness of Social-ecological Systems from an
  Institutional Perspective.\\ In: Ecology and Society 9 (1), 18.
\bibitem{Bertalanffy1950} Ludwig von Bertalanffy (1950). An outline of General
  System Theory. The British Journal for the Philosophy of Science, vol. I.2,
  134–165.
\bibitem{Goldovsky1983} B.I. Goldovsky (1983).
  \foreignlanguage{russian}{Система закономерностей построения и развития
    технических систем}.
  \url{https://triz-summit.ru/triz/metod/gold/regular/}.
\bibitem{KFK2000} Klaus Fuchs-Kittowski (2000).
  Wissens-Ko-ProduktionVerarbeitung, Verteilung und Entstehung von
  Informationen in kreativ-lernenden Organisationen.\\ In: Fuchs-Kittowski
  u.a. (Hrsg.). Organisationsinformatik und Digitale Bibliothek in der
  Wissenschaft. Wissenschaftsforschung, Jahrbuch 2000. Gesellschaft für
  Wissenschaftsforschung, Berlin.
  \url{http://www.wissenschaftsforschung.de/JB00_9-88.pdf}
\bibitem{Goldberg2016} Jörg Goldberg, André Leisewitz (2016). Umbruch der
  globalen Konzernstrukturen.\\ Z 108, S. 8--19.
\bibitem{Graebe2019} Hans-Gert Gräbe (2019).  A discussion about TRIZ
    and system thinking reported in my Open Discovery Blog.
    \url{https://wumm-project.github.io/2019-08-07}.
\bibitem{Graebe2020} Hans-Gert Gräbe (2020). Reader zum 16. Interdisziplinären
  Gespräch \emph{Das Konzept Resilienz als emergente Eigenschaft in offenen
    Systemen} am 7.2.2020 in Leipzig.
  \url{http://mint-leipzig.de/2020-02-07/Reader.pdf}.
\bibitem{Gumilev1994} Lev N. Gumilev (1994). \foreignlanguage{russian}{Конец и
  вновь начало}.  (Ende und wieder Anfang).  Aufsätze, zusammengestellt von
  N.V.Gumilev. Tanais Verlag, Moskau.
\bibitem{Gumilev2017} Lev N. Gumilev (2017). Von der Rus zu Russland.
  Monsenstein und Vannerdat, Münster.
\bibitem{Hamilton2008} Patrick Hamilton (2008). Wege aus der Softwarekrise:
  Verbesserungen bei der Softwareentwicklung. ISBN 978-3-540-72869-6.
\bibitem{Holling2000} C.S. Holling (2000). Understanding the Complexity of
  Economic, Ecological, and Social Systems. In: Ecosystems (2001) 4, 390–405.
\bibitem{Jacobasch2019} Gisela Jacobasch (2019). Bienensterben -- Ursachen und
  Folgen.  Leibniz Online 37 (2019).
  \url{https://leibnizsozietaet.de/bienensterben-ursachen-und-folgen/}
\bibitem{KlixLanius1999} Friedhart Klix, Karl Lanius (1999). Wege und Irrwege
  der Menschenartigen.  Kohlhammer, Stuttgart.
\bibitem{KoltzeSouchkov2017} Karl Koltze, Valeri Souchkov (2017).
  Systematische Innovation.\\ Hanser Verlag, München. Zweite Auf"|lage. ISBN
  978-3-446-45127-8.
  \bibitem{Kropik2009} Markus Kropik (2009). Produktionsleitsysteme in der
    Automobilfertigung.\\ ISBN 978-3-540-88991-5.
\bibitem{TESE2018} Alexander Lyubomirskiy, Simon Litvin, Sergey Ikovenko,
  Christian M. Thurnes, Robert Adunka (2018). Trends of Engineering System
  Evolution. Eigenverlag, Sulzbach-Rosenberg.  ISBN 978-3-00-059846-3.
\bibitem{MEW23} Karl Marx (MEW 23). Das Kapital, Band 1. Dietz Verlag, Berlin.
\bibitem{MEW42} Karl Marx (MEW 42). Grundrisse der Kritik der politischen
  Ökonomie.  Dietz Verlag, Berlin.
\bibitem{Ostrom2007} Elinor Ostrom (2007). A diagnostic approach for going
  beyond panaceas.  Proceedings of the national Academy of sciences, 104(39),
  15181--15187.
\bibitem{Rubin2007} Michail S. Rubin (2007). \foreignlanguage{russian}{О
  выборе задач в социально-технических системах}. (Über die Wahl von Aufgaben
  in sozial-technischen Systemen). In: \foreignlanguage{russian}{ТРИЗ Анализ.
    Методы исследования проблемных ситуаций и выявления инновационных
    задач}. (TRIZ-Analyse. Methoden zur Untersuchung von Problemsituationen
  und zur Identifizierung innovativer Aufgaben). Hrsg. von S.S. Litvin,
  V.M. Petrov, M.S. Rubin. \foreignlanguage{russian}{Библиотека Саммита
    Разработчиков ТРИЗ}, Moskau. S. 35--46.
  \url{https://www.trizland.ru/trizba/pdf-books/TRIZ-summit2007.pdf}.
\bibitem{Rubin2010} Michail S. Rubin
  (2010). \foreignlanguage{russian}{Филогенез социокультурных систем. Секреты
  развития цивилизаций}.  (Phylogenese soziokultureller Systeme. Geheimnisse
  der Zivilisationsentwicklung).
  \url{http://www.temm.ru/en/section.php?docId=4472}.
\bibitem{Szyperski2002} Clemens Szyperski (2002). Component Software: Beyond
  Object-Oriented Programming. ISBN: 978-0-321-75302-1.
\bibitem{TT} Target Invention (2020). TRIZ Trainer.
  \url{https://triztrainer.ru}.
\bibitem{VDMA2019} VDMA. Maschinenbau in Zahl und Bild 2019. 
\bibitem{Vernadsky1997} V.I. Vernadsky (1997, Original 1936--38). Scientific
  Thought as a Planetary
  Phenomenon. \url{https://wumm-project.github.io/Texts.html}
\end{thebibliography}
\end{document}
