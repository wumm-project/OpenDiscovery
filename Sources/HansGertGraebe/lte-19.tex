\documentclass[11pt,a4paper]{article}
\usepackage{a4wide,url}
\usepackage[german]{babel}
\usepackage[utf8]{inputenc}

\parindent0cm
\parskip3pt

\title{Zur Evolution Technischer Systeme} 
\author{Hans-Gert Gräbe, Leipzig}
\date{Version vom 9. Dezember 2019}
\begin{document}
\maketitle

Anlass dieser kleinen Skizze zum Thema „Gesetze und Trends der Evolution
Technischer Systeme“, die im TRIZ-Kontext in verschiedenen Varianten
diskutiert werden, war die Fassung dieser Gesetze und Tendenzen in den
Aufgaben des TRIZ-Cups 2019/20\footnote{Siehe dazu
  \url{https://triz-summit.ru/contest/cup-tds-2019-2020/contest-2019-2020/}
  (in Russisch) sowie die Synopse
  \url{https://wumm-project.github.io/Upload/lte.pdf}. }.

In jener Ausschreibung wurde auch ein „Gesetz der Verdrängung des Menschen“
(erste Fassung) formuliert, das in einer späteren Fassung zu einer „Tendenz
der Verdrängung des Menschen aus technischen Systemen“ präzisiert
wurde\footnote{Die fremdsprachigen Originale sind hier und im Weiteren vom
  Autor ins Deutsche übertragen. Es existiert eine russische Version dieses
  Textes, in der die russischsprachigen Zitate im Original zu finden sind.}.

Eine solche Tendenz ist weder in Altschullers Liste von acht solchen
Gesetzen\footnote{Siehe dazu etwa \url{https://de.wikipedia.org/wiki/TRIZ}.}
noch in der Auf"|listung von fünf Gesetzen und zehn Tendenzen in
(Koltze/Souchkov)\footnote{Karl Koltze, Valeri Souchkov. Systematische
  Innovation. ISBN 978-3-446-45127-8.} zu finden.

Sie widerspricht auch grundlegenden Setzungen in der deutschen kybernetischen
Schule, was ich in einer Note vom 8.11.2019 an die Autoren der Aufgaben des
TRIZ-Cups festgehalten habe:
\begin{quote}
  Das „Gesetz der Ersetzung des Menschen“ ist nicht in der Liste im deutschen
  TRIZ-Standardwerk (Koltze/Souchkov) zu finden und ich bin absolut anderer
  Meinung, dass dies ein Entwicklungsgesetz für technische Systeme sei.
  Zumindest in der deutschen Literatur gibt es bereits seit den 1980er Jahren
  eine lange Diskussion über dieses Thema. Zum Beispiel unterstreicht Klaus
  Fuchs-Kittowski in einer Zusammenfassung seiner
  Arbeiten\footnote{\url{http://www.informatik.uni-leipzig.de/~graebe/Texte/Fuchs-02.pdf}}
  \begin{quote}
    „Unsere Antwort auf die Frage war immer: Der Mensch ist die einzig kreative
    Produktivkraft, er muss Subjekt der Entwicklung sein und bleiben.  Daher
    ist das Konzept der Vollautomatisierung, nach dem der Mensch schrittweise
    aus dem Prozess eliminiert werden soll, verfehlt!"
  \end{quote}
  Die Ersetzung des Menschen als Gesetz der technischen Entwicklung wurzelt in
  einem sehr merkwürdigen Verständnis des Begriffs \emph{Technik}, welches das
  Offensichtliche vergisst -- es gibt keine \emph{technischen Systeme},
  sondern nur \emph{technosoziale Systeme}.
\end{quote}

Michail Rubin präzisierte in einer PM vom 10.11.2019 seine Position wie folgt:
\begin{quote}
  Dies erfordert eine gesonderte Diskussion. Wir verweisen auf die Arbeit von
  Lubomirsky und Litvin, die sich auf die Verdrängung des Menschen aus
  technischen System bezieht.  Wir sind uns einig, dass dieses Phänomen kein
  Gesetz ist, sondern ein Trend, der einem anderen Gesetz folgt: dem Gesetz
  der Erhöhung der Autonomie von Systemen.  Wir haben die Liste von Gesetzen
  und Trends in der Ausschreibung entsprechend modifiziert. Sie haben absolut
  Recht, dass technische Systeme nicht unabhängig sind in ihrer Entwicklung
  und allgemeiner sozio-technische Systeme betrachtet werden müssen. Gesetze
  der Entwicklung sozio-technischer Systeme unterscheiden sich aber von den
  Gesetzen der Entwicklung technischer Systeme. Für rein technische Systeme
  kann wirklich der Trend der schrittweisen Herauslösung menschlicher
  Beteiligung beobachtet werden. Statt eines Ruderbootes erscheint ein Boot
  mit einem Motor. Die ganze industrielle Revolution des 17. Jahrhunderts ist
  auf der Verdrängung des Menschen durch Motoren und Maschinen aufgebaut. Die
  nächste technologische Revolution ist mit der Verdrängung des Menschen aus
  dem Bereich der Kontrolle durch Automatisierung und Computer verbunden. Das
  heißt aber nicht, dass aus technologischer Sicht der Mensch aus dem
  sozio-technischen System verdrängt wird. Im Gegenteil, der Mensch bleibt die
  Hauptanforderungsquelle für technische Systeme. Aber diese Anforderungen
  werden zunehmend ohne menschliches Eingreifen erfüllt. Dieser Trend ist auch
  charakteristisch für das Kino als technisches System\footnote{Das Thema der
    Aufgaben des TRIZ-Cups.}. Es ist klar, dass der Mensch weder aus dem
  Prozess der Schaffung von Filmwerken, noch von Kunstwerken, noch aus dem
  Prozess des Konsums von Kinoprodukten herausgedrängt wird, er bleibt das
  Zentrum all dieser Prozesse.
\end{quote}

In diesem Zusammenhang ergeben sich für mich eine Reihe von Fragen, die auch
in einer ersten Diskussion auf
Facebook\footnote{\url{https://www.facebook.com/groups/111602085556371}} nicht
ausgeräumt werden konnten.
\begin{enumerate}
\item Was ist ein \emph{technisches System} im Gegensatz zu einem
  \emph{sozio-technischen System}?
\item Wie ist der Ansatz \emph{Evolution technischer Systeme} zu verstehen?
  Gibt es eine Evolution einzelner technischer Systeme oder kann deren
  Evolution nur in der Gesamtheit technischer Systeme oder nur in noch
  umfassenderen gesellschaftlichen Strukturen sinnvoll besprochen werden?
\item In welchem Verhältnis steht \emph{der Mensch} zu einzelnen technischen
  Systemen und zur Gesamtheit seiner technischen Schöpfungen? In welchem
  Umfang ist bei dieser Frage zwischen dem \emph{Menschen als Gattungssubjekt}
  (dem verfügbaren Verfahrenswissen), einzelnen Menschen als handelnden
  \emph{Akteuren in Mittel-Zweck-Verhältnissen} (dem privaten
  Verfahrenskönnen) und kooperativen Akteuren als \emph{Betreiber der
    einzelnen technischen Systeme} (den institutionalisierten
  Verfahrensweisen) zu differenzieren?
\end{enumerate}

Derartige Fragen ergeben sich insbesondere beim Studium von sozio-ökologischen
Systemen, in welche die Wirkungen technischer Systeme ja offensichtlich
eingebettet sind.  Siehe hierzu etwa die Ansätze von Elinor
Ostrom\footnote{Anderies, John M., Marco A. Janssen, Elinor Ostrom (2004).
  Framework to Analyze the Robustness of Social-ecological Systems from an
  Institutional Perspective. In: Ecology and Society 9 (1), 18. -- Ostrom,
  Elinor (2007). A diagnostic approach for going beyond panaceas.  Proceedings
  of the national Academy of sciences, 104(39), 15181--15187.}, die in unserem
Leipziger
Seminar\footnote{\url{https://github.com/wumm-project/Leipzig-Seminar}.}
gerade besprochen werden.

\section{Was sind technische Systeme?}

Die große Mehrzahl der von Menschen erschaffenen technischen Systeme sind
Unikate. Der Wirtschaftszweig, der mit der Herstellung solcher Unikate befasst
ist, heißt \emph{Industrieanlagenbau}. Auch die Mehrzahl der Infomatiker ist
mit der Erstellung solcher Unikate befasst, denn die IT-Systeme, die derartige
Anlagen steuern, sind ebenfalls Unikate.  Dasselbe gilt auch für die Ämter,
Behörden und öffentlichen Einrichtungen. So ist zum Beispiel die Leipziger
Stadtverwaltung aktuell damit befasst, ihre Verwaltungsprozesse zu
„digitalisieren“, was unter Führung des Dezernats Allgemeine Verwaltung und
zusammen mit dem städtischen IT-Dienstleister Lecos erfolgt.

Natürlich wird dabei das Fahrrad nicht dauernd neu erfunden --
Komponententechnologien bilden die Grundlage jeder ingenieur-technischen
Arbeit und auch die Informatik hat nach einer mehr als 25 Jahre dauernden
Softwarekrise\footnote{Patrick Hamilton (2008). Wege aus der Softwarekrise:
  Verbesserungen bei der Softwareentwicklung. ISBN 978-3-540-72869-6.} zu
komponentenbasierten Entwicklungsmethodiken gefunden\footnote{Clemens
  Szyperski (2002). Component Software: Beyond Object-Oriented
  Programming. ISBN: 978-0-321-75302-1.}.  In diesem Kontext haben sich
allerdings auch die Berufsbilder der Informatiker differenziert in
Komponentenentwickler („design for component“) und Komponentenmonteure
(„design from component“).  Erstere entwickeln Komponenten für einen größeren
Markt, zweitere entwickeln daraus weiterhin die großen Unikate („Systeme“ auch
in der informatischen Fachsprache).

Wir bewegen uns dabei klar im Bereich der Standard-TRIZ-Terminologie eines
\emph{Systems von Systemen} -- ein technisches System besteht aus Komponenten,
die ihrerseits technische Systeme sind, deren \emph{Funktionieren} (sowohl im
funktionalen als auch im operativen Sinn) für die aktuell betrachtete
Systemebene vorausgesetzt wird. Der Begriff eines technischen Systems hat
damit eine klar epistemische Funktion der „Reduktion auf das Wesentliche“.
Einstein wird der Ausspruch zugeschrieben „make it as simple as possible but
not simpler“. Das \emph{Gesetz der Vollständigkeit eines Systems} bringt genau
diesen Gedanken zum Ausdruck, allerdings nicht als \emph{Gesetz}, sondern als
\emph{Modellierungsdirektive}.

Der Begriff \emph{technisches System} ist in einem solchen
planerisch-realweltlichen Kontext vierfach überladen
\begin{itemize}
\item [1.] als realweltliches Unikat,
\item [2.] als Beschreibung dieses realweltlichen Unikats
\end{itemize}
und für in größerer Stückzahl hergestellte Komponenten auch noch
\begin{itemize}
\item [3.] als Beschreibung des Designs des System-Templates sowie
\item [4.] als Beschreibung der Auslieferungs- und Betriebsstrukturen der nach
  diesem Template gefertigten realweltlichen Unikate. 
\end{itemize}

Insbesondere der letzte Punkt, der Zusammenhang zwischen einer Komponenten als
Konzept und den realweltlich verbauten Komponenteninstanzen, ist komplex, da
die produktiven Strukturen der Herstellung und des Einsatzes dieser
Komponenteninstanzen gewöhnlich auseinanderfallen, die Komponenteninstanzen
nach der Herstellung also verschickt und an ihrem Einsatzort für den konkreten
Gebrauch vorbereitet und verbaut werden müssen. In der Theorie einer
\emph{Software aus Komponenten} werden dabei die drei Phasen \emph{deploy,
  install, configure} deutlich unterschieden. 


\end{document}
