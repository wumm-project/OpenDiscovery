\documentclass[11pt,a4paper]{article}
\usepackage{a4wide,url}
\usepackage[russian,german]{babel}
\usepackage[utf8]{inputenc}

\parindent0cm
\parskip3pt

\title{Zur Entwicklung Technischer Systeme} 
\author{Hans-Gert Gräbe, Leipzig}
\date{Version vom 18. Januar 2020}
\begin{document}
\maketitle

\section{Vorbemerkungen}

Anlass dieser kleinen Skizze zum Thema „Gesetze und Trends der Entwicklung
Technischer Systeme“, die im TRIZ-Kontext in verschiedenen Varianten
diskutiert werden, war die Fassung dieser Gesetze und Tendenzen in den
Aufgaben des TRIZ-Cups 2019/20\footnote{Siehe dazu
  \url{https://triz-summit.ru/contest/cup-tds-2019-2020/contest-2019-2020/}
  (in Russisch) sowie die Synopse
  \url{https://wumm-project.github.io/Upload/lte.pdf}. }.

In jener Ausschreibung wurde auch ein „Gesetz der Verdrängung des Menschen“
(erste Fassung) formuliert, das in einer späteren Fassung zu einer „Tendenz
der Verdrängung des Menschen aus technischen Systemen“ präzisiert
wurde\footnote{Die fremdsprachigen Originale sind hier und im Weiteren vom
  Autor ins Deutsche übertragen. Es existiert eine russische Version dieses
  Textes, in der die russischsprachigen Zitate im Original zu finden sind.}.

Eine solche Tendenz ist weder in Altschullers Liste von acht solchen
Gesetzen\footnote{Siehe dazu etwa \url{https://de.wikipedia.org/wiki/TRIZ}.}
noch in der Auf"|listung von fünf Gesetzen und zehn Tendenzen in
(Koltze/Souchkov 2018) zu finden.  Zu finden ist ein solche Tendenz allerdings
in der 2018 herausgegebenen von MATRIZ autorisierten Version der \emph{Trends
  of Engineering System Evolution} (TESE), siehe (Lyubomirskiy u.a. 2018).

Damit steht natürlich die Frage nach den kontextuellen Annahmen, die zu diesen
verschiedenen Positionen führen. In der marxistischen Literatur wird ein
solcher Herauslösungsprozess ebenfalls betrachtet.  Im „Maschinenfragment“
(MEW 42:570 ff.) -- einem frühen Rohentwurf der eigenen ökonomischen Theorie
-- entwickelt Marx die Vision einer Gesellschaft, in welcher der
„gesellschaftliche Stoffwechsel“ (MEW 23:37) auf eine Weise organisiert ist,
dass
\begin{quote}
  es nicht mehr der Arbeiter [ist], der modifizierten Naturgegenstand als
  Mittelglied zwischen das Objekt und sich einschiebt; sondern den
  Naturprozess, den er in einen industriellen umwandelt, er als Mittel
  zwischen sich und die unorganische Natur [schiebt], deren er sich
  bemeistert.  (MEW 42:592)
\end{quote}
Weiter stellt Marx dar, dass die Entwicklung der Produktivkräfte
\emph{notwendig} auf eine solche Weise der Organisation des gesellschaftlichen
Stoffwechsels zusteuert.
\begin{quote}
  In den Produktionsprozess des Kapitals aufgenommen, durchläuft das
  Arbeitsmittel aber verschiedene Metamorphosen, deren letzte die
  \emph{Maschine} ist oder vielmehr ein \emph{automatisches System der
    Maschinerie} (System der Maschinerie; das \emph{automatische} ist nur die
  vollendetste adäquateste Form derselben und verwandelt die Maschinerie erst
  in ein System), in Bewegung gesetzt durch einen Automaten, bewegende Kraft,
  die sich selbst bewegt; dieser Automat bestehend aus zahlreichen mechanischen
  und intellektuellen Organen, sodass die Arbeiter selbst nur als bewusste
  Glieder desselben bestimmt sind. (MEW 42:584)
\end{quote}
Dieser Gedanke ist weitgehend singulär und im übrigen Marxschen Werk nirgends
ausgearbeitet. So wenigstens (Goldberg/Leisewitz 2016).

Die Probleme solcher „in Bewegung gesetzten Automaten“ werden mittlerweile in
einer ökolo\-gischen Krise planetaren Ausmaßes sichtbar, so dass die Frage
steht, ob ein behaupteter „Trend der Verdrängung des Menschen aus technischen
Systemen“ nicht eine grundsätzliche theoretische Fehlkonstruktion markiert,
die im aktuell verwendeten Komplex dieser Gesetze und Trends der Entwicklung
technischer Systeme enthalten ist. 

Der Ansatz einer solchen Verdrängungstendenz widerspricht auch grundlegenden
Setzungen in der deutschen kybernetischen Schule, was ich in einer Note vom
8.11.2019 an die Autoren der Aufgaben des TRIZ-Cups festgehalten habe:
\begin{quote}
  Das „Gesetz der Ersetzung des Menschen“ ist nicht in der Liste im deutschen
  TRIZ-Standardwerk (Koltze/Souchkov) zu finden und ich bin absolut anderer
  Meinung, dass dies ein Entwicklungsgesetz für technische Systeme sei.
  Zumindest in der deutschen Literatur gibt es bereits seit den 1980er Jahren
  eine lange Diskussion über dieses Thema. Zum Beispiel unterstreicht Klaus
  Fuchs-Kittowski in der Zusammenfassung (Fuchs-Kittowski 2001) seiner
  Arbeiten
  \begin{quote}
    „Unsere Antwort auf die Frage war immer: Der Mensch ist die einzig kreative
    Produktivkraft, er muss Subjekt der Entwicklung sein und bleiben.  Daher
    ist das Konzept der Vollautomatisierung, nach dem der Mensch schrittweise
    aus dem Prozess eliminiert werden soll, verfehlt!“ (ebenda:10)
  \end{quote}
  Die Ersetzung des Menschen als Gesetz der technischen Entwicklung wurzelt in
  einem sehr merkwürdigen Verständnis des Begriffs \emph{Technik}, welches das
  Offensichtliche vergisst -- es gibt keine \emph{technischen Systeme},
  sondern nur \emph{technosoziale Systeme}.
\end{quote}

Michail S. Rubin präzisierte in einer PM vom 10.11.2019 seine Position wie
folgt:
\begin{quote}
  Dies erfordert eine gesonderte Diskussion. Wir verweisen auf die Arbeit von
  Lubomirsky und Litvin, die sich auf die Verdrängung des Menschen aus
  technischen System bezieht.  Wir sind uns einig, dass dieses Phänomen kein
  Gesetz ist, sondern ein Trend, der einem anderen Gesetz folgt: dem Gesetz
  der Erhöhung der Autonomie von Systemen.  Wir haben die Liste von Gesetzen
  und Trends in der Ausschreibung entsprechend modifiziert. Sie haben absolut
  Recht, dass technische Systeme nicht unabhängig sind in ihrer Entwicklung
  und allgemeiner sozio-technische Systeme betrachtet werden müssen. Gesetze
  der Entwicklung sozio-technischer Systeme unterscheiden sich aber von den
  Gesetzen der Entwicklung technischer Systeme. Für rein technische Systeme
  kann wirklich der Trend der schrittweisen Herauslösung menschlicher
  Beteiligung beobachtet werden. Statt eines Ruderbootes erscheint ein Boot
  mit einem Motor. Die ganze industrielle Revolution des 17. Jahrhunderts ist
  auf der Verdrängung des Menschen durch Motoren und Maschinen aufgebaut. Die
  nächste technologische Revolution ist mit der Verdrängung des Menschen aus
  dem Bereich der Kontrolle durch Automatisierung und Computer verbunden. Das
  heißt aber nicht, dass aus technologischer Sicht der Mensch aus dem
  sozio-technischen System verdrängt wird. Im Gegenteil, der Mensch bleibt die
  Hauptanforderungsquelle für technische Systeme. Aber diese Anforderungen
  werden zunehmend ohne menschliches Eingreifen erfüllt. Dieser Trend ist auch
  charakteristisch für das Kino als technisches System\footnote{Das Thema der
    Aufgaben des TRIZ-Cups.}. Es ist klar, dass der Mensch weder aus dem
  Prozess der Schaffung von Filmwerken, noch von Kunstwerken, noch aus dem
  Prozess des Konsums von Kinoprodukten herausgedrängt wird, er bleibt das
  Zentrum all dieser Prozesse.
\end{quote}

In diesem Zusammenhang ergeben sich für mich eine Reihe von Fragen, die auch
in einer ersten Diskussion auf
Facebook\footnote{\url{https://www.facebook.com/groups/111602085556371}} nicht
ausgeräumt werden konnten.
\begin{enumerate}
\item Was ist ein \emph{technisches System} im Gegensatz zu einem
  \emph{sozio-technischen System}?
\item Wie ist der Ansatz \emph{Entwicklung technischer Systeme} zu verstehen?
  Gibt es eine Entwicklung einzelner technischer Systeme oder kann deren
  Entwicklung nur in der Gesamtheit technischer Systeme oder nur in noch
  umfassenderen gesellschaftlichen Strukturen sinnvoll besprochen werden?
\item In welchem Verhältnis steht \emph{der Mensch} zu einzelnen technischen
  Systemen und zur Gesamtheit seiner technischen Schöpfungen? In welchem
  Umfang ist bei dieser Frage zwischen dem \emph{Menschen als Gattungssubjekt}
  (dem verfügbaren Verfahrenswissen), einzelnen Menschen als handelnden
  \emph{Akteuren in Mittel-Zweck-Verhältnissen} (dem privaten
  Verfahrenskönnen) und kooperativen Akteuren als \emph{Betreiber der
    einzelnen technischen Systeme} (den institutionalisierten
  Verfahrensweisen) zu differenzieren?
\end{enumerate}

Derartige Fragen ergeben sich insbesondere beim Studium von sozio-ökologischen
Systemen, in welche die Wirkungen technischer Systeme ja offensichtlich
eingebettet sind.  Siehe hierzu etwa die Ansätze von Elinor Ostrom in
(Anderies u.a. 2004) sowie (Ostrom 2007), die in unserem Leipziger
Seminar\footnote{\url{https://github.com/wumm-project/Leipzig-Seminar}.}
gerade besprochen werden.

\section{Was sind technische Systeme?}

\subsection{Einige vorbereitende Überlegungen}

Die große Mehrzahl der von Menschen erschaffenen technischen Systeme sind
Unikate. Der Wirtschaftszweig, der mit der Herstellung solcher Unikate befasst
ist, heißt \emph{Industrieanlagenbau}. Auch die Mehrzahl der Infomatiker ist
mit der Erstellung solcher Unikate befasst, denn die IT-Systeme, die derartige
Anlagen steuern, sind ebenfalls Unikate.  Dasselbe gilt auch für die Ämter,
Behörden und öffentlichen Einrichtungen. So ist zum Beispiel die Leipziger
Stadtverwaltung aktuell damit befasst, ihre Verwaltungsprozesse zu
„digitalisieren“, was unter Führung des Dezernats Allgemeine Verwaltung und
zusammen mit dem städtischen IT-Dienstleister Lecos erfolgt.

Natürlich wird dabei das Fahrrad nicht dauernd neu erfunden --
Komponententechnologien bilden die Grundlage jeder ingenieur-technischen
Arbeit und auch die Informatik hat nach einer mehr als 25 Jahre dauernden
Softwarekrise (Hamilton 2008) zu komponentenbasierten Entwicklungsmethodiken
gefunden (Szyperski 2002).  In diesem Kontext haben sich allerdings auch die
Berufsbilder der Informatiker differenziert in Komponentenentwickler („design
for component“) und Komponentenmonteure („design from component“).  Erstere
entwickeln Komponenten für einen größeren Markt, zweitere entwickeln daraus
weiterhin die großen Unikate („Systeme“ auch in der informatischen
Fachsprache).

Wir bewegen uns dabei klar im Bereich der Standard-TRIZ-Terminologie eines
\emph{Systems von Systemen} -- ein technisches System besteht aus Komponenten,
die ihrerseits technische Systeme sind, deren \emph{Funktionieren} (sowohl im
funktionalen als auch im operativen Sinn) für die aktuell betrachtete
Systemebene vorausgesetzt wird. Der Begriff eines technischen Systems hat
damit eine klar epistemische Funktion der „Reduktion auf das Wesentliche“.
Einstein wird der Ausspruch zugeschrieben „make it as simple as possible but
not simpler“. Das \emph{Gesetz der Vollständigkeit eines Systems} bringt genau
diesen Gedanken zum Ausdruck, allerdings tritt er hier nicht als
\emph{Gesetz}, sondern als \emph{Modellierungsdirektive} in Erscheinung.

Der Begriff \emph{technisches System} ist in einem solchen
planerisch-realweltlichen Kontext vierfach überladen
\begin{itemize}
\item [1.] als realweltliches Unikat,
\item [2.] als Beschreibung dieses realweltlichen Unikats
\end{itemize}
und für in größerer Stückzahl hergestellte Komponenten auch noch
\begin{itemize}
\item [3.] als Beschreibung des Designs des System-Templates sowie
\item [4.] als Beschreibung und Betrieb der Auslieferungs- und
  Betriebsstrukturen der nach diesem Template gefertigten realweltlichen
  Unikate.
\end{itemize}

Insbesondere der letzte Punkt, der Zusammenhang zwischen einer Komponenten als
Konzept und den realweltlich verbauten Komponenteninstanzen, ist komplex, da
die produktiven Strukturen der Herstellung und des Einsatzes dieser
Komponenteninstanzen gewöhnlich auseinanderfallen, die Komponenteninstanzen
nach der Herstellung also verschickt und an ihrem Einsatzort für den konkreten
Gebrauch vorbereitet und verbaut werden müssen. In der Theorie einer
\emph{Software aus Komponenten} werden dabei die drei Phasen \emph{deploy,
  install, configure} deutlich unterschieden. 

\subsection{Kommentar von Nikolay Shpakovski, 8.12.2019}

Gesetze und Entwicklungslinien werden aktiv bei der Lösung von situativen und
prognostischen Aufgaben eingesetzt. Es geht um das System, aber sehr wenig,
und das habe ich schon lange verstanden.

In letzter Zeit denke ich oft an das Konzept des „technischen Systems“. Dieses
Konzept ist ein wichtiger Teil des Prozesses zur Lösung von Problemen nach
unserem Ansatz. Ich finde nichts Falsches am Ansatz des VDI\footnote{Auf
  Facebook schrieb ich dazu: Als zentrale Frage steht für mich, was überhaupt
  ein „Technisches System“ ist. Ist dieser Begriff in der Mehrzahl, wie im
  TRIZ-Kontext wie selbstverständlich gebraucht, überhaupt sinnvoll
  verwendbar? Der VDI -- Verein Deutscher Ingenieure -- als
  Standesorganisation, der in der VDI-Richtlinie 3780 den Technikbegriff
  normiert, ist in dieser Frage uneins, indem er von einer „Menge von
  Systemen“ spricht und Technik in folgenden drei Dimensionen betrachtet: 
  \begin{itemize}
  \item Menge der nutzenorientierten, künstlichen, gegenständlichen Gebilde
    (Artefakte oder Sachsysteme);
  \item Menge menschlicher Handlungen und Einrichtungen, in denen Sachsysteme
    entstehen und
  \item Menge menschlicher Handlungen, in denen Sachsysteme verwendet werden.
  \end{itemize}}, alles stimmt, alles auf der Welt kann als System betrachtet
werden.  Jedes System kann als „System von Systemen“ dargestellt werden, wir
wählen einfach irgendeine Ebene aus und sagen -- das ist ein System.  Dann
ergibt sich sofort die Möglichkeit zu sagen, dass es Obersysteme und
Subsysteme gibt.

Du hast eine konkrete Frage gestellt - was ist der Unterschied zwischen den
Konzepten „System -- Subsysteme“ und „System -- Komponenten“. Es ist einfach
-- die Komponente ist ein noch nicht systematisierter Teil des Systems, ein
potenzielles Teilsystem.
\begin{quote}
  Anmerkung HGG: Das widerspricht aber dem Verständnis der
  Komponententechnologie, nach dem die Komponenten zur Bauzeit des Systems,
  also \emph{vor} dessen Betrieb vorhanden sein müssen.
\end{quote}

Das Konzept des „technischen Systems“ ist in der TRIZ schrecklich verstrickt.
Als technisches System wird eine Reihe von Mechanismen betrachtet, die eine
neue Qualität ergeben, zum Beispiel ein Auto, ein Stift, eine Uhr. Als
technisches System wird ein System zur Durchführung einiger Funktionen
bezeichnet, beispielsweise zum Transport von Gütern, wozu außer dem Auto noch
viel mehr gehört. Das ist nicht schlimm, das Problem ist, dass diese
Definitionen kühn vermischt werden, was zu schrecklicher Verwirrung führt.
Den Fahrer in das System Auto einbeziehen oder nicht? Was ist mit Benzin? Ist
Luft ein Teil des Autos oder nicht? Menschen lenben mit diesen Verwirrungen
gut, bauen ganze Theorien und führen Seminare durch, was diese Verwirrungen
nur noch verstärkt.

Für mich unterscheide ich
\begin{enumerate}
\item ein technisches System (systematisiertes technisches Objekt, eine
  Maschine auf dem Lager),
\item ein funktionierendes System (was im Patent als „Maschine in Arbeit“
  bezeichnet wird),
\item ein nützliches technisches System (das, was ein nützliches Produkt
  herstellt).
\end{enumerate}

Natürlich verwirrt das Wort „technisch“ hier viel, aber in dieser Situation
ist das so zu verstehen, dass ein technisches System ein System ist, das Bezug
zur Technik (Ingenieurwesen) hat oder zur Technik des Durchführens irgendeiner
nützlichen Handlung. Wirf besser dieses Wort komplett weg. Das Wichtigste, das
Nützlichste zur Lösung des Problems ist ein nützliches System. Auf dieser
Ebene verliert das Wort „technisch“ seine Bedeutung, weil es ein Elektriker
sein kann, der eine Glühbirne einsetzt oder ein Raumschiff geht in die
Umlaufbahn oder ein Anwalt oder ein Computerprogramm. Das Hauptkriterium ist,
ob dies ein nützliches Ergebnis ergibt oder es sich um „Mozhaiskis
nicht-fliegendes Flugzeug“ handelt\footnote{Ein im russischen Kontext
  berühmtes Beispiel ähnlich dem „Schneider von Ulm“ im Deutschen, siehe
  \url{https://de.wikipedia.org/wiki/Geschichte_der_Luftfahrt}.}?

\subsection{Kommentar von Michail S. Rubin, 31.12.2019}

\paragraph{1.}
Was ist ein technisches System im Gegensatz zu einem sozio-technischen?

Die Antwort ist ganz einfach: Bei der Betrachtung eines technischen Systems
berücksichtigen wir keine anderen bestehenden Beziehungen (soziale,
wirtschaftliche, politische, wirtschaftliche, Marketing usw.) im System, mit
Ausnahme von Objekten und Beziehungen technischer Natur. Diese externen
(menschlichen, kulturellen) Beziehungen können durch zusätzliche Anforderungen
oder Einschränkungen an technische Objekte ersetzt werden.

Bei der Betrachtung von Systemen als sozio-technisch werden eine Reihe
technischer Objekte und Zusammenhänge berücksichtigt, beispielsweise wenn die
TRIZ-Analyse von Produktionsunternehmen nicht nur als technisches System
(Maschinen, Geräte), sondern die Fabrik als sozio-technisches Objekt
betrachtet wird: Bestellsystem und Marketing, Personalpolitik, Finanzen und
die wirtschaftliche Lage des Unternehmens, Systeme der Entscheidungsfindung
usw. Offensichtlich verändert dies den Gegenstand der Überlegungen und die
Forschungsinstrumente grundlegend.

Ich verweise dazu zum Beispiel auf meine Aufsätze (Rubin 2007) und (Rubin
2010).

\paragraph{2.}
Ist es möglich, die Entwicklung technischer Systeme isoliert von sozialen
Strukturen zu betrachten?

Unter dem Gesichtspunkt der Entwicklung der Materie in der Natur sollten
technische Systeme als soziokulturelle Systeme auf der Ebene wirtschaftlicher,
finanzieller, politischer und anderer Systeme klassifiziert werden, die als
Teil der menschlichen Kultur in einem zivilisierten Umfeld entstanden sind
und nur dort existieren können.

Gleichzeitig werden Geschäftssysteme, politische Systeme, ethische Systeme
usw. als eigen\-ständige Objekte betrachtet. Trotz der direkten Verbindung
dieser Systeme mit der Zivilisation, mit soziokulturellen Systemen, haben sie
ihre eigenen Gesetze und gelten als eigenständige Forschungsobjekte. Es ist
nicht überraschend, dass technische Systeme unabhängig voneinander betrachtet
werden können, aber auch als sozial und technisch betrachtet werden können
(d.h. das soziale Umfeld einschließen).

\paragraph{3.}
Zwischen Menschen und technischen Systemen gibt es viele sehr unterschiedliche
Zusammenhänge. Welche Aspekte dieser Verbindungen interessiert Sie?  Erzeugung
und Entwicklung, Betrieb, ästhetische Bindungen, ethische, wirtschaftliche,
politische, künstlerische, biologische -- es gibt eine Vielzahl von Schichten
des Verhältnisses von technischen Systemen und Menschen.

\subsection{Kommentar von Hans-Gert Gräbe, 18.01.2020}

Mir ist bewusst, dass ich mit meinen Fragen einen großen diskursiven Korpus
berühre, so dass sich kurzschlüssige Antworten oder gar Urteile verbieten.

Das wird in (Rubin 2010) besonders deutlich, mit dem Sie die von mir
aufgeworfenen Fragen in einen sehr weiten Kontext stellen. Dies ist zweifellos
möglich und wahrscheinlich auch sinnvoll. Sie beziehen sich dabei besonders
auf die Ergebnisse der sowjetischen Ethnografie, die im Westen wenig bekannt
sind.  Ich selbst bin mit wesentlichen Ansätzen von L.N.Gumilev wie
Passionarität und seinem Ansatz einer etwa 1500-jährigen „S-Kurven“-artigen
Entwicklung von Zivilisationen aus dem Sammelband (Gumilev 1994) etwas
vertraut.  Ins Deutsche übersetzt wurde bisher nur sein Text (Gumilev 2017),
in dem er seinen Passionaritätsansatz auf den Übergang von der Kiewer zur
Moskauer Rus als zivilisatorischen „S-Kurven-Wechsel“ entwickelt.  Andere
Autoren wie (Adji 2011) weisen darauf hin, dass die Kiewer Rus zu einer Zeit
entstanden ist, wo das Gebiet faktisch eine normannische Kolonie war und das
Establishment von Warägern gebildet wurde. Nach diesem Ansatz ist die Quelle
der Entwicklung dort dem Aufeinandertreffen der Kulturen der Eroberer und der
Eroberten geschuldet.  Die Migrationsströme auf der Karte (Rubin 2010:Karte 5)
müssen in dem Kontext seit wenigstens 10\,000 Jahren auch als Ströme
kultureller und technologischer Entwicklungspotenziale und Vermischungen
gesehen werden. So lange währt in diesem Verständnis bereits die
„Globalisierung“. 

Über diese Zusammenhänge gibt es wichtige Arbeiten von V.I. Vernadsky,
insbesondere dessen Noosphärenansatz (Vernadsky 1997). Welche Rolle spielen in
Ihrem Systemansatz diese Überlegungen?

Auch scheint nach (Klix/Lanius 1999) klar zu sein, dass diese
Migrationsbewegungen oft durch lokalen Klimastress in Folge globaler
Klimaveränderungen ausgelöst wurden, ja dieser Klimastress viel ursächlicher
für technologische Weiterentwicklungen war als Migration und
kulturell-technische Vermischung, die ja nur das global bereits Verfügbare
fusioniert, nicht aber zu neuen Ufern aufbricht.

Trotz Sesshaftigkeit seit der Neolithischen Revolution spielen derartige
Migrationsbewegungen noch bis in die Neuzeit eine wichtige Rolle.  Wir haben
uns in letzter Zeit -- durch die recht ungenauen Angaben in (Adgi 2011)
angeregt -- etwas genauer für die Migrationsbewegungen in Europa im Zeitraum
1000 v.Chr. bis 400 n.Chr. interessiert und dabei für die von Gumilev in
diesem Territorium beschriebenen Entwicklungen, die in (Rubin 2010:Karte 17)
referenziert werden, aber ja weitgehend spekulativen Charakter haben, wenig
Bestätigung gefunden.

Ich komme zu meiner Ausgangsfrage über das Verhältnis technischer und
sozio-kultureller Systeme zurück. Meine Studentengruppe löst gerade die
Aufgabe „Schiffsmast“\footnote{Siehe
  \url{https://triztrainer.ru/tasks/section-1/ship-mast/}.} aus dem Minsker
TRIZ-Trainer.  Betrachtet man das Boot als Teil eines Wassertransportsystems,
dann ist die Lösung einfach: Gib in dein Navi ein, dass die Brückendurchfahrt
gesperrt ist, und lasse dieses einen anderfen Weg suchen. Im Kontext eines
autonom fahrenden Boots (der Mensch ist -- in Ihrem Verständnis -- vollkommen
aus dem System „Boot“ verdrängt) gibt es wohl auch keine andere Lösung
(\textbf{These:} Wo kein Mensch, da auch kein TRIZ).

Natürlich ist das keine „ordentliche“ Lösung, man soll das System „Boot“
analysieren.  Für dessen Hauptfunktion (Wassertransport) ist der Mast komplett
entbehrlich, dessen beide Funktionen (Träger von Antenne und Oberlicht)
gehören zur Komponente „Steuerung“ des Systems „Boot“. Also muss man diese
Komponente als Teilsystem weiter analysieren -- wozu Antenne und wozu
Oberlicht?

Oberlicht: Im Ober-Ober-System „Wassertransport“ gibt es die Vorschrift
(technisch? sozio-kulturell?), einen hohen Mast zu beleuchten.  Diese vom
Menschen (in einem komplizierten sozio-kulturellen Prozess der
Institutionalisierung) vereinbarte Vorschrift entfaltet also ihre konkrete
Wirkung als technische Vorgabe (so jedenfalls ist das nach der von Ihnen
angegebenen Vorgehensweise in die Anforderungen an das System „Radio“ zu
importieren, dessen Komponente die Antenne ist, das wiederum ein Teil des
Systems „Steuerung“ ist). Dieses Problem kann also nach Studium der Regeln aus
dem Ober-Ober-System einfach gelöst werden -- kein Mast, auch kein Oberlicht
erforderlich.

Antenne: Gehört offensichtlich zur Komponente Funkverbindung als Teil des
Systems „Steuerung“. Auch wenn der Mensch in einem komplett autonom fahrenden
Boot eliminiert wäre -- auch jenes autonom fahrende Boot benötigte die von
Menschen betreute und betriebene Navigationsinfrastruktur, auf die der
Bootsführer im gegebenen Kontext über die Funkverbindung zugreift.

Zusammenfassend: Ihre Argumentation greift nach meinem Verständnis zu kurz. 

\section{Literatur}
\raggedright
\begin{itemize}
\item Murad Adji (2011). \foreignlanguage{russian}{Азиатская Европа}. (Das
  asiatische Europa).  Astrel, Moskau. ISBN 978-5-271-36400-6.
\item J.M. Anderies, M.A. Janssen, E. Ostrom (2004).  Framework to
  Analyze the Robustness of Social-ecological Systems from an Institutional
  Perspective.\\ In: Ecology and Society 9 (1), 18. 
\item Klaus Fuchs-Kittowski (2001). Wissens-Ko-ProduktionVerarbeitung,
  Verteilung und Entstehung von Informationen in kreativ-lernenden
  Organisationen.\\ In: Fuchs-Kittowski u.a. (Hrsg.). Organisationsinformatik
  und Digitale Bibliothek in der Wissenschaft. Wissenschaftsforschung,
  Jahrbuch 2000. Gesellschaft für Wissenschaftsforschung, Berlin.
  \url{http://www.wissenschaftsforschung.de/JB00_9-88.pdf}
\item Jörg Goldberg, André Leisewitz (2016). Umbruch der globalen
  Konzernstrukturen.\\ Z 108, S. 8--19.
\item Hans-Gert Gräbe (2020). Reader zum 16. Interdisziplinären Gespräch
  \emph{Das Konzept Resilienz als emergente Eigenschaft in offenen Systemen}
  am 7.2.2020 in Leipzig. \url{http://mint-leipzig.de/2020-02-07/Reader.pdf}.
\item Lev N. Gumilev (1994). \foreignlanguage{russian}{Конец и вновь начало}.
  (Ende und wieder Anfang).  Aufsätze, zusammengestellt von
  N.V.Gumilev. Tanais Verlag, Moskau.
\item Lev N. Gumilev (2017). Von der Rus zu Russland. Monsenstein und
  Vannerdat, Münster.
\item Patrick Hamilton (2008). Wege aus der Softwarekrise:
  Verbesserungen bei der Softwareentwicklung. ISBN 978-3-540-72869-6. 
\item Friedhart Klix, Karl Lanius (1999). Wege und Irrwege der
  Menschenartigen.  Kohlhammer, Stuttgart.
\item Karl Koltze, Valeri Souchkov (2017). Systematische Innovation. Hanser
  Verlag, München. Zweite Auf"|lage. ISBN 978-3-446-45127-8.
\item Alexander Lyubomirskiy, Simon Litvin, Sergey Ikovenko, Christian
  M. Thurnes, Robert Adunka (2018). Trends of Engineering System
  Evolution. Eigenverlag, Sulzbach-Rosenberg.  ISBN 978-3-00-059846-3.
\item Karl Marx (MEW 23). Das Kapital, Band 1. Dietz Verlag, Berlin.
\item Karl Marx (MEW 42). Grundrisse der Kritik der politischen Ökonomie.
  Dietz Verlag, Berlin.
\item Elinor Ostrom (2007). A diagnostic approach for going beyond panaceas.
  Proceedings of the national Academy of sciences, 104(39), 15181--15187.
\item Michail S. Rubin (2007). \foreignlanguage{russian}{О выборе задач в
  социально-технических системах}. (Über die Wahl von Aufgaben in
  sozial-technischen Systemen). In: \foreignlanguage{russian}{ТРИЗ Анализ.
    Методы исследования проблемных ситуаций и выявления инновационных
    задач}. (TRIZ-Analyse. Methoden zur Untersuchung von Problemsituationen
  und zur Identifizierung innovativer Aufgaben). Hrsg. von S.S. Litvin,
  V.M. Petrov, M.S. Rubin. \foreignlanguage{russian}{Библиотека Саммита
    Разработчиков ТРИЗ}, Moskau. S. 35--46.
  \url{https://www.trizland.ru/trizba/pdf-books/TRIZ-summit2007.pdf}.
\item Michail S. Rubin (2010). \foreignlanguage{russian}{Филогенез
  социокультурных систем. Секреты развития цивилизаций}.  (Phylogenese
  soziokultureller Systeme. Geheimnisse der Zivilisationsentwicklung).
  \url{http://www.temm.ru/en/section.php?docId=4472}.
\item Clemens Szyperski (2002). Component Software: Beyond Object-Oriented
  Programming. ISBN: 978-0-321-75302-1.
\item V.I. Vernadsky (1997, Original 1936--38). Scientific Thought as a
  Planetary Phenomenon. \url{https://wumm-project.github.io/Texts.html}
\end{itemize}


\end{document}
