\documentclass[11pt,a4paper]{article}
\usepackage{a4wide,url}
\usepackage[russian]{babel}
\usepackage[utf8]{inputenc}

\parindent0cm
\parskip3pt

\title{Об эволюции технических систем} 
\author{Hans-Gert Gräbe, Leipzig}
\date{Версия от 9 декабря 2019 г.}
\begin{document}
\maketitle

Причина этого маленького наброска на тему «Законы и тенденции развития»
Технические системы в контексте ТРИЗ в разных вариантах была версия этих
законов и тенденций в Объявление о Кубке ТРИЗ 2019/20 \footnote{\raggedright
  Задачи ТРИЗ-Кубка
  \url{https://triz-summit.ru/contest/cup-tds-2019-2020/contest-2019-2020/} а
  также версии \url{https://wumm-project.github.io/Upload/lte.pdf}. }.

В этом тендере также «закон репрессий человека» стал (первая версия), которая
в более поздней версии стала «тенденцией вытеснение человека из технических
систем» стал.

Такой тенденции нет ни в списке Альтшуллера из восьми таких
Законы\footnote{Смотрите про \url{https://en.wikipedia.org/wiki/TRIZ}} до сих
пор в списке пяти законов и десяти тенденций в (Кольце /
Сучков) \footnote{Карл Кольце, Валерий Сушков. систематическая Инновации. ISBN
  978-3-446-45127-8.}.

Это также противоречит основным положениям немецкого кибернетического Школа,
что я написал в записке от 8.11.2019 для авторов задач Проведены кубки ТРИЗ.
\begin{quote}
  «Закон замещения человека» отсутствует в списке в (Кольце / Сучков) и я
  совершенно не согласен с этим закон о разработке технических систем. По
  крайней мере, в Немецкая литература существует с 80-х годов Обсуждение на
  эту тему. Например, Клаус указывает Фукс-Киттовский в кратком изложении его
  Работа \footnote{\url{http://www.informatik.uni-leipzig.de/~graebe/Texte/Fuchs-02.pdf}}
  \begin{quote}
    «Наш ответ на этот вопрос всегда был: человек единственный творческий
    Производительной силой он должен быть и оставаться субъектом
    развития. следовательно это концепция полной автоматизации, согласно
    которой человек постепенно быть исключенным из процесса, пропущено!
  \end{quote}
  Замена человека как закон технического развития коренится в очень странное
  понимание термина \emph{техника}, который является Очевидное забвение -- нет
  \emph{технических систем} но только\emph {техносоциальные системы}.
\end{quote}

Михаил Рубин разъяснил свою позицию в личном кабинете от 10.11.2019 следующим
образом:
\begin{quote}
  Для этого требуется отдельная дискуссия. Мы ссылаемся на работу Любомирского
  и Литвина, в которой говорится о вытеснении человека из технической системы.
  Мы согласны с тем, что это явление не закон, а тенденция, которая происходит
  в рамках другого закона: повышения уровня автономности систем. Обновленная
  система законов и тенденций добавлена в файл, который приложен к этому
  письму.  Вы абсолютно правы в том, что технические системы не являются
  самостоятельными в своем развитии и более общими являются
  социально-технические системы. Законы развития социально-технических систем
  отличаются от законов развития технических систем. Для чисто технических
  систем действительно можно наблюдать тенденцию постепенного исключения
  участия человека. Весто весельной лодки появляется лодка с мотором. Вся
  промышленная революция XVII века была построена на вытеснении человека
  двигателями и машинами. Следующая технологическая революция также была
  связана с вытеснением человека из области управления за счет автоматизации и
  компьютеров. Это совсем не означает, что из техники, как
  социально-технической системы вытесняется человек. Наоборот, человек
  остается главным источником требований для технических систем. Но выполнение
  этих требований все в большей степени происходит без участия человека. Эта
  тенденция характерна и для кинематографа, как технической системы. Понятно,
  что ни из процесса создания кинопроизведений, как произведений искусства, ни
  из процесса потребления продуктов кинематографа человек не вытесняется, он
  остается центром всех этих процессов. 
\end{quote}
В этом контексте у меня тоже возникает ряд вопросов в первом обсуждении Не
фейсбук\footnote{\url{https://www.facebook.com/groups/111602085556371}} может
быть очищено.
\begin{enumerate}
\item Что такое \emph{техническая система} в отличие от
  \emph{социально-техническая система}?
\item Какой подход \emph{эволюция технических систем} понять?  Есть ли
  эволюция отдельных технических систем или они могут быть Эволюция только в
  совокупности технических систем или только в неподвижных значимые социальные
  структуры осмысленно обсуждаются?
\item В каких отношениях \emph{человек} относится к отдельным техническим
  Системы и к совокупности его технических творений? В котором Сфера действия
  этого вопроса находится между \emph{человек как общий предмет} (имеющиеся
  знания процесса), отдельные люди как действующие \emph{актеры в отношениях
    среднего назначения} (частный Процедуры) и кооперативные акторы как
  \emph{оператор индивидуальные технические системы} (институционализированные
  Процедуры) дифференцировать?
\end{enumerate}

Такие вопросы возникают, в частности, при изучении социально-экологических
Системы, в которых влияние технических систем очевидно встроены Смотрите о
подходах Элинор Остром\footnote{Anderies, John M., Marco A. Janssen, Elinor
  Ostrom (2004).  Framework to Analyze the Robustness of Social-ecological
  Systems from an Institutional Perspective. In: Ecology and Society 9 (1),
  18. -- Ostrom, Elinor (2007). A diagnostic approach for going beyond
  panaceas.  Proceedings of the national Academy of sciences, 104(39),
  15181--15187.}, в нашем Лейпциг
семинар\footnote{\url{https://github.com/wumm-project/Leipzig-Seminar}.}
просто чтобы обсудить.

\section{Что такое технические системы?}

Подавляющее большинство искусственных технических систем Уникальный. Отрасль,
которая занимается производством таких уникальных предметов называется
\emph{строительство промышленного предприятия}. Также большинство информеров с
созданием таких уникальных предметов, потому что ИТ-системы, такие Системы
управления также уникальны. То же самое относится и к офисам, Власть и
общественные институты. Таков пример Лейпцига Администрация города в настоящее
время занята своими административными процессами «Оцифровка», который
возглавлял Департамент общего управления и совместно с муниципальным
поставщиком ИТ-услуг Lecos.

Конечно, велосипед не постоянно переизобретен - Компонентные технологии
составляют основу каждого инженерно-технического Работа, а также информатика
длились более 25 лет Кризис программного обеспечения\footnote{Patrick Hamilton
  (2008). Wege aus der Softwarekrise: Verbesserungen bei der
  Softwareentwicklung. ISBN 978-3-540-72869-6.} найдены методологии разработки
на основе компонентов\footnote{Clemens Szyperski (2002). Component Software:
  Beyond Object-Oriented Programming. ISBN: 978-0-321-75302-1.}.  В этом
контексте стали Тем не менее, профессиональные профили компьютерных ученых
дифференцированы в Разработчики компонентов («дизайн для компонентов») и
установщики компонентов («Дизайн из компонента»). Первые разрабатывают
компоненты для большего Рынок, второй продолжают разрабатывать большие
уникальные предметы («системы» также в технической терминологии).

Мы находимся в области стандартной терминологии ТРИЗ \emph{системы систем} -
техническая система состоит из компонентов, которые, в свою очередь, являются
техническими системами, чьи функции \emph{работают} (как в функциональным, а
также оперативным) для рассматриваемого в настоящее время Системный уровень
предполагается. Понятие технической системы имеет таким образом, ясная
эпистемическая функция «сокращения к основам».  Эйнштейну приписывают
высказывание «сделай это как можно проще, но не проще ". \emph{закон полноты
  системы} приносит точно Эта идея выражается не как \emph{закон}, а как
\emph{модель} директивы.

Термин \emph{техническая система} находится в одном в четыре раза перегружены
контекстом планирования-реального мира
\begin{itemize}
\item [1.] как уникальный мир,
\item [2.] как описание этого уникального реального мира
\end{itemize}
а также для компонентов, производимых в больших количествах
\begin{itemize}
\item [3.] как описание дизайна шаблона системы
\item [4.] в качестве описания доставочной и эксплуатационной структур Сделано
  из этого шаблона в реальном мире уникальным.
\end{itemize}

В частности, последний момент, отношения между компонентом как Концепция и
реальный мир, созданный компонентами экземпляров, является сложным, потому что
производственные структуры производства и использования этих Экземпляры
компонентов обычно распадаются, экземпляры компонентов после того, как
производство отправлено и на их месте работы для бетона Использование должно
быть подготовлено и установлено. В теории \emph{Программное обеспечение из
  компонентов} будет состоять из трех этапов \emph{deploy, install,
  configure}. 
\end{document}
