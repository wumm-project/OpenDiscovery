\documentclass[a4paper,11pt]{article}
\usepackage{od}
\usepackage[main=russian,english]{babel}
\setlist{noitemsep}

\author{Lebedev Yuri Valentinovich}
\title{Классификация потоков в технических системах
  (Classification of flows in technical systems)}

\date{April 18, 2011}

\newcommand{\frameit}[1]{\par\fbox{
    \begin{minipage}{\textwidth}#1\end{minipage}}\par
}

\setlist{noitemsep}

\begin{document}
\maketitle

\begin{quote}
  Source: \url{https://www.metodolog.ru/node/967}
  
  Contains a large number of comments (in Russian).

  The text itself has an inconclusive numbering of the sections, possibly
  parts of the text are missing in the source. In the table of contents, an
  automatic numbering is given first and the numbering from the source is
  given second.
\end{quote}

\tableofcontents

\section{RESUME}

В данной работе предпринята попытка дать классификацию типов потоком,
существующих в технических системах. Приведены основные субтренды оптимизации
различных типов потоков. На мой взгляд, детальная классификация потоков по
типам позволяет проводить более полный и адекватный анализ систем.

Для всех вводимых типов потоков приведен сквозной пример: потоки в двигателе
внутреннего сгорания. Не будучи специалистом в ДВС, автор не берется давать
исчерпывающую характеристику этих потоком. Поэтому все примеры следует
рассматривать как чисто иллюстративные.

В основном я старался применять терминологию, принятую в школе GEN3. В
необходимых случаях вводил новые определения.

\subsection{1.1.1. Введение (проблемы существующей формулировки)}

Литвин и Любомирский формулируют 4 основных тренда данного закона (паразитные
потоки как частный случай вредных описаны почти слово в слово с вредными):
\begin{itemize}
\item Повышение эффективности использования полезных потоков
\item Снижение повреждающей способности вредных потоков
\item Повышение проводимости полезных потоков
\item Снижение проводимости вредных потоков.
\end{itemize}
Первые два тренда очевидны до тривиальности и тривиальны до полной
неинструментальности. Зато и спорить с ними совершенно невозможно. Правда,
очевидные вещи бывает очень полезно сформулировать в явном виде в формате
аксиом, что и было сделано.

А вот с двумя другими трендами есть серьезные проблемы.
\begin{itemize}
\item Повышение проводимости полезных потоков
\item Снижение проводимости вредных потоков.
\end{itemize}
Прежде всего, "проводимость потока" звучит примерно как "сопротивление
электричества" или "проводимость воды в трубе". Речь все-таки должна идти о
проводимости каналов (трактов) потока ("сопротивление проводника
электрическому току"). Должен сказать, что речь идет не о простом согласовании
падежей в русском тексте, а о выделении существенно различных категорий, ранее
смешанных в едином термине.

\section{1. Краткая История развития закона}

Предшественником закона является сформулированный Генрихом Сауловичем
Альтшуллером закон минимальной энергетической проводимости систем \cite{1}. В
ходе развития системы ЗРТС этот закон оставался долгое время (с 1975 по 2002
г, если судить по датам публикаций) практически неизменным. В частности, в
книге Генриха Альтшуллера, Аллы Зусман, Бориса Злотина и др. \cite{2} закон
упомянут мельком, как часть закона повышения согласованности в системах; в
работе Юрия Саламатова \cite{3} закон воспроизводится почти дословно.

В книге Владимира Петрова \cite{4} закон рассмотрен как увеличение удельной
энергонасыщенности системи является субтрендом закона перехода системы на
микроуровень. Но здесь он уже рассматривается не как требование минимально
необходимого уровеня, а именно как линия развития систем.

В работе Саймона Литвина и Алекса Любомирского \cite{5} закон полностью
переработан и рассматривается как "закон повышения эффективности использования
потоков вещества, энергии и информации".

Нетрудно увидеть, что у Петрова и Литвина с Любомирским это по существу
совершенно новый закон, не столько развивающий предшественника, сколько
находящийся рядом:

Во-первых, новый закон трактует не о возможности существования системы, а о
путях усовершенствования уже работоспособной системы. Во-вторых, у Литвина и
Любомирского существенно расширен круг рассматриваемых потоков с энергии до
всех видов существующих в системе потоков.

Описание закона в \cite{5} является не столько трендом (линией развития),
сколько перечнем механизмов (по существу: рекомендаций по улучшению потоков в
системе). Список приведенных механизмов весьма обширен и составляет 42 пункта.
Структурирование дано по типам потоков и разделению механизмов на "изменение
проводимости потоков" и "изменение эффективности потоков".

\section{2. Определения} 

Серьезным недостатком большинства теоретических работ в ТРИЗ является, на мой
взгляд, недостаточное внимание к терминам и определениям.

Для решения поставленной задачи сначала попробуем уточнить некоторые ключевые
термины.

В данной работе
\begin{itemize}
\item Будем называть потоком такое перемещение материальных объектов, энергии
  или информации в системе, при котором отдельные части потока перемещаются по
  одному и тому же закону одни за другими (частично поток может перемещаться в
  надсистеме, но ключевым является его наличие и перемещение в рассматриваемой
  системе),
\item Будем называть источником потока компонент системы, формирующий поток и
  задающий его первоначальные параметры,
\item Будем называть каналом потока компонент системы, непосредственно внутри
  которого поток перемещается по системе (при этом канал может распределенным
  в пространстве (не иметь четких однозначно заданных границ) или
  отсутствовать вовсе),
\item Будем называть изделием тот компонент системы, который преобразуется под
  воздействием данного полезного потока, или тот, который непосредственно
  повреждается рассматриваемым вредным потоком.
\end{itemize}

Примечания:
\begin{itemize}
\item[A.] Указанные формулировки не являются абсолютно общими для всех
  мыслимых случаев, но достаточны для прикладных целей (анализ ТС и выработка
  предложений по их совершенствованию).
\item[B.] Отнесение тех или иных элементов системы к одному из определенных
  компонентов не является безусловным и определяется особенностями решаемой
  задачи.
\item[C.] В значительной части систем поток является объектом преобразования,
  а не субъектом. Соответственно, "изделие" является субъектом и его
  правильнее было бы назвать "преобразователь" потока. Однако, для нужд
  практического потокового анализа существенной разницы нет, поэтому эту
  специфическую особенность из дальнейшего рассмотрения опускаем.
\end{itemize}
При такой формулировке оказываются разделены четыре функционально существенно
разных по отношению к рассматриваемому потоку компонента системы:
\begin{itemize}
\item сам поток, как субъект преобразования в рабочем органе
\item источник потока,
\item канал для удержания/ограничения/направления потока,
\item "изделие" {--} компонент, на который поток непосредственно воздействует,
  изменяя хотя бы один из его параметров.
\end{itemize}
В терминах "объект -- функция -- субъект" канал преобразует поток, поток
преобразует изделие. Усложнение модели (введение новых дополнительных
элементов) кажущееся. \textbf{В функциональной модели наличие канала по
  умолчанию подразумевает как наличие потока, так и конца потока (изделия). В
  потоковой модели наличие потока также умолчанию подразумевает наличие
  канала, по которому тот движется, и изделия.}

На мой взгляд, явная запись компонента, ранее подразумевавшегося по умолчанию
\begin{itemize}
\item упрощает анализ модели,
\item выявляет прямую не опосредованную связь между потоковой и функциональной
  моделями, а также потоковой модели с ЗРТС. 
\end{itemize}
В частности:
\begin{itemize}
\item В потоковой модели появляется возможность оперировать другими
  компонентами системы (источником, каналом, изделием).
\item В функциональной модели кроме компонентов появляются все рассматриваемые
  потоки.
\item В явном виде появляется возможность согласования четырех компонентов
  системы (источника, потока, канала и изделия), конкретизируются компоненты,
  за счет которых может расти управляемость потоком (источник и канал) и т.д.
\end{itemize}
В предложенных формулировках достаточно четко разделяются и две пары трендов,
предложенных Литвиным и Любомирским:
\frameit{\begin{itemize}
\item Повышение эффективности использования полезных потоков
  \textbf{изделием}. 
\item Снижение повреждающей \textbf{изделие} способности вредных потоков.
\end{itemize}}
НО:
\frameit{\begin{itemize}
\item Повышение проводимости \textbf{каналов} полезных потоков.
\item Снижение проводимости \textbf{каналов} вредных потоков.
\end{itemize}}
В такой формулировке тренды становятся гораздо более понятными. При этом:
\begin{itemize}
\item становится очевидной неполнота существующей формулировки закона,
\item вторая пара трендов по-прежнему вызывает серьезные возражения.
\end{itemize} 

Простые примеры:
\begin{itemize}
\item Полезный поток топлива в ДВС. При повышении проводимости канала этого
  потока дополнительное поступление топлива в камеры сгорания приведет к
  неполному сгоранию, что, в свою очередь, приведет к целому ряду серьезных
  проблем.
\item Полезный поток горячей воды или пара в рубашке теплообменника. При
  повышении проводимости этого канала тепло будет удаляться из системы, хотя
  нам нужно наоборот.
\item Вредный поток джоулева тепла при прохождении электрического тока по
  электронной схеме при снижении проводимости канала этого потока будет
  разогревать плату, хотя нужно опять-таки наоборот.
\item Полезный поток электрического тока в лампе накаливания при повышении
  проводимости проводника приведет к изменению ее номинала, а выше
  определенного предела попросту -- к перегоранию лампы.
\item Вредный поток канализационных стоков при понижении проводимости фановой
  трубы приведет сами знаете к чему.
\item Полезный поток полуфабриката к какому-то исполнительному механизму
  (потребителю потока) при повышении проводимости выше определенного предела
  приведет к затовариванию потребителя и/или необходимости вводить
  буфер-накопитель.
\item И т.д.
\end{itemize}
Разумеется, можно оговориться разными способами. Типовой способ при ЗРТС
анализе выглядит так: "в данном случае действует другой субтренд...". Однако,
\textbf{такого рода оговорки (и даже сама по себе их необходимость) резко
  снижают инструментальность метода} (все-таки, когда я провожу анализ, я
хотел бы в результате узнать, куда мне двигать систему; подобные же оговорки
означают, что я узнаю это уже ПОСЛЕ решения задачи (нахождения ответа другими
способами) или что я должен перебрать все субтренды, не зная заранее, который
из них мне пригодится).

Разумеется, увеличение проводимости канала потока часто оказывается очень даже
полезным. Приемы такого увеличения подробно рассмотрены в действующей версии
закона и остаются верными и безусловно полезными. То же самое касается случаев
уменьшения проводимости канала.

Задача данной работы -- попытаться определить, когда необходимо применять те
или иные приемы.

\subsection{1.2 Классификация потоков}

Чрезмерно упрощенное деление потоков только на полезные и вредные приводит к
чрезмерно общим выводам тапа "улучшить вообще", либо к неструктурированному
списку, как получилось в существующей версии. Поэтому для начала необходимо
дать более детальную классификацию существующих потоков. Разумеется, слишком
детальная классификация не менее вредна, поэтому необходимо выбрать какой-то
оптимальный уровень детализации.

Разделение полезных потоков по признаку функциональности
\begin{itemize}
\item Главный полезный поток -- поток, необходимый для выполнения рабочим
  органом системы главной полезной функции этой системы.
\item Вспомогательный полезный поток -- поток, необходимый для адекватного
  преобразования главного (или, то же самое, для выполнения вспомогательных
  полезных функций).
\end{itemize}
Разделение полезных потоков по источнику
\begin{itemize}
\item Первичный (внешний) полезный поток -- поступающий в систему извне.
\item Вторичный полезный поток -- поток, формируемый внутри системы.
\end{itemize}
Разделение вредных потоков по источнику
\begin{itemize}
\item Первичный (внешний) вредный поток -- поступающий в систему извне.
\item Вторичный вредный поток -- поток, формируемый внутри системы и
  являющийся прямым и неизбежным следствием выполнения какой-либо полезной
  функции.
\item Паразитный поток -- поток, возникший в результате преобразования
  полезного потока, но не предопределенный принципом действия (т.е., не
  неизбежный хотя бы в принципе).
\end{itemize}
Разделение потоков по признаку "конь-всадник":
\begin{itemize}
\item Поток-функционал
\item Поток-носитель
\end{itemize}
Разделение полезных потоков на замкнутые и открытые.
\begin{itemize}
\item Открытый поток -- поток, выходящий из источника и заканчивающийся на
  изделии или в надсистеме.
\item Замкнутый поток -- поток, возвращающийся к источнику (на практике --
  будучи уже преобразованным; поэтому замкнутый поток можно рассматривать и
  как два разных потока -- выбор формата рассмотрения зависит от конкретной
  задачи проекта).
\end{itemize}
Для примера предложенной классификации удобно рассмотреть двигатель
внутреннего сгорания (там такая классификация выглядит очевидной и вполне
наглядной).
\begin{itemize}
\item Разделение "источник/канал/поток/изделие" вполне очевидно:

  Потоками являются потоки бензина, воздуха, выхлопных газов, охлаждающего
  воздуха и т.д. Каналами для них являются трубка подачи бензина, трубы подачи
  воздуха, охлаждающего воздуха, выхлопная труба и пр. Источником выхлопных
  газов является камера сгорания. Изделием охлаждающего воздуха является
  теплообменник и т.д.

\item Главными полезными потоками являются (по принципу действия ДВС)
  \begin{itemize}
  \item поток бензина,
  \item поток воздуха в двигатель,
  \item поток охлаждающей жидкости.
  \end{itemize}
\item Вспомогательными полезными потоками являются: поток всевозможных
  присадок в бензине, поток масла в двигателе.
\item Внешними полезными потоками являются два из указанных главных (бензин и
  окислитель), а также поток присадок в бензине.
\item Внутренними полезными потоками являются: поток масла, поток сгоревших
  газов в цилиндры и из цилиндров, часть тепловых потоков, охладитель.
\item Внешними вредными потоками являются различного рода загрязнения в
  бензине и воздухе.
\item Вторичные вредные потоки:
  \begin{itemize}
  \item Поток углекислого газа и паров воды, полученных при сгорании бензина.
  \end{itemize}
\item Паразитные потоки:
  \begin{itemize}
  \item Поток окислов азота и моно-окиси углерода
  \item Поток продуктов распада присадок к бензину
  \item Поток загрязнений в масле
  \end{itemize}
\item Разделение "конь-всадник" не менее очевидно:
  \begin{itemize}
  \item Воздух (или жидкость) для охлаждения - типичный носитель ("конь")
  \item Поток тепла (холода) является типичным функционалом ("всадником"),
    что, собственно, и позволяет легко менять носитель.
  \item Поток присадок к бензину является функционалом по отношению к потоку
    бензина. Собственно бензин является при этом носителем потока присадок
    (одновременно являясь при этом важным функциональным потоком!)
  \end{itemize}
\item Замкнутые потоки: поток моторного масла, поток охлаждающей жидкости,
  поток загрязнений в масле.

  (разумеется, как всегда, необходимо рассматривать систему не вообще, а
  применительно к цели проекта. Поэтому, в зависимости от решаемой задачи,
  возможны другие варианты:

  главный полезный поток -- кислород воздуха, а азот и другие газы в воздухе --
  внешний вредный поток. Поток бензина и присадок в нем можно считать единым
  потоком, а поток охлаждающих агентов -- вспомогательным полезным, и т.д.)
\end{itemize}
Разумеется, такая классификация гораздо сложнее, чем в существующей версии
закона.  Поэтому уровень детализации нужно выбирать минимально достаточным для
целей каждого конкретного проекта.

В то же время, подобная классификация все равно возникает при любом
мало-мальски детальном потоковом анализе. Без чего он превращается из анализа
ситуации в рисование более или менее красивых картинок, что имеет отношение к
презентации результатов, а не их получению (хотя разумеется, грамотная и
аккуратная презентация результатов ТРИЗ анализа часто имеет важное
самостоятельное значение).

Зато совершенно очевидно, что разные потоки требуют разных подходов к
усовершенствованию (оптимизации, повышению эффективности etc). Но раз так,
безусловно полезно и разделение потоков по разным полочкам.

Приведенный пример потоков в ДВС хорош явным разделением. Разумеется, возможно
огромное количество сочетаний и подробностей. Например, поток джоулева тепла в
лампе накаливания -- является ли он вторичным вредным потоком или
вспомогательным полезным или даже главным полезным? Разумеется, ответ
полностью зависит от задачи проекта.  В то же время поток того же джоулева
тепла в светодиоде является заведомо вредным; пожалуй -- паразитным, хотя может
трактоваться и как вторичный.

\subsection{1.3 Субтренды для отдельных типов потоков и их каналов}

\section{3. Возникновение вредных и паразитных потоков в системе}

\subsection{3.1 Снижение проводимости каналов внешних вредных потоков \emph{на
    входе в систему} (вплоть до формирования непроницаемого барьера этому
  потоку)}

Внешние потоки являются потоками надсистемными, поэтому нашему регулированию
почти не подлежат. Источником внешних вредных потоков с точки зрения
рассматриваемой системы является вход в систему. Так что мы чаще всего можем
регулировать только точку входа потока в систему.
\begin{itemize}
\item Фильтры, устанавливаемые на входе в ДВС, являются типичным примером
  проявления указанного субтренда.
\item Другим примером являются различные способы ограничения въезда
  большегрузного транспорта в центр городов.
\end{itemize}
Данный подход вообще применяется очень широко, но есть мощный подводный
камень: канал для внешнего вредного потока очень часто является каналом для
какого-то полезного потока. Если в комнате сквозняк, естественное желание --
закрыть форточку. Но мы тут же перекрываем поток свежего воздуха!
\textbf{Поэтому наличие на входе в систему внешнего вредного потока (пусть
  даже и остановленного!) является индикатором возможного наличия
  проблемы}. Все тот же пример с фильтрами: фильтры не только останавливают
потоки загрязнений, но и тормозят полезные потоки. С другой стороны, увидев
фильтры на входах в совершенно незнакомую нам техническую систему, мы можем
сразу же уже при первичном знакомстве с системой, безо всякого анализа,
сделать как минимум три вывода:
\begin{itemize}
\item На входе имеется вредный поток.
\item Остатки этого потока есть внутри системы.
\item Проводимость канала полезного потока снижена.
\end{itemize}
Как видим, уже сама по себе дополнительная классификация потоков и связанных с
потоками компонентов системы может быть вполне инструментальным приемом.

\subsection{3.2. Управление формированием вторичных вредных потоков (снижение
  удельных характеристик потока)}

Внутренний поток возникает по определению как результат работы рабочего органа
(рассматриваемого в данном случае как источник вредного потока) по выполнению
некоей полезной функции.

На первый взгляд кажется, что именно усовершенствование рабочего органа должно
оказаться главной мерой по борьбе с вредными потоками. Но это не так. В силу
данного выше определения, вторичный вредный поток неизбежен при выполнении
полезной функции. Устранить его можно только при переходе на новый принцип
действия.

Обычно можно уменьшить повреждающие параметры этого потока, но в силу закона
неравномерности развития частей системы (единственного, пожалуй, закона,
который не ставится под сомнение ни одной из школ ТРИЗ) ресурсы развития
рабочего органа использованы, как правило, уже на ранних стадиях развития
системы в целом. Поэтому, когда мы начинаем вычищать детали и
усовершенствовать "трансмиссию" {--} рабочий орган уже с трудом поддается
улучшениям.

Например, в ДВС формируются два мощных вторичных вредных потока: тепло,
которое необходимо отвести от цилиндров на третьей ветви цикла Карно, и
углекислый газ. Оба эти потока можно не создавать только при отказе от
принципа действия (например, при переходе на электромотор), но мы-то
анализируем ДВС!

Количество углекислого газа можно несколько уменьшить при переходе на газ и
далее спирт. НО

Во-первых, не ликвидировать, а всего лишь несколько уменьшить; во-вторых, этот
тренд имеет очень серьезные надсистемные ограничения, понемножку реализуется,
но, главное - из совершенно иных соображений.

Поэтому ставить задачу на смену принципа действия ради уменьшения вторичных
вредных потоков \emph{в системе} -- обычно не удается.

\subsection{3.3. Управление формированием паразитных потоков}

Паразитные потоки являются частным случаем вторичных вредных, но имеют ту
особенность, что их появление не определяется принципом действия системы.
Поэтому появляется важный субтренд развития системы: минимизация паразитных
потоков. \textbf{При выявлении паразитных потоков именно их минимизация
  (вплоть до полного устранения) является первым направлением
  усовершенствования.}

Например:
\begin{itemize}
\item Моно-окись углерода, связанная с неполным сгоранием топлива, является
  типичным паразитным потоком. Первым (лучшим) решением должно бы стать
  решение, направленное на полное сгорание топлива (предотвращение паразитного
  потока).
\item Упоминавшееся джоулево тепло в светодиодах не может, конечно, быть
  устранено полностью, но теоретически может стремиться к нулю (в отличие от
  ламп накаливания). Поэтому при любом достигнутом уровне этот поток можно
  пытаться уменьшить. В частности, для этого в светодиодах (как и в любых
  полупроводниковых приборах) стремятся к как можно более низкоомной подложке.
\item Важный вариант паразитного потока -- всевозможные утечки полезного
  потока. Естественно стремление устранить и/или предотвратить эти утечки
  (здесь упомянуто просто для полноты, хотя в принципе решение очевидно: если
  у вас протекает кран, то его просто нужно починить).
\end{itemize}

\subsubsection{3.3.1. Формирование паразитного потока как признак
  противоречия} 

За исключением простейшего (хотя и важного) случая утечек \textbf{наличие в
  системе паразитного потока всегда является признаком противоречия}:
паразитный поток по данному выше определению не является неизбежным следствием
принципа действия.  То есть, какие-то свойства системы с одной стороны должны
быть реализованы (необходимость чего определяются необходимостью выполнения
тех или иных полезных функций), но в то же время, должны быть изменены, чтобы
предотвратить возникновение паразитного потока.

В ДВС типичным паразитным потоком является поток окислов азота в продуктах
сгорания. Противоречие выглядит как "температура сгорания топлива должна быть
высокой, чтобы обеспечить высокий коэффициент полезного действия ДВС, но в то
же время малой, чтобы предотвратить появление окислов азота". ИЛИ: "в составе
окислителя азот должен быть, чтобы можно было использовать самый дешевый и
доступный из всех окислителей воздух, но азота не должно быть, чтобы не
возникали его окислы".

Важно при этом еще и то, что \textbf{мы не просто легко обнаруживаем
  противоречие (одна из важнейших целей анализа в ТРИЗ), но и локализуем его в
  том компоненте системы, которая выступает как источник потока} (т.е., сразу
же автоматически определяем оперативную зону и оперативное время).

В этом примере возникает еще одна очень любопытная коллизия: в это же
оперативное время и в той же оперативной зоне возникает другой паразитный
поток -- моно-окиси углерода. Противоречие здесь выглядит примерно так:
"количество кислорода должно быть избыточным, чтобы предотвратить неполное
сгорание углеводородов, но количество кислорода должно быть стехиометрическим,
чтобы предотвратить окисление азота (а также предотвратить ряд иных проблем)".

Т.е., возникает необходимость решать два разных противоречия, сошедшихся
строго в одно время в одном месте: если кислорода много -- возникают окислы
азота, если кислорода мало -- возникает моно-окись углерода. В ТРИЗ, насколько
мне известно, нет инструментов для одновременного согласованного решения двух
или более противоречий. Было бы интересно и полезно такие инструменты
разработать!

\section{Канал потока как основной инструмент управления потоком}

Выше уже упоминалось о том, что оптимизация потоков в основном происходит за
счет усовершенствования канала и в гораздо меньшей степени за счет
усовершенствования источника и изделия.

Это происходит в силу закона неравномерности развития частей системы: как
изделие, так и источник являются рабочим органом (не обязательно главным).
Поэтому их развитие происходит ранее развития канала (трансмиссии в
формулировке Альтшуллера, когда он говорил о потоках энергии в механических и
электромеханических системах). Поэтому, когда в реальном рабочем проекте
встает задача оптимизации потоков, оптимизация рабочих органов уже, как
правило, проведена и они находятся на том уровне развития, который доступен
для данной системы здесь и сейчас.  Задача на усовершенствование источников
потоков и их изделий (приемников потоков) ставиться может, но почти всегда в
качестве задач на перспективу.

В ДВС имеется вредный поток загрязнений в масле. Как источником, так и
изделием этого потока являются узлы трения (случай замкнутого потока). Помимо
установки фильтров (управление каналом) можно поставить задачу на снижение
трения. Например, магнитные подшипники полностью или почти полностью
ликвидируют трение и, тем самым резко снижают загрязнения. Но нужно понимать,
что такое преобразование источника если и делается, то из других соображений
(снижение трения в движущихся узлах само по себе задача гораздо более важная,
чем снижение загрязнения в масле).

Поэтому:
\begin{itemize}
\item Такая работа проделывается из других соображений и не имеет, как
  правило, отношения к оптимизации потока масла.
\item Поэтому в любой конкретной системе существующий уровень трения задан
  либо техническими возможностями, либо экономическими и др. ограничениями и,
  как правило, не поддается управлению на этом этапе развития системы.
\item Тем не менее, помнить о такой возможности управления вредными потоками
  следует (хотя использовать ее чаще всего не удается).
\end{itemize}

Другой пример в ДВС: полезный поток бензина в двигатель можно регулировать
проводимостью канала потока (управляемый вентиль -- в этом качестве выступает
карбюратор), но можно и источником потока -- бензонасосом. Здесь оказывается
гораздо удобнее регулировать канал. Можно сделать бензонасос, например,
подающий бензин в виде импульсов с изменяемой интенсивностью, но это будет
сложно и дорого. С другой стороны, по отношению к потоку бензина
непосредственно в камеру сгорания карбюратор следует уже рассматривать как
источник потока. Выбор точки зрения полностью определяется целями и
особенностями проекта.

Другой пример. На производственной линии поток брака является типичным
паразитным потоком, который нужно снижать всегда. Для совершенствования
процесса изменяются все три компонента:
\begin{itemize}
\item Оборудование, на котором возникает брак (источник потока).
\item Система отбраковки и удаления брака (канал потока).
\item Оборудование следующей операции (изделие).
\end{itemize}
Разумеется, нужно совершенствовать все три.

В перспективных проектах целесообразно разрабатывать оборудование с меньшим
выходом брака (источник) и с бóльшими допустимыми отклонениями (изделие). В
неТРИЗовской среде это формулируется как требование к разработчикам
оборудования обеспечить больший диапазон входных и меньший диапазон выходных
параметров. Но ставить такую задачу целесообразно для фирмы-производителя
оборудования, то есть в этом случае мы рассматриваем совсем другую систему.
Если же работа ведется для усовершенствования технологического процесса,
использующего указанное оборудование, то параметры оборудования являются
надсистемным фактором, с трудом поддающемся изменениям (производителю хлеба
бессмысленно предлагать решения, связанные с разработкой новой печи).

Поэтому в проектах текущего усовершенствования системы управление отбраковкой
(каналом потока) часто является единственным доступным способом.

\section{5. Управление каналом вредных потоков}

После того, как вредный поток тем или иным способом но все-таки оказался в
системе, разница между первичным, вторичным и паразитным становится чисто
академической. Далее необходимо с ним бороться.

Как уже сказано, изменение источника вредных потоков (управление их
источником) чаще всего затруднено. По той же причине затруднено изменение
параметров изделия. Поэтому \textbf{первым (а часто -- основным) методом
  борьбы с вредными потоками всех видов является управление их каналами}.

Здесь возможны два существенно разных варианта:
\begin{itemize}
\item Формирование искусственного управляемого канала для вредного потока и
  вывод потока в надсистему (канализация вредных потоков).
\item Управление сформировавшимся каналом вредного потока (при этом канал
  всегда ведет к повреждаемому компоненту системы -- изделию).
\end{itemize}

\subsection{5.1.  Канализация вредных потоков -- \emph{повышение} проводимости
  каналов удаления вредных потоков, движущихся внутри системы}

\frameit{\textbf{Первым способом управления вредными потоками является
    формирование канала для удаления потока из системы}.}

Тезис полностью противоречит существующей концепции, но метод является очень
важным, эффективным и широко применяемым на практике.
\begin{itemize}
\item В примере с ДВС тепло от цилиндров на третьей ветви цикла Карно нужно
  отвести как можно быстрее. Точно также, продукты сгорания топлива из
  цилиндров нужно удалить, а вовсе не задержать.
\item В промышленном производстве возвратные отходы несут в себе стоимость
  всех технологических операций, которые прошло сырье. Поэтому как можно более
  ранняя и эффективная отбраковка таких отходов (т.е. -- повышение
  проводимости каналов этих отходов, например, путем сокращения длины канала)
  является экономически очень эффективным решением.
\item Вентиляторы и радиаторы являются мерами по увеличению проводимости
  вредного потока тепла в электронных приборах и вообще во всех электрических
  машинах (по Литвину и Любомирскому нужно было бы ставить теплоизолирующие
  кожухи!).
\item В химической промышленности быстрое удаление неиспользованных продуктов
  реакции резко увеличивает полезный эффект и/или скорость основной реакции (а
  что бывает, когда падает тяга в дымовой трубе -- известно).
\end{itemize}
Так что тезис о том, что вредный поток, гуляющий по системе, всегда следует
тормозить, выглядит категорически неверным.

\subsection{5.2. Утилизация вторичных вредных и паразитных потоков}

Также часто используемый метод (чем-то сходный с 22-м приемом устранения
противоречий "обратить вред в пользу").

Уже очень давно на производствах собирают всевозможные отходы на предмет их
вторичного использования (хорошо известны сбор металлолома и макулатуры, но
этот список может быть продолжен очень далеко).

Суть метода как раз в торможении вредного потока внутри системы, что внешне
соответствует существующей концепции закона. Однако, делается следующий шаг --
поток останавливается полностью, аккумулируется в каком-то сборнике и затем
все-таки выводится из системы по новому специально организованному каналу
\textbf{(то есть, это та же канализация вредного потока, просто поток
  переводится в новый канал при подавлении ранее существовавшего)}.

\subsubsection{5.2.1.  Использование вторичного потока внутри системы}

Частный случай предыдущего. Метод используется довольно редко именно в силу
того, что вторичный вредный поток возникает как отходы, связанные с принципом
действия, т.е., как поток непригодный к использованию.

Довольно часты попытки использования вторичного тепла. Результат бывает не
всегда, но когда удается -- эффект бывает очень неплохим.

Пожалуй, единственный пример из ДВС -- использование сбросового тепла от
двигателя для подогрева кабины автомобиля ( но уже вне ДВС как такового).

Так что следует учитывать, что такое использование всегда направлено на
выполнение какой-либо вспомогательной или дополнительной функции.

Возможный вариант использования данного субтренда: можно поставить задачу на
использование сбросового тепла внутри ДВС. Где в ДВС необходим такой ресурс,
как тепло? Например, можно попытаться подогревать этим теплом бензин и воздух
непосредственно перед подачей в камеру сгорания. Или, например, для подогрева
соседнего цилиндра на обратном ходу. Разумеется, это всего лишь постановка
задачи, которую еще нужно решить (и вполне возможно -- отказаться от ее
решения), но если задачу не ставить, то не будет и решения. Для ее аккуратного
решения нужно провести как минимум функциональный анализ системы.

\subsubsection{5.2.2. Утилизация паразитных потоков}

Должна применяться тогда, когда устранение этих потоков затруднено или
нецелесообразно (например, экономически). Значимых отличий от утилизации
других вредных потоков я здесь не вижу, кроме одного важного частного случая:

\paragraph{5.2.2.1 Использование производственного брака в качестве возвратных 
  отходов.}

Возвратными отходами на производстве называются такие виды брака, которые
могут быть возвращены \textbf{в то же} производство в рамках того же или
следующего производственного цикла (не следует путать с вторичными ресурсами
-- отходами, используемыми в другом месте для других нужд). Как правило, это
брак, не прошедший еще полного производственного цикла, а отсеянный на более
ранних этапах.

Например: При изготовлении различных заготовок из теста (и вообще -- любой
смеси) часть заготовок имеет неправильную форму и снимается из дальнейшего
процесса. Такие заготовки отправляются обратно в смеситель и в составе новой
порции теста формуются заново. Экономический смысл этого вполне понятен.

Но следует иметь ввиду, что эта часть полуфабриката проходит некоторые
операции дважды, т.е., себестоимость конечной продукции будет выше возможной.

В этом случае применяются все три способа (субтренда), описанных выше и именно
в предложенной последовательности:
\begin{itemize}
\item Первым делом нужно принять меры по уменьшению брака (совершенствование
  именно той/тех подсистемы, где возникает брак -- задача выходит из области
  потокового анализа, переходя в область функционального анализа).
\item Далее следует принять меры по утилизации паразитных отходов -- т.е., как
  раз использование брака в качестве возвратных. При всей очевидности этим
  занимаются далеко не всегда, так что здесь очень часто есть хорошие ресурсы.
\item Не всякий брак удается, конечно, использовать в качестве возвратных. Но
  иногда удается использовать его как-то еще. Помимо упоминавшегося сбора
  вторичного сырья решения бывают самые неожиданные.
\end{itemize}
Например, на "Светлане" (СПб) из отбракованных колб вакуумных ламп делали
рюмки и бокалы. Более того, материалы, распыляемые в лампах в качестве
адсорберов, иногда тоже бывали некачественными. А вот декоративность таких
распыленных пленок не страдала и эти рюмки декорировали именно из таких
отбракованных слитков. Самое забавное, что производство этих рюмок сохранилось
и после закрытия основного лампового производства -- но это уже к вопросу
четвертого этапа развития системы.

\subsection{5.3. Управление каналом вредного потока внутри системы}

В тех случаях, когда не удается быстро вывести вредный поток из системы,
необходимо уменьшить его вредное воздействие.

Примечание: необходимо помнить, что в подавляющем большинстве случаев
материального или энергетического потока он все же будет выведен за пределы
системы просто в силу законов сохранения энергии и вещества. Исключение могут
составлять одноразовые системы, накапливающие собственные отходы.  Но хороший
убедительный пример даже нелегко подобрать. Поэтому речь идет о том, что
вредный поток не удается вывести за пределы системы достаточно быстро и
управляемым образом.

В этих условиях, действительно, как правило, нужно затормозить вредный поток.
При этом либо сформируется другой канал, который будет выводить поток из
системы по менее вредному пути, либо произойдет диссипация потока. В этом
случае действует субтренд, описанный у Литвина и Любомирского: "Уменьшение
проводимости канала вредного потока".

Приемов для реализации субтренда существует довольно много, причем, они
подробно описаны в частных областях техники. В основном они сводятся к
следующим:

\subsubsection{5.3.1.  Снижение проводимости отдельных звеньев канала}

Достаточно очевидное решение. Если необходимо снизить интенсивность
какого-либо потока, резонным является снизить проводимость канала. В свою
очередь, снизить проводимость канала можно одним из двух способов:

\paragraph{5.3.1.1.  Уменьшение проводимости существующих звеньев канала.} 

В частности, широко применяются:
\begin{itemize}
\item Увеличение длины отдельных звеньев канала.
\item Уменьшение удельной проводимости отдельных звеньев канала.
\end{itemize}

В ДВС один из вторичных вредных потоков -- поток газов (рабочего тела) через
уплотнения в конструкции корпуса. Этот тот случай, когда канализация вредного
потока, по-видимому, невозможна. Увеличение ширины уплотнения как раз является
способом увеличить длину канала для этого потока. Разного рода конструктивные
решения (рифленые фланцы и т.п.) являются способами уменьшить его удельное
сечение.

Разумеется, задача герметизации стоит отнюдь не только в ДВС. Как раз в ДВС
эта проблема не очень актуальна, здесь приведена только для единства сквозного
примера.

\paragraph{5.3.1.2.  Введение в канал «бутылочных горлышек»}

\frameit{Под бутылочным горлышком будем понимать дополнительное звено канала с
  заведомо более низкой проводимостью.}

Такое определение отличает данный прием от прямого уменьшения проводимости уже
имеющихся звеньев канала. Решение весьма часто употребляется в технике.

Одним из наиболее типичных примеров является фильтр: в тех случаях, когда
вредный поток идет по одному каналу с полезным, фильтр разделяет эти два
потока и тормозит один из них (вредный). Другим типичным примером бутылочного
горлышка является вентиль, который перекрывает (частично или полностью) канал
и тем самым управляемо снижает его проводимость (но уже для всех видов
потоков, идущих этим каналом).

Например, во время работы ДВС формируется такой специфический поток, как поток
загрязнений в масле. Быстро и надежно отфильтровать и сбросить эти загрязнения
технически можно, но достаточно дорого. Поэтому применяется фильтр, который
просто тормозит указанный вредный поток. Вывод потока из системы производится
лишь иногда при замене фильтра.

Другие примеры фильтров и вентилей в различных отраслях техники каждый инженер
легко найдет по вкусу в достаточных количествах и в соответствии со своей
специализацией.

\paragraph{5.3.1.3.  Введение в канал «застойных зон».}

В общем случае, введение застойных зон не является самостоятельным приемом.
Любое снижение интенсивности вредного потока в любых звеньях канала, будь то
снижение проводимости существующего или постановка дополнительного звена,
перед этим звеном зону повышенной концентрации потока, что как раз и является
застойной зоной.

Выделение "застойной зоны" в отдельный прием имеет тот смысл, что появление
таких зон довольно часто позволяет более эффективно бороться с вредным
потоком. Например, появляется возможность сформировать канал канализации или
какой-то обработки вредного вещества (энергии) именно из застойной зоны.
Эффективность такого канала будет тем выше, чем выше концентрация вредного
фактора в застойной зоне.

Например, в ДВС фильтр является не только бутылочным горлышком для различного
рода загрязнений, но и концентратором этих загрязнений. Это позволяет
сравнительно легко удалять загрязнения путем простой смены фильтра. Этот же
факт позволяет поставить задачу на канализацию загрязнений -- например,
введением самоочищающихся фильтров.

Разумеется, и в этом примере речь идет о фильтрах не только в двигателе, но и
в любых других технических системах с потоками.

Другой пример: отстойники в системах водоочистки являются типичными застойными
зонами. За счет повышенной концентрации загрязнений их обработка (в том числе
-- химическая) оказывается гораздо более дешевой и эффективной.

\section{6. Полезные потоки}

На первый взгляд кажется, что действия в отношении полезных потоков должны
быть инверсны (симметричны) действиям в отношении вредных потоков. Однако, это
не так.

Если вредные потоки в любом случае подлежат либо удалению, либо ослаблению, то
полезные потоки лишь изредка подлежат именно усилению. Как правило, имеется
определенный оптимальный набор параметров, превышение которых как минимум
непродуктивно, а часто -- вредно.

Ряд примеров, подтверждающих это, приведен в самом начале и этот ряд можно
легко продолжить для любой области техники. Поэтому \frameit{\textbf{главным
    методом улучшения потоков является \emph{согласование параметров потока} с
    другими параметрами системы}.}

Таким образом, закон оптимизации потоков в части полезных потоков является
субтрендом закона повышения согласованности.

Это не должно нас удивлять: если ЗРТС является не просто перечнем трендов, а
их системой, такие взаимные пересечения неизбежны и объективны. Недостатком
существующих версий ЗРТС как раз является недостаточная фиксированность этих
связей.

\subsection{6.1.  Согласование главных полезных потоков по интенсивности}

Одним из основных параметров любого потока является его интенсивность.

Под интенсивностью потока будем понимать количество вещества, энергии или
информации в единицу времени. Например, электрический ток ($[A]=[Q/t]$),
мощность ($[W]=[J/s]$), расход бензина (литр/час), поток полуфабриката
(кг/сек), автомобильный трафик (машин/час) и т.д.

Интенсивность потока на входе в канал, очевидно, равна мощности источника.
Необходимо заметить что, в отличие от вредного потока, источник полезного
потока поддается изменениям. Хотя бы потому, что очень часто единственная
полезная функция этого компонента системы -- именно формирование данного
полезного потока.

При этом закономерность состоит в том, что
\frameit{\begin{itemize}
\item в процессе развития системы мощность источника главного полезного потока
  все более согласовывается с проводимостью канала потока,
\item в процессе развития системы проводимость канала главного полезного
  потока все более согласовывается со скоростью преобразования потока в
  изделии.
\end{itemize}
В большинстве случаев проводимость канала должна стремиться к максимальной
скорости преобразования потока (производительности) в изделии (рабочем
органе). Мощность источника должна стремиться к максимальной проводимости
канала (она часто согласована с производительностью еще на этапе
проектирования, но этот момент следует проверить).}

Немножко повторюсь.

Существующая версия "Повышение проводимости полезных потоков" или "Повышение
эффективности использования полезных потоков" {--} в переводе на обычный русский
это звучит как необходимость либо увеличить проводимость канала, либо изменить
поток как-нибудь по-другому. При этом не дается никаких рекомендаций на тему,
когда какой из субтрендов применять. В этом виде закон оказывается безусловно
верным, но абсолютно бесполезным ("договорились встретиться либо в пятницу,
либо в какой-нибудь другой день").

Значительную сложность представляет здесь необходимость согласования сразу
трех разных объектов -- источника потока, канала его передачи и рабочего
органа. Но именно поэтому имеет смысл разделение единого тренда на два
частных.

Еще несколько примеров сверх приведенных в самом начале статьи:
\begin{itemize}
\item В ДВС система впрыска топлива является источником потока топлива в
  цилиндры. Совершенно очевидно при этом, что интенсивность потока топлива
  должна быть согласована с потреблением топлива в цилиндрах. При этом
  интенсивность должна измеряться не просто средним во времени расходом, но и
  заданной периодичностью. Мощность системы впрыска должна также быть
  согласована с необходимым расходом. И более того, как пропускная способность
  канала (проводимость канала), так и максимальная производительность системы
  (мощность источника) должны обеспечить нужный расход при режиме
  максимального потребления (максимальной мощности, развиваемой ДВС). Но и
  увеличивать эти параметры сверх необходимых не следует.
\item Важным частным случаем ТС со сложными системами потоков являются
  технологические производственные линии. Главным полезным потоком в
  технологии является поток полуфабриката (сырья в начале процесса,
  полуфабриката -- в середине, готового продукта -- в конце). Увеличивать
  проводимость канала, т.е. пропускную способность транспортных подсистем
  (будь то транспортер, трубопровод или что-то иное) сверх производительности
  оборудования как минимум нерентабельно по экономическим соображениям. Часто
  это и вовсе требует установки дополнительных буферных устройств.
\end{itemize}
Очень часто (на оборудовании непрерывного действия) транспортный канал для
потока объединен с преобразователем (потребителем) потока. В таких случаях
избыточная производительность оборудования по отношению к предыдущему или
последующему приводит к серьезной рассогласованности всего процесса.

\subsubsection{6.1.1.  Буферизация потоков как инструмент согласования}

Наличие буферов (накопителей) между источником потока и изделием является
важным признаком недостаточной согласованности источника и канала или канала и
изделия по производительности, что позволяет сразу, на самых первых шагах
анализа, увидеть имеющиеся недостатки, но с другой стороны, простейшим и
поэтому очень распространенным способом ликвидации имеющейся
рассогласованности.

Причины и следствия буферизации весьма разнообразны и им можно было бы
посвятить отдельное исследование. Однако, в данной работе оно было бы,
пожалуй, излишним.

В автомобильных ДВС между бензонасосом и карбюратором стоит фильтр. Помимо
основной функции он выполняет также роль небольшого буфера, регулирующего
скорость подачи бензина в карбюратор. Если бы производительность насоса была
полностью согласована с потребностями карбюратора, этот буфер не понадобился
бы. Другое дело, что в этом случае применен простой и дешевый способ
согласования и ставить задачу на повышение согласованности нецелесообразно.

Продолжать традиционный в ТРИЗовских работах ряд примеров было бы излишним --
явление настолько распространенное, что каждый инженер без труда приведет для
себя примеры буферов по вкусу.

\paragraph{6.1.1.1.  Встроенные источники ресурсов.}

Встроенный источник создается для обеспечения автономности системы от внешнего
источника. Поэтому функционально встроенный источник является типичным буфером
между надсистемным источником потока и каналом. Несогласованность заключается
в практическом отсутствии канала подачи потока на этапе функционирования
рабочего органа. Но одновременно встроенный источник является источником
потока по отношению ко всем последующим элементам системы. Одну из этих двух
функций встроенного источника необходимо выбирать в зависимости от целей
анализа.

Необходимость учета в потоковом анализе встроенных источников отсылает нас к
закону повышения полноты частей системы, но эта же необходимость показывает и
ограниченность этого закона: встроенный источник -- все-таки буфер хотя бы в
том смысле, что накапливает ресурс, поступающий откуда-то извне. Но любая
транспортировка и хранение ресурса (особенно, если она связана с его
преобразованием) неизбежно требует прямых материальных затрат, затрат ресурса
работоспособности системы, усложнения системы и т.д. Поэтому встроенный
источник используется только и именно как неприятная необходимость, что
полностью противоречит основному тезису закона повышения полноты частей
системы.

Очень характерны в этом смысле транспортные системы. Железные дороги
оказываются наиболее дешевыми из таких систем. При этом при первой возможности
их переводят на электрическую тягу даже ценой снижения автономности.

В автомобильных ДВС типичным (но не единственным) встроенным источником
являются бензобак и дополнительная канистра. Указанная противоречивость
требований к их объему проявляется здесь в полный рост: на автомобилях для
ралли или путешествий бензобак стараются делать как можно больше, да еще и
засунуть в багажник канистру. Для обычных разъездов по сколько-нибудь развитой
дорожной сети канистру оставляют дома. Для гонок по трассе размер бензобака
делают как можно меньше (болиды Ф1 дозаправляют при каждой смене резины и
экономят при этом на массе машины).

\subsection{Согласование вспомогательных полезных потоков по интенсивности} 

Общая закономерность состоит в том, что в процессе развития системы 
\frameit{\begin{itemize}
\item Мощность источника главного полезного потока все более согласовывается с
  проводимостью канала потока.
\item Проводимость канала главного полезного потока все более согласовывается
  со скоростью преобразования потока в потребителе.
\item Интенсивность вспомогательного полезного потока должна быть согласована
  с интенсивностью главного потока.
\end{itemize}
В большинстве случаев проводимость канала должна стремиться к максимальной
скорости преобразования потока (производительности) в рабочем органе. Мощность
источника должна стремиться к максимальной проводимости канала (она часто
согласована с производительностью еще на этапе проектирования, но этот момент
следует проверить).}

Первые два тезиса и комментарий полностью повторяют таковые для главного
потока и выглядят очевидными. Третий тезис связан с тем, что вспомогательный
поток по определению не имеет самостоятельного значения, но используется для
увеличения эффективности главного потока (присадки в бензин, растворенные в
воде соли в теплообменниках, катализаторы, смазочное масло в движущихся частях
машин, служебные и мета-файлы в информационном потоке и т.д.). Требование
согласования этих потоков с главным, ради которого они и вводятся в систему,
выглядит очевидным.

Это заметно усложняет как анализ, так и решение задачи -- требуется
одновременное взаимное согласование уже четырех-шести объектов! Но именно
поэтому явно зафиксированное требование такого согласования может иметь очень
большое практическое значение: не счесть ситуаций, когда некое очевидное
требование становилось требованием по умолчанию и с ходом времени просто
забывалось.

Типичным примером вспомогательного потока в ДВС является поток масла в
двигателе. В этом случае совершенно необходимым является согласование
интенсивности потока с режимами работы двигателя. Согласование достигается
тем, что масляный насос (источник потока) запитан от оси двигателя. Тем самым
интенсивность оказывается согласована с режимом работы двигателя (режим
обратной связи). В отсутствии такого согласования избыточный поток приведет
как минимум к излишнему расходу ресурса. При этом прямой зависимости между
необходимой интенсивностью этого вспомогательного потока с интенсивностью
главного полезного потока нет, поэтому этот вид согласования здесь не
требуется.  В то же время, очевидным недостатком такого решения (обратной
связи) является необходимость использования более мощного насоса, который
большую часть времени используется не на полную мощность.

Очень важный и частый случай (субтренд?), существенно упрощающий количество
одновременно согласовываемых подсистем:

\frameit{Если количество вспомогательного ресурса однозначно связано с
  количеством главного ресурса и используются они строго одновременно
  (присадки в бензине, соли в растворе воды, мета-файлы в информационном
  потоке...), эффективным способом согласования является их предварительное
  (часто еще в надсистеме) смешивание.}

В этом случае не требуется ни дополнительного канала, ни его согласования
(соответствует использованию потоком канала для другого потока из текущей
версии закона).

Еще одним типичным примером вспомогательного потока в ДВС является поток
присадок к бензину. Поскольку эти присадки нужны строго в то же время, в том
же месте и в жестко заданной пропорции к бензину, согласование проводится
путем предварительного смешивания еще в надсистеме. Но поскольку всегда за все
нужно платить, при применении этого приема фактором расплаты оказывается
снижение управляемости системой. В данном случае октановое число бензина
оказывается строго задано надсистемой и для регулирования применяется
дополнительный компонент системы (октан-корректор), регулирующий опережение
зажигания.

\section{7.  Согласование потоков по другим параметрам}

Крайне важный момент, который в текущей версии учтен двумя приемами в ряду
других и никак не оговорен особо:
\begin{itemize}
\item Повышение удельных характеристик полезного потока.
\item Снижение удельных характеристик вредного потока.
\end{itemize}
Вообще говоря, сказанного достаточно для случая вредного (и паразитного)
потока. Чем меньше эффективность его вредного воздействия, тем лучше для
системы. Поэтому:

\frameit{В процессе развития технической системы источники вторичных вредных
  потоков и паразитных потоков изменяется так, что удельные характеристики
  потока снижаются.}

Например, соотношение углекислого газа и воды в составе выхлопных газов
изменяется в сторону увеличения процентного состава воды в ряду "бензин -- газ
-- спирт".

Следует отметить также, что этот субтренд -- один из самых слабых и редко
реализуемых. Вторичный вредный поток предопределен принципом действия. Поэтому
и параметры этого потока трудно поддаются изменению. Паразитный же поток
должен быть по мере развития системы ослаблен вплоть до полного устранения и
поэтому борьба за его характеристики всегда носит локальный (временный)
характер.

Следует также отметить, что речь идет о \emph{вторичных} вредных потоках.
Первичные потоки задаются надсистемой. Если полезные потоки часто могут как-то
регулироваться (мы можем выбрать нужную нам марку бензина на заправочной
станции), то вредные потоки такой регулировке поддаются с трудом (мы не можем
выбрать состав воздуха, подаваемый в ДВС).

Что касается полезных потоков, то:

\frameit{В процессе развития технической системы источники полезных потоков
  изменяется так, что удельные характеристики потока согласовываются с
  параметрами изделия (приемника потока). При этом характеристики канала
  потока согласовываются с параметрами самого потока (источника).}

Несложно увидеть, что эти формулировки как для вредного, так и для полезного
потока почти повторяют сказанное выше. Разница состоит в том, какой компонент
менять в процессе согласования.  Выше говорилось о необходимости изменения
канала для его согласования с изделием или источником. Здесь говорится о том,
что источник или изделие также могут изменяться для достижения
согласованности.

Разумеется, довольно часто требуется увеличение параметров полезного потока
(например, для ДВС -- теплота сгорания топлива). Но в общем случае требуется
именно согласование этих параметров с параметрами других компонентов (изделием
и каналом). Например, если мы в качестве охлаждающей жидкости ДВС применим
жидкий азот, то ничего путного, скорее всего, не получится. Хотя задачу,
конечно, ставить можно.

\section{8. Управляемость полезных потоков}

Очень важный элемент системы потоков в любой ТС. Он оказался полностью
проигнорирован в существующей версии - видимо, по причине того, что об этом
трактуется в законе повышения управляемости. Но, как и в случае повышения
согласованности, в результате \textbf{управление потоками оказалось полностью
  потеряно в потоковом анализе!} Между тем,

\frameit{\emph{Любой} полезный поток непременно должен иметь систему управления,
  как минимум на уровне включено/выключено.

Вредный поток должен иметь систему управления в тех случаях, когда его
изменение может привести к неисправности системы или неприемлемым вредным
воздействиям в надсистеме.}

Очевидные примеры в ДВС:
\begin{itemize}
\item Бензонасос включается только тогда, когда мы включаем мотор.
\item При перегреве двигателя он выключается (в противном случае он все равно
  выключится, но в форме аварии).
\end{itemize}
Развитие системы управления потоками -- один из немногих субтрендов ЗРТС, для
которых типовые оговорки вроде "как правило" или "статистически достоверно" {--}
не только не нужны, но и недопустимы.

Обязательность подсистемы управления полезным потоком очевидна уже потому, что
на разных этапах жизненного цикла ТС потребление ею потока различно (как
минимум на уровне есть/нет).

Все мы читали и многие помнят, какими сложными, изощренными, оригинальными
приемами удалось инженерам в 1986 году отключить систему блокировки на
Чернобыльской АЭС. Мир праху их, но отключили они именно систему управления
главного полезного потока.

Поэтому разработчики боевых систем используют уже не "защиту от дурака", а
"защиту от умника", что оказалось гораздо сложнее (нужно исходить уже не из
того, что отключить блокировку попытается солдатик с вомьмилеткой -- его-то
мы, конечно, перехитрим, но толковый грамотный инженер ...).

И это -- вовсе не фрагмент из мемуаров бывалого, а простой пример того, что
бывает, когда у потока нарушена простейшая функция управления хотя бы на
уровне "включено/выключено" (при всей сложности систем защиты функционально
они сводятся именно к этому). Итак:

\frameit{В любой системе с явно выделенным полезным потоком ВСЕГДА имеется
  подсистема управления потоком. Развитие этой подсистемы полностью
  определяется законом повышения управляемости.}

\subsection{Два типа управления интенсивностью потока}

После того, как мы разделили для анализа источник потока и его канал,
становятся очевидными и два типа управления: управление проводимостью канала
("вентиль") и управление интенсивностью потока ("насос").

Управление типа "вентиль" выглядит обычно более простым в реализации, поэтому
используется значительно чаще. Однако, "вентиль" имеет существенный недостаток
перед насосом: он обязательно тормозит поток -- является искусственно созданным
бутылочным горлышком!

Поэтому чаще более прогрессивным (но, как правило, более трудным в реализации)
является управление типа "насос", когда мы не трогаем канал, а непосредственно
регулируем интенсивность самого потока. Субтренд замены "вентиля" на "насос"
не является частым, но всегда полезным, когда возможно.

Очевидно, вентиль в канале полезного потока является подсистемой согласования
интенсивности потока. При этом сам канал является избыточным в среднем, хотя в
каждый данный момент его проводимость согласована с потребителем потока (в
соответствии со сказанным выше максимальная проводимость канала полезного
потока должна быть равна максимальной для всех возможных режимов, т.е. --
канал используется в среднем не на полную мощность). Поэтому уже само наличие
"вентиля" в канале потока означает наличие технического противоречия типа
\begin{itemize}
\item Проводимость канала должна быть равной производительности потребителя
  потока, чтобы обеспечить оптимальные режимы работы и наименьшую стоимость
  системы,

  НО

\item Проводимость канала должна быть выше производительности потребителя
  потока, чтобы обеспечить управляемость подсистемы.
\end{itemize}

Точно также и управление типа "насос" требует, чтобы проводимость канала была
выше средней и поток не полностью использует пропускную способность своего
канала -- возникает аналогичное по форме, хотя иное по существу противоречие.
Но в случае "насоса" избыточным оказывается также и мощность источника.

Таким образом, система управления потоком сама по себе является генератором и
индикатором противоречий в системе.

В ДВС примерами системы управления потоком типа "насос" являются именно насосы
(бензонасос, маслонасос, насос охлаждающей жидкости). Примером системы
управления типа "вентиль" являются клапаны впуска топлива и выпуска
отработанных газов.

\section{9. Особенности замкнутых потоков}

Выше мы назвали замкнутым потоком такой поток, который возвращается к
источнику, т.е., такой, канал которого замнут.

Разумеется, проходя через канал и изделие, поток всегда меняет какие-то свои
параметры, поэтому в принципе любой замкнутый поток можно рассматривать как
два разных потока, состоящих из одних и тех же материальных компонентов, при
условии, что изделие первого является источником второго и наоборот. Также
совершенно очевидно, что никакой поток не бывает абсолютно замкнутым. Тем не
менее, использование этого понятия часто облегчает анализ.

Важной особенностью замкнутых потоков является также и то, что это всегда --
потоки вещества. Потоки энергии никогда не могут считаться достаточно
замкнутыми, кроме как в мысленных физических экспериментах. Для потоков
информации это понятие вообще, кажется, не имеет смысла.

В ДВС примерами замкнутых потоков являются:
\begin{itemize}
\item Поток охлаждающей жидкости -- полезный поток.
\item Поток масла в двигателе -- полезный поток,
\item Поток загрязнений в масле -- вредный поток.
\end{itemize}
Другие примеры -- оборотная вода в огромном количестве разных технических
устройств; поток ионов в аккумуляторах, поток муниципального транспорта,
движущегося от кольца к кольцу и обратно и т.д.

В целом замкнутые потоки обладают всеми свойствами потока, но их особенности
предоставляют дополнительные возможности оптимизации.

При анализе замкнутого потока канал следует разделить на два разных участка:
\begin{itemize}
\item прямой канал -- часть канала, проходящая от источника к изделию,
\item возвратный канал -- часть канала, по которой поток возвращается к
  источнику.
\end{itemize}

\subsection{9.1. Полезный замкнутый поток}

На прямом канале полезного потока действуют в полной мере все правила
усовершенствования потоков, как обобщенные в данной работе, так и принятые в
каждой конкретной отрасли инженерных работ.

А вот на возвратном канале замкнутый полезный поток ВСЕГДА является паразитным
(не выполняя полезную функцию в системе, он обязательно потребляет какие-то
ресурсы: как минимум это сам возвратный канал, который нужно проложить, и
энергия, которую необходимо потратить на его перемещение). Поэтому на
возвратном канале полезного замкнутого потока в полной мере действуют те
правила, которые были сформулированы Литвиным и Любомирским: необходимо
обеспечить как можно большую проводимость канала, как можно большую
интенсивность потока, попытаться придать потоку какие-то дополнительные
функции.

\subsection{9.2.  Вредный замкнутый поток}

Для этого типа замкнутого потока также верны все правила, сформулированные для
вредного потока вообще, но есть важная особенность: в этом случае носитель
вредной функции многократно проходит через изделие, в связи с чем, его вредное
воздействие также многократно увеличивается.

Поэтому наличие в системе вредного замкнутого потока всегда является признаком
более серьезного недостатка, чем вредный открытый поток.

Но имеется и дополнительный прием устранения указанного недостатка: необходимо
разомкнуть этот поток.

Например, в случае замкнутого потока загрязнений в масле (а это, в том числе
-- абразивные частицы, наличие которых приводит к более быстрому износу)
применяется обычный фильтр, который тормозит поток. Но имеет смысл поставить
задачу разомкнуть поток, т.е. удалять загрязнения.

\section{10. Поток-носитель и поток-функционал}

Очень часто в системе существуют потоки, чьей единственной полезной функцией
является переносить другой поток (иногда -- одной из нескольких полезных, но
существенной). В существующей версии закона оптимизации потока об этом
упоминается вскользь в ряду способов оптимизации "Использование одного потока
в качестве переносчика второго".

Вместе с тем, поток-носитель часто играет очень важную роль в системе, иногда
-- ключевую.

Можно сформулировать следующую лемму: В любой системе, в которой существует
явно выраженный полезный поток, существует как минимум два потока-носителя.

Действительно, для формирования, направления и управления материальным потоком
требуются энергия и информация. Но и энергия, и информация никогда не
существуют и не передаются сами по себе, но обязательно требуют наличия
материального носителя.

Таким образом, разделение потоков на носитель и функционал является
существенно важным и заслуживает самостоятельного анализа.

Примеры:
\begin{itemize}
\item Углеводородное топливо часто является носителем разного рода присадок (а
  получающаяся при этом смесь является носителем химической энергии).
\item Поток энергии
\item Поток информации
\item При групповой обработке обрабатываемые заготовки (полуфабрикаты)
  располагаются в кассетах (боксах, подложках и т.д.).
\item Тара -- типичный поток носитель.
\item Уже упоминалось, что вода является носителем вспомогательного потока
  солей при приготовлении разного рода рассолов (а сами рассолы часто
  используются как носители потока "холод").
\item Поток трудозатрат оператора (специфический, но вполне реальный поток,
  стоящий работодателю реальных денег). Полезными для производства потоками
  являются потоки принимаемых решений и предпринимаемых действий. Поток затрат
  "Labor cost" является их носителем (не всегда обязательным и очень часто --
  неадекватным ...).
\item Etc, etc.
\end{itemize}
Полезный поток-носитель имеет ряд особенностей по сравнению с полезным потоком
прямого использования. Соответственно, рассмотрение потока как носителя дает
ряд дополнительных возможностей для совершенствования системы:

\subsection{10.1.  Заменимость носителя}

Главная (часто -- единственная) полезная функция потока-носителя -- перемещение
потока-функционала. Поэтому

\frameit{Поток носитель чаще всего не является единственным возможным вариантом
  и может быть заменен.}

\begin{itemize}
\item Явно выраженный тренд замены бензиновых двигателей автомобиля на
  газовые, спиртовые, а также на отказ от ДВС в пользу электродвигателей
  является попыткой заменить поток-носитель энергии.
\item Общеизвестный пример: если в печи заменить воздух на электромагнитное
  поле, получится микроволновка.
\item Когда в различных баках-нагревателях мы заменяем паровую рубашку на ТЭНы
  мы меняем поток-носитель, доставляющий энергию к баку.
\item Появление мобильных телефонов полностью определено переходом на другой
  поток-носитель информации.
\end{itemize}

Субтренд не является всеобщим, но для нужд методики можно было бы даже убрать
оговорку "как правило": даже в случае неудачи попытка увидеть альтернативные
носители может оказаться полезной. Во всяком случае, найти безальтернативный
носитель не так-то просто (легко найти вариант, когда альтернатива
экономически неэффективна или технически нереальна на сегодняшнем уровне
развития техники -- но совсем уж безальтернативный носитель найти очень
трудно).

\subsection{10.2.  Вредность носителя}

Поток-носитель сам по себе (если отвлечься от его функции нести собственно
полезный поток) почти всегда является вредным сам по себе (все, что он делает
в системе помимо своей главной функции нам не нужно, а ресурсы всегда
потребляются).

\subsubsection{10.2.1. Основные субтренды оптимизации потока-носителя}

Из двух указанных главных свойств носителя (особенно -- последнего) вытекают и
основные приемы (субтренды) его оптимизации:
\begin{itemize}
\item Увеличение плотности потока-функционала на носителе (вплоть до замены
  носителя).

  Например:
  \begin{itemize}
  \item Использование в качестве охлаждающей жидкости вместо воды различных
    спиртосодержащих жидкостей.
  \item Использование более высокого рабочего напряжения в электрических
    силовых системах.

    Это далеко не всегда возможно из других соображений (в частности -- из
    соображений безопасности), но, например, в линиях электропередачи -- это
    тренд давний и очень эффективный.
  \item Более плотная загрузка транспорта (грузовой транспорт -- типичный
    носитель полезного потока «груз»).
  \item Последняя революция в информатике связана с появлением широкополосных
    систем передачи данных.
  \item Повышение производительности труда вполне можно рассматривать как
    увеличение плотности полезных операций на единицу затраченных усилий --
    пример не вполне корректный, зато абсолютно актуальный.
  \end{itemize}
\item Уменьшение стоимости носителя.

  Разумеется, это актуально для всех видов ресурсов. Однако, если для главного
  потока носителя это бывает неоправданно (велик риск потери качества), то для
  потока-носителя, который уже не будет присутствовать в конечном продукте --
  это актуально всегда.
  \begin{itemize}
  \item Тенденция замены бензина на "альтернативные виды топлива" существенно
    ускоряется в период дорогой нефти (хотя не спадает и в периоды дешевой).
  \item Очень ярко это видно в стремлении всячески сократить стоимость
    промышленной тары.
  \end{itemize}
\item Придание потоку-носителю дополнительных полезных функций.

  Например:
  \begin{itemize}
  \item Придание потребительской таре функций информации, защиты etc.
  \item Использование в качестве кассет групповой обработки деталей будущих
    корпусов приборов.
  \item И т.д.
  \end{itemize}
\end{itemize}

\subsubsection{10.2.2.  Использование субтрендов для вспомогательных полезных
  и вторичных вредных потоков}

Поток-носитель является частным случаем вспомогательного полезного потока ДО
освобождения от функционала и типичным вторичным вредным потоком ПОСЛЕ.
Поэтому все правила оптимизации, потоков, изложенные в соответствующих
параграфах, верны, но с обязательным учетом ДО/ПОСЛЕ (канализация носителя
была бы неуместна до освобождения от функционала; точно также, после
использования никакое согласование само по себе уже не нужно -- нужно от него
избавиться).

\subsection{10.3.  Оборотные потоки-носители}

Поток-носитель очень часто может использоваться неоднократно, возвращаясь к
пункту "загрузки функционалом". Но делается это далеко не всегда: ровно
половину пути/времени возвращаемая "тара" крутится пустой. Критерий здесь, в
общем-то, очевиден: если стоимость возврата ниже стоимости самого
потока-носителя, то его следует возвращать. И наоборот.

Соответственно, возникают два субтренда:
\begin{itemize}
\item Уменьшение стоимости носителя (см. выше).
\item Уменьшение стоимости оборота носителя (см. «замкнутый поток»).
\end{itemize}
Эти два субтренда вполне могут развиваться одновременно.
\begin{itemize}
\item Типичный пример: оборотная тара. Голубая мечта производственников --
  бестарное производство. Но, поскольку это удается нечасто, огромные усилия
  тратятся на реализацию этих двух субтрендов. Кто помнит сбор пустых бутылок
  -- тому других примеров уже не надо.

  Вместе с тем, внутризаводская промежуточная тара бывает оборотной очень
  часто -- оборот короткий, легко управляемый и т.д.

\item Другой пример: грузовой транспорт. Штука сильно недешевая. Поэтому при
  каждой возможности пытаются использовать попутный транспорт (довольно
  крупные фирмы на окраинах больших городов неплохо живут использованием этого
  тренда). Универсальные контейнеры и собственно транспортные системы, строго
  заточенные под такие контейнеры, также неплохо помогают увеличить
  оборотность.
  
\item Если в качестве теплоносителя используется вода, ее часто сбрасывают в
  канализацию после использования. Но как только используется какой-нибудь
  рассол, тосол и т.п. -- он становится многооборотным.
\end{itemize}
Общее правило (необязательное, но часто реализуемое): если поток-носитель
полностью крутится в системе, его можно сделать многооборотным и, как правило,
это очень эффективно. Если же поток выходит в надсистему, эффективность его
возврата резко падает.

Яркий пример: системы оборотного водоснабжения внутри одного производства,
пусть даже крупного, легко могут быть замкнутыми, но уже для небольшого
поселка (часто гораздо меньшего, чем упомянутое производство) -- почти никогда.

\section{11. Особенности потоков вещества, энергии и информации}

Всюду выше говорилось о потоках вещества, энергии и информации. Очевидно,
однако, что эти три вида потоков должны иметь некоторые особенности.
Соответственно, оптимизация этих потоков тоже должна несколько различаться.

\subsection{11.1.  Потоки энергии и информации как потоки-функционалы}

Как я отмечал выше, практически всегда потоки энергии информации требуют
поток-носитель. Соответственно, к ним в полной мере применимо все, что сказано
выше в главе 10.

\subsection{11.2.  Неприменимость законов сохранения к потокам информации}

Информация не подчиняется законам сохранения. По крайней мере, таким, которые
характерны для вещества и энергии. Эта особенность потоков информации
определяет и особенности оптимизации этих потоков. В частности:
\begin{itemize}
\item Использование потока в источнике не обязательно приводит к изменению
  информации (число одновременно читающих этот текст может быть каким угодно,
  что никак не отразится на самом тексте).
\item Увеличение интенсивности потока не обязательно приводит к
  пропорциональному увеличению собственно информации (я могу увеличить длину
  этого текста раз в пять, но читатель вряд ли будет знать на эту тему в пять
  раз больше).
\item Канализация вредного потока информации в надсистему сама по себе не
  ликвидирует ее в системе (можно опубликовать сплетни из жизни офиса, но они
  останутся и в офисе).
\item С другой стороны, вредную информацию можно полностью устранить без
  канализации.
\end{itemize}

\section{1.4 Выводы}

На мой взгляд, предложенные формулировки потоков и трендов их развития более
адекватно отражают реальные потоки в системах и пути их усовершенствования.

Все эти примеры, разумеется, можно охарактеризовать и существующими
субтрендами разных законов. Моей задачей не было все переиначить, но просто
структурировать анализ, сделав его более инструментальным.

\begin{thebibliography}{xxx}
\bibitem{1} Альтшуллер Г.С., Творчество как точная наука. -- М.: Сов. радио,
  1979.  Законы развития систем; 2.  Закон «энергетической проводимости»
  системы. \url{http://www.altshuller.ru/triz/zrts1.asp#12}

\bibitem{2} Альтшуллер Г.С., Злотин Б.Л., Зусман А.В. и др., Поиск новых идей:
  от озарения к технологии (теория и практика решения изобретательских задач),
  Кишинев, "Каpтя Молдовеняскэ", 1989 г.
  \url{http://www.trizway.com/content/poisk_novih1.pdf} ч. 1, стр. 56

\bibitem{3} Юрий Петрович Саламатов, 1991-1996 г. "Система Законов Развития
  Техники (Основы Теории Развития Технических Систем)".
  \url{http://www.trizminsk.org/e/21101440.htm}

  Yuri Salamatov, TRIZ: the Right Solution at the Right Time: a Guide to
  Innovative Problem Solving
  \url{http://vietnamwcm.files.wordpress.com/2008/07/inovative-problem-solving.pdf}

\bibitem{4} Владимир Петров, Серия статей «Законы развития систем», 24
  сентября 2002 г.
  \url{http://www.trizland.ru/trizba/pdf-books/zrts-12-microlevel.pdf} стр.2;
  \url{http://www.trizland.ru/trizba/pdf-books/zrts-16-energo.pdf}

\bibitem{5} Литвин С. С., Любомирский А.Л. Законы развития технических систем,
  февраль 2003.  \url{http://www.metodolog.ru/00822/00822.html} п. 5.1.4.
\end{thebibliography}

\section{Комментарии}

\subsection*{Классификация потоков в технических системах, akyn 2011-04-19.}

Дорогой Юрий!

Большое Се-Се (спасибо – кит.) за интересную и очень полезную работу!

Полагаю, по ходу подробного изучения возникнут еще вопросы и замечание, но
одно лежит на поверхности. Вы рассмотрели и устройства и процессы «в одном
флаконе». Тогда как личный опыт говорит о том, что для этих объектов есть как
общие закономерности, так и особенности.

А что считаете по этому поводу Вы?

С уважением, akyn

\subsection*{Re: Классификация потоков в технических системах. priven
  2011-04-19} 

Согласен с АКыном. Статья очень хороша, конкретна и требует внимательного
прочтения.

Пока что "ход в сторону":
\begin{quote}
  В этом примере возникает еще одна очень любопытная коллизия: в это же
  оперативное время и в той же оперативной зоне возникает другой паразитный
  поток -- моно-окиси углерода. Противоречие здесь выглядит примерно так:
  "количество кислорода должно быть избыточным, чтобы предотвратить неполное
  сгорание углеводородов, но количество кислорода должно быть
  стехиометрическим, чтобы предотвратить окисление азота (а также
  предотвратить ряд иных проблем).

  Т.е., возникает необходимость решать два разных противоречия, сошедшихся
  строго в одно время в одном месте: если кислорода много -- возникают окислы
  азота, если кислорода мало -- возникает моно-окись углерода. В ТРИЗ,
  насколько мне известно, нет инструментов для одновременного согласованного
  решения двух или более противоречий. Было бы интересно и полезно такие
  инструменты разработать!
\end{quote}

То есть, имеем следующие варианты:
\begin{itemize}
\item Низкая Т $\to$ нет паразитных потоков, но малый КПД
\item Высокая Т $\to$ высокий КПД, но есть паразитные потоки
\item Высокая Т + много окислителя $\to$ окислы азота
\item Высокая Т + мало окислителя $\to$ монооксид углерода
\end{itemize}
В принципе, здесь можно, вроде бы, использовать и АРИЗ: вначале выбираем
половинку «температурного» противоречия (выбираем высокую температуру,
поскольку при низкой температуре малый КПД неустраним), а затем выбираем
половинку «поточного» противоречия, и при высокой температуре боремся либо с
окислами азота, либо с монооксидом углерода. Наверное, есть и иные способы.

Но вот какой вопрос конкретно по этому примеру: а если точно отрегулировать
соотношение окислитель / топливо наилучшим способом (при оптимальной
температуре, определяемой по полезной функции), то останутся ли проблемы
существенными, или же с их решением можно будет повременить до (так или иначе
неизбежного, наверное) массового перехода на электромобиль? Другими словами,
нужно ли это противоречие именно устранять, или же в данном случае достаточно
минимизировать НЭ?

Заранее благодарен,
Александр.

\subsection*{Re: Классификация потоков в технических системах. Gregory
  Frenklach 2011-04-19}
  
Использование понятия "поток" в ТРИЗ лично мне всегда казалось лишним.
Относился и отношусь к этому понятию, как к дополнительной сущности с весьма
сомнительной пользой.

\subsection*{Re: Классификация потоков в технических системах. Alex L.
  2011-04-19} 

Мне кажется, статья весьма и весьма интересная и полезная. Не со всем в ней
согласен, но надо еще подумать. Спасибо!

\subsection*{Re: Классификация потоков в технических системах. priven
  2011-04-20} 

Gregory Frenklach wrote:
\begin{quote}      
  Использование понятия "поток" в ТРИЗ лично мне всегда казалось лишним.
  Относился и отношусь к этому понятию, как к дополнительной сущности с весьма
  сомнительной пользой.
\end{quote}

Поскольку я по специальности химик-технолог, понятие "поток" кажется мне
совершенно естественным и понятным: ведь любая химическая технология -- это,
так или иначе, есть именно описание потока или потоков, а уже потом только --
устройств и способов. То же самое могу сказать и в отношении своей третьей
специальности (базы данных): там надо проектировать и анализировать именно
потоки данных прежде всего, а уже исходя из них, определять оптимальные
структуры данных и способы их обработки (если поступать наоборот, то база
будет очень хорошей и правильной по своей структуре, но работать будет со
скоростью "в час по чайной ложке"). Возможно, для инженера-конструктора
ситуация иная -- здесь судить не берусь.

\subsection*{Re: Классификация потоков в технических системах. akyn
  2011-04-20}

Как и обещал, являясь заинтересованным лицом продолжаю свои несущественные
мелочные замечания:
\begin{itemize}
\item[1.] Во введении дана увязка Потоков только с Функцианальным Анализом.
  Тогда как потоки используются и в Технической Системе и в Стандартах.
\item[2.] Использовано без объяснения понятие «Вредный поток».
\item[3.] Введено понятие «Вспомогательный поток». А может быть стоит тогда
  определить и другие используемые в рамках ТРИЗ термины: дополнительный,
  недостаточный, избыточный? Это позволило бы лучше увязать Потоковый Анализ с
  другими инструментами.
\end{itemize}

\subsection*{Re: Классификация потоков в технических системах. Gregory
  Frenklach. 2011-04-20}

priven wrote:
\begin{quote}
  Поскольку я по специальности химик-технолог, понятие "поток" кажется мне
  совершенно естественным и понятным: ведь любая химическая технология -- это,
  так или иначе, есть именно описание потока или потоков, а уже потом только -
  устройств и способов. То же самое могу сказать и в отношении своей третьей
  специальности (базы данных...
\end{quote}
Остаюсь при своём мнении.  Потоки -- штука для ТРИЗ "чужая", НО для некоторых
ДРУГИХ методик анализ потоков -- это то, что надо и могла бы получится отличная
синергия.

Но к этой работе это никакого отношения не имеет.  Что же касается работы -- я
не согласен с той её частью, где говорится об информации.

\subsection*{Re: Классификация потоков в технических системах. lebedur
  2011-04-20} 

Коллеги, доброго времени суток и большое спасибо за отклики.  С нетерпением
ожидаю обещанного «внимательного прочтения» и следующего за этим размазывания
по стенке. Тут, конечно, есть о чем поговорить и с чем не соглашаться.  Пока
спешу ответить на то, что уже написано.

Quote: Akyn
\begin{quote}
  Вы рассмотрели и устройства и процессы «в одном флаконе». Тогда как личный
  опыт говорит о том, что для этих объектов есть как общие закономерности, так
  и особенности.
\end{quote}
И личный опыт, и общественный и какой угодно другой. Мое мнение здесь примерно
такое: когда мы рассматриваем некую систему, мы говорим об устройствах и
процессах, то мы говорим о функциональной модели системы-устройства и
системы-процесса. Но при этом понимаем, что система-то одна и та же, но
рассмотренная с разных сторон. Есть сторонники рассматривать тот или иной
вариант модели или оба одновременно. Имеются разной успешности попытки
объединить эти два вида моделей (можно упомянуть работы О. Фейгенсона и
Александра Кашкарова – из числа известных мне). Потоковая модель является еще
одним видом модели. Связанной, разумеется, с первыми двумя, но к ним не
сводящейся.

Quote: Priven
\begin{quote}      
  Другими словами, нужно ли это противоречие именно устранять, или же в данном
  случае достаточно минимизировать НЭ?
\end{quote}
Можно и так, конечно. Вообще, использование ТРИЗ вовсе не является
единственным способом решать технические задачи. Я здесь говорил немного о
другом: когда мы формулируем технические противоречия, мы всегда их разделяем
и далее решаем порознь. Но может быть ситуация, когда противоречия ни во
времени, ни в пространстве не разделить. Решать их нужно бы одновременно. Да и
вообще, это могло бы оказаться очень полезным – научиться решать связанные
противоречия вместе. Я просто поставил вопрос. Ответа не знаю, увы.

Что же до конкретной задачи: никогда никакой параметр не может быть задан
абсолютно точно. В частности, соотношение реагентов в реакции. Поэтому при
подгоночном решении задачи всегда будет лететь немножно окислом азота и
немножко закиси углерода.

Quote: Gregory Frenklach
\begin{quote}
  Даже не знаю, что сказать.  «Использование понятия "поток" в ТРИЗ лично мне
  всегда казалось лишним.  Относился и отношусь к этому понятию, как к
  дополнительной сущности с весьма сомнительной пользой».
\end{quote}
Дело-то все в том, что потоки в системах существуют независимо от того что мы
с вами о них думаем, и даже независимо от того, думаем ли мы вообще. Отчего же
бы их и не проанализировать!

Quote:
\begin{quote}
  «Потоки - штука для ТРИЗ "чужая"». 
\end{quote}
Мсье, вам шашечки или ехать? Считанные разы в совершенно исключительных
случаях задача проекта ставится как «устранить недостаток системы методами
канонической ТРИЗ». Гораздо чаще просят «устранить недостаток системы». Может
быть, конечно, это мое исключительное невезение…

Quote:
\begin{quote}
  «Что же касается работы -- я не согласен с той её частью, где говорится об
  информации».
\end{quote}
Принял к сведению. Теперь бы еще узнать, что именно вас там не устраивает –
ваши комментарии, когда по существу, обычно очень интересны и как минимум
неожиданны.

Quote: Alex L
\begin{quote}
  «Не со всем в ней (статье) согласен, но надо еще подумать». 
\end{quote}
Саша, спасибо за «не совсем». Еще в прошлый раз с некоторым трепетом ждал
твоей реакции и был ею приятно удивлен. Было бы странно, если бы ты согласился
здесь со всем.

Три вопроса Akyn'а, которые он почему-то назвал мелочными придирками. Ни фига
себе мелочи!

1. Про техническую систему не понял. Стандарты – способ решения задачи, я же
говорю, в основном, об анализе. Но вообще-то связка потокового и
функционального анализов настолько естественна (на мой взгляд), что
используется очень широко (разумеется, только там, где потоковый анализ вообще
используется).

2. Насчет вредного потока – согласен полностью. Нужно вводить. Случилось из-за
того, что я не придумал анализ потоков, а только серьезно изменил уже
имеющийся подход (имею ввиду ЗРТС по Литвину и Любомирскому). Но для меня этот
подход настолько естественен, что не приводил тех определений, с которыми
согласен (а в тексте есть и другие). Но это – всего лишь объяснение того, как
так получилось, что вовсе не служит оправданием. Спасибо. И за будущие такие
«мелочи» -- спасибо заранее.

3. Ввести понятия можно, но не думаю, что нужно. Я здесь вводил те, которые
возникали у меня сами собой при решении разных задач. Стоит ли вводить
дополнительные определения, если не знаешь, что с ними дальше делать?

\subsection*{Re: Классификация потоков в технических системах.  Gregory
  Frenklach 2011-04-20}

Вы меня зря к защитникам "чистоты" канонической ТРИЗ причисляете. Не в этом
дело.  Я просто не вижу (уже более 20-ти лет), как этот подход помогает решать
(или хотя бы облегчить) решение задач, нерешаемых (или сложно решаемых) с
помощью других инструментов ТРИЗ.  Но, как я уже писал, в рамках других
методик (например, теория ограничений) этот подход будет очень органичен и
весьма полезен.  Это мысли по поводу подхода "вааще" и никак не критика Вашей
работы.

Теперь по поводу информации...  Вы не определили, что Вы понимаете под
информацией в общем и под потоком информации в частности.  Например, в одном
месте Вы в качестве увеличения интенсивности потока информации берёте длину
текста.  В другом случае Вы связываете количество читающих с использованием
информации, но сводите всё к тексту.  В третьем случае под "канализацией"
информации в надсистеме понимаете публикацию этой информации.

У меня непонятки или у Вас? Ах да, я ведь написал, что не согласен, но не
написал с чем.  Я не согласен с тем, что длина текста, например -- это
интенсивность потока.  И с тем, что Вы называете "канализацией информации".
Думаю, что это из-за того, что у нас с Вами разное понимание того, что считать
информацией и что считать потоком информации.

Я бы для начала начал с малого и определился бы с тем, что можно использовать
в качестве информации о технической системе, а потом бы перешёл к тому, что
есть потоки информации в технических системах.

А тексты, офисы и головы читателей в качестве примеров для потоков информации
не использовал бы, чтобы... ну хотя бы, чтобы таких, как я не путать:)
Успехов!

\subsection*{Re: Классификация потоков в технических системах. akyn
  2011-04-21}

К сожалению, химик, особенно химик-аналитик (пришлось и этим заниматься),
просто обязан быть занудой. Поэтому продолжаю придирки.

lebedur wrote:
\begin{quote}
  Мое мнение здесь примерно такое: когда мы рассматриваем некую систему, мы
  говорим об устройствах и процессах, то мы говорим о функциональной модели
  системы-устройства и системы-процесса. Но при этом понимаем, что система-то
  одна и та же, но рассмотренная с разных сторон.
\end{quote}
Не понял, как эти системы могут быть одинаковыми. Действительно, между
Устройствами и Процессами много общего, но есть и специфика. В том числе и при
построении функциональных моделей.

Связь Потокового Анализа с Функциональным действительно очевидна. Но в
классической модели Системы есть поток энергии от Источника через Двигатель,
Трансмиссию и Инструмент к Объекту обработки. ВеПоль, являясь моделью
взамодействия Инструмент-Объект тоже завязан на поток Энергии. Поэтому-то и
появился вопрос о единстве терминологии.

По поводу ссылок на авторитеты. Юра, это ТВОЯ работа, и отдуваться тебе. Когда
будем обсуждать работу Любомирского – тогда он и «ответит за базар».

А вот с Григорием по поводу потоков не согласен. Впрочем, бог ему судья. Не
попадалась ему за 20 лет задача с потоками – повезло значит. Значит такие
задачи решал. Впрочем: «Ты суслика видишь? – Нет! – А он есть!...» (с).

А вот по поводу информации я с ним согласен. Действительно, у Информации много
общего с Энергией. Но есть и много отличий. Так что, как и в случае
Устройство-Процесс, так и здесь я бы выделил особенности Информации более
четко.  Или, в первом варианте работы, пока бы их не трогал.

\subsection*{Re: Классификация потоков в технических системах. priven
  2011-04-21} 

Gregory Frenklach wrote:
\begin{quote}
  Я не согласен с тем, что длина текста, например - это интенсивность потока.
  И с тем, что Вы называете "канализацией информации". Думаю, что это из-за
  того, что у нас с Вами разное понимание того, что считать информацией и что
  считать потоком информации.

  Я бы для начала начал с малого и определился бы с тем, что можно
  использовать в качестве информации о технической системе, а потом бы перешёл
  к тому, что есть потоки информации в технических системах.
  
  А тексты, офисы и головы читателей в качестве примеров для потоков
  информации не использовал бы, чтобы... ну хотя бы, чтобы таких, как я не
  путать:)
\end{quote}

Еще одна прекрасная иллюстрация того же самого, что "всплыло" в нашей статье
про модели: отсутствие четкого определения понятия, которое "и так понятно"
автору.

К замечанию Григория полностью присоединяюсь по форме, но не по существу
работы. Меня как раз ничего не смутило в описании потоков информации, потому
что я привык работать с компьютерными информационными системами и базами
данных, а в них информация измеряется битами -- то есть, в первом приближении,
"объемом текста". А сам термин "информация" я понял не как "информация О
СИСТЕМЕ", а как "информация, циркулирующая В системе": она может быть (и чаще
всего бывает) информацией, не имеющей к самой системе прямого отношения. Когда
мы говорим по телефону, в телефонной сети циркулирует поток информации -- но
это (чаще всего) информация отнюдь не о телефонной сети.

Не согласен здесь с Юрием в двух моментах.

1. Не согласен, что к информации не применим закон сохранения. Он применим,
хотя здесь часто возникает путаница. И применим он как раз к потоку
информации: объем информации на выходе потока равен объему на входе, но часть
полезной информации может превращаться в "информационный шум", то есть во
вредный поток (который часто воспринимается как "отсутствие информации" или
"потеря информации"). А вот за пределами потока информация в самом деле не
сохраняется.

2. Не вполне согласен, что вредную информацию можно "устранить без
канализации", -- это зависит, опять же, от того, как рассматривать информацию:
как информацию в потоке или информацию в системе. Если в системе (в
статическом виде) -- тогда да. А если в потоке -- вопрос, как минимум,
спорный.

В любом случае, уточнение термина "информация" и того процесса, к которому она
имеет отношение в логике автора, было бы очень полезным.

\subsection*{Re: Классификация потоков в технических системах.  lebedur
  2011-04-21} 

В целом согласен со всеми, в частностях, как всегда в нашей дивной науке никто
ни с кем до конца не согласен. Может быть, это не так и плохо?

Полностью согласен, что было бы неплохо иметь общепризнанный глоссарий и им
пользоваться. Но, во-первых, кто бы взялся его написать? А во-вторых, как бы
уговорить друг друга с ним согласиться? Поэтому пока приходиться пользоваться
терминами так, кто их как понимает. Все-таки, в каждой статье давать полный
перечень определений – нелепо.

Но абсолютно согласен с тем, что термин «информация» я использую плохо. Тут
Григорий и Александр абсолютно правы. Подумаю.

При этом в данном случае я говорил об информации скорее в том смысле, что
привел Александр. Т.е., как о чем-то, измеряемом в битах или длине текста. При
этом вполне можно удлинить любой текст в заданное количество раз, сохранив
количество полезной информации, но увеличив количество информационного мусора.

Но в целом этот раздел проработан слабо. Если честно, то в тех проектах, где
мне приходилось работать с информационными потоками, речь шла обычно о
конкретном сигнале, имевшем конкретного носителя.

Что же до законов сохранения – я на всякий случай и написал осторожненько, что
они у информации скорее другие, чем привычные нам для энергии и вещества.

Что же до полезности понятия потока, то мое мнение такое:

1. Тут опять же вопрос терминологии. Если в каком-то проекте у меня что-то все
время булькает и куда-то течет, то я сначала использую понятие потока, а уж
потом думаю ТРИЗ это или нет.

2. В нашем курятнике потоковый анализ причисляют к вполне ТРИЗовским
методикам. О том, что многие думают иначе я вообще узнал меньше года назад.
Решать поставленные начальством задачи эта штука помогает. Тихо надеюсь, что в
новом варианте, который предлагаю я, будет помогать лучше. А все остальное –
терминология…

Акыну. Саша, я не ссылаюсь на Любомирского как на авторитет, чьим именем
следует затыкать недовольным рты. Я просто пояснил, откуда взялась моя ошибка:
там, где я учился ТРИЗу, была некая общепринятая терминология, которую не
приходилось оговаривать. Значительная часть этой терминологии пришла из
текстов Литвина и Любомирского. Моя же ошибка (полностью согласен), что я с
той терминологией вышел на другую аудиторию. Разумеется, при этом нужно
расшифровывать термины. Просто некоторые термины, вроде «вредный поток»,
просто не приходит в голову дефинировать. Это, конечно, всплыло не только в
вашей последней статье и не только здесь – большая беда…

Что же до моделей, то мое мнение здесь такое: Имеется Система. Мы ее
рассматриваем в зависимости от обстановки и решаемой задачи то как процесс, то
как конструкцию. Не знаю конструкции, которая не была бы вовлечена в
какой-нибудь процесс. Не знаю процесса, которому для протекания не нужна какая
ни на есть конструкция. И точно также одной из моделей может быть потоковая.

Александру Привеню. Насчет удаления вредной информации. Я вот тут стер пару
файлов из компьютера. Удалил. Но не канализировал.

Остаюсь с уважением и ожиданием новой критики
Юра Лебедев

\subsection*{Re: Классификация потоков в технических системах.  priven
  2011-04-21}

lebedur wrote:
\begin{quote}  
  В данном случае я говорил об информации скорее в том смысле, что привел
  Александр. Т.е., как о чем-то, измеряемом в битах или длине текста. При этом
  вполне можно удлинить любой текст в заданное количество раз, сохранив
  количество полезной информации, но увеличив количество информационного
  мусора.  ...
  
  Насчет удаления вредной информации. Я вот тут стер пару файлов из
  компьютера.  Удалил. Но не канализировал.
\end{quote}
Юрий, нет ли в этих словах противоречия?

С одной стороны, если информация измеряется в битах, то стоит ли говорить о
"полезной" и "бесполезной" информации в тексте? Биты же вроде бы и на
полезную, и на бесполезную информацию одинаковые.

С другой стороны, если говорить об информации именно как о том, что существует
в памяти компьютера и передается посредством технической системы, то что
означает "стер файл"? Это ведь на самом деле всего лишь метафора такая. А в
действительности ничего Вы не "стерли", а просто заменили на диске одни биты
другими битами. При этом общее количество бит на диске совершенно не
изменилось -- просто некоторые из них Вами не используются, но они ведь все
равно существуют. Опять же, физически при "стирании" никуда не деваются и биты
самого файла (поэтому его и можно "восстановить"), а просто ЕДИНСТВЕННЫЙ бит
(причем этот бит находится ВНЕ самого файла) устанавливается в положение,
сигнализирующее о том, чтобы система не показывала этот файл пользователю...
Но даже и при перезаписи, и даже при форматировании диска количество бит на
нем все равно не изменяется -- и до тех пор, покуда информационная система
работает в штатном режиме, "закон сохранения информации" действует в ней столь
же неотвратимо, как и закон сохранения энергии. А вот если винт случайно на
пол уронить -- тогда различие между энергией и информацией и обнаружится...

С третьей стороны, говоря о том, что разные (в том числе и по размеру) тексты
несут одинаковую информацию, Вы уже намекаете на семантику с семиотикой, в
которой понятие "информация" трактуется совершенно не так, как в технике, -- и
это уже ближе к тому смыслу, который в этот термин, видимо, вкладывает
Григорий...

В общем, я поддерживаю предложение АКына -- исключить пока что
"информационный" блок Вашей системы из рассмотрения. Исключить не навсегда, а
до тех пор, пока не "утрясется" четкое понимание как самих терминов
"информация", "поток информации", "носитель информации", так и тех операций,
которые с этой информацией производятся в рассматриваемой (технической -- или
какой-то иной -- какой именно?) системе. Если у понятия нет определения --
полбеды, определение можно дать и "задним числом", в конце концов, лишь бы
само понятие использовалось в тексте в одном и том же смысле.

\subsection*{Re: Классификация потоков в технических системах.  Gregory
  Frenklach 2011-04-21}

akyn wrote:
\begin{quote}
  ...с Григорием по поводу потоков не согласен. Впрочем, бог ему судья. Не
  попадалась ему за 20 лет задача с потоками – повезло значит. Значит такие
  задачи решал. Впрочем: «Ты суслика видишь? – Нет! – А он есть!...» (с).
\end{quote}
Не знаю, не знаю...  Можно ЛЮБУЮ из решённых мной задач представить с помощью
"потоков" {--} было бы желание.  А вот нужно ли?

Ещё раз -- в рамках других методик использование наработок, сделанных в ТРИЗ
по потокам, будет весьма полезным.  Посмотрите, кстати, как органично
вписываются потоки вещества, энергии, информации в метод Коллера.  Вот об этом
я и говорю.

Впрочем, бог Вам судья:)
Не хотите прислушаться -- не надо.

\subsection*{Re: Классификация потоков в технических системах.  lebedur
  2011-04-21} 

Григорий,

Как раз хочу прислушиваться и прислушиваюсь. Более того, на этот раз почти
согласен. Анализ, он и есть анализ. Анализировать всегда можно с разных точек
зрения. Когда в проекте есть время и другие ресурсы, то это даже полезно.
Разумеется, можно обойтись без потокового анализа. Можно, наоборот, обойтись
только потоковым. Но на мой взгляд, правильнее выбирать то, что удобнее в
каждом конкретном случае. И разумеется – для каждого конкретного человека.
Знаю людей, которые терпеть не могут ФСА. Знаю теперь еще одного человека,
который предпочитает обойтись без потокового. Но как раз для того, чтобы такой
выбор был, лучше иметь все эти методики в арсенале. Как инструменты на полке.
Можно, конечно, построить одним топором. Иногда получается красиво. Но все же
лучше, чтобы в запасе была еще хотя бы пила. Считайте, что я просто решил
слегка заточить пилу – вдруг кому пригодится!

Александр,

Согласен и с вами, но опять же отчасти. Давайте уберем пока из рассмотрения
информационные потоки – там все сыро.

Тем не менее: на чистом диске битов нет. Там есть только ячейки памяти. Биты
же информации возникают тогда, когда я упорядоченно эти ячейки зарядил. И если
я диск отформатировал (полным форматированием, на это хватает даже моей
компьютерной грамотности), то ячейки-то остались, а вот информация стерлась. 

Удачи всем нам, Ю. Лебедев

\subsection*{Re: Классификация потоков в технических системах.  Gregory
  Frenklach 2011-04-21}

\begin{quote}
  lebedur wrote: Знаю теперь еще одного человека, который предпочитает
  обойтись без потокового.
\end{quote}
Я предпочитаю не обойтись без потокового анализа, а найти ему более подходящую
(чем ТРИЗ) надсистему.  Он тогда и работать лучше будет.

Одна из таких более подходящих надсистем -- это теория ограничений.  В рамках
другой (метод Коллера) потоки уже используются.  Почему я так "зациклился" на
том, что это не ТРИЗ?  Какая в конце концов разница?

Дело в том, что если относить потоковый анализ к ТРИЗ, то нужно обеспечить его
органичную стыковку с основными понятиями этой самой ТРИЗ, чтобы не выглядело,
как чужеродное тело.  Как это сделали, когда стыковали ТРИЗ и ФСА, например.

А это (стыковка и замазка стыков) -- большая и (на мой взгляд) лишняя работа.
В то же время стыковка потокового анализа с теорией ограничений потребует
меньше работы и даст лучший результат. В результате получится очень неплохая
модификация теории ограничений, где потоковый анализ будет "на месте".

\subsection*{Re: Классификация потоков в технических системах. priven
  2011-04-21} 

\begin{quote}
  lebedur wrote: Александр, Согласен и с вами, но опять же отчасти. Давайте
  уберем пока из рассмотрения информационные потоки – там все сыро.
  
  Тем не менее: на чистом диске битов нет. Там есть только ячейки памяти. Биты
  же информации возникают тогда, когда я упорядоченно эти ячейки зарядил. И
  если я диск отформатировал (полным форматированием, на это хватает даже моей
  компьютерной грамотности), то ячейки-то остались, а вот информация стерлась.
\end{quote}

Юрий, я ведь кого зря по мелочам не критикую. Если критикую по мелочам --
значит, по большому счету, как правило, согласен. Но давайте все же не будем
про биты, которые -- на самом деле -- никуда не деваются даже при
форматировании диска: сколько их там, Вы можете увидеть, если глянете емкость
этого самого диска. "Стирая" информацию, Вы на самом деле просто записываете
одни биты вместо других -- вот и всё. С той единственной разницей, что
"старые" биты были Вам когда-то полезны (но стали бесполезными или вредными),
а "новые" не являются ни полезными, ни вредными по определению. Но ведь
полезность определяется ВАМИ, а не техникой, -- для нее всё едино, хоть
"стертый" файл, хоть "не стертый". Да и само понятие файла -- уже само по себе
абстракция, ну нет на диске никаких таких "файлов", это только лишь образ в
мозгу пользователя.

Я к тому, за что Вы, вроде бы, и сами ратуете: за четкость. Информационные
потоки -- это тоже потоки. Но нужно все же определить, что именно Вы понимаете
под "информацией", и, исходя из этого, идти дальше. В техническом смысле
"стирание файла" как "потеря информации" -- смешно и глупо, простите (не
сочтите за личную критику -- я глупостей делаю немало и всегда благодарен тем,
кто мне на них указывает). В семантическом -- это, напротив, самое то и есть:
информация "была, и нету". Но в последнем случае объем информации измеряется
отнюдь не числом знаков в тексте: там совсем другие "измерители", к битам
прямого отношения не имеющие.

Я думаю (могу, как всегда, ошибаться!), что все же в рамках Вашего подхода
больше подойдет формальное описание информации: то, что передается технической
системой и измеряется в битах, безотносительно к полезности оной информации
для человека. Мне кажется (снова могу ошибаться!!!), так Вам самому легче
будет.  Ну, а как Вам поступать, -- естественно, Ваше полное право решать.

С уважением, Александр.

\subsection*{Re: Классификация потоков в технических системах. priven
  2011-04-21} 

Юрий,

Позволю себе еще одно замечание тоже "вбок", но с рекомендацией "в лоб".

Я понимаю, что публикация -- это как живой организм, в котором "что выросло,
то выросло". Но все же, раз Вы уж взялись, ни много ни мало, структурировать
классификацию, хотелось бы увидеть больше "структурности" и в самой статье, и
в ее выводах.

Вот просто список заголовков верхнего уровня, по сушеству -- общий план Вашей
статьи:
\begin{itemize}
\item[1.] Краткая История развития закона
\item[2.] Определения
\item[3.] Возникновение вредных и паразитных потоковв системе
\item[4.] Канал потока как основной инструмент управления потоком
\item[5.] Управление каналом вредных потоков
\item[6.] Полезные потоки
\item[7.] Согласование потоков по другим параметрам
\item[8.] Управляемость полезных потоков
\item[9.] Особенности замкнутых потоков
\item[10.] Поток-носитель и поток-функционал
\item[11.] Особенности потоков вещества, энергии и информации
\end{itemize}
Мне кажется, что здесь просматриваются несколько "линий развития":
\begin{itemize}
\item Классификация самих потоков по нескольким основаниям (полезный --
  вредный -- паразитный ..., основной -- вспомогательный -- дополнительный
  ..., замкнутый -- разомкнутый, "конь" -- "всадник", и т.д.);
\item Классификация содержимого потоков (вещество -- поле, энергия --
  информация, комбинированные потоки);
\item Общая классификация способов управления потоками;
\item Особенности управления разными видами потоков;
\item Инструменты управления потоками и их специфика по отношению к потокам
  разного вида.
\end{itemize}
Не могли бы Вы, пусть и вне самой статьи, а как дополнение к ней (на этой
ветке или на отдельной), подытожить все-таки то, что Вами сделано? Хорошо было
бы собрать выводы по статье не в полутора ничего не значащих общих фразах
(реально выводов в Вашей статье ведь нет совсем!), а в виде конкретных
классификаций, то есть -- списков, схем и т.д., которые бы более-менее легко
запоминались и с помощью которых было бы проще разобраться, что к чему. Это, я
думаю, повысило бы и инструментальность самой статьи.

Опять же, исхожу из заявленной Вами цели:
\begin{quote}
  В данной работе предпринята попытка дать классификацию типов потоком,
  существующих в технических системах.  [...]
  
  Моей задачей не было все переиначить, но просто структурировать анализ,
  сделав его более инструментальным.
\end{quote}
То есть -- классифицировать и структурировать некое (пока что не очень
структурированное и не очень классифицированное) описание. На мой взгляд, эта
цель пока еще ни в коей мере не достигнута, хотя очень похоже на то, что Вы
сильно приблизились к ее достижению. Надеюсь, что Вы сделаете решающий шаг,
который в итоге и превратит Ваше текстуальное описание в заявленную
структурированную классификацию.

Заранее благодарен, Александр.

\subsection*{Re: Классификация потоков в технических системах.  lebedur
  2011-04-22} 

Григорий, спасибо. Теорию ограничений скачал и читаю с большим удовольствием.
Метод Коллера в книге не нашел, но то, что смог на скорую руку увидеть в
Интернете, впечатления пока не произвело. Впрочем, нужно поискать книгу. Еще
раз спасибо. Тем не менее, не вижу и препятствий к использованию потоков в
ТРИЗ. Во всяком случае, в школе GEN3 они используются давно и продуктивно (а
это отнюдь не только сотрудники GEN3).

Мое мнение: если говорить об анализе, то он очевидным образом делится на две
связанные, но не идентичные части: поиск «правильного» недостатка (в ТРИЗ – в
формате противоречия) и его, недостатка, дальнейший анализ уже на предмет
разрешения. По крайней мере, на первой стадии потоковый анализ вполне
пригоден. На решательной стадии – труднее, но в частности поэтому и нужна
более точная классификация.

Александр.  Назвать эту критику мелочами можно только в порядке
самоуничижения. Зато сильно по существу.

Я выбрал те классификационные признаки, с которыми приходилось сталкиваться на
практике (в том числе и задолго до знакомства с ТРИЗ). Можно ли ввести другие?
Безусловно. Будет ли это полезно? Не знаю, не уверен.  Можно ли сделать эту
классификацию лучше? Ну, конечно!

Классификации способов и особенностей управления потоками не вводил
(собирался, но пока ничего не вышло). Даны только наиболее употребительные (на
мой взгляд и вкус) способы и наиболее характерные особенности. В этой части
классификации пока просто нет, наговариваете вы на меня, сэр.

Что же схем, диаграмм и прочих способов \emph{отображения} классификации –
опять согласен, хотелось бы. Но не получилось. Возможно, как раз из-за того,
что классификация получилась очень подробная. Та слишком простая
классификация, которая дана в работе Литвина и Любомирского, как раз
разложилась по полочкам очень легко и изящно. Но тогда теряется множество
существенных признаков, важных на практике. Такое вот противоречие, преодолеть
которое я пока не сумел (а если кто сумеет – буду очень рад).

Что же до информации, то я уже согласился не рассматривать ее (здесь и пока).
Однако ж, воля ваша, но бит и информация – вещи сугубо и принципиально разные.
Бит есть единица измерения информации. Так что на форматированном диске биты
есть, но информации – нет. Если совсем точно – есть, но какая-то совсем
другая! Точно также и с понятием вредной/полезной информации. Разумеется, одна
и та же информация может быть полезной или вредной строго в зависимости от
целей проекта. Вас же не смущает это в отношении понятия «функция»?

Единственно с чем я тут согласен, так это с тем, что хорошо бы оперировать
однозначно дефинированным понятием, понимаемым одинаково. А то так и будем: вы
говорите, что шарик синий, а я – что круглый.

Фил, насчет определения потока – согласен. Получилось несколько не того. В то
же время, могу предложить такую лемму: любое высказывание может быть оспорено
более, чем одним способом, было бы желание. Разумеется, с определениями нужно
быть поаккуратнее. Тем более, если взялся их вводить. Но вот дальше вас унесло
в желание возразить просто ради желания возразить.

«Источник "источает", а не формирует. Формирует поток форсунка или ее аналог.
А источником потока является какой-то накопитель вещества, энергии,
информации»; «Изделие не является компонентом системы. Оно является внешним по
отношению к ТС. Приходит как сырье, уходит как изделие. Но всегда вне самой
системы» -- вы ведь просто дали другие определения. Теперь бы еще показать,
чем они лучше.

«Вот это уже весьма интересно. Как это отсутствующий канал мы будем называть
каналом потока? Отстутствующее называть присутствующим? Оригинально» -- ничего
особенно оригинального. Я дал определение канала и оговорил его возможное
отсутствие. Вам никогда не приходилось описывать явление, которое имеет место
не всегда и не везде?

«Классификация -- это последовательное деление понятия. Правильная
классификация -- это всегда дерево, в котором отдельные листья не
пересекаются.  Нет дерева -- нет классификации» -- опять же, не более, чем
ваше определение понятия классификации с вашими же из него выводами. А вот,
например, Брокгауз и Ефрон дают такое определение: «Классификация --
логический прием, основанный на логическом делении понятия и употребляемый в
эмпирических науках для распределения предметов на роды и виды» -- и ни слова
про дерево. Скучно с вами, не буду я вам больше отвечать. Уж простите, если
сможете.

\subsection*{Re: Классификация потоков в технических системах. priven
  2011-04-22} 

Юрий,

Я бы в данном конкретном случае не стал так уж сразу списывать замечания Фила
со счетов. Особенно про классификацию. Возьмите ЛЮБУЮ мало-мальски известную
классификацию, и увидите, что Фил прав: это ВСЕГДА либо список, либо дерево.
Других работающих классификаций я просто не знаю -- а я, поверьте, с
классификациями работал немало, и не только в естественных науках, и не только
в науке вообще. Так что попробуйте послушаться (в кои-то веки!) умного совета
Фила и все-таки сделать нечто либо "списочное", либо "древообразное". Уверяю
Вас, будет только лучше.

По поводу Вашего ответа на мое замечание. Юрий, я могу легко понять, что
занудное-презанудное занятие составления классификаций не вполне Вам по духу.
Но Вы сами взялись ... А раз взялись -- то буьте добры соответствовать.
Классификация (как и компьютерная программа) только тогда хороша и
инструментальна, когда у нее есть простое и понятное отображение. Нет
графического отображения -- значит, нет и самой классификации, а есть только
некий набор возможных вариантов -- что совсем не одно и то же!

Ведь назначение классификации совершенно не в том, чтобы перечислить несколько
возможных вариантов чего-то там, -- это и безо всякой классификации сделать
можно.  Задача состоит в том, чтобы пользователь классификации, видя СВОЮ
СОБСТВЕННУЮ ситуацию, мог с помощью простой и понятной процедуры "увязать" ее
со списком уже известных вариантов и, соответственно, определить (уже
известные) типовые действия в этой ситуации. У Вас про это немало есть -- но
это все разрозненно и на общую классификацию не сильно тянет. А очень хочется,
чтобы тянуло.

А про информацию -- просто отложим вопрос. Вы все-таки, пожалуйста,
определитесь, какой подход исповедуете -- "технический" (биты и длина текста)
или "семантический" (смыслы и структура образа), тогда и будет о чем говорить.
Смыслы размером текста не измеряются, а вот информационный поток -- измеряется
именно им. Но об этом поговорим позднее и, наверное, безотносительно к этой
статье.

Всяческих Вам успехов!

Александр.

\subsection*{Re: Классификация потоков в технических системах. akyn 2011-04-22}

Привет Юра!

Молодец, удар держишь хорошо.

Что касабельно Фила -- так он и есть Фил. Это не ник, это АНТИдиагноз.

Что касается Колера – не уверен. Мужик умный, по крайней мере не глупее
Голдрата. Если надо пришлю, что есть.

А в целом – ТАК ДЕРЖАТЬ!!!

\subsection*{Re: Классификация потоков в технических системах. Gregory
  Frenklach 2011-04-23}

1. С потоками работает также Лин.  Это тоже "кандидат" стать надсистемой
потокового анализа.

2. Операции Коллера, по моему, могли бы стать очень неплохим дополнением,
поскольку это операции с потоками вещества, энергии, сигналов.  Они
(операции), кстати, больше под синтез "заточены".

А этого-то как раз и не хватает.

\subsection*{Re: Классификация потоков в технических системах. lebedur
  2011-04-23} 

Добрый день, Александры.

Мне тоже нравится ваш стиль ведения разговоров. Жаль только, что нам теперь
специально приходится оговаривать удовольствие от элементарной спокойной
дискуссии вместо скучной кухонной склоки.

Что же до древовидной структуры, то я от нее не отказался, она у меня просто
не получилась. Морфоящик тоже.

У меня получилось 5 классификационных признаков:
\begin{itemize}
\item Основа (su-fi-in),
\item Функциональность,
\item Источник,
\item Комплексность,
\item Замкнутость.
\end{itemize}
Наиболее логичным был бы формат пятимерного ящика, но я не умею его нарисовать.
На дереве же аналогично: каждая ветка имеет ровно такую же аналогичную. И
выглядит все вместе вполне нелепо. Я попробовал по-всякому. Ничего красивого
пока не получилось. Если припомните, я именно на эту тему вам и писал
незадолго до публикации.

Конечно, жаль, что красивая табличка потерялась. Разумеется, нужно продолжать
попытки. Если кто чего путного подскажет – буду рад. Пока утешаюсь тем, что
перед Дарвином был Линней, но перед Линнеем тоже был кто-то, кто составил
описательную картину ситуации. Будем считать, что я пока этот кто-то.

Любопытно следующее: я пробовал приложить оба варианта к реальным проектам.
Благо, у меня сейчас три подряд проекта имеют хорошую потоковую составляющую.
Получилось любопытно: на этапе анализа я пользовался той классификацией,
которую дал сейчас – успешно. На этапе решения удачнее была прежняя простая
классификация, хотя она все равно не слишком хороша.

По другим пунктам:

Мне не кажется, что АРИЗом мало пользуются из-за отсутствия зрительного
графического образа. Там есть системные проблемы. АРИЗ 2010 имени Рубина выдан
в графическом формате древовидного алгоритма. Тем не менее, мне не кажется,
что эта штука будет работать (при всем моем глубоком и искреннем уважении к
Михаилу Семеновичу).

Пример с МАТХЭМом также неубедителен, но уже лично и только для меня. Ну, не
люблю я его и не умею им пользоваться. Это, кстати, лишний раз показывает, что
многочисленность инструментов – благо. Но не в том смысле, что каждый решатель
имеет широкий арсенал, а в том, что каждый решатель может выбрать для себя
что-то из существующего арсенала и переложить на свою полочку.

Lox пишет: «накопитель есть аккумулятор, а не источник».

В целом – да. Но тогда я вынужден считать, например, источником шурупов на
входе в какую-нибудь машину по заворачиванию шурупов, шахту железной руды
где-нибудь сильно далеко от моих интересов.

По самому большому счету аккумулятором можно назвать все, а источником только
и единственно ту, пока учеными не определенную, штуку, которая когда-то очень
давно привела к Большому Взрыву (а я вдобавок, знаю людей, которые убеждены,
что Большой Взрыв – сплошное надувательство, или же вовсе диверсия неких
гадов).

Поскольку же инженеру приходится иметь дело с конкретной системой, причем
рассматриваемой ЗДЕСЬ и СЕЙЧАС, то я считаю источником любой накопитель, в
который ресурс попал не важно откуда и когда. Если для моей задачи важно
откуда и когда – тогда я и называю его буфером (а вы аккумулятором, что никак
не меняет дело).

Акыну: Спасибо на добром слове. Удар держу по-всякому, в зависимости от
настроения. Но когда имело с хамами и/или дураками, что часто одно и то же,
настроение мое обычно портится…

Колера же, пожалуйста, пришли. По рассказам о его методе в инете у меня
сложилось впечатление, что там есть много интересного.

Григорию: Про Лина пока не ничего не знаю, если что есть – пришлите,
пожалуйста. А если есть что полное про теорию Коллера – тоже. Голдратта
(точнее – Детмера) читаю с нарастающим удовольствием. Спасибо.

Правда, синтез – это не то, о чем мы сейчас говорим. Я как-то думал, что
написал про анализ. Если же кто сумеет взять из моего опуса нечто полезное для
синтеза – буду очень рад, но прошу со мной поделиться…

Удачи всем нам, и, по примеру Акына, всем полная каса хамнида.

Ю. Лебедев

\subsection*{Re: Классификация потоков в технических системах.  Gregory
  Frenklach 2011-04-23}

Что такое лин посмотрите тут.  Потом сделайте поиск на интернете.

По Коллеру у меня только "бумажки", но если сделать поиск, используя ключевые
слова, можно кое что найти и на интернете.  В том числе и на сайте
Методолог: \url{http://www.metodolog.ru/instruments.html#KOLER}

P.S. Lean
\begin{itemize}
\item \url{http://www.metodolog.ru/00906/00906.html}
\item \url{http://www.metodolog.ru/01123/01123.html}
\end{itemize}

\subsection*{Re: Классификация потоков в технических системах.  Александр
  Кудрявцев 2011-04-23}

Юрий, только что вернулся из командировки, не имел возможности написать по
поводу "потоковой" работы. Надеюсь в ближайшие дни наверстать.

Пока -- выдержка из письма, полученного сегодня: "Про "потоки" было полезно.
Мы столкнулись ... с проблемкой, так она почти без изменений легла на текст.
При случае поблагодари от моего имени автора.

Пока проблему не решили, но поиск зоны быстро проявился. Наверное, и сами бы
смогли... Да ленивые мы. А тут кстати и статья САМА в руки, да споры вокруг
неё".

Всего доброго,

\subsection*{Re: Классификация потоков в технических системах.  priven
  2011-04-23}

\begin{quote}
  lebedur wrote: У меня получилось 5 классификационных признаков: ...

  Наиболее логичным был бы формат пятимерного ящика, но я не умею его
  нарисовать.
\end{quote}
1. А что такое в данном случае "in"? Дюймы, что ли?

2. В "пятерку" пока что признаки не сильно ложатся. Я знаю три вида
"пятерочных" классификаций -- но здесь не вижу ни одного из них. Похоже, надо
как-то дробить. Скажем, по формальному (грамматическому) признаку: "основа --
источник" и "функциональность -- комплексность -- замкнутость". Но не худо бы
сперва дать всем пятерым определения. Особенно понятию "комплексность". А там,
глядишь, что цельное и вырисуется.

В общем и целом, классификации бывают, как минимум, двух разных видов, и
составлять любую из них -- не слишком простая и часто не очень благодарная
работа. Тем не менее, если Вы знаете какие-то классификации, отличающиеся от
дерева и морфоящика, то расскажите, пожалуйта, какие именно и как они
выглядят, а если нет -- то придется все-таки делать что-то из этих двух...

С наилучшими пожеланиями, всё на свете парамнида и за всё вихаё,

Александр.
\end{document}

    Log in or register to post comments

Re: Классификация потоков в технических системах

Submitted by priven on чт, 28/04/2011 - 13:43

Господа, не надо путать РАЗНЫЕ модели.

Андрей рассуждает в рамках ФИЗИЧЕСКОЙ модели переноса энергии. Эта модель описывает ФИЗИЧЕСКУЮ систему. Там никаких "двигателей" и "трансмиссий" нет вообще: там есть источник энергии, среда - проводник энергии (в частности, это может быть и вакуум - без разницы) и приемник энергии.

АКын рассуждает в рамках ТРИЗОВСКОЙ ("функцциональной2) модели, которая описывает ТЕХНИЧЕСКУЮ систему, то есть - систему, специально созданную для удовлетворения человеческой потребности с использованием тех или иных природных эффектов, связанных с переносом энергии (применение ТРИЗ к информационным системам - все еще предмет дискуссий). А в этой ТЕХНИЧЕСКОЙ ЭНЕРГОПРЕОБРАЗУЮЩЕЙ системе есть и двигатель, и трансмиссия, и инструмент, и все прочее, о чем говорит ТРИЗ. "Среда, передающая энергию", в этой модели ровно такой же нонсенс и абсурд, как и "инструмент" или "двигатель" - в модели физической.

А в модели ПОЛЬЗОВАТЕЛЬСКОЙ телевизор, к примеру, состоит из экрана, вилки питания, антенны, пульта и программы передач - больше НИЧЕГО в этой модели нет, и там совершенно наплевать, какие физэффекты в этом телевизоре используются и используются ли они там вообще.

Как только мы определяемся с конкретной моделью, сразу же становятся яснее ответы на многие вопросы. Между прочим, не только в ТРИЗ.

С уважением,

Александр.

    Log in or register to post comments

Re: Классификация потоков в технических системах

Submitted by akyn on чт, 28/04/2011 - 14:00
Изображение пользователя akyn.

    priven wrote:
    Господа, не надо путать РАЗНЫЕ модели.
    Как только мы определяемся с конкретной моделью, сразу же становятся яснее ответы на многие вопросы. Между прочим, не только в ТРИЗ.


+100!

    Log in or register to post comments

Re: Классификация потоков в технических системах

Submitted by priven on чт, 28/04/2011 - 18:00

Не знаю, какие существуют методы демагогии. Но знаю (поскольку занимался этим весьма предметно), что в так называемых "описательных дисциплинах" (к числу которых пока что с очевидностью принадлежит ТРИЗ) с помощью математики можно очень легко сжульничать, и крайне сложно сделать что-то реально полезное.

Чтобы стало наоборот, должны появиться, как минимум, количественные законы - если и не аналоги законов Ньютона, Ома, Кулона, Кирхгоффа и т.д., то хотя бы аналоги Периодического закона Менделеева, автор которого приближенно предсказывал атомные массы еще не открытых элементов. Где в ТРИЗе есть что-то подобное? Ау... Нет ответа... не дает ответа...

Или все-таки дает? Тогда - что это за ответ, и на какой именно вопрос? Запросто могу чего-то не знать - буду благодарен за полезную информацию.

    Log in or register to post comments

Re: Классификация потоков в технических системах

Submitted by Александр Кудрявцев on чт, 28/04/2011 - 18:33

    priven wrote:

    Чтобы стало наоборот, должны появиться, как минимум, количественные законы - если и не аналоги законов Ньютона, Ома, Кулона, Кирхгоффа и т.д., то хотя бы аналоги Периодического закона Менделеева, автор которого приближенно предсказывал атомные массы еще не открытых элементов. Где в ТРИЗе есть что-то подобное? Ау... Нет ответа... не дает ответа...


Увы, вынужден согласиться. Количественных, точных законов у нас нет. В нашем понимании "закон" - это нечто "социологическое". (Если увеличить зарплату женщинам, то количество матерей - одиночек вырастет...) Из такой конструкции государство многое могло бы использовать, но вся беда в том, что применяем мы эти законы на уровне конкретных советов конкретному человеку. Что делает наши построения конечно же шаткими и вероятностными.

    Log in or register to post comments

Re: Классификация потоков в технических системах

Submitted by priven on пт, 29/04/2011 - 02:47

Александр Владимирович, а почему же "увы"? Наоборот, это же здорово! С одной стороны, это показывает, что и БЕЗ количественных законов МОЖНО решать вполне себе практические и в том числе инженерные задачи - что для тех, кто занимается точными науками, отнюдь не есть очевидный факт. А с другой стороны - как много у нас еще впереди...

Андрей ведь, по большому счету, прав. Маркс как-то сказал: "Наука лишь тогда достигнет совершенства, когда она обогатится математикой" (запомнил, потому что на входе в школу метровыми буквами висело) - наверное, не он первый сказал что-то подобное. Просто, я считаю, не нужно забегать вперед и пытаться "натянуть" математику на не подготовленную еще для нее тризовскую "почву", а нужно эту самую "почву" готовить, то бишь - открывать те самые количественные законы, не "социологические", а хотя бы в духе упомянутого профессора Менделеева или его полу-однофамильца монаха Менделя. Вот тогда и можно будет проверять правильность чего-либо математикой.

Осталось-то до этого прекрасного момента, я считаю, всего ничего! Ну, месяц, квартал, ну, хорошо, год (это с хорошим запасом, я думаю), - но никак не десятилетия.

Я, разумеется, не к тому, что, мол, "ну вот, осталось чуток потерпеть - и каааак зааааживем!!!", - а к тому, что таков естественный ход событий, через это прошли многие дисциплины, пройдет и ТРИЗ. Предпосылки-то все налицо, если разобраться, и количественные прогнозы развития техники, особенно для тех, кто этим занимается не первый год (в отличие от Вашего покорного слуги), вещь, по сути, давно уже освоенная. Просто наступает пора "подобрать хвосты" и, обобщив то, что нам уже и так известно, сделать следующий шаг - аналогично тому, как его сделали упомянутые господа в химии и биологии.

Всех проблем, разумеется, это не решит - но может дать новые, более эффективные инструменты для их решения. В частности, я ожидаю в самое ближайшее время серьезных изменений в концепции Идеального Конечного Результата: пока что мы видим там только "конец пути" и "то, что перед самым концом", но хорошо бы видеть еще и "ближайшую остановку", и "маршрут", не правда ли? А где "маршрут" - там и "карта местности", а это уже что-то вполне "измеримое", пусть точные "карты", уровня GSM-навигации, появятся и не сразу. Это только один из примеров, где "переход качества в количество" напрашивается, на мой взгляд, сам собой. А там, глядишь, и прочая математика подоспеет.

Я думаю, что здесь надо, не торопя события (как Фил, например), последовательно, шаг за шагом развивать наши представления о законах развития техники, приближаясь постепенно к той самой строгой математике, о которой говорит Андрей. От судьбы все равно не убежать - а "судьбу" в ТРИЗ я вижу именно там. Собственно, мы с АКыном так прямо и написали - и по ЭТОМУ поводу нам никто вроде бы и не возразил... и в самом деле - разве есть на этот счет сомнения?

    Log in or register to post comments

Re: Классификация потоков в технических системах

Submitted by Александр Кудрявцев on пт, 29/04/2011 - 07:54

    priven wrote:
    Александр Владимирович, а почему же "увы"? ...

    Андрей ведь, по большому счету, прав. 

"Увы" потому, что Андрей по большому счету прав.

    Quote:
    Осталось-то до этого прекрасного момента, я считаю, всего ничего! Ну, месяц, квартал, ну, хорошо, год (это с хорошим запасом, я думаю), - но никак не десятилетия.

Возможно, что такой срок остался до появления какой-нибудь работы, в которой даются предложения по математизации процесса. Но отнюдь не до момента широкого использования этой математики. Потому что математика без глубокого осознания - пустышка, чему Вы ранее привели множество примеров. А нового, глубокого представления, в корне меняющего картинку, следует ждать все же через какой-то Гегельсовский путь - сбора и учета всего накопленного опыта. Или, хотя бы проверки и адаптации выдвинутой теории кэтому самому опыту. Путь не скорый, так как накоплено фактуры немало.

    Quote:
    Я, разумеется, не к тому, что, мол, "ну вот, осталось чуток потерпеть - и каааак зааааживем!!!", - а к тому, что таков естественный ход событий, через это прошли многие дисциплины, пройдет и ТРИЗ. 


Да вроде и сейчас как-то живем. Дело не в этом, не в наличии - отсутствии математики саомй по себе. Дело в недостаточном уровне понимания предмета и плацдарма, на котором он действует. Дело в том, что плацдарм этот усложняется уже буквально на глазах, а мы отвечаем на это чрезвычайно медленной внутренней перестройкой.

    Quote:
    Предпосылки-то все налицо, если разобраться, и количественные прогнозы развития техники, особенно для тех, кто этим занимается не первый год (в отличие от Вашего покорного слуги), вещь, по сути, давно уже освоенная. Просто наступает пора "подобрать хвосты" и, обобщив то, что нам уже и так известно, сделать следующий шаг - аналогично тому, как его сделали упомянутые господа в химии и биологии. 

Прогнозы количественные пока в стадии лабораторных экспериментов. Дело чрезвычайно нужное, но в реальные проекты пока численные результаты вроде бы не вставляются - рановато чутка. Да и не сводится наш предмет только к прогнозированию.

    Quote:
    Всех проблем, разумеется, это не решит - но может дать новые, более эффективные инструменты для их решения. В частности, я ожидаю в самое ближайшее время серьезных изменений в концепции Идеального Конечного Результата: пока что мы видим там только "конец пути" и "то, что перед самым концом", но хорошо бы видеть еще и "ближайшую остановку", и "маршрут", не правда ли? А где "маршрут" - там и "карта местности", а это уже что-то вполне "измеримое", пусть точные "карты", уровня GSM-навигации, появятся и не сразу. Это только один из примеров, где "переход качества в количество" напрашивается, на мой взгляд, сам собой. А там, глядишь, и прочая математика подоспеет. 

Сценариев развития действительно может быть много. Любопытно было бы обменяться представлениями на этот счет.

    Quote:
    Я думаю, что здесь надо, не торопя события (как Фил, например), последовательно, шаг за шагом развивать наши представления о законах развития техники, приближаясь постепенно к той самой строгой математике, о которой говорит Андрей.

Конечно, здесь у меня акцент чуть иной - приближаться надо не к строгой математике, а к полному понимаю. Запишем его в виде формулы - отлично. Осознаем его как объединяющий морально- нравственный императив, тоже прекрасно.

    Quote:
    От судьбы все равно не убежать - а "судьбу" в ТРИЗ я вижу именно там.

Но это не значит, что она именно такова - не так ли? (Это я просто к тому, чтобы не превращаться сейчас в фаталистов, а видеть все открывающиеся перспективы.)

    Log in or register to post comments

Re: Классификация потоков в технических системах

Submitted by priven on пт, 29/04/2011 - 19:29

Соглашаюсь с Александром Владимировичем почти по всем пунктам, за исключением буквально пары моментов.

Что касается математики - вот сегодня как раз статью Бушуева вывесили, очень интересно, надо будет смотреть и разбираться. Надеюсь, появятся и другие статьи. Но я малость не это имел в виду - а именно то, что не требует больших интеллектуальных усилий для применения. Как закон Менделя: скрещиваем желтый горох с зеленым и получаем заданную пропорцию в потомстве, думать не нужно совершенно - бери себе и применяй.

С пониманием чуть сложнее. Менделеев ведь совершенно не понимал, почему свойства элементов меняются периодически, а, тем не менее, закон открыл. Кулон тоже не понимал, почему в его формуле именно квадрат расстояния, но тоже закон открыл. Я так думаю, что сначала надо эмпирически найти эти самые законы (которые можно применять не только на больших массивах, но и для конкретных систем - как Менделеев для конкретных, еще не открытых элементов), а потом уже пытаться их понять. Наоборот можно идти долго, нудно и - зачастую безрезультатно...

А что касаемо судьбы, то я ведь специально оговорился, что "я вижу" ее там - что уже само по себе предполагает возможность и иного вИдения, согласитесь. На абсолютную истину я никода не претендовал - и здесь тем более не пытаюсь это делать.

    Log in or register to post comments

Re: Классификация потоков в технических системах

Submitted by Gregory Frenklach on сб, 30/04/2011 - 14:28
Изображение пользователя Gregory Frenklach.

Математика начинается при переходе от "технического" решения к "расчётному" и это (по-моему) в любой области.
Но мы (ещё) "не там"

    Log in or register to post comments

Re: Классификация потоков в технических системах

Submitted by lebedur on сб, 30/04/2011 - 17:10

Добрый день, коллеги, надеюсь не помешал?
Раз уж никому неинтересно про потоки, поговорим, пожалуй, про математику.
В получившейся дискуссии, на мой взгляд, все правы понемножку. Как оно чаще всего и бывает. Хочу сразу оговориться, что высказываю только свою точку зрения и на своей безусловной правоте не настаиваю. Тема поднята важная, поэтому получилось длинно, прошу прощения.
1. Мне кажется, неверным сравнивать ТРИЗ с естественными науками и делать на этом основании вывод, что ТРИЗ это такой гадкий утенок, хороший, но малость недоразвитый. ТРИЗ - это просто о другом. Естественные науки трактуют нам о том, как устроена техническая система, ТРИЗ о том, что нам с этим делать. Хорошее внятное описание ТС, безусловно, нужно. Если есть возможность получить хорошую подробную математическую модель – совсем хорошо. Но вот настаивать на математизации ТРИЗ или хотя бы строгости ее выводов – неправильно. Хотя бы уже потому, что задача «что с этим делать» может ставиться по-разному от проекта к проекту, меняться во времени и даже меняться в течение проекта.
Поэтому и законы наши НЕ МОГУТ БЫТЬ точными математическими. Не потому, что мы не можем их так сформулировать, а просто потому, что они о другом. Об этом, кстати, писал еще Альтшуллер, предупреждая, что и законы будут меняться (не формулировки, а по существу), и уж приемы/стандарты тем более. Поэтому, кстати, я считаю метод контрольных ответов на учебные задачи принципиально неверным: наша методика не приспособлена для этого. А не приспособлена именно потому, что не для этого предназначена.
Разумеется, все правы в том, что точное численное описание ТС необходимо. Всем, наверное, случалось слышать что-то вроде «зачем нам численные значения, мы решаем задачу по ТРИЗ». Глупость, на мой взгляд, несусветная. Но точные значения и математический аппарат для их анализа нужны в рамках естественнонаучного описания. Опять-таки, все решатели наверняка сталкивались с ситуацией, когда недостаточный учет численных значений приводил к неработоспособным решениям, а то и вовсе нелепым. Байек на эту тему тоже ходит множество. Педагоги, те, которые обучают уже толковых инженеров, а не студентов, также обычно не пренебрегают такими примерами. Но всюду здесь разговор о необходимости численных данных и сопутствующей математике именно в естественнонаучном смысле.
2. Если вернуться к моим любимым потокам, несть числа примеров, когда потоковый анализ выполнен без наполнения его числами. Но и цена такому анализу – гнутый грошик в базарный день. Потоковый анализ придуман ведь не ТРИЗовцами, он есть в той или иной форме в большинстве естественных наук, и особенно - прикладных. Собственно говоря, любая «чего-то динамика» - это и есть анализ потоков «чего-то». Если бы я затеял писать общую теорию потоков, я, разумеется, написал бы систему уравнений. Как минимум в ней были бы
 Уравнение движения
 Уравнение непрерывности для непрерывных потоков (аналог законов Кирхгофа из курса школьной физики)
 Условие непрерывности (требование статики)
 Уравнение потерь на сопротивление среды (аналог закона Ома)
 Уравнение диссипации потока (аналог закона Ватта)
 Наверное, что-то еще.
Но я-то писал не об этом! И все остальные авторы ТРИЗовских разработок пишут не об этом. Поэтому я и не стал рисовать графическую структуру своей классификации, за что меня долго ругали. Но она здесь и не нужна.
Как я этим пользуюсь:
Я ввел четыре новых классификационных признака (вредные/полезные были и раньше).
Теперь я просто вставляю в таблицу потоков потоковой модели четыре новых столбика, соответствующие этим четырем признакам. И заполняю их. Заполняются они легко, результат получается наглядный, удобный и полезный. За последние полгода я проверил это дело на четырех старых проектах и четырех новых – действительно, удобно. По крайней мере, мне.
3. Что же собственно математики, то мне уже приходилось приводить этот пример, приведу еще раз: «Всякое тело продолжает удерживаться в состоянии покоя или равномерного и прямолинейного движения, пока и поскольку оно не понуждается приложенными силами изменить это состояние» - и ни одного математического символа и вообще ни звука про математику. А сколько разнообразных ВЫВОДОВ!
Аналогично закон сохранения энергии и некоторые другие. Математика появляется все-таки чуть позже, там, когда и где она нужна.
Точно также математика появляется и в ТРИЗ, но не в том смысле, что чуть позже, а именно в том, что когда и где нужно. Например, правила тримминга из ФМ полностью формализованы. На мой взгляд, даже несколько чересчур: там есть некоторые очень даже не всеобщие моменты, которые, тем не менее, зашиты в Техоптимайзер полностью и жестко. Вообще все методы, где мы что-то сравниваем (чаще ранжируем) – формализованы ровно в той мере, в какой это имеет смысл.
В моем примере с потоками получившуюся таблицу вполне можно назвать матрицей и обрабатывать ее сколько душе угодно – просто незачем. ЕЩЕ РАЗ: НЕ ТО, ЧТОБ НЕ УМЕЕМ, А НЕЗАЧЕМ.
Еще пример. В прекрасной книге Петрова о ЗРТС в главе о законе повышения согласованности дана подробная, детальная классификация, нарисованы деревья и все такое. Просто ждешь появления формул. Но формулы не появляются. И не потому, что плох закон или неумел автор. Просто – о другом. Поэтому там появляются «линии эволюции», что имеет отношение не к выводам о том, что с чем рассогласовано, а к тому, что нам с этим делать. А делать каждый намерен что-то свое. В конце концов: 5 мм – это мало или много? Правильно, смотря для чего.

    Log in or register to post comments

Re: Классификация потоков в технических системах

Submitted by priven on сб, 30/04/2011 - 19:34

Юрий,

Не могу не удержаться, чтобы не вставить свои пять копеек.

Конечно, Вы правы: математика, введенная преждевременно, только портит и ничему не помогает. Если мы можем только определять ранги - то бессмысленно рассчитывать точные значения. Если мы можем только составлять текстовое описание - то бессмысленно расставлять ранги. Если у нас нет твердых фактов - то бессмысленно их описывать. Если у нас нет обратной связи - то бессмысленно писать дифуры, определяющие зависимость параметров от времени. И так далее.

А вот в части направления движения - я все же считаю, что "мы пойдем своим путем" не самая лучшая позиция, и если множество научных дисциплин прошли очень похожий путь, то нет особых оснований полагать, что ТРИЗ сия чаша минует, - если, конечно, мы не хотим превратиться в религиозную секту. Поэтому в направлении "математизации" все равно двигаться придется, хотим мы того или не очень. Если не хотим - то просто уступим свое место тем, кто хочет, и про ТРИЗ все забудут, как про сладкий (или кошмарный - на выбор) сон.

"Наша наука не про то" - это, простите, от лукавого. Дело не в том, про что наука, а в том, что мы в ней знаем и умеем. Как только будем знать и уметь что-то реальное считать - сразу же наука станет "про то". Численные законы в ТРИЗ еще только предстоит открыть - но из этого совершенно не следует, что этим не стоит заниматься. Как мне кажется, здесь как раз мы слишком консервативны, и в ближайшие месяц-год-два ТРИЗ моут ожидать в этом плане очень серьезные изменения. Не захотим меняться - значит, уступим свое место тем, кто захочет, только и всего. (Разумеется, на абсолютную истину здесь не претендую.)

Поэтому давайте не говорить о том, что "наша наука" "про то" или "не про то", а исходить из конкретной ситуации, которая может (в принципе) ОЧЕНЬ быстро измениться, сделав возможным и даже необходимым то, что еще вчера считалось фантастическим. И чем быстрее мы это поймем - тем, по-моему, будет нам лучше.

    Log in or register to post comments

Re: Классификация потоков в технических системах

Submitted by lebedur on вс, 01/05/2011 - 06:04

Добрый день, Александр.
Про «не могу не удержаться» - это вы на себя наговариваете. Как видим, вполне можете. Впрочем, получается это у вас здорово.
Если же всерьез, то, боюсь, вы меня неправильно поняли. Впрочем, это означает всего лишь, что я плохо высказался.
Вы были бы абсолютно правы, если бы ТРИЗ была естественной наукой в ряду других, но со своим предметом. Тогда, разумеется, неточность ТРИЗ была бы ее недостатком. И оставалось бы либо этот недостаток устранять, либо признать ТРИЗ принципиально бесперспективной.
Но, на мой взгляд, ТРИЗ есть не наука в общепринятом смысле слова, а метод. В ряду других, например, 6S, VE, НАССP, бережливое производство, тот же мозговой штурм и прочие. Да, я считаю метод ТРИЗ более перспективном и продвинутым в этом ряду (по крайней мере, из известных мне). Но это никак не наука в общепринятом смысле. ТРИЗ не дает нам новых знаний о предмете (где предмет – ТС), но позволяет их по-другому систематизировать и найти неочевидные пути решения задач. В этом смысле ТРИЗ имеет, скорее, отношение к теории систем.
В этой части формализация, разумеется, нужна и, более того, имеет место и развивается.
Возьмите, например, весьма формализованные процедуры ФСА (я предпочитаю в формате ФМ), бенчмаркинг, потоковую модель, MPV. Я как раз и пытался в своей работе дать дополнительные критерии для потоковой модели.
В этой части не могу согласиться с парой частностей:
«Если мы можем только определять ранги - то бессмысленно рассчитывать точные значения» - но все ровно наоборот. Мы ранжируем объекты (компоненты, системы, функции) исходя из значений, желательно - как можно более точных. Чем меньше эта точность, тем хуже ранжирование.
«Если мы можем только составлять текстовое описание - то бессмысленно расставлять ранги» - очень даже осмысленно. А что прикажете делать, если система есть, а численных параметров нет? Отказываться делать проект в ожидании развития теории? Так ведь начальство заругает. Но, кстати, довольно часто удается ранжировать и так. Более того, придуман даже весьма изысканный метод, позволяющий ответить на старинный вопрос: что все-таки больше, метр или килограмм? Правда, только в отношении ДАННОЙ ТС, рассматриваемой ЗДЕСЬ и СЕЙЧАС в рамках ЭТОЙ ЗАДАЧИ. Метод называется MPV, работает не как часы, требует изворотливости ума и все такое, но все же…
«Если у нас нет твердых фактов - то бессмысленно их описывать» - если твердые факты есть, то описывать их уже не нужно. Если же их нет, остается только и именно описывать. «С одной стороны…, но с другой стороны… а вот если…». Опять же, а что остается?
«Если у нас нет обратной связи - то бессмысленно писать дифуры, определяющие зависимость параметров от времени» - оно конечно так, хотя зависимость параметров от времени и не всегда определяется обратной связью.
Можно ли считать существующий уровень формализации достаточным? Да ни боже упаси. НО:
- во всех перечисленных (Вами!) ситуациях речь идет о недостатках естественнонаучного описания ТС. Как здесь может помочь формализация ТРИЗ?
- если есть точные знания и точная математическая модель ТС, полученные естественнонаучными дисциплинами, тогда нас уже не зовут, решают задачу сами (такие вот злые нехорошие у нас заказчики – как что попроще, так норовят сами).
- процесс формализации этой работы в ТРИЗ идет и довольно бодро. Литвин, например, любит повторять, что если раньше соотношение формального и творческого в нашей работе было где-то 30:70, то теперь 70:30 и стремительно движется к 80:20. Я с ним в этом, правда, не совсем согласен, но вот, например, MPV появился уже во время моей, не слишком долгой, работы в ТРИЗ. Я даже принимал некоторое участие в его создании. Конечно, не считаю себя участником Бородинского сражения, но я все же был в числе тех, кто старательно подносил Вербицкому патроны. А один раз мне даже дали пострелять…
- говоря о математизации ТРИЗ вы все же, мне кажется, имеете ввиду другое. Скорее, как я мог понять, систему ЗРТС. Согласен с тем, что она более чем несовершенна. Прежде всего, ЗРТС есть, и даже несколько версий. А вот СИСТЕМЫ нет. Еще точнее, она везде, с самой первой версии Альтшуллера, намечена, но не более того. Наиболее полно системный фактор просматривается в версии Фила. Там, правда, на мой взгляд, другие проблемы…
- так что впереди большая работа. Но вот тут наши подходы различаются кардинально. Я считаю, что не следует пытаться описывать все системы единообразным формализованным вплоть до математического аппарата образом. Резон простой: любое обобщение вынужденно пренебрегает частностями. А дьявол-то как раз и прячется в мелочах. Возможно, с точки зрения игры разума – это интересная задача. Но я практик. Моя задача – изгонять того самого дьявола. Поэтому мне интереснее такое развитие, которое лезет в те самые мелочи. Впрочем, пусть и далее расцветает тысяча роз.
PS: фразой про "в ближайшие месяц-год-два" заинтриговали уже дальше некуда. С нетерпением жду (все остальные, по-моему, тоже) :-)
с уважением, Ю. Лебедев

    Log in or register to post comments

Re: Классификация потоков в технических системах

Submitted by akyn on вс, 01/05/2011 - 07:58
Изображение пользователя akyn.

Привет, Юрий!

Позволю себе не согласиться. К Иловайского (увы, покойного) была хорошая идея рассматривать ТРИЗ, как Инженерное Творчество. Это не наука, но научная дисциплина, типа "Процессы и аппараты".
Естественно, что за мои идеи Фил ответственности не несет.

С уважением,
Ваш akyn - Кынин (в переводе: с татарского - ножны для сабли, с китайского - "преодаление трудностей", с корейского - воин).

    Log in or register to post comments

Re: Классификация потоков в технических системах

Submitted by GIP on вс, 01/05/2011 - 13:28
Изображение пользователя GIP.

    lebedur wrote:

    Будем называть потоком такое перемещение материальных объектов, энергии или информации в системе, при котором отдельные части потока перемещаются по одному и тому же закону одни за другими (частично поток может перемещаться в надсистеме, но ключевым является его наличие и перемещение в рассматриваемой системе),


Если перемещение материальных объектов (МО) - направленное? - назвать потоком, то это вызывает ассоциацию с некоторой траекторией, следом в пространстве, который оставляет (различной плотности в направлении протяженности) череда перемещающихся МО.
Но что тогда есть перемещение энергии, в смысле - чего она череда? Аналогично, чего череда информации?
Другими словами, что перемещается потоком в случае энергии, а также - в случае информации?

Осознание

==ИИ-->

    Log in or register to post comments

Re: Классификация потоков в технических системах

Submitted by Ромащук Александр on вс, 01/05/2011 - 14:10

Уважаемый, Ю.Лебедев!
Мне очень близка Ваша идея о методической, а не научно-познавательной направленности ТРИЗ:

    lebedur wrote:
    Но, на мой взгляд, ТРИЗ есть не наука в общепринятом смысле слова, а метод. В ряду других, например, 6S, VE, НАССP, бережливое производство, тот же мозговой штурм и прочие. Да, я считаю метод ТРИЗ более перспективном и продвинутым в этом ряду (по крайней мере, из известных мне). Но это никак не наука в общепринятом смысле. ТРИЗ не дает нам новых знаний о предмете (где предмет – ТС), но позволяет их по-другому систематизировать и найти неочевидные пути решения задач. 


И в этом плане у меня методический вопрос по материалам Вашей статьи: Вы предполагаете, что представленная Вами классификация является методическим средством решения задач или она включает в себя классификацию этих средств (т.е. каждое из описаний видов поток включает метод по решению задач, содержащих подобный вид потока)? У меня в этом плане чисто прагматический вопрос: содержит ли статься "инструкцию по применению" или это пока общее описание базовой схемы (т.е. ближе к научному анализу)? Просто не смотря на большую насыщенность примеров у меня сходу не получилось воспользоваться данной классификацией более чем в распознавательных целях (да и в последних я сильно запутался, поскольку не понял в какой очередности "подставлять" виды потоков и т.п.). Ответ мне нужен чтобы понять насколько это мои собственные проблемы (связанный еще и с тем, что я не совсем чисто ТС пытался смотреть) и насколько это запланированный для статьи результат.
С уважением, Александр

    Log in or register to post comments

Re: Классификация потоков в технических системах

Submitted by lebedur on вс, 01/05/2011 - 16:02

Геннадий Иванович, добрый день.
Разумеется, перемещаются всегда только материальные объекты. Я помню, что энергия – всего лишь «мера различных форм движения материи», а информация и вовсе – «свойство материальных объектов и явлений …» (далее по Википедии). Тем не менее, понятие потоков энергии и информации очень удобно для описания разного рода моделей и применяется весьма широко (дать исчерпывающие, академически точные определения не могу).
И именно потому, что ни энергия, ни информация никогда не существуют сами по себе в отрыве от МО, я и ввел понятие потоков-носителей и потоков-функционалов.
Простой пример. Разумеется, в компьютере происходит перемещение только электронов. Когда в молодости я занимался физикой полупроводников, я именно этот аспект и рассматривал. Но уже при переходе к простым элементам гораздо удобнее оперировать понятием электрического тока. А в сколько-нибудь сложных схемах – потоков информации. Разумеется, информация при этом переносится током, а он в свою очередь – поток электронов, кто бы спорил.
Александр, добрый день и вам.
Я вовсе не противник «научно-познавательной направленности ТРИЗ», просто мне интересней методическая сторона дела. Попробую ответить на ваш вопрос, но очень коротко не получится, уж не обессудьте.
Я написал работу в развитие потокового анализа. Здесь меня опять подвел бэкграунд: бóльшую часть своей ТРИЗовской биографии я провел в GEN3, где понятие потокового анализа общеизвестно и общеупотребительно. Я все время забываю, что очень многие в ТРИЗ этого понятия не знают; многие из тех, кто знает, не пользуются; некоторые из тех, кто пользуется, не считают это ТРИЗом – еще раз прошу прощения.
Итак, эта работа является попыткой усовершенствовать имеющийся потоковый анализ. Прочитать о нем можно в диссертации Олега Герасимова http://www.triz-summit.ru/ru/section.php?docId=4639. Там сведено вместе очень много всего, зато про каждый отдельный инструмент – достаточно компактно. Наверное, можно где-нибудь еще.
При этом я дополняю потоковую модель, данную в виде графа, таблицей потоков (аналогично двум способам представления ФМ) – так мне удобнее.
Что я сделал в этой работе.
1. Я ввел четыре новых классификационных признака:
По функциональности
Главный
Вспомогательный
По источнику
Первичный
Вторичный
Паразитный (только для вредных)
«Конь/Всадник»
Функционал
Носитель
По открытости
Открытый
Замкнутый

В таблицу потоковой модели я теперь ввожу четыре дополнительных столбца (полезность/вредность выделена и сейчас) и заполняю их соответствующей характеристикой – это несложно.
В результате у меня получается более развернутый диагноз (что-то вроде «полезный, вспомогательный, вторичный замкнутый поток-функционал). Все-таки, это гораздо более точный диагноз, чем просто «полезный». Конечно, в старину лекари делили все болезни на два вида: трясучка и лихоманка – но чем полнее диагноз, тем успешнее лечение (в среднем на круг). Разумеется, этот диагноз имеет смысл только при учете параметров потока.
2. В этой же работе я назвал чисто назывным порядком основные или наиболее часто применяемые методы лечения. Обоснования давал только там, где смог. Кое-что просто оставил за бортом как почти очевидное – иначе статья могла бы вылиться в книгу, чего я никак не хотел.
Более полный список приемов, но вовсе не структурированный, дан в работе Литвина и Любомирского http://www.metodolog.ru/00822/00822.html. Несколько месяцев назад я попробовал несколько структурировать этот список http://www.metodolog.ru/node/850 - получилось, правда, средненько. Мой список приемов (методов, субтрендов – кому как нравится) неполон, но каждый может его дополнить. Особенно, если специализируется в конкретной области.
Попробовал ответить насколько смог. Спрашивайте, с удовольствием отвечу. Но, чтобы не заполнять эфир вопросами, которые не очень интересны другим, можно спрашивать и по почте lebedur@yandex.ru
Ю. Лебедев

    Log in or register to post comments

Re: Классификация потоков в технических системах

Submitted by Александр Кудрявцев on вс, 01/05/2011 - 17:52

Как это, "не очень интересны другим"?
Очень интересны.
Помните, у Галича - "А из зала мне кричат: "Давай подробности!"

    Log in or register to post comments

Re: Классификация потоков в технических системах

Submitted by priven on вс, 01/05/2011 - 20:01

Здравствуйте, Юрий.

Я думаю, Ваши возражения обусловлены, скорее всего, непониманием. Ситуация в точности аналогична Вашей: я недостаточно понятно сформулировал и не был понят. Только так и могу объяснить эти самые возражения. Приношу извинения.

Наверное, сказывается тот факт, что я "обсчитвывал" десяток или полтора всяких кандидатских и докторских диссертаций по гуманитарным наукам, и многие вещи мне кажутся "самоочевидными", тогда как для обычного инженера они таковыми вовсе не являются. А ТРИЗ - как ни крути - дисциплина пока что сугубо описательная.

Когда я говорил про ранги, то имел в виду так называемую "ранговую шкалу" - понятие, очень хорошо известное в гуманитарных (большей частью описательных) дисциплинах, но редко используемое в технических. Это такая шкала, в которой для любых двух объектов можно записать одно из трех соотношений: "точно больше", "точно меньше", "примерно равно", но насколько больше или меньше - неизвестно. По сути, это и есть то самое, о чем Вы пишете, говоря про MVP - опять же, это метод, который в гуманитарных дисциплинах абсолютно "затасканный" и общеизвестный. В этой ситуации можно, конечно, приписать объектам некие числа (и это и делают), но с этими числами нельзя производить такие операции, как ,скажем, сложение или умножение - их можно, максимум, вычитать друг из друга. Никаких точных значений эти числа не дают и не могут дать по определению, сколь бы точно Вы их ни "подсчитывали". Можно попросить сто миллионов экспертов поставить баллы от 0 до миллиона, скажем, качеству мобильных телефонов маркок А, В и С, но, какую бы номинальную точность Вы ни получили в результате усреднения данных, Вы все равно не сможете сказать, во сколько раз один из них лучше другого: операция деления для таких результатов принципиально не определена, и результат этой операции абсурден по определению. Поэтому и бессмысленно рассчитывать точные значения.

Когда я говорил про дифуры от времени, то подразумевал именно то, что эти дифуры должны быть чем-то полезными. Когда связь прямая, дифуры чаще всего не нужны - достаточно обычных алгебраических уравнений (бывают исключения, о чем я и написал, но это именно исключения). А вот когда в системе есть обратная связь (что означает, что эту связь необходимо учитывать на практике для решения конкретной задачи, - ибо обратные связи есть, конечно, в любой системе) - то уже алгебраические уравнения являются скорее исключением, и надо-таки писать дифуры.

Когда я говорил об описании фактов, то, опять же, имел в виду совершенно стандартную для описательных дисциплин ситуацию: что-то "наблюли", но что именно - толком не понятно. В этом случае эти факты как раз и надо ОПИСЫВАТЬ: в каких условиях производили наблюдения или эксперимент и что объективно получили. Поскольку точных формул никаких нет, описание является единственным способом фиксации этого самого факта.

Все эти вещи я совершенно четко вижу в ТРИЗ, и, поверьте, я не вижу здесь ровным счетом ничего такого, что бы принципиально отличало ТРИЗ от прочих описательных дисциплин, таких, как психология, медицина, социология, юриспруденция или политология.

А общая тенденция такова, что описательные дисциплины со временем становятся точными, но никак не наоборот - по крайней мере, об обратных случаях (когда бы точная наука стала описательной и отказалась от имеющейся в ней математики) мне пока еще ничего не известно. Медицина уже давно на пути к точным наукам, психология "на подходе", в социологии всякие математические модели используются уже второе столетие, даже в юриспруденции уже математические методы начинают применять. И нас не минет чаша сия, поверьте.

В общем, на мой взгляд, никакая ТРИЗ не "методика", а самая обычная гуманитарная наука. И если именно с этих позиций к ней подходить, то и не надо выдумывать всяких определений типа "не наука, а методика", а надо просто развивать ее сообразно ее нынешнему (а не "послезавтрашнему"!) статусу. Ведь физика тоже далеко не сразу стала точной наукой...

Ну и в заключение про "месяц-год-два". Вы-то сами как считаете, неужели до открытия "настоящих" (в обычном естественнонаучном смысле) законов и реального использования количественных формул в ТРИЗ надо ждать еще полтыщи или полсотни лет? Не многовато ли ждем? По-моему, месяц или год - куда более правильный срок, чтобы нАчать и затем углУбить... Впрочем, история нас, надеюсь, очень быстро рассудит :).

С уважением,

Александр.

    Log in or register to post comments

Re: Классификация потоков в технических системах

Submitted by Ромащук Александр on вс, 01/05/2011 - 21:52

Юрий, раз мне дали карт-бланш на дополнительные вопросы, то попробую пояснить свой вопрос!
Правильно ли я понял, что Вы предлагаете определенные усовершенстования в определенных местах процедуры потокового анализа, которая описана О.М.Герасимовым в своей диссертации в п. 4.4.9 (по Вашей ссылке)? Просто и в диссертации процедура (т.е. собственно метод) описана не так чтобы слишком подробно, но в этом случае более понятной становится методическая часть Вашей статьи. Хотя мне кажется, что инструктивную часть процедуры добавить было бы не сложно (особенно если она такая же небольшая как в диссертации Герасимова) и это куда больше акцентировало бы собственно методическую направленность (от познавательно-классификационной). Особенно это полезно было бы для таких малознакомых с потоковым анализом как я
С уважением, Александр

    Log in or register to post comments

Re: Классификация потоков в технических системах

Submitted by lebedur on пн, 02/05/2011 - 16:58

Добрый день, коллеги.
Прошу прощения, но вынужден сегодня обращаться по фамилиям.
Александру Кудрявцеву.
Конечно, я готов говорить о любых подробностях. Просто мне показалось, что Александр Ромащук близок к тому, чтобы начать задавать вопросы, ответы на которые, судя по предыдущей переписке, неинтересны другим. Если он предпочитает задавать вопросы на форуме, разумеется, буду на форуме же и отвечать, чем прямо сейчас и займусь.
Александру Привеню.
Наш диалог вскорости начнет походить на диалог кукушки и петуха. По-моему, нам уже можно подсократить количество экивоков и начать по умолчанию исходить из того, что мы относимся друг к другу с достаточным уважением, чтобы не подчеркивать это через фразу. Но все же напоследок: нет, это я неправильно выразился (:-)). Вовсе не считаю ТРИЗ совсем уж не-наукой. Помнится, даже писал, что ТРИЗ наука, скорее, гуманитарная…
Но в этом нашем смешном деле есть еще и методический аспект. Завзятые методисты исходят из необходимости писать методики в «инструктивном стиле»: делай раз, делай два… (это я уже потихоньку начинаю отвечать и Александру Ромащуку). Я не сторонник жестких инструкций. Точнее, я их ценю и даже пользуюсь. Но хочу оставить за собой право в любой удобный для себя момент отменить или изменить любой из пунктов любой инструкции. Здесь, в Корее, я в этом убеждении еще раз утвердился. Но для этого нужно не просто знать пункты инструкций, но и понимать их смысловую составляющую.
Как тут нам недавно резонно напомнили, раздавать указания может и дрессированная шимпанзе. Считая себя немножко разумнее шимпанзе, предпочитаю понимать смысл своих действий. Вот, собственно, в этом и заключается азарт в моих работах, не вполне методических, но и не чисто теоретических – так, серединка наполовинку.
Что же до математизации ТРИЗ, то мы уже вот-вот начнем повторяться. Может быть, стоит перенести этот разговор в другую ветку?
Александру Ромащуку.
Мне кажется, что я все же предлагаю нечто большее, чем некоторые уточнения в описанной Олегом процедуре. Эту ссылку я дал для того, чтобы
- Стала понятнее «методическая часть» моей статьи – кажется, это удалось
- Просто не знаю других источников в Интернете. Не то, чтобы это было как-то сильно секретно (Олег же выложил), просто в инете редко выкладывают инструкции.
Теперь подробности. Я не писал и не хочу писать инструктивных материалов (см. выше). Олег изложил один из вариантов построения потоковой модели (и цель его диссертации была другая). Знаю еще как минимум два и пользуюсь ими. А вообще, таких вариантов ровно столько, сколько авторов потоковых моделей, и даже чуть больше. Именно поэтому не стал расписывать процедуру построения модели.
Давайте считать потоковой моделью описание системы потоков в ТС и ближайшем окружении. Описание может быть разным. Почти всегда сначала рисуют графическую модель. Как правило, они сводятся к двум основным типам (но не только):
- в узлах компоненты, куда и откуда поток течет
- в узлах результат действия потока (новое химическое вещество, тепло, какое-то изменение в компоненте и т.д.).
Первый вариант сильно напоминает модель конструкции, второй – модель процесса.
Например, недавно я делал потоковую модель для деградации аккумулятора и сделал это вторым способом. Чуть раньше делал потоковую модель топливной ячейки, и делал первым способом. А еще когда-то делал потоковую модель, располагая потоки на планировке цеха. А еще раньше…. И так далее.
В любом случае возникает список потоков с некоторым их описанием.
«Инструктивная часть» моей работы могла бы свестись к:
- Предложению такой список составить в табличной форме,
- Предложению ввести в эту таблицу четыре новых столбца и их заполнить,
- Далее по моему вчерашнему тексту.
Насчет малого знакомства с потоковой моделью. Собственно говоря, всю известную мне «теоретическую» часть процесса я и написал (+ описание Герасимова). По моему глубокому убеждению, дальше полезна только практика. Конечно, хорошо бы, чтобы эта практика была под менторством хорошего знающего педагога, но тут я пас. Я меньше всего педагог. Помочь составить модель – могу, устроить ее разбор – тоже, но преподавать просто не умею.
Впрочем, по-прежнему готов отвечать на вопросы как более-менее общие - на форуме, так и о сугубых частностях - в личке.
Удачи,
Ю. Лебедев

    Log in or register to post comments

Re: Классификация потоков в технических системах

Submitted by Ромащук Александр on чт, 05/05/2011 - 11:11

Спасибо большое, Юрий, за подробное разъяснение!
Взял небольшой тайм-аут, чтобы устоялось собственное понимание. Насколько я понял, изначально Вы не заострились на "инструктивной части" из-за возможности разных вариантов встраивания в уже существующие "инструкции" (варианты цельного потокового анализа), т.е. попытались выделить инвариантную по отношению к этим вариациям метода часть. С учетом Ваших уточнений процедуры применения такого "инварианта" все становится куда более понятно.
При этом только хотел бы предположить, что акцент метод "описания" повышает эффективность перехода от обсуждения собственно методической части к условно научно-познавательной (от "как" к "что"). Другими словами, если дается процедура использования термина "информация", то чтобы под этим термином не понималось, можно обсуждать сложности использования процедуры, неоднозначность ее применения на разных конкретных задачах и т.п. Без процедурной части очень легко и просто увлечься разбором что есть "информация" вообще и как ее понимает автор... Написал это больше как пример общей опасности обсуждений на форуме, а не особенность обсуждения именно Вашей статьи
С уважением, Александр
ПС: большое спасибо за предложение обращаться с трудностями использования потокового анализа. Буду иметь в виду такую возможность :)

    Log in or register to post comments

Re: Классификация потоков в технических системах

Submitted by lebedur on чт, 05/05/2011 - 16:41

Александр.
Кажется, вы поняли меня именно так, как я и хотел сказать. Разве что я попытался обойтись без терминов "инвариант" и тому подобных, но по смыслу мы друг друга поняли. Я рад.
Насчет "Без процедурной части очень легко и просто увлечься разбором что есть "информация" вообще и как ее понимает автор... " - согласен и уже покаялся в том, что имея ввиду рассказать о своих соображениях всем, реально рассказывал людям из GEN3 - я не очень опытный публикатор и постараюсь впредь такую ошибку не допускать.
Ю. Лебедев

    Log in or register to post comments

Re: Классификация потоков в технических системах

Submitted by Ромащук Александр on чт, 05/05/2011 - 20:42

Юрий, я тем более рад, что в целом Вас понял (а уж как это выразить терминологически...)! Про раскаяние - боюсь, что хотя Вы несколько утрировали свою позицию, но в целом это общая опасность форума. Сваливание на квазинаучные обсуждения и, тем самым, уменьшение профильного и профессионального для большинства участников (к которым себя, конечно, не отношу) - методической работы. Вполне вероятно, что я описываю только собственные страхи
С уважением, Александр
