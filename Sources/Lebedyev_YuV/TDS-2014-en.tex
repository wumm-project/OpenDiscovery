\documentclass[a4paper,11pt]{article}
\usepackage{od}
\usepackage[russian,main=english]{babel}

\setlist{noitemsep}

\author{Lebedev Y.V., Logvinov S.A.}
\title{Integration of Flow and Functional Analysis}
\date{TRIZ Developer Summit, 2014}

\begin{document}
\maketitle

\begin{quote}
  Translated from the Russian Original available at
  \url{https://triz-summit.ru/confer/tds-2014/article/} 
\end{quote}

Flow analysis is an effective tool for analysing technical systems in TRIZ. It
is a good complement to functional analysis (FA), as it allows to identify
problems not identified by other means of analysis.

Nevertheless, until now:
\begin{itemize}
\item There is no accepted methodology for conducting flow analysis.
\item No information from FA is used in flow analysis.
\item The results of flow analysis and FA are integrated only when building
  the causal chain of undesirable effects.
\end{itemize}

This paper attempts to consider flow analysis as a special case of functional
analysis: flows in a technical system (TS) are considered as a special case of
components of the TS that have important features.

The functional relationship of flows to other ('stationary') components of the
system is considered: to source, channel, receiver and control system of the
flow, which form a functionally complete technical subsystem.

The proposed approach opens up the following possibilities:
\begin{itemize}
\item The possibility of applying well-established functional analysis
  techniques in flow analysis.
\item Partial integration of the processes of flow and functional analyses.
\item Identification of components interacting with the flows to be improved.
\end{itemize}

\section*{Problem Statement}

The practice of flow analysis is highly effective. A properly constructed flow
model can identify problems that are poorly detectable by FA tools. The reason
is clear -- FA takes almost no account of information about the spatial
structure of the TS being analysed. But the flow model intuitively takes this
information into account. The functional model (FM) is a "snapshot" of the TS
taken at a given point in time. But this completely or almost completely
ignores the dynamics of the TS. When building a FM of the system at work, one
has to go to all sorts of tricks or simply has to disregard the "canonical
purity" of the description. A typical example: component 1 -- moves --
component 2. When strictly following the rules of composing the FM, such a
formulation is no longer correct, as it implies different positions of the
components at different time.

The introduction of flows into the FM removes this kind of difficulty by
\textbf{treating flows as dynamic components of a system} whose lower system
level components move in space under the influence of static components.
However, unlike FA, to date flow analysis is clearly unsatisfactory
formalised.  In fact, there is no unified methodology for conducting it.  In
addition, there are no tools for exchanging information between FA and flow
analysis models. The two analyses are carried out independently and their
results are combined only when the causal chain of undesirable effects is
built.

There have been repeated attempts to combine the two instruments. One of the
most elaborated methodologies is described in A. Kashkarov's dissertation
\cite{1}.  Unfortunately, his proposed methodology is quite cumbersome, which
completely kills one of the main advantages of flow analysis -- its intuitive
clarity and transparency.

This article discusses approaches that allow for the integration of FA and
flow analysis without losing these inherent merits.

\section*{Choice of a Methodological Prototype}

To continue the discussion, we select one of the existing variants of flow
analysis methodology as a basis for further improvement. Apparently, the
relationship between flow analysis and other TRIZ tools is best described in
O. Gerasimov's dissertation \emph{Technology of Selecting Tools for Innovative
  Design basaed on TRIZ-FVA} \cite{2}. However, this description is very
brief, which makes it difficult to use in practical work.

In addition, this description uses a number of concepts introduced within
ZRTS\footnote{ZRTS -- Laws of Development of Technical Systems.}  (flow
analysis appeared as a development of one of the laws of ZRTS -- the Law
\emph{Improve the Efficiency of the Use of Flows of Matter, Energy and
  Information}). It is useful to recall the history of the emergence and
development of this law.

The precursor of the law is the law of minimal energy conductivity of systems
formulated by G. Altshuller \cite{3}. In the course of development of the
system of ZRTS this law remained practically unchanged for a long time (from
1975 to 2002, judging by dates of publications). In particular, in
\cite[p. 56]{4} the law is briefly mentioned as part of the law of increasing
coherence in systems; in Salamatov's book \cite{5} the law is reproduces
almost verbatim.

In V. Petrov's book \cite{6} the law is considered as an increase in specific
energy saturation of systems and is a sub-trend of the law of system
transition to the micro-level. But here it is no longer considered as
requiring a minimal necessary level, but as a line of systems development.

In the work of S. Litvin and A. Lyubomirsky \cite{7}, the law is completely
revised and considered as "law of increasing the efficiency of matter, energy
and information flows".

It is not difficult to see that both V. Petrov and S. Litvin, A. Lyubomirsky
have, in essence, a completely new law, not so much developing the predecessor
but located nearby:
\begin{itemize}
\item Firstly, this formulation of the law refers not to a system to be
  constructed, but to ways of improving an already viable system.
\item Secondly, \cite{7} significantly extends the range of flows considered
  with "energy flows" to "matter, energy and information flows", i.e. to all
  kinds of flows existing in the system.
\end{itemize}
In doing so, the description of the law is not so much a trend (a line of
development) as a list of mechanisms (essentially recommendations to improve
flows in the system).  The list of mechanisms is quite extensive at 42 items,
structured by type of flow and divided into two groups, implying "changes in
the conductivity of flows" and "changes in the efficiency of flows". It is
this version of the law and its recommendations for conducting flow analysis
that we will choose as prototype for future work.

\section*{What are we Going to Improve?}

First of all, definitions. A serious shortcoming of most theoretical work in
TRIZ is the lack of attention to notions and definitions.  In particular,
S. Litvin and A. Lyubomirsky formulate four main trends of this law (parasitic
flows, as a special case of harmful ones, are described almost word by word
as harmful ones):
\begin{itemize}
\item Improve the efficiency of use of useful flows.
\item Reduce the damaging power of harmful flows.
\item Increase the conductivity of useful flows.
\item Reduce the conductivity of harmful flows.
\end{itemize}
The first two trends are obvious to triviality and trivial to complete
non-instrumentality. But it is also completely impossible to argue with them.
Of course, it can be useful to formulate obvious things explicitly as axioms,
and this has been done.

But there are serious problems with the other two trends.
\begin{itemize}
\item Increase the conductivity of useful flows.
\item Reduce the conductivity of harmful flows.
\end{itemize}
First of all, \textbf{conductivity of flow} sounds similar to the concepts of
\textbf{resistance \underline{of} electricity} or \textbf{conductivity of
  water in a pipe}.  But we have, after all, to talk about conductivity of
flow \textbf{channels} (paths) ("resistance of the \textbf{conductor}
\underline{to} electric current"). It is not a matter of simple matching of
words, but of distinguishing substantially different categories previously
mixed in a single term.

So first let's try to clarify some key terms.  In this paper
\begin{itemize}
\item We call \textbf{a flow} such a movement of material objects, energy or
  information in a system in which individual parts of the flow move according
  to the same law one after the other (part of the flow may move in a
  supersystem, but the key is its presence and movement in the system in
  question).
\item We call the \textbf{flow source} the component of the system that
  generates the flow and sets its initial parameters.
\item We call \textbf{flow channel} the system component which defines the
  path of the movement of the flow (the channel may be distributed in space,
  and may have no clear, unambiguous boundaries).
\item We call the \textbf{consumer} the component of the system which
  is transformed by a given useful flow, or one that is directly damaged by
  the harmful flow in question.
\end{itemize}

\begin{center}
  Add Diagram
\end{center}

Notes:
\begin{itemize}
\item[A.] The above formulations are not absolutely general for all
  conceivable cases, but are sufficient for applied purposes (analysis of TS
  and elaboration of proposals for their improvement).
\item[B.] The assignment of elements of a system to one or another of the
  defined components are not unconditional and are determined by the specifics
  of the task to be performed.
\item[C.] In many systems, the flow is the object of transformation, not the
  subject.  Accordingly, the "consumer" is the subject and should more
  properly be called the "transducer" of the flow. However, for the needs of
  practical flow analysis there is no significant difference, so we omit this
  peculiarity from further consideration.
\end{itemize}
This formulation separates four types of system components that functionally
significantly differ in relation to the flow in question:
\begin{itemize}
\item the flow itself, as the subject of the transformation,
\item the source of the flow,
\item the channel for holding/limiting/directing the flow,
\item the consumer -- a component that is directly affected by the flow by
  changing at least one of its parameters.
\end{itemize}
In terms of "subject -- function -- object", the channel transforms the flow,
the flow transforms the product. An essential detail is that a minimum FM
consists of a flow element and a channel. Source and consumer may not be
included in the FM.

The complexity of the model (introducing new additional elements) only seems
to exist. The explicit introduction of a component which was previously
implied by default
\begin{itemize}
\item simplifies the analysis of the model,
\item reveals a direct, unmediated link between the flow model and the
  functional model, and also between the flow model and ZRTS.
\end{itemize}
In particular:
\begin{itemize}
\item In the flow model it is possible to operate with other components of the
  system (source, channel, consumer).
\item In the functional model, in addition to the components appear all of the
  flows in question.
\item The four system components (source, flow, channel and product) can be
  explicitly aligned, the components are specified by which the manageability
  of the flow (source and channel) can grow, etc.
\end{itemize}
In the proposed formulations also the pairs of trends are mentioned quite
clearly:
\begin{itemize}
\item Improving the efficiency of the use of useful flows \textbf{by the
  consumer}.
\item Reducing the damaging power of harmful flows in relation to
  \textbf{other elements of the TS}.
\end{itemize}
BUT:
\begin{itemize}
\item Increase the conductivity of \textbf{channels} of useful flows.
\item Reduce the conductivity of \textbf{channels} of harmful flows.
\end{itemize}
In this formulation, trends become much clearer. Thus:
\begin{itemize}
\item the incompleteness of the current wording of the law becomes apparent,
\item the second pair of trends is still highly objectionable.
\end{itemize}
Simple examples:
\begin{itemize}
\item The useful flow of fuel in an internal combustion engine (ICE). If the
  conductivity of this flow channel is increased, additional fuel flows into
  the combustion chambers which will result in incomplete combustion, which in
  turn will lead to a number of serious problems.
\item Useful flow of hot water or steam in the heat exchanger. Increasing the
  conductivity of this channel heat will be removed from the system, although
  we need the opposite.
\item The harmful Joule heat flux from the passage of electric current through
  an electronic circuit when the conductivity of this flow channel is reduced
  will additionally heat up the board, although again the opposite is
  required.
\item If the conductivity of the conductor in the useful flow of electric
  current in a light bulb is increased, it will result in a change in its
  rating and beyond a certain limit the lamp simply burns out.
\item If the conductivity of the useful flow of a semi-finished product to
  some kind of actuator (to the consumer of the flow) rises above a certain
  limit, the consumer will become overstocked and/or a store as buffer has to
  be introduced.
\end{itemize}
Of course, there are different ways to paraphrase this. A typical way to
overcome these ambiguities in a flow or ZRTS analysis is to say "in this case
there is a different sub-trend ...".  However, \textbf{these kinds of clauses
  (and even the need for them itself) drastically reduce the instrumentality
  of the method}.

Of course, increasing the conductivity of the flow channel is often very
useful. The techniques for such an increase are detailed in the current
version of the law and remain true and certainly useful. The same goes for
cases where the channel conductivity is reduced.

The proposed refinements to the concepts, while seemingly simple, make it
possible to significantly improve the effectiveness of flow analysis.

\section*{An Algorithm to Conduct the Analysis}
The following logic is proposed for conducting a flow analysis.
\begin{enumerate}
\item Build a flow model using the existing prototypical methodology.
\item Identify the flow deficiencies as recommended.
\item Clarify the identified deficiencies, identify their type:
  \begin{itemize}
  \item Inadequate parameters and functions of the flow
    \begin{itemize}
    \item Inadequate and harmful flow functions
    \item Excessive expenses for the functioning of the flow
    \end{itemize}
  \item Inadequate parameters and functions of the channel
    \begin{itemize}
    \item Inadequate and harmful channel functions
    \item Excessive expensesfor the formation and functioning of the channel
    \end{itemize}
  \end{itemize}
\item ONLY for the identified problematic areas of the flow we construct
  refined flow models, in which we explicitly separate the flow itself and the
  channel.
\item Based on step 4, we construct a functional model for the components
  associated with the problematic flow and conduct a standard FA. (As a matter
  of fact, this is a standard and widely used technique by practitioners: to
  build a detailed FM at a deeper system-level for the problematic part of the
  system identified by the model at the higher level.)
\item To eliminate the deficiencies identified in the construction of the FM
  according to point 5 use the recommendations of the law to improve the
  efficiency of matter, energy and information flows. In doing so, it is
  convenient to make use of the classification of flows and of trends which
  are characteristic for different types of flows. An attempt at such
  classification and trends is given in \cite{8}.
\end{enumerate}

\section*{Conclusions}
Suggested approach:
\begin{itemize}
\item It is not a direct combination of the two methodologically quite
  different tools. Essentially, the proposed approach allows for the
  application of well-established and repeatedly proven techniques and methods
  of functional ANALYSIS to a convenient and illustrative flow-based MODEL.
\item It is easy to apply and can be quickly mastered by practitioners being
  familiar with either FA or flow analysis. The approach can be applied by
  specialists, familiar with different variants of both FM and flow models. It
  is no secret that real-world models are constructed a little differently by
  each professional.  Therefore, a rigidly prescribed algorithm may not be
  suitable for many practitioners.
\end{itemize}
Of course, it would be tempting to fully merge the two models, as was done,
for example, in \cite{1}. But at the output we get a very cumbersome model,
overloaded with both information and graphic elements. The proposed approach
allows, in our view, to obtain sufficiently weighty results with not too much
effort.

\begin{thebibliography}{xxx}
\bibitem{1} Kashkarov A.G.
  \foreignlanguage{russian}{Вещественно-энергетические преобразования в
    технической системе.  Методика построения и анализа моделей} --
  Substance-energy transformations in a technical system. A methodology of
  model construction and analysis. 2009.
  \url{http://www.triz-summit.ru/ru/203864/204357/204590/}
\bibitem{2} Geramsimov O.M. \foreignlanguage{russian}{Технология выбора
  инструментов инновационного проектирования на основе ТРИЗ-ФСА} -- Technology
  for selecting innovation design tools based on TRIZ-FVA.  2010.
  \url{http://www.triz-summit.ru/ru/203864/204737/204739/}
\bibitem{3} Altshuller G.S., Creativity as an exact science. -- Moscow:
  Sov. Radio, 1979. Laws of Systems Development; 2nd Law of "Energy
  Conductivity" of a System. \url{http://www.altshuller.ru/triz/zrts1.asp#12} 
\bibitem{4} Altshuller G.S., Zlotin B.L., Zusman A.V. et al.
  \foreignlanguage{russian}{Поиск новых идей: от озарения к технологии (теория
    и практика решения изобретательских задач)} -- Search for new ideas: from
  insight to technology (theory and practice of inventive problem solving).
  Kishinev, 1989.
  \url{http://www.trizway.com/content/poisk_novih1.pdf}
\bibitem{5} Salamatov Y.P. 1991-1996. \foreignlanguage{russian}{Система
  Законов Развития Техники (Основы Теории Развития Технических Систем)} -- The
  System of Technological Development Laws (Fundamentals of Technical Systems
  Development Theory).   \url{http://www.trizminsk.org/e/21101440.htm}

  Yuri Salamatov. TRIZ: the Right Solution at the Right Time: a Guide to
  Innovative Problem Solving.  Institute of Innovative Design.
\bibitem{6} Vladimir Petrov. \foreignlanguage{russian}{Серия статей «Законы
  развития систем»} -- Series of papers on Laws of Systems Development, 24
  September 2002
  \url{http://www.trizland.ru/trizba/pdf-books/zrts-12-microlevel.pdf}
  \url{http://www.trizland.ru/trizba/pdf-books/zrts-16-energo.pdf}
\bibitem{7} Litvin S. S., Lyubomirsky A.L. \foreignlanguage{russian}{Законы
  развития технических систем} -- Laws of development of technical systems,
  February 2003. \url{http://www.metodolog.ru/00822/00822.html}
\bibitem{8} Lebedev Y. V. \foreignlanguage{russian}{Классификация потоков в
  технических системах} -- Classification of flows in technical systems, 2011.
  \url{http://www.metodolog.ru/node/967}
\end{thebibliography}
\end{document}
