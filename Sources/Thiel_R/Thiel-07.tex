\documentclass[11pt,a4paper]{article}
\usepackage{od}
\usepackage[utf8]{inputenc}
\usepackage[ngerman]{babel}

\title{Zur Lehrbarkeit dialektischen Denkens – Chance der
  Philosophie, Mathematik und Kybernetik helfen} 
\author{Rainer Thiel, Storkow}
\date{Version vom Dezember 2007.} 

\begin{document}
\maketitle

\begin{quote}
  Dieser Text ist die Grundlage eines Vortrags von Rainer Thiel auf der
  Konferenz der Leibniz-Sozietät am 8. November 2007. Er wurde im
  Protokollband
  
  Fuchs-Kittowski, Klaus / Zimmermann, Rainer E.:\\ “Kybernetik, evolutionäre
  Systemtheorie und Dialektik”,\\ ISBN 978-3-89626-919-5
  
  dieser dreitägigen Konferenz veröffentlicht, der 2012 im Trafo-Verlag Berlin
  erschienen ist.  Mehr zum genaueren Ablauf der Konferenz siehe\\
  \url{http://www.leipzig-netz.de/index.php/HGG.2019-09}.   
\end{quote}

\section{Vorbemerkungen}

Hegel schrieb 1807: „Erst was vollkommen bestimmt ist, ist zugleich
exoterisch, begreiflich, und fähig, gelernt und das Eigentum aller zu sein.
(Phänomenologie des Geistes, S. 17.) Was dem vorausgegangen war in aller
Philosophie, das hat Hans Heinz Holz mit seinem „Weltentwurf und Reflexion“
durchleuchtet. Er hat die Grundlegung der Dialektik vollendet. Sein Werk habe
ich erst jetzt lesen können, seit vierzig Jahren stehe ich außerhalb
philosophischer Institute. Indessen frei vom Blick auf Bürokraten. Freiheit
habe ich mir erlaubt, weil in der DDR keine Arbeitslosigkeit zu befürchten
war. Ohne besondere Intelligenz konnte ich Akademikern voraus sein. Seit
langem wirke ich in sozialen Netzen, da bleibt für Bücher auch kaum Zeit. Doch
Praktikum hat Leibniz gut geheißen: „Theoriam cum praxi coniungere.“

Ein Zweites sei erinnert. Die Paderborner Gruppe „Erwägen Wissen Ethik“ (EWE)
hatte zum Thema „Dialektik als Heuristik“ aufgerufen. Da habe ich geäußert:
Dialektik als Heuristik „JA“. Aber es wurde zu viel drum herum geschrieben,
auch vom Thema abgewichen. Statt Dialektik zu entwickeln – zu viel
Mechanistik.

Indessen gilt: Menschen machen Geschichte, auch wenn es nur per Stillehalten
ist und übel ausgeht. Menschen gestalten Geschichte, wenn sie über sich
hinausgehen. Wissen muss man, was passieren kann. Rückwärtsblickend hilft die
Rede vom Determinismus weiter, vorwärts aber nicht. Marx und Engels haben
Möglichkeiten erkannt. Aber Voraussagen? Nein, sagen Marx und Engels. (MEW 22
Seite 509) Goethe war nur wenig zu weit gegangen, als er schrieb: „Was macht
gewinnen? Nicht lange besinnen.“ Ähnlich Clausewitz (Ausgabe 1957 Seite 434).
Das habe ich in „Mathematik – Sprache - Dialektik“ verarbeitet, um Mathematik
vom Geruch zu befreien, Kochbuch für Buchhalter zu sein. Mit Mathematik kann
man Sprechen und Denken über Dialektik. Um das anno 1975 gedruckt zu kriegen
war List vonnöten und Solidarität. Zu spät und zag hat Herbert Hörz ein
Löchlein in das Beamten-Brett zu bohren versucht.

Kurzum: Alles, was man vorwärtsblickend wissen kann, macht Geschichte einem
Fußballspiel oder einer Ehe eher vergleichbar als Planetenbahnen und
Gasmoleküle. Auch Ballspiele und Ehen werden von Tätern gestaltet. Hinterher
kann man fragen, wie alles gekommen ist, so weit reicht der Determinismus.
Vorwärts kommt es aufs kreative Handeln an. Das an der Mechanik orientierte
Paradigma „Determinismus“ ist aber, selbst wenn es auf Zufälle Rücksicht
nimmt, seit Hegel auch deshalb obsolet, weil es lebendige Objekte nicht
zugleich als Subjekte wahrnimmt, die über sich selber hinausgehen. Das kann
mit dem Terminus „innere Widersprüchlichkeit“ designiert werden. Zufälle
werden in der Regel nur als äußere Beeinflussungen von Objekten gesehen. Das
ist Mechanik. Dialektik ist die Selbstbestimmung von Systemen, von Subjekten
und Populationen sowie die Theorie davon.

Zum Fußball gehören Trainer. Außerhalb des Fußballs bräuchten wir
professionelle Dialektiker. In der schönen Literatur bestimmen sich
menschliche Populationen und Subjekte, der Leser perzipiert das intuitiv, das
Verständnis der innewohnender Dialektik könnte durch Philosophen zum vollen
Bewusstsein ausgeprägt werden. Marx/Engels haben – übers Heute hinaus denkend
- Möglichkeiten erkannt, Hypothesen gebildet, Mut gemacht. Das ist Dialektik
als Heuristik. Und auch nachlesbar. Doch das wurde zugewischt. Der Schaden ist
gewaltig und trug zum vorläufigen Abbruch hoffnungsvoller Anfänge bei.

Vorwärts nun mit Hans Heinz Holz. Sein zwanzigstes Kapitel, sein Ausblick,
beginnt etwa so: „Von den jetzt gewonnenen Einsichten aus lässt sich [\ldots]
eine [\ldots] konstruktive Systematisierung der Dialektik vornehmen.“ Das ist
Aufruf, Dialektik lehrbar zu machen. Konstruktive Dialektik hat Hans Heinz
Holz schon selber praktiziert, indem er zeigte, was „Negation“ und was
„Widerspiegelung“ ist. Konstruktiven Geistes war Hegel, als er schrieb: Das
Individuum hat das Recht zu fordern, „dass ihm die Wissenschaft wenigstens die
Leiter reiche“. (Phäno. 25) Das passt zu Hans Heinz Holz: Dialektik werde „auf
die vier Grundzüge zurückkommen“, die in didaktischen Schriften millionenfach
verbreitet worden sind. Das ist eine der untersten Sprossen. So habe ich mich
geäußert, auch nach der EWE-Diskussion in einem Schreiben an alle Teilnehmer,
in höchster Kürze. Auch heute ist Kürze geboten. Als Häretiker hatte ich immer
schon Verzicht zu üben. Kompression sollte erleichtern, Ungewohntes gedruckt
zu bekommen. Doch es hat auch erleichtert, Gedanken eines Schülers von Karl
Marx zu unterdrücken.

\section{Nun zur Sache selbst:}

Der Kürze halber lasse ich Worte zu den Grundzügen 1 und 2, also zu den Topoi
„Zusammenhang“ und „Entwicklung“, heute ganz weg. Ich lasse auch weg einen
5.~Grundzug, der dem „Differenzieren“ gewidmet sein müsste. Erwähnt sei jetzt
nur, dass sich Einsichten zu den Topoi „Zusammenhang“, „Entwicklung“,
„Differenzierung“ und zugleich Einsichten in soziale Bewegungen ergeben, wenn
man als Intellektueller mittendrin ist. Dann erkennt man auch gnoseologische
und soziologische Aspekte der Dialektik. In „Zusammenhängen“, in „Entwicklung“
und „spezifizierend“ zu denken ist fast allen Menschen ungewohnt. Es ist
schwer, mit ihnen über Entwicklung und Differenzierung in Politik und
Bürgerbewegung zu sprechen. Vielen fällt es schon schwer, den Zusammenhang mit
einem Partner zu realisieren. Sie melden sich am Telefon, zum Beispiel
„Stefan“. Doch welcher Stefan ist es von den vielen, an die der Hörer denken
muss? Und gar zwei Zusammenhänge gleichzeitig im Auge zu haben fällt den
meisten Menschen schwer. Lieber versteifen sie sich auf erste Eindrücke und
klopfen sich mit Redepartnern, die einen anderen Zipfel der Realität am
kleinen Finger haben.

Also jetzt nur zum 4. und danach zum 3.~Grundzug der Dialektik, zum
dialektischen Widerspruch und danach zum Verhältnis von Quantum und Quale.

\section{Anmerkungen zum dialektischen Widerspruch:}

Dabei blicke ich auf Lehrbücher, die bis 1989 erschienen sind. Ein Teil der
zehntausend Zeilen dort gilt gutwilligen Lesern als unbestreitbar. Zu dem
anderen Teil möchte ich jetzt sechs Anmerkungen vortragen:

\paragraph{1. Anmerkung:}
Zur Dialektik polarer Verhältnisse wird bis 1989 verwiesen auf Beispiele von
Marx/Engels, nicht immer bewältigt im Sinne ihrer Spender, aber immerhin. Doch
Anregungen aus der Kybernetik? 1967 durch Georg Klaus auf einem Berg von
Erkenntnissen! Danach war Schluss damit in Lehrbüchern zum DiaMat. Kybernetik
hatte um 1960 geholfen, das Verhältnis von Wechselwirkung und Zielstrebigkeit
zu verstehen, ich hatte das exemplifiziert an Rückkopplungs-Systemen in
Marxens Kapital, mit Rückhalt von Georg Klaus gedruckt 1962, und anno 1967
weiter ausgeführt unterm Titel „Quantität oder Begriff? Der heuristische
Gebrauch mathematischer Begriffe“. Neulich hat Günter Kröber daran erinnert in
einem Sammelband „Kybernetik steckt den Osten an. Aufstieg und Schwierigkeiten
einer interdisziplinären Wissenschaft in der DDR“ (Berlin 2007). Mein
druckfertiges Manuskript für diesen Sammelband mit Auszügen aus meinen
Veröffentlichungen von 1962/67 war ohne Rücksprache mit mir unterdrückt
worden.  Doch ein Teil der Erkenntnisse findet sich bei Kröber. Meine Arbeit
von 1962 scheint also auch 45 Jahre später interessant, obwohl ich selber
heute darüber hinaus bin. Trotzdem freue ich mich über jeden, den ich mit der
Zeit von der Gültigkeit auch der ersten Anfänge habe überzeugen können.
Erkenntnisse von anno 1962 schmücken also den Sammelband von anno 2007. In
meinem druckfertigen, doch unterdrückten Beitrag waren auch Erkenntnisse von
1975 zum Thema „Mathematik – Sprache – Dialektik“ referiert. Heute in diesem
kleinen Kreise darf ich das erwähnen: Als Mitgestalter des Aufstiegs einer
interdisziplinären Disziplin, auch in praktischen Fragen, meine ich, dass der
Sammelband nicht rundum gelungen ist. Laut muss ich sagen, dass verschwiegen
wurden auch die Initiativen von Friedhart Klix und anderen im Forschungsrat
der DDR, an denen ich teilgenommen habe und die der Akademie der
Wissenschaften abgerungen wurden. Andere Einzelheiten, die in „Quantität oder
Begriff“ (1967) und in „Mathematik – Sprache – Dialektik“ (1975) nachgelesen
werden können, auch die Beiträge zu einer Spezifikation des dialektischen
Widerspruchs, lasse ich heute weg. Bedauerlich bleibt, dass im erwähnten
Sammelband von 2007 zum x-ten Mal der Eindruck erweckt wird, man hätte sich
der Gefahr erwehren müssen, dass Philosophie durch Kybernetik ersetzt werde.
Stattdessen wäre hervorzuheben gewesen, dass durch Kybernetik belebt wurde,
was in der Philosophie von Hegel und Marx längst angedacht und endlich
wahrzunehmen war.

In den sechziger Jahren hatte es noch einiges mehr gegeben, um schrittweise
eine Lehre der Dialektik zu schaffen. Der Philosophie-Historiker Gottfried
Stiehler hat 1966 das Maximum an Klarheit erreicht, das ohne Mathematik
erreichbar ist. Stiehlers Buch von 1966 zeugt von Ernst, auch heute kann man
daraus lernen. Doch bald ist in Lehrbüchern nur noch ein Konglomerat von
Worten wie Widerspruch, Gegensatz, Antagonismus, Differenz von Soll und Sein.
Keine Begriffe, keine Spur von System und Verständlichkeit. Unausgefüllte
Gerüste blieben die beiden Ansätze von 1966 und 1967. Verschlampert wurden
Ansätze zum Sozialismus.

\paragraph{2. Anmerkung:}
In den Lehrbüchern bis 1989 sind die umfangreichen Texte mit der
Arbeiterklasse im Zentrum fixiert auf erstarrte Bilder gesellschaftlichen
Geschehens. Ich war vierzig Jahre lang Mitglied der SED, in den ersten Jahren
habe ich viel gelernt, das gebe ich nicht auf, Arbeiter und Funktionär bin ich
selbst gewesen. Aber bis 1989 ist in Lehrbüchern ausgeblendet die Spaltung der
Arbeiterklasse in Werktätige und Ritualienpfleger. Die Spaltung begann vorm
ersten Weltkrieg. Nach dem zweiten Weltkrieg äußerte sie sich verschieden in
Ost und West. Dass sie beginnen konnte, liegt in der Arbeiterklasse selbst.
Das wäre zu sehen gewesen mit Marxens Entfremdungs-Lehre. (Dazu Rudolf Bahro
1979 „Die Alternative“ und neunzehn Jahre zu spät von mir: „Marx und Moritz –
Unbekannter Marx – Quer zum Ismus“) Doch in Lehrbüchern des DiaMat wurde das
nicht reflektiert, auch nicht das Phänomen der „Spaltung“ selber, das bis
hinein in Seelen reicht. Fingerzeige darauf wurden aus Texten zum
dialektischen Widerspruch gestrichen. Das fehlt nun in den Lehrbüchern, die
kein gutes Zeugnis ablegen für das Land, für das sie stehen sollten.

\paragraph{3. Anmerkung:}
Unterbelichtet ist in den Lehrbüchern der Umgang mit Widersprüchen. Es fehlt
„De-Eskalation versus Eskalation“. Es fehlt das Philosophikum „Kreativität“,
also „Schöpfertum“. Ein Professor des histMat meinte, „Kompromisse lösen
Widersprüche.“ Aus Achtung vor seinem lauteren Charakter verschweige ich
seinen Namen. Doch auch die Theorie strategischer Spiele dient nur der
Selektion von Varianten im Rahmen konstanter Repugnanzen. Hingegen wäre
kreativ, durch neue Strategien Widerspruchslösung anzubahnen. Deutlich wird
das vorm Hintergrund mathematischer Modelle. Was unter „Optimierung“ fällt,
ist Kompromiss, also Änderung im Rahmen bestehender Verhältnisse. Unter Lösung
fällt dagegen die strukturelle Änderung bestehender Konstellationen. Das führt
hin zum Wesen von „Kreativität“.

Deshalb sei ein Phänomen angedeutet, das in der Deutschen Demokratischen
Republik auf Dialektik und Kreativität orientieren sollte. Das Phänomen wurde
„Erfinderschule“ genannt, etwas unglücklich, weil unter „Erfinden“ oft
„Märchenerzählen“ verstanden wird. Besser wäre gewesen „Workshop zu
Widerspruchszentrierter Innovations-Methodik“. Die Methodik wurde geprägt von
Hans-Jochen Rindfleisch, Rainer Thiel und Hansjürgen Linde, letzterer führt
das Phänomen weiter in Bayern. Zwei der Autoren sind Verdiente Erfinder der
DDR und promovierte Ingenieure, Linde wurde in Bayern Professor, leitet zwei
Institute und ist gefragter Partner der Industrie. Das Phänomen
„Erfinderschule“ wurde mehrmals dokumentiert, durch den Ingenieurverband der
DDR und danach mit Hilfe von Freunden an Rhein und Isar sowie mit
Fördermitteln eines Bundesministeriums.

Kurzum, das Phänomen wurde in mehrtägigen Workshops mit Ingenieuren
praktiziert an Problemen aus realen Betrieben der DDR. Brainstorming diente
der guten Laune. Danach zwecks Provokation das „Inverse Brainstorming“.
Anschließend viele Fragen: Welchen Bedürfnissen entspringt das Problem? Wie
hat es sich entwickelt? Welche technischen, ökologischen, ökono"|mischen
Parameter bestimmen es? Aus Kundensicht und aus der Sicht des Betriebes? Das
alles wurde in einer Matrix erfasst. Und dann ging es richtig los: Wie müssten
wir die Parameter-Werte ändern, wenn etwas Gutes entstehen soll? Wir
forderten: „Kollegen, schraubt die Werte hoch, habt Mut, wir wollen etwas
Neues, das obendrein vernünftig ist!“ Wenn nun -- mit der Tabelle
experimentierend -- die Ingenieure beginnen, die wünschbaren Werte-Variationen
miteinander in Beziehung zu denken, z.B. Werte der Geschwindigkeit, des
Gewichts, der Sicherheit, der Handhabbarkeit, der Kosten und alles das bei
extrem knappen Ressourcen, dann dauert es nicht lange, und die Ingenieure
rufen: „Das geht nicht, da kommen wir in Widersprüche.“ Dann habe ich gesagt:
„Aha, die Profs haben euch irregeleitet.“ Ich füge hinzu: Im Fachwissen waren
die Ingenieure der DDR vortrefflich ausgebildet. Aber betreffend Dialektik
waren sie irregeführt. Man hat ihnen eingetrichtert: Wenn in Ingenieuraufgaben
Widersprüche auftreten, müssen die wünschbaren Parameter-Werte heruntergedreht
werden. Und von Philosophen wurden wir Heuristiker befehdet, weil wir
Dialektik praktizierten. Hörz war einsame Ausnahme. Auf einem Kolloquium
wurden wir beide von jüngeren Philosophen angegriffen.

Wir Erfinderschul-Methodiker haben mit führenden Profs der Hochschulen
erbittert gerungen. Erst anno 1992 hat deren Primus öffentlich bekannt: „Ja,
in einer Ingenieuraufgabe, die auf Neues zielt, müssen Widersprüche konzipiert
werden, um Neues zu entwickeln.“ Natürlich hatten wir Erfinderschul-Leute eine
tief gegliederte Methodik geschaffen, auf dreihundert Druckseiten nachlesbar.
Dort haben wir in hundert Schritt-Empfehlungen und vielen Erläuterungen
gezeigt, wie man durch Antizipieren von Entwicklungs-Widersprüchen zu
Lösungsansätzen kommt. Lösungen haben wir auch erarbeitet. Die
Lösungs-Empfehlungen sind ihrerseits durch Dialektik inspiriert, zum Beispiel
„Spalten von Objekten“ und „gegenseitiges Kompensieren der Komponenten“.
Einfachste Paradigmen sind das Kompensationspendel und die nachempfundene
Erfindung des Schiffsankers. (Vergleichbare Kompensationen werden auch in der
Mathematik praktiziert, z.B. beim Integrieren per Substitution oder -- zwecks
Radizieren der quadratischen Gleichung -- in Gestalt der quadratischen
Ergänzung.)

Kurzum, indem wir in Systeme von Parametern durch Variation oder durch Spalten
von Objekten gleichsam Power einbrachten, begannen in den antizipierbaren
Werte-Verlaufslinien Divergenzen zu entstehen bis zur Unliebsamkeit. Wir
präsumierten, wie per Variation Widersprüche entstehen. Das Gesamtgeschehen
aus Spaltung bis zum Gegensatz ist der dialektische Widerspruch.

Die erste neuzeitliche Anregung, nachzudenken über die Spaltung von Monolithen
in auseinandergehende, zuerst nur differenzische, bei fortgesetzter Variation
bald auch entgegengesetzte Komponenten, die erste Anregung jenseits von Hegel
und Marx empfing ich durch Genrich Saulowitsch Altschuller (Baku, Moskau).
Altschuller hatte darauf hingewiesen und anhand einer Tabelle erläutert, dass
bei extensiver (tatsächlicher oder antizipierter) Variation technischer
Objekte deren Parameter -- physikalisch und unterm Gesichtspunkt ihrer
Nutzbarkeit auch ökonomisch -- oft „in Widerspruch“ zueinander geraten. Im
Deutschsprachigen wurde erstmalig darauf verwiesen von mir in Deutsche
Zeitschrift für Philosophie 1976. Bald entwickelte ich dazu eine mehr
mathematisch anmutende, simple Darstellungsweise, die 1982 auch Eingang
gefunden hat in das erste Erfinderschulmaterial des Ingenieurverbands. Das war
den meisten Ingenieuren und Naturwissenschaftlern anfangs zu neu, einige
begeisterten sich nur an der von mir verwandten Symbolik. Sofort verstanden
wurde es von dem auch theoretisch hochgebildeten Erfinder Dr.-Ing. Hans-Jochen
Rindfleisch, sodass es in den Berliner Erfinderschulen bald zur praktischen
Anwendung kam, über die ich soeben berichtet habe. Hansjürgen Linde hat eine
eigene Version geschaffen unter dem treffenden Titel „Widerspruchsorientierte
Innovations-Strategie“ (WOIS), zum ersten Mal zusammenhängend dokumentiert in
Lindes Dissertation (TU Dresden 1988), die längst zur Grundlage zahlreicher
Schriften, Workshops und Kongresse geworden ist. In der prononciert
kybernetischen Literatur habe ich darauf noch keine Bezüge gefunden.

\paragraph{4. Anmerkung:}
In der kybernetischen Literatur ist aber von Anfang an Bezug genommen auf die
Interaktionen zwischen technischen Objekten und ihrem Umfeld, indirekt auch
innerhalb technischer Objekte, wenn man diese als Systeme sieht. Daraus ergab
sich auch eine fundamentale Vertiefung des simplen Wechselwirkungsbegriffes
der Philosophie, die am mechanischen Materialismus orientiert war und noch
ist. Als Paradigma gelten dort die Newtonschen Grundgesetze. Sie bleiben
relative Wahrheiten. Nicht alle Wechselwirkung ist Rückkopplung, aber ohne den
Rückkopplungsbegriff ist der traditionelle Wechselwirkungsbegriff arm und kann
bestenfalls dem sog. 1.~Grundzug der Dialektik zugeordnet werden, durch den
auf die Omnipräsenz und Vielfältigkeit von Zusammenhängen hingewiesen wird.

Dass geringe Störungen, die aus Gespaltetsein einheitlicher Aggregate
resultieren, sich hochschaukeln können, lehrt die Kybernetik positiver
Rückkopplungen. Das Paar aus Störung und Rückkopplung macht dialektischen
Widerspruch. Nachvollziehbar ist das mit Differentialgleichungen. Darüber habe
ich 1962/63 berichtet in Deutsche Zeitschrift für Philosophie und
differenzierter in „Quantität oder Begriff“ 1967, wobei ich auch auf
Forschungen von Lewis F. Richardson von 1960 (englischer Physiker und
Friedensforscher) zurückgegriffen habe. Leider ist das von Philosophen und
Wissenschaftstheoretikern, deren es Hunderte gab, nicht zur Kenntnis genommen
worden.

\paragraph{5. Anmerkung:}
Nicht alle Träume betreffs Applikation von Differentialgleichungen außerhalb
des technisch-physikalischen Bereichs sind in Erfüllung gegangen. Das erkannte
ich während der Arbeit an dem umfangreichen Text, den ich 1967 betitelte
„Quantität oder Begriff?“, und als ich das Vorwort verfasste, war mir klar
geworden: Das nächste Buch muss „Mathematik – Sprache – Dialektik“ heißen.
Simpelste Überlegungen deuten an, worauf das hinausläuft:

Weil dialektische Widersprüche in Wachstumsprozessen entstehen, müssen
Wachstumsprozesse verstanden werden. Das wird aber verhindert, weil die
öffentliche Meinung darauf dringt, höchstens einzelne Ereignisse zu
betrachten.  Nicht besser steht es, wenn Ökonomen den volkswirtschaftlichen
Prozess in Jahresabschnitte und Wachstumsraten, zum Beispiel in jährliche
Zinsraten, zerstückeln: Das kontinuierlich verlaufende Jahr wird auf den
Silvesterabend reduziert. In Wirklichkeit können großflächige, zum Beispiel
volkswirtschaftliche Prozesse als annähernd stetige Veränderungen zum Beispiel
des Inlandsprodukts verstanden werden. Von einzelnen Ökonomen wird das
tatsächlich anerkannt, sie experimentieren mit Wachstumsmodellen in der
Sprache der Differentialgleichungen, wobei sie natürlich oft auf hypothetische
Werte von Koeffizienten angewiesen sind. Wenn Makro-Ökonomie annähernd
verstanden werden soll, ist beides unvermeidbar: Die Differentialgleichungen
und der hypothetische Charakter von Parameter-Werten. Die meisten Ökonomen
aber und fast alle Normalverbraucher verstehen nichts von
Differentialgleichungen.  Deshalb wird Makro-Ökonomie fast überhaupt nicht
verstanden und ist ein Tummelfeld für Kartenleger, mit allen Konsequenzen für
das politische Leben.

In meinem Umfeld bemerkten einzelne Ingenieure und Physiker, dass hinter der
gebräuch"|lichen Zinseszinsformel die Differentialgleichung
\begin{gather*}
  y = a\cdot \frac{d\,y}{d\,t}
\end{gather*}
steht, deren Lösung die Exponentialfunktion ist. Nächster Schritt war die
Einsicht, dass nicht alle Bäume in den Himmel wachsen. Es muss also auch an
Sättigungsprozesse gedacht werden, im einfachsten Fall an die logarithmische
(richtig: logstische -- HGG) Funktion. Also wäre auch an komplexere
Differentialgleichungen zu denken gewesen. Diese hätten anregen können, über
die zugrunde liegenden makro-ökonomischen Prozesse und deren Beeinflussung
nachzudenken. Doch das Publikum, dessen Aufklärung den Philosophen oblegen
hätte, verweigerte sich schon den allerersten Einsichten. Das war einer allzu
primitiven, doch universell verbreiteten Auffassung von Realität geschuldet:
Die meisten Menschen berufen sich in ihren Urteilen auf den augenblicklichen
Zustand der Objekte, die sie zu sehen glauben. Mit Heftigkeit und
Leidenschaft, die bis zum Fanatismus geht, behaupten sie: „Ich bin Realist!“
Muss man das glauben?

Wer auch nur ein wenig Umgang mit Differentialgleichungen hat, fühlt sich zur
Widerrede herausgefordert. Die sich Realisten nennen, berufen sich auf den
augenblicklichen Zustand, auch wenn sie zurecht unterstellen könnten, dass das
Objekt mitsamt vielen seiner Eigenschaften veränderlich ist. Aber sie greifen
sich aus der Lebenskurve, die man hypothetisch in ein Koordinatensystem
eintragen könnte, nur den Augenblickswert, also einen einzigen Punkt $y = t_0$
der Kurve. Man braucht aber keinen großen IQ zu haben um zu wissen, dass in
der Regel jedes $y$ einer Funktion zugeordnet ist und dass insbesondere diese
Funktion Differentialquotienten enthalten wird. Diese bringen zum Ausdruck,
dass das $y$ nicht nur schlechthin veränderlich ist, sondern auch mit einer
gewissen Geschwindigkeit (Steilheit), Beschleunigung (Steilheit der Steilheit)
und so weiter.

Die Leute, die am lautesten schreien, Realisten zu sein, sind es nicht. Das
zeigt sich massenhaft in Diskussionen, in denen Überwindung von
Unzuträglichkeiten ansteht. Ihr Geschrei ist eine sich selbst erfüllende
Behauptung: Man behauptet, es bewegt sich nichts, also bewegen sich die
Menschen nicht und warten, bis das sogenannte Sein über sie kommt. Die sich
Realisten nennen, begreifen nicht, dass auch die Veränderung in jedem
Zeitpunkt zur Realität gehört. Wer das nicht begreift, wird auch nicht kreativ
werden. Wer Umgang mit Differentialgleichungen hatte, dem hat sich diese
Welt-Ansicht eingeprägt. Schon in dieser elementaren Bewandtnis hat sich
Mathematik als Sprache der Dialektik gezeigt.

Daraus folgt, dass Philosophie, welche die Sinn-Fragen des Lebens beantworten
will, an den Fragen nach Struktur der Wirklichkeit nicht vorbeikommen kann.

\paragraph{6. Anmerkung:}
Systeme von Differential-Gleichungen gehören zu den Paradigmen, an denen sich
die philosophische Widerspruchs-Dialektik hochziehen kann. Sogar multipolare
Systeme gewinnen da an Transparenz. Und sind Gleichungen nichtlinear, können
neue stabile, auch unumkehrbar unerwünschte Zustände eintreten. Dann ist mit
Störgrößen-Ausregeln nichts mehr zu machen. Das begriff ich – ungewollt -- als
mathematisch interessierter Bürger der DDR vor 45 Jahren. Im Ausland aber –
was mir erst zehn Jahre später auffiel - sind weitere Formen der
Nichtlinearität erkannt worden. Dazu einige Worte.

Schon im Gymnasium lassen quadratische Gleichungen einen Spaltpilz im
Lösungsgeschehen erkennen. Längst werden auf Computern brisantere
Nichtlinearitäten realisiert: Werden nichtlineare Ausdrücke, im einfachsten
Fall der quadratische Iterator
\begin{gather*}
  y = x_{n+1} = a\cdot x_n (1-x_n),
\end{gather*}
immer wieder auf sich selber angewendet, und werden zusätzlich Koeffizienten
exzessiv variiert – das entspricht Energie-Einträgen in das Geschehen --, dann
zeichnet sich auf dem Bildschirm das sogenannte \emph{Feigenbaum-Diagramm} ab:
Anfangs einheitliche Bahnen spalten sich in zwei und mehr Zweige. Fachleute
subsumieren das in der \emph{Chaos-Theorie}, die eine dialektische
Prozess-Theorie ist. Das Feigenbaum-Diagramm und einfachste Implikationen habe
ich vor Jahren in „Die Allmählichkeit der Revolution“ deutlich zu machen
versucht, weil es auch für Quale-Umschlagen relevant ist. Schon bescheidenste
Auswertungen dieser Theorie erbringen Aufschlüsse darüber, wie dialektische
Widersprüche entstehen.

Der quadratische Iterator und Hegels Entwicklung von „Sein“ lassen ahnen, was
Dialektik ist. Wird der quadratische Iterator praktiziert, kommt (bei manchem
Anfangswert $x_1$ und manchem Wert von $a$) eine überraschende Folge von
Werten $x_n$ heraus. Hegels Begriffsentwicklung beginnt mit dem „Sein“. Und
was tut der Dialektiker Hegel? Er entwickelt – mit der Sturheit eines Schelms,
wie ein Computer – den Inhalt des „reinen Seins“:

„Sein, reines Sein, -- ohne alle weitere Bestimmung. In seiner unbestimmten
Unmittelbarkeit ist es nur sich selbst gleich und auch nicht ungleich gegen
Anderes, hat keine Verschiedenheit innerhalb seiner, noch nach außen. Durch
irgendeine Bestimmung oder Inhalt, der in ihm unterschieden, oder wodurch es
als unterschieden von einem Andern gesetzt würde, würde es nicht in seiner
Reinheit festgehalten. Es ist die reine Unbestimmtheit und Leere. – Es ist
nichts in ihm anzuschauen, wenn von Anschauen hier gesprochen werden kann;
oder es ist nur dies reine, leere Anschauen selbst. Es ist ebensowenig etwas
in ihm zu denken, oder es ist ebenso nur dies leere Denken. Das Sein, das
unbestimmte, unmittelbare, ist in der Tat Nichts, und nicht mehr noch weniger
als Nichts.“ Könnte das nicht jeder Bürger nachvollziehen, der gefragt wird:
Was könnte dir einfallen, falls dich jemand nach dem reinen Sein befragt?

Vom „Nichts“ aus entwickelt Hegel das „Sein“ und aus beiden, die nicht
dasselbe sind, doch sich als dasselbe erweisen, das „Werden“. Was Hegel hier
geschrieben hat, ist eine gewollte Persiflage des Dialektikers auf die dumme
Philosophie. Es ist, als hätte sich Hegel damit warm gelaufen, denn jetzt wird
es ernst. Jetzt nämlich beginnt Hegel erst richtig: Alle wesentlichen Begriffe
der Philosophie, alle wesentlichen Semanteme, die Menschen benutzen, um über
Probleme des Erkennens zu sprechen, entwickelt Hegel aus ihren elementaren
Stadien und in ihren gegenseitigen Relationen, darunter die Semanteme
„Quantität“ und „Qualität“, sodass auch sichtbar wird, wie viele verschiedene
Bedeutungen mit derartigen Worten verbunden werden, ohne dass sich Nutzer
solcher Worte dessen bewusst sind. Damit hat Hegel – die Geschichte der
Philosophie und des menschlichen Erkenntnisvermögens nachvollziehend -- nicht
nur ein dialektisches System philosophischer Begriffe geschaffen, sondern auch
ein System, das als ein Muster „Konstruktiver Systematik“ der Dialektik gelten
kann, wie Hans Heinz Holz gefordert hat und wie ich angeregt habe seit
Jahrzehnten. Solche Muster müssten – nach dem Vorbild Hegels – in größerer
Anzahl geschaffen werden. Eine kurze Charakteristik seiner Dialektik gibt
Hegel 1820 in den „Grundlinien der Philosophie des Rechts“ § 31.

Dabei wird es hilfreich sein, Hegels „Logik“ zu didaktischen Zwecken in
vereinfachter Form darzustellen, als Handreichung zum Lernen, als erste
Anregung zum Verstehen, wie „konstruktive Dialektik“ aussehen kann. Schon die
Fähigkeit zum Verständnis von Hegel, Marx und aller Dialektik muss trainiert
werden.

In Hegels „Logik“ steckt zugleich die tiefe Wahrheit: Kommt heraus aus der
Kontemplation, aus dem Frust, seid aktiv, handelt, entwickelt die Dinge aus
sich selbst heraus, gleich, ob es die inhaltsvollen nachfolgenden Begriffe wie
„Quantität“ und „Qualität“ sind oder ob es der Begriff „Zahl“ ist und die
Zahlensysteme – von der Mathematik und von Hegel entwickelt -- oder ob es der
quadratische Iterator oder sonst was ist. Selbstentwicklung ist geradezu das
Wesen des Iterators, Mathematiker sprechen von „Rekursiver Funktion“, besser
hieße es „Prokursive Funktion“.

Eine Abart solcher Entwicklungen demonstriert der Graphiker
M. C. Escher\footnote{Titelbild von (Thiel 2000).}: Escher beginnt mit
simplen, völlig exakten Dreiecken. In den Augen des Künstlers sind das
prokursive Objekte; er sieht in ihnen die Anlage zur Selbstbewegung. Sogleich
lässt Escher die Dreiecksseiten zu sanften Linien aufwallen wie die Oberfläche
ruhenden Wassers, wenn es heiß und immer heißer wird. In einem zweiten Schritt
lässt Escher die Wellung stärker werden, in einem dritten Schritt noch mehr,
wobei sich zugleich die Ecken des ehemaligen Dreiecks auszustülpen beginnen,
immer mehr, bis sie sich der Gestalt von Flügeln nähern. So geht es weiter.
Beim elften Schritt – annähernd allmählich -- sind aus den ursprünglichen
toten Dreiecken hochvitale Möwen geworden, die sich in den Lüften vergnügen.

Der Graphiker Maurits Cornelis Escher hat eine Vision gehabt und hat die
Entwicklung -- der Vision entsprechend und als Künstler gestaltend – den
Dreiecken zukommen lassen: Die Dreiecke sind zunächst Symbole der Starre, sie
werden zu Möwen, zu Symbolen der Vitalität. Das scheint einem Elementarprozess
zu entsprechen, von dem auch die Entwicklung von Personen in der schöngeistige
Literatur lebt. Natürlich sind in der Regel in einem Werk zwei und mehr
solcher Elementarprozesse miteinander verflochten. Dem entsprechen die beiden
ersten Grundzüge der Dialektik nach Lenin:
\begin{quote}
  1) die Bestimmung des Begriffs aus ihm selbst / das Ding selbst soll in
  seinen Beziehungen und in seiner Entwicklung betrachtet werden; 2) das
  Widersprechende im Ding selbst / das Andere seiner,/ die widersprechenden
  Kräfte und Tendenzen in jedweder Erscheinung.“ (LW 38, S. 212-13)
\end{quote}
Das zielt ins Innerste des Lebens und seiner literarischen Gestaltung, wurde
aber in der DDR nicht zitiert. Wenn es auch richtig ist, dass man literarische
Werke nicht zu Tode analysieren soll – man könnte probieren, durch Hervorheben
von Linien der Entwicklung und der Selbstentwicklung die Dialektik so manchem
Leser lebendig werden zu lassen.

Vorstehende Beispiele – pädagogischer gestaltet – könnten als Muster des
lehrbaren, trainierbaren dialektischen Denkens fungieren, das zu
Gegensatzumschlägen und zu Spaltungen des scheinbar Monolithischen führt.
Zugleich lässt sich darüber nachdenken, dass „Gegensatz-Umschlagen“ und
„Spaltungen“ komplementäre Etiketten für dialektische Prozesse sind. So lernt
ihr, liebe Leute, auch euch selber zu entwickeln, ihr seid doch keine
Trauerklöse, keine Monolithen.

Inzwischen habe ich auch empirisches Material zu Spaltungen in
Bürgerbewegungen und Parteien. Ich musste Ursachen und Formen solcher
Spaltungen erkennen, auch unter gnoseologischen und psychologischen Aspekten.
Das alles wäre Stoff zur Lehre von Dialektik der Ausgebeuteten und
Gedemütigten. Dem kann man das Etikett „innere Widersprüche“ anheften, aber
man muss es begreifen, um es zu gestalten. Dazu wiederum muss die Dialektik
von quantitativen und qualitativen Wandlungen verstanden werden. Dem war in
Stalins Nomenklatur der 3.~Grundzug der Dialektik zugeordnet:

\section{Anmerkungen zum dritter Grundzug der Dialektik (Stichwort
  „Quale-Wandel“)} 

Stalin hatte mit seinem dritten Grundzug der Dialektik viele Menschen
beeindruckt. Das Personal des sog. Marxismus-Leninismus kam bis heute nicht
los von Stalin: Stalin quer zu Marx, niemand bemerkte es.

Stalin unterstellte, Wandlungen seien anfangs nur quantitativ, nicht
qualitativ von Anfang an, Quanta müssten sich erst ansammeln, um ins
Qualitative umzuschlagen. Erst das eine, dann das andere. Daraus folgert
Stalin, Quale-Wandel würde plötzlich eintreten. Das ist Bürokraten-Logik: Erst
gar nichts, dann alles auf einmal. Es ist leider auch die Logik des Kleinen
Mannes. Aber es stimmt nicht einmal für das Wasser in realen Gefäßen. Niemand
hat je erlebt, dass flüssiges $H_2O$ im Kochtopf plötzlich verdampft.
Schmelzpunkt und Siedepunkt werden lokal erreicht, dabei entstehen
retardierende Prozesse, komplizierte Wechselwirkungen. Integral gesehen
wandeln sich reale Wassermengen allmählich. Stalin widerspricht dem
Augenschein.  Das ist georgische Priesterschule.

Doch überall, wo Augen-Schein wirklich trügt, besteht Stalin auf dem Schein.
Die Frage ist nämlich überhaupt nicht, ob Quale-Wandel von Anfang an sichtbar
ist. Materialisten hätten Stalin subjektiven Idealismus vorwerfen müssen. Wahr
ist nämlich: Wir abstrahieren von qualitativem Wandel. Gründe liegen in der
objektiv bedingten Praxis. Indirekt lernen das Ingenieure und Physiker im
ersten Semester. Sie arbeiten mit vereinfachten, mit linearisierten Formeln,
sonst wird alles zu umständlich. Bei exzessiven Wandlungen – das wissen
Physiker und Ingenieure – gelten aber Funktionen, die nichtlinear sind, wo
also Variable in einer von eins verschiedenen Potenz stehen. Nur wird das aus
praktischen Gründen vernachlässigt. In polemischer Überspitzung kommentierte
ein Mathematiker: Mit überzogenen Linearisierungen hat man „die einzige
Möglichkeit eingebüßt, [\ldots] sich mit der Realität auseinanderzusetzen“.
(Leon O. Chua. Zitiert nach Peitgen, Jürgens, Saupe, C.H.A.O.S Seite 211)

Gewiss ist das überspitzt. Es gibt Lehrbücher der dezidiert Nichtlinearen
Elektro-Technik. Auch Ballistiker der Artillerie kennen die Nichtlinearität.
Doch Unteroffiziere konnten damit in Schwierigkeiten geraten. Erich Loest
erzählt in seiner Biografie, wie ein Feldwebel den Hitlerjungen erklärte, „das
Geschoss würde nach dem Verlassen des Laufs eine Weile geradeaus fliegen, bis
Erdanziehung und Luftwiderstand die Flugbahn krümmten,“ worauf die
Gymnasiasten behaupteten, „das stimme nicht, sofort wirkten diese Faktoren,
schon im ersten Millimeterbruchteil.“ Der Feldwebel wiederholte seine Ansicht,
doch der Gymnasiast Erich Loest „blieb hartnäckig, der Feldwebel jagte den
Aufsässigen um den Block. [\ldots] Die Unteroffiziere sahen in L. einen
Schnösel von der Oberschule, der sich über sie lustig machte“.

Praxisbedingt sind viele Ingenieurformeln linearisiert, und ausschließlich
linearisiert zu denken sind die meisten Menschen gewöhnt. Für die Philosophie
aber geht es um mehr. Hegel war es, der die Nichtlinearität erkannt hat, als
er die Kategorien „Qualität“ und „Quantität“ untersuchte. In „Wissenschaft der
Logik“, Lehre vom Sein, spricht Hegel von Potenzen-Verhältnissen. Damit wird
von Anfang an nicht nur quantitativer, sondern auch qualitativer Wandel
ausgewiesen. Das wäre im gesellschaftswissenschaftlichen Pflichtstudium in der
DDR leicht erklärbar gewesen. Man hätte nur Marx und Engels lesen müssen.
Diese beiden benutzten zur Erläuterung ein Beispiel nach Napoleon mit
unterschiedlichen Reiterverbänden:
\begin{itemize}\itemsep0pt
\item 2 Mameluken schlagen jeweils 3 Franzosen.
\item 100 Mameluken sind 100 Franzosen gleichwertig.
\item 300 Mameluken können von 300 Franzosen besiegt werden.
\item 1500 Mameluken werden jedes Mal von 1000 Franzosen geworfen.
\end{itemize}
Nachlesbar in (MEW 14/308, 20, 23). Dort findet man die Nichtlinearität (MEW
20, S. 120). Marx war durch das Modell angeregt, die Möglichkeit von
Mehrwertproduktion zu begründen. Verschieden große Kooperationen nebeneinander
hat sich Marx vorgestellt, also etwa eine Kooperation aus 10 Werktätigen, dann
aus 20, aus 50, aus 100 usw. Marx zeigte, dass sich Möglichkeit zur
Mehrwert-Produktion aus nichtlinear variierenden Größenverhältnissen ergibt
(MEW 23, Kapitel „Kooperation“), analog, wie durch Wandel von Quanta das Quale
der Reiterverbände von Anfang an wächst. Einzelkämpfer-Quale schlägt
allmählich um. Mit Blick auf allmählichen Wandel sagte schon Goethe: „Vernunft
wird Unsinn, Wohltat Plage“, auch das war Friedrich Engels aufgefallen. Quale
geht allmählich über aus Unter- und in Überlegenheit, sie wandelt sich
permanent mit dem Quantum. Und selbst, wenn Bürokraten einen Punkt markieren,
der Übergang vollzieht sich allmählich. Fuzzy-Geometrie macht das noch
deutlicher. Clausewitz hat das vorweggenommen. Wer Meilensteine setzt, will
sich vor allem selber feiern.

Hält man sich das Modell von Napoleon und Marx wiederholt vor Augen, erkennt
man auch den Zusammenhang zwischen Nichtlinearität und dem Weltgesetz „Das
Ganze ist mehr als die Summe der Teile“. Das Ganze einer Wandlung äußert sich
von Anfang an gegenüber linearem Wachstum als Surplus-Effekt, ausgedrückt in
Nichtlinearität. Man spricht dann auch von progressivem bzw. degressivem
Wachstum. Leider gilt in den Schulen fast nur die lineare Algebra, wo es egal
ist, ob und wie man in mehrgliedrige Additionen Klammern einstreut. So wird
durch den Schulunterricht nicht nur ein schiefes Bild der Mathematik erzeugt,
sondern auch eine Abstraktion vom Ganzen, das mehr ist als die Summe der
Teile. Eben nichtlinear.

Die bürokratische Fasson der linearen, rein linear-summativen Anhäufung
quantitativen Wandels hätte ersetzt werden müssen durch die Frage: Wieso
unterscheiden wir überhaupt quantitative und qualitative Änderungen?
Steinzeit-Menschen kannten diese Unterscheidung nicht. Wieso? Welche Rolle
haben Abstraktionsprozesse gespielt?

Stalin hat religiöse Illusionen gestützt per Abstraktion vom permanenten
Quale-Wandel. Er hätte sagen können: Wenn deutlich wird, dass sich die
Abstraktion nicht mehr halten lässt, dann glauben wir, es träte ein
plötzlicher Übergang ein. Auch die Alltags-Menschen nehmen die Illusion für
bare Münze, weil sie überwiegend keine Prozesse wahrnehmen, sondern singuläre
Ereignisse. Die Medien sind ganz geil, den Menschen Ereignisse zu bieten und
nichts als Ereignisse, die punktuell sind, ohne Entwicklung zu reflektieren.
Das ist Opium.

Von religiösem Wahn möchte ich auch sprechen, wenn Leute, die sich als links
verorten, der Meinung huldigen: Jetzt haben wir Kapitalismus, da können wir
sowieso nichts machen, da bleiben wir am besten zu Hause, bis ein großer
Kladderadatsch den Kapitalismus hinweggefegt hat, dann sind wir wieder da.

Bei derart absurder Auffassung vom Quale-Umschlagen ist auch das Verhältnis
von Reformen verschiedenen Typs der Dialektik entzogen. Es gibt nämlich
Reformen ganz verschiednen, ja entgegengesetzten Typs. Typ A ist
entwicklungsneutral. Beispiel: Hartz IV als Zusammenlegung von
Arbeitslosenhilfe und Sozialhilfe. Typ B festigt bestehende
Herrschafts-Verhältnisse. Beispiel: Hartz IV als Instrument zur
De-Vitalisierung von Arbeitslosen und als Schreckmittel für die, die noch
einen Job haben. Die Typen A und B überwogen bisher in der Geschichte. Dagegen
bewirkt Reform des Typs C zweierlei: Ein rasches Ergebnis wird erzielt, das
zugleich ein Ergebnis ist, welches den Wandel des gegenwärtigen Quale in eine
neue Gesellschaft voranbringt. Das Kräfteverhältnis wird gewandelt, Spielräume
werden verändert: Enger für die Konzerne, weiter für die Notleidenden. Ein
Beispiel wäre allgemeine Verkürzung der Arbeitszeit, Arbeit und Freizeit für
alle, Spielräume und Kraft, um politische Freiheiten wahrzunehmen, die durch
das Grundgesetz verbrieft sind: Kampf um Menschenwürde und Schach dem
Eigentum, das seine Pflichten verletzt. Menschliche Kräfte würden der Abtötung
entrissen. Würden Reformen des Typs C erkämpft, ist der Kapitalismus nicht
mehr wie zuvor -- es entstehen Elemente einer Gesellschaft Aufrecht gehender
Bürger. Das Philosophikum ist strategisch bedeutsam. Aber es ist ja noch nicht
mal zur Kenntnis genommen worden, dass Karl Marx in seinem Hauptwerk \emph{Das
  Kapital, Erster Band} geschrieben hat: „[\ldots] abstrakt strenge
Grenzlinien scheiden ebensowenig die Epochen der Gesellschafts- wie der
Erdgeschichte.“ (MEW 23, S. 391)

Die Menge der Marx-Engels-Dokumente, die dasselbe bedeuten, ist erdrückend,
die Gedanken zur Nichtlinearität in MEW 14, 20 und 23 gehören dazu. Zwölf
Jahre nach der Wende verwies ich einen Professor für Marxismus-Leninismus
darauf. Da schrie der Professor „nein, nein, nein.“ Als ich danach in meinem
Vortrag ausführlicher über die Quellen gesprochen hatte, sagte der ML-Prof.
nur das eine: Marx wäre eben auch nur ein Mensch gewesen. Da hatte ich Mühe,
meine Verachtung zu zügeln.

Indem Hegel die Zwangsläufigkeit von Potenzen-Verhältnissen enthüllt, also von
Nichtlinearität, beweist er, dass sich Quale permanent wandelt, wenn sich
etwas quantitativ wandelt. Ausgerechnet in diesem Punkt versagt Lenin, der
Hegel hoch verehrt hatte. Lenin schreibt: „Ohne Studium der höheren Mathematik
ist das alles unverständlich.“ (LW 38, S. 110f).

Da Lenin Hegels „Potenzenverhältnis“ nicht verstehen konnte, fragte er:
Wodurch unterscheidet sich der dialektische Übergang von einer Qualität zur
anderen? Lenins Antwort: „Durch das Abbrechen der Allmählichkeit.“ (LW 38,
S. 272. Siehe auch S. 339). Lenin spricht von „Sprung“. Heute reden Schwätzer
gar von „Quantensprung“. Hegel hatte aber gemeint: Veränderung des Quantums
„ist zugleich wesentlich der Übergang einer Qualität in eine andere“ (LW 38,
S. 345). Hegel konzediert Allmählichkeit, er fügte nur hinzu: Allmählichkeit
erklärt nicht den Quale-Wandel. Hegels Antwort liegt in der Dialektik, die er
enthüllt auch via Nichtlinearität.

Hat Lenins Fehlinterpretation Einfluss gehabt auf die Geschichte? Ich glaube
„ja“, und zwar unmittelbar nach dem Oktober-Aufstand. Das habe ich vor fünf
Jahren untersucht. Erst im April 1918 relativiert Lenin seine
Plötzlichkeitsthese (LW 27, S. 264) und ersetzt sie durch die Frage „Langsamer
oder schneller“. Das ist aber auch noch nicht Hegel oder Marx. Einheit von
Quantum und Quale ist keine Frage der Zeit, sondern der Dialektik eines
Phänomens, das durch Abstraktion zerspalten ist. Abstraktion ist wie Feuer,
von welchem Schiller sagt: „Wohltätig ist des Feuers Macht, wenn es der Mensch
bezähmt, bewacht. [\ldots] Doch furchtbar wird die Himmelskraft, wenn sie der
Fessel sich entrafft.“

Die Himmelskraft vom Stamme „Abstraktion“ hat Hegel verstanden. Von Anfang an
in seiner Begriffsentwicklung sagt Hegel vom Quantum: „Die Gleichgültigkeit
der Bestimmtheit macht seine Qualität aus, d.i. die Bestimmtheit, die an ihr
selbst als die äußerliche Bestimmtheit ist.“ (S. 215). Oder: „Die Qualität des
Quantums [\ldots] ist seine Äußerlichkeit überhaupt.“ (S. 323). Oder „Die
Äußerlichkeit der Bestimmtheit ist die Qualität des Quantums.“ (S. 332). Und
so geht das bei Hegel von Anfang an in seiner Begriffsentwicklung, in der er
Wohltätigkeit und verheerende Kraft der Abstraktion recherchiert als
Dialektiker und Kriminalist (vgl. Logik I, S. 115). Menschen haben in
Jahrtausenden „Quantität“ durch Abstrahieren von „Qualität“ geschieden und
verselbständigt. Das hat beigetragen, Welt zu erkennen. Doch es hat auch
Folgen gehabt, Risiken und Nebenwirkungen. Dialektische Widersprüche sind
zwischen „Quantität“ und „Qualität“ entstanden. Hegel hat sie kenntlich
gemacht und zu überwinden gelehrt. Goethe, der sich mit Hegel gut verstand,
hat dazu ein Dichterwort parat: „Natur ist weder Kern noch Schale, alles ist
sie mit einem Male.“ Doch Bürokraten betonieren Abstraktionen. Bürokraten
applizieren Brachialgewalt, wie das heute an ostdeutschen Siedlungen und
Schulen praktiziert wird.

Mathematik und Philosophie pflegen unterschiedliche Ambitionen und Sprachen.
Hegel hat sie zum Nutzen beider Wissenschaften genial zur Korrespondenz
gebracht, auf 200 Seiten, auch Physik und Chemie im Blick. Hegel hat sogar
einige Begriffsentwicklungen der Mathematik vorausgespürt. Vor allem wollte
Hegel Dialektik als Wissenschaft. Ein ungeheures Anliegen! Hegels Logik hätte
da eine 3.~Auflage verdient, doch Hegel wurde nur 61 Jahre alt. Wo er den
Begriff „Matrix von Maßverhältnissen“ einbringt, stimmt noch alles, dort
klingt sogar ein fraktaler Gedanke an, nur das Wort „Matrix“ kennt Hegel noch
nicht, seine Interpreten erst recht nicht.

Hegel hat recht gegen die Philosophen. Nur -- auf den letzten 18 Seiten hat
Hegel Korrespondenzen realer Bewandtnisse nicht angemessen spezifiziert, das
haben seine Interpreten auch nicht bemerkt. Im Umfeld seines Reizwortes
„Knotenlinie von Maßverhältnissen“ hat Hegel Bezugsebenen von Beispielen
vermengt. Unbemerkt. Die so entstandenen Vogelscheuchen hat er dann
beschossen.  Das Reizwort „Knotenlinie“ sollten wir vergessen.
Matrix-Darstellung muss an seine Stelle treten.

Ganz richtig aber blieb Hegel durchgehend dabei, die Allmählichkeit von
Wandlungen erkläre nicht den Quale-Wandel. Doch permanenten Quale-Wandel hat
er nachgewiesen. Nach titanischer Arbeit ist er erschöpft. Da unterläuft ihm
der Fehler, den er auf 200 Seiten überwunden hatte. Dergleichen kann den
Cleversten passieren. Dem wackeren Einzelkämpfer Hegel werde das verziehen,
doch ganzen Scharen Lehrbuch-Machern? „Quod licet Iovi, non licet bovi.“

Übrigens hat Hegel auch die sogenannten Elenchen kommentiert. Nur hat noch
kein Hegel-Interpret bemerkt, was Hegel über Nichtlinearität alias
Potenzen-Verhältnis und über Elenchen schrieb. Lenin dagegen ist ehrlich
gewesen.

Hegel hat allmähliche Proportionsverschiebungen beim Wachstum von Städten
gesehen: Das Quantum ist die Seite, an der ein Dasein unverdächtig angegriffen
wird. Es ist die List des Begriffes, ein Dasein anzufassen, wo seine Qualität
nicht ins Spiel zu kommen scheint (346). Obwohl es daran keinen Zweifel gibt,
hat für die unausgegorene Idee überdimensionierter Luftschiffe und
Abwasseranlagen – für Giga-Projekte -- die Regierung Brandenburgs Hunderte
Millionen verschleudert. Da müsste Strafgesetzbuch § 266 „Untreue“ greifen.

Auch Wärmehaushalt und Körpermechanik von Tieren hängen ab vom Verhältnis
zwischen Körperlänge, Oberfläche und Volumen des Körpers: Oberfläche wächst in
zweiter Potenz zur Schulterhöhe, Volumen in dritter Potenz. Das beeinflusst
allen Stoffwechsel der Lebewesen und die Evolution. Jahrzehnte nach Darwin
wird dergleichen „Allometrie“ genannt und ist unübersehbar.

Nichtlinearität im Größen-Wandel eines Objekts bedeutet: Proportionen zwischen
dessen Komponenten – also Eigenschaften -- ändern sich. Mathematik macht das
verständlich. Zum Beispiel fürs Verhältnis „Kapital/Arbeit“ bedeutet das
Verschiebungen in den Handlungsspielräumen. Also ist Kapital nicht gleich
Kapital.

Quale-Wandel, der sich in Schaumkronen andeutet, wenn er in den Tiefen längst
im Gange ist, können wir verstehen, wenn wir nicht abstrahieren von der
Nichtlinearität in der Entwicklung von Relationen innerhalb eines Quale. Wenn
sich Relationen verschieben, dann wandeln sich Eigenschaften. Das hatte auch
Lenin Hunderte Male richtig gesehen. Eigentlich sind es Bürokraten und kleine
Leute, die allmählichen Quale-Wandel verleugnen, weil sie ihn in ihrer
begrenzten Weltsicht nicht wahrnehmen, sie ergötzen sich an Ereignissen als
den Schaumkronen auf der Oberfläche von Flüssen. Quale-Wandel von der
Daseinsform „Zeit“ her zu definieren ist schlichtweg unangemessen. Von
„Allmählichkeit der Revolution“ spreche ich, um zu provozieren. Dialektik im
Inneren kann sich in zeitlicher Form äußern, doch die Geschwindigkeit ist
nicht ihr Wesen.

Hegel hat im Zusammenhang mit Potenzverhältnissen nicht nur das Wort „Maß“
gebraucht. Hegel hat mehr noch gedacht an multiple Maß-Verhältnisse. Das sind
zugleich Indikatoren der Multipolarität des Widerspruchssyndroms. Multiple
Maßverhältnisse im Sinne habend verallgemeinert Hegel den Begriff des
Exponenten einer Variablen. Exponent im Sinne Hegels kann im Rahmen multipler
Maßverhältnisse ein System nichtlinearer Gleichungen sein, auch nichtlinearer
Operator-Gleichungen. So hat Hegel das Prinzip der Nichtlinearität in der
Philosophie verankert. Allmählichkeit erklärt nichts, aber Nichtlinearität
erklärt das Quale-Umschlagen und dessen Allmählichkeit.

Vereinzelt war Hegel nicht exakt. Richtig sagt er, die Änderung der Größe ist
dem Etwas „nicht gleichgültig“, es bleibt nicht, was es ist, “sondern die
Änderung änderte seine Qualität.“ (S. 343) Nur heißt das nicht, dass ein fixes
Quantum existiere, wo das Etwas „zugrunde ginge“. (343) Zwischen „Änderung“
und „Untergang“, zwischen „Untergehen als Prozess“ und „vollendetem (oder gar
plötzlichem Untergang)“ ist wohl zu unterscheiden. Statt „Untergang“ wäre
korrekt gewesen: „Ein Etwas schickt sich über sich hinaus“. Im Vorwort zur
zweiten Auflage der Logik bat Hegel um Nachsicht. Das war am 7. November 1831.
Sieben Tage später hatte ihn die Cholera dahingerafft.

\section{Lehrbarkeit der Dialektik als pädagogisches Problem}

Grundzüge der Dialektik sind gut, um allererste Aufmerksamkeit zu erregen: Was
ist Dialektik? Gut ist auch zu wissen, dass biologische Arten und
Gesellschaftsformationen sich entwickelt haben. Manchmal wird ein Freund der
Dialektik seinen Zeitgenossen sagen: Leute, wendet die Entwicklungslehre an.
Selbst Lenin hat zuweilen so gesprochen (z.B. in „Staat und Revolution“), und
wenn man die Entwicklungslehre „anwendet“, ist das ein erster Schritt zum
dialektischen Denken.

Doch nachhaltig ist das nicht. Eine Doktrin auf Objekte „anzuwenden“ wird der
Dialektik nicht gerecht. Man muss trainiert sein, Objekte, Zustände, Begriffe
gedanklich zu explorieren. Man muss deren Eigenschaften (begriffliche
Bestimmungen) aus den Keimen entwickeln. Das muss dem Weltbürger in Fleisch
und Blut übergehen, dann wird er Dialektiker. Eine Vorstellung davon hat Hegel
gegeben, als er explorierte, was „Sein“ und „Nichts“ ist. Hegels Logik ist
eine riesige Exploration philosophischer Begriffe. Vergleichbar damit ist
\emph{Das Kapital} von Marx. Beide Werke sind für den Nutzer sehr
anspruchsvoll. Mit einmal Lesen ist es nicht getan. Manchmal ist schon mit
einzelnen Gruppen von Sätzen zu ringen. Allmählich versteht man die „Logik“,
Verzeihung – Dialektik.

Geniale Menschen sind unbewusste Dialektiker. Sie haben so etwas in ihrem
Hinterkopf. Als Dialektiker im Geiste Hegels und als früher Kybernetiker
exemplifiziert das Clausewitz („Vom Kriege“, Zweites Buch, 6. Kapitel)
vermittels Napoleons Handlungsproblematik im italienischen Feldzug 1797. In
Erfindungen hervorragender Ingenieure habe ich Dialektik gesehen. Aber erstens
wird das in der Patentschrift nicht zum Ausdruck gebracht, und zweitens sind
Erfinder zunächst schockiert, wenn man sie bittet: Lassen sie mich mal ihre
Erfindung mit meinen Worten ausdrücken. Danach sind sie angenehm überrascht,
als hätten sie das Christkindlein gesehen.

Dialektik zu erlernen und zu trainieren ist sehr aufwändig. Zwischen den
Extremen „Gar nichts“ und „Hegel/Marx“ sind Sprossen auf Hegels Leiter zu
finden, Zwischenstufen. Die Gruppe der Grundzüge kann ein erstes
Zwischenstadium sein. In vorstehendem Text sind mehrere Vorschläge enthalten.
Mathematik und Kybernetik gehören dazu. Jetzt noch ein weiterer Vorschlag:

So wichtig auch immer Grundzüge der Dialektik für alle Propädeutik sind –
Begriffsentwicklungen müssen auch zum täglichen Leben in Bezug gesetzt werden.
Auch das ist in „Die Allmählichkeit der Revolution“ begonnen worden. Leicht
sieht man, dass z.B. Hegels Figur des „Fürsichseienden“ für „Apartheit“ steht,
wohin der Liberalismus tendiert und wo die Parteien einschließlich Linkspartei
angekommen sind. Das bremst den Aufrechten Gang zu einer humanen Welt, in der
kein Mensch mehr gedemütigt würde. Zu solchen Dialektika habe ich Material aus
praktischer Arbeit als Bürgerrechtler. Und dann gibt es auch noch ganz
trivialen Stoff, nämlich alltägliche Sätze, in denen das Wort „aber“ vorkommt.
Was will man ausdrücken, wenn man das Wörtchen „aber“ verwendet? Ist also auch
im Alltäglichen manchmal ein kleines bisschen Dialektik?

Nun schlage ich vor, die Leibniz-Sozietät möge einen Studienkreis für
Dialektik bilden, zwei Drittel der Mitglieder mit einer eins in Mathematik
oder einem höheren Zertifikat. Vielleicht würden meine Ingenieur-Kollegen
mitwirken. Drei Voll-Mathematiker müssen dabei sein. Wir sollten Brücken
schlagen zu Kennern der Künste und zu Promotern der De-Eskalation. Doch ohne
Hegel geht es nicht: „Erst was bestimmt ist, ist exoterisch und fähig, gelernt
und das Eigentum aller zu sein. Die verständige Form der Wissenschaft ist der
Allen dargebotene und für Alle gleichgemachte Weg zu ihr [\ldots]“.

\section{Anhang: Drei bisher nicht veröffentlichte Supplemente}

\subsection*{I.}

\paragraph{a)}
Der Deutschen Gesellschaft für Kybernetik, vor allem Prof. Siegfried
Piotrowski (Paderborn), gebührt Dank für das Interesse zur Rekonstruktion des
Aufstiegs und der Schwierigkeiten der Kybernetik in der DDR. In einer Reihe
von Kolloquien ab 1999 hatte die Deutsche Gesellschaft für Kybernetik
Zeitzeugen und Aktivisten des schwierigen Aufstiegs zusammengerufen. Nicht
alle haben den Aufstieg betrieben, doch sie haben erneut das Wort ergriffen.
Die Deutsche Gesellschaft für Kybernetik hat recherchiert und Aktivisten nach
vielen Jahren erneut zusammengeführt. In einem Vortrag und in mehreren
Diskussionsbeiträgen hatte ich vornehmlich die tiefgreifenden Korrespondenzen
von Dialektik und Mathematik/Kybernetik behandelt. Früher oder später wird das
auch noch publiziert werden.

\paragraph{b)}
Alsbald nahm die Deutsche Gesellschaft für Kybernetik auch Verbindung zur
Leibniz-Sozietät auf. In einem korporativ veranstalteten zweitägigen
Kolloquium 2002 wurde versucht, den 1912 geborenen und 1974 verstorbenen Georg
Klaus zu würdigen. Auch dort habe ich die Korrespondenz von Kybernetik und
Dialektik zur Sprache gebracht („Georg Klaus, die Dialektik, die Mathematik
und das lösbare Problem disziplinärer Philosophie“). In einer kurzen Notiz
habe ich auch auf die Arbeit der Kybernetik-Kommission des Forschungsrates der
DDR hingewiesen. Daran hätte im neuesten Sammelband angeknüpft werden können.

\paragraph{c)}
Beide Korporationen luden für November 2007 zu einer Veranstaltung ein mit den
zwei sinnverwandten Titeln „Kybernetik – evolutionäre Systemtheorie –
Dialektik“ und „Kybernetik und Dialektik“. Zum ersten Mal nach vielen
Jahrzehnten wurde versucht, dem Begriffspaar Dialektik/Kybernetik ein ganzes
Tagungsprogramm zuzuordnen.

\paragraph{d)}
Schon Jahre zuvor hatten beide Korporationen den Sammelband „Kybernetik steckt
den Osten an – Aufstieg und Schwierigkeiten einer interdisziplinären
Wissenschaft in der DDR“ inauguriert. Für den Anfang 2007 gedruckten
Sammelband hat Frank Dittmann in engagierter, mühevoller Kleinarbeit 20
schriftliche Beiträge acquiriert und druckfertig formatiert. Ich weiß nicht,
ob Frank Dittmann dafür den hochverdienten Lohn empfangen hat. Deshalb bin ich
Frank Dittmann auch nicht böse, dass einer meiner beiden in seiner Hand
befindlichen druckfertigen Beiträge der Weiterleitung an den Verlag entgangen
ist. Niemand ist in der Lage gewesen, Herrn Dittmann für seine anspruchsvolle
Arbeit einen dotierten Forschungsauftrag zu vermitteln. Deshalb wäre es
unbillig, ausgerechnet ihn dafür verantwortlich zu machen, dass mit der
umfangreichen Publikation nicht alle Probleme gelöst worden sind, die
zwangsläufig auftreten, wenn man möchte, eine komplizierte Komponente der
Wissenschaftsgeschichte Revue passieren zu lassen. Frank Dittmann hat es
vermocht, einen engagierten Berichterstatter von Werken des früh verstorbenen
Manfred Peschel in die Edition einzubeziehen, nämlich Herrn Seising.

Freilich wären durch Zusammenarbeit mit Zeugen und Mitgestaltern der
Kybernetik in der DDR einige Defizite vermeidbar gewesen: Wichtige Ereignisse
wären ausnahmslos richtig eingeordnet worden, die Leistungen des
Forschungsrates der DDR ab 1968 wären gewürdigt worden. Kenntlich gemacht
worden wären auch die verderblichen Folgen der Ungleichmäßigkeit des Aufstiegs
der Kybernetik in der DDR, die sich ab 1968 in Karrierismus und Opportunismus
äußerten. Vor allem ab 1968 rief das den Widerstand seriöser, wenn auch
konservativer Wissenschaftler und Ingenieure hervor. In meinem Lebenslauf
werde ich darüber berichten. Einige Andeutungen unten in den beiden folgenden
Supplementen.

\subsection*{II.}

Nach wiederholter Durchsicht von Texten im Sammelband „Kybernetik steckt den
Osten an – Aufstieg und Schwierigkeiten einer interdisziplinären Wissenschaft
in der DDR“ gebe ich hiermit zu Protokoll:

\paragraph{a)}
Zum Widerspruch herausfordernd ist die Überschrift eines Kapitels auf
Seite~13, welche lautet: „Das Verdikt von 1969“. Diese Formel nimmt
unmittelbar bezug auf den Titel des Sammelbandes und ist deshalb von
grundlegender Bedeutung. Bekanntlich können geschichtliche Prozesse nicht
monokausal erklärt werden. Gewiss hat es 1969 eine restriktive, den Aufstieg
bremsende autoritative Verlautbarung gegeben von Kurt Hager, Mitglied des
Politbüros und des Sekretariats des ZK der SED. Hager war dort zuständig für
die Bereiche Hoch- und Fachschulwesen, Volksbildung, Gesundheitswesen und
darüber hinaus für ideologische Fragen. In der Praxis sprach man kurz von
„Bereich Hager“. Multi-kausal gesehen könnte Hagers „Verdikt“ partiell auch
inspiriert gewesen sein durch den ausufernden Karrierismus, vor dem ich ja
selber auch gewarnt hatte. (s.u.)

Hager war aber nicht verantwortlich für die Bereiche Wirtschaft,
wirtschaftsnahe Forschung und Forschungsrat der DDR. Dafür war sein
gleichrangiger Kollege Günter Mittag zuständig, in der Praxis sprach man von
„Bereich Mittag“. Als Hager restringierte, war im Bereich Mittag eine neue,
der Kybernetik förderliche Initiative angelaufen. Deshalb führt die Formel
„Das Verdikt von 1969“ in die Irre. Unzufrieden mit der Entwicklung der
kybernetik-relevanten Forschung in der DDR hatte sich der hochangesehene,
mathematisch beschlagene Psycholog Prof. Friedhart Klix, der wenig später auch
zum Präsidenten der Weltföderation der Psychologen gewählt wurde, an den
Vorsitzenden des Forschungsrates der DDR, Prof. Max Steenbeck (Physiker,
Magneto-Hydro-Dynamik) gewandt.

Der Forschungsrat der DDR war ein demokratisch arbeitendes Organ. Seine
Mitglieder wurden vom Ministerrat zu dessen Beratung berufen. Seine primären
Gliederungen waren sog. Gruppen, z.B. für Mathematik, Physik, Chemie,
Maschinenbau, Medizin, insgesamt schätzungs"|weise knapp 100 Personen.
Außerdem gab es etliche Zentrale Arbeitskreise (ZAK) mit Mitgliedern aus der
Akademie der Wissenschaften, aus Hochschulen und aus der Industrie,
schätzungsweise 300 Personen. Mitte 1968 wurde Klix von Steenbeck eingeladen,
die Beratung fand in Steenbecks Residenz an der Otto-Grotewohl-Straße statt
und dauerte drei Stunden. Klix wurde von Steenbeck gebeten, eine
Kybernetik-Konzeption für den Vorstand des Forschungsrates auszuarbeiten und
geeignete Mitwirkende zu gewinnen. Als Mitarbeiter des Ministeriums für
Wissenschaft und Technik habe ich an der Beratung teilgenommen, im Auftrag des
Ministers hatte ich die Kybernetik-Kommission hinfort als deren Sekretär zu
unterstützen.

Gleich zu Beginn überschritt ich meine dienstliche Kompetenz und fertigte
einen Entwurf für die Konzeption. Der Minister erfuhr davon, mir wurde
hinterbracht, er habe geflucht. Da meldete ich mich beim Minister und wurde
sofort empfangen. Der Minister belehrte mich freundlich und ließ erkennen,
dass er meinen Eifer hoch schätze. Ich erwähne das, um zu dokumentieren, dass
man als verantwortungsbewusster Bürger der DDR sehr wohl Möglichkeiten hatte,
sich bemerkbar zu machen. Ich habe das auch später genutzt.

Ende August 1968 (es muss der 22. des Monats gewesen sein) trat die
Kybernetik-Kommission zum ersten Mal zusammen. Ihr gehörten an: Prof. Karl
Reinisch (Ilmenau), Prof. Helmut Thiele (Berlin), Prof. Hans Drischel
(Leipzig), Prof. Günter Tembrock (Berlin), Prof. Ulrich (Greifswald),
Prof. Friedhart Klix als Vorsitzender.
\vfill
\ccnotice
\end{document}
