\documentclass[12pt,a4paper]{article}
\usepackage{od}
\usepackage[utf8]{inputenc}
\usepackage[russian,main=ngerman]{babel}

\title{Altschuller, TRIZ und ProHEAL}
\author{Rainer Thiel, Storkow}
\date{24.11.2016} 

\begin{document}
\maketitle
\begin{quote}
  Beitrag zur 21. Leibniz-Konferenz „Systematisches Erfinden“,
  24.--25. November 2016 in Lichtenwalde.
\end{quote}

Alle folgenden Thesen werden von mir nummeriert. Weil ich auch Kritisches zu
TRIZ anmerken möchte, lassen Sie mich bitte mit drei Reminiszensen beginnen:

\paragraph{1.}
Im Jahre 1973 war ich von Altschuller hellauf begeistert, da war ich
aufmerksam geworden auf Altschullers Buch\footnote{G.S.Altschuller. Erfinden –
  (k)ein Problem. Eine Anleitung für Neuerer und Erfinder. Verlag Tribüne,
  Berlin 1973. } „Erfinden – (k)ein Problem“, herausgegeben von Dr. Kurt
Willimczik (Berlin-Pankow) im Verlag der Gewerkschaft FDGB. Ich publizierte
eine Streitschrift\footnote{(??) R. Thiel. Über einen Fortschritt in der
  Aufklärung schöpferischer Denkprozesse. Deutsche Zeitschrift für Philosophie
  1976, Nr. 3} mit Seitenhieben auf die Systematische Heuristik von Johannes
Müller, in der es mir an Dialektik fehlte. Doch ich veranstaltete ein großes
Kolloquium, Titel „Methodologie und Schöpfertum“. Ich nutzte meine Position
als kleiner Forschungsstellenleiter im Zentral-Institut für Hochschulbildung
an der Humboldt-Universität zu großem Lob für Altschuller.  Und Altschullers
TRIZ-Buch\footnote{G.S. Altschuller. \foreignlanguage{russian}{Творчество как
    точная наука. — М.: Советское радио, 1979.}} brachte ich 1984 in deutscher
Übersetzung auf den Markt, insgesamt drei Auflagen\footnote{G.S. Altschuller.
  Erfinden. Wege zur Lösung technischer Probleme. Verlag Technik Berlin. Drei
  Auflagen: 1984, 1986, 1998.}. Die Hauptlast der Übersetzung trug meine Frau,
an Sonntagen, nach 60-Stunden Arbeitswoche als Chefin einer Zeitschrift. Mein
heftiger Einsatz wurde vom Chef des größten und von Professoren gesteuerten
Technik-Verlags anerkannt, er sagte: „Genosse Thiel, Sie haben hervorragend
gekämpft, wir machen das Buch“. Weil sich das lange hinzog, ließ ich als
Provisorium ein Altschuller-Lehrmaterial von Tschus und
Dantschenko\footnote{Quelle} aus Dnjepropetrowsk übersetzen und breit streuen.

\paragraph{2.}
Was mich umgetrieben hatte, verschmolz sich mit praktischer Arbeit im
Ingenieurverband KDT. Vor allem gelangte ich ins Netz des Verdienten Erfinders
Dipl.-Ing. Michael Herrlich (Leipzig). Stimmungsmacher war auch der Direktor
des Instituts für Schweißtechnik in Halle, der Physiker Dr. Werner Gilde. Und
Micha Herrlich, der energiestrotzende, eloquente, gesellige, auch zu Witzen
aufgelegte Revolutionär, hatte Mitstreiter gesammelt, um einen Schneeball zur
Entstehung von Erfinderschulen zu wälzen. Einzelne Amtsträger des
Ingenieurverbandes KDT wie Dipl.-Ing. Rudi Höntzsch – Mitglied des Präsidiums
– waren mitgerissen. Andere Amtsträger waren skeptisch oder gleichgültig. Im
Bezirksverband Berlin der KDT fand ich Anerkennung vom ehrenamtlichen
Vorsitzenden Dr. Georg Pohler, dem Generaldirektor des VEB Kabel-Kombinats
Oberspree. Leider wurden wir innerhalb der KDT bald dem zögerlichen Bereich
„Weiterbildung“ zugeordnet. Doch schon 1981 war die erste bezirks-basierte
Erfinderschule der DDR praktiziert, angesiedelt im VEB Berliner
Werkzeugmaschinenfabrik Marzahn unterm Schirm des Direktors für Forschung und
Entwicklung Dr.-Ing. Werner Bahmann.

Trainer und Moderator der meisten Erfinderschulen in Berlin war der Verdiente
Erfinder Dr.-Ing. Hans-Jochen Rindfleisch vom VEB Transformatorenwerk
Oberschöneweide. Die ersten Verdienten Erfinder, mit denen ich zusammenwirkte,
waren Micha Herrlich, Hans-Jochen Rindfleisch sowie Karl Speicher vom
Turbinenhersteller VEB Bergmann-Borsig in Berlin-Pankow. Karl Speicher
versuchte, das Ministerium für Hoch- und Fachschulwesen zu aktivieren.  Doch
das Ministerium schlief bis zur Auflösung 1990, und die Dresdner Professoren
waren erleichtert, als sie bemerkten, dass das Kolloquium „Methodologie und
Schöpfertum“ 1977 nicht vom Minister angewiesen, sondern von mir einberufen
worden war. Bald sammelten sich im Haus des KDT-Bezirksvorstands Berlin
angeregte Ingenieure und erwarteten, dass etwas geschah.  Doch keiner wollte
den Anfang machen. Auch ich sträubte mich, weil ich kein Ingenieur bin. Doch
immer wieder fiel die Wahl auf einen Philosophen.  Das war ich. Abgesetzt
wurde ich nur drei Jahre später, im Hause war ein neuer Parteisekretär berufen
worden. Ersatz wurde nicht geschaffen, ich setzte meine Arbeit fort.

Anfang der achtziger Jahre entwickelte sich auch Freundschaft und Kooperation
mit Verdienten Erfindern aus anderen Bezirken der DDR: Mit Hansjürgen Linde
aus einem VEB in Gotha (Thüringen) und mit Dr. rer. nat. habil. Dietmar Zobel
aus dem VEB Stickstoffwerke Piesteritz bei Wittenberg. Linde aus Gotha hatte
selber schon mit Erfinderschulen begonnen. Mir imponierte, wie er über
Ingenieur-Sein und Erfinden sprechen konnte. Da animierte ich ihn zu einer
außerplanmäßigen Aspirantur. Seine Dissertation 1988 an der TU Dresden
benannte er „Widerspruchsorientierte Innovations-Strategie
WOIS“\footnote{Genaue Verweise siehe weiter unten.}. In der Verteidigung an
der TU Dresden begegnete er souverän allen Anrempelungen durch Professoren.
Drei Jahre später wurde er Professor in Bayern. Und der Chef unserer
gemeinsamen Widersacher in Dresden bekannte 1992 in internationaler
Fachpresse: Linde hat recht: Anspruchsvolle Ingenieure müssen sich
Widersprüchen stellen! 1993 sagte mir die Geschäftsführerin der Nürnberger
Erfindermesse im Telefongespräch: „Und dann, Herr Thiel, haben wir noch etwas
Besonderes. Bei uns spricht ein Professor aus Coburg über
Widerspruchsorientierte Innovationsstrategie“. Da konnte ich erwidern: „Die
kenne ich, sie ist in meiner Wohnung konzipiert worden.“ Ich verriet nur
nicht, wo ich wohnte: in Berlin Ost.

\paragraph{3.}
In den achtziger Jahren wurden von Hans-Jochen Rindfleisch und Rainer Thiel
rund 25 Erfinderschulen initiiert und praktiziert. Hans-Jochen war Moderator,
Trainer und Wortführer, zielstrebig und geistreich. Er begann mit elf Notizen
auf einem Blatt Papier. Seine Eigenwilligkeit sollte bald deutlich werden,
rasch wuchs sein Spickzettel. Wir beide begannen, über Altschullers Buch von
1973 hinaus zu denken. Gleichzeitig suchte ich die Übersetzung und Publikation
von Altschullers TRIZ-Buch von 1979\footnote{Siehe oben.} zu bewerkstelligen.
Ich kroch gleichzeitig auf zwei Pfaden. Die Übersetzung musste dem Wortlaut
Altschullers folgen, doch 1984 im Titel des neuen Altschuller-Buches vermied
ich das Label „TRIZ“, also „Theorie des Lösens von Erfindungsaufgaben“. Ich
meinte: Von einer Theorie des Erfindens kann noch keine Rede sein, ich gab dem
übersetzten Buch den neutralen Titel „Erfinden – Wege zur Lösung technischer
Probleme“. Anno 1998 hatte Professor Möhrle, aus dem Saarland kommend, die
3. Auflage zu Wege gebracht. Kurz zuvor hatte ich mich an die VDI-Zentrale
gewandt. Von dort kam die Antwort: Erfinderschulen sind in der Bundesrepublik
nicht denkbar.  Zuvor aber, im Juli 1990, formell existierte noch die DDR,
besuchte uns in Ostberlin der Chef der Deutschen Aktionsgemeinschaft „Bildung,
Erfindung, Innovation“, Dr. Matthias Heister aus Bonn. Er sagte: „Sie haben
Erfinderschulen gemacht.  Das ist doch Silbernes, das die DDR einbringt in die
Einheit.  Schreiben Sie ihre Erfahrungen auf.“ Das geschah. Bald erschien ein
dickes Buch\footnote{Erfahrungen mit Erfinderschulen. Ein aktueller Bericht
  für das ganze Deutschland, seine Unternehmer, Ingenieure und Erfinder. Mit
  einer zusammenfassenden Übersicht von R. Thiel und H.-J. Rindfleisch.
  Herausgeber: Deutsche Aktionsgemeinschaft Bildung, Erfindung, Innovation
  (DABEI), Bonn und Berlin 1993.}, finanziert durch Erlöse eines
Benefiz-Konzerts des Amateur-Orchesters der Patent-Behörden in München. Anno
93 war das Buch ausgedruckt, da lief ich zur Filiale des Bundesministeriums
für Bildung und Wissenschaft in Berlin.  Dort sagte man: „Machen Sie doch auch
ein Buch für uns, wir bezahlen das.“ Dieses Buch\footnote{Hans-Jochen
  Rindfleisch, Rainer Thiel. Erfinderschulen in der DDR. Eine Initiative zur
  Erschließung und Nutzung von technisch-ökonomischen Kreativitätspotentialen
  in der Industrieforschung.  Rückblick und Ausblick.  Trafo Verlag, Berlin
  1994.}  erschien 1994.

\paragraph{4.}
In Berlin-Ost hatten sich also zwei Entwicklungslinien miteinander
verschränkt: Popularisierung von Altschuller und Entwicklung über Altschuller
hinaus. So entstand vor allem dank Jochen Rindfleisch unsere gemeinsame Arbeit
\emph{ProHEAL}, also \emph{Programm zum Herausarbeiten von Erfindungsaufgaben
  und Lösungsansätzen}. Dieses Programm beruht auf den revolutionären Anstößen
von Altschuller (Baku und Moskau). Dieser Genrich Saulowitsch Altschuller
hatte gewirkt wie ein Kolumbus, der einen unbekannten Kontinent entdeckte.
Altschuller wirkte auch wie ein Alexander von Humboldt, Natur- und
Kunstschätze beschreibend, die noch nicht beleuchtet worden waren. Altschuller
definierte auch Ideale wie zum Beispiel „Ideale Maschine“ und schrieb gewandt
wie ein Krimi-Autor. Davon bin ich bis heute aufgeladen. Also las ich
Altschuller von 1973 und 1984 immer wieder und notierte meine Beobachtungen.
So entstand allmählich eine Liste mit kritischen Notizen. Jeder einzelne Satz
von Altschuller erscheint einleuchtend, doch manche Sätze widersprechen sich,
auch 1984. Das merkt man, wenn man nachfolgende Sätze liest. Manche Sätze
enthalten Versprechungen, diese werden wiederholt, jedoch nicht immer
eingelöst. Auch das bemerkt man, wenn man weiterliest. Altschullers Texte
hätten viel kürzer ausfallen können. Doch das tut meiner Hochachtung vor
Altschuller keinen Abbruch. Sein Projekt war neuartig, revolutionär, in sich
hoch komplex, schwer überschaubar. Da wird er wohl unterschwellig gehofft
haben: „Vielleicht fällt mir beim Schreiben noch etwas Besseres ein.“ So
erkläre ich mir den großen Umfang seiner Äußerungen. Wir in Berlin begannen
viele Jahre später als Altschuller und hatten es viel einfacher, uns viel
kürzer auszudrücken.

Vor allem beim Verdienten Erfinder Dr.-Ing. Rindfleisch ging das sehr schnell.
Er begann gleich mit der Frage: Wie entdeckt man eine treffende
Problemstellung?  Und noch etwas: Er war ja nicht nur Erfinder, er war auch
durch die strenge Schule der theoretischen Elektrotechnik gegangen. Und so
vollzog sich Dialektik: Hans-Jochen entwickelte sehr schnell das ProHEAL, und
ich half mit. Dieses Programm ruht nicht nur auf mancherlei Kritik an
Publikationen Altschullers, vor allem setzt es neue Akzente. Das möchte ich
nunmehr andeuten, bevor ich das ProHEAL als solches skizziere.

\paragraph{4.1.}
ProHEAL wendet sich nicht nur an erfindungswillige Ingenieure. Vielmehr
setzt es an den Anfang die intellektuelle und moralische Auflassung gegenüber
allen Technikern, ihre Tätigkeit an den Bedürfnissen der Gesellschaft zu
orientieren und nicht primär an den Aufträgen des Chefs. Dazu sind alle
Techniker verpflichtet und nicht nur die Liebhaber von Erfindungen. Deshalb
beginnt ProHEAL kurz und bündig mit den Worten: \emph{„Worin besteht das
  gesellschaftliche Bedürfnis?“} Das soziale, ökologische, wirtschaftliche
Bedürfnis? Man wird ja nicht gleich gefeuert, wenn man den Chef daraufhin
anspricht. Deshalb werden in ProHEAL anstelle des Begriffs „Auftrag“ die
Begriffe „Problem“ und „Problemlösung“ verwendet.

\paragraph{4.2.}
ProHEAL vertraut darauf, dass Techniker gut ausgebildet sind und sich in
Physik, in Chemie und Bionik zurechtfinden können. ProHEAL vertraut auch
darauf, dass Techniker sich Einblick verschaffen können in soziale,
ökonomische und ökologische Zusammenhänge: Entweder sie misstrauen gängigen
Medien oder sie verschaffen sich Einblick mittels \emph{kritischer} Medien.

\paragraph{4.3.}
ProHEAL setzt darauf, dass Techniker rationale Anforderungen und Erwartungen
respektieren und dass sie gegebene Bedingungen und Restriktionen nicht
fürchten.  Schon in (Altschuller 1973:89) hatte Altschuller geschrieben: „In
der Praxis der Erfinder besteht die Hauptsache oft darin, den technischen
Widerspruch aufzudecken. Ist er erst aufgedeckt, ist es nicht mehr schwer, ihn
zu überwinden. Oft ist auch die präzise Formulierung des Widerspruchs
entscheidend. Der Erfinder muss genau bestimmen, was \emph{das nicht zu
  Vereinbarende} und was \emph{das zu Vereinbarende} ist. Hier ist es wichtig,
psychische Trägheit zu überwinden.“ Gerade darauf zielt ProHEAL, ohne viele
Worte.

\paragraph{4.4.}
Problemlösungen sind Innovationen. Doch Innovationen, die nur an Marktgesetzen
und Patentrecht orientiert sind, sind nicht generell Problemlösungen.  Was
patentiert wird, muss international eine Neuheit sein. So war es auch in der
DDR. Doch die Patent-Amtler in der DDR wollten auch Problemlösungen sehen.
Deshalb war es in der DDR zunehmend schwerer geworden als in der BRD, ein
Patent zu erlangen. Patent-Einreicher sagten oft: Wenn wir aus Berlin kein
Patent bekommen, melden wir in München an. Altschuller aber hatte seine
Prinzipe zur Problemlösung überwiegend aus Patentrecherchen extrahiert.
ProHEAL dagegen respektiert den Unterschied zwischen Innovation und
Problemlösung von vornherein. Altschullers WEPOL-Darstellung ist bedeutsamer
als die Liste der 40 Prinzipe, weil sie zur Ausnutzung verfügbarer
Wechselwirkungen anregt.  Leider ist 1984 bei Altschuller die Verwendung
dieser beiden Paradigmen – die vierzig Prinzipe einerseits und WEPOL
andererseits – nicht aufeinander abgestimmt. Da geht es streckenweise
durcheinander.  Vorteil: Da müssen die Leser grübeln.

\paragraph{4.5.}
Die meisten der 40 Prinzipe scheinen mir nicht mehr zu sein als Erinnerungen
an Kniffs, die jedem Facharbeiter auch ohne Literatur in den Sinn kommen.
Hoch interessant ist, wie Dietmar Zobel in seinem Buch „Systematisches
Erfinden“\footnote{Dietmar Zobel. Systematisches Erfinden: Methoden und
  Beispiele für den Praktiker. Expert Verlag, Renningen 2009.} den meisten der
40 Prinzipe eine Funktion zuweist: Weniger zum Erfinden geeignet, sondern zur
fachgerechten Ausführung technischer Leistungen. Von großem
erfindungsrelevanten Kaliber scheint mir nur das Prinzip „Keil durch Keil“
(Altschuller 1973:304), besser noch „Kompensation“ (Altschuller 1973:28, 29,
32, 50) und „Umwandlung von Schädlichem in Nützliches“ (Altschuller 1984:139)
zu sein. Symbol ist das Pendel-Problem nach Karl Duncker\footnote{Karl
  Duncker. Zur Psychologie des produktiven Denkens. Springer, Berlin
  1935.}. Doch bei Altschuller fehlt das Symbol.  Ich habe getestet: Von 250
Ingenieuren fand nur ein einziger die Lösung von selbst. Doch von Altschuller
wird der Psychologe Carl Duncker gescholten. Schade.

\paragraph{4.6.}
Altschullers Riesentabelle mit den vierzig Lösungsprinzipen ist gewiss ein
Mutmacher. Doch Lösungen aus der Tabelle ablesen? Ich betone „ablesen“. Wer
nur abliest, hat noch nicht die \emph{Entstehung} von Widersprüchen im Blick:
\emph{Warum} sind Widersprüche entstanden? Danach zu fragen empfiehlt auch
Dietmar Zobel. Wer nur abliest, ist noch nicht im Bilde. Wer nur abliest, ist
auch noch nicht in der Therapie, wo Geist und Seele massiert werden, um
kreativ zu werden.

\paragraph{4.7.}
WEPOL – also Stoff-Feld-Analyse – regt schon eher zum Denken an. Die
WEPOL-Graphik ist ein Medium zum Denken wie unsere Wortsprache. Doch WEPOL
schränkt auch das Blickfeld ein. Es gibt ja Beziehungen auch zwischen Körpern
und Beziehungen zwischen Feldern. Darüber täuscht WEPOL hinweg.

\paragraph{4.8.}
ProHEAL bietet ein matrix-förmiges Format mit 16 Tabellenfeldern, in das
sich Parameter gemäß 4.2 und 4.3 leicht eintragen lassen. ProHEAL fordert dazu
auf, die Werte dieser Parameter (bzw. deren Kehrwerte) in dieser Matrix bis
zur Entstehung von Widersprüchen in die Höhe zu treiben, also Widersprüche
auf Basis 4.1, 4.2 und 4.3 herauszufordern, zu sollizitieren.

\paragraph{4.9.}
So hat es auch Hansjürgen Linde gesehen. 1990 begann er, unterm Label „WOIS“
in renommierten Industrie-Unternehmen ganz Deutschlands seine Workshops zu
praktizieren.

\paragraph{4.10.}
Eine Frage: Ist jetzt nicht auch der Widerspruch entstanden zwischen der
Reife literarisch explorierter Methodologie des Erfindens und dem Umfang der
Literatur, der von den Technikern bewältigt werden müsste? Den Technikern im
Beruf und im Studium?

\paragraph{5.}
Nun werfen wir einen Blick auf das Kernstück von ProHEAL, ein matrix-förmiges
Gebilde mit 16 Feldern. (Siehe Anhang bzw. auch in (Thiel 1994:114). In den
Matrix-Feldern waren 1988 beispielhaft Eintragungen für Kraftfahrzeuge
vorgenommen worden).

\paragraph{5.1.}
Dabei kann man einen Spaß und zwei Beinahe-Redundanzen bemerken. Begonnen
hatten wir, dem \emph{gesellschaftlichen Bedürfnis} entsprechend die
\textbf{A}nforderungen, die \textbf{B}edingungen und die
\textbf{R}estriktionen zu notieren, also \emph{die A, die B und die R}. Da
meinte Hans-Jochen: Nehmen wir doch gleich noch die \textbf{E}rwartungen mit
dem Anfangsbuchstaben E hinzu, das duftet zwar nach Redundanz, denn
Anforderungen, Bedingungen und Restriktionen haben wir schon im Blick, doch
wir haben statt ABR jetzt \textbf{ABER}. Das ist nicht nur ein schönes Logo
aus vier Buchstaben, das ist auch ein Alarm-Signal beim Brainstorming, dem ein
inverses Brainstorming folgen muss. Da spielt in deutscher Sprache \emph{die
  Kopula „aber“, also der kritische Einwand}, eine produktive Rolle. Eine
weitere Beinahe-Redundanz erlaubten wir uns, indem wir außer den \emph{ABER}
auch noch \emph{vier Zielgrößen-Komponenten} ins Blickfeld zogen. So entstand
aus der eindimensionalen ABER Liste eine zweidimensionale Matrix mit insgesamt
4 hoch 2 gleich 16 Feldern. Und Re\-dundanz wegen der Begriffsverwandtschaft
der Zeilen- und der Spalten-Eingänge? Das ist ein Glücksfall. Denn jetzt
entsteht für den Techniker ein (4-hoch-2)-Generator, \emph{das Denken
  anzukurbeln}. Das ist Wind, um Widersprüche herauszuarbeiten. Wir brauchten
nur noch zu sagen: „Und jetzt, liebe Kolleginnen und Kollegen, treiben wir die
üblichen Parameterwerte in die Höhe.“ In unseren Erfinderschulen hat das
funktioniert. Die Techniker waren aufs höchste angeregt, auch
\emph{auf}geregt, und bald kamen laute Zwischenrufe. Verdutzte Techniker
riefen: „Da kommen wir doch in Widersprüche!“ Ja, genau das wollen wir, das
hatte auch Altschuller gewollt, ich hatte es schon zitiert. Und ich konnte den
Technikern sagen: „Da sind Sie von ihren Professoren getäuscht worden, von
wegen in einer Ingenieur-Aufgabe dürfe nie ein Widerspruch auftreten.“ Wir
aber haben gelernt von Hegel und von Marx.  Dort lernten wir auch, dass zur
Dialektik gehört: Bäume wachsen nicht bis zum Himmel. Wer darauf hinwirkt,
provoziert dialektische Widersprüche.  Auch diesen Gedanken von Hegel und Marx
fand ich einmal, aber nur einmal bei Altschuller.

Die ABER-Matrix ist (mit redaktionellen Wandlungen) auch von Hansjürgen Linde
in seiner Dissertation TUD 1988\footnote{Hansjürgen Linde. Gesetzmäßigkeiten,
  methodische Mittel und Strategien zur Bestimmung von Erfindungsaufgaben mit
  erfinderischer Zielstellung.  Dissertation, TU Dresden 1988.} und 1993 in
der Druckausgabe der WOIS\footnote{Hansjürgen Linde, Bernd Hill. Erfolgreich
  erfinden. Hoppenstedt, Darmstadt 1993.} vielfach verwendet worden, ebenso
Schreibweisen in Textfassungen, 1980 von mir eingeführt, um Texte Altschullers
komprimieren zu können. Linde und Thiel waren in herzlicher Freundschaft
verbunden. Eines Tages rief er mich an: „Morgen muss ich ins Krankenhaus.“
Drei Wochen später empfing ich eine Nachricht aus seinem Institut: Sein
reiches Leben hatte sich vollendet.

Nun kommen nur noch drei Positionen mit wenigen Worten, also:

\paragraph{5.2.}
ProHEAL vertraut darauf, dass Techniker willens und fähig sind, ihr Fachwissen
gründlich zu nutzen, notfalls auch zu ergänzen, um vor dialektischen
Widersprüchen nicht zurückzuschrecken, also kreative Wege einzuschlagen.

\paragraph{5.3.}
ProHEAL bietet auf wenigen Druckseiten einen algorithmus-analogen Leitfaden,
zunächst leicht lösbare Widersprüche zu identifizieren und schnell zu lösen
oder im Bedarfsfall tiefergehende technisch-technologische und noch tiefer
gehende technisch-naturgesetz\-liche Widersprüche genau und somit angreifbar
zu formulieren. Leicht lösbare Widersprüche fanden wir schnell und sehr oft.

\paragraph{5.4.}
ProHEAL ist auf wenigen Druckseiten nachlesbar, an denen sich Techniker
hinreichend orientieren können. Zusätzlich werden Erläuterungen angeboten, die
sich wahlweise wahrnehmen lassen, sie vergrößern die Lust zum Problemlösen.

\section*{TRIZ, ProHEAL und unsere Zukunft}

Zwei kurze Anmerkungen von Rainer Thiel zu den täglich sichtbaren
Leuchtschriften „Innovation“ und „Wachstum“ in der Abschlusssitzung der
Konferenz

\paragraph{1. Innovationen, unser Zeitalter.}
Da ist ja etwas dran. Aber nur etwas. Ich zitiere aus einem
Buch\footnote{Sahra Wagenknecht. Reichtum ohne Gier. Campus Verlag,
  Frankfurt/M. 2016.}, das gerade in Frankfurt und New York erschienen ist und
sich auf umfangreiche Literatur stützt, u.a. auf eine Fraunhofer-Studie.  In
diesem Buch heißt es – ich zitiere: „[\ldots] dass ein immer größerer Teil der
Patentanmeldungen nicht mehr dadurch motiviert ist, eigene Innovation vor
Imitation zu schützen. [\ldots] Stattdessen dominiere das Ziel, die Anwendung
bestimmter Technologien durch Konkurrenten zu blockieren. [\ldots] Oder es
werden Verfahren patentiert, denen überhaupt keine Innovation zugrunde liegt.
Immer öfter würden Patente nicht [deshalb] angemeldet, um sie zu nutzen,
sondern um die Nutzung einer den eigenen Produkten gefährlichen Innovation zu
\emph{verhindern}.“

Und noch ein Wort zum Worte „Innovation“: Einer der kreativsten Menschen aller
Zeiten, Albert Einstein, ein Humanist, den Kommunisten zugetan, forderte den
amerikanischen Präsidenten auf, die Entwicklung der Atombombe administrativ
einzuleiten, um damit Hitler zuvorzukommen. Doch die cleveren US-Geheimdienste
hatten gar nicht bemerkt, dass Hitler noch vor seiner Atombombe besiegt werden
konnte. Also begann in Los Alamos die Entwicklung von Atombomben. Als der
Krieg schon entschieden war, wurde aus machtpolitischen Gründen auf Hiroshima
und Nagasaki je eine Bombe geworfen.

\paragraph{2. Die Leuchtbuchstaben „Wachstum“.}
Forciert wird wirtschaftliches Wachstum, das die Bewohnbarkeit unserer
kosmischen Heimat, unserer Erde, untergräbt. In nördlichen Industrieländern
wird Menge und Vielfalt von Konsumgütern und Waffen hemmungslos vergrößert.
Schon im 19. Jahrhundert begannen Philosophen und Dichter davor zu warnen:
Rousseau, Jean Paul, Karl Marx. Der Dichter Jean Paul erzählt, wie er sich an
einen Freund wandte: Kannst Du denn nicht sehen, „dass die Menschen toll sind
und schon Kaffee, Tee und Schokolade aus besonderen Tassen, Früchte, Salate
und Heringe aus eigenen Tellern, und Hasen, Früchte und Vögel aus eigenen
Schüsseln verspeisen. – Sie werden aber künftig, sag‘ ich Dir, noch toller
werden und in den Fabriken so viele Fruchtschalen herstellen, als in den
Gärten Obstarten abfallen [\ldots], und wär‘ ich nur Kronprinz oder
Hochmeister, ich müsste Lerchenschüsseln und Lerchenmesser, Schnepfenschüsseln
und Schnepfenmesser haben, ja eine Hirschkeule von einem Sechsender würd‘ ich
auf keinem Teller anschneiden, auf dem ich einen Achtender gehabt
hätte.“\footnote{Jean Paul. Siebenkäs. Insel Verlag, Frankfurt/M. 1987.
  Original 1796.}. Ich füge hinzu: „So leben wir. Die Schränke voll und
voller. Dicht und dichter gedrängt verdecken Sachen die Sicht auf Sachen, die
schon da sind: Verdeckt, vermisst und abermals gekauft. Man tröstet sich, das
Neue sei moderner \ldots.  Bis schließlich nur noch Röcheln ist: Wir können
nicht anders. Fahren wir zum Kaufhaus.“\footnote{Rainer Thiel. Marx und
  Moritz. Unbekannter Marx. Quer zum Ismus. Trafo Verlag, Berlin 1998.} Dort
auch das Marx-Zitat (MEW 25:784): „Selbst eine ganze Gesellschaft, eine
Nation, ja alle gleichzeitigen Gesellschaften zusammengenommen, sind nicht
Eigentümer der Erde. Sie sind nur ihre Besitzer, ihre Nutznießer, und haben
sie als boni patres familias den nachfolgenden Generationen verbessert zu
hinterlassen.“

Ist das nicht unsere Wirtschaft seit Jahrzehnten? Nichts gegen Märkte, wir
brauchen sie. Sie werden durch mittelständische, genossenschaftliche,
gemeinnützige Unternehmen belebt. [Sahra Wagenknecht 2016, Hans Küng 2010.]
Doch das Gerüst unbegrenzter Marktwirtschaft strebt Richtung Himmel, und das
hat längst neue Widersprüche hochgepuscht. Beunruhigt sind Mitbürger
christlichen Glaubens, Naturfreunde, Nichtregierungsorganisationen NGO und
einige Linke. Bei ATTAC gibt es eine Arbeitsgruppe „Transformation statt
Wachstum“. Ich war Mitbegründer. Doch Techniker sind kaum dabei.

Was machen wir nun mit den extensiven Texten zu TRIZ? Altschuller hatte in
einem Land gewirkt, in dem noch vieles fehlte, was uns im Westen längst
Gewohnheit war. Auch in Asien und Afrika fehlt es an vielem. Muss aber in
Entwicklungsländern alles wie in nördlichen Industrieländern geraten?
Deshalb ist „Transformation statt Wachstum“ eine Kiste mit vielen Problemen,
mit Widersprüchen, vor denen wir alle stehen. Wir müssen sie erkunden. Mit
ProHEAL und seiner ABER-Matrix sind die Probleme ganz direkt ansprechbar.

Wenn wir Freunde von TRIZ sein wollen, müssen wir auch diese Widersprüche
erkunden. TRIZ darf nicht missbraucht werden. Wir wollen keine Sklaven des
großen Kapitals sein. Wir sollten überlegen: Wie muss TRIZ genutzt werden, um
unsere kosmischen Heimat zu sichern? TRIZ im Gepäck, und ProHEAL wird sein.  Es
lebe das Brot, und es lebe der Wein.
\ccnotice
\newpage

\begin{center}
  \providecommand{\abox}[1]{\parbox{2.5cm}{\footnotesize\raggedright #1}}
  \providecommand{\bbox}[1]{\parbox{2.5cm}{\vspace*{4pt}\centering
      #1\vspace*{4pt}}} 
  \begin{tabular}{|c|c|c|c|c|}\hline
    & \bbox{Zweckmäßig\-keit} & \bbox{Wirtschaft\-lichkeit}
    & \bbox{Beherrsch\-barkeit} & \bbox{Brauch\-barkeit}\\\hline
    \textbf{A}nforderungen
    & \abox{Leistungsfähig\-keit und Fahr\-tüchtigkeit bis Fahrgeschwindigkeit 
      von $x$ km/h}
    & \abox{1. Kraftstoff\-sparend\par 2. Abgaswärme nutzend}
    & \abox{1. Leicht bedien\-bar, Verschleißteile leicht zugänglich\par
      2. Ersatzteile an Bord verfügbar (mitführbar)}
    & \abox{\vspace*{1em} 1. Anpassbar an örtlich gegebene
      Verkerkehrs\-bedingungen\par 2. Verwendbar als Zug\-maschine,
      Liefer\-wagen und Reisewagen\vspace*{1em} } \\\hline
    \textbf{B}edingungen
    & \abox{1. Verkehrstauglich\par 2. Zugbetriebstauglich}
    & \abox{1. Servicefreundlich\par 2. Lastentransportdienlich}
    & \abox{\vspace*{1em} 1. Kurzzeitig auf $x$-fache Normallast
      überlastbar\par 2. Fahr\-verhalten (unverzögert), Lenkung
      folgend\vspace*{1em} } 
    & \abox{1. Steinschlag abhaltend\par 2. Hitze abwendend\par
      3. Temperatur\-haltend\par 4. Feuchte\-ausgleichend} \\\hline
    \textbf{E}rwartungen
    & \abox{1. Hohes Be\-schleunigungs\-vermögen\par 2. Verzöge\-rungs\-freie
      Beschleu\-nigung}
    & \abox{1. Transport\-ergiebigkeit\par 2. Preisgünstig}
    & \abox{\vspace*{1em} 1. Schleuder\-bewegungen selbsttätig
      ausgleichend\par 2. Auf rasch veränderliche Fahrbahnbedingungen selbst
      einstellend\par 3. Selbst überwachend\vspace*{1em} }
    & \abox{1. Unabhängig von Tankstellen\par 2. Un\-empfind\-lich gegen tiefe
      Tempe\-ra\-turen (z.B. beim Starten)} \\\hline
    \textbf{R}estriktionen
    & \abox{1. Antriebs- und Brems-System spurgetreu\par
      2. Verkehrs\-regelgemäße Licht- und Signalanlage}
    & \abox{1. Anspruchslos in bezug auf Instandhaltung\par2. Genügsam in bezug
      auf Kraftstoff\-qualität }
    & \abox{1. Verkehrs\-sicher\par 2. Rüttelfest\par 3. Stoß- und
      schlagfest\par 4. dieb\-stahlsicher}
    & \abox{\vspace*{1em} 1. Verträglich mit Abgasnorm\par
      2. Korrosionsbeständig bei Tausalzein\-wirkung\par 3. Unbedenklich für
      inner\-städti\-schen Ver\-kehr\vspace*{1em} }\\\hline
  \end{tabular}\par \vskip1em 
  Zielgrößen und Komponenten. Die ABER-Matrix von Hans-Jochen Rindfleisch und
  Rainer Thiel.  Matrix-Felder sind hier ausgefüllt mit Beispielen aus 1988.
\end{center}

\end{document}
