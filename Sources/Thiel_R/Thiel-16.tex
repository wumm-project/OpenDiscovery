\documentclass[11pt,a4paper]{article}
\usepackage{od}
\usepackage[utf8]{inputenc}

\title{Erfinderschulen – Problemlöse-Workshops.\\[6pt] Projekt und Praxis}
\author{Rainer Thiel, Storkow}
\date{Veröffentlicht am 03.07.2016 in LIFIS-Online.\\[4pt] DOI:
  \url{10.14625/thiel_20160703}} 

\begin{document}
\maketitle
\begin{abstract}\noindent
  In diesem Text werden wesentliche Aspekte der Erfinderschulbewegung in der
  DDR von einem der über viele Jahre aktiv in diese Entwicklungen involvierten
  Augen- und Zeitzeugen beschrieben.  Siehe dazu auch die Publikation
  \emph{Erfinderschulen der DDR -- Silbernes fürs ganze Deutschland} desselben
  Autors in \emph{Horst Jäkel (Hrsg.). DDR – unauslöschbar}.  Schkeuditz 2008.
\end{abstract}

\section*{1. Das Projekt}

Das Projekt entstand in der DDR: Ein bis zwei Dutzend Ingenieure aus einem
Industrie-Betrieb versammeln sich zwei Mal je eine Woche in einem
betriebseigenen Heim, um Methodik des erfinderischen Problemlösens
kennenzulernen und ein bis drei Probleme des Betriebes erfinderisch zu lösen,
in ein bis drei Gruppen Gemeinschaftsarbeit. In der ersten Woche werden ca. 12
Stunden Vorträge geboten. In ca. 40 Stunden Teamwork wird ein Problem exponiert
und ein Lösungsansatz geschaffen. Die Moderation je einer Gruppe – idealerweise
je 7 Teilnehmer – wird von einem erfahrenen Erfinder geleistet, er wirkt als
Methodiker und Trainer. In den nachfolgenden Wochen wird im Betrieb das
Patentstudium vertieft, es werden Berechnungen und Handversuche, auch
Laborversuche angestellt.  Schließlich folgt eine zweite Woche im Internat des
Betriebes, um Patentanmeldungen fertigzustellen und den Start zur Nullserie
einzuleiten.

Den Teilnehmern wurde ab 1983 ein eigens entwickeltes methodisches
Hand-Material -- ein kleines Buch -- zur Verfügung gestellt. Autoren: Michael
Herrlich und andere. 1988/89 war das Material erheblich weiterentwickelt und
stand nun in zwei kleinen Büchern zur Verfügung, beide sehr anspruchsvoll.
Autoren: Hans-Jochen Rindfleisch und Rainer Thiel.  Herausgeber dieser drei
Bücher im Eigenverlag und oft auch Träger von Erfinderschulen war der
Ingenieur-Verband „Kammer der Technik“ KDT.

In der DDR gab es zwischen 1981 und 1990 ca. 300 Erfinderschulen mit ca. 7000
Teilnehmern. Leitbild der Durchführung der Erfinderschulen war das Projekt nach
dem Stand von 1983. Und die erfindungsmethodische Literatur von
G.\,S.~Altshuller (siehe unten) fand zunehmend Beachtung.  Vor allem in Berlin
entstanden von der anzustrebenden Methodik wesentlich weiter führende
Vorstellungen. Sie fanden gedruckten Ausdruck in den beiden Materialien von
1988/89.  Deren Verbreitung litt aber schon unter den Verfalls-Erscheinungen
der DDR. Aus Süddeutschland liegt eine Anfrage vor, dieses Material erneut zum
Druck vorzubereiten.

Nicht immer war es wie vorgesehen zu einer zweiten Erfinderschulwoche gekommen.
Es darf geschätzt werden, dass trotzdem 600 Patentanmeldungen und 1000
praxiswirksame Problem-Lösungen erzielt wurden. Da 1990 das
Entwicklungspotential der Betriebe liquidiert, das Ingenieur-Personal auf 15
Prozent reduziert und zu 85 Prozent in alle Winde zerstreut wurde, kam der
Bildungseffekt der Erfinderschulen vor allem westdeutschen Unternehmen zugute,
kann aber nicht konkreter beurteilt werden. Im Osten fanden Erfinderschulen nur
noch sehr wenige statt.

Die Möglichkeiten, das Projekt in der Bundesrepublik fortzusetzen, wurden
zwiespältig beurteilt. Doch der Vorstandsvorsitzende der Deutschen
Aktionsgemeinschaft Bildung – Erfindung – Innovation (DABEI, Sitz in Bonn)
äußerte noch zu Zeiten der DDR zu Erfinderschul-Trainern: „Sie haben
Erfinderschulen gemacht. Das ist Silbernes, das die DDR einbringt in die
Einheit.  Schreiben Sie Ihre Erfahrungen auf!“ Das geschah mit 16
Einzelbeiträgen von Trainern und mit einer ausführlichen Gesamt-Darstellung
(88 Druckseiten) von Dr.-Ing. Hans-Jochen Rindfleisch und
Dr.\,phil.\,habil. Rainer Thiel (Thiel, Rindfleisch 1993).  Der Druck 1993
wurde finanziert aus Mitteln eines Benefiz-Konzerts, das der Präsident des
Deutschen Patentamts München mit dem Liebhaber-Orchester München der deutschen
Patent-Behörden arrangiert hatte. Als schließlich das fertige Buch auch im
Bundesministerium für Bildung und Wissenschaft präsentiert wurde, empfahl dort
ein Abteilungsleiter: „Machen Sie doch eine solche Edition auch für uns. Wir
können das mit Fördermitteln unterstützen.“ So geschah es.  Zahlreiche
Manuskripte einzelner Trainer benutzend verfassten Rindfleisch und Thiel eine
Gesamtdarstellung, die akademischen Maßstäben gerecht wird: „Erfinderschulen
in der DDR. Eine Initiative zur Erschließung von technisch-ökonomischen
Kreativitätspotentialen in der Industrieforschung“. (127 Druckseiten, trafo
verlag Berlin 1994).

Dazu im Gegensatz befand der deutsche Ingenieurverband VDI (Sitz in
Düsseldorf), Erfinderschulen wie in der DDR könne es in der Bundesrepublik
nicht geben. Das wurde bald widerlegt. (Siehe unten)

Nach 1990 kam es mehrmals zu freundschaftlichen Treffen mit Sprechern von
Erfinder-Ver"|bänden der BRD und Westberlins. Leider konnte ein gemeinsames
Konzept nicht gefunden werden: Projekt und Methodik „Erfinderschule“
entsprachen nicht ihren Vorstellungen. Höchst bedeutsam war aber das Projekt
„Widerspruchsorientierte Innovations-Strategie“ von Hansjürgen Linde: Begründet
in Gotha (Thüringen), mit großem Erfolg fortgesetzt im neuen Standort Coburg in
Bayern. Darüber wird weiter unten informiert.

\section*{2. Antriebe zur Entstehung der Erfinderschulen\\ in der DDR}

Ab 1961 -- nach dem sog. Mauerbau -- entstand in der DDR
wissenschaftlich-technische Aufbruchstimmung.

\textbf{a)} 
Berühmt wurde jetzt Dr.\,rer.\,nat.~Werner Gilde, der Direktor des Instituts
für Schweiß-Technik in Halle. Unter dem Namen „Ideenkonferenz“ machte er das
Brainstorming bekannt, begleitet mit seinem eigenen Ausspruch „Geht nicht
gibt's nicht.“ In Leipzig regte sich der vielfache Erfinder Dipl.-Ing. Michael
Herrlich, Schöpfer von Spezial-Maschinen und -Anlagen im Kombinat Süß- und
Dauerbackwaren, ausgezeichnet mit dem staatlichen Titel „Verdienter Erfinder“.
Für Ideenkonferenzen sammelte er Sprecher (Initiatoren) aus dem ganzen Land,
vor allem Ingenieure und einige Psychologen. Sie nutzten auch Bezirks-Verbände
der KDT. Mit deren Hilfe wurden sie in Industriebetrieben wirksam. Herrlich
schuf auch Grundlagen zu ihrer aller Kommunikation und für ihre Sammlung in
halbjährlich stattfindenden Wochenend-Treffen sowie beim Präsidium der KDT. Zum
ersten Workshop namens „Erfinderschule“ kam es dank Herrlich 1980: Eine Woche
lang mit Teilnehmern aus dem ganzen Land, bestückt mit Vorträgen und
Ideen-Konferenzen.

\textbf{b)} 
Herausragender Einzelkämpfer war der Maschinenbau-Ingenieur Karl Speicher vom
Dampf-Turbinen-Hersteller VEB Bergmann-Borsig in Berlin-Pankow, staatlich
ausgezeichneter Verdienter Erfinder mit 70 größtenteils realisierten Patenten
zur Sicherheits-Steue"|rung von Turbinen, seinerzeit den sowjetischen
Problem-Lösungen überlegen. Karl Speicher (geb. 1925) hatte die Entstehung
seiner Erfindungen protokolliert und versuchte, das Hochschul-Ministerium zu
gewinnen, um in studentischen Workshops Erfinder heranzubilden. Dort sollten
Verdiente Erfinder „vormachen, wie sie es selber gemacht hatten“, und sie
sollten Studenten zu eigenen Entwürfen anregen, die vom gestandenen Erfinder
kritisch gefördert werden wie andernorts Musik-Studenten als Meisterschüler
durch den gestandenen Meister. Gegenüber dem Ministerium konnte sich Karl
Speicher nicht durchsetzen. Doch dank seiner genialen Erfindungen und seines
fortgeschrittenen Alters wirkte er als Nestor der Erfinderschul-Bewegung.

\textbf{c)} 
Im Jahre 1973 beginnend wurde in der DDR bekannt Altschuller, Genrich
Saulowitsch, (Baku und Moskau) mit dessen grundlegendem Werk „Algoritm
isobretenija“, Verlag „Moskauer Arbeiter“ 1969. Die deutsche Übersetzung (310
Seiten) erschien 1973 unter dem Titel „Erfinden (k)ein Problem?“ im Verlag des
Gewerkschafts-Bundes der DDR. Initiator und Übersetzer war der Außenseiter
Dr.\,phil. Kurt Willimczik, Germanist, tätig im Informations-Institut eines
Industriezweiges. Dr.\,phil. Rainer Thiel fand eine Information darüber in
einer Zeitschrift und machte ab 1974 das Buch bekannt in Kreisen von Michael
Herrlich und von Konstruktions-Methodikern, also im Vorfeld der
Erfinderschulen. Thiel war begeistert, dass Altshuller anknüpfte an das Prinzip
des dialektischen Widerspruchs als roten Faden der Definition und der
vorsätzlichen Lösung technischer Probleme. Im Zentrum steht eine Matrix mit 32
Zeilen und 32 Spalten, welche durch je einen technisch-ökonomischen Parameter
definiert werden.  Diese Parameter geraten in Konflikt zueinander, falls ihre
Werte über den Stand der Technik hinaus erhöht werden. In die Matrix-Felder
trug Altschuller jeweils passende Lösungs-Vorschläge ein, genauer gesagt deren
Nummern aus seiner Liste „Die 35 bzw. 40 Prinzipe zur Lösung technischer
Widersprüche“.  Zum Beispiel hieß Prinzip Nr. 22 „Umwandlung des Schädlichen in
Nützliches“, Prinzip Nr. 23 hieß „Überlagerung einer schädlichen Erscheinung
mit einer anderen“. In den nachfolgenden Jahren wurde dieser Ansatz von
Altschuller in mehreren Büchern weiterentwickelt und ergänzt. Bemerkenswert ist
vor allem seine „WePol“ („Stoff-Feld“) Analyse.

\textbf{d)} 
In der DDR fiel Altschullers Werk mitten hinein in die forcierte Verbreitung
ingenieur-methodischer Denkmittel: Die Konstruktions-Methodik von Friedrich
Hansen und anderen, bekannt vor allem im Maschinenbau, und die „Systematische
Heuristik“ von Johannes Müller und Peter Koch. Letztere zielte vor allem auf
die Systematisierung der Ingenieur-Arbeit mit Blick auf die bevorstehende
Computer-Nutzung durch Ingenieure. Das war auch den Freunden des Projekts
willkommen, das bald zur Entwicklung der Erfinderschulen führen sollte. Von der
politischen Führung der DDR wurde die Systematische Heuristik als
General-Instrument zur Erzielung höchster Effektivität der Ingenieur-Arbeit
empfunden und eindringlichst empfohlen bis zur Entstehung von Widerwillen bei
gestandenen Hochschul-Absolventen. Durch die Werke von Altschuller entstand
jedoch eine neue Lage: Kritikwürdig erschien zwar nicht die methodische
Rationalisierung der Ingenieur-Arbeit, wohl aber der Mangel an deren
Orientierung auf widerspruchslösende Erfindungen. Das wurde erstmals
ausgesprochen in einer Denkschrift von Rainer Thiel, in Deutsche Zeitschrift
für Philosophie 3/1976: „Über einen Fortschritt in der Aufklärung
schöpferischer Denkprozesse“. Thiel war damals im zentralen Institut für
Hochschulbildung Forschungsgruppenleiter für wissenschaftstheoretische
Grundlagen der Hochschulbildung. Aus freien Stücken publizierte er 1977 einen
Forschungsbericht und arrangierte ein Kolloquium „Methodologie und
Schöpfertum“. Dort kam es auch zum Zusammenprall mit Führungskräften der
Systematischen Heuristik.  Letztere bemerkten zwei Jahre später, dass Thiel
nicht auf Befehl des Ministers für Hochschulbildung gehandelt hatte, sondern
aus freien Stücken. Neunzehn von neunzig Teilnehmern reichten ihre Beiträge
nachträglich in Schrift-Form ein. Das Protokoll wurde veröffentlicht (170
Seiten).

\section*{3. Die ersten Erfinderschulen}

Etwa 1980 hatte Michael Herrlich dem Präsidium der KDT hinreichend Mut
eingeflößt, zur ersten (symbolischen) Erfinderschule zu rufen. Wenige Monate
später gelang im Bezirksverband Berlin der KDT die erste Erfinderschule,
unterstützt vom Direktor für Forschung und Entwicklung (F/E) der Berliner
Werkzeugmaschinenfabrik Marzahn, Hersteller von Innenrund-Schleifmaschinen für
Wälzlager-Ringe in großen Stückzahlen für den Export in die Sowjetunion: Die
Rotationsgeschwindigkeit der Schleifkörper war rasant gesteigert worden, die
Schleifkörper wurden rasant verschlissen, die Maschinen mussten zum
Schleifkörperwechsel immer rascher angehalten werden. Die Entwicklung hatte zu
einem eklatanten Widerspruch geführt. Die Erfinderschulwoche wurde geleitet von
Dr.-Ing. Hans-Jochen Rindfleisch, Verdienter Erfinder in einem Betrieb der
Elektro-Industrie.  Rindfleisch wurde zur Ausübung der Funktion als Methodiker
und Trainer eine Woche lang freigestellt durch seinen Chef in dem ganz anderen
Industrie-Zweig, so, wie das bald immer wieder geschehen sollte und auch
anderen Trainern in der DDR widerfuhr: Betrieb A stellt seinen Erfinder dem
Betrieb B in einem anderen Industrie-Zweig für eine oder zwei Wochen zur
Verfügung.

Im Bezirksverband Berlin der KDT war der Nicht-Ingenieur Rainer Thiel von
Ingenieuren und Psychologen gegen seinen Willen zum Vorsitzenden der
Arbeitsgruppe „Erfindertätigkeit und Schöpfertum“ gewählt worden.  Funktionen
zu übernehmen, mit denen hohe Erwartungen verbunden sind, war in der Regel
nicht begehrt. Nun aber besuchten Mitglieder der Gruppe auch Ingenieur-Gruppen
in verschiedensten Betrieben. Dabei werden Rindfleisch und Thiel miteinander
bekannt: Der Moderator und Trainer für die erste Berliner Erfinderschule ist
gefunden! Rindfleisch leitet die Problemlöse-Gruppe für das
Schleifkörperproblem. Er ist weniger Pädagoge denn Analytiker und konstruktiver
Problem-Löser. Ein Teil der Teilnehmer bleibt passiv, drei Teilnehmer sind von
seinen Lösungsschritten begeistert.  Die Lösung heißt „Fliegender
Schleifkörperwechsel“ und wird auch vom Direktor für Forschung und Entwicklung
begrüßt. Leider verlor sich der Weg zum devisenbringenden Endergebnis im
Dschungel der Führung des Industrie-Zweigs.

Doch das intellektuelle Ergebnis ermutigte Rindfleisch und Thiel, in Berlin
Erfinderschulen zu arrangieren. Thiel antichambrierte in Betriebsleitungen
verschiedener Industrie-Zweige, zunehmend Interesse von Direktoren
erfahrend. Und Rindfleisch als Trainer entwickelte – von Thiel assistiert --
die Methode des Herausarbeitens und Lösens von Erfindungsaufgaben. Die Achtung
auch von Direktoren für die Erfinderschulen wuchs. Schon nach der 3.~Berliner
Erfinderschule sagte ein Direktor vor allen Teilnehmern, an Rindfleisch
gewandt: „Dass wir unser Problem so angehen müssen, wie Sie es uns gezeigt
haben, hätten wir uns vor drei Wochen noch nicht vorstellen können. Dass Sie es
uns zeigen konnten, verdanken Sie Ihrer Methode.“

In Berlin verliefen an die 30 Erfinderschulen erfolgreich, die von Rindfleisch
moderiert wurden. Ähnlich blieb ihnen das ökonomisch entscheidende Endergebnis
versagt, zum Beispiel so: Nach einer Erfinderschulwoche war in einem
holzverarbeitenden Betrieb das erfinderisch konzipierte Modell einer
Klapp-Couch von einem Abteilungsleiter eigenhändig gebaut worden. Das Modell
wurde in einer Betriebs-Ausstellung vorgeführt. Da wird begeistert gerufen:
„Das müssen wir gleich unseren westdeutschen Einkäufern vorführen. Das bringt
uns Millionen von Devisen.“ Da ertönt der Gegenruf: „Vorsicht, Kunde droht mit
Einkauf.“

Wieso denn das? Die neue Produktlösung hätte die Bereitstellung von
Presswerkzeugen im Wert von 50\,000 Mark erforderlich gemacht, also Peanuts.
Unser Ehrgeiz war, durchgreifende Lösungen zu entwickeln, die mit Peanuts
realisierbar sind. Das gelang auch, denn Rindfleischs Methodik ist gerade dafür
konzipiert: Lösung mit Aufwand nahe null. Doch in der DDR wurden Werkzeugmacher
immer häufiger in der laufenden Produktion verschlissen. Deshalb fehlten sie im
Werkzeugbau. Und eine Peanut aus dem Westen zu importieren war auch nicht
einfach. Ähnliches Ergebnis auch in anderer Weise: Für die Entschwefelung von
Rauchgas in einem Heizkraftwerk wurde eine Lösung gefunden und patentiert:
Einfach realisierbar und obendrein geeignet zur Herstellung eines Baustoffs.
Der Direktor für Forschung und Entwicklung hält das Ergebnis für
aussichtsreich, bereit auch, die Endlösung in seinem Kombinat zu erarbeiten.
Doch er fügt hinzu: „Leider mussten wir die Zuständigkeit für die ganze
Arbeitsrichtung abgeben ans Institut für Kraftwerke in Vetschau.“ Aber dort
wird man sagen: „Nicht hier erfunden“ oder „not invented here“, NHE oder NIH,
wir haben keine freien Kapazitäten, und Berlin ist weit weg.

Bekannt geworden ist, dass in den meisten Erfinderschulen von 8 bis 22 Uhr
gearbeitet wurde, auch während der reichlich bemessenen Pausen. Oft wurden im
Abendprogramm Vorträge von kreativen Mitbürgern verschiedenster Observanz
geboten, oft auch von Sportlern, die über ihren Trainingseifer, über ihre
ausgeklügelten Trainingsprogramme und Wettkampf-Methoden berichteten.

Beeindruckt waren die meisten Erfinderschul-Teilnehmer auch von der Lockerheit
und Aufgeschlossenheit, mit der ernsthafteste Arbeit geleistet wurde. Sie
berichteten vor ihren Chefs: So müsste das auch im laufenden Betrieb geschehen.
Mangelhaft blieb aber noch lange die Vorbereitung der Teilnehmer: Sie sollten
Informationen über Probleme mitbringen, die in ihren Betrieben zu lösen waren,
über Anforderungen von Kunden, über Zulieferer-Engpässe und über
Patentrecherchen.  In den Berliner Erfinderschulen, die prinzipiell der Lösung
betrieblicher Probleme galten, wurden stets mehrere Stunden aufgewendet, um
durch Befragung von Teilnehmern Informationslücken zu reduzieren. Meist blieben
Lücken, die spekulativ zu überbrücken waren.

Von Rindfleisch und Thiel genau wie von allen Erfinderschul-Aktivisten wurde
die einschlägige Arbeit (abgesehen von der Freistellung für jeweils eine Woche
Erfinderschule) ehrenamtlich ausgeübt. Für je eine Stunde Vortrag wurden vom
veranstaltenden Betrieb allenfalls 25 Mark Honorar gezahlt. Selbstverständlich
war es unter diesen Umständen nicht möglich, das Schicksal der gefundenen,
meist auch patentierten Lösungen bis zum möglichen Endprodukt zu verfolgen. Die
nachträglich angefertigten Berichte verschiedener Erfinderschul-Aktivisten sind
leider meist sehr allgemein gehalten. Zu erwarten sind heute nur noch
gründlichere Ausarbeitungen von Michael Herrlich: Kein anderer Aktivist hat so
viele Erfinderschulen gemanagt und moderiert wie dieser hoch-engagierte
Erfinder.  Leider hatte auch Herrlich nach der Wende 1989/90 aufs Härteste
um seinen Lebensunterhalt zu ringen, ohne auch nur einen einzigen Tag der
Freiheit, um seine Erfinderschul-Arbeit nachvollziehbar zu dokumentieren.
Ähnlich erging es Rindfleisch. Thiel wurde bald von ganz anderen Problemen in
Anspruch genommen.

\section*{4. Altschuller, Berliner Erfinderschulmethodik und\\ Systematische
  Heuristik} 

Altschuller erreichte um 1980 das Maximum seiner Produktivität. Er publizierte
ein Buch nach dem anderen und erweckte Neugier, noch mehr aus seiner Werkstatt
zu erfahren. Willimczik brachte in deutscher Übersetzung 1983 im Urania-Verlag
heraus „Flügel für Ikarus. Über die moderne Technik des Erfindens“, gemeinsam
mit A. Seljuzki. Thiel und seine Frau übersetzten „Tvortschestvo kak totschnaja
nauka“ – „Schöpfertum als exakte Wissenschaft“ – deutscher Titel „Erfinden –
Wege zur Lösung technischer Probleme“. 1986 erschien bereits die zweite
Auflage.  Die erste Auflage 1984 musste hart erstritten werden: Der führende
Technik-Verlag stand unter dem Einfluss von Hochschulprofessoren, auch der
Systematischen Heuristik, und wollte nicht. Thiel erstritt sich die Fürsprache
des Kultur-Ministers für eine Anhörung vor 10 Chefs von 10 Verlagen. Dort
stritt Thiel drei Stunden lang. Endlich meinte der Chef des führenden „Verlag
Technik“: „Genosse Thiel, Sie haben engagiert gekämpft. Wir machen das Buch!“
Im Jahre 1998 wurde die 3.~Auflage von Prof Möhrle (Kaiserslautern und TU
Cottbus, Lehrstuhl für Planung und Innovationsmanagement) herausgegeben.

Altschuller hatte oft aus Patentschriften zitiert. Das waren total
redundanzfreie Texte, deshalb nur schwer übersetzbar. Da mussten Experten
helfen, doch was die Experten vorschlugen, war nicht mit dem russischen
Original-Text vereinbar. Also erneuter Start zur Übersetzung.

Der Inhalt dieses Buches war eine Weiterentwicklung von „Erfinden (k)ein
Problem“, völlig neu war die WePol-Analyse, ein Verfahren, die
Funktions-Komponenten technischer Gebilde zu Zwecken der Analyse und
Problemlösung unter Nutzung graphischer Mittel darzustellen, zu erfinderischen
Zwecken leicht zu handeln. Solches Handling erfolgte ständig auch im Kopfe
Rindfleischs und war ein Kompass seiner verbalen Vorschläge.

Mit Altschuller war von Thiel inzwischen auch Dr.\,rer.\,nat.\,habil. Dietmar
Zobel bekannt gemacht worden: der eigenständige, literarisch hochkultivierte
und anstiftende Verdiente Erfinder, Chef der Phosphor-Fabrik im VEB
Stickstoffwerk Piesteritz bei Wittenberg. Genuss zu lesen war sein Buch mit dem
allzu bescheidenen Titel „Erfinderfibel – Systematisches Erfinden für
Praktiker“, Verlag der Wissenschaften 1985, Lust aufs Erfinden erzeugend. Wir
wurden Freunde. Zobel erkannte sehr schnell die Bedeutung Altschullers und
zelebrierte auch in Berlin eine Erfinderschule mit pädagogischem Geschick.
Leider fiel es ihm als habilitiertem Chemiker nicht leicht, die von Hans-Jochen
Rindfleisch – dem primär theoretischen Elektrotechniker -- geprägte Berliner
Methodik zu adaptieren.

In den achtziger Jahren entwickelte Hans-Jochen Rindfleisch seine methodischen
Vorstellungen. Altshullers Widerspruchsgedanke wurde erst jetzt in aller
Konsequenz expliziert. Nach Altshuller wird der Widerspruch in der Technik vom
erfinderisch aufgelegten Ingenieur einfach nur angetroffen wie ein statisches
Verhältnis, anhand der erwähnten Matrix technisch-inhaltlich klassifiziert und
vermittels der erwähnten Matrix-Felder mit höffigen Lösungsprinzipien
ausstaffiert.

Rindfleisch dagegen ist der konsequentere Dialektiker, in dreierlei Bezug:

\textbf{a)} 
In Betracht gezogen wird von Rindfleisch zunächst der längerfristige
technisch-ökonomische Entwicklungsprozess: die gesellschaftlichen Bedürfnisse,
die ihn vorangetrieben oder ihn entbehrt haben, und die technischen Potenzen,
die ihm zufolge geschaffen wurden. Auf diese Weise entsteht das Analogon einer
Landkarte, auf der sich der erfinden wollende, zum Erfinden gezwungene
Ingenieur bewegt, bis er bei der Situation anlangt, in der sich aktuell sein
Betrieb zurechtfinden muss: Welchen gesellschaftlichen Bedürfnissen muss er
sich aktuell stellen? Und welchem Stand der internationalen Technik muss er
sich stellen? Dazu muss der Erfinderschul-Teilnehmer Informationen aus seinem
Betrieb mitbringen und auch zu Patentrecherchen bereit sein. Also Vorsicht vor
plötzlichen Einfällen, das Brainstorming wird lediglich zur Belustigung
gewohnheitsgeprägter Ingenieur-Bürokraten genutzt! In der Erfinderschule nach
Rindfleisch wird – am Vormittag nach dem Brainstorming – das \emph{inverse
  Brainstorming} praktiziert, der Entwicklungs-Dialektik gemäß: Auszusprechen
ist, was den schnellen plötzlichen Ideen entgegensteht. Im Gegensatz zum
Brainstorming als einer reinen Lockerungsübung gingen wir davon aus, dass der
beste Weg zur erfinderischen Problemlösung die gründliche Analyse des Problems
ist. Dazu publizierte Thiel 1989 eine Sammlung von Zitaten berühmter Forscher
wie zum Beispiel Heisenberg: „Die richtige Fragestellung ist oft mehr als der
halbe Weg zum Erfolg.“

\textbf{b)} 
In dieser Phase wird eine Matrix zum Ordnen der anzustrebenden Gedanken
genutzt: Was sind die \textbf{A}nforderungen, die \textbf{B}edingungen für
Herstellung und Gebrauch, die \textbf{E}rwar"|tungen, die über aktuelle
Anforderungen hinausgehen, und die \textbf{R}estriktionen, die total über die
Bedingungen hinausgehen (z.B. die Verkehrssicherheit)? Die sog. \textbf{ABER}.
Diese definieren die Zeileneingänge einer Matrix. In dem Bestreben, das
vorläufig noch abstrakte „Sollen“ noch stärker mit der Ingenieurerfahrung
assoziierbar zu machen, schlug Rindfleisch ein weiteres Quadrupel von
Assoziations-Anregern vor, sogenannte \textbf{Zielgrößen}: Diese werden nun als
die Spalteneingänge der im Entstehen befindlichen Matrix definiert: Die
Zweckmäßigkeit, die Wirtschaftlichkeit, die Beherrschbarkeit und die
Steuerbarkeit des Produktes oder Verfahrens, dessen Konzept gefunden werden
soll.

Beide Quadrupel konstituieren eine Matrix, die mit ihren definierten Zeilen-
und Spalteneingängen dem Betrachter zuruft: Sei unzufrieden mit Deinen noch
diffusen Vorstellung des Problems, mit dem Du konfrontiert bist, und fülle die
Felder der folgenden Matrix aus mit konkreten Angaben relevanter Parameter und
wünschenswerter Entwicklung ihrer Werte:

\begin{center}\small
  \begin{tabular}{|c|c|c|c|c|}\hline
    & Zweckmäßigkeit & Wirtschaftlichkeit & Beherrschbarkeit &  Brauchbarkeit
    \\\hline  
    \textbf{A}nforderungen &&&& \\\hline
    \textbf{B}edingungen &&&& \\\hline
    \textbf{E}rwartungen &&&& \\\hline
    \textbf{R}estriktionen &&&& \\\hline
  \end{tabular}
\end{center}
Es erscheint uns also entscheidend, die wesentlichen Parameter durch Analyse
von tech"|nisch-ökonomischen Belangen zu finden, sodann durch kräftige,
extensive Parametervariation über das Vorgefundene (und damit über Altshuller)
hinausgehend – überhaupt erst Widersprüche gedanklich vorwegzunehmen (zu
antizipieren) und analysierbar zu machen. Deshalb also die $(4\times4 =
16)$-Felder-Matrix mit den ABER und den Zielgrößenkomponenten. So findet der
Ingenieur selbstständig zur Analyse von vorwegzunehmenden Widersprüchen im
technisch-ökonomischen Denkfeld. Der Ingenieur gewinnt an Zielklarheit und an
Lust, kreativ zu werden. So gewannen wir einen Assoziations-Generator für den
Ingenieur, um dessen Erfahrungen zu aktivieren für gründliche Recherche der
Bedürfnisse von Nutzer und Hersteller, Kunde und Fabrikant, und um den
Ingenieur zu motivieren, weiteres Material aus Literatur und
Nachbar-Abteilungen seines Betriebes zu beschaffen. Dass die Zeilen- und die
Spalten-Inhalte sich redundant überdecken können, stört nicht. Es geht darum,
die Assoziation möglichst stark anzuregen und Widersprüche sichtbar zu machen,
sogar zu provozieren.

Ende der achtziger Jahre haben wir – zumindest in Berlin -- diese Matrix in
Erfindeworkshops konsequent angewandt. Das erste Ergebnis war verblüffend. Beim
Ausfüllen der Felder, mit denen fixiert wird, was alles \emph{gleichzeitig}
erreicht werden soll, bricht immer ein Ingenieur aus in den Ruf: „Da kommen wir
ja in Widersprüche.“ Unsere Antwort: „Gerade das sollen Sie ja! Jetzt sollen
Sie Ihrer Phantasie keine Zügel anlegen, jetzt – mit dieser Matrix vor Augen –
sollen Sie kühn und frech sein!“ Und manchmal fügten wir hinzu: „Mit Ihrem
Ausruf {\glq}da kommen wir ja in Widersprüche{\grq} zeigen Sie, dass etwas
gefehlt hat in Ihrer Ausbildung. Sie sind von ihren Professoren in die Irre
geführt worden.“ Die Denkarbeit, die zu dieser matrix-förmigen Tabelle führte,
hatte – über Altshuller hinausgehend -- 1980 begonnen mit dem Vorschlag zu
einer Notierungsweise technisch-ökonomischer Widersprüche, die auch zitiert
wurde in dem ersten Erfinderschule-Lehrmaterial, das von Michael Herrlich
verfasst worden war. Die weitere Ausgestaltung hat fünf Jahre in Anspruch
genommen.  Damit waren Rindfleisch und Thiel zum zweiten Mal zur Dialektik
aller Entwicklung vorgedrungen und zum ersten Mal über das Widerspruchskonzept
von Altshuller hinausgegangen.

\textbf{c)} 
Freude hatte ursprünglich ausgelöst, dass Altshuller in seine Tabellenfelder –
von ihm als fest vorgegeben -- sogleich auch die von ihm als relevant
angenommenen Lösungsverfahren eingetragen hatte. (Den Nachweis ihrer
Multivalenz empfanden wir als schwach.) Gewiss kann die Kurzerhand-Zuordnung
von Standard-Lösungsverfahren für manchen Nutzer Anregung bieten. Wir meinen
aber auch heute noch, dass manchem Nutzer Zweifel kommen, ob Lösungen immer auf
diese Weise gefunden werden können. Das war uns auch Grund, die
Lösungsverfahren aus der Altshuller-Tabelle herauszulösen, um sie zu einem
späteren Zeitpunkt des schöpferischen (kreativen) Prozesses effektiver ins
Spiel bringen zu können. Es nützt nichts, sie zu früh anwenden zu wollen, wenn
das Problem noch gar nicht hinreichend bestimmt ist, ebenso wenig wie beim
Brainstorming.
\enlargethispage{-2em}

Wir suchten und fanden aber einen dritten Anlass, die Dialektik aller
Entwicklung in der Methodik des Erfindens geltend zu machen: Die Bewertung der
von Altshuller vorgeschlagenen vierzig Lösungsprinzipe. Wir verwerfen sie
nicht.  Doch in ihrer Relevanz für die Ausprägung einer erfinderischen
Denkweise und Effektivität unterscheiden sie sich in grundlegende und allzu
spezielle.  Grundlegend sind Prinzip Nr. 22 „Umwandlung des Schädlichen in
Nützliches“ und Prinzip Nr. 23 „Überlagerung einer schädlichen Erscheinung mit
einer anderen“. Das kann auch geschehen durch Spaltung des Einheitlichen in
entgegengesetzte Komponenten, die sich – zum Beispiel bei thermisch bewirkten
Längenänderungen -- gegenseitig kompensieren. Das hatte Duncker erkannt, als er
das Beispiel „Uhrenpendel“ als den entscheidenden „Witz“ erfinderischer
Lösungen rühmte und in seinem Testprogramm für erfinderische, gleichwohl auch
erlernbare Fähigkeiten explizierte. Einige unserer Erfinderschul-Begründer und
Verdiente Erfinder, Autoren erfolgreich angewandter Patente hatten diese
dialektischen Prinzipe intuitiv angewandt, ohne von Altshuller oder Duncker
gewusst zu haben.

Fünfzig Jahre nach Duncker wurde von Thiel bei der Weiterbildung von
Patentingenieuren und Funktionären der Neuererbewegung mit eben der
Dunckerschen Pendel-Aufgabe getestet. Dabei wurde festgestellt, dass fast alle
Ingenieure kolossale, komplizierte, kostspielige, sogar wirkungslose Anlagen
vorschlugen, statt innerhalb von fünf Minuten konzentrierten Nachdenkens die
geniale dialektische Lösung der Längen-Regulierung des Pendels aus ihrem
eigenen Kopf herauszuholen. Auch von Michael Herrlich wurden solche Lösungen
hoch geschätzt als „raffiniert einfache Lösungen“: Das technische Objekt wird
so konzipiert, dass es die erwünschte Funktion selbsttätig ausführt.

Das Prinzip hatte Thiel schon als Kind erspürt beim Versuch, die Wirkungsweise
des Toiletten-Spülkastens zu verstehen. Doch im Physik-Unterricht der Schulen
werden solche Probleme nicht behandelt.  Ersatzweise wurde von Thiel auch das
Beispiel „Gierfähre“ ins Gespräch gebracht. Ersatzweise wurde von ihm auch eine
Kollektion von Schul-Beispielen konstruiert, indem er die vermutliche
Entwicklungsgeschichte des Schiffsankers spekulativ rekonstruierte. (Siehe
„Erfindungsmethodische Grundlagen“, Material für Lehrkräfte, KDT 1988,
Abschnitt 1.9).

Später wurde von Prof. Klaus Stanke (Dresden) der Fall folgender Problemlösung
bekannt gemacht: Es stehen 6 Streichhölzer zur Verfügung, um vier Dreiecke zu
bilden. Unmöglich, sagt der Ingenieur. Da wird von Stanke empfohlen: Nimm 3
Streichhölzer, um in der Ebene ein Dreieck zu bilden. Dann hast Du noch drei
Streichhölzer und gehe in die nächsthöhere Dimension: Von jedem Eckpunkt in der
Ebene strecke ein Streichholz in die dritte Dimension hoch und führe diese drei
Streichhölzer in der nächsthöheren Dimension in einem Punkt mit den zwei
anderen Streichhölzern zu einer gemeinsamen Spitze zusammen: Übergang in die
nächsthöhere Dimension. (Das muss nicht immer eine räumliche Dimension sein.)
Das Stichwort „nächsthöhere Dimension“ findet sich auch bei Altshuller als
Lösungsprinzip 17, doch die dialektische Substanz wird dort nicht erkennbar,
der Gedanke wird dort für den Ingenieur auch nicht nachvollziehbar.

Natürlich ist „Übergang in die nächsthöhere Dimension“ nicht zwangsläufig
lösungs"|träch"|tig. Doch dieser Begriff eröffnet willkommene Möglichkeiten,
wenn die Prinzipe von Rindfleisch und Thiel als ein Übergang in die
nächsthöhere Denk-Dimension aufgefasst und dabei als inhaltserfüllend
(konkretisierend) verstanden werden. Und auch die Umkehrung ist möglich:
Verstehe das Beharren~(!) im Zustand – zum Beispiel der Ebene oder des primären
Denkfeldes -- als das Schädliche, wähle das konträre Gegenteil und lasse dieses
mit dem Urzustand – zum Beispiel dem Dreieck in der Ebene oder der
unerwünschten, thermisch bedingten Längenveränderung des Pendels –
korrespondieren. Das ist interpretierbar als „Überlagerung einer schädlichen
Erscheinung mit einer anderen schädlichen Erscheinung“ oder auch als Spaltung
des Einheitlichen – des traditionellen Pendel-Stabes oder des Sets der 6
Streichhölzer -- zu korrespondierenden Komponenten: des Dreiecks in der Ebene
und seine Nutzung für Dreiecke im Raum.

Diese hervorhebenswerten Prinzipe anzuwenden bedeutet selber einen Übergang,
nämlich den Übergang in eine höhere Denkebene: in die geradezu philosophische
Denkebene. Für den denkaktiven Ingenieur reicht das oft schon, den konkreteren
Rest für die Lösung der Erfindungsaufgabe zu finden.  Diese wenigen, doch
hervorhebenswerten Prinzipe lassen sich mit Beispielen illustrieren und vom
Ingenieur sehr leicht \emph{verinnerlichen}, sodass sie sein Denken bestimmen.
Sie sind erfindungsgenetische Substanz, die im Kopfe des Ingenieurs ihren Platz
findet. Der Ingenieur wird sich daran gewöhnen, auch viele Resultate der
erfolgreich gewordenen Technik der Vergangenheit als Ausdruck jener
hervorhebenswerten Prinzipe zu erkennen. So kann er sein erfinderisches
Vermögen – rein nebenbei, ohne nennenswerten Zeitaufwand – ständig
trainieren. Er braucht dazu nicht die Liste aller 40 Prinzipe von Altshuller
immer wieder durchzuchecken. Darauf ist auch von Dietmar Zobel hingewiesen
worden.

Während die vorstehenden drei Dialektik-Muster a), b), c) von Rindfleisch und
Thiel für weitere Kreise erkennbar wurden, bewegte sich auch die Systematische
Heuristik in ihrer Wahrnehmung von Altschuller. Dieser Pionier wurde nun nicht
mehr ignoriert, sondern auch empfohlen. Doch es war der Altschuller in jener
Version, die wir unter a), b) und c) als methodisch und dialektisch
unzureichend erkannt hatten. Vor allem die Forderung von Rindfleisch und Thiel,
Parameter und Parameter-Werte bis zur Entstehung von Widersprüchen
hochzutreiben, wurde von Experten der Systematischen Heuristik und der
Konstruktions-Systematik heftig angegriffen. Das fand ein Ende erst 1992. Dazu
der nachfolgende Abschnitt 5.

Zuvor noch ein Blick auf das heuristische Programm, das sich aus den
Erkenntnissen a), b) und c) ergab. Es wurde von Hans-Jochen Rindfleisch
\emph{ProHEAL} genannt: \emph{Programm zum Herausarbeiten von
  Erfindungsaufgaben und Lösungsansätzen}. Damit wird die oben dargebotene
Matrix – jenes erste Orientierungsmuster – konkretisiert.

Dieses Programm wurde von Jochen Rindfleisch (Mitwirkung Rainer Thiel) sogleich
in drei Ausdrucksformen dargeboten:
\begin{itemize}\itemsep0pt
\item einem erzählenden Text in „Erfindungsmethodische Grundlagen“,
  KDT-Lehrmaterial 1988, Kapitel 1, Seite 11 bis 52,
\item einer algorithmus-ähnlich dargebotenen Schrittfolge (in
  KDT-Erfinderschule Lehrbrief 2, KDT 1989, Seiten 4 bis 30, Kurzfassung
  einschließlich begrifflicher Erläuterungen Seiten 32–37. Zusätzlich ab Seite
  53 Erläuterungen zu den Begriffen, die im ProHEAL verwendet werden) und
\item graphischen Darstellungen der Struktur des ProHEAL. Diese sind im Bereich
  der vorgenannten Seitenangaben zu finden.
\end{itemize}

Nachdem Hans-Jochen Rindfleisch ProHEAL in dieser dreifachen Ausfertigung
geschaffen hatte, tat er noch ein Übriges. Er versetzte sich selber in die
Rolle eines Anwenders der dreifachen Ausfertigung von ProHEAL und schilderte,
wie ihm ProHEAL dabei zustatten kam. (Lehrbrief 2 ab Seite 73) Und schließlich
versetzte sich Rindfleisch in die Rolle anderer Personen, die ProHEAL in ihrem
Betrieb anzuwenden gedenken. („Erfindungsmethodische Grundlagen“ ab Seite 78)
Auf diese Weise entstehen in beiden Fällen Beleuchtungen des ProHEAL, in
letzterem Fall mit dem Blick anderer Personen in verschiedenen Bereichen der
Technik und in verschiedenen Situationen, in denen geprüft wird mit Blick auf
die ABER und die Zielgrößen-Komponenten: Sind Widerspruchslösungen
erforderlich? Wie verfahren wir? Rindfleisch schildert ausführlich und
hochkonzentriert vierzehn Beispiele aus verschiedenen Betrieben bzw.
Erfinderschulen, auch von unseren Erfinderschul-Kollegen, bei denen
Erfordernisse zu Widerspruchslösungen ausführlich exponiert und Lösungen
schrittweise erarbeitet wurden. Dabei wurde nicht psychologisch spekuliert oder
auf zufällige Einfälle gewartet, es wurde gründliche Gedankenarbeit geleistet.
Dr.-Ing. Hans-Jochen Rindfleisch war durch die harte Schule der theoretischen
Elektrotechnik gegangen.

Alles, was Rindfleisch aufgezeichnet hat, ist gerade deshalb in einfachen,
prägnanten, kurzen, perfekt geformten Sätzen ausgeführt, ohne überflüssige
Floskeln, doch auch ohne Lücken in der Durchführung. So liest es sich gut. Von
ebendieser Güte sind auch die Graphiken. Da war Hans-Jochen Rindfleisch
achtundfünfzig Jahre alt. Thiel hatte schon zu dessen Manuskripten Korrektur
gelesen und kaum je einen falschen Buchstaben gefunden. Sechsundzwanzig Jahre
später liest Thiel abermals, nun auch so, als hätte er im Auftrag eines Verlags
strengste Korrektur zu lesen. Auch was Thiel im Jahre 2015 liest, enthält
weniger Anlässe zu einer Korrektur als üblicherweise ein Buch nach seiner
ersten Drucklegung. Vor allem erscheinen auch heute die von Rindfleisch
aufgeschriebenen Gedanken als zwingend wie in einem ausgereiften Lehrbuch der
Physik oder der Hochschul-Mathematik.

Ein Meister in der alltäglichen Kommunikation mit Kollegen war Jochen nicht.
Seit 1994 hatte Thiel kaum noch Kontakte zu ihm. Einst fragte Thiel, ob wir
unsere beiden Lehrbriefe erneut publizieren sollten, er sähe keine Anlässe zu
Änderungen. Jochen antwortete, es gäbe schon einiges besser zu machen. 2013
verstarb er, im Alter von achtundsiebzig Jahren. Nach dem Urnengang in
Berlin-Köpenick bat ich seine Witwe, den Nachlass gut zu verwahren. Ich bin
glücklich, der Freund eines klar denkenden und energiegeladenen Genies gewesen
zu sein.

\section*{5. Die Widerspruchsorientierte Innovations-Strategie WOIS von
  Hansjürgen Linde aus Gotha (Thüringen) und ihre Erfolge in den alten
  Bundesländern}

Das kam so: Michael Herrlich versammelte jedes Jahr zu Pfingsten – zum Beispiel
auf der Hohen Sonne im Anblick der Wartburg – die Kollegen der
Erfinderschul-Szene.  Thiel kam zufällig zu sitzen neben einem Hansjürgen Linde
aus Gotha, Verdienter Erfinder und Abteilungsleiter im VEB „Rationalisierung
der bezirksgeleiteten und Lebensmittel-Industrie“.  Schnell war zu bemerken,
dass Linde über seine Ingenieur-Arbeit sprechen konnte wie kein anderer außer
Rindfleisch. Thiel schlug Linde vor, eine Forschungs-Aspirantur zu beantragen:
In drei Jahren je 6 Wochen Freistellung von der Arbeit im Betrieb zwecks Arbeit
an einer Dissertation für den Dr.-Ing. Wegen der Ingenieur-methodischen
Orientierung kam dafür nur die TU Dresden als Hochburg der Systematischen
Heuristik und der Konstruktions-Systematik infrage. Linde kannte sich gut aus
in diesem Sujet und war auch fähig, über die Entstehung seiner eigenen Patente
druckreif zu berichten. Linde hospitierte zunächst in mehreren Erfinderschulen
im Lande.  Die Erfinderschulen von Rindfleisch gefallen ihm am besten. Es kommt
zu Zusammenkünften mit Thiel, vorhandene Texte sowie Entwürfe für die künftige
Dissertation werden besprochen. Linde hat auch sofort die Überschrift parat:
„Widerspruchsorientierte Innovations-Strategie“. Die Worte „Erfindung“ und
„Erfinderschule“ will er vermeiden, zu Recht, denn sie werden oft missbraucht.
In einem deutsch-englischen Wörterbuch heißt es: „To invent – erfinden, lügen“.
Besonders oft heißt es im Volksmund „Ausreden erfinden“.

\enlargethispage{-2em}
Linde kommt schnell voran mit seiner Dissertation. Rindfleisch erkennt die Nähe
zu unserem weit fortgeschrittenen ProHEAL und fühlt sich nicht wohl dabei.
Thiel beschwichtigt und sagt: ProHEAL ist derart substantiell und neuartig, von
uns auch noch nicht hinreichend publiziert, dass wir glücklich sein können: Es
entsteht eine zweite Version von ProHEAL und mit vielen neuen Eigenschaften.
Linde als Maschinenbauer erstrebt auch andere Darstellungsweise. Das muss uns
willkommen sein. Wir können unsere Arbeit an Linde spiegeln, überprüfen,
manches wird manchem auch leichter verständlich sein, und unsere Substanz wird
durch Lindes Arbeit geprüft und auseinandergenommen, neu zusammengesetzt und
voll bestätigt. Thiel wurde als dritter Betreuer und als dritter Gutachter von
Lindes Dissertation von der zuständigen Fakultät akzeptiert. Im Februar 1988
wurde die Dissertation an der TU Dresden verteidigt. Titel der Dissertation:
„Gesetzmäßigkeiten, methodische Mittel und Strategien zur Bestimmung von
Entwicklungsaufgaben mit erfinderischer Zielstellung“. Also brauchte Thiel gar
nicht von ProHEAL zu sprechen.

Lindes Aspirantur war problemlos genehmigt worden, vielleicht hatte auch eine
Rolle gespielt, einen Verdienten Erfinder und Praktiker nunmehr zum engeren
Kreis innerhalb der Fakultät zählen zu können. Erst kurz vor der Verteidigung
schien man bemerkt zu haben, welches Problem diesem engeren Kreis mit Linde ins
Haus geraten war: Dieselbe dialektische Substanz, die man jahrelang heftig
angegriffen hatte. Und nun hatte sich Linde gegen unwürdige Angriffe zu
verteidigen: Es wurde gerügt, dass Linde auf Fragen nicht nur mit „Ja“ oder
„Nein“ antwortete, sondern mit konkreten und präzisen Erläuterungen, es wurde
sogar gerügt, dass Linde seine visuellen Overheadprojektor-Darstellungen nicht
auf die Größe des Raumes mit den unerwartet vielen Hörern eingestellt hatte.
Nach ca. zwei Stunden zogen sich die Gutachter und der Vorsitzende der
Prüfungs-Kommission zur Beratung zurück.  Man kam nicht umhin, die Dissertation
anzuerkennen. Doch mit welcher Note? Wenn ich mich recht erinnere mit „cum
laude“, vergleichbar der Drei in den Schulnoten. Thiel plädierte auf „Summa cum
laude“, vergleichbar der Eins in den Schulnoten. Wenn ich mich recht erinnere,
stand am Ende „magna cum laude“, vergleichbar der Zwei.

Anfang 1990 wechselte Linde von Gotha nach München zu BMW, um drohender
Arbeitslosigkeit zu entgehen. Sehr schnell gelangen ihm Patent-Lösungen und
Workshops mit BMW-Kollegen. 1992 wurde Linde – ein Ossi!  -- als Professor an
die Fachhochschule Coburg berufen. Dort entwickelte er seine Workshop-Arbeit
unter dem Titel „Widerspruchsorientierte Innovations-Strategie“ und gründete
neben dem staatlichen FHS-Institut zusätzlich ein privates Institut. Seine
Dissertation publizierte er 1993 beim Hoppenstedt-Verlag in Darmstadt, nun
unter dem Titel „Erfolgreich erfinden.  Widerspruchsorientierte
Innovationsstrategie für Entwickler und Konstrukteure“, 314 großformatige
Seiten, 154 Abbildungen und mit Ergänzungen von Bernd Hill (damals noch in
Erfurt) aus bionischer Sicht.  Lindes Workshops wurden zunehmend von vielen
namhaften, meist auch großen Industrie-Unternehmen aus der ganzen
Bundesrepublik geordert.  Aller zwei Jahre veranstaltet Linde mehrtägige
Konferenzen mit bis zu zweihundert Teilnehmern aus der Industrie und
ausführlichen Dokumentationen. Im Verlaufe von Jahren gelang ihm der Aufbau
eines Teams mit jungen Mitarbeitern. Sie müssen nun die Arbeit allein
fortsetzen.  Anno 2012 wurde Hansjürgen Linde von einem Krebsleiden
dahingerafft.

Kurz vor dem Ende meines Berichts gestatte ich mir, eine Anekdote zu erzählen:
Im Jahre 1993 hatte ich mit der Erfindermesse Nürnberg zu tun.  Da sagte mir
die Geschäftsführerin am Telefon: „Und dann, Herr Thiel, haben wir noch etwas
Besonderes. Bei uns spricht ein Professor aus Coburg über
{\glq}Widerspruchsorientierte Innovationstrategie WOIS{\grq}. Seien Sie
herzlich eingeladen.“ Natürlich wäre der befremdlich klingende Name in der DDR
eher ein Hemmnis für die Öffentlichkeit gewesen. Nun aber freute ich mich und
konnte antworten: „Ich kenne WOIS, die ist in meiner Wohnung in Berlin beraten
worden.“ Ich verschwieg nur, dass das in Berlin-Ost gewesen ist. Zu jener Zeit
war es noch riskant, bekannt werden zu lassen, woher ein Experte wie Linde
gekommen ist. Linde selbst hatte mich gebeten, Vorsicht walten zu lassen. Das
ist aber bald überflüssig geworden. Es war mir auch vergönnt zu beobachten, wie
Linde inmitten eines Kreises seiner Professoren-Kollegen stand und von seiner
Arbeit erzählte.

Und was ist aus der akademischen Fachwelt zu vernehmen? Ein führender Kopf der
Systematischen Heuristik und der Konstruktionslehre äußerte sich in der
führenden Fachzeitschrift „Konstruktion“ Nr. 44 (1992) Seiten 57–63 unter dem
Titel „Kreatives Problemlösen in der Konstruktion“ zunächst in allgemeinen
Worten zum Thema. Schließlich gelingt ihm ein Set von Leitsätzen, überschrieben
mit den Worten „Transformieren und Übertreiben von Problemen“. Natürlich hätte
sich Linde prägnanter ausgedrückt, doch Lindes Kernaussagen schimmern für den
Kundigen hindurch und werden nun auch in folgenden Regeln angedeutet:
„Polarisieren durch Hervorheben innerer Gegensätze, z.B. echter Diskrepanzen in
einem Sachverhalt, einander bedingende Widersprüche, scheinbar paradoxe
Formulierungen {\ldots} Vorstellen der idealen Lösung des Problems, Einführen
extremer Bedingungen, {\ldots} Zuspitzen des Problems durch extreme
Formulierungen, die z.B. {\ldots} extreme Anforderungen an das Ergebnis
enthalten, {\ldots} Das Gegenteil vom allgemein Bekannten ausdrücken. {\ldots}
Bei stark divergenter Problemformulierung sind unerwartete Lösungen am
wahrscheinlichsten, {\ldots} Stark inspirierend wirken vor allem
widersprüchliche Problemsituationen“. Das wird der prägnanten Dissertation von
Linde oder dem ProHEAL mit ihren konkreten Ausführungen nicht ganz gerecht.
Trotzdem freute sich der bescheidene Hansjürgen Linde und sagte zu mir: „Jetzt
hat Professor H. meine Arbeit anerkannt: Der Ingenieur muss die ABER bis zum
Widerspruch treiben, wenn er Probleme lösen will, mit anderen Worten: wenn er
kreativ sein will.“

\section*{Apercu}

\enlargethispage{-2em}
Die DDR war ein Bücher-Land. Doch wer wollte {TRIZ} drucken? Der größte
Technik-Verlag stand unter dem Einfluß der Hochschulprofessoren, die uns nicht
wohl gesonnen waren. So verfiel ich auf eine List. Ich verfaßte drei Briefe für
drei verschiedene Instanzen: Das erste Schreiben für den Präsidenten des
Patentamtes, dem war ein Briefentwurf für den Präsidenten des
Ingenieurverbandes beigefügt, und diesem ein Schreiben an den Minister für
Kultur, der verantwortlich war für das Verlagswesen. Der Minister bat mehrere
Cheflektoren, mir eine Anhörung zu gewähren, und so rang ich drei Stunden lang
mit fünf Cheflektoren, bis der Chef des geeignetsten Verlages sagte: „Genosse
Thiel, Sie haben hart gekämpft und mich überzeugt.“ 1984 war das {TRIZ}-Buch
gedruckt mit dem Titel „Erfinden – Wege zur Lösung technischer Probleme“. 1986
erschien eine zweite Auflage. Die dritte Auflage erschien 1998 auf Initiative
von Professor Möhrle, der von Saarbrücken nach Cottbus übergesiedelt war.

Auch Rohübersetzungen jüngerer Werke Altschullers konnte ich beschaffen. Doch
für mich bleibt Altschullers Werk „Algoritm izobretenija“, Moskau 1969
bahnbrechend – in deutscher Sprache 1973 herausgebracht von Kurt Willimczik
unter dem Titel „Erfinden (k)ein Problem“, eingangs von mir schon genannt. Auf
Altschullers Schultern stehend, als Promoter von Altschullers
Fundamental-Arbeit und durch eigene Praxis waren wir anspruchsvoller geworden.
Das bezieht sich unter anderem auch auf die 35 (in „TRIZ“ 39) Prinzipe zur
Lösung technischer Widersprüche. Als erster brachte unser Kollege Dr.\,habil.
Dietmar Zobel, Verdienter Erfinder und Leiter einer großen Produktionsanlage in
der chemischen Industrie, zu Papier, dass es eine Auswahl dieser Prinzipe
verdient, stets als anregend beachtet zu werden, andere Prinzipe hingegen
weniger. In ProHEAL verweisen wir nichtdestoweniger auf Altschullers Prinzipe,
auch auf seine Standards und seine WePol-Analyse. Aber wichtiger erscheinen uns
18 Empfehlungen, von denen die wohl bedeutendste weiter unten besprochen werden
soll.  (Eine Konfrontation mit einigen wichtigen Standards und mit der
WePol-Analyse stehen noch aus.)

Doch vor allem mussten Ingenieure aus der Industrie gewonnen werden, mit uns
Erfinde-Workshop zu praktizieren.  1980 gelang es Michael Herrlich, über das
Zirkelwesen hinausgehend einen einwöchigen Workshop zu veranstalten, im Namen
des Präsidiums des Ingenieurverbands „Erfinderschule“ genannt. Geboten wurden
Brainstorming, morphologischer Kasten und Erfahrungsberichte von Erfindern. In
den Betrieben war gelacht worden: „Haha, man will uns zu Erfindern ausbilden.“

Seit 1980 wurden Praxis und Methodik der Erfinde-Workshops entwickelt. Bis 1990
wurden landesweit etwa 300 mehrtägige Problemlöse-Workshops mit schätzungsweise
10\,000 Teilnehmern durchgeführt. Allmählich wurde das methodische Repertoir
erweitert. Die Lösungsprinzipe von Altschuller fanden schnell Anklang.
Schwieriger war es mit den anderen fünf Kernen und ihrer Weiterentwicklung.

Schrittweise wurden Fallbeispiele aus Literatur und eigenen
Erfinder-Erfahrungen durch aktuelle Real-Probleme von Industrie-Betrieben
ergänzt.  In Berlin dominierten bald die aktuellen Real-Probleme. In den
Erfinde-Workshops, die Hans-Jochen Rindfleisch in Berlin als Trainer leitete,
wurden Probleme im Real-Zeit-Regime bearbeitet, die ich aus Industrie-Betrieben
besorgte. Das waren meist flüchtig hingeworfene Brocken von ein paar Dutzend
Worten aus der Praxis, arm an brauchbarer Information.

\section*{Literatur}

\begin{itemize}
\item Genrich Saulowitsch Altschuller: \emph{Erfinden – (kein) Problem? Eine
  Anleitung für Neuerer und Erfinder}. Verlag Tribüne, Berlin 1973. Aus dem
  Russischen ins Deutsche übertragen von Kurt Willimczik. Original:
  \foreignlanguage{russian}{Алгоритм изобретения. — М.: Московский рабочий. —
    1969.}
\item Rainer Thiel: Über einen Fortschritt in der Aufklärung schöpferischer
  Denkprozesse. Deutsche Zeitschrift für Philosophie 1976, Nr. 3.
\item Rainer Thiel: \emph{Methodologie und Schöpfertum}. Forschungsbericht und
  Konferenz-Protokoll 1977. Zwei Manuskript-Drucke aus dem Institut für
  Hochschulbildung Berlin.
\item Rainer Thiel: Dialektische Widersprüche in Entwicklungsaufgaben. Berlin
  1980.  Ormig KDT, integriert in das erste Lehrmaterial für Erfinderschulen
  der KDT 1983.
\item Genrich Saulowitsch Altschuller: \emph{Erfinden. Wege zur Lösung
  technischer Probleme}. Aus dem Russischen übertragen von Katrin und Rainer
  Thiel. VEB Verlag Technik Berlin. Drei Auflagen: 1984, 1986, 1998. Original:
  \foreignlanguage{russian}{Творчество как точная наука. — М.: Советское
    радио, 1979.}
\item Dieter Herrig, Herbert Müller, Rainer Thiel: Technische Probleme –
  methodische Mittel – erfinderische Lösungen. In Maschinenbautechnik, Nr. 6
  und Nr. 7, 1985.
\item Rainer Thiel: Wird unseren Ingenieurstudenten die Dialektik des realen
  technischen Entwicklungsprozesses gelehrt? Denkschrift an Kurt Hager und
  ca.\ 80 prominente Intellektuelle, darunter Helmut Koziolek, Erich Hahn und
  Herbert Hörz. 1986.
\item Hans-Jochen Rindfleisch, Rainer Thiel: Programm zum Herausarbeiten von
  Erfindungsaufgaben. Bau-Akademie der DDR, 1986.
\item Rainer Thiel: Zweite Denkschrift an Kurt Hager und ca. 80 prominente
  Intellektuelle, darunter Helmut Koziolek, Erich Hahn und Herbert Hörz. Darin
  {\glqq}Wie ernst nehmen wir es mit der Dialektik?{\grqq} sowie Info über die
  Erfinderschulen. 1987.
\item Hansjürgen Linde: Gesetzmäßigkeiten, methodische Mittel und Strategien
  zur Bestimmung von Erfindungsaufgaben mit erfinderischer Zielstellung.
  Dissertation, TU Dresden, 1988. Betreuer und Gutachter R. Thiel.
\item Hans-Jochen Rindfleisch, Rainer Thiel: Erfindungsmethodische Grundlagen.
  Lehrmaterial zur Erfinderschule. Lehrbriefe 1 und 2, Kammer der Technik,
  Berlin 1988 und 1989.
\item Rainer Thiel: Komplexitätsbewältigung – Dialektikbewältigung, theoretisch
  und \linebreak praktisch.  Deutsche Zeitschrift für Philosophie 1990,
  Nr.~5. Darin weitere Literatur-Angaben.
\item Hansjürgen Linde, Bernd Hill: Erfolgreich erfinden. Darmstadt 1993.
\item Hansjürgen Linde: Forschungsberichte des WOIS-Instituts in Coburg,
  fortlaufend.
\item Erfahrungen mit Erfinderschulen. Ein aktueller Bericht für das ganze
  Deutschland, seine Unternehmer, Ingenieure und Erfinder. Mit einer
  zusammenfassenden Übersicht von R. Thiel und H.-J. Rindfleisch.  Deutsche
  Aktionsgemeinschaft Bildung, Erfindung, Innovation (DABEI).  Bonn und Berlin
  1993.
\item Hans-Jochen Rindfleisch, Rainer Thiel: Erfinderschulen in der DDR. Eine
  Initiative zur Erschließung und Nutzung von technisch-ökonomischen
  Kreativitätspotentialen in der Industrieforschung. Rückblick und Ausblick.
  Trafo Verlag, Berlin 1994.
\end{itemize}
\ccnotice
\end{document}
