\documentclass[11pt,a4paper]{article}
\usepackage{od}
\usepackage[ngerman]{babel}
\usepackage[utf8]{inputenc}

\newenvironment{frage}{\begin{quote}}{\end{quote}}

\title{Rainer Thiel und die Kybernetik}
\author{Rainer Thiel, Storkow}
\date{Version vom 3. Oktober 2020} 

\begin{document}
\maketitle

Das Interview wurde von \emph{Sebastian Bär} geführt und liegt einem Text zu
Grunde, der in der Sonderbeilage des ND zum 3. Oktober 2020 veröffentlicht
wurde.

\begin{frage}
  Wie kamst Du auf die Kybernetik?
\end{frage}
Ich hatte das Glück, vorbereitet zu sein: Zwei Jahre Studium der Mathematik,
vier Jahre Studium der Philosophie. Diplomarbeit „Newton, Marx und Einstein“.
Publiziert in der Monatsschrift „Aufbau“. Der wichtigste meiner Lehrer war
Genosse Professor Georg Klaus. Schon 1951 hatte Klaus vor uns Studenten das
grundlegende Werk von Norbert Wiener aus den USA in der Hand gehabt und
vorgezeigt: \emph{Cybernetics or Communication und Control in the machine and
  the animal}. Der Name Cybernetics ist abgeleitet vom altgriechischen Wort
für „Steuermann“.

Den entscheidenden Anstoß, mich damit zu befassen, empfing ich in der
Silvesternacht 1958. Mein Gast war der Mathematiker Klaus Matthes, der später
auch Akademie-Mitglied wurde. Für mich war er auch Vorsitzender der
SED-Grundorganisation „Mathematik und Physik“ an der Humboldt-Universität zu
Berlin, ich war Vorstandsmitglied. Matthes hatte aber den Georg Klaus immer
nur als parteitreuen Philosophen gesehen. Und nun sagte er mir: „Ihr müsst
euch mal mit dem Werk „Signal“ von Poletaew aus der Sowjetunion befassen. Das
zündete bei mir. Meine Frau leistete den größten Teil der Übersetzungsarbeit,
ich lief zum VEB Deutscher Verlag der Wissenschaften, Georg Klaus schrieb eine
Einführung, 1962 war das Buch gedruckt, die 2. Auflage erschien 1963. Von
Mitarbeitern Norbert Wieners wurde es respektvoll behandelt.

\begin{frage}
  Was waren denn nun die Auswirkungen?
\end{frage}
Zunächst erwähne ich, dass auch Georg Klaus nun schnell ein Buch schrieb
\emph{Kybernetik in philosophischer Sicht}, erschienen im Dietz Verlag Berlin
1961, dem zentralen Partei-Verlag. Es war zu spüren, dass Walter Ulbricht
hinter den Bemühungen stand, die Kybernetik bekannt und wirksam zu machen. Nur
sehr wenige Genossen verstanden das. Doch es war das theoretische Partei-Organ
„Einheit“, das nun wenigstens ein Symposium mit Experten wollte. Ich sollte
Namen vorschlagen. Antwort: „Hast du denn auch Parteilose auf deiner Liste?
Nein? Was wir als Genossen sagen wollen, können wir auch gemeinsam mit
Parteilosen.“ So kam es. Ich musste protokollieren, daraus wurde 1961 eine
Sonderbeilage der Zeitschrift „Einheit“.
\newpage 
\begin{frage}
  Hast du selber auch etwas Originelles geschrieben?
\end{frage}
Meine Dissertation, publiziert 1962 in Deutsche Zeitschrift für Philosophie.
Erstes Kapitel: Karl Marx und die kybernetischen Grundbegriffe „Negative und
positive Rückkopplungen in „Das Kapital“. Zweites Kapitel „mathematische
Differential-Gleichungen in der Wirtschaftsplanung“. Das war damals noch zu
weit gegangen. Drittes Kapitel „Konsequenzen für die Entwicklung der
philosophischen Dialektik“. Nämlich Konsequenzen für das grundlegende Gesetz
des Umschlagens quantitativer in qualitative Wandlungen. Später entwickelte
ich daraus Folgerungen für den Revolutionsbegriff, stets im Einklang mit Marx
und Engels.

\begin{frage}
  Aber da waren Klaus und Thiel immer noch im Literarischen befangen. Gab es
  denn auch Auswirkungen auf die staatliche Planung und Leitung der
  Volkswirtschaft?
\end{frage}
Wir wollten diese Auswirkungen. Doch trotz Walter Ulbricht waren wir immer
noch viel zu schwach. Zunächst wollten wir die akademische Basis erweitern.
Deshalb setzte Georg Klaus beim Generalsekretär der Akademie durch, eine
Kybernetik-Konzeption der Akademie anstreben zu dürfen und eine
Kybernetik-Kommission der Akademie ins Leben zu rufen. So geschah es. Zu
Mitgliedern der Kommission wurden hochrangige Mathematiker, Techniker und
Biologen berufen. Die Organisation war meine Aufgabe als Sekretär. Es gab
Konferenzen in der Akademie der Wissenschaften, auch eine Konferenz speziell
für Ökonomen. Weil Georg Klaus krank war, musste ich dort vortragen. Ich war
noch nicht mal zum Dr. phil. promoviert, doch vor mir saßen zweihundert
Professoren der Ökonomie. In einem Pausengespräch sagten sie mir: „Dann können
wir ja unsere Daten in einen Rechenautomaten werfen, der macht das dann für
uns“. Genau so sprachen auch Soziologen im zentralen Partei-Institut.

\begin{frage}
  Das war wohl allzu naiv.
\end{frage}
Genau. Damit kommen wir zum grundlegenden Problem der sozialistischen Leitung
und Planung und damit auch der Entwicklung des Sozialismus. Wir mussten ja
nicht nur befürchten, dass vom Westen her unsere Staatsgrenze angegriffen
wird. Die Bundeswehr hielt schon Manöver ab bis an unsere Staatsgrenze, die
wir seit 1961 gerade erst gesichert hatten. Und das zweite Problem: Unsere
Wirtschaftskader waren junge Leute, die gerade erst in der Arbeiter- und
Bauern-Fakultät das Abitur gemacht hatten. Und die wenigen älteren Genossen,
die es noch gab, hatten im Klassenkampf bis 1945 zu ringen, um Arbeiter für
den Sozialismus zu gewinnen. Auch 1960 war das Problem noch längst nicht
gelöst. Und nun die Kybernetik, die ihnen allen als fremdartig erschien, noch
dazu aus den USA über den Ozean gebracht.

\begin{frage}
  Hat denn der energische Walter Ulbricht auch mal auf den Putz gehauen? 
\end{frage}
Ei freilich. Es war auf dem 7. Parteitag der SED 1967. Da sagte Ulbricht: „Die
Anwendung der Erkenntnisse der Kybernetik wird für die Planung und Leitung
gesellschaftlicher Prozesse im Sozialismus beträchtlich an Bedeutung
gewinnen.“ Und er setzt noch eins drauf: „Und wenn [\ldots] die Kybernetik
hilft, die Arbeitsproduktivität beträchtlich zu steigern, dann werden wir uns
[\ldots] so lange in die Kybernetik hineinknieen, bis wir sie vollständig
beherrschen.“ (a.a.O. S. 295). Vom Parteitag einstimmig beschlossen. Doch über
die Folgen möchte ich sprechen, nachdem ich auf die Grundbegriffe der
Kybernetik gekommen sein werde.

\begin{frage}
  Wie war das denn in der Sowjetunion? 
\end{frage}
In der Sowjetunion bestand Interesse wegen der militärischen
Landesverteidigung, doch kein Interesse für zivile Zwecke. Aber bald gab es
ein paradoxes Phänomen. Ein tschechischer Genosse war wegen drohender
Hinrichtung von revolutionären Querdenkern nach den Rayk-Kostoff-Prozess in
die Sowjetunion geflohen. Mit Kybernetik im Kopf fand er den gleichgesinnten
Axel Iwanowitsch Berg, die beiden brachten die Sache in Gang. Georg Klaus hat
das erkundet und für die Junge Welt eine Sondernummer geschrieben.

Bis jetzt haben wir über Grundprobleme der Wirtschaftsführung gesprochen. Doch
was sind denn nun die Begriffe der Kybernetik, die verstanden werden müssen?
Genosse Georg Klaus unterschied zunächst vier Komponenten der Kybernetik:
\begin{itemize}[noitemsep]
\item[1.] Regelungs-Theorie, 
\item[2.] Informations-Theorie,
\item[3.] Algorithmen-Theorie,
\item[4.] Theorie der strategischen Spiele. 
\end{itemize}
Ich beschränkte mich auf den regelungstheoretischen Aspekt, der für die
Entwicklung der philosophischen Dialektik am interessantesten ist. Übrigens
hatte die Kybernetik auch noch eine „Cousine“: Die „Operationsforschung“, die
mit mathematischen Matrizen-Gleichungen arbeitete, um volkswirtschaftliche
Bedarfs- und Produktions-Verflechtungen zu analysieren.

\begin{frage}
  Nun aber los mit der Kybernetik!
\end{frage}
Grundbegriffe sind die Rückkopplungen, negative und positive Rückkopplungen.
Philosophen sprechen nur sehr allgemein von Wechselwirkungen. Auch im
Physik-Unterricht der Schulen wird das verschwiegen. Was negative Rückkopplung
ist, fiel mir aber schon auf, als ich zehn Jahre alt war: An der Wasserspülung
in der Toilette. Hat man am Griff gezogen, schießt Wasser von oben nach unten.
Wenn genug Spülwasser geflossen ist, ist der Wasserfall von selber am Ende.
Doch der Spülkasten über unseren Köpfen füllt sich wieder ganz von selbst mit
Wasser, ohne unser Zutun. Wie denn das?  Ganz einfach. Im Spülkasten hängt
eine Kapsel, die im Wasser schwimmt, der sogenannte Schwimmer, der per Hebel
mit der Wasserleitung verbunden ist und den Wasser-Zufluss stoppen kann. Wenn
nun das Wasser aus dem Spülkasten heraus ist und das Klosett-Becken gereinigt
hat, ist auch der Schwimmer auf den Boden des Spülkastens herabgesunken, und
per Hebel gibt er den Zufluss neuen Wassers frei. Ist das neue Wasser genügend
gestiegen – frei zur neuen Spülung des Klosettbeckens -- schließt der
Schwimmer den weiteren Zufluss von Wasser. So ist der nächste Akt perfekt
vorbereitet. So einfach ist das. Das verstand ich als Zehnjähriger, weil mir
meine Eltern die angeborene Neugier nicht geraubt hatten. Doch in
Schul-Lehrbüchern zur Physik geht es immer nur um Newtonsche Mechanik und
elementarste Elektrik, aber nichts Dialektisches.

\begin{frage}
  Und was ist nun „positive Rückkopplung“? 
\end{frage}
Das ist vielen Menschen bekannt als Teufelskreis: Ein Unglück zieht weitere
Unglücke nach sich. Und wenn nicht eine negative Rückkopplung dazwischen
kommt, wächst das nächste Unglück heran, im Extremfall bis zu großen
Katastrophen. Immerhin haben friedliebende Politiker und Diplomaten ein Gefühl
dafür.

\begin{frage}
  Wie kann man nun die Gefühle für negative und positive Rückkopplung zu
  vollem Bewusstsein verstärken, ein für allemal auch lehrbar machen?
\end{frage}
Das ist im allereinfachsten Fall mit ein paar Strichen auf Papier machbar, am
einfachsten mit je einem Kreis. Das Material für die Rückkopplung ist für die
Kybernetik nicht interessant. Hauptsache, die Struktur stimmt. In
komplizierteren Fällen sind mehrere Kreise miteinander zu einem System
verflochten. Verschiedene Typen von Rückkopplungen beeinflussen sich
gegenseitig zu Systemen. Natürlich beherrschen das nur die Fachleute. So ist
Dialektik von Wechselwirkungen. Die graphische Darstellung – die Formatierung
- nennt man „Modell“. Mit solchen Modellen muss man in der Kybernetik
arbeiten. Doch da gab es z.B. bei Philosophen mit ihrer akademischen
Sprachgewohnheit das nächste Problem. Als ich in einem Doktoranden-Seminar von
Modellen sprach, gab es Protest. Beim Worte „Modell“ dachten sie an
Schaufenster für Damenmoden und an Laufstege.

\begin{frage}
  Und wie war das bei Ökonomen?
\end{frage}
Die Spezialisten für Ökonomie zeichneten mit Bleistift Kreise auf Papier. Doch
sie unterließen es, die Kreise wenigsten auch mit einem Pluszeichen
beziehungsweise Minuszeichen zu kennzeichnen. Ironiker nannten das
„Kästchen-Malerei“. In jener Zeit aber sagte Walter Ulbricht sein energisches
Wort. Primäre Folge war, dass manche Doktoren und Professoren sich plötzlich
zur Kybernetik bekannten, um eine höhere Position an den Hochschulen zu
ergattern.

\begin{frage}
  Hat das denn niemand gebremst?
\end{frage}
Ein Beispiel: Wieder einmal hatte ich Genossen Professor Georg Klaus zu
vertreten: Im Zentral-Institut für sozialistische Wirtschaftsführung am
Müggelsee, wo Minister und Kombi\-nats-Direktoren weitergebildet wurden. Da
kritisierte ich auch die sogenannte Kästchen-Malerei. Doch der moderierende
Professor fiel mir ins Wort: Wieso kritisierst Du denn plötzlich die
Kybernetik? Da wurde ich zum Glück vom Institutsdirektor beschützt, der mich
verstanden hatte. Das war Gen. Prof. Helmuth Koziolek, Mitglied des
Zentralkomitees der SED. Als ich nach der „Wende“ Koziolek fragte, wer denn
ebenso dachte wie er, kam die Antwort: „Es waren nur drei Personen. Doch das
weiß niemand.“

Interessant, dass ich ausgerechnet vom Wirtschafts-Historiker Thomas Kuczynski
verstanden worden war. Das entdeckte ich, als ich mir zufällig seine Website
im Internet ansah. Er rühmte meine Habilitationsschrift, die als dickes Buch
mit dem Titel \emph{Quantität oder Begriff} 1967 im VEB Deutscher Verlag der
Wissenschaften erschienen war. Mit dem Titel wollte ich ausdrücken: Nicht nur
Zahlen und Mengen, sondern Begriffe und Modelle von Systemen.

\begin{frage}
  Und was war denn nun in der Akademie der Wissenschaften mit ihren vielen
  Mitgliedern aus allen Disziplinen herausgekommen?
\end{frage}
Eine Kybernetik-Konzeption ist unter Georg Klaus ausgearbeitet worden. Doch
dann war Schluss. Mein Freund Heinz Liebscher hat das aufgeschrieben.
\newpage
\begin{frage}
  Es soll aber danach eine zweite „Kybernetik-Welle“ gegeben haben.
\end{frage}
Ja. Das kam so: An der Humboldt-Universität lehrte und forschte der geniale,
auch in Mathematik beschlagene Psychologie-Professor Friedhart Klix. Er war
bald auch Präsident der Welt-Föderation der Psychologen. Sein Hauptwerk hieß
\emph{Information und Verhalten. Kybernetische Aspekte der organismischen
  Informationsverarbeitung} (VEB Deutscher Verlag der Wissenschaften, 1973),
gewidmet seinen strebsamen Mitarbeitern. Doch das genügte ihm nicht. Zurecht
wandte er sich an Prof. Max Steenbeck, den Präsidenten des Forschungsrates der
Deutschen Demokratischen Republik, in dem Experten naturwissenschaftlicher und
technischer Disziplinen demokratisch miteinander verbunden waren. Das
Ministerium für Wissenschaft und Technik hatte den Forschungsrat zu
unterstützen. Doch die Abteilung für Technik und Naturwissenschaft wollte
nichts von Kybernetik wissen. Aber die Genossen wandten sich an mich: Begleite
du den Klix zum Steenbeck.

\begin{frage}
  Und was kam heraus?
\end{frage}
Nach drei Stunden entschied Steenbeck: Erster Schritt: Demokratische Beratung
im Vorstand des Forschungsrats. Ziel: Eine Kommission und eine Konzept für
Kybernetik-Forschung. Die Mathematiker unterstützten uns, die Physiker waren
dagegen. Einer von ihnen meinte, die Physik der Atomhülle wäre die beste
Kybernetik. Da sprang ein Technik-Professor für uns ein. Damit war der Weg
frei gemacht u.a. für die Gründung eines großen Instituts für Kybernetik und
Informationsprozesse. Klix und Thiel machten sich ans Werk. Doch der Präsident
der Akademie der Wissenschaften war schon wieder dagegen. Thiel geriet mit ihm
in Streit, als es um die Räume für das Institut ging. Endlich waren zwei alte
Baracken in Berlin Adlershof gefunden. Und bald meckerte die Akademie der
Wissenschaften erneut. Das habe ich 42 Jahre später in meinem Lebenslauf-Buch
erzählt. Heute frage ich mich: Soll man sich nach 42 Jahren immer noch darüber
ärgern? Nein, man soll den Kontrahenten längst wieder die Hand gereicht haben.
Doch Geschichte muss trotzdem verstanden werden. Die Akademie gehörte zum
Führungskreis „Gesellschaftswissenschaften“ des Zentralkomitees der SED, also
der Philosophie und der Bildung, auch der Ausbildung von Ökonomen.

\begin{frage}
  Bist du nun fertig?	
\end{frage}
Nein, immer noch nicht. Auch E. Honecker und einige Genossen Physiker wollten
uns bremsen. Und die Akademie machte weiter Sperenzien. Auch VEB Carl Zeiss
Jena wollte nicht mitmachen wegen hochvertraulicher Kooperation mit der
Sowjetunion.  Endlich feierlicher Gründungsakt in einem großen Saal. Als Thiel
sich dem Eingang näherte, wurde er begrüßt von dem Physiko-Mathematiker
N. J. Lehmann aus Dresden, dem Schöpfer des weltweit-ersten
programmgesteuerten Kleinrechners. Und was sagte N. J. Lehman zu Thiel? „Herr
Klix und Thiel sind die Väter des Instituts.“ Es wurde bald zum produktivsten
Institut der Akademie. Und N. J. Lehmanns Nachlass wird vom Deutschen Museum
in München aufgenommen, gepflegt, und der weltweit erste Kleinrechner wurde
von Konrad Zuse geweiht, der den weltweit ersten Rechenautomaten gebaut hatte.
\newpage
\begin{frage}
  Du hast Dich vor allem mit Philosophen und Ökonomen gebalgt. Kannst Du mit
  einem einzigen Satz ausdrücken, wie sich das Studium der Kybernetik auf sie
  auswirken müsste? Kannst Du das wenigsten für die Komponente der Kybernetik
  sagen, die du favorisiert hast?
\end{frage}
Durch Denkanregungen und Ergänzung von dialektischen Strukturen im Denken
aktiv handelnder Menschen.

\begin{frage}
  Und was war denn mit den Praktikern der Industrie?
\end{frage}
Das waren Ingenieure, die gründlich studiert hatten, hochengagiert waren,
keine Ruhe kannten und dabei ihr Denken tainierten.

Als ich mich ein paar Jahre später mit der Methodik des Erfindens und der
Gründung von Erfinder-Schulen befasste – heute würde man sagen ein- oder
zwei-wöchige Workshops mit Ingenieuren aus der Industrie, gemeinsam mit
Verdienten Erfindern der Deutschen Demokratischen Republik –, bekam ich es zu
tun mit einigen VEB-Leitern und Generaldirektoren von Kombinaten. Der Reihe
nach: Der Direktor für Forschung und Entwicklung des VEB Berliner
Werkzeugmaschinenfabrik Marzahn. Dann der Chef des VEB Berlin-Chemie und sein
Stellvertreter. Dann der Forschungsdirektor des VEB Energie-Kombinat
Berlin. Dann der Generaldirektor des Kombinats Kabelwerke Oberspree. Dann der
Chef der Phosphorfabrik des Düngemittelwerks Piesteritz. Zuvor schon der Chef
des VEB Glaswerke Jena. Und schließlich Prof. Dr.-Ing. Karl Döring, der
Generaldirektor des Kombinats Stahl- und Walzwerke EKO Eisenhüttenstadt. Nach
der sog. Wende war das EKO privatisiert worden, schließlich in den Besitz
eines indischen Milliardärs übergegangen, des größten metallurgischen
Super-Unternehmers in Europa. Doch Karl Döhring blieb der technische Chef in
Eisenhüttenstadt.
\vfill
\ccnotice
\end{document}
