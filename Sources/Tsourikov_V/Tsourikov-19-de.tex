\documentclass[11pt,a4paper]{article}
\usepackage{od}
\usepackage[utf8]{inputenc}
\usepackage[russian,ngerman]{babel}

\title{30 Jahre Projekt „Invention Machine“}

\author{V. Tsourikov, Minsk}
\date{TRIZ Summit 2019 in Minsk}

\begin{document}
\maketitle
\begin{quote}
  Original: \foreignlanguage{russian}{30 лет проекту «Изобретающая
    машина»}\footnote{\url{http://triz-summit.ru/file.php/id/f304802-file-original.pdf}.}.
  Proceedings of the TRIZ Developers Summit, June 13--15, 2019, Minsk,
  Belarus.  Übersetzt von Hans-Gert Gräbe, Leipzig. 
\end{quote}

\begin{abstract}  
  Der Bericht diskutiert die Entwicklung des Invention Machine Projekts von der
  Idee des „Computers als intelligenter Assistent des Erfinders“ zum Konzept des
  „Computers als realer Erfinder, als ebenbürtiger kreativer Partner des
  Menschen“. Anstelle von Hilfestellungen in Form von Effekten oder Zugängen
  erzeugt das intellektuelle System Erfindungen auf dem Bildschirm.
\end{abstract}

\emph{Schlüsselwörter:} Generative Inventions, automatisch generierte
Erfindungen, True Machina\footnote{ Invention Machine ist eine Marke der
  Invention Machine Corp.  True Machina ist eine Marke von Prediso.  }

\section{Lem, Turing und ein Programm für Minsk-22}

Unmittelbar nach dem Eintritt in das Fachgebiet Computertechnik des Minsker
radiotechnischen Instituts hatte ich das Glück, Lems „summa technologiae“ zu
lesen und das Konzept der universellen Turingmaschine kennenzulernen.  Die
grenzenlosen Möglichkeiten der kybernetischen Welt Lems und die unglaubliche
Schönheit von Turings abstrakter Mathematik bildeten zusammen eine explosive
Mischung, die mich stark bewegten. Ein Computer kann alles! Es blieb, ein
würdiges Thema für meine eigenen Forschungen in der künstlichen Intelligenz zu
finden. Den Computer in einen Erfinder verwandeln -- klang verrückt, aber
spannend.

\subsection{Der kombinatorische Synthetisator}

Als Diplomprojekt sollte ein Digital-Analog-Wandler entworfen und gebaut
werden.  Die Arbeit verzögerte sich aufgrund der großen Streuung der Parameter
der einzubauenden Widerstände. Es entstand die Idee, ein Programm zu
schreiben, das eine Vielzahl von Verbindungsvarianten von Widerständen
generieren und aus diesen die passendste Konfiguration auswählen kann.  Das
wurde auf einer Minsk-22, in der Sprache Autocode ENGINEER umgesetzt, der
Algorithmus auf einem Lochstreifen „gespeichert“.

Das Programm funktionierte, es war ein Qualitätskonverter für schlechte
Widerstände entstanden. An die emotionale Wirkung des Wunders erinnere ich
mich bis heute. In diesem Algorithmus aus dem Jahr 1973 sind beide Teile des
kreativen Prozesses enthalten: die Ideengenerierung und die parametrisierte
Qualitätsbewertung der Ideen.

Das Programm wurde von einem Menschen geschrieben, aber den gesamte kreative
Zyklus der Suche nach der besten Lösung führte der Computer aus. Der Computer
kann erschaffen, und zwar auf seine eigene Weise -- das war eine wichtige
Schlussfolgerung.

\subsection{Das intelligente System \emph{Pulsar}}

Der nächste Schritt war das intelligente System \emph{Pulsar} für die
automatische Synthese neuer Methoden zur Auswertung kosmischer Signale im
SETI-Projekt.  Neun Jahre der Arbeit, zweihundert Softwaremodule -- das
Ergebnis: die automatische Synthese von zwanzigtausend neuen
Signalverarbeitungsalgorithmen [1]. Dieses System verfügte bereits über eine
Wissensbasis aud mathematischen Theoreme (als \emph{mathematische Effekte}
bezeichnet), aus denen automatisch Signaldetektorstrukturen erzeugt und deren
Qualität durch eine Monte-Carlo-Methode bewertet wurde.

Im Wesentlichen arbeitete der 1983er Pulsar wie Edison bei der Materialauswahl
für einen Faden seiner Glühbirne. Dies war zugleich die erste wirkliche
Erfinder-Maschine, allerdings in einem sehr engen Bereich. Das System
\emph{Pulsar} schlug dem Benutzer keine Tricks oder Effekte vor, sondern
synthetisierte neue Erfindungen und bewertete deren Qualität.

\subsection{Die Schule des jungen Erfinders}

Das Pulsar-Projekt wurde von niemandem finanziert, obwohl es das vage Gefühl
gab, dass dies in Zukunft passieren könnte, und dann würden begeisterte
Entwickler benötigt. Ich verteilte an die Gruppen-Ältesten mehrere Dutzend
Kopien des Programms eines Kurses (künstliche Intelligenz, Psychologie der
Kreativität, Algorithmus der Erfindung) und bestellte einen Hörsaal.  So wurde
im Herbst 1976 die \emph{Schule des jungen Erfinders} geboren, die eine
außerordentlich wichtige Rolle im IM-Projekt spielte. Der Kurs erforderte
ernsthaftes Arbeiten über zwei Semester, mit Hausaufgaben und
Abschlussprojekten.  Jedes Jahr kamen bis zu 80 Personen in die ersten
Veranstaltungen, den Abschluss erreichten nicht mehr als 12. Aber was für
Absolventen!  Jeder ein geborener Leiter, mit innerem kreativem Feuer.  Die
Absolventen der Schule träumten nicht nur von Kreativität, sie waren ernsthaft
bereit und innerlich überzeugt, sich in das Projekt hineinzuknien.

Es ist bezeichnend, dass später, als sich in Minsk Software-Unternehmen
ausbreiteten und mit hohen Gehälter winkten, nur wenige der Jungs aus der
ersten Gruppe des IM-Projekts sich im langweiligen Outsourcing versuchten.

\section{Die Erfinde-Maschine (IM)}
\subsection{NILIM -- das wissenschaftliche Forschungslabor für
  Erfindemaschinen.  TRIZ als Basis für die IM}

1987 gelang es dem MRTI\footnote{Minsker Radiotechnisches Institut}, ein Labor
für intelligente Systeme zu eröffnen, in dessen Bestand die Absolventen der
Schule aufgenommen wurden. Die Begeisterung war sehr groß, aber auch die
Bürokratie war nicht schwach und fraß die meiste Zeit des Laborleiters.

Wir beschlossen, eine Genossenschaft zu gründen. Die Firma
\emph{Wissenschaftliches Forschungslabor für Erfinde-Maschinen} wurde am
12. April 1989 registriert. Das Motto der Firma war „Hohe Qualität und neueste
Technologie“. Es wurde eine Lizenz für die Sprache PROLOG vom britischen
Unternehmen LPA erworben, gegen die künftigen Lieferung von Software erhielten
wir importierte Personalcomputer vom Minsker Zahnradwerk. Wir mieteten ein
Büro in einer großartigen Lage, neben einem Café, in dem Sie einen starken
Kaffee brühten. Ab da verlief die Entwicklung in hohem Tempo.

Die Atmosphäre in der Firma war freundlich und unglaublich kreativ. Zweimal im
Jahr veranstalteten wir Ausstellungen und Seminare für Anwender, die Vorträge
wurden veröffentlicht. Jedem Mitarbeiter waren vier Firmen zugeordnet, die ein
IM-System gekauft hatten, der regelmäßig die Anwender anrief und so wertvolles
Feedback aus erster Hand erhielt. Übrigens waren die Erfahrungen der Firma
NILIM nützlich bei der Gründung der TRIZ-Vereinigung in Petrosawodsk.

Die auf TRIZ-Methoden basierende Invention Machine [2] war kein automatischer
Ideen-Synthetisator, sondern unterstützte die Erfinder intellektuell und war
aus der Perspektive der künstliche Intelligenz ein Schritt zurück im Vergleich
zum Pulsar-System, dafür aber war sie für eine viel größere Anzahl von
Anwendern interessant, insbesondere beim Erlernen der TRIZ-Methoden und der
Logik der Funktions-Wert-Analyse.

\subsection{Die Praxis der Verwendung von IM}

Die Gründung der Firma \emph{Invention Machine Corporation} in Boston am
7. April 1992 erlaubt die Beschaffung von Kapital von Motorola, Intel, von
Risikokapitalfonds und die Bereitstellung von Schulungen und Beratung für
große technische Firmen (Motorola, Kodak, Xerox, Intel, Procter und Gamble,
General Electric, NASA, BMW, Ferrari, Mitsubishi, Honda, Unilever usw.).  Der
\emph{TechOptimizer} war gut für den Unterricht einzusetzen, aber die Anzahl
der aktiven Benutzer wuchs langsam.

Bei der Lösung realer Probleme bemerkten die Ingenieure schnell die
Einschränkungen der TRIZ-Wissensbasis und verloren einfach das Interesse.
Mehr noch, selbst richtig gute Empfehlungen des Systems wurden von den
Ingenieuren oft abgelehnt, ohne auch nur zu versuchen, die empfohlenen Zugänge
oder Wirkung anzuwenden wegen der Barriere zwischen zu abstrakten
Systemempfehlungen und dem konkreten Problemmodell im Kopf des Anwenders.

Um die Möglichkeiten des Systems zu erweitern, wurden zwei Projekte gestartet:
ein Ursache-Wirkungs-Synthetisator und ein semantischer Prozessor für die
Suche nach notwendigem Wissen im Internet. Beide Projekte erforderten viel
Kapital und Aufwand.

\subsection{Synthetisator für Effektketten}

Direkte logische Schlussketten (von Fakten zu einem Ziel) und inverse (von
einem Ziel zu Fakten) bilden den Kern kreativen Denkens. Warum nicht beide
Arten von Schlussketten zur wissenschaftlichen Datenbasis hinzufügen, damit
die Ingenieure das neue System aktiver anwenden können? Das war ein Schritt in
Richtung der Erweiterung des Raums möglicher Lösungen, und wir rechneten mit
einer starken Zunahme der Zahl aktiver Nutzer. Die Realitäten waren etwas
anders.  Das System erzeugte interessante Wirkungsketten, aber die Barriere
der Missverständnisse blieb bestehen und, mehr noch, verstärkte sich, da
anstelle eines Effekts nun gleich mehrere zu analysieren waren. 

Ich habe persönlich Fälle beobachtet, in denen die Invention Machine sehr
vielversprechende zusammengesetzte Konzepte ausgab, aber der Anwender nicht
einmal versucht hat, diese zu studieren. Drei unbekannte Effekte zu gleicher
Zeit -- die Ingenieure weigerten sich, solche Konzepte zu analysieren.  Dieses
Problem brachte uns auf die Idee, dass, wie schwierig dies auch sein mag, die
Ideen nicht in Form von Effekten oder Wirkungen zu präsentieren, sondern in
Form grafischer Darstellungen technischer Systeme. Das heißt, es war ein
echter KI-Erfinder zu programmieren, der nicht nur Vorschläge macht, sondern
selbst erfindet und die Erfindung auf dem Bildschirm darstellt. Diese Idee
wurde später zur Grundlage des Projekts \emph{True Machina}.

\subsection{Der semantische Prozessor}

Der Einsatz des IM-Systems zur Lösung praktischer Probleme in führenden
technischen Unternehmen der USA, in Westeuropa und Asien führte zum
Verständnis, dass die korrekte Formulierung der Erfinderaufgabe in Form einer
technischen Funktion von der sofortigen Bereitstellung der erforderlichen
Wissensquellen nicht nur aus der Datenbasis, sondern aus dem ganzen Internet
begleitet werden muss, in denen die Implementierung dieser und ähnlicher
Funktion zu sehen ist.

Angenommen, Im Prozess der Analyse einer Aufgabe wurde als wichtige technische
Funktion identifiziert \emph{Erhöhen Sie die Stärke der Turbinenschaufeln}. Es
wäre dannnicht schlecht, Zugang zu den neuesten Forschungsergebnissen in
diesem Bereich zu haben, und zwar nicht in Form einer endlosen Liste von
Webadressen, sondern in Form von kurzen kausalen Beziehungen der Art
\emph{Eine chaotische Struktur erhöht die Stärke der Schaufeln}. Wir haben es
geschafft, einen leistungsstarken und genauen \emph{Miner of Knowledge}
basierend auf einer tiefer Semantik zu bauen [3].

Wie es manchmal in der Softwareentwicklung vorkommt, wurde das semantische
Modul von einem Hilfsprodukt später zum Hauptprodukt der Firma \emph{IM Corp}.
Das semantische Netzwerk reflektiert nicht einfach die Struktur des Wissens,
sondern enthält auch argumentative Elemente und logische Schlussfolgerungen
aus dem Wissen.

Der semantische Prozessor ist bei Kunden der IM Corp. sehr beliebt; wir haben
auch versucht, Investitionen für den Start einer semantischen Suchmaschine
einzuwerben, aber in der Wirtschaft begann eine Rezession und die Investoren
wollten nicht mehr ernsthaft investieren. Außerdem mussten wir mit dem
Zusammenbruch der Aktien an den Börsen den geplanten Einstieg des Unternehmens
an der NASDAQ abblasen.

\subsection{Die wichtigsten Schlussfolgerungen aus den Erfahrungen mit der
  Implementierung des IM-Systems}

Intelligente Unterstützungssysteme zur Lösung erfinderischer Probleme auf der
Basis von FSA und TRIZ sind sehr nützlich, um gleichzeitig TRIZ und FSA zu
lernen, allerdings werden zur Lösung praktischer Probleme die Systeme nur von
einem kleinen Kreis von Enthusiasten verwendet. Zum Teil wegen der Komplexität
der Theorie und der fehlenden Ergebnisgarantie.

Das semantische Modul hat die Anzahl der Anwender erhöht, allerdings hatte das
nur indirekt Auswirkungen auf die Ideen-Generierung; die schöpferische Arbeit
verlagerte sich in Richtung der Textanalyse von Artikeln und Patenten. Eine
deutliche Erhöhung der Anzahl aktiver Anwender ist unter zwei Bedingungen
möglich: der deutlichen Vereinfachung der Komplexität der Logik und der
Wissensbasis, der maximalen Reduktion der Barriere vom Hinweis bis zur
kompletten Lösungsidee sowie der ständigen Aktualisierung der Wissensbasis
über Probleme, wissenschaftliche Effekte, neue Materialien und technische
Prozesse.

\section{True Machina als wahre KI-Erfinderin}

Das Konzept des True Machina Projekts unterscheidet sich stark von der
IM-Philosophie auf der Grundlage der TRIZ.  Das IM-System basierte auf der
Logik der FSA und der TRIZ-Wissensbasis, die für Menschen erstellt wurde, so
dass neue Ideen im Kopf des menschlichen Benutzers entstehen.

Das Hauptkonzept des neuen Projekts besteht darin, anstatt Hinweise zu
generieren, die dem Menschen beim Problemlösen helfen, den Computer selbst
neue Lösungen in Form von Erfindungen mit Abbildungen, Beschreibung, Titel und
Nummer der Erfindung erzeugen zu lassen. Der Unterschied ist grundlegend: Wenn
bisher die Idee der Erfindung im Kopf des Benutzers des Programms entstand, so
wird nun die Idee im Speicher des Computers generiert, der Benutzer sieht auf
dem Bildschirm keinen Hinweis, sondern eine Reihe neuer technischer Lösungen.

Das intellektuelle System erfindet selbst und unterstützt nicht nur den
Menschen beim Erfinden.  Künstliche Intelligenz beginnt die Rolle des Partners
zu spielen: Der Mensch generiert Ideen und die \emph{True Machina} generiert
ebenfalls Ideen. Beide am kreativen Prozess Beteiligten entwickeln
Erfindungs-Ideen, aber sie tun es auf unterschiedliche Weise. Dies ist die
Stärke der Partnerschaft zwischen Mensch und \emph{True Machina}.

Die Funktionalität des Systems True Machina [4]:
\begin{itemize}
\item Synthese von Erfindungen in Form von Grafiken und Texten -- Quelle des
  Wissens sind die neuesten wissenschaftlichen Veröffentlichungen;
\item Modus \emph{Generative Inventions}: Neue Erfindungen werden sofort nach
  dem Auftreten eines neuen Effekts, Materials oder Problems in der
  Wissensbasis generiert;
\item Vergleich der erzeugten Erfindungen nach einem quantitativen
  Hauptparameter und nach eine Gruppe von Qualitätsindikatoren;
\item Kontinuierliche Auffüllung der Wissensbasis;
\item Anpassung an die Probleme großer Anwenderunternehmen.
\end{itemize}

\subsection{Der kreative Mensch-Maschine-Dialog. Wo ist die KI stark?}

Der Computer ist in der Lage, Informationen in gigantischem Umfang zu
speichern, diese schnell und einfach zu verarbeiten und in visueller Form
auszugeben. Heute ist er auch lernfähig und kann argumentieren, aus Wissen
Schlüsse ziehen, was für die Kreativität von grundlegender Bedeutung ist.

In Anbetracht der Möglichkeiten heutiger Computer und KI-Systeme stellen wir
uns folgende Aufgabe: Programmiere einen Systemkern, der ständig eine Basis
wissenschaftlicher Erkenntnisse und experimenteller Forschungsergebnisse
aktualisiert und mit einer hierarchischen Liste technischer Probleme oder
Funktionen semantisch abgleicht.

Es gibt also \emph{zwei} sich ständig aktualisierende Wissensdatenbanken.
Wenn als Eingabe im System als kurzer Eintrag eine Beschreibung eines neuen
Forschungsergebnisses auftaucht, dann durchsucht das System sofort alle
technische Probleme und versucht, die Probleme zu entdecken, die mit der neuen
Erkenntnis (besser) gelöst werden können. Wenn das TM-System solche Probleme
gefunden hat, so erzeugt es neue Erfindungen in Form eines synthetisierten
Musters, einer kurzen Beschreibung und Titel, das heißt, eine Beschreibung
einer neuen Erfindung wird erzeugt. Praktisch werden so wissenschaftliche
Erkenntnisse automatisch in technologische Innovationen übersetzt.

Wir haben diesen Prozess als \emph{Generative Inventions} bezeichnet. Es gibt
noch keinen russischen Begriff dafür. Die Essenz dieses Ansatzes sieht vor,
dass die konzeptionelle Lösung eines technischen Problems nicht generiert
wird, wenn der Ingenieur danach fragt, sondern in dem Moment, wenn
wissenschaftliche Erkenntnisse für dessen Lösung in das System eingespeit
werden. Dies verkürzt die Zeit von der wissenschaftlichen Entdeckung bis zur
technischen Erfindung erheblich.

Die im Projekt verwendete Technologie nennen wir \emph{kombinatorische
  Semantik}.  Kombinatorik erlaubt es, alle möglichen Erfindungen aus einem
neuen Effekt oder Material für den aktuellen Baum der technischen Funktionen
zu entwickeln.  Die Semantik ist verantwortlich für kausale logische
Schlussfolgerungen, das heißt nmittelbar für den kreativen Prozess. Faktisch
kehren wir zur Architektur des Pulsar-Systems zurück, aber ergänzt um
Semantik, welche die Mächtigkeit des Systems stark erhöht.

\begin{quote}
  Abb. 1. True Machina generiert automatisch neue Erfindungen aus der
  Wissenschaft und experimentelles Wissen -- hier weggelassen.
\end{quote}

\subsection{Milliarden Erfindungen}

Wir bauen den Kern des True Machina Systems so, dass es möglich ist, mehr als
einer Milliarde Erfindungen automatisch zu erzeugen, was die Anzahl der auf
der Welt geschaffenen Patente um fast zwei Größenordnungen übertrifft. Ein
solches Ziel kann mit entsprechenden finanziellen und strategischen Partnern
in wenigen Jahren erreicht werden.

Die Entstehung einer solchen Datenbank mit einer Milliarde Erfindungen im
Internet wird zweifellos eine große Wirkung bei Ingenieuren, Beratern und
Patentanwälten haben. Die Basis wird kontinuierlich aktualisiert durch die
automatische Erzeugung neuer Erfindungen unmittelbar nach der Veröffentlichung
der Ergebnisse fortgeschrittener wissenschaftlicher Forschung. So wird True
Machina wissenschaftliche Ergebnisse sofort in technische Innovationen
umsetzen.  Es besteht die Möglichkeit, dass True Machina der dominierende
Erfinder wird.

\section{Fazit}

Die intelligenten Systeme des Projekts Invention Machine helfen bei der Lösung
erfinderische Aufgaben auf der ganzen Welt. Die Mitarbeiter der Firma NILIM
(Minsk, St. Peterburg), die IM Corp., NILITIS, Method NauchSoft haben
tatsächlich eine neue Klasse von Softwarelösungen erstellt -- intelligente
technische Unterstützungssysteme für Innovationen.

Das nächste Ziel ist der Aufbau eines Systems künstlicher Intelligenz, das
laufend neue Erfindungen und deren textuell-grafische Beschreibungen erzeugen
kann als gleichwertiger kreativer Partner eines Ingenieurs, Forschers oder
Erfinders.

\section{Referenzliste}
\raggedright
\begin{itemize}
\item [1.] Tsourikov V. \foreignlanguage{russian}{Разработка быстрых ранговых
  обнаружителей сигналов в сложных условиях связи}. Dissertation. Minsk, 1983.
\item [2.] Altshuller G. \foreignlanguage{russian}{Творчество как точная
  наука. Советское радио}, 1979.
\item [3.] Tsourikov V., Batchilo L., Sovpel I. Document semantic
  analysis/selection with knowledge creativity capability utilizing
  subject-action-object (SAO) structures. United States Patent 6\,167\,370
  vom 26. Dezember 2000.
\item [4.] Tsourikov V. Generative inventions. TRIZfest, Conference
  Proceedings. September 14-16, 2017, Kraków, Poland, 178--184.
\end{itemize}
\end{document}
