\documentclass[11pt,a4paper]{book}
\usepackage{od}
\usepackage[russian,english]{babel}
\usepackage[utf8]{inputenc}

\newcommand{\Missing}{
  \begin{quote} Here is text missing in the translation.\end{quote}
}
\newcommand{\addpicture}[1]{
  \begin{quote} Add picture: #1\end{quote}
}

\title{Features of the use of TRIZ for solving organizational-managerial
  tasks (OMT): schematization of an inventive situation and work with
  contradictions.\\[2em]}

\author{Anton Kozhemyako, Chelyabinsk}
\date{August 2019}

\begin{document}
\begin{titlepage}\centering\Large\mbox{}\vskip3em
  Anton Kozhemyako\vskip3em
  
  {\LARGE Special Features of the Use of TRIZ for Solving\\[12pt]
    Organizational-Managerial Tasks (OTM): Schematization of an\\[12pt]
    Inventive Situation and Work with Contradictions

  }\vfill

  Thesis for the title \emph{TRIZ Master}\vskip1em

  Scientific advisor: TRIZ Master Valery Souchkov\vskip1em
  
  Chelyabinsk, 2019
\end{titlepage}
\thispagestyle{empty} \mbox{}\vfill

Original:\\[1em] \foreignlanguage{russian}{Антон Кожемяко.\\ Особенности
  применения ТРИЗ для решения организационно-управленческих задач:\\
  схематизация изобретательской ситуации и работа с противоречиями}.

Translated by Hans-Gert Gr\"abe, Leipzig.
\vskip2em
\ccnotice
\cleardoublepage
\tableofcontents
\cleardoublepage
\chapter{Introduction}

This work belongs to the field of Theory of Inventive Problem Solving.

The work consists of 2 sections and appendices.

The first section is devoted to schematization used in the analytical phase of
an inventive situation\footnote{This is \emph{a situation characterized by the
    presence of necessity to satisfy the demand of a specific supersystem
    without a clearly defined set of problems for the further solution or
    direction of solving the problem} [42].} solving OMT. It is shown that
most problems in organized social systems are set by the stakeholders in a
form that is not enough informative for its processing using TRIZ tools. What
is typical in general also for tasks in any other areas, for example:
«increase productivity of the line by 5\%.» Such tasks always come with high
uncertainty, since resources to achieve the goal in these tasks are drawn from
soft systems such as human and their interaction.

The most convenient way of initial presentation of OMT in a form convenient
for further processing, according to the author, is schematization used by the
followers of G.P. Shchedrovitsky, which was initially developed by the Moscow
methodological circle (MMC) specifically for the analysis of such a class of
tasks. In the section it is described in detail how TRIZ tools «are arranged»
for a preliminary analysis of the problem using schematization.

The second section is devoted to the choice of the operational
zone\footnote{Part of the physical spaces where the conflict or undesirable
  effect arizes that generates the inventive situation [42]. It is worth
  noting that the area in which the conflict in solving OMT is located, is not
  necessarily defined as a physical point in space -- remark by author.} for
OMT and the allocation of resources in the operational zone. This is also
addressed by M.S. Rubin in his works: «The character of the interaction field
in business systems determines the different nature of the space or zone of
conflict. This is not a physical space, as it is usually the case in technical
systems, but rather a multidimensional space-set, consisting of conflicting
elements and relationships between them» [44].

In this work the scope of application of this tool is described and explained
why this approach is preferable precisely for this class of problems.

In the Appendix the practical application of the described methods is shown.

\section{Relevance of the research topic}

Since the 1990th, in the TRIZ environment the application of the methodology
to the solution of problems in social systems has been actively discussed
([1], [2], [3], [4], [5], [36], [37] and many others sources). A number of
TRIZ specialists successfully applies its tools in projects on business
systems and other organized social systems, for example, to the evolution of
creative teams [37], to other organized social systems, whose purpose is not
to make profit (government structures, military units, police, judicial
system, healthcare, etc.). For that time TRIZ specialists have accumulated
many successful cases in solving problems in social systems, which indicates
the intensive development of TRIZ in this direction (in a number of works,
business systems are not classified as social ones [5], but as information
systems [1]). Some examples of TRIZ applications to business tasks are also
described on the website of the author [6].

As any artificially created (organized) system a social system tends to
entropy [23], [38], hence in such systems an obligatory function is control.
When implementing a control function in a dynamically changing supersystem,
modern managers face many challenges, which the author calls
organizational-managerial [11]. The notion «organizational-managerial task»
(OMT) will be unfolded in more detail below.

Most of such problems usually do not cause difficulties to managers and are
solved by analogy, since most situations with which a manager is faced in
everyday practice, are of known type [10]. However in conditions of modern
rapidly changing economic and social reality the manager is faced with many
inventive situations [39], which are difficult to resolve in the usual way
[11].  It should be noted that the use of TRIZ for solving OMT ist still not
structured, up to initial provisions. Take at least an emphasis in the
definition of business systems: a number of authors (N.N. Khomenko [37],
V.A. Korolev [5], B.V. Shmakov [8], etc.) call such systems social, and, for
example, E.A. Sosnin and B.N. Poyzner [1] call them informational).  In a
number of works such systems belong to some indefinite set -- so called
«non-technical systems» (for example, [7]) or business systems that is already
somewhat more accurately [2]. Even if, of course, business systems are
important, they are still subsystems of a larger class of systems -- organized
social systems, which include, as mentioned above, not only profit oriented
systems. OMT can arise in any of the listed types of organized social systems,
as in any of these systems there is a management function, during the
implementation of which inventive situation arize.

Many attempts have been made to transfer TRIZ tools designed for solving
problems of transforming technical systems to OMT, some of which have taken
well, whereas some of them are the product of a direct, but ineffective
transfer from one sphere to the other, therefore the use of such tools is
doubtful, for example, the «direct translation» of the matrix of
G.S. Altshuller into the language of business systems [8] (it should be noted
here that along with attempts to directly but ineffectively transfer the
methods of resolution of technical contradictions there are also deeply
developed versions for solving problems in the field of business, for example,
the matrix of D. Mann, author of \emph{Hands-on Systematic Innovation for
  Business and Management} [9]), however, these methods can only be applied
after formulating a contradiction, which is hardly possible at the beginning
of work on OMT.

In general, the problem of studying the inventive situation before applying
TRIZ tools is a separate large-scale task. In many aspects the application of
TRIZ for solving such problems is “lame” precisely because of lack of tools
for reliable inventive situation analysis.

In addition, there is an assumption that the term «business system» [2]
includes partly (but does not completely cover) both social systems and
informational ones.  However, this classification does not outline sharp edges
of such systems and it is difficult to understand where the social system ends
and the informational one begins. Since the systems described by the above
concepts are very intertwined and the boundaries between them blurred, the
author does not consider it appropriate to separate in business systems social
and informational subsystems.  It’s enough to understand that business systems
are part of a larger system: organized social systems.  The author believes
that it is easier to deal with the concept of
\textbf{organizational-managerial tasks} (MT), this concept indicates that
\textbf{the task is set in any organized social system by a subject which has
  a goal of producing a certain improvement in the interaction of the elements
  of the business system bound by informational and social relationships}
       [40].

Why does the author call this type of task organizational-managerial?  It is
known that organizational tasks are associated with optimisation of the
allocation of resources with the goal of getting the most impact out of them.
Below it will be shown that organizational tasks can be posed at the level of
both filling and generalized objects in business system (definitions of
generalized objects and filling are given below). \textbf{Organizational
  tasks} are connected with the organization of links between generalized
objects in the business system and filling generalized objects of a business
system according to its properties.  \textbf{Managerial tasks} are tasks
related to the \textbf{improvement of the efficiency of activities} of
elements of a business system that are in certain relationships. Since the
subject usually needs to increase the effectiveness of a business system or
some subsystem of it, it most often initates both organizational changes and
managerial influence.  In the literature, these types of influences are not
often distinguished (for example, [23]), however, these concepts are sharply
divorced in the work of G.P. Schedrovitsky [10], therefore, the author uses
this classification and talks about OMT if it is required to increase the
effectiveness of a organized social system (in particular, a business system)
or any of its subsystems. \textbf{Below the author uses the term «OMT» in the
  context of tasks related to improving the efficiency of organized social
  system or its subsystems}.

It is worth noting that the author did not meet generally accepted
terminology, which is clearly describing similar systems, with the exception
of the generally accepted position that an organized social system is designed
for the goals of the customer (short, medium or long term). Terms describing
the structure of organized social systems, not counting the generally accepted
classification describing the hierarchy of the internal structure --
organization, top-management, departments, sections and project teams [26],
[28], [32] -- are not known to the author.

Thus, the urgent question is the elaboration of an inventive situation for OMT
aiming at the possibility of further analysis provided that most solvers get
such tasks in insufficient formalized form, therefore, a method of preliminary
analysis is required for similar tasks. It should be noted that with the
problem of formalization of OMT are faced not only specialists in the field of
TRIZ, but also the members of the Moscow Methodological Circle (MMC) under the
leadership of G.P. Schedrovitsky [10] were actively engaged in such problems.
As a result of their activities, they developed the method of schematization
of such tasks [41], which perfectly copes with this problem, but generates
another one: this tool perfectly helps in the initial «entry to the task»,
that is, in the initial analysis of the inventive situation, but is
practically useless for its further solution, however, for the problem of
«solving in depth» management tasks through the identification of
contradictions and their subsequent resolution TRIZ mechanisms are well
suited. This thesis is confirmed by work experience of the author together
with representatives of the methodological school of G.P. Shchedrovitsky in a
number of projects.

In addition, trying to use the IFR operator to resolve inconsistencies in OMT
[11], the solver inevitably faces difficulties in determining the operational
zone (OZ) and the resources that can be mobilized for the search for the most
effective solution, since the boundaries of the OZ are outlined by abstract
concepts, rather than by the physical frame of conflict, as in technical
tasks. In this paper, an attempt is made to formalize the selection of the OZ
when resolving contradictions in managerial tasks.  The author shows how in
this way it is possible to outline the OZ not as a point in space, as in
technical tasks, but in the plane of abstract concepts that are often used in
the description of conflicts in organized social systems (motives, incentives,
reaction, values, desire, competencies, key performance indicators, etc.),
which is the core of a number of OMT [11]. Application of a similar approach,
including the use of a shortened version of ARIZ and its elements to solve
such problems is given in a number of cases on the author’s site [6].

The practical need for preparing OMT for further analysis using TRIZ tools is
long overdue. Also obvious is the need to offer a simple and convenient
mechanism for determining the operational zone, since the lack of a
methodologically developed mechanism for determining OZ restrains the use of
ARIZ for this class of problems [11]. The author believes that short ARIZ
versions (in 6-7 steps) are an excellent tool for solving OMT, which managed
to prove itself in practice as a reliable tool that provides sustainable
results.

\section{Goals and objectives of the research}
The objective of this work:
\begin{itemize}
\item Propose a way to formalize business tasks using schematization and draw
  up a roadmap for applying schematization to solution of OMT in order to
  further successful application of TRIZ tools.
\item Develop areas of transition from schematization to TRIZ tools, including
  the level of the terminological apparatus. This question is relevant also
  because between TRIZ specialists and SMD methodologists (followers of the
  school of G.P. Shchedrovitsky) an active exchange of information has been
  going on during the last several years, however the issue of transition
  between tools until now no one worked out;
\item Develop a method for determining the operational zone in OMT, taking
  into account the specifics of formulation of conflicting elements in similar
  tasks.
\end{itemize}

\section{Scientific novelty of the research}

The scientific novelty of this work is as follows:
\begin{itemize}
\item The author has developed a method for applying schematization to
  training OMT for the further use of mechanisms TRIZ as an indispensable
  condition for the analysis of inventive situations in the field OMT.

  The author conducted a detailed analysis of the work of G.P. Shchedrovitsky
  and based on the studied material developed a sequence of schematizations
  for the analysis of inventive situations simplifying the further use of TRIZ
  tools in order to solve such problems. The author considers analogues of
  such approaches used in TRIZ (system operator and functional modeling during
  the FVA), and concludes: \textbf{schematization has a unique mechanism for
    determining managerial layers, and also concepts of generalised object and
    filling, which provides new opportunities for statement of partial tasks
    in solving OMT, with the ability to scale the obtained solutions}. Such
  opportunities are missing in the existing scope of TRIZ tools, which
  significantly prevents the use of TRIZ mechanisms to solve OMT.
\item \textbf{The author proposes a system of problem setting} based on the
  results of schematization of inventive situations using by means of
  sequential analysis [11]:
  \begin{itemize}
  \item the model of a viable system (MVS) at the system-supersystem border;
  \item degrees of controllability by layers in the scheme;
  \item interconnections (links, functions, processes);
  \item generalized objects and their filling.
  \end{itemize}
  
\item A roadmap was compiled:
  \begin{itemize}
  \item Identify the problem situation.
  \item Define the conflict area and identify conflicting pairs (objects and
    subjects of OMT)
  \item Add system elements around the conflicting pair and identify secondary
    problem situations related to the task.
  \item Define the relationship between the elements of the system at the
    level of generalized objects («generalized object» is s term, which is
    explained in detail in the text of the thesis. The term does not replace,
    but complements the notion of an «element of the system», is a subsystem
    of a system element). If necessary identify processes.
  \item Identify the nearby elements of the system, including «regulators».
  \item Identify conflicting areas of generalized objects and fillings.
  \item Set up a system of tasks by carrying out an analysis:
    \begin{itemize}
      \item of a model of a viable system (MVS) at the system-supersystem
        border;
      \item of the degree of controllability by layers in the scheme;
      \item of the interconnections (links, functions, processes);
      \item of the generalised objects and their content.
    \end{itemize}
  \end{itemize}

\item A method for determining the operational zone in the OMT was developed
  that is characterized by a high abstraction of descriptive characteristics,
  as a result of which it is impossible to outline a part of real space, in
  which the conflict develops (different to most technical tasks).  This
  method makes it possible to use ARIZ mechanisms to resolve contradictions in
  OMT at a high degree of abstraction. \textbf{The novelty is that using the
    method, developed by the author, the solver can not only determine the
    operational zone as a physical contour of space (it is worth noting that
    this possibility reinforces the use of schematization, where it’s very
    convenient to elaborate such parts), but also to determine the operational
    zone directly from the formulation of a technical contradiction, and
    subsequently identify resources of the operational zone in the form of
    factors determining the system state and the properties indicated in the
    technical contradiction.} The ability to allocate resources from abstract
  concepts is the most important skill in solving OMT, since the physical
  contour of space can just not to be at the disposal of the solver [11].
\item A roadmap is compiled:
  \begin{itemize}
  \item Formulate a pair of TC (technical contradictions);
  \item Choose the working TC;
  \item The conflicting pair of the selected TC forms the operational zone,
    including tool and product;
  \item Identify the resources as a group of factors affecting the tool and
    the product;
  \item Further on, we work in the ARIZ logic: assign an IFR rule, substitute
    product and tool resources in the IFR rule, etc.
  \end{itemize}
\end{itemize}
It is worth noting that the author came across with the opinion of a number of
experts that in relation to OMT, it is incorrect to use the term «technical
contradiction». Some TRIZ experts consider it worth to identify market,
organizational, interpersonal and psychological (intrapersonal) contradictions
[40]. The author does not agree with this principle of division, since TC is a
form of conflict representation, and the listed contradictions do not relate
to the form of presentation of information, but to the level of solving the
problem (in TRIZ initially considered as macro and micro level, and such a
classification applies clearly to the level of problem solving. If the term
«technical contradiction» introduces some embarrassment, on can be use the
already the well-established notion of a «dialectical contradiction of the
first kind») [11]. Certainly understanding of typical levels of formation of
contradictions when solving OMT is an important information for the solver,
simplifying the formulation of contradictions, but terms describing levels in
OMT and the concept of «technical contradiction» are not identical, and
therefore interchangeable.

It is worth noting that when solving OMT, the method for quickly resolving
technical contradictions is actively used, which is especially important for
the solution of OMT characterized by a variety of contradictions. The method
was developed in the System Restriction Theory (SRT) for working with a
«thundercloud» (the method is described in detail by Darrell Mann already in
2000 and published in \emph{The TRIZ Journal} [12]), however, since in TRIZ it
is used a slightly different form of graphic representation of the
contradiction, the author adapted this tool, and as a result, the method of
express analysis of contradictions became much simpler and now requires much
less time than in the original version [12], which makes this tool extremely
practical [11].  However, the author decided not to include a description of
the modified contradiction analysis tool in the dissertation, as the author’s
innovation related to the transfer of SRT approaches for TC resolution in TRIZ
do not have sufficient novelty and in one form or another are used by many
TRIZ specialists [12], [40]. If desired, more details on the application of
this approach by examples of solving practical problems are given in
\textbf{appendix 3}.

\section{The practical relevance of the research}
\begin{itemize}
\item [1.] The proposed methodology of schematization in order to formalize
  OMT allows you to:
  \begin{itemize}
  \item Define the system contours without missing important details, and on
    the other hand, exclude «extra» elements of the system, taking into
    account the objectives of the task at the expense of system visualization
    and highlighting the position of the solver. The system operator is in the
    author’s opinion not an alternative to schematization, since this tool has
    no means to describe the relationship between elements of the system, it
    describes only its composition;
  \item Set a system of tasks to be further solved by TRIZ means, without
    omitting important aspects of the organisational-managerial task;
  \item Several times reduce the time for communication within the team during
    analysis of the inventive situation.
  \end{itemize}
\item [2.] The proposed methodology for identifying resources as factors
  affecting elements of the operational zone allows:
  \begin{itemize}
  \item Reduce communication time to identify the operational zone, previously
    very lengthy considerations had to be made in order to identify the OZ in
    an organisational-managerial task in the case the solver intended to apply
    ARIZ to resolve a contradiction;
  \item Identify the resources in the operational zone as significant factors
    affecting tool and product in the operational zone without searching for
    objects in the business system, thus dramatically increasing the speed of
    analysis and the quality of inventive solutions.
  \end{itemize}
\end{itemize}
All this makes the proposed methods suitable for practical use in consulting
projects.  Detailed application examples in consulting projects prove the
instrumental nature of the proposed methods (see appendix, as well as sources
[6], [11]).

\section{Key Points for the Defence}
\begin{itemize}
\item [1.] \textbf{Use of schematization to process invention situations in
  OMT and the relation between schematization and
  tools adopted in TRIZ}.
  \begin{itemize}
  \item The goals of applying schematization in solving business problems.
  \item The method of defining business tasks through schematization.
  \item Terminological apparatus of schematization.
  \item The scope of application of schematization and its application in
    conjunction with others TRIZ tools.
  \item Conclusions on the use of schematization.
  \end{itemize}
\item [2.] \textbf{Method of identification of the operational zone in OMT
  from the model of technical contradiction}.
  \begin{itemize}
  \item The objectives of the identification of the operational zone in
    business tasks.
  \item In what cases is it necessary to resort to the identification of the
    operational zone in business task.
  \item Difficulties with determining the operational zone in business tasks.
  \item The method for extracting the operational zone from the model of
    technical contradiction.
  \item Definition of tools and products and identification of resources as
    factors, affecting the tool and product in the operational zone.
  \item Conclusions on the application of the method for identifying the
    operative zone from the model of technical contradiction.
  \end{itemize}
\end{itemize}
\section{Personal contribution of the applicant:}
\begin{itemize}
\item[1.] The use of schematization according to G.P. Schedrovitsky for
  pretreatment of poorly explained inventive situations in OMT with the
  purpose of obtaining a system of partial tasks, which are then processed
  with the TRIZ arsenal.  Connection of schematization with TRIZ tools.
\item[2.] Application of the categories «Generalized Object» and «Filling» in
  order to obtain scalable solutions when using TRIZ tools in solving OMT. The
  terms «Generalized Object» and «Filling» are explained in detail below.
  Briefly: «Generalized Object» and «Filling» are subsystems of a system
  element, these concepts clarify the concept of «system element» and are of
  great practical importance when analysing OMT from the perspective of
  scaling solutions;
\item[3.] Development and testing of the identification of the operational
  zone in OMT.
\end{itemize}
\section{Work approbation}
\begin{itemize}
\item[1.] Scientific conference «TRIZ. The practice of applying methodological
  tools». Moscow, 2016;
\item[2.] TRIZ Training in full-time and distance format, trained more 300
  specialists. During the training, students solved problems from their
  practice under the author’s guidance and used these tools in their projects;
\item[3.] At the time of writing the dissertation, the author has completed
  more than 50 consulting projects using these tools;
\item[4.] The book \emph{TRIZ. Solution of business problems} /
  A. Kozhemyako. Synergy University, Moscow 2017. -- 288 pp., Ill. With
  answers to questions from readers who applied the recommendations in their
  projects.  In 2019 the 2nd edition, revised and supplemented, was issued.
\item[5.] Scientific conference “TRIZ-Summit“, Minsk, 2019.
\end{itemize}
\section{Publications on the topic of the dissertation}
\begin{itemize}
\item[1.] A. Kozhemyako. \emph{TRIZ. Solution of business problems}. 
  Synergy University, Moscow 2017. -- 288 p., Ill .;
\item[2.] A. Kozhemyako. Non-technical TRIZ: experience in solving
  OMT, limitations and tools. Materials of the
  VIII anniversary conference \emph{TRIZ. The practice of using methodological
    tools and their development}.
\item[3.] A. Kozhemyako. Ideas for the joint use of TRIZ and SMD for solving
  business problems. Published on the site \url{https://www.bmtriz.ru}.
\item[4.] A. Kozhemyako. Ideas for the joint application of TRIZ, SMD and TOS
  for solving business problems.  Part 2. Published on the site
  \url{https://www.bmtriz.ru}.
\item[5.] A. Kozhemyako. Some points about systems thinking of the head of a
  sales department.  We apply system analysis. Sales Management Magazine, 03
  (98), 2018.
\item[6.] A. Kozhemyako. Schematization of the inventive situation in
  OMT. Materials of the conference
  \emph{TRIZ-Summit}.
\item[7.] A. Kozhemyako. Specifics of the application of TRIZ in
  OMT. Materials of the conference
  \emph{TRIZ-Summit}.
\item[8.] Morphological analysis to solve business problems. Published in the
  journal \emph{Management today}.
\end{itemize}

\section{Structure and scope of work}
The work consists of an introduction, three main sections, a conclusion
section, and six appendices, including examples of practical application of
the proposed methods, set out on 83 pages; includes 28 figures, 10 tables, a
list of references with 46 titles, including books and publications by the
author on the topic of the dissertation.

\chapter{Research results}

\begin{center}\Large
  Schematization of inventive situations.
\end{center}

\section{Objectives of the research}
A solver using TRIZ to solve OMT is sharply faced with the question of a
detailed clarification of the inventive situation.  Practice shows that
usually the solver receives such tasks from stakeholders in an insufficiently
formalized form [11]. Simply put, we have to deal with “slogan” formulations
of the problem [40], albeit accompanied by «digitized» indicators:
\begin{itemize}
\item it is necessary to reduce the cost of our products by 10\% within 3
  months;
\item reduce labor expenses by 15\% within 6 months;
\item increase the influx of target leads [16] to 300 units per week until
  the end of 2018;
  
  Etc.
\end{itemize}
It should be noted that not only TRIZ experts are faced with the problem of
formalization of OMT. The members of the Moscow Methodological Circle (MMC)
under the guidance of G.P. Schedrovitsky were actively dealing with this
problem [10], and as a result of their activities, a schematization technique
for such problematic situations [41], which is able to qualitatively cope with
this problem, but does not have system tools for further OMT, which, in turn,
are provided by TRIZ.

\textbf{Objective of the research: to show the benefits of applying
  schematization in preliminary analysis of the inventive situation and
  develop a method to apply schematization in conjunction with TRIZ tools}.
Consider existing TRIZ approaches to formalisation of inventive situations and
show areas of similarity and areas of difference. Show which benefits offers
schematization compared to other tools of preliminary analysis of inventive
situations. Link schematization with TRIZ tools solving OMT.

\section{Minimum Viable Business System}
Since the dissertation is devoted to tools for working with OMT, which, as
indicated, can be set by stakeholders when trying to improve an organized
social system, it is necessary better to understand the functioning of such
systems. Consider typical elements of an organized social system on the
example of a business system.  We begin by defining the concept of the
viability of a business system developed by Valery Souchkov [2]:

\addpicture{Fig. 1 Minimum viable business system.}

As can be seen from the diagram in Fig. 1, for a business system to be
minimally viable, it must contain a supply engine, an added vakue creation
engine, and a delivery engine supplying value to the market. In addition, the
viability of the system is supported by business processes and management
functions (goal setting and control of their implementation), carriers of
these functions is the personnel of the company, in particular managers.
\textbf{In business systems, people are the most significant subsystems and
  supersystems}. The problen is that a person is a system with a high degree
of uncertainty (low probability of predictability of behavior), and therefore,
the same parameter value at the input to the system can give a large spread of
output values depending on the specific state of the system element, and it is
often possible to observe a result that is very far from forecast.

%% -------------------------------------
Since a sustainable development of a business system is a primary goal, the
functions of setting tasks and monitoring their implementation requires a
clear vision from the manager [13], such that in business systems arise many
inventive situations, the resolution of which can clarify the vision of the
manager. It’s known that with many inventive tasks, a manager is not coping
due to the lack of a standard solution. In modern management it is estimated
that more than 90\% of managerial decisions are standardized [10], [23]. but
remaining unsolved tasks may reduce the effectiveness of business systems,
remaining unresolved years, as practice shows. The author faced similar tasks.
in the field of document management, staff recruitment and training, marketing
and sales, in areas of CRM systems ... The situation is aggravated by the fact
that it is not always easy to find and adapt decisions from generally accepted
practices - organizational and managerial the task often needs to be closed
pretty quickly.

Until these “white spots” can be overcome and clarity appears design in the
mind of the manager, the function of setting tasks for execution and
controlling them Executions cannot be fully implemented. The business system
is located in constant movement, therefore such “white spots“ in the
activities of modern managers arise with enviable frequency, and time to
overcome them less and less is given away from year to year. Similar tasks
called in this paper organizational and management, have good potential for
application TRIZ tools and approaches.

Since the purpose of this work is to study the features application of TRIZ to
the class of tasks set in business systems should be given attention to the
definition of “organizational and managerial tasks.“

The author believes that in solving this class of problems it is worth
distinguishing between three forms of activity that form the basis of
management [41]:
\begin{itemize}
\item organization;
\item leadership;
\item management.
\end{itemize}
Organization is the process of forming supersystems and / or subsystems of
various level in business systems: association, organization, department,
workplace finally.  Organization is precisely the formation of a structure,
that is, elements and their interconnections.

Leadership (from words to lead, lead by hands) - this is the setting of tasks
performers and monitoring their implementation.

Management is a change in the activities of performers. That is, when the
structure organized, all tasks are set (including tasks for feedback), but we
are not satisfied with the performance of performers, we are trying to change
them activity in the direction of improvement, that is, we begin to manage to
manage them activity.

All three of these activities form what can ultimately be called “Activities
in business systems.“ Therefore, I propose to call such a class of tasks
“Organizational and managerial“. \emph{Of course, from the standpoint of using
  TRIZ, we only interested in inventive situations that arise when solving
  organizational and management tasks.}  Further, at used term “Organizational
and managerial tasks“, the requirement of inventive situations will be
implied.

However, before using TRIZ tools, an inventive situation should be as
previously analyzed. Similar analysis in organizational management tasks has
differences from pre-processing tasks, delivered in technical systems,
primarily because an element of the system is a man - \emph{a complexly
  organized element that has its own goals}. Moreover, the processes going on
in such systems are often quite confused, contain a lot of conventions and
require a special approach to identification and description.

\Missing

\section[The SMART method]{Application of the SMART method to formalize
  organizational management tasks}

SMART [14] (the extended formula “SMARTER” is also known) is a mnemonic
abbreviation for the principle of setting goals. This model is today the main
standard when setting tasks to subordinates, we can say that this The main
international standard in this field. According to SMART, the task should be
specific, measurable, attainable, meaningful (relevant), correlated with a
specific period (time-bounded). Perhaps this is the most world-famous way of
formalizing organizational and managerial tasks.  Examples of tasks set in
accordance with the SMART model:
\begin{itemize}\it
\item [1)] 10\% of the employees are consistently late for work for 5-10
  minutes, another 3\% of the team are late for more than 15 minutes. “Stable
  Delays ”are considered delays more than 5 times a month. Required to lower
  maximum stable lateness of employees up to 3, lateness of 15 minutes and
  above completely exclude.
\item[2)] It is required to increase sales of the company's product (dietary
  supplements from natural raw materials) 20\% until December 31, 2017. Growth
  through expansion of territories unacceptable, an increase should occur in
  existing territories.  Allowable increase in marketing budget -- 10\%.
\end{itemize}

This method of formalizing organizational-managerial tasks is much better than
the absence of any scheme, however, it is easy to notice that such a statement
of the problem is good for execution, in case the performers have all
necessary resources to complete the given task and if the performers know ways
to accomplish it (they know how to apply these resources to achieve the given
goal). However, such a model is completely insufficient for formalization of
the task in order to find the most effective conceptual solutions if there is
no standard solution to the problem. Consequently, the SMART model does not
open up the possibility of applying TRIZ methods, and therefore, it cannot be
considered as a method of preliminary formalization of the inventive
situation.  In fact, the use of the SMART model does not allow the solver to
switch from the inventive situation to a task (more precisely, to a system of
tasks) suitable for further processing using TRIZ tools. \textbf{In the
  opinion of the author, what makes the use of TRIZ in the first place
  difficult for the solution of organizational-managerial tasks is the lack of
  reliable and relatively simple tools, allowing to move from an inventive
  situation to a system of tasks in the field of business systems}.

The SMART model, though, makes important refinements to the understanding of
the inventive situation, does not solve the problem of preparing an inventive
situation in organizational-managerial tasks for further processing by TRIZ
methods. Currently, the SMART method is a generally accepted method of goal
setting all over the world, mastery of the SMART method is considered as one
of the most important competencies of a modern manager, as almost
indispensable management basics. However, to formalize the inventive situation
in organizational-managerial tasks, this generally accepted method does not
give the required result -- it does not help to isolate a system of tasks from
an inventive situation, which can subsequently be processed with TRIZ tools.

There is also another opinion. For example, a number of TRIZ specialists are
convinced that if there are no standard methods or not enough resources to
solve the posed problem, then any TRIZ specialist will formulate a
contradiction using standard formulas.  The author strongly disagrees with
this argument, since before a contradiction can be formulated according to
standard formulas, for a successful solution of the
organizational-managemerial task it is required to describe a model of a
functioning business system (MFS) [15] and set up a system of particular tasks
to its elements (subsystems and supersystems).  Otherwise, we get a
contradiction that is of extremely general nature, and therefore, there is a
high risk of getting trivial decisions as output. The author recommends to
formulate contradictions to each of the particular tasks posed after the
analysis of the MFS. Otherwise, the solver risks that the obtained solutions
are to "narrow", and as a result, a situation is possible when the solver will
not finally reach the goal [10], [11].

It is worth noting that the tree of contradictions also cannot be used on this
stage, since at the stage of formulating the organizational-managerial tasks
the set of contradictions is not yet visible -- see the example at the end of
the section (the problem of implementation a new sales system in the company).

\textbf{Conclusion: the SMART method helps to specify the task to be
  performed, but cannot be used as a tool for initial processing of an
  inventive situation.}

\section{Description of business processes} 
Description of business processes [34] is another method that is actively used
in an consulting environment for preliminary processing of an inventive
situation.  It is usually resorted to the description of business processes if
the solver believes that an organized social system does not function
rationally, which means, unlike the previous method of tasks setting, has a
narrower scope. As a rule, this tool is used to analyze the inventive
situation in order to increase the efficiency of standard, well-established
processes in organized social systems.

The technology for describing business processes in the worldwide practice is
not bad standardized and described by generally accepted \emph{notations},
e.g. IDEF0, BPMN and others [34]. In TRIZ, a similar method is also actively
used in the form of \emph{flow analysis}. And although the descriptions of
business processes and the flow analysis are not quite the same thing [11]
(flow analysis describes the transfer of matter, energy or information, has
sources, consumers and a flow path, whereas the description of a business
process focuses on a list of operations performed by the process owner and
members), these methods have a lot in common.

An example of a business process description:

\addpicture{Fig. 2. An example of a description of a business process.}

In fig. 2 an example of a description of a business process of decision-making
by a bank on a client’s loan application is shown using BPMN notation. As can
be seen from fig. 2, a similar method is well suited for preliminary analytics
of only one class of organizational-managerial tasks: optimization of
regularly recurring operations in various parts of the organized social
system.  But if the task is connected not only with established processes,
what to do in this case? After all, business processes is only one aspect of
the development of organized social systems, there are also other levels --
making strategic decisions, developing of new directions of activities,
relationships of process participants and other tasks.

\textbf{Conclusion: the tool is very useful in solving a particular class of
  organizational-managerial tasks but does not possess the required
  versatility and is suitable exclusively for optimization of regularly
  ongoing processes, and therefore, as a single tool for preliminary
  processing of an inventive situation for solving organizational-managerial
  tasks cannot be recommended.}

\section{System operator}
Let's consider the "classic" TRIZ tools that claim to be used for primary
analysis of an inventive situation in organizational-managerial tasks.

There are known cases, where TRIZ practitioners use the system operator for
this purpose [25].  The author also applies the system operator to solve
problems in business systems. One example of the application of the system
operator to solve a organizational-managerial problem is shown in Appendix 2.
With the help of the system operator it is possible to see the dynamics of
development of a system and predict its structure in the future, and also to
consider the structure of the functioning system -- to describe supersystems
and subsystems of the given system (... sub-subsystems, subsystems, system,
supersystems, super-supersystems ...).

Indeed, the system operator has a significant drawback -- it is looking at the
development of the system through the composition of its elements, but does
not take into account the layers of the system and the relations between its
elements. This, of course, cardinally limits the capabilities of this tool for
a descriptions of organized social systems. In addition, the structure of a
functioning system is taken only element-wise, in contrast to the
schematization that looks through the layers, groups, relations, processes and
functions. And if the supersystems are on completely different management
layers, what to do in this case? How to set specific tasks for effectiveness
of management of system elements?  The system operator does not reflect
management phenomena (fig. 3).

The advantages of the system operator include its versatility and adjustable
level of detail of system elements, as well as the ability to trace the
evolution of the system, its subsystems and supersystems, what determines the
prognostic value of the tool.

The downside is the impossibility of an adjustable decomposition of elements
of the supersystem, the impossibility of drawing relationships between
elements of the system and the supersystem (the structure is taken as if in
isolation, without indicating the connections between the elements of the
system), as well as the impossibility of demonstrating in the model the
schemes of control, which is a critically nessecary moment when studying the
inventive situation with the aim solving organizational-managerial problems.

\addpicture{Fig. 3. The structure of the system operator.}

Nevertheless, the author considers the system operator a perfectly applicable
tool to solve organizational-managerial tasks, but not for the purpose of
preliminary analysis of the inventive situation, but \textbf{in order to study
  the system in the context of its evolution} [11], which is important for a
number of \textbf{strategic} tasks, where situational interactions of elements
of the system should not be taken into account.

\textbf{In contrast to the specific inventive situation posed by the customer
  in an organized social system, tasks that should be studied using the system
  operator, are usually general and strategic in nature (see Appendix 2).}

\section{Structural Analysis and Functional Modeling}
These types of analysis applied in a FVA [19], most accurately describe the
inventive situation when solving organizational-managerial problems, since
they show the composition of the system, elements of the supersystem, the
relationship between the elements of the system, functions, and modern
versions of FVA allow you extract groups of elements and even take into
account the processes within the system [17]. An example of a functional
scheme is shown in fig. 4:

\addpicture{Fig. 4. Simplified functional diagram of the construction helmet
  (in red harmful functions are shown).}

It is this approach that makes it possible to conduct a preliminary analysis
of the inventive situation with the aim of setting particular tasks, which
subsequently can be solved using TRIZ tools. However, functional modelling has
all the same key flaw defect as any method developed for analysis of technical
systems: the method does not take into account the management context between
elements of the system. Moreover, a part of the elements of a working system
are good modifiable objects of the material world, and some are subjects, that
is, people, performing defined functions, but with their own goals,
dynamically changing emotions, patterns of behavior, etc., that is, objects
with behavior which is hard to predict.  The description of such objects
requires a special language in order to develop solutions that can later be
replicated.

When solving organizational-managerial tasks, it is impossible to ignore such
phenomena, since people in organized social systems are essential (and often
the most important) elements of the system.

\textbf{Conclusion: The author believes that functional analysis has great
  potential for solving organizational-managerial tasks -- in the author’s
  book [11] an example of a solution of a similar problem is explained in
  detail --, but this tool is more likely applicable to a certain class of
  organizational-managerial tasks related to the optimization of organized
  social systems and their subsystems [39] and is not quite convenient for
  preliminary analysis of inventive situations arising in such systems, since
  it does not take into account the specifics of a description of people as
  subsystems of a social system, nor does it take into account the dynamically
  changing driving influence of the elements of the system to each other in
  the context of the inventive situation under consideraton (i.e., it does not
  take into account the peculiarities of "soft" systems).}

\section[Schematization at MMK]{Schematization developed at the Moscow
  Methodological a circle (MMK) under the leadership of G.P. Shchedrovitsky
  and its application for preliminary inventive analysis of inventive
  situations}

This method was formed in the Moscow Methodological Circle under the
leadership of G.P. Schedrovitsky [10] in order to organize the thought
activity (\foreignlanguage{russian}{мыследеятельность}) of a group of
professionals who discuss problematic situations in the field of organization
and management. The meaning of this tool was to gradually to put on the "Map"
the elements of the system that are significant for the task being solved and
relations between them, that is, get a tool for the manager, similar to the
main tools of military strategists -- a map, the details of which are
appearing gradually, in the course of the thought activity of the officers
planning the military operation. The author considers such consideratons to be
an excellent alternative to the methods described above from the point of view
of formalizing the inventive situation, but getting a synthesis of
schematization and TRIZ in its purest form was not so easy.

Essentially, schematization is a visualization of an inventive situation
worked out in accordance with the \textbf{categories of the system} and
understanding the functionality of the system [10]. It is this tool that the
author considers optimal for a primary analysis of an inventive situation when
solving organizational-managerial tasks. It should be noted that the author
had to restore the following categories of schematization on his own, based on
a detailed study of the works of G.P. Shchedrovitsky, as currently
schematization is greatly simplified and many SMD methodologists are building
schemes without taking into account the categories of systems described below.
Some of them even generally simplified schematization to scribing [35].

Therefore, the author had to restore the principles of schematization from the
work of G.P. Schedrovitsky, and then develop on his own methods for setting
tasks based on the results of a schematization of the inventive situation in
organizational-managerial tasks.

Based on the analysis of the works of G.P. Schedrovitsky the following
categories of schematization can distinguished:
\begin{itemize}[noitemsep]
\item[1.] MWS (model of a working system -- frame);
\item[2.] Layers;
\item[3.] Groups;
\item[4.] Relations;
\item[5.] Functions;
\item[6.] Processes;
\item[7.] Generalized Objects;
\item[8.] Fillings.
\end{itemize}

A caveat is required: the category \textbf{"system frame"} in the
terminological apparatus of G.P. Shchedrovitsky can be replaced by the
category \textbf{"model of a working system"} developed by Nikolai Shpakovsky
[15].

Since the author spent a significant part of his work in departments of
marketing and sales of international corporations and is the author of a
three-volume book "The era of smart sales ...", where he summarized the
experience gained and described his methodological basics in this area, and
currently leads a significant part of projects in the field of marketing and
sales in the B2B market [27], many examples of organizational-managerial tasks
the author takes precisely from these areas. Of course, this does not mean
that organizational-managerial tasks are restricted to the field of marketing
and sales. The converse is true: problems from marketing and sales are a
special kind of organizational-managerial tasks.

\textbf{The model of a working system (MWS) is determined in two stages:}

\subsubsection*{1. Isolation of the kernel.}
The inventive situation presented by the stakeholder indicates that there is
resistance of managers who do not accept the new sales system, implemented at
the enterprise. It’s known that before that, managers didn’t work chaotic,
their work was determined by a different sales technology supported at the
enterprise, to which they adopted. After conducting these simple arguments, we
identified the structural core, which can be represented as part of a scheme
(Fig. 5):

\addpicture{Fig. 5. The core of the problem, presented in the form of a
  scheme.}

In fig. 5 we see the main stakeholder in the task -- the “resisting” managers
themselves, what they resist is the complex sales system being implemented as
solution and the existing sales system, to which they adopted during time.
The two sales systems conflict with each other (this is a logical conflict),
in this case -- differ in content and requirements, as shown by a dashed
arrow, in addition, managers also conflict with the newly introduced sales
system, as it requires them to restructure their work, which causes
dissatisfaction with the salespeople.  The new system is promoted by the head
of the sales department, so between the head of the sales department and the
managers, we also observe a conflict of interest. Further on the diagram we
show also this connection. \textbf{The core of the scheme always represents
  the minimum scheme of conflict given by the inventive situation.}

\subsubsection*{2. Definition of connections from the core of the task to the
  supersystem and the final definition of the MWS (frame).}

To whom the managers express their dissatisfaction, who perceives their
resistance? In the course of communication with the client, it turns out that
first of all -- to the head of the sales department (in the diagram in Fig. 6
he is designated as ROP). And maybe also to customers, which is inacceptable
to the company. In any case, customers should be introduced in the scheme,
since they are ultimately affected by the internal changes in the sales
department.

Why was the sales system created? To increase conversion rates, efficiency of
work with the client, and as a result -- to sign more transactions on large
amounts without hiring additional sales staff. What else “hurts” the sales
system? The CRM system (Customer Relationship Management [16]) implemented in
the company. The new sales system requires major changes to the work with the
main software in the sales department -- the CRM system. Therefore, the scheme
depicts a conflict of the new sales system with the existing CRM system, to
which specialists of the sales department adopted. The problem is not that the
installed system does not have certain options, this is a secondary problem, a
smaller one than the one we trying to formalize. The problem is that the
“relationship“ of the sellers changes concerning the new requirements of the
CRM system.

In addition, the work of the sales department relates to activities of other
company services -- production, warehouse, logistics, accounting ... This is
important, but at the stage of setting the task it is too early to engage in a
deeper detailing of these processes; therefore, we denote thee “touch points“
of the sales department with other company services as “cross-business
processes“ (\foreignlanguage{russian}{сквозные бизнес-процессы}), which can
also be transformed under the influence of the new sales model. So we got a
formed \textbf{model of a working system} consisting of: sales staff, head of
sales, the existing sales system, the new sales system that is replacing it,
the target client groups, CRM systems and cross-business processes:

\addpicture{Fig. 6. A model of a working system, presented in the form of
  a diagram. Abbr.: ROP -- Head of Sales Department.}

\subsubsection*{The concept of a layer in a scheme.}

A layer in a scheme symbolizes that an element of the system element or a
group of elements, placed on a higher layer controls an element or group of
elements, located on the lower layers in the context of the task (i.e. a layer
is a graphical representation of the fact of controlling the activities of one
element in relation to another).  And although the system is in most cases
defined as a set of interconnected elements, it is obvious that the elements
can have different hierarchical positions in the system to be studied, that
is, there is a hierarchy of control between the elements from the viewpoint of
a \emph{defined action} considered in the context of the problem to be solved.
The concept of a layer allows to consider the elements of the system, taking
into account their dynamically changing hierarchy, where the hierarchy is
determined on the basis of understanding which element is governing, and which
element carries out the “orders” of the more “senior” elements of the system,
and not in general, but only in the context of this specific activity related
to the considered organisational-managerial task.

It is worth noting that the concept of a layer does not replace the concepts
of a sub-sub-system, subsystem, system, supersystem, etc., so layers are not
at all the same as we depict in the system operator.  \emph{If subsystems,
  supersystems, etc. are fairly rigidly fixed, the layers are constantly
  changing in the context of the activities of the elements depicted in the
  scheme.}

That is, as an important concept for the analysis of business systems, the
author suggests to use schematization with the allocation of “layers“,
explaining this with the basic properties of organizational-managerial systems
built on a \emph{dynamically changing hierarchy of governance in the context
  of the studied activity of elements of the system}.

It is important to note that subsystems are subsets of the system, the system
itself is a subset of the supersystem. As you know, subsets have the property
of similarity to the sets of which they are a part. In other words, a
subsystem is a set of elements that make up the system, broken down by some
criteria into subsets [45].

For example, a subset of the decision center [16] in an organization is a
seller of the company -- a potential supplier of the product, since the seller
is a subset from a decision-making perspective (he makes a decision based on
information transmitted by the seller, but not only. In the process of
decision the decision center is influenced also by other subsets). But which
element rules which activities of other elements? This is a big question. The
answer can be given only in the context of the task.  Suppose, in the process
of implementing his functions, the manager conducts research on customer
needs, on the basis of which he subsequently prepares a commercial proposal.
Which element of activity controls which? \emph{Of course, the client controls
  the further actions of the seller based on the information provided about
  his needs}. Moreover, if we consider a different situation, where the needs
of the customer is a product, then the layers will change again, that is, the
seller through questions will manage the activities of the decision center. At
the same time, the subsystem and supersystem remained at their places, only
layers change!  Another example. The seller forms a picture of the world of
the customer, creating value in his offer. Which element is now driving the
activity of which? Now the seller drives the activities of the client from a
position of acceptance of the solutions.  the subsystem also governs
activities of the supersystem in this context.  Please note: subsystem and
supersystem did not change places, but layers -- changed!

This is the most important understanding from the point of view of finding a
solution to a problem in an organizational-managerial context, therefore, the
concept of a layer is not a substitute for the concepts of supersystem and
subsystem.

\textbf{Rule for representing layers in a diagram:} If any elements of the
system are depicted in the diagram higher than the others, then the solver
ranks them in the upper layer, i.e. in the context of this inventive situation
considers them as controlling elements. The main thing is to always remember
the rule: layers may vary depending on the contemplated inventive situation.

In fig. 6 layers are clearly visible in the context of the task: Cross-cutting
business processes have largely determined the configuration of an existing
CRM system. The CRM system, which was configured based on the requirements of
the existing sales system, at the moment is a deterrent to the implementation
of the new sales system, as it is reliably connected with end-to-end business
processes in company, which implies the use of its reports by a number of
related units. It does not support the functions required for the new sales
system, but the transition to another CRM system, although it will provide the
necessary functions for the introduction of the new sales system, is highly
likely to give rise to similar conflicts at the border of the sales department
with other divisions of the company. From here it’s easy to conclude that the
sales staff are on the next layer -- they are “controlled” by the existing
business processes in the company and the “accustomed” to it CRM system that
supports the current sales system, which, in its turn, does not suit the head
of the sales department and the best employees, since it does not support
effective work with clients in recent conditions on a highly competitive
market.

It is not hard to notice that the schematic representation of the MWS taking
into account the layers makes a lot of important refinements in understanding
the inventive situation. The analysis of the schemes shows that there are at
least several contradictions hidden in the problem (\emph{of course, such a
  representation of an inventive situation requires the solver to develop
  schematization skills, so in the process of constructing a scheme a
  bidirectional thought process is going on: the scheme is built in the
  process of analyzing the layers, but on the other hand, layers “appear” in
  the process of constructing the scheme. Therefore, in practice, the scheme
  of the MWS is usually redrawn several times, until full clarity is
  established in the description of the inventive situation between the
  problem owner and the solver}).

To this point, the author received the remark that since a scheme is rarely
created in one passage, this approach strongly resembles the method of trial
and error. The author does not consider the method of trial and error and
iterative approach as identical concepts, as during several iterations the
scheme is refined, and is not newly created in a random manner.  The solver
also clarifies the definitions when formulating a technical contradition or an
IFR.  It is far from always possible to get the final wording in the first
attempt. The same happens when applying schematization.

\subsubsection*{Groups of elements of the MWS (groups)}

A group is a combination of elements of a system to fulfill a specific
function. If the group is taken as a separate system, then we can talk about
the MUF (main useful function) of the group. \emph{The author believes that
  instead of the term “group”, introduced by G.P. Shchedrovitsky, it’s
  completely appropriate to use the TRIZ terms -- system, subsystem,
  sub-subsystem ...}. Usually, if solving a problem requires more general
consideration of the elements of the system, or more detailed onew (in the
course of work on the task such transitions are assumed and become apparent in
the communication process between the solver and the problem provider), it is
convenient to give on the scheme the required detailing of the elements and
their relationships for deeper analysis, but in the moment when more
superficial analysis is required, combine such elements (sub-subsystems) into
groups (subsystems) and emphasize the general function of the group. In
practical applications of schematization for the analysis of an inventive
situation, such a division may be very important, for example, for the task of
increasing the efficiency of a department (in this case, the sales department)
the following scheme was developed for the analysis of an inventive situation:

\addpicture{Fig. 7. Scheme of the sales department with the allocation of
  layers and groups. Abbr .: KAM -- key account manager (key account manager),
  lead -- potential client who in one way or another responded to marketing
  communication.}

The groups in the diagram are:
\begin{itemize}[noitemsep]
\item marketing department; 
\item sales department; 
\item customers.
\end{itemize}
For example, why did the company create a marketing department? To prepare and
conduct marketing promotions? To study the market? For advertising campaigns?
For marketing analytics? To carry out public relations (PR)? Marketing should
carry out all these functions, but they are not the main ones. They are
auxiliary ones. All these functions are needed in order to promote your
product on the market. So the MUF of the group of marketing (department) is
the promotion of the company's product on the market by influencing target
client groups (again groups, only now they are combined by functional features
of consumption). The MUF of the group (department) sales is the same, but with
some differences -- promoting a company's product on the market by personal
impact on the client. The difference between them is essentially in the way
how they impact on the target audience.

Groups can be not only departments and divisions. It can be project groups or,
for example, groups of employees united by some social characteristic or, say,
the time of work in the company. It all depends on the condition of the task.
Only one very important rule should be remembered. Any group has its own MUF
that is significant for the problem to be solved. Arbitrary division of
elements into groups within the frame of a working system is unacceptable, as
this complicates the task, burdens it with unnecessary information.  The
question arises: is it worth to introduce this concept or to stick to the
notion of a subsystem, traditionally used in TRIZ?  Is it possible to use the
concept of "aggregation of elements" instead of "groups" also known in TRIZ?
At this stage the author proposes to leave this debatable question open.

\paragraph{Functions.}
Functional language is perfectly developed in TRIZ and, perhaps, is the main
language used in the analysis and description of a system: first of all, with
the use of FVA. Therefore in this work, a detailed description of the concept
of a “function” is not required.

\paragraph{Processes and communications.}
These are the most important categories of schematization.

A \emph{process} is a course, the development of a phenomenon in time. In this
definition the key word is “phenomenon.“

\emph{Relationships} are a designation on the scheme showing that in the
context of the task it is important for us that element A affects element B,
but what processes are going on is not important for us.

Since the scheme in Fig. 7 is functional, processes are not indicated on it,
but functions are shown as arrows. But if when creating a functional scheme it
is not managed to get a working model, that is, when analyzing the scheme,
disruptions and promising trajectories of the problem solution became not
visible, then you should embark into processes.

Interestingly, in TRIZ this problem of the process level has already been
discussed, for example, in relation to \emph{advanced functional analysis}
developed by Naum and Oleg Feigenson [17] when, as a result of the FVA not
one, but several functional models are built, each for a specific state of the
system due to the processes occurring in the system.

Here we can clearly see the fact that in most problems the solver may remain
at the functional level (then on the scheme only functions and relations are
present), but there are situations when the system has to be considered in
various states depending on the processes occurring in it. Similarly, when
conducting schematization of an inventive situation, there are situations when
the scheme requires to indicate also the processes going on between the
elements of the system. Understanding the difference between processes and
relationships ist very important for a qualitative schematization of the
inventive situation when solving organizational-managerial problems.

You can see an example of such a scheme in the diagram below:

\addpicture{Fig. 8. The scheme of analysis of the inventive situation of the
  problem of increasing the effectiveness of the mentoring process.}

In the diagram shown in fig. 8, the modelling of the mentoring process is
visible, that is organized in the sales department. The processes indicated by
arrows are considered in detail that go on when planning a strategy for
concluding a deal, the student’s work in "field conditions" and the processes
that occur during the organization of feedback from the student to the mentor.
In addition, the task involves in-depth study of the processes taking place in
the period of supervision by the mentor of the development of the student's
skill in using the tool, which the latter should develop on the instructions
of the mentor.

\textbf{The author does not have data on the existence of a methodology to
  clearly determine during schematization whether a transition to a process
  level is required, or is it sufficient to remain on the level of a scheme
  containing the necessary elements, their functions and relationships.} The
practical recommendation is as follows: first we build a scheme describing the
inventive situation at the level of relationships and functions, and then, if
the data is clearly not enough for further processing by TRIZ tools, we begin
to delve into the consideration of processes between system elements, for
which flow analysis or analysis of business processes with using standardized
notations is applied [34].

Introducing processes in the scheme leads to inevitable complications of the
scheme, describing the inventive situation, therefore, when conducting
schematization one should be guided by the principle of appropriateness and
introduce processes only where is a need for their detailed study (see the
case of the introduction of a new sales systems below, appendix 5).

\paragraph{General objects and filling.}

Let's move on to the concept of “general object” and “filling”. In the opinion
of the author, from the categories developed by G.P. Shchedrovitsky, these
concepts are one of the key in terms of applying system analysis to inventive
situations arising in the field of organization and management. These
categories applied to schematization are responsible for scaling the received
soultion.  \textbf{Scaling is the most important characteristics of the
  obtained solution in the organizational-managerial sphere.}

When conducting schematization from the position of the categories “general
object” and “filling”, not only sharply increases the likelihood of a
scalable solution, but its stability increases, that is, the ability to
withstand environmental disturbances.

In fact, at the stage of schematizing the inventive situation, the foundation is laid
qualities of the future decision.
So, a place is a vacant unit, which is also an element.
system. For example, a driver’s seat in a car. It is very different from
other places are equipped in a completely different way. The location is in conjunction with
other elements of the system - controls, instruments, expensive .
Places have properties that specify the requirements for their content .
For example, the place of the head, the place of the teacher, the place of the milling machine operator in the workshop.
Places give us tight connections in the process, however, so that the system
functioned, places must be filled. Who should be a milling machine
supervisor, driver - for example, a computer program or a person (see. Fig. 1
- it is indicated that a person is a subsystem or supersystem organized
social system (for example, a business system), although if you look in more detail,
then the subsystem or super-system is not a person as such, but a place + material).
Material is a system filling a place in accordance with its requirements, somehow
a person with relevant competencies or a computer program,
having certain characteristics corresponding to the requirements of the places.
In technology G.P. Shchedrovitsky requirements places are called properties-
functions , and material properties — attributive properties . So in organizational

management tasks there is a special class of tasks - coordination tasks
requirements of places (properties-functions) and material properties (attributive properties).
When solving organizational and managerial tasks, the solver must understand
on which layer should he stop. If the problem is solved in the space of the organization or
its units (department, department, site), it is important to try to find a solution on
level of places, which will determine its further scaling. If the task is
to coordinate the material and location, the solver must warn the customer that
that most likely the solution found will be special for this case and
scaling of the resulting solution will be fraught with certain difficulties
(if at all possible). The exception is the tasks associated with the search
the principle of such coordination (for example, such problems arise in the field of selection
staff).
Places can be represented as subjects (this is a place that takes
people, and objects (computer program, any document, robot, etc.).
The same can be said about the material (Fig. 9).
Some notation used in the diagrams :
Fig. 9. Some designations used in the scheme of the inventive situation.

Subjects and objects can be represented on the diagram both in places and in material.
- depending on the task. It is preferable to represent objects and subjects as
places, because in this case we get a scalable solution.
Setting objectives based on the results of schematization
Two options for further analysis after
schematization:
Option 1. Work with existing gaps. We find gaps , i.e.
discrepancies between the “how it should be” and what is currently depicted on
scheme. Particular tasks can be set for gaps ( see example in the appendix)
1 ).
Next, we highlight a list of unwanted effects (NE). Work with the
undesirable effects in TRIZ well developed, most often for this purpose
a causal analysis is applied. Therefore, according to the results of schematization
Inventive situation, we can advance in a variety of ways:
We get a list of NEs and conduct a causal analysis. This way
applicable if all NEs in the diagram are obvious;
We identify the gaps in the diagram and set the task to eliminate them. To received
tasks, you can apply the primary processing mechanisms of the task, long and successfully
used in TRIZ - stream analysis, functional analysis, causal
investigative analysis, benchmarking ... [18]. There are many variations, we must move from
context. No one forbids applying schematization to clarify in detail
gap structure, if the obtained task is a new inventive situation with
many unknowns [11];
Highlight technical inconsistencies and continue to work with them (technical
contradiction: a situation that arises when trying to solve an inventive problem
by improving a specific feature (parameter) of the system, which leads to
unacceptable degradation of another attribute (parameter) of the same system [42] ) . In that
in the event decisive TRIZ mechanisms come into operation.
Option 2. Choosing a promising roadmap and setting private
tasks. This is the path we took, solving the problem of a multiple increase in sales
construction equipment in the channel “road construction“ (Fig. 10). The problem was
that over the past three years, the company's product margin has been halved, and
sales are steadily falling. Marginality is known to add up to structure
costs and market value. We decided to start from the second and analyze the structure
construction equipment market (Fig. 10 gives an example of a rough analysis of the situation, but in order
in order to understand that the current business strategy of the company has been chosen incorrectly,
sketching without fine detail turned out to be quite enough):

Fig. 10. Preliminary “rough” design scheme of the construction machinery market
in the southern Urals.
To come to the choice of a promising road map, we plotted
budget allocation for road construction and sorted out the hierarchy of this
distribution - so the layers in this diagram were grouped by region
(groups). Further, understanding the price segmentation of construction equipment, it turned out
it is easy to compare these two parameters with the practice of the tasker, and then note two
points on the diagram - layers the company is working with now and as much as possible
a layer close to the consumer, capable of acquiring the technology of this segment
based on the data plotted on the scheme and its typical needs. So decided
promising roadmap, which was significantly different from the existing
company’s market strategy. To implement the resulting roadmap
it was required to pose a number of tasks and resolve the contradictions that arose.
At first glance, it may seem that this scenario is similar to the construction
functional model during the PSA, as the scheme and functional model
describes the structure of the system. However, the schema refines the layers, showing the hierarchy
control allows you to simulate the depth of interaction between elements
system, by introducing a chain of concepts of communication-function-process, reveals
the composition of the subsystems, if necessary from the position of the problem being solved (groups or, as
it is customary to call groups in TRIZ - aggregation of elements ), including the ability to
selection of groups passing through layers “diagonally”, for example, if by condition
tasks there is a need to analyze the work of project teams taking into account
employee performing two or more roles (for example, a member of the project team -
financial officer), when conducting schematization, the solver has
the ability to distinguish between material properties and site requirements, which is critical
important from the point of view of scalability of the resulting solution (another example
inventive situation is depicted in the diagram in Fig. eleven).

Fig. 11. The scheme of selection of employees according to the competency model.
In fig. 11 shows a diagram compiled by the author to describe
inventive situation that arose during the selection of employees for certain
positions taking into account changes in the requirements of places during the evolution of the organization
according to the model of I. Adizes (Fig. 11 shows the so-called PAEI code (P -
production of results, A - administration, E - entrepreneurial function and I -
integration. PAE is more about places, I is more about material). The PAEI code is shown in fig.
11 as a function of the requirements of places and properties of the material, also visible in the diagram
significant roles of process participants located on three layers and elements
recruitment systems related to places (competency model) and material
(competency assessment and personality type and ability assessment).
Therefore, despite a certain similarity with the functional model, the circuit
a slightly different tool. Its main purpose is to isolate the system of tasks from
primary inventive situation . The author suggests using schematization
to formalize the inventive situation when solving organizational
management tasks. Schematization should be applied immediately after clarification.
problems and goals of the solver before using the usual TRIZ tools.
The fundamental difference between schematization and functional model
Select layers . The functional model used in the FSA does not imply
building a hierarchical scheme with the allocation of layers, where the subject of management is
parent layer in comparison with the control object. Similar hierarchy of elements in
the scheme is very important for the analysis of the inventive situation in the organizational
management tasks (the very concept of organizational and management tasks requires
representations of system elements depending on the control hierarchy in terms of
set task).
Representation of a system element as a Place or Material, clear
the realization whether we are solving the problem at the level of the place or at the level of the material . Author
pointed out above that business systems are soft systems, as they include

human as the main subsystems. It’s important to separate, we look at a person as
on a subject endowed with variable properties or as a place with
specific requirements. The author indicated above that in the process of analysis
inventive situation, the solver must ensure that the problem is solved on
level of places, because otherwise there are problems with scaling
decisions received.
The author emphasizes once again that a simple tool transfer is not possible
from technical to business systems. To handle an inventive situation
business systems require specific tools that can prepare
task to use the “standard“ TRIZ tools.
The algorithm for working with the circuit
The algorithm for working with the circuit is as follows:
\item based on the situational analysis that we carry out on the scheme,
the most acceptable way of conducting further
transformations, which is defined as a priority (road
map of future traffic).
\item as soon as the roadmap is selected, we immediately attach it to the existing one
system, we see the secondary tasks in the form of directions or NE, which
will appear in the system when implementing the selected roadmap.
\item If necessary, analyze selected tasks with tools
primary processing tasks adopted in TRIZ.
\item If necessary, we form technical contradictions. Next for
permissions
selected
contradictions
apply
famous
TRIZ tools.
It should be noted that many companies, starting to realize many NEs at
the implementation of the new roadmap, they see before themselves not a set of separate tasks,
to be resolved, but an insurmountable wall and retreat, preferring “not
rock the boat. “ Applying schematization, isolating the system of tasks from
inventive situation with the subsequent use of TRIZ tools for
obtaining solutions, we have reliable tools for solving
organizational and managerial tasks of increased complexity. The author believes that
application of schematization lays a solid foundation for successful
the use of the TRIZ arsenal in solving organizational and managerial problems.

An example of the use of schematization for setting objectives for
organizational and managerial task (the case is described in detail in
Appendix 5):
A system consisting of:
sales department of an industrial enterprise,
manufacturing tooling from heat-resistant steels, presented
head of sales, sales staff and the current system
sales (Fig. 12).
The essence of the problem: the head of sales (ROP) implements a new sales system,
having advantages over the previous one in terms of depth of study
customers and, as a result, allowing to increase the average amount of the contract and
conversion, however, managers resist and are in no hurry to leave the “beaten”
rails. “
Required: to make managers use only the tools of the new system
sales in their activities (as the task sounded in the original formulation, that in
the process of analyzing the circuit turned out to be not quite the correct goal setting - see table).
Below, we compose a model of a functioning system, presented in the form of a diagram.
The process of constructing a circuit for this task is described above (see explanations for Fig. 6):
Fig. 12. The scheme of the inventive situation in the problem of changing the sales system.

The tasks set according to the scheme (Fig. 12) using the categories of schematization:
No.
Object of analysis in
IFS
Tasks
No.
Task
one
interaction
the system
(dotted line) with
elements
supersystems
1.1.
CRM system. The conflict arose largely due to the fact that the existing CRM-system is not
adapted to the requirements of the new sales system, which creates significant
inconvenience → make the CRM system meet the requirements of the new
sales system and supported her .
1.2
end-to-end business processes. The new sales system is changing end-to-end business processes,
this is especially true in collaboration with the design department and production
→ you need to configure end-to-end processes so that the requirements of the new
sales systems were provided .
1.3.
customers. The new sales system increases the time to contact
by the customer → how to make the depth of the customers' work increase without
increase time managers?
2
Layers
2.1
The implemented sales system manages the actions of managers, imposing on them
specific requirements → How to make sales system requirements
performed, but did the managers spend as little effort as possible?
2.2
Managers are faced with the fact that for a number of clients the requirements of the new system
redundant, which does not increase, but rather reduces efficiency ( from this point of view
managers “manage“ the reaction of customers, hence the distribution of layers on
scheme ) → Differentiate customers and introduce a new sales system only
in relation to such client groups in which conversion increase is expected
and the average weight of the transaction when applying this system .
3
Communications
Partially analyzed in paragraphs 1 and 2, additionally:
3.1
A logical conflict between two systems, for example, the approach to
identification of needs, the stages of the transaction are radically different →
Compare the requirements of the existing and new systems, identify areas
similarities and cardinal discrepancies, disassemble into elementary steps of the area
cardinal differences, thereby simplifying the implementation (such a statement of the problem
allows the solver to rely on existing resources).
3.2
Communication defects ROP managers → Define metrics and reference points in the new
sales system, which should be feedback from manager to
to the leader. Simplify data retrieval by reference point managers.
3.3
Establish a relationship CRM-system - ROP → Having solved the tasks 3.2, bring the CRM-system in
in accordance with the decisions received, amend the procedure accordingly
meetings, strengthening communication on reference points and reducing
communication on non-essential items.
four
Processes and
the functions
4.1
The task appeared after setting the task 1.2: to conduct a detailed analysis of business
processes between the sales department and the design department, as well as between the department
sales and production department (pre-mapping processes with
using BPMN notation) . Highlight bottlenecks and set tasks for them
overcoming.
4.2.
After solving task 1.1, set the task to simplify the entry of the required data into
CRM system by entering patterns and rules.
five
Groups
5.1
Negative phenomena within a group of managers - the effect of the adoption of new
technology modeled by J. Moore → how to use innovators and early adopters
as a resource for introducing a new sales system? How to identify and
to neutralize the influence of “farther“?
5.2.
Customer groups, which follows from the analysis of the task 2.2. Dividing customers into
categories A, B and C. Define customer categories and target customer groups, for
which the new sales system is redundant. Set a task to synchronize work
department, which should apply both sales systems, if the hypothesis is confirmed and
the existing sales system would be appropriate to maintain for certain
customer groups amid the introduction of a new one.
6
Places and material 6.1.
To conduct training of “good middles” in the new sales system after solving problems
from paragraphs 1-5 and determine whether they have reached the level of “stars“ after a given time. If not,
to conduct a comparative analysis of the work of both and conduct additional training of “good
average “according to the performance model (the performance model explains
what specific competencies make stars stars by comparing their competencies with
competencies of “good middle peasants” in the team and identification of discrepancies) .

And that, according to the results of applying schematization and analysis of the scheme, taking into account
13 categories describing the success of introducing a new
sales systems in the practice of the sales department of this production company. Author
pays special attention to the tasks set in paragraphs 2 and 6 of the table. If not
enter categories of layers and categories of place-material, these tasks could be
not delivered. And the tasks of meeting the requirements of places and material properties are
the most important decisions of organizational and managerial tasks set in
organized social systems! Some tasks from the table do not require
the use of TRIZ tools - they can be put to execution using the method
SMART [14]. Some tasks require the use of “primary processing” tools
Tasks ”: streaming analysis, causal analysis, comparative analysis.
An attempt to solve some problems will lead to the formulation of technical contradictions [11].
So, schematization allows a primary analysis of inventive
situations when solving organizational and managerial problems and get a set of private
tasks, including taking into account the layers and the compliance of the material properties with the requirements
places for the solution of which the TRIZ standard tools are excellent.
The second option is to extract unwanted effects from the resulting scheme and
subsequently we work with them - see appendix 1 .
The algorithm of work with the organizational and managerial task with
subsequent use of TRIZ tools:
The algorithm for solving organizational and management problems using
TRIZ schematization and methods according to the results of project implementation are presented
as follows:
Fig. 13. The algorithm of work with organizational and managerial tasks. Scheme.

Method for identifying the operative zone in
organizational and management
tasks from a pair of TP
Goals and objectives of the study
It’s known that the operational zone is the location of the conflict,
which is the cause of the invention [42]. Author
emphasizes that when solving organizational and managerial tasks most often,
there are several operational zones, that is, usually we are not talking about one conflict,
being the cause of the inventive situation, and about their
totality. And it’s not at all a fact that, having performed a causal analysis, the solver
guaranteed to establish one single reason for their appearance, unless, of course,
will move along the causal chain “from inside out”.
The definition of the operational zone is required, first of all, for localization
places for choosing a resource, since decisions obtained using resources taken from
areas of conflict are closest to ideal [22].
In technical tasks, the operational zone is much easier to determine, there
it is always localized in space, and its localization is defined by the boundary
tasks. For example, if during a drilling operation, the cutting edge of the drill
blunts, then the operational zone is in the zone of cutting the metal of the workpiece, to
transition to the micro level, of course. When moving to the micro level, operational
the zone will be in the layers of the material of the cutting edge drill, possibly
will go to the level of the grain boundaries of the metal, etc. For specialists in
material science operational area is quite obvious.
When solving organizational and managerial tasks, such certainty is not
observed. For example, if you assume that you are a scientist and spent a long
analysis of the problem of employee motivation using tools
pre-processing tasks, as a result of which they were able to reach the level
neurobiology and localize the biochemical processes of the brain in the operative zone, then
in this case, the operational area will be determined by the same principles as
and in technical tasks, that is, it will be a piece of space in which there is
conflict leading to an inventive situation. What blocks
sufficient dopamine release [29]? Why not happening
positive reinforcement for which serotonin is associated with other
neurotransmitters [29]? The task becomes material, belonging
physical objects ...
However, the vast majority of managers are not neurophysiological scientists.
Even psychologists primarily operate with abstract concepts and
prefer not to touch the physical levels [23], [29]. Therefore, when deciding
organizational and physical tasks physical layer usually not available

solver. And this is not an exception, but a rule. For example, what is motivation? By
by and large, motivation is the person’s inner ability to overcome
resistance to rest to achieve the goal [29] Motivation is always
concerns the inner world of man, in contrast to stimulation (external
impact) . Analyzing the phenomenon of motivation, we are forced to analyze such
categories like human abilities, peace, goal, human values ​​and
etc. - even a cursory analysis of these categories gives an understanding that the physical level
solving such problems is usually not available to the solver. At least in
present. Continue: components of motivation: goal, willpower,
self control. Goals are set by the context, environment and human value system
[24]. It’s very difficult for a solver to “grab” “hard” resources, that is, resources for
physical level. Talking about motivation, we can say a lot about conflicts,
without mentioning a single point in space.
Therefore, if the operational area is the location of the conflict,
which contains a tool (an object that performs a negative impact),
product (an object that perceives this effect) and the environment surrounding
conflicting pair [3], and the operational zone in organizational and managerial
tasks often can not be described at the physical level, then operational
a zone in such a class of tasks is nothing more than a conflicting pair isolated from
TP pairs (TP1 or TP2).
Why is there a point in solving organizational and managerial problems
first highlight the contradictions, and then move on to the allocation of the operational zone?
This is due to the fact that in such problems the contradictions are primary, usually they
manifest either in the conflict of interest of key stakeholders ( author
repeatedly used the following move in projects: after schematization
inventive situation and highlighting key stakeholders carried out
MPV analysis [43] , which allows to identify stakeholder requirements included in
contradictions, in fact - a conflict of interest. These contradictions are analyzed and
fixed in the form of a pair of TP).
Or contradictions appear when you try to make some
changes, for example, after highlighting tasks in the wake of schematization solver
comes to the conclusion that certain changes are necessary, but with mental
the projection of these changes on your system sees a secondary undesirable effect,
giving rise to a contradiction. At the junction, it is easy to formulate a new pair of TPs, which
usually done. For example, in Sberbank a similar thought experiment with
recently entrenched as an organizational and managerial norm (from
conversation of the author with TRIZ corporate training participants for employees
Sberbank TT Group) .
Understanding that operational and organizational tasks
the zone is determined by the conflicting pair in the working TP with the addition of a description
environment of negative interaction between the tool and the product (note : tool -
the subject of the negative “processing” of the object, that is, the product) , gives the key to
the source of resources closest to the source of the conflict in such tasks and
allows full application of ARIZ mechanisms to organizational
management tasks (the tool will be the state of the system in
previously selected working TP , and the product - a deteriorating consumer property
systems). The author has already indicated that to address organizational and managerial

tasks it makes sense to use only shortened, “combat” forms of ARIZ in several
steps.
The main feature of the method of describing health care for management systems,
which the author describes is the use of a new parametric approach
in the description of the operative zone. This method is well suited to describing OZ in
complex social systems with objects distributed over time, in
space and other characteristics parameters. For example, if in a department
company procurement changes have occurred in significant areas of the business process then
such changes will entail both positive actions and undesirable effects in
other departments, that is, immediately there are many operational areas! Arises
question: how to track them, especially in large companies? And whether you need to track them,
Isn’t it easier to switch to another method of description - parametric? Besides,
decisions can have a delayed effect, and in different areas of the business - in
different time. Therefore, in solving organizational and managerial problems, much
it’s more efficient to consider which parameters influence the parameters indicated in
identified contradictions, and use them already as resources [11].
That is, we are dealing with a situation where the system contains many
elements, each of which is described by a set of parameters:
[P] 1  E 1 ; [P] 2  E 2 ; [P] 3  E 3 ... [P]  E
(one)
Formula (1) shows that each element “E“ is described by a set
“[П]” parameters, and such elements in any business system, albeit a finite number,
but there are a lot. Relations between elements in a business system
It’s also very difficult, especially since the connections between the elements are constantly
rebuild depending on the influence of external and internal systems
factors. That is, to select elements in operational areas, and even more so, to catch
communication between them can be extremely difficult, therefore, to identify operational
the zones associated with the studied undesirable effect are almost impossible.
However, if we are dealing with contradictions arising in business
systems (Fig. 14, Appendix 5), in practice it is much easier to determine which
parameters determine the parameters in contradiction than to look for conflicting
pairs in the business system associated with the investigated contradiction, and describe them
operational areas, and later to explore the parameters of these elements. As it was
noted above, the essential parameters of the X-elements of the business system,
conflicting defining parameters are much simpler
detect than the elements themselves, in addition, the detected parameters are easy
use as resources to solve the problem. That is why
a parametric approach in working with the operational zone when solving organizational
management tasks should be considered as promising.
The author believes that this conclusion, which was obtained and confirmed in
the course of participation in more than 50 projects allows universalizing ARIZ approaches
and apply them equally successfully to both technical and organizational
management tasks.
The analysis of the ARIZ task remains the same for both organizational and
managerial and technical tasks, with the only difference being operational
the zone of the technical problem is the physical area of ​​space, and in the organizational

management tasks the operational zone is formed by a conflicting pair
the state of the system is a consumer property along the negative branch, in fact -
conflicting pair expressed in abstract terms (not physical
area of ​​space).
Otherwise, the procedure for applying ARIZ to any artificial systems
remains the same, which allows the efficient use of ARIZ for solving
organizational and management tasks.
Note that the use of ARIZ for solving this class of problems
appropriate if the analysis of the contradiction is not satisfactory
results.
An example of the allocation of the operational zone in organizational
management task
We apply the found principle to the resolution of the contradiction depicted in
fig. 9. Recall the essence of the problem (in the form of a pair of TP), which will be the starting point in
our further considerations:
If the number of transactions is simultaneously worked out by a department employee
15 sales, then the managers go to the targets faster, however, for
fulfillment of sales plans with existing indicators of transaction conversion
increase the number of managers in the sales department, which is unacceptable (TP1).
On the other hand, if the number of transactions is at the same time
sales person 25, then to fulfill sales plans when
existing conversion rates need to hire fewer employees in
sales department, however, sales staff slowly reach the planned
indicators, which is unacceptable (TP2).
Further, according to the ARIZ logic, the working
technical contradiction, for which we carry out the following reasoning:
A sales department is created for personalized work with a customer on
creating a high subjective value of the proposal, overlapping the value
payment. Personalized work is the core of definition. If possible
the same value is created by influencing immediately on a group of consumers, the sales department does not
needed, it should be curtailed as an extra link in the business [16]. If the sales team
is present in the company and is not a consequence of psychological inertia
managers (sellers should be, because they have always been here), then
personalization of customer impact is the only way for the company
convey the high value of your proposal.
Of course, in this regard, sellers should serve the greatest
the number of transactions at the same time without loss of conversion and without
Reducing the average weight of a transaction within a specific client category . From here
it is clear that one manager should not have deals in simultaneous processing
15, 25. Therefore, the working TP is as follows (Fig. 14):

Fig. 14. The choice of working TP from a pair of TP
We need to increase the workload of managers, so we select TP2 : if in
At the same time, the manager is working on 25 projects, the number of
managers in the sales department will decrease, but the manager will go on
planned targets are longer, which is unacceptable.
Then the conflicting pair will be as follows :
25 projects were given at the same time being worked out by one manager and 6
months of reaching planned sales figures.
There is clearly a conflict here: 25 projects are being worked out at the same time
one manager (tool), negatively processes the exit to the planned
sales figures (product):
Fig. 15. The operational area, including the tool, product and their environment
interactions. 25 projects - a tool, planned indicators - a product.
Next, decompose the tool and product included in the model
operational zone (Fig. 15), on components that can subsequently be
use as resources, substituting them in the formula for an ideal end result
RBI

We give an example of such a decomposition [11]:
Operational Zone Element
The role of the element Resource as a subsystem of each element
25 projects at the same time
at one manager
Tool
Decision making scheme in a category A project
Decision making scheme in category B project
Stage of the transaction (sales funnel)
Sales channels
Project related work
Errors in recruiting a customer base
Going to targets 6 months
Product
The number of leads (responses to marketing
activity)
Lead quality
Sales funnel conversion
The average frequency of a transaction per year
Employee Competencies
Client base recruitment regulations
Collaboration with colleagues
As a result, we received an impressive list of resources for solving
tasks. Some can be further decomposed, for example: sales channels,
related work of the manager in the project, etc.
If there are many resources received, they can be subjected to the procedure
prioritization, for example, according to the following logic (priority falls on the left
to the right) [25]:
1. The element of the operational zone: Product → Environment → Tool
2. Quantity: Unlimited → Sufficient → Limited
3. Quality: Harmful → Neutral → Useful
4. Value: Free → Penny → Dear
Prioritization of resources: the higher the final score, the higher the priority [11]:
Operational Zone (OZ) Resources
Element oz
amount
Quality
Value
ITO
G
Ed
e
l
and
e
Wed
e
Yes
Ying
page
mind
e
nt
N
e
about
gr
ani
what
nn
th
D
about
stato
h
th
ABOUT
gr
ani
what
nn
th
Vre
dny
(about
tho
dy)
N
e
th
tr
alny
P
about
l
e
know
B
e
spl
atomic
TO
about
ne
e
h
th
D
about
R
about
go
th
Decision making scheme in a category A project
one
3
one
one
6
Decision making scheme in category B project
one
3
one
one
6
Stage of the transaction (sales funnel)
one
3
2
one
7
Sales channels
one
2
2
2
7
Project related work
one
3
2
2
eight
Manager errors when recruiting a customer base
one
2
3
3
9
Number of leads
one
one
one
one
four
Lead quality
3
one
one
one
6
Sales funnel conversion
3
one
one
one
6
The average frequency of a transaction per year
3
2
one
2
eight
Employee Competencies
3
one
2
one
7
Set-up regulations
2
3
one
one
7
Collaboration with colleagues
2
2
one
one
6

In our example, three resources allocated in
table in gray. Therefore, they should be used first.
For this problem, more than 10 solutions were obtained using dedicated
resources, and this is for one pair of TP! For example, I would like to show how it worked
harmful resource - manager errors that get the maximum score.
In the ARIZ logic, the ideal final result (RBI) rule is assigned and then
instead of the X-element, the selected resources are substituted. (RBI - decision
inventive task, allowing to obtain the desired result with
zero compensation factors. As follows from the laws of physics, such
a solution can never be reached and therefore the concept of perfect
the final result serves to reduce the degree of psychological inertia in
the process of solving the problem by orienting the problem solver to search
solutions with the highest degree of ideality [42] ).
We demonstrate these steps:
1. Rule of RBI : <X-element> itself provides access to planned indicators
manager for 3 months ( condition of the task manager ), provided
performance index of 25 projects at the same time exploring ( with
increasing the load on managers to achieve planned targets in
companies occurred on average for 6 months ).
2. Manager mistakes when recruiting a database of projects themselves provide access to
manager's planned targets for 3 months, subject to fulfillment
indicator of 25 projects at the same time.
3. Since it was not possible to directly obtain a solution from RBI, we proceed to
the formation of physical contradiction (FP) around the selected resource
( FP - a situation that occurs when a certain attribute
the object of interest to us must have two different meanings
at the same time to ensure the desired result [42]): errors in
recruitment of the base should lead to the correction of technology sales manager,
in order to reach the target indicators for 3 months, and errors in the selection of the base are not
lead to the correction of the manager’s sales technology, since the manager doesn’t
has sufficient skills to reflect errors that occur during recruitment
customer base .
Since the company that set this task has implemented an adaptation system
sales staff, it was easy to take control of the mentor process
recruitment of the base of projects to newly arrived managers and to carry out its reflection
first, 2 times a week, then - 1 time per week, then 1 time in 2 weeks,
thereby making his mistakes a resource for correcting further work. Similar
reflection is carried out according to performance models developed for mentors
[sixteen].

After the formation of the FP, the solution involving this resource turned out to be
the obvious is managing the employee’s reflection process in the process
initial set of customer base.
Roadmap for working with the operational zone in the organizational
management tasks
\item Formulate a pair of TP;
\item Determine the operating TP (TP1 or TP2);
\item Select the conflicting pair in the working TP;
\item Highlight the operational area, additionally defining the interaction environment
tools and products (the operational area consists of a “tool”,
carrying out harmful effects and “products” perceiving
harmful effects [3]);
\item Allocate resources to the operational area and determine their priority, if resources
lot.
\item Formulate a RBI rule.
\item Substitute resources in the RBI rule instead of the X-element. If the decision is not
obtained at this stage, then form around the selected resource
physical contradiction (we act in the logic of ARIZ).

Conclusion: conclusions and recommendations
The effectiveness of the proposed methods
The effectiveness of the proposed methods is practically confirmed:
1. Schematization - the tool is used in more than 20 projects;
2. Express analysis of contradictions - in more than 50 projects;
3. Formulation of the operational area in management tasks - in more than 30
projects.
These tools are included in the training program implemented by the author in full-time
format and format of the online workshop. 200 online training programs trained
people, according to the full-time program - about 150 people. During the training, students
(students are company specialists) carry out projects in the field of their
activities and protect projects based on learning outcomes. By citing the numbers above,
the author had in mind only the most high-quality projects of the training participants.
Scope and limitations of the proposed methods
The author assumes the use of these techniques to solve organizational
management tasks set in any organized social systems.
These tools can only be used provided that the solver
owns the subject of research, or works closely with specialists,
possessing the required substantive competencies in the field of strategic and
regular management, marketing, sales, financial planning,
psychology etc. [2], [13], [23].
The author’s practice shows that the greatest efficiency in application
tools can be achieved in team mode, if the work
teams are effectively supported by flexible project management tools,
First of all, Scrum technology [26].
The author's recommendation for use in organizational and managerial tasks:
1. Always apply schematization to the full clarification of inventive
situations subject to the use of TRIZ in organizational and managerial
tasks;
2. Apply the method of formulating the operational area and working with resources
of the operational zone proposed by the author only if
formulating a TP pair, the solution is not obvious and the solver expressed a desire
continue to move in the logic of ARIZ.

The possibility of further development of techniques
The author believes that TRIZ specialists should take a closer look at
G.P. Shchedrovitsky [10], [41] and study the application of categories of systems,
proposed by the author, for a more accurate and quick description of inventive
situation.
A special methodological study requires the category of “layer“, as well as
“Place“ and “material“, practical recommendations for a more conscious
the use of these concepts in solving organizational and managerial problems.
The author believes that research in this direction should be continued.
The author believes that the proposed description of the operational area is not
final. The author believes that it is necessary to develop a special
methodological language for the description of the tool, product and, in particular, their environment
interactions . As a result of this description, the allocation of operational resources
zones can be much more accurate, therefore, will give even more interesting
practical results.
The development of criteria according to
with which the solver can make a detailed analysis of the resources of the operational zone.
The use of TRIZ for organizational and managerial tasks today
a day far from established discipline, there is a significant
research work.

ANNEX 1
An example of using schematization for analyzing inventive
situations in conjunction with S-curve analysis.
Objective: to increase staff productivity and reduce time costs
the first person of the company through a change in the employee motivation system. Note: in
This example does not show the final solution to the problem, it is demonstrated
exclusively an analysis of the inventive situation.
Tasks: pyrotechnic company “Fast and the Furious“, St. Petersburg.
From fig. 13 shows that during the existence of the company, the author changed three
motivational models with which he inspired his close-knit team.
As the MPV (main parameter of value) adopted “employee productivity“,
expressed in the number of operations per shift with the required level
quality . Since the concept of “operation” is predetermined, and the operations themselves
reflected in the technological maps that are compiled for each event,
perform a performance calculation and determine the level of quality of performance
work is easy.
Fig. 16. Change of various motivation systems in the company.
We describe the motivation system shown in Fig. sixteen:
Curve No. 1 - at the beginning of the existence of the company, in the late 90s - for the promotion
It was considered to be arranged in an organization where there is a normal social package,
stable salary and healthy relationships in the team. At the beginning of this approach
perfectly stimulated employees to work, compared with others not quite
“White“ companies. But over time, the “white package“ began to be accepted as the norm, and
stability was no longer a motivating factor, but perceived as
due.
Performance
employees

Curve No. 2 - the company introduced a system of cash bonuses. She gave
tangible
growth
performance
and
responsibility
but
then
performance declined. Previous bonuses were no longer available for motivation
to work. Numerous studies have been conducted on the effect of money on motivation,
of which it is known that the award is perceived by the employee as a motivator approximately
three months, after which he begins to take it for granted. According to the findings
S. Covey [30], money for business is like air, without them the company cannot work
can, and the employees quit. However, for life, a person needs not only
breathe. Therefore, such a system has very limited resources for its
application.
Curve No. 3 - a decision was made to develop an intangible system
encouragement, which would be based on their own moral and ethical values
employees who must match the values ​​of the founder of the company, given
requirements of the pyramid of needs A. Maslow [31]. Today the system is in
the beginning of this curve, and according to the director’s forecasts, it will give a smoother, but
steady and continuous growth of labor productivity and personal
responsibility of employees, will spread its influence both on the selection of employees,
and on their retention. Naturally, the value motivation system is by no means
It does not cancel the system of monetary incentives, it supplements it. Observations give
reason to believe that a value approach combined with a powerful system
training can give about a twofold increase in
selected MPV.
From fig. 16 shows that the studied system in this company is located in
the very beginning of the third S-curve, and therefore, all the main efforts of the leader
must be spent on tuning the system - you need to create such conditions that
the system began to work steadily. There are no other priorities at this stage.
Now you need to set tasks, for which it is proposed to parse the system in
the form in which it exists now, and then determine the desired parameters
system, its future configuration ( Fig. 16 shows that the transition to the third
the curve in the company has just occurred while substantial
value shift in the minds of employees takes time and does not occur
instantly ).
To describe the current inventive situation, we apply
schematization:

Fig. 17. The use of schematization for the analysis of an inventive situation.
Since we resorted to schematization after applying the analysis on S-
curve, we got some new knowledge. From fig. 16 we see that introduced
the bonus system (curve 2) was implemented quite successfully, which resulted in
MPV growth, although the system reached its saturation quite quickly.
Naturally, in the diagram (Fig. 17) we fixed the structure of this system in a section
“It was“.
Then there was a transition to curve 3 (Fig. 16) as a more promising
naturally, with the preservation of the bonus, that is, between the two curves occurred
continuity . If the company didn’t do this, then the transition occurred
there would be a significant “drawdown“ of MPV (Fig. 16). Yes and no cash reward
Motivation systems have no prospects, any normal leader knows this.
In addition, the company determined the position of the new motivation system at S-
curve - this is stage I (Fig. 16). In accordance with the objectives of the first stage, we say not
as much about efficiency, how much about the potential of the system and its minimum
health. Therefore, in the diagram in Fig. 17 we depict the structure based on
assigned tasks: to ensure the minimum performance of the new system
motivation, but taking into account its configuration (at the heart of the system proposed
task manager, the pyramid of needs A. Maslow). The author does not consider the pyramid
A. Maslow is an exceptionally correct model, but to describe the current situation, she
perfect - over the years of work in this company, employees have “grown“ from the point
view of values ​​and the pyramid of A. Maslow it simply and reliably demonstrates.
Carrying out the analysis of the circuit in Fig. 17, we found unwanted effects
(NE) and recorded them in the table:
No. Condition of elements
existing
system, “It was“
Item Status
the new system, “It has become”
Tasks
No.
NJJ Description
one
Prize
distributed
directively
the director
Premium distributed
collective according
contribution
SJS 1
Grievances and their hidden
conflicts
SJ 2
Manipulation of employees in relation
to colleagues

2
Staging
detailed
strictly defined
SMART tasks
Outlining frames
statement of general tasks
and setting constraints
employees themselves
SJS 3
Recently arrived employees cannot
work in this mode, since they don’t
lack of knowledge
SJS 4
Distracting experienced staff for
control tasks less
experienced
Transition to a single environment
planning
for example to flexible
design system
management for small
teams - SCRUM
SJ 5
Regular weekly planning
takes extra time, usually -
2 ... 3 hours a week.
SJ 6
Irritation from repetitive operations
planning, team briefings, etc. →
decreased attention, attitude to the system
planning as an unnecessary load
four
Employee values
at level 1-2 in A.
Maslow
Mature employees
having values ​​3-4
A. Maslow level
SJF 7
Often, such employees want to open their
business, so they leave the company
SJ 8
Such employees have their own
opinion on working matters with them
need to agree. Manage such
people - it’s the same as grazing cats.
SJ 9
Need to maintain interest in
all areas of motivation - money,
emotions, intelligence, meaning and contribution to
a society that requires significant
efforts from the director / owner
SJS 10
Such an employee has versatile
interests, not the fact that manufacturing
tasks will be paramount for him
five
Primary control
carried out by the director
Director carries out
general control
indicators, control
quality operations
do it yourself
employees
SJ 11
With the loss of workplace value for
employee risk of deterioration
quality of operations, which
may go unnoticed
SJ 12
Even when an employee sincerely tries,
he is subject to the factor
eyes “, that is, simply does not see
own flaws that are easy
see from the side. However external
detailed control abolished.
The selection of NE in the analysis of the circuit in Fig. 17.
Thus, 12 tasks were set, the solution of which can provide
working capacity
selected
the system
motivation
staff.
Analysis
inventive situation allowed to quickly identify and formulate 12
specific tasks, which would be difficult without using a system
approach, given the fact that the selected system is at stage I of development according to S-
figurative curve and did not pass approbation.
It was decided to introduce a motivation system taking into account the implementation of the found
solutions to the tasks given in the table.

APPENDIX 2
The task was set by the director of one company as follows: how
register business processes independently, without complicated terminology and unnecessary
paperwork? It was required to give a minimal template that would allow
perform work on the description of the company's business processes so as not to produce
unnecessary information. It would not happen that the developed business processes
would slow down the company, deprived of its required dynamics. At the same time, work on
the intuition, as before, is no longer possible, the young company was faced with the first
growth disease. The company faced a serious controversy.
A lot of literature is available on how to prescribe business processes. But almost
it is not indicated anywhere how recommendations for describing business processes in
depending on what stage the company is at. And even if such
There are recommendations; they are quite heavy and bulky. We set the task to give
capacious and accurate recommendations, differentiated for different stages of development
business, which will answer exactly the question: what model
Need to prescribe business processes for this particular company ?
We apply the system operator in order to better understand the system by
Description of the company's business processes. The structure of the system operator is shown in
fig. 3 in the main part of the dissertation.
System Operator:
1) PRESENT.
1.1. The investigated system: business processes.
The system of business processes, special attention to cross-cutting business processes (affecting
work of 2 or more departments or groups). Process flexibility.
Quote: “Most companies are organized according to a functional principle, but they
should work in conditions of interfunctional interaction. ... processes
break the hierarchical structure. “
1.2. Supersystem
Strategic management, balanced scorecard. Horizontal
employee interaction. Quality management system - as a methodology. Market,
competitors. The dynamics of the environment. Changes to the law.
Quote: “From the point of view of the process approach, the organization appears as a set
processes. The management of such an organization is based on process management.
Each process has its own goal, which is its criterion.
effectiveness. The goals of all processes are lower level goals, through
the implementation of which top-level goals are achieved - the goals of the company. ”
1.3. Subsystems:

Business process system (model), responsibility management, management
personnel, process regulation, personnel reporting, process automation,
process performance management.
2) PAST 30s 20th century.
1.1.
System:
A person in the workplace, instructions of managers (namely “instructions”).
A.K. Gastev focused on the human factor. He believed that the main thing
the role in the work of the enterprise is played by man. Quote: “organizational effectiveness
begins with the personal effectiveness of each person in the workplace, in particular
with the efficient use of time ”(development of a description technique
production processes at this time is primarily associated with the name of this
wonderful person).
The most important problem, according to A.K. Gastev, there was an inability of a working man
obey, work in a team and strictly follow the instructions of the leaders.
1.2.
Supersystem
The survivals of the agricultural system, industrialization, leadership, production pace.
Rigid hierarchical structure in the enterprise.
A.K. Gastev noted that “workers do not know how to keep a single production pace
and work as well as their European counterparts do. Way of life
peasant Russia without rich European working traditions. ”
1.3.
Subsystems:
Workplace; Personal qualities: a sense of time, personal efficiency at work
place, the ability to obey.
A.K. Gastev emphasized that “Russian workers lack a sense of time.
Russia, in which the workers are former serfs who went to the free
bread, this way of life did not initially contribute to the acquisition of the European “installation on
time“.
3) THE PAST 70-80s. 20th century.
3.1. System:
SADT standard (Structured Analysis and Design Technique ), functional methodology
Simulation IDEF0 (Integration Definition For Function Modeling) .
One of the best-known methodologies for describing organizations as organizational-
technical systems, has become the methodology of structural analysis and design
SADT systems (Structured Analysis and Design Technique ). It was developed
American Douglas Ross (D. Ross) in 1973. Particularly widespread use
received one of the SADT subsets - functional modeling methodology

IDEF0 (Integration Definition For Function Modeling ). The initiator of its development
and further standardization was the US Department of Defense. Methodology
IDEF0 was successfully used in military, commercial organizations to solve
a wide range of tasks (from software development for defense
systems prior to the development of logistics and management systems
finance). Availability and experience of using IDEF0 in various subject areas
areas, along with growing computer support, made it even more affordable
in use. This, in turn, also led to the widespread use of IDEF0
as a methodology for describing the business processes of organizations. In many ways, the popularity
functional modeling methodology IDEF0 due to the ease of notation,
the main elements of which are the function block and arrow.
Also in the USSR at the beginning of the 70s, an Integrated Management System was introduced in the USSR
product quality (CC UKP) . Management was based on the logic of mass
production, economies of scale, centralized control, and also resulting
low rate of change and a rapid loss of relevance.
The control system inherited from the USSR is based on the concept of mass
production, which dominated the entire national economy. The main purpose of this
systems - get the economic effect of the growth of production. Than
the larger the volume of production, the lower the cost per unit of output. At
it’s easier to standardize and unify processes, and also easier
carry out centralized control. Such a system allowed to produce
a huge amount of TRU (goods, works, services), but in order to change something
had to spend a huge amount of resources due to lack of flexibility
in management and processes. As a result, it turned out that in the international arena, our
enterprises were uncompetitive due to lack of flexibility and
the inability to quickly adapt to the needs of the market.
3.2 Supersystem:
Strategic management, balanced scorecard. System
quality management. Competitors, market ... Relative stability, gradual,
smooth change of scenery (a significant difference from the NS “Real“ ). Acting
legislation.
3.3 Subsystems:
IDEF0 Principles, Process Diagrams, Process Performance Management, System
business processes (model), responsibility management, personnel management,
process regulation, staff reporting.
4) FUTURE.
1.1. System:
Flexible business process cards integrated into CRM systems and more
high level (ERP).
1.2. Supersystem
Self-developing business (company), further development of LEAN, CRM-system, ERP-
systems with integration of machine learning algorithms, BigData, distributed
registries.

1.3. Subsystem:
Instant access to self-updating information. Flexible business process system
(model), responsibility management, personnel management, regulation
processes, automatic reporting by indicators, flexible management
process efficiency. Automation, robotization, competency development system,
knowledge management.
From the analysis of the system using the system operator, we can distinguish
following:
1. Subsystems: Quick access to information. Business Process System (model),
responsibility management, personnel management, process regulation,
personnel reporting, process automation, performance management
processes. We see that business processes must be able to quickly
be extracted from the information environment, have high flexibility, have
reference points that show what will change in the supersystem when changing
business process at the level of a specific position.
2. Strategic management, balanced scorecard. Lean
production,
system
management
quality.
Market,
competition...
Relative stability, gradual, smooth change of scenery
( significant difference from the NS “Real“ ). Current legislature.
When designing business processes, a system should be developed
indicators: KPI (key performance indicators) and managerial
indicators by which we track the effectiveness of achieving KPI.
The future shows us that business processes should be included with
knowledge management system, that is, a system should be developed
indicators tied to a competency model. Provide here
no gap! Automatic collection of statistics on indicators
special attention should be paid, to develop a culture of working with numbers,
gradually preparing the control system for the application of methods
machine learning in the future.
3. The human factor significantly affects performance and efficiency
processes. Therefore, when the responsibility matrix is ​​prescribed, the functional
defined, it is necessary to select people in a team with psychological and
competency portrait suitable for the position. Otherwise,
no one guarantees that the processes will work correctly and be fully implemented
volume.
It follows that business processes should not only be tied to
knowledge management system, but also with the profile of the position, which, in general,
is logical.
4. Take into account that the sense of time in humans has evolved since AK. Gastev, but
still far from ideal, so business processes should be
Automated in a CRM system with automatic notification, but in
in any case, before preparing the ToR for CRM, which ultimately will get
BP description, a paper document is created.
It’s important to consider that trying to regulate everything in a row is silly, but in
small companies, such attention to administration is fraught with loss
business. Therefore, before the regulation of processes should be determined
company position on the S-curve (most conveniently according to I. Adizes) and based on this
set the “scale” of regulation, that is, determine the extent
detailing the process. It is also important to determine the degree of freedom of acceptance.
employee decisions in changing business processes in order to increase them
effectiveness. As stated above, allow for

operational process change, but with setting markers, which of
Related processes will be involuntarily affected. Should provide
differentiation of access rights to change processes.
5. Studying the success of IDEF0 shows us that for presenting business processes
you should try to get as far away as possible from text instructions in favor of
charts - infographics, drawings with short explanations. If more is needed
detailed explanation, it can be given as a note to
corresponding paragraph of the infographic. Such instructions are perceived and
memorized much better, but there are pitfalls. Good
infographics - the best option in terms of perception of instructions
by the user, but she has a huge minus in that drawing circuits is very
expensive and long in time. Not all employees can do this.
Today, this problem is resolved. In 2016 - 2017, the present
boom in integrating graphical display of business processes in CRM-
systems according to IDEF0 recommendations. Hence it’s clear that it’s worth
pay attention to CRM-systems that have just such
opportunities and use them. It is important to consider monitoring for
indicators indicated above, differentiation of access rights, signaling by
reference points when making changes.
6. An integrated product quality management system (CS UKP) of the USSR may be
interesting only in the case of large-scale reengineering of business processes in
large corporations. In other cases, you should not contact her.
You should pay attention to the company’s standard for designations and
corpus of concepts. The corpus of concepts should be the same for everyone in the company and
as much as possible unified with the practice accepted in the world. “Translations“ of terms
inside the company is too expensive. Therefore, together with the development
business processes should deal with the standard adopted by the company.
It’s better to immediately lay down standardized concepts and notation than later
spend a lot of time and effort to fix it.
7. When choosing a CRM system and a method for preparing a description of business processes
should take into account the rapid change in the environment, the system should be able to
make changes quickly, better - without the involvement of IT-specialists. Otherwise
case, the dynamics can be lost, and the PSU will turn into empty trash and
will stop working.
You should not engage in self-written programs, but use ready-made ones
expandable systems to provide the above
functions.
8. When describing the BP should take into account the interaction between units. Exactly
at the junction of departments there is the greatest defect in communications, distortion
information and various kinds of failures.
Recommendations - see above. Optional: when determining a personality profile
the place should not be considered in isolation, but viewed in conjunction with units
and process owners with whom business processes are intertwined most
closely. The problem is solved at the level of places, do not go into the properties of the material (see
recommendations for schematization)!
9. In the future, the impact of IT technology will increase, so the final product will be
CRM-system with embedded PSU, giving hints in real time.
In the form of a list of documents BP will exist only at the time of implementation, in
quality of project documentation. Next is only the electronic format.
Pay attention to software manufacturers focusing on
issue of tips and statistics increased attention.
As a result of the analysis of data concentrated in the system operator,
a matrix for the implementation of a system for describing business processes depending on
the stage of development of the company, the author has not met analogues of this matrix in any
specialized literature, nor in their practical activities (stages

development are given in lines according to I. Adizes [13]). The columns are informational
blocks required in the description of business processes for each stage:
Designations and abbreviations:
+ ... +++ - the degree of detail of the documentation;
Business Processes (BP)
Organizational Structure (Organ.)
Job Description (CI)
Reporting (Report)
Regulations (P)
Standards Management (Ex. Art.)
Compliance Monitoring Standards (CIS)

Performers in the description of business processes:
Designations and abbreviations:
“-“ missing;
C - himself;
K - team;
Co - consultant (expert).
Case shows that using the system operator provides interesting
results in solving organizational and management problems, but more suitable
to find strategic decisions .
When solving situational organizational and managerial tasks, application
schematization makes it possible to more accurately analyze inventive
the situation and set particular tasks including, applying the categories of layers and places-
material.
It seems that this state of affairs outlines the scope
system operator and schematization to solve this class of problems.

APPENDIX 3
Task:
A modern, fast-growing Chelyabinsk IT company developing
platform solutions for working with digital content using
artificial intelligence (number of staff: approx. 200 people), repeatedly changed
their structure, trying to find a balance between the structure, which are based
departments (marketing, finance and accounting, several development departments, broken
competencies) and the structure based on project teams,
including a variety of specialists, depending on the objectives of the project.
The company faced several inconsistencies, one of which
consisted of the following:
Fig. 18. A pair of TAs related to the organizational structure of an IT company.
Contradiction analysis:
TP1. In order to optimize the loading of specialists, the basis of the structure
companies should have departments (unit units), as the head
unit determines how much time one employee needs to
one or another project and, since he owns information about all projects, he
can optimize the work of the employees of his department. That is, loading
unit specialists are optimized as there is a centralized
work planning by an experienced specialist (department head), who
sets tasks to department specialists and monitors their implementation .
TP2. In order for specialists to better understand the nuances of the project, they must be
completely involved in it, that is, belong to one or another project team,
since discussions of all the nuances of the project take place in the presence of the project
teams, and at all stages of the project - from the inception and appearance of MVP (minimum
viable product - the minimum viable product) [32] to obtain a full-fledged
market product, that is, they take part in the discussion of all the nuances
project in all its stages .

Combining new entities with opposing states of the system on
fig. 18:
1. Is it possible to make the project team at the heart of the structure?
the company had a single coordinating center for competencies, which would put
tasks for subject specialists belonging to different teams, and
would control the execution of tasks?
Here came the idea of ​​applying flexible principles of project management at the level of
the whole company, that is, with a built-in system for recording time and project indicators.
Agencies supporting flexible planning principles exist (Agile and SCRUM
[26]), while allowing you to shoot high-quality analytics (such as Asana [33],
eg). At the same time, the top management of the company gets a special,
a guiding and inspiring role; much attention needs to be paid
continuous training of staff, especially project managers, to which
special requirements.
Conclusion: a flexible planning system with integrated performance monitoring
accept, but completely reorient the company to “confederation“
project teams - a risky strategy.
2. Is it possible to make departments remain at the core of the structure, but their
experts would take part in the discussion of projects not occasionally, but
constantly, at all stages of the project with a wide range of
specialists involved?
Here came the idea of ​​creating a “managerial club” within the company, members
which meets at least 1 time per week and discusses projects in
development in the company. In addition, “design circles” have been created that are not found
less than 1 time per week, in which all specialists involved in
this particular project.
Management Club and Project Clubs are supported by flexible methods
planning and related tools.
This solution is partially implemented; full implementation in the company's practice
planned for 2018.
Transformations in the company for 2018:
\item We create a unified system for assessing the success and contribution of the project and
resource allocation.
\item We introduce a flexible methodology in which company employees participate in
project level.
\item The previously adopted structure of the company, based on which unit-
units from which resources are drawn into projects.
\item A set of measures is being developed, according to which, employees will
involved in the product life cycle.
\item Developing practices of the “Management Club”.
\item “Project circles” are being created.

APPENDIX 4
Task.
Find a market way to get the company out of the crisis (company “X” takes
about 10\% of the Russian market, while being the second producer in the Russian Federation by market share).
The crisis is generated by the actions of a more powerful competitor, occupying a leading
position in the Russian market and owning a share of more than 50\% of the market.
The main objective of the project (after conducting a preliminary analysis): how
increase the number of branches and managers without resorting to attracting
a significant amount of credit?
The main results.
Found an organizational solution that will allow the company to successfully
compete with a more powerful target competitor without borrowing
cash, as well as without changing the main product line, that is
exclusively in a market way, which was the main requirement of the customer.
Introduction
Company “X“ is a manufacturer of concrete additives, takes 2nd place in
market share of the Russian Federation, occupying a little more than 10\% of the Russian market. Main competitor
company X occupies more than 50\% of the Russian market share, and its share continues to grow,
and the market share of company “X”, on the contrary, shows a steady decline.
Market analysis of concrete products and ready-mixed concrete in Russia, as well as volume analysis
cement production in the Russian Federation (indirect indicator) shows that sales decline
company “X“ in the last two years can not be caused by the fall of the target market
customers - manufacturers of ready-mixed concrete and reinforced concrete products (see market analysis). As
source data of the analysis used data obtained by the analytical department
company “X“ (see. Fig. 19).
An analysis of the products of company “X” and its main competitor revealed a complete
similarities in the composition and quality of products of these two firms, and the survey data
target customer groups show even slight excess
product quality of the company “X“ in comparison with the product of the main competitor.
It should be noted that there is a promising direction in the market of concrete additives
- polycarboxylates. These are more advanced supplements that allow for much
lower concentration give more stable properties of concrete. Manufacturers
such additives are foreign companies. But neither company “X“ nor its main competitor
does not produce such additives, moreover, their market is still small, despite
undoubted perspective.
Product comparison data provided by the analytical department
company “X“, a survey of target customer groups conducted by us.
Product cost analysis also showed full compliance with the proposal.
company “X“ and its main competitor.

Thus, by exclusion, we can conclude that the main
the difference is in customer service , which means we are dealing not with technical, but with
organizational and management task.
Since we are talking about the B2B market, then studying market promotion, first
the turn should be paid to sales managers [27]. Study
in the form of a customer survey showed that the average qualifications of managers of both
companies are at the same level. At the same time, customer focus
managers “X” marked by customers as much higher compared to
the main competitor, while the share of company “X” continues to fall amid growth
market share of a key competitor.
In quantitative terms, there is a huge difference - the number
managers in company X is about 6 times lower than the number of managers in
company “Y“. The situation is similar with the number of branches - the number of branches
Company X is 5 times lower than its main competitor. Needless to say
that the management of company “X“ is well aware of this quantitative
dissonance, however, only after this comparative analysis became
it is clear that the size of the company, and therefore a multiple difference in financial
opportunities - the only reason for the stable loss of the market by the company “X“ on
the background of the growing market at the time of solving this problem .
Maintain a larger staff of managers and branches does not seem
possible - not enough money. Financial Opportunities for X and Its
main competitor is almost an order of magnitude different.
Hence the main objective of the project: how to increase the number of
branches and managers, without resorting to attract a significant amount
loan funds ?
Subtask: it is desirable to make sure that the competitor could not copy
this decision .
Immersion in the task. MARKET ANALYSIS 2007 - 2012.
Charts are based on data provided by company analysts.
“X.“

Fig. 19. Analysis of the Russian market of cement and ready-mixed concrete.
As can be seen from the above diagrams, the market for ready-mixed concrete and cement after
overcoming the crisis of 2008 - 2009 growing steadily.
Analysis of the sales of company “X” and its most dangerous competitor 2007 -
2012.
According to the end of 2012, in the Russian market of additives for production
ready-mixed concrete the following situation has developed: the main market share is
the main competitor of the company “X” (it owns more than 50\% of this market),
the second place is occupied by company “X” (about 10\%) of the market, followed by other players.
The diagrams are removed from the example for obvious reasons. The diagrams show that
Y sales are growing amid a growing market for precast concrete and ready-mixed concrete.
Sales of the company “X” are falling against the backdrop of a growing market for precast concrete and ready-mixed concrete, and
also amid growing sales of the main competitor. Gradually picture
develops. But while there is no clarity - is it the sales or the product itself?
Analysis of the product offer of company “X“
A comparative analysis of the product is not given here in view of its significant
volume.
Comparative analysis performed by specialists of the company “X” and a survey
customers, showed full compliance with the products of the company “X“ and its main
competitor, and the survey revealed that the quality of the product of company “X“ is negligible
superior to the quality of the product of its main competitor in the parameter “dosage“ -
that is, a number of products of the company “X“ makes it possible to obtain the desired effect
at lower dosages.
The analysis shows that:
product quality indicators of both companies are on the same level
(the conclusion was made on the basis of a study by specialists of the company “X” and a survey
customers).

the range of products is completely identical, moreover
promising additives - polycarboxylates - none of the studied
companies.
Key market indicators - production of concrete products, ready-mixed concrete and cement
As of the end of 2012, they are showing steady growth.
The market share of the main competitor of the company is growing steadily.
The market share of company “X” is steadily declining.
As stated above, the only significant difference is
market activity of companies: the number of managers and branches of company “X”
less than 6 times than its main competitor.
The problem is that you need to align the number of branches and
managers between competitors, for which you need to increase staff and number
affiliates about 5 times. However, this is impossible due to lack of funds, and
gap in financial capabilities of company “X” and its main competitor
constantly growing.
A key contradiction arises between the need to increase the number of
branches and managers and the inability to do this due to lack of funds:
Fig. 20. A pair of TP in this task.
Working TP isolated from fig. 18: affiliates and managers should be
much to provide the required consumer coverage, but it will require the company
investing a significant amount of cash, which is unacceptable.
Comment: The branch in this task is not just representation
company. The peculiarity of this business is that most
consumers prefers to receive the supplement in liquid form, however, the properties
liquid additives are quickly lost, so each branch is a
a small production site with an additive dilution unit. One such
the node is able to reliably supply consumers in a radius of about 250 km, not

more. Of course, one node is not enough - you need to have staff
managers and technologists involved in the sale and subsequent implementation
additives at the enterprise of customers. Therefore, each branch should be endowed
functions of production and product promotion. It turns out to work with
consumers, we need branches with production functions and managers,
carrying out the function of sale. But to keep up with a competitor, you need
significant cash .
Operational Zone (OZ). The conflict area in this task is between
a large number of branches and managers of the company “X“ and a small number
branches and managers of the company “Y“.
In fact, a conflict arises between great financial opportunities
company “Y” and small opportunities of company “X”, processing the market
(of the same consumers).
Fig. 21. The operational area (OZ).
OZ Resources :
Tool: market supply of companies (preliminary analysis
revealed the identity of the proposals of both companies)
Product: final consumers (precast concrete plants and manufacturers of commodity
concrete), traders (resellers).
Let's try to transfer the branch function to other elements of the system, first
the queue - to the health resources.
RBI-1 to OZ resources: The final consumer carries the functions
branch.
The customer has already tried this strategy - but enterprises with
a competitor’s branch, they are not going to bear the costs for the performance of this function -
Largely due to this market strategy, business X collapses, although products
Company Y
Company X
Final
Consumers
Traders

identical and even somewhat superior in stability to the properties of a competitor, about which
said above. This is a dead end.
RBI-1: The trader himself carries the functions of a branch. The customer objects to
of this formulation of RBIs: a trader is only a reseller, a reseller. He doesn't care
what to make money on. Nevertheless, we recorded that a trader is a possible
source of funding, however, like the plant. But there is a difference between them -
the trader needs to do business in the region with numerous customers, and the plant -
have the most convenient supplier.
RBI-2: Traders, working on large logistics sites throughout
Territories of the Russian Federation themselves ensure the impossibility of further development of a competitor
company “X“.
Then the problem arose in a new formulation: how to make it so that
more Y will open branches and the more hiring managers, the company X
will be better? - that is, at this stage the acceptance of TP authorization “surfaced”
No. 22 - to turn harm in favor.
In general, such a statement of the problem removes the fear that the competitor will repeat
the decision of the company “X“. After all, the more it repeats, the more it will “stoke“
by myself.
From the course of strategic marketing it is known that such a task
is interpreted by the expression “you need to find a competitor’s weakness in his strength“, i.e. what
need to do something that Y can't repeat? (in other words, cannot
abandon his own strategy and therefore will “drown“ himself).
And here the RBI “works”: the trader himself carries the functions of a branch .
Solution Idea:
You need to turn the trader (reseller) into a full-fledged distributor,
for whom technological customer support is part of his business, and
creation of a logistics platform - an object of investment of a partner of company “X” with
predicted payback period.
To implement this solution, company X will need to guarantee
exclusive to one or two partners in the specified territory for the billing period
time (the question now comes down to calculating the payback period plus - the time when
distributor makes a profit; this period is, after all, a subject
negotiations of the parties);
Provide training to distributor technologists, transfer to distributor
technology to work with end customers, to provide an opportunity to represent interests
enterprises in a given territory (franchise).
Verification of the solution found.
Is the original contradiction allowed? Yes. Even a superficial analysis
shows that funds are required several times less than for opening
own branches, while the number of managers at the traders is enough for

elaboration of category B and part C clients. Category A clients may
worked out by the company “X“ with the subsequent transfer to the trader.
Verification of the decision on the impossibility of repetition by the main competitor:
The competitor has already opened more than 30 branches - in all significant regions. For
repeating the strategy he will either have to close the branches in which they are invested
funds, or compete with your own distributors in
regions! Distributors will not accept such conditions. Therefore, the implementation of this
a strategy for a competitor of company “X“ will be difficult - this is the weakness
competitor, concluded in his strength. The decision was received completely on organizational-
managerial level, as required by the customer at the stage of setting the task.
(!) It is interesting that before solving the problem using TRIZ company
periodically raised the task of developing traders in order to increase
sales. However, the company's specialists have not previously seen a tool in traders
resolving this contradiction and did not try to develop a strategy
development of a full distribution network. Just earlier management
the company did not set a task like that .

APPENDIX 5
Cross-cutting case demonstrating joint use
schematization and work with contradictions as proposed by the author
schemes.
A system consisting of:
sales department of an industrial enterprise,
manufacturing tooling from heat-resistant steels, presented
head of sales, sales staff and the current system
sales (Fig. 12).
The essence of the problem: the head of sales (ROP) implements a new sales system,
having advantages over the previous one in terms of depth of study
customers and, as a result, allowing to increase the average amount of the contract and
conversion, however, managers resist and are in no hurry to leave the “beaten”
rails. “
Required: to make managers use only the tools of the new system
sales in their activities (as the task sounded in the original formulation, that in
the process of analyzing the circuit turned out to be not quite the correct goal setting - see table).
Below, we compose a model of a functioning system, presented in the form of a diagram.
The process of constructing a circuit for this task is described above (see explanations for Fig. 6):
Fig. 12 (repetition). Scheme of an inventive situation in the problem of changing the system
sales.

The tasks set according to the scheme (Fig. 12) using the categories of schematization:
No.
Object of analysis in
IFS
Tasks
No.
Task
one
interaction
the system
(dotted line) with
elements
supersystems
1.1.
CRM system. The conflict arose largely due to the fact that the existing CRM system
not adapted to the requirements of the new sales system, which creates significant
inconvenience → how to make the CRM system meet the requirements
new sales system and supported it?
1.2
end-to-end business processes. The new sales system is changing end-to-end business processes,
this is especially true in collaboration with the design department and
production → you need to configure end-to-end processes so that
The requirements of the new sales system were provided .
1.3.
customers. The new sales system increases the time to contact
by the customer → how to make the depth of the customers' work increase without
increase time managers?
2
Layers
2.1
The implemented sales system manages the actions of managers, imposing on them
specific requirements → How to make sales system requirements
performed, but did the managers spend as little effort as possible?
2.2
Managers are faced with the fact that for a number of clients the requirements of the new system
redundant, which does not increase, but rather reduces efficiency ( from this point of view
managers “manage“ the reaction of customers, hence the distribution of layers on
scheme ) → Differentiate customers and introduce a new sales system
only in relation to such client groups in which increase is expected
conversion and average transaction weight when using this system .
3
Communications
Partially analyzed in paragraphs 1 and 2, additionally:
3.1
Logical conflict between two systems, for example, the approach is completely changing
to identify needs, the stages of the transaction are radically different →
Compare the requirements of the existing and new systems, identify areas
similarities and cardinal discrepancies, disassemble into elementary steps of the area
cardinal differences, thereby simplifying the implementation (such a statement of the problem
allows the solver to rely on existing resources).
3.2
Communication defects ROP managers → Define metrics and reference points in
the new sales system, which should be feedback from
manager to leader. Simplify data retrieval by managers
reference points.
3.3
Establish a relationship CRM-system - ROP → Having solved the tasks 3.2, bring the CRM-system in
in accordance with the decisions received, make appropriate changes to
order of meetings, strengthening communication on reference points and
reducing communication on irrelevant moments.
four
Processes and
the functions
4.1
The task appeared after setting the task 1.2: to conduct a detailed analysis of business
processes between the sales department and the design department, as well as between
sales department and production department (pre-mapping
processes using BPMN notation) . Highlight bottlenecks and set targets for
to overcome them.
4.2.
After solving task 1.1, set the task to simplify the introduction of the required
data into the CRM system by entering patterns and rules.
five
Groups
5.1
Negative phenomena within a group of managers - the effect of the adoption of new
technologies according to the model of J. Moore → how to use innovators and early
followers as a resource to introduce a new sales system? how
to identify and neutralize the influence of “farther”?
5.2.
Customer groups, which follows from the analysis of the task 2.2. Carry out customer separation
on categories A, B and C. Define customer categories and target customer groups,
for which the new sales system is redundant. Set a task for synchronization
the work of the department, which should apply both sales systems, if a hypothesis
will be confirmed and the existing sales system will be advisable to maintain
for certain groups of customers amid the introduction of a new one.
6
Places and material 6.1.
Provide training for the “good middle peasants” of the new sales system after the decision
tasks from pp 1-5 and determine whether they have reached the level of “stars“ after a given time.
If not, conduct a comparative analysis of the work of those and others and conduct further training
“Good middle peasants” according to the performance model (performance model
explains exactly what competencies make stars stars by comparing them
competencies with the competencies of “good average“ in the team and identifying
discrepancies) .

Next, we divide the subtasks into elementary actions that are necessary
implement on the project. If necessary, we indicate the tools that should be used in
further process this task. Also create unwanted effects
and contradictions, if any, arise during the course of the project.
No.
ass
aci
The task
No.
ass
en
and
I
Task for execution /
further analysis
Secondary NE /
TP (draft)
1.1
How to make CRM-
the system matched
new system requirements
sales and supported her?
1.1.1
Carry out ABC analysis and describe portraits
customers by category in each channel
1.1.2. Organize a multi-funnel in CRM (process
and resultant) in money and quantities
NE: Manager mistakes during
funnel selection
1.1.3
Provide the ability to assign to one
deal several counterparties indicating
status and affiliation of the transaction in the card
customer
1.1.4
Provide the ability to assign
categories to deal and counterparty
1.1.5
Customize process and result reports
funnel according to TK
1.2
End-to-end configuration required
processes so that
new system requirements
sales were secured
1.2.1
Describe existing business processes
between designated departments in
BPMN notations and highlight points
inconsistencies of the current process with
requirements of the new sales system (after
solutions to problem 1.2.3) *
1.3
How to make depth
customer research
increased without increasing
time-consuming managers?
1.3.1
Develop standard decision making schemes
for priority sales channels indicating
customer entry points and tactics
categories A and B
NE: With 5 priority
channels - this is at least 10
circuits that still need
identify correctly.
1.3.2
Develop channel matrices
TP: many hypotheses
needs - more chance
create UTP but difficult
keep in mind
(need to reproduce
quickly in the conversation)
1.3.3
Develop rules for providing bonuses
to customers
TP: many bonuses –large
probability of getting into
need but choose
complicated.
1.3.4
Develop a base of typical questions at work
at different stages of the funnel
TP: a lot of questions - more
chance to create UTP but quickly
we will tire the client
2.1
How to make
sales system requirements
performed but managers
spent as little as possible
effort?
2.1.1
Provide CRM system with tips
2.1.2 CRM system itself pulls data from
customer portraits by average
2.1.3
Provide a description of the business process
manager for each stage
process and result funnel, highlight
areas of greatest time loss and
set tasks to eliminate them.
2.2
Differentiate
customers and introduce a new
sales system only
relation to such client
groups expected
increase conversion and
average deal weight at
application of this system
2.2.1
Retain existing sales system for
category C customers. Introduce a new system
sales only for customers of categories A and B.
TP: a contradiction in the choice
sales department work schemes
- both systems are implemented
all employees, or
differentiation is carried out?

3.1
Compare requirements
existing and new systems,
identify areas of similarity and
cardinal discrepancy,
disassemble on
elementary area steps
cardinal differences
simplifying implementation
Partially solved in solving problems 1.3.2 -
1.3.4
3.1.1
The difference in actions at the stage of evaluating options:
create a list of typical selection criteria in
channels for target centers
decision-making (supplement the scheme with
solving problem 1.3.1)
NE: Managers forget
perform actions at the stage
evaluation options let
gravity process
3.1.2
Difference of actions: a new item was added -
economic justification. Give examples
(cases) payback calculations that
manager can use in preparation
commercial offers.
3.1.3
Stage work with objections replaced by stage
“Resolution of doubt“, which causes
difficulties. Describe the background
doubt, manifestation
doubt, train to work with doubt.
NE: Managers often
miss occurrence
doubt centers acceptance
decisions even if
have been trained.
3.2
Define metrics and
reference points in the new
sales system by which
reverse should be carried out
communication from manager to
to the leader. Simplify
receiving data
reference managers
points.
3.2.1
Develop quantitative indicators in
funnel in the process and resultant funnels
3.2.2
Develop quality indicators in
sales funnel
NE: Difficulty of control
quality indicators in
sales funnel
3.2.3. Suggest application control options
data to the CRM system by managers
daily ( e-commerce facilities in
no company )
3.2.4
To provide for the formation in the CRM system
Lead from email manager and site
companies provide automatic accounting
leads.
NE: The risk of appearing in the database
data of the same
client under different
names
3.3
Make appropriate
changes in order
meetings reinforcing
reference communication
points and reducing
communication on
irrelevant moments
3.3.1
Create report in CRM-system “movement in
funnel for 1 week “for each
manager
4.1
Conduct a detailed analysis
business processes between
sales department and
design department, and
also between sales and
production department
Included in task 1.2.1
4.2
Set a task to simplify
entering the required data into
CRM system by entering patterns and
regulations
Tasks 1.1.1 - 1.1.5 and a number of other tasks
4.2.1
After completing tasks 1.1.1; 1.3.1-1.3.4
integrate these documents with
CRM system
5.1
How to use innovators and
early followers in
as a resource for implementation
new sales system? how
identify and
neutralize influence
“Bumpier”?
5.1.1
Create a report in the CRM system for use
its capabilities (which entities are actively
are used).
5.1.2
Provide a correlation report
Entity Use - Conversion -
average transaction weight - gross margin per month
(use data when conducting
meetings)
5.2
Set a task for
department work synchronization,
which should apply both
sales systems if hypothesis
confirmed and existing
the sales system will be
advisable to save for
Task 2.2.1

specific customer groups
amid the introduction of a new
6.1
Provide training for “good
middle peasants “new system
sales after solving problems from
pp 1-5 and determine if achieved
whether they are the level of “stars“ through
set time. If not,
conduct a comparative analysis
the work of both and spend
retraining “good
middle peasants “according to the model
performance
6.1.1
Create a performance model indicating
weighting factors on
positive deal outcome
TP: When using
no theoretical data
takes into account the specifics
enterprises but accounted for
new system requirements
sales, and if you take the data
by the enterprise then they
match the specifics, but
accumulated by
the requirements of the previous
sales systems
6.1.2
Observe the work of the “stars“ and
refine the performance model
TP: If you watch for a long time, then
get an adequate model
performance but learn by
it will not work right away. BUT
need to be taught immediately, otherwise
the system will not work.
6.1.3
Create a training program /
mentoring using data
obtained in the course of solving tasks 6.1.1 - 6.1.2
TP: mentoring program
must implement
best practice staff
but at the same time they come off
from core business
* in bold the table indicates tasks that require further analytical
work.
And there are 28 tasks for execution, of which 26 can be put to execution,
if you remove the selected unwanted effects (6) and resolve the contradictions,
formed at this stage in draft form (7), after which you can
finally formulate SMART execution tasks [14] and implement
planning, for example, using SCRUM technology [26]. That is, proceed to
“Managerial“ part of the project.
2 more tasks require further analysis, for example, using notations
descriptions of business processes [34].
We highlight the contradictions in graphical form that we encountered (8 contradictions
arose at the stage of formulating tasks for execution, another 7 are secondary
unwanted effects that occur when trying to fulfill the set
tasks (transferring the remedy to the system). Execution tasks not containing
contradictions, are not included in the table, they will be fixed on the task board [26] and go
to work.
If the resolution of the contradiction is obvious, write it to the right in the table. If not, then
We expose the selected TPs to further analysis.
Tasks 1.2.1 and 2.1.3 will be further analyzed, as indicated above,
after which TRIZ tools can also be applied to them.

No.
TP in graphical form
Authorization / Next Steps
1.1.2
-
one
+
When setting the trade category marker “C”
in the CRM system it is automatically offered
process funnel if marker is installed
“A” and “B” are the result.
Requirements taken into account
two systems
of sales
Funnel in CRM
Errors in
choosing
funnels
+
2
-
1.3.1
-
universal
+
It is solved when performing task 2.1.2, so
how typical schemes are provided
manager in the form of a hint CRM system
when setting a deal marker and channel
sales. Solution: make the channel marker
sales “active field.
Accuracy of description
the situation
Decision making schemes
Complexity
identification
scheme
+
Broken by channels and
categories
-
1.3.2
+
Lot
-
Break down hypotheses by sales channels and
decision centers. In preparation
to the meeting use only the target
matrix, pre-selecting ~ 5
key hypotheses (empirical figure).
The target matrix is ​​output by analogy with
solved problem 1.3.1
Brighter and
reasoned
USP (unique
trade
sentence)
Hypotheses of needs
customer
Play
from memory during
dialogue with
by customer
-
Few
+
1.3.3
+
Many options
-
Bonus classification matrix: row by row
- classification by groups (financial,
logistics, consulting ...), by
columns - classification by authority:
Manager, Head of Sales
CEO.
Probability
to get in
need
Bonuses to customers
Complexity
of choice
_
Few options
+
1.3.4
+
Lot
-
The contradiction is resolved by mentoring -
you need not to remember questions, but to learn
design them quickly with
using typical hypotheses
needs and criteria.
Brighter and
reasoned
USP (unique
trade
sentence)
Customer questions
Play
from memory during
dialogue with
by customer
-
Few
+
2.2.1
+
Differentiated by
sales patterns
-
We try to solve through analysis
TOC contradictions [12]
Optimal
using
qualifications
Department staff
of sales
amount
supported by
standards
-
All employees
apply both schemes
+
3.1.1
-
Single stage of work with
needs
+
Set marker in CRM system,
indicating the duration of stay on
each stage of the funnel depending on
correlation of the stage with the category of the transaction. At
exceeding the limit of being at the stage
“Need recognition“ marker
signals manager about high
the probability of a change in stages.
Localization
work with
issues and
criteria
of choice
Sales funnel
Ease
identification
stages
+
Work with needs
beaten in 2 stages
-
3.1.3
+
Stage “permissions
doubt “
-
Let's try to solve using RBI and
resources [11] (previously acted on a hunch,
need to find reliable and easy
decision to identify doubts
Decision Centers)
Conformity
project realities
of sales
Sales funnel
Complexity
identification
-
Stage “work with
objections
+
3.2.2
-
Only quantitative
+
Bring quality indicators to
clear digital form (result -
identified at least 3 needs
leading to benefits not highlighted
less than 3 criteria leading to
benefits, etc.). Provide for
CRM-system markers, allowing to conduct
counting such indicators and display them in
special field in the transaction passport.
Adjustment
transaction movements
→ conversion
Funnel indicators
of sales
Complexity
extract
of information
+
Quantitative and
quality
-

3.2.4
-
Manually
+
Lida approves in the CRM system only
marketing assistant. Also
assistant distributes leads for
further elaboration → required to enter
in a CRM system, a marker for the appearance of a lead,
notifying employee to whom
attached lead assistant department
marketing.
Lead Preservation
(guarantee that lead
not lost)
Leading in CRM-
the system
Opportunity
appearance in the database
defective
records
+
Using
automatic services
-
6.1.1
+
Prepared from
using
experience
-
We try to solve through analysis
TOC contradictions [12]
Using
available data
on application
the system
Model
performance
Conformity
programs
learning
requirements
new system
-
Prepared from scratch
+
6.1.2
+
Based on
current understanding
-
We try to solve through analysis
TOC contradictions [12]
Preparation time
models (launch in
work)
Model
performance
Efficiency
learning
(impact
strictly to the point
growth)
-
Based on
form matching
the behavior of the middle peasants
with the stars
+
6.1.3
+
“Serednyachki“ with experience
from 1 year
-
We try to solve through analysis
TOC contradictions [12]
Time of the best
employees on
performance
main
the activities
Who is conducting
training / mentoring
“Concentrate“
practical
skills
-
Best practics
+
So, straight away we managed to find solutions to 8 contradictions out of 13. The remaining 5 contradictions
We will allow by other means.
Resolution of the contradiction 2.2.1 (solved through analysis of TOC):
Fig. 22. Analysis of a pair of TP 2.2.1
Differentiated by
sales patterns
Number of supported
standards
Department staff
of sales
Apply both schemes
(no differentiation)
Optimal use
employee qualifications
+
-
+
-
There is no transition of qualified
less skilled employees
work and vice versa
You need to have only one circuit
training new employees

The branch “how to make employees differentiated by sales technology,
but at the same time we had only one scheme of employee training ”can be seen as
promising.
Solution : for the duration of the internship, the employee must be trained according to the program of work with
category C customers, and then, upon detection of potential, it can be transferred
to work with category A and B clients after retraining. In doing so, we
we get a single training system for sales staff, but differentiation by
sales patterns saved.
Super effect : gradual manifestation of employee potential, additional
motivation by complicating tasks with increasing income, reducing the number of unsuccessful
personnel decisions.
Resolution 3.1.3 (use resources and RBIs):
Fig. 23. The choice of working TP from a pair of TP 3.1.3.
Resources:
Stage of “resolving doubts”
(tool)
The process of identifying an “item”
processing (doubt) “- product
Approach to the contract
Personal communication with the adoption center
solutions
New large customer
Information from other stakeholders
Characteristics of the personality of the adoption center
solutions
Experience in similar transactions
Top Management Attention
Behavior Center Analysis
solutions
RBI rule:
The X-element itself provides an unmistakable identification of customer doubts (item
processing), significantly affecting the outcome of the transaction.
As a result of the substitution of resources in the RBI formula, we obtain the following:
1. When approaching the conclusion of a contract in the case of work with large transactions
the head of sales (ROP) takes the transaction under personal control;
Includes Stage
“Resolution of doubt”
Identification difficulty
“Subject of processing” (doubts
hard to detect)
Sales funnel
Includes the “work with
objections
Compliance with realities
project sales
+
-
+
-

2. If the client has not worked with us before, a major transaction is planned,
we enter information about the personal qualities of the counterparty in the contact passport - type
individuals by DISC topology (or use the enneatype model by topology
Adizes-Madanes), interests, features of behavior. For briefing
managers detailed cases, examples are shown.
3. If the transaction is large, then at the stage of recognition of needs we study in detail
requirements of decision makers on this transaction. Special attention
we turn to this point if the transaction is significantly larger than those that were made with
this client earlier or the client switched from a competitor. Further, when
promoting the transaction, we fix deviations from the above requirements identified
deviations and will be reliable indicators of doubt.
4. If personal communication with the decision center is interrupted for a while
for objective reasons (the transaction goes to the stage where work
carried out with other decision centers (DPC), periodically
conduct personal communication with the CPR under various pretexts. Periodicity
contacts - at least 1 time per month.
5. Maintain information with persons influencing the decision and record
deviations from the current trajectory. In case of detection
significant deviations come in contact with the DPC.
6. Use the experience of similar transactions - on an extended monthly basis.
retrospective analysis of sales
to study the issue of identifying doubts arising in the most
material transactions.
Resolution of the contradiction 6.1.1.
A cursory analysis of the contradiction led to the understanding that express analysis by TOC is unlikely
will give a quality result. Therefore, we will resolve this contradiction with
allocation of resources of the operational area and the application of RBI.
Fig. 24. The choice of working TP from a pair of TP 6.1.1
Prepared from
using
experience gained
Program compliance
learning the requirements of the new
the system
Model
performance
Prepared “from a clean
sheet “
Using accumulated
statistics
+
-
+
-

Resources:
Performance Model Preparation
“From scratch” - a tool
Using accumulated statistics
- product
Specification of behaviors
overlapping with existing
sales system
Accumulated Cases
Unique Form Specification
conduct
Cross-channel conversion data
of sales
Manager Reports
Documented Results
meetings
RBI rule:
X-element itself ensures the use of accumulated statistics in the model
performance designed for the new sales system.
As a result of the substitution of resources in the RBI formula, we obtain the following:
1. Describe behaviors in the performance model in relation to the new
sales system and highlight those forms of behavior that are similar to the previous
system. Use existing cases and accumulated experience when working out
“Intersecting” forms of behavior;
2. Analyze in channels where the conversion was above the average for transactions
categories A and B. Use the experience in these transactions as a reference
to verify the adequacy of the performance model created for the new
sales systems.
3. To analyze in channels where the conversion was below the average for category transactions
A and B. Put a thought experiment in the application of the performance model for
a new sales system for such transactions. Mark those forms
Behaviors experienced by experienced staff could correct the situation.
When training employees, pay special attention to the highlighted forms
behavior.
4. To check the adequacy of the conclusions in paragraphs 2 and 3, use the reports of managers and
meeting minutes created during the development of analyzed transactions.
5. Use data from manager reports and meeting minutes to
case studies when teaching selected forms of behavior (if the form
new behavior, then as a case, you can bring a situation where the presence
this form of behavior could have a qualitative impact on successful
transaction outcome).

Resolution of the contradiction 6.1.2.
Fig. 25. Analysis of TP 6.1.2 pair and application of RBIs.
The application of RBI suggests itself:
RBI 6.1.2-1: the X-element itself ensures the selection of forms that are “not enough” by the middle peasants
behavior without a comparative analysis of the work of the “average“ and “stars“.
Solution : analysis of the situation by the “stars“ themselves: if the “stars“ hold a series of meetings
together with the “middle peasants“, having on hand a common list of forms of behavior, then spending
subsequent reflection with the “stars“ is easy to catch the difference. This operation is not
will require more than 1 week.
Based on
current understanding
Learning Effectiveness
(we act strictly on
growth points)
Model
performance
Based on
job comparisons
“Average“ and “stars“
Model preparation time
+
-
+
-
Do not spend time on comparative
analysis of the work of the “average“ and “stars“ by
behaviors
We are not wasting development efforts
behaviors that
“Average“ and so good
reproduce
RBI

Resolution of the contradiction 6.1.3.
Fig. 26. Analysis of TP pair 6.1.3.
Both branches are recognized as dead ends, no solution found.
We try through resources and RBI:
Fig. 27. The choice of working TP from a pair of TP 6.1.3
Serednyachki with experience from
1 year
Practical concentrate
skills
Who is conducting
training
Employees with experience from 5
years having high
results
Time best employees on
basic tasks
+
-
+
-
Top employees excluded from the process
mentoring
Already shallow all that does not give
steady result
“Serednyachki“ with experience from
1 year
Practical concentrate
skills
Who is conducting
training
Employees with experience from 5
years having high
results
Time best employees on
basic tasks
+
-
+
-

Resources:
Experienced staff → mentors -
tool
Time
experienced
employees
on
basic tasks - product
Project Sales Experience
Meeting Planning Time
Company product knowledge
Time to analyze transaction information
Accumulated customer base
Time to contact customer
Fame in the professional environment,
reputation
Time to prepare reports in CRM-
the system
Expertise in the client’s business
Time for lunch, rest during working hours
of the day
Psychological competence
RBI rule:
The X-element itself provides an exception to the time spent by the best employees on
training / mentoring.
As a result of substituting resources into the RBI formula, we obtain the following (we see that with
the RBI data better manages the resource of the product):
1. Adaptation of a new employee is divided into 2 parts: introductory course new employee
passes with the employee who conducts the initial introduction to the course of affairs and
answers some of the beginner's questions. The introductory course is conducted by an employee who has
1 year work experience and stable sales results. After passing
Introductory course, the new employee falls into the second stage of adaptation.
2. The second stage of adaptation is the “shadow”. That's what technology is called when new
the employee is trying to copy the actions of the wizard, in the course of playing it
action he is faced with a lot of obscure moments that
should fix in the form of questions. Then prepared questions new
the employee asks the mentor. The same technology can be used during
contacting an experienced employee with a client.
3. This is where the practice entrenched in a number of companies comes from - a progress diary
new employee. That is, the new employee does not record anywhere, but in
paper or electronic workbook with pre-prepared fields. So
information is better structured and amenable to subsequent analysis.
4. After successfully completing the second part of the adaptation, an experienced employee
allocates time in his schedule for “polishing” the skills of a new employee in
the process of monitoring its preparation for responsible meetings and contacts with
by the customer.
Those. to completely remove experienced employees from the mentoring process is irrational,
but it is possible to reduce their time in the process of mentoring by 2-3 times without reducing
process efficiency. The contradiction is partially resolved.
Based on the obtained solutions, the task manager developed an implementation program
new sales system, including a step-by-step action plan drawn up in the environment
www.trello.com using SCRUM technology [26]. Currently going
the process of implementing this program of activities .

APPENDIX 6
General algorithm of work with organizational and managerial
tasks indicating the most commonly used tools.
When solving organizational and managerial tasks, we applied the most
various tools from the TRIZ arsenal [11]. Some tools from the arsenal
TRIZ are used in organizational and management tasks without any
methodological completion, and some had to be converted to requirements
“Soft” systems, primarily, for the requirements of business systems [2, 11].
In fig. 28 presents a simplified scheme of using TRIZ for solving business
tasks used by the author [46]. To describe the groups of tools, we introduce two important
concepts: product and result [10]. These terms are widely used in modern
management. A product is something that can be transferred to the next stage of work with a task,
that is, the “exit“ of the instrument. The result is a refined understanding of the system or
process (analysis result).
Most often, such a scheme is used not so much to solve situational problems,
how much when performing projects in business systems:
Fig. 28. General scheme for the use of TRIZ tools in the implementation of projects in
organizational systems.

So, the work on the task consists of five blocks (Fig. 28):
\item Formulation and formalization of the task;
\item Initial processing of the task;
\item Highlighting a list of key contradictions;
\item Decisive mechanisms;
\item Verification of the decisions received.
We will reveal these mechanisms in more detail.
1. Formulation and formalization of the problem (steps 1–5, Fig. 28).
Process: communication in a team of professionals looking for a solution, designed
to understand the conditions of the problem.
The result: an understanding of the system’s design — what are the key elements
the composition of the system and how the most significant relationships between them are organized, as well as
what paths (step 5) should further transform the system.
Product: A set of unwanted effects that make up the problem situation.
(step 4) and the trajectories of further work with the task (step 5).
2. Initial processing of the task (steps 6–10, Fig. 28).
In fig. 1 not all the tools that we use in TRIZ are presented,
but only the most commonly used.
Process: we analyze the cause-effect relationships, parameters and structure
system, determine the principles of its work.
Result: a deeper understanding of the connections in the system, a description is made
elements and their key parameters.
Product: key unwanted effects, secondary tasks, solution ideas.
Some ideas might appear at the previous stage, but when using
there are much more tools from this block.
3. Highlighting a list of key contradictions (step 11, Fig. 28).
Process: we determine the parameters that conflict with each other.
Highlight secondary adverse effects associated with counteraction
system. We are building contradictions.
Result: sharpening the task to the limit, which allows you to see the nodal points
tasks, to cut off all unnecessary - “white noise”, which is always in large quantities
accompanies organizational and managerial tasks.
Product: list of contradictions.
4. Decisive mechanisms (steps 12–13, Fig. 28).
There are several such mechanisms, most often we use two of them.
(steps 12 and 13).
Note that the process of selecting an operational zone and determining resources within
The operational area is part of step 12.
Process: using conflict resolution algorithms, we find ideas for
solutions to the problem.
The result and the product here coincide: ideas for solving the problem and tasks of the third
order. Such problems arise if the solutions found require further
analysis (then back to steps 3, 7-10) or come across a significant
system resistance (then go to step 11 and allow new
contradictions).

5. Verification of the obtained solutions (step 15, Fig. 28).
Process: we analyze the received decisions, we compare them with the goals of customers
and key stakeholders. We pass a verdict: are solutions suitable for
introductions? If the answer is yes, then we proceed to the verification by the ratio
benefits to costs. If the benefits are incomparably greater than the cost of solving this
tasks (which happens, of course, not always), then proceed to planning
implementation, if not - go back to the beginning of the algorithm and select
additional tasks.
Result: understanding the next steps: we move on to planning implementation
or set new tasks.
Product: solutions and secondary tasks.
Now a little about the “parking“ (step 14). “Parking“ we call the place where
“Stored” ideas found in the process of working on a task (of course, parking
- this is not our invention, G.S. Altshuller proposed solutions
labeled as “GI” is a brilliant idea). Ideas can come up on a variety of
stages of processing the task, starting from step 3 to step 15 inclusive. Their important
“park” on time and divide into categories so as not to lose and subsequently
translate them into an action plan for the implementation of ideas received during implementation
project.
In fig. 28 shows a general scheme for using TRIZ tools in projects,
implemented in organizational systems. Of course, in practice it’s not necessary
All tools depicted in this diagram apply. Instruments
used selectively. So, in the example from Appendix 5, primary tools
task processing was not used at all, but decomposition was applied
secondary tasks obtained by analyzing the circuit in Fig. 12. The decision process
tasks remains largely a creative process and, unfortunately, still
insufficiently formalized and largely depends on experience, competencies and
psychological features of the solver.

SOURCES
1. Sosnin E.A., Poyzner B.N. Social Engineering Workbook
(interdisciplinary project) - Tomsk: Tomsk University Press,
2001.
2. V. Soushkov. TRIZ and Systematic Business Model Innovation, 2010. Bergamo
University Press. ISBN: 978-88-9633359-4
3. Technology of creative thinking / Mark Meerovich, Larisa Shragina. - M .:
Alpina Business Books, 2008.
4. Korolev V.A. Fundamentals of the system-process theory of convenience and
the life of organizations. Manuscript deposited in TRIZ fund
CHOUNB.
5. Korolev V.A. Modeling of social objects, 2004. Manuscript
deposited in the TRIZ CHOUNB fund.
6. Examples of the application of TRIZ to organizational and managerial tasks:
http://bmtriz.ru/articles/categories/1/
7. Fundamentals of the general theory of strong thinking, Khomenko NN, 1997. Manuscript
deposited in the TRIZ CHOUNB fund.
8. Shmakov BV, Schepetov EG Compliance of technical and
social systems, Chelyabinsk State University.
9. D. Mann. Hands-on Systematic Innovation for Business and Management “, IFR
Press, 2004. ISBN 1898546738
10. Reus A.G., Zinchenko A.P. Guide to Organization Methodology,
Leadership and Management / Readings on the work of G.P. Shchedrovitsky
M .: Alpina Publisher, 2012.
11. TRIZ: Solving business problems / A. Kozhemyako. - M .: Synergy University,
2017.
12. D. Mann. Physical Contradictions and Evaporating Clouds (Case Study
Applications of TRIZ and the Theory of Constraints), 2000
13. Adizes I.K. How to overcome management crises. Diagnostics and solution
managerial problems. - M .: Mann, Ivanov and Ferber, 2014.
14. Use SMART goals to launch management by objectives plan:
https://www.techrepublic.com/article/use-smart-goals-to-launch-management-by-
objectives-plan /
15. TRIZ. Analysis of technical information and the generation of new ideas: educational
allowance / N.A. Shpakovsky. - M.: Forum, 2010.
16. A. Kozhemyako. The era of smart sales in the B2B market. How to audit
commercial service on your own and break away from competitors, 2016
(electronic edition:
http://bmtriz.ru/anton_kozhemyako_era_umnyh_prodazh_kak_provesti_audit_ko
mmercheskoy_sluzhby_svoimi_silami_i_otorvatsya_ot_konkurentov /)
17. N. Feigenson. Advanced functional approach in TRIZ.
Report at the Scientific Conference “TRIZ. Application practice
methodological tools. “ Moscow, 2017
18. Kudryavtsev A.V. TRIZ - tools for creating innovation for development
enterprises. Textbook, 2013.
19. G.S. Altshuller, B.L. Zlotin et al. Search for new ideas: from insight to
technology. - Chisinau: Cartya Moldavanske, 1989.
20. O. Cowan, E. Fedurko. Fundamentals of the theory of constraints. Library
strategic decisions of CBT, 2012. ISBN 978-9949-9148-1-4

21. Discovering the organization of the future / F. Lalu. - M .: Mann, Ivanov and Ferber,
2016.
22. Yu.P. Salamatov, I.M. Kondrakov. Evolution model of technical systems,
1986. The manuscript was deposited in the TRIZ CHOUNB fund.
23. Management / Drucker Peter F., Macjarello Joseph A. - LLC
“I.D. Williams“, 2011.
24. Willpower. How to develop and strengthen / Kelly McGonigal; per. from English Ksenia
Chistopol. - 2nd ed. - M .: Mann, Ivanov and Ferber, 2013.
25. Find an idea: Introduction to TRIZ - the theory of solving inventive problems /
Heinrich Altshuller. - 3rd ed. - M .: Alpina Publishers, 2010.
26. Scrum. Revolutionary Project Management Method / Jeff Sutherland;
per. from English M. Geskina - M .: Mann, Ivanov and Ferber, 2016.
27. The era of smart sales in the b2b market / A.P. Kozhemyako. - M.: Moscow
Synergy Financial and Industrial University, 2013.
28. The era of smart sales. Strategy and management / A.P. Kozhemyako. - M .:
Moscow Financial and Industrial University “Synergy“, 2013.
29. Psychological effects in management and marketing. 100+ directions
increasing efficiency in management / A.P. Kozhemyako. - M.: Moscow
Synergy Financial and Industrial University, 2015.
30. Eighth skill: From efficiency to greatness / Stephen R. Covey; Per. from English
- 4th ed. - M .: Alpina Publishers, 2009.
31. Abraham H. Maslow. Motivation and Personality (2nd ed.) NY: Harper & Row,
1970; St. Petersburg: Eurasia, 1999. Translation by A. M. Tatlybaeva.
32. Think like a designer. Design thinking for managers / J. Lidtka, T.
Ogilvy. - M .: Mann, Ivanov and Ferber, 2015.
33. Asana: The easiest way to manage team projects and tasks:
https://asana.com/product.
34. Business process modeling notations:
http://www.businessstudio.ru/products/business_studio/notations/
35. Scribing: description and tools:http://nitforyou.com/scribe/
36. New appointment. Kuryshev V.A., 1994
37. Methodological materials for the 45th author’s workshop N. Khomenko
“Modern intellectual technologies based on TRIZ”, Minsk
1996.
38. Anti-fragility. How to capitalize on chaos / N. N. Taleb. - M .:
Publishing Group “Alphabet-Atticus“, 2014.
39. V. Sushkov. Ideality in business, 2015
40. V.G. Siberians. Invention in business or “development” through
Contradictions, 1999
41. G.P. Shchedrovitsky. Organizational thinking: ideology, methodology,
technology (lecture course). - M.: Publishing house of the studio Artemy Lebedev,
2015 year
42. GLOSSARY OF TRIZ AND TRIZ-RELATED TERMS, VERSION 1.2. Valeri
Souchkov, The International TRIZ Association - MATRIZ, 2018.
43. A.V. Efimov. Methodology of MPV analysis, 2008
44. M.S. Ruby. TRIZ in small business - competitive handicap, 2004
45. Yu.G. Chernikov. System analysis and operations research. - M .:
Publishing House of Moscow State Mining University, 2006.
46. ​​A. Kozhemyako. A little bit about the systems thinking of the department head
sales. We apply system analysis. Sales Management, 03 (98),
2018 year

\end{document}
