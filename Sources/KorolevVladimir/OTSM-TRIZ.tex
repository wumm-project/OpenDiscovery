\documentclass[11pt,a4paper]{article}
\usepackage{a4wide,url,graphicx}
\usepackage[utf8]{inputenc}
\usepackage[russian]{babel}

\parindent0pt
\parskip3pt

\title{Аксиоматика ТРИЗ-ОТСМ} 

\author{Владимир Королёв}

\date{2019}

\begin{document}
\maketitle

\begin{quote}
  Source: \url{http://www.triz.org.ua/works/wx18.html}
\end{quote}

Квалификация специалиста по ТРИЗ предопределяется уровнем знания и понимания её аксиоматики. А она определена Альтшуллером Г.С. как диалектический материализм. Соответственно, развитие ТРИЗ возможно только как следствие развития науки. Отказ от этой основы означает переход от ТРИЗ к идеалистической теории. Без жёсткого следования собственной аксиоматике нельзя ни планомерно развивать ТРИЗ вместе с её приложениями с уходом от привычного метода проб и ошибок, ни расширять её применение на другие области деятельности человека, ни толком обучать людей. Статья разработана в обоснование соответствующего раздела Справочника терминов ТРИЗ-ОТСМ.

Предыдущие версии Аксиоматики с момента публикации Аксиоматики ТРИЗ-ОТСМ (8) утрачивают свою силу, сохраняя свою ценность только как предмет исторического интереса.

Начался ХХІ век, но вместо обещанного футурологами необычайного прогресса во всех сферах человеческого бытия мы наблюдаем глобальное помутнение умов и поворот к дремучему мракобесию.

А. Зиновьев

Теория решения задач

 

Каждому человеку приходится решать задачи. Много и разных. При этом он обычно руководствуется здравым смыслом.

А что такое «здравый смысл»? Это правильное применение опыта при решении задач.

А что такое «опыт»? Это усвоенные причинно-следственные зависимости («если – то»), отличающиеся от множества других «силой» или повторяемостью производимого впечатления. Первые обычно выглядят как утверждения (аксиомы), вторые – как обобщения (теории). Опыт бывает личный, но преимущественно – усвоенный обобщённый опыт (знания) множества людей.

А что такое «правильное применение»? Это умение (способность) преобразовывать беспорядочное множество частных случаев опыта в систему, которая соответствует требованиям задачи. И вот с этим-то совсем не просто.

Во-первых, у каждой причины обычно бывает более чем одно следствие, каждое из которых тоже становится причиной других следствий. Поэтому каждое действие человека порождает неограниченно растущее причинно-следственное «дерево». Чем больше опыт, тем больше следствий человек способен учесть. Но обычно знаний и времени хватает лишь на «кустик» из нескольких коротких «веточек». На большее мало кто способен от природы.

Во-вторых, на переход от причины к следствию всегда влияют обстоятельства. Они разные и переменные. Из-за чего и следствия одной и той же причины бывают разные. Поэтому чем больше и ветвистей «дерево», тем выше неопределённость итога рассуждений. Ведь вместо расчётов приходится полагаться на случайность, на «метод» проб и ошибок (МПиО). Справиться с неопределённостью в решении таких задач совсем уж мало кто способен от природы.

Хорошо ещё, что большинство задач во всех областях деятельности простые и решение их сводится к усвоению простых действий (и даже их технологических последовательностей) с жёстко предопределёнными следствиями. Со временем такие действия закономерно передаются технике и ЭВМ. Плохо, что такой работы для человека находится всё меньше.

Соответственно, задачи остающейся части становятся всё сложнее, затраты на поиск решения становятся всё больше, а цена ошибок становится всё выше. Неограниченно выше. Такие задачи принято называть творческими (ныне модно называть их креативными) из-за содержания в них практически бесконечной неопределённости. Поэтому уже давно стала очевидной (хотя и не всем) необходимость создания алгоритмов для таких задач. В конце концов, не за пять минут, а за сто лет, но люди как-то справляются с такими задачами. Хотя бы и посредством последовательного приближения с множеством ошибок.

Если отложить пока в сторону творческие задачи из области т.н. искусства (музыка и всё такое прочее), то наиболее важными для жизнедеятельности человека представляются наука, управление и техника (технология). Тем более, что между этими направлениями очень много общего. Но несмотря на множество попыток создать для них алгоритмы, дальше добрых советов дело почти не двигалось.

Прорыв наметился только в середине 20-го века в области инженерных творческих задач: была создана Теория решения изобретательских задач (ТРИЗ, Альтшуллер Г.С., СССР). Название обусловлено тем, что такие задачи часто называют изобретательскими, так как хорошее решение может быть признано изобретением.

Вообще говоря, первоначально ТРИЗ (далее эта аббревиатура будет применяться исключительно как теория, разработанная её автором; она же – классическая) была задумана как всеохватывающая теория. Но поскольку наиболее представительный документированный материал (патенты) для исследований существует только в технической области в виде патентов, постольку область применения теории оказалась ограниченной только техническими же задачами.

К сожалению, спешка с созданием прикладной части ТРИЗ (способы решения задач) привела к тому, что разработка её аксиоматики была отложена, оставшись только в виде самых общих ссылок на диалектический материализм (он же – научный подход) [2]. По этой причине оказались неудачными позднейшие попытки Альтшуллера Г.С и его соратников применить ТРИЗ в нетехнических областях. Хуже того, довольно скоро остановилось развитие собственно ТРИЗ. А застой положил начало деградации её и уже найденных алгоритмических способов (технологий) решения задач.

Выше упоминались «-измы», которые сразу лишают обычного человека всякого желания читать дальше и, тем более, разбираться. Но всё гораздо проще, чем кажется: любой человек обладает собственным набором принципов мировосприятия и поведения. Это-то и называется философией. Всё разнообразие философий можно разделить всего на две группы: материалистические (научные) и идеалистические (все остальные).

Философия диалектического материализма занимается предельным обобщением основных принципов множества научных теорий, применение которых доказало их достаточное для применения соответствие действительности. Тем самым она показывает границы знания нашей науки, выходить за которые нельзя при решении прикладных задач. Поэтому она не даёт готовые к применению ответы, но препятствует антинаучным рассуждениям и расчётам. Пренебрежение философией диалектического материализма мешало и будет мешать толком освоить ТРИЗ и понять её. А также не позволит развить её на творческие задачи в других областях деятельности человека.

Таким образом, каждому(!), кому необходимо сократить издержки на решение так называемых творческих задач в своей деятельности, придётся разобраться в аксиоматике ТРИЗ как логике диалектического материализма (ЛДМ) [8,9] в технической (технологической) области [1] и, главное, сделать её аксиоматикой своего мировосприятия. А далее можно уверенно развивать ТРИЗ и её прикладную часть, расширяя область применения на остальные виды деятельности человечества.

 

 

ТРИЗ и наука

 

Согласно современной методологии науки, любая теория для выполнения своего предназначения должна состоять из пяти частей [9]:

1) Исходные основания – фундаментальные понятия, законы, уравнения, аксиомы и т.п.

2) Абстрактная модель предмета теории (к примеру, "абсолютно черное тело", "идеальный газ" и т.п.).

3) Логика теории – совокупность правил и способов доказательства теорем, а также понятийный аппарат для описания теории, организации и изменения используемых знаний.

4) Установки философские и социокультурные. Они указывают мировоззрение, в рамках которого разрабатывается теория.

5) Следствия из данной теории.

 

Поскольку предметом данной статьи являются только исходные основания, постольку прочие части будут затрагиваться в меру необходимости. Не так уж трудно заметить, что теория начинается именно с исходных оснований - аксиоматики. Нет аксиоматики – нет и теории, кто бы что ни говорил. А не воспринимаешь аксиоматику – не воспринимаешь и не понимаешь теорию, что и кто бы себе ни воображал. Даже если наловчился пользоваться её понятиями. Но неизбежно – с сомнительной эффективностью, неминуемо толкающей ТРИЗ на всевозможные «улучшения», абы подогнать теорию к своему упрощённому, а то и вовсе ошибочному мировосприятию.

Основания ТРИЗ исходно отражены в пяти утверждениях [2]:

1. Теоретической основой ТРИЗ являются законы развития технических систем. Прежде всего, это законы материалистической диалекти­ки.

2. Эти законы можно познать и использовать для сознательного – без множества «пустых» проб – решения изобретательских задач.

3. Главный закон развития технических систем – стремление к увеличению идеальности.

4. Процесс решения изобретательской задачи можно рассматривать как выявление, анализ и разрешение технического противоречия.

5. Современная ТРИЗ превращается в ТРТС – теорию развития технических систем.

 Эти утверждения должны были конкретизировать материалистическую диалектику применительно к предмету ТРИЗ: «процессу решения изобретательской задачи». Вместе с тем достаточно очевидно, что решение творческой задачи в любой области деятельности – это тоже изобретение, хотя и патентуемое. Разница лишь в системах, где возникают задачи. И, стало быть, при некоторой «доводке» ТРИЗ должна быть применима и к системам другой природы.

Но получилось так, что Альтшуллер Г.С. вместе с соавторами ТРИЗ руководствовались сугубо идеалистической «диалектической логикой» вместо заявленного диалектического материализма. Поэтому в прикладной части ТРИЗ без труда можно найти отчётливые следы идеалистических философий. В частности, античного материализма Демокрита (обособленность движущихся материальных частиц), идеалистической диалектики Г. Гегеля (в отношении терминологии) и диалектической логики Ильенкова Э.В. (по сути) [1], эволюционной теории Ч. Дарвина (обобщившего учения предшественников от Анаксимандра и Эразма до Ж. Кондорсе и Т. Мальтуса) и панпсихизма персоналистов, А. Уайтхеда и др. Это выразилось в представлении о независимых от человека путях развития техники и т.н. «диалектического противоречия» как объективной, якобы присущего технике и природе вообще, и т.д. [6]. А идеализм никогда не мог решить ни одной практической задачи. Поэтому естественные неопределённости алгоритмов ТРИЗ были дополнены идеалистическими неопределённостями.

Приведённые утверждения оснований ТРИЗ [2] и по содержанию можно назвать аксиоматикой лишь условно. Преимущественно они описывают направления разработки способов применения диамата (в тогдашнем понимании) к решению задач в технической области. Об этом говорилось в первых статьях Альтшуллера Г.С. [3, 4].

Беда в том, что в понятие «материалистическая диалектика» (диалектический материализм) разные люди вкладывают разный смысл (наследие уже отвергнутой диалектической логики). В частности, сводя её к противоречиям [5]. Но противоречие – это только признак непонимания обстоятельств возникновения задачи и поэтому на деле применяются в ТРИЗ как способ уточнения задачи.

А с «технической системой» того хуже: каким-либо образом упорядоченную (обычно – в представлении человека) кучу «объектов» (в просторечии – «железяк»). При этом не учитывается, что «железяки» и их кучи – только средства создания и обеспечения технологических процессов, образующие систему только при их протекании. Хотя инженеры-технологи или экономисты, приучены работать с процессами и создаваемыми ими системами. Они стихийные приверженцы диалектического материализма, да не подозревающие этого. И они понимают, что «объект» – это условность, под которой понимается совокупность отличительных признаков каким-либо образом объединяемой совокупности наблюдаемых процессов. Кто из них не приучен, тот обречён на ошибки, будучи не способным на достаточную полноту и ясность действительности. К сожалению, к числу неприученных относятся инженеры-конструкторы.

Без знания и понимания аксиоматики нет и понимания ТРИЗ, без чего она и её прикладные технологии превращаются в груду утверждений, с которой каждый волен делать всё, что ему заблагорассудится. К примеру, наблюдаются неуклюжие попытки «изобретения» своих «ТРИЗ». Неуклюжих потому, что в обоснование не приводится ничего, кроме заурядного «я так думаю», «я верю только собственным выводам», «система – это то, что я считаю системой» и т.п. Такая неопределённость препятствует развитию прикладных технологий ТРИЗ.

Пяти утверждений [2] недостаточно, чтобы поднять ТРИЗ до уровня Общей теории сильного мышления (ОТСМ). Иначе не удастся распространить её применимость не только на задачи высшего уровня [10], но и за пределы техники вообще. А это возможно только на основе аксиоматики, предельно соответствующей ЛДМ и, разумеется, современному уровню естествознания. Поэтому излагаемая ниже аксиоматика уточнена этой логикой и называется «Аксиоматика ТРИЗ-ОТСМ».

Необходимо понимать невозможность существования неизменной аксиоматики любой теории. ТРИЗ-ОТСМ – не исключение. Аксиоматику, а затем и теорию необходимо развивать в соответствии новыми данными науки. Которая, кстати говоря, тоже не имеет «забронзовевших» аксиом и даже терминов. К примеру, если термин «материализм», отражающий сугубо научный подход к изучению Природы, уже тысячелетия остаётся основой науки, то его содержание меняется. Ныне он совсем не соответствует атомистическим представлениям древних греков. Но углубление в тонкости физических и философских теорий – пока за пределами ТРИЗ-ОТСМ и её Аксиоматики по причине отсутствия достаточных сведений. Хотя иногда и приходится углубляться, во избежание заблуждений. Не зря бывалые люди советовали «никогда не делать то, что может сделать специалист».

Таким образом, ТРИЗ-ОТСМ:

1) опирается на естествознание и его достижения,

2) на соответствующую им ЛДМ,

3) не является ни тем, ни другим,

4) но использует их в меру своих возможностей.

Остальное – лишь следствия.

ТРИЗ – это не столько новый подход к решению технических задач, сколько представление материалистической, рационалистической философии, приспособленное к особенностям мировосприятия, прежде всего, инженеров.

Отличить очередную «ТРИЗ» от оригинала не так уж и затруднительно. Альтшуллер Г.С. выстраивал ТРИЗ опираясь на научный метод, на материализм (диалектический), который в наиболее явном виде представлен в физике. Поэтому можно спокойно отметать без рассмотрения теории и способы решения изобретательских задач, если они явно или косвенно опираются на антинаучные утверждения (к примеру, что энергия или сознание – сущности).

 

 

Аксиоматика

 

Аксиоматика ТРИЗ-ОТСМ – это набор утверждений, принимаемых в качестве бесспорных начал (оснований) для дальнейших логических рассуждений в области развития и применения ТРИЗ-ОТСМ. Её аксиомы при необходимости могут быть заимствованы из других областей знания, где они признаны доказанными или приняты как аксиомы.

Истинность аксиоматики и построенной на ней теории подтверждается отсутствием отрицающих её фактов, а не подтверждающих.

В отличие от формальной логики, современная научная аксиоматика (следовательно, и аксиоматика ТРИЗ-ОТСМ), представляет собой утверждения, принимаемые в качестве истинных только в рамках своей области применения, а вне её – только рабочими предположениями в областях, откуда они были заимствованы. Научной эта аксиоматика называется по той причине, что формальная логика Аристотеля была разработана для юридических и политических нужд. Вершиной её применения в иных областях стали «Начала» Евклида. Однако её ограниченность была установлена ещё Аристоном Хиоским, учеником Зенона-стоика. Много позднее – Леонардо да Винчи и Галилеем. С тех пор для нужд науки и техники начали разрабатывать иные логики на иных аксиоматиках (Декарт, Паскаль, Лейбниц и другие). Более того, по мере вытеснения идеализма из области исследования мышления, формальная логика начала замещаться на ЛДМ и в этой области. Вместе с тем, она сохраняет свою полезность, будучи набором правил, обеспечивающих однозначность и одинаковость понимания передаваемых сообщений.

 

Аксиома 1. Философской основой ТРИЗ является логика диалектического материализма.

Пояснения. Эта логика (ЛДМ) представляет собой отражение в мозге внешних закономерностей [1]. Полнота отражения зависит от врождённой и приобретённой способности мозга учитывать их при выработке сигналов, управляющих поведением человека. Поэтому ЛДМ опирается на естественные закономерности природы и развивается в соответствии с новейшими достижениями науки.

 

Аксиома 2. Физической основой ТРИЗ является диалектический материализм.

Пояснения. 2.1. Обычно диалектический материализм представляют через три закона (утверждения!): «Единство и борьба противоположностей», «Отрицание отрицания» и «Переход количества в качество». Однако эти краткие обобщения тогдашних достижений науки даны на тогдашнем же языке философов и плохо применимы для современных задач во всех областях жизнедеятельности человека. Главный недостаток этих законов – следы идеализма Г. Гегеля. Кроме того, не учтены отсутствовавшие тогда теория систем и другие достижения нынешней науки об устройстве Вселенной. К сожалению, всё знакомство с этой основой ТРИЗ ограничивается только этими древними утверждениями и в дальнейшем никак не сказывается на мировосприятии [1].

2.2. Главное утверждение современного диалектического материализма: существует только движущаяся материя, изотропная во всей Вселенной. Всё взаимосвязано, взаимопроникаемо и взаимозависимо, а значит всё влияет друг на друга! Всё является частью одного целого.

2.2.1. Движущаяся материя – это наиболее общее название всего существующего («сущего»). Она неразрывна и проявляется в восприятии человека как изменения (процессы), события и явления, порядок (упорядоченность) и беспорядок (хаос), размеры и время (расстояния), обособленность (отличие признаков) и среда (общность признаков).

2.2.2. Человек, будучи частью движущейся материи, не может представить её сколько-нибудь целостно и поэтому воспринимает только её области, различающиеся некоторыми признаками, как условно отдельные объекты. Сделанные человеком средства наблюдения усиливают его возможности различения, не более того.

2.2.3. Наиболее общим признаком движения материи и её переходов (процессов) из одних форм в другие является энергия – скалярная физическая величина. Наиболее общие, повторяющиеся отношения между событиями в зависимости от вида восприятия (и, соответственно, способа измерения) воспринимаются как пространство и время. Это положение отражено в LT-системе единиц измерения. Остальные физические единицы измерения производны от метра и секунды. Из-за неразрывности движущейся материи её стороны взаимозависимы, что представляется как «законы», показывающие связь их признаков через меры (количественно).

2.3. Из представления о неразрывности движущейся материи следует, что всё – процессы (изменения, движение). Они изменчивы и конечны.

В каждом процессе происходит колебание величины определённости-неопределённости (устойчивости) его режима, зависящей от столь же изменчивых факторов (других процессов).

При достижении критической величины определённости/неопределённости своего режима каждый процесс вступает в точку бифуркации (область хаоса), в которой происходит его видоизменение (событие) в последующий процесс (процессы). Из чего следует неустойчивость организации систем, мерцающими по своей природе.

2.4. Совокупность взаимозависимых процессов образует одну большую систему – бесконечную изотропную Вселенную, организованную подобно фракталу. Фрактальность воспринимается человеком как законы природы, приводящие к возникновению сходных (закономерных), на самых низких уровнях – одинаковых образований на разных уровнях организации движущейся материи.

Каждая её область (выделяемая человеком по любым признакам) – часть система. В любой точке Вселенной движущаяся материя – и ламинарная, и турбулентная (хаотичная) на разных уровнях своей организации. Из чего следует, что всё – процессы, и всё – системы.

 

Следствие 1 Аксиомы 2. Мышление есть форма от­ра­же­ния действительности мозгом.

Пояснения. Это утверждение Ф. Энгельса было теоретически рассмотрено и обосновано [14]. Но «без познания механизма этого отражения (механизма работы мозга) повисает в воздухе знаменитый философский вопрос о первичности материи или сознания» [15]. Естественно, сугубо материалистического механизма.

 

Следствие 2 Аксиомы 2. Логика и интуиция – это частные виды отражения воспринимаемых закономерностей. Отражение всегда искажено и неполно, а искажение и неполнота – индивидуальны.

ЛДМ применительно к общественно-технической (технологической) среде означает необходимость исследования существования в этой среде закономерностей развития систем как совокупностей взаимосвязанных процессов: общественных и технологических. И далее – создания алгоритмов исследования возникновения и решения задач.

 

Следствие 3 Аксиомы 2. Приспособление человека к окружающей среде происходит путём создания и развития общественно-технических (технологических) систем как средств и способов обеспечения собственного гомеостаза путём создания энтростата и гомеостатического ансамбля.

Энтростат человека – это, прежде всего, природная среда, приспосабливаемая для нужд человека. По мере естественной эволюции из совокупности людей постепенно образовался над-биологический организм как способ организация совместной жизнедеятельности, повышающий возможности для выживания. Событием, запускающим процессы создания и развития энтростата, является нарушение или угроза нарушения гомеостаза, для устранения которых недостаточна естественная скорость биологической и общественной эволюции.

 

Следствие 4 Аксиомы 2. Обстоятельства в области возникновения задачи – это область неопределённости (хаоса)

Неопределённость снижается с построения предварительной физической модели этого явления. Далее строится системно-процессная модель совокупности обстоятельств, прямо или косвенно связанных с нежелательным явлением и выявляется причинно-следственная последовательность. Она позволяет постепенно уточнить предварительную. Физическую модель и тем самым ещё более снизить неопределённость. И так уточнять до тех пор, пока не будет выявлен источник нежелательного эффекта и возникнет возможность построить модель задачи.

 

Следствие 5 аксиомы 2. Изотропность Вселенной проявляется и в общественно-технической (технологической) области, где организации (системы) фрактально в той или иной мере повторяются во всех частных случаях.

Выявляемые и применяемые человеком «законы» и «закономерности» в общественно-технической (технологической) области только на поверхностный взгляд кажутся разными. Поэтому, если их выявлено более одного, то «общепринятые» определения понятий «закон» и «закономерность» ошибочны или недостаточно точны. Тем самым порождается хаос и искусственное разделение единого как в науке, так и во всё остальном.

 Хаос устраняется посредством разработки наиболее общего определения понятия и общего же способа его преобразования для применения в частных случаях преобразования процессов и систем. Это касается «веполей», «приёмов», «линий», «ресурсов» и всего остального.

 

Следствие 6 Аксиомы 2. Закон — это описание взаимообусловленности (взаимозависимости) явлений, отражающее отношение между причиной и следствием. Описание содержит отношения (обычно количественные) между явлениями (событиями) и особенности их проявлений в той или иной среде.

Среда – это всё, что находится за пределами рассматриваемого явления и влиянием чего можно пренебречь. Повторяющиеся следствия образуют закономерность. Закон проявляется через закономерность, а она – через существование системы [9]. Представление о существовании закономерностей применительно к развитию техники стало главным утверждением классической ТРИЗ. В дальнейшем предметом исследования должны быть технологии (технологические процессы), а не техника, которая является средством обеспечения протекания процессов.

 

 

Заключение

 

Аксиоматика ТРИЗ-ОТСМ разработана взамен утверждений Альтшуллера Г.С. в «Справке ТРИЗ-88». Сравнить по пунктам из затруднительно, но можно отметить главное отличие: эти утверждения вместе с дальнейшими построениями, начиная от АРИЗ, опираются на представления античных материалистов и идеалистическую логику, а не на заявленный диалектический материализм [1]. Впрочем, судя по всему, на них всё равно мало кто обращает внимание и тем более – вникает в смысл.

К сожалению, «из всего, на что смотрим, мы видим лишь то, что привыкли или хотим видеть». Следовательно, и приведённую здесь Аксиоматику каждый будет понимать по-своему; старые навыки уходят не просто. Поэтому придётся поработать над собой.

Данное утверждение особенно важно для преподавателей: мировосприятие слушателей обычно далеко от научного. Из чего следует задача перевода Аксиоматики в доступные образы без потери смысла. К примеру, на уроках астрономии не говорят школьникам, что Солнце – это лучезарный Гелиос в огненной колеснице. Им объясняют, что это лишь ближайшая звезда. А Гелиоса могут упомянуть как наивные представления древних греков. Хотя уже сейчас уже наше представление о Солнце подвергается сомнению [11].

Надо также учитывать зависимость развития мышления человека от образования и рода занятий. Беда в том, что хотя человек рождается с врождённым диалектическим мышлением, необходимость в дальнейшем передать ему огромный объём знаний за короткое время приводит их сжатию в набор из отдельных, малосвязанных обрывков. Вследствие этого диалектическое мировосприятие подавляется. Однако некоторые профессии принуждают восстанавливать природные диалектические способности. Скажем, инженеры-технологи принуждаются к этому работой с технологическими процессами, экономисты – с процессами оборота капитала в денежном выражении, управляющие предприятиями – с процессами оборота капитала во всех его формах и с многолетней длительностью цикла, строители – процессами строительства…

К сожалению, ТРИЗ создавалась преимущественно для инженеров-конструкторов как авторов изобретений, что и отражено в аббревиатуре. А механика (одна из старейших наук!) уже давно доведена до состояния набора готовых технических решений (теория машин и механизмов, Артоболевский, Анурьев, сопромат и т.д.). Это естественно сводит деятельность конструктора к комбинаторике и мозговому штурму, обычно не требующим диалектики. Поэтому разработанные на основе ТРИЗ способы решения технических задач (АРИЗ, ПУТП и другие) в действительности призваны только сужать область перебора.

Приводимая Аксиоматика достаточно надёжна для построения если пока ещё не ОТСМ, то по-настоящему диалектической, материалистической ТРИЗ, как доработанной классики – вполне. По мере освоения она постепенно создаёт возможность научиться видеть то, на что смотришь, во всех областях жизнедеятельности человека. На её основе можно создавать более совершенные и приближённые к естественному мышлению способы решения задач. В пределе – сразу видеть обстоятельства возникновения задачи, её суть и решение.

Более того, появление Аксиоматики наконец-то создаёт возможность разработки полноценного учебника. В пределе – для преобразования всей организации обучения новых поколения страны.

 

 

Королёв В.А.

Киев

24.07.2019 г.

 

Аннотация. Квалификация специалиста по ТРИЗ предопределяется уровнем знания и понимания её аксиоматики. А она определена Альтшуллером Г.С. как диалектический материализм. Соответственно, развитие ТРИЗ возможно только как следствие развития науки. Отказ от этой основы означает переход от ТРИЗ к идеалистической теории. Без жёсткого следования собственной аксиоматике нельзя ни планомерно развивать ТРИЗ вместе с её приложениями с уходом от привычного метода проб и ошибок, ни расширять её применение на другие области деятельности человека, ни толком обучать людей. Статья разработана в обоснование соответствующего раздела Справочника терминов ТРИЗ-ОТСМ.

Предыдущие версии Аксиоматики с момента публикации Аксиоматики ТРИЗ-ОТСМ (8) утрачивают свою силу, сохраняя свою ценность только как предмет исторического интереса.

 

Литература:

1. С144 Королёв В.А. «ТРИЗ – прикладной диамат», 2019г., 10 с., http://triz.org.ua/works/wx13.html

2. Альтшуллер Г.С. «Справка «ТРИЗ-88»

3. Альтшуллер Г.С., Шапиро Р.Б. «О психологии изобретательского творчества», ж. «Вопросы психологии» № 6, 1956г., стр. 37-49.

4. Альтшуллер Г., Шапиро Р. «Изгнание шестикрылого Серафима. Как мы встретились с шестикрылым Серафимом», Журнал "Изобретатель и рационализатор", 1959, № 10.

5. Королёв В.А. С134. «Противоположность» в физике и её роль в законах материалистической диалектики (О заблуждениях стихийного диалектика - 3)» (8 с., 2016г.), http://triz.org.ua/works/wx07.html.

6. Б.И. Голдовский «Некоторые комментарии к эвристическим возможностям противоречия в технической системе», 2015 г., http://www.metodolog.ru/node/1949.

7. Главное отличие материализма от идеализма: с позиций материализма сознание вторично по отношению к материи и поэтому может быть познано наукой. Именно поэтому ТРИЗ-ОТСМ будет оставаться научной теорией ровно до тех пор, пока опирается на диамат и его логику.

8. Эпитет «диалектический» означает, что материализм воспринял современные достижения науки и отказался от понимания материи как множества обособленных аристотелевских тел в пользу неразрывной движущейся материи. Аристотелевское понимание материи даёт хотя и приближённый, но вполне приемлемый для практики результат. Для многих, но не для всех. Поэтому для АРИЗ оказалось необходимым перейти к действиям с процессами и системами без античных «свойств». Интересующимся современными философскими баталиями в отношении аксиом диамата лучше обратиться к первоисточникам. Много интересного в изобилующей цитатами статье Вейника В.А. «Первичная аксиоматика материализма»: http://elib.org.ua/philosophy/ua_readme.php?archive=&id=1200798366&start_from=&subaction=showfull&ucat=1.

9. Королёв В.А. С133. «Закон суров – 3» (43 с., 2016г.). http://www.triz.org.ua/works/wx06.html

10. Г.С. Альтшуллер, Б.Л. Злотин, В.И. Филатов «Профессия – поиск нового (функционально-стоимостной анализ и теория решения изобретательских задач как система выявления резервов экономии)», Кишинёв, Картя Молдовеняскэ, 1985 г. стр. 14-16.

11. В. Минковский «Физика Вселенной. Новые теории и гипотезы», 60 с., Одесса, 2016г., http://magru.net/pubs/7241/#59.

12. Ф. Энгельс, в кн.: Маркс К. и Энгельс Ф., Соч., 2 изд., т. 20, стр. 554-555.

13. Ленин В. И., Поли. собр. соч., 5 изд., т. 29, стр. 172.

14. Королёв В.А. С125 «Мышление как форма отражения» (16 с., 2015 г., http://www.triz.org.ua/works/ws83.html).

15. Руслан Хазарзар «Скептический взгляд на диалектический материализм» http://khazarzar.skeptik.net/books/kh/causa.htm).
© 1998-2019 Владимир Королёв


    Новости
    Энциклопедия
    Социум
    АРИЗ
    Вопросы теории
    Разное
    Отзывы
    Публицистика

С145. Аксиоматика ТРИЗ-ОТСМ (8)

    Квалификация специалиста по ТРИЗ предопределяется уровнем знания и понимания её аксиоматики. А она определена Альтшуллером Г.С. как диалектический материализм. Соответственно, развитие ТРИЗ возможно только как следствие развития науки. Отказ от этой основы означает переход от ТРИЗ к идеалистической теории. Без жёсткого следования собственной аксиоматике нельзя ни планомерно развивать ТРИЗ вместе с её приложениями с уходом от привычного метода проб и ошибок, ни расширять её применение на другие области деятельности человека, ни толком обучать людей. Статья разработана в обоснование соответствующего раздела Справочника терминов ТРИЗ-ОТСМ.
    Предыдущие версии Аксиоматики с момента публикации Аксиоматики ТРИЗ-ОТСМ (8) утрачивают свою силу, сохраняя свою ценность только как предмет исторического интереса.

Начался ХХІ век, но вместо обещанного футурологами необычайного прогресса во всех сферах человеческого бытия мы наблюдаем глобальное помутнение умов и поворот к дремучему мракобесию.

А. Зиновьев

Теория решения задач

 

Каждому человеку приходится решать задачи. Много и разных. При этом он обычно руководствуется здравым смыслом.

А что такое «здравый смысл»? Это правильное применение опыта при решении задач.

А что такое «опыт»? Это усвоенные причинно-следственные зависимости («если – то»), отличающиеся от множества других «силой» или повторяемостью производимого впечатления. Первые обычно выглядят как утверждения (аксиомы), вторые – как обобщения (теории). Опыт бывает личный, но преимущественно – усвоенный обобщённый опыт (знания) множества людей.

А что такое «правильное применение»? Это умение (способность) преобразовывать беспорядочное множество частных случаев опыта в систему, которая соответствует требованиям задачи. И вот с этим-то совсем не просто.

Во-первых, у каждой причины обычно бывает более чем одно следствие, каждое из которых тоже становится причиной других следствий. Поэтому каждое действие человека порождает неограниченно растущее причинно-следственное «дерево». Чем больше опыт, тем больше следствий человек способен учесть. Но обычно знаний и времени хватает лишь на «кустик» из нескольких коротких «веточек». На большее мало кто способен от природы.

Во-вторых, на переход от причины к следствию всегда влияют обстоятельства. Они разные и переменные. Из-за чего и следствия одной и той же причины бывают разные. Поэтому чем больше и ветвистей «дерево», тем выше неопределённость итога рассуждений. Ведь вместо расчётов приходится полагаться на случайность, на «метод» проб и ошибок (МПиО). Справиться с неопределённостью в решении таких задач совсем уж мало кто способен от природы.

Хорошо ещё, что большинство задач во всех областях деятельности простые и решение их сводится к усвоению простых действий (и даже их технологических последовательностей) с жёстко предопределёнными следствиями. Со временем такие действия закономерно передаются технике и ЭВМ. Плохо, что такой работы для человека находится всё меньше.

Соответственно, задачи остающейся части становятся всё сложнее, затраты на поиск решения становятся всё больше, а цена ошибок становится всё выше. Неограниченно выше. Такие задачи принято называть творческими (ныне модно называть их креативными) из-за содержания в них практически бесконечной неопределённости. Поэтому уже давно стала очевидной (хотя и не всем) необходимость создания алгоритмов для таких задач. В конце концов, не за пять минут, а за сто лет, но люди как-то справляются с такими задачами. Хотя бы и посредством последовательного приближения с множеством ошибок.

Если отложить пока в сторону творческие задачи из области т.н. искусства (музыка и всё такое прочее), то наиболее важными для жизнедеятельности человека представляются наука, управление и техника (технология). Тем более, что между этими направлениями очень много общего. Но несмотря на множество попыток создать для них алгоритмы, дальше добрых советов дело почти не двигалось.

Прорыв наметился только в середине 20-го века в области инженерных творческих задач: была создана Теория решения изобретательских задач (ТРИЗ, Альтшуллер Г.С., СССР). Название обусловлено тем, что такие задачи часто называют изобретательскими, так как хорошее решение может быть признано изобретением.

Вообще говоря, первоначально ТРИЗ (далее эта аббревиатура будет применяться исключительно как теория, разработанная её автором; она же – классическая) была задумана как всеохватывающая теория. Но поскольку наиболее представительный документированный материал (патенты) для исследований существует только в технической области в виде патентов, постольку область применения теории оказалась ограниченной только техническими же задачами.

К сожалению, спешка с созданием прикладной части ТРИЗ (способы решения задач) привела к тому, что разработка её аксиоматики была отложена, оставшись только в виде самых общих ссылок на диалектический материализм (он же – научный подход) [2]. По этой причине оказались неудачными позднейшие попытки Альтшуллера Г.С и его соратников применить ТРИЗ в нетехнических областях. Хуже того, довольно скоро остановилось развитие собственно ТРИЗ. А застой положил начало деградации её и уже найденных алгоритмических способов (технологий) решения задач.

Выше упоминались «-измы», которые сразу лишают обычного человека всякого желания читать дальше и, тем более, разбираться. Но всё гораздо проще, чем кажется: любой человек обладает собственным набором принципов мировосприятия и поведения. Это-то и называется философией. Всё разнообразие философий можно разделить всего на две группы: материалистические (научные) и идеалистические (все остальные).

Философия диалектического материализма занимается предельным обобщением основных принципов множества научных теорий, применение которых доказало их достаточное для применения соответствие действительности. Тем самым она показывает границы знания нашей науки, выходить за которые нельзя при решении прикладных задач. Поэтому она не даёт готовые к применению ответы, но препятствует антинаучным рассуждениям и расчётам. Пренебрежение философией диалектического материализма мешало и будет мешать толком освоить ТРИЗ и понять её. А также не позволит развить её на творческие задачи в других областях деятельности человека.

Таким образом, каждому(!), кому необходимо сократить издержки на решение так называемых творческих задач в своей деятельности, придётся разобраться в аксиоматике ТРИЗ как логике диалектического материализма (ЛДМ) [8,9] в технической (технологической) области [1] и, главное, сделать её аксиоматикой своего мировосприятия. А далее можно уверенно развивать ТРИЗ и её прикладную часть, расширяя область применения на остальные виды деятельности человечества.

 

 

ТРИЗ и наука

 

Согласно современной методологии науки, любая теория для выполнения своего предназначения должна состоять из пяти частей [9]:

1) Исходные основания – фундаментальные понятия, законы, уравнения, аксиомы и т.п.

2) Абстрактная модель предмета теории (к примеру, "абсолютно черное тело", "идеальный газ" и т.п.).

3) Логика теории – совокупность правил и способов доказательства теорем, а также понятийный аппарат для описания теории, организации и изменения используемых знаний.

4) Установки философские и социокультурные. Они указывают мировоззрение, в рамках которого разрабатывается теория.

5) Следствия из данной теории.

 

Поскольку предметом данной статьи являются только исходные основания, постольку прочие части будут затрагиваться в меру необходимости. Не так уж трудно заметить, что теория начинается именно с исходных оснований - аксиоматики. Нет аксиоматики – нет и теории, кто бы что ни говорил. А не воспринимаешь аксиоматику – не воспринимаешь и не понимаешь теорию, что и кто бы себе ни воображал. Даже если наловчился пользоваться её понятиями. Но неизбежно – с сомнительной эффективностью, неминуемо толкающей ТРИЗ на всевозможные «улучшения», абы подогнать теорию к своему упрощённому, а то и вовсе ошибочному мировосприятию.

Основания ТРИЗ исходно отражены в пяти утверждениях [2]:

1. Теоретической основой ТРИЗ являются законы развития технических систем. Прежде всего, это законы материалистической диалекти­ки.

2. Эти законы можно познать и использовать для сознательного – без множества «пустых» проб – решения изобретательских задач.

3. Главный закон развития технических систем – стремление к увеличению идеальности.

4. Процесс решения изобретательской задачи можно рассматривать как выявление, анализ и разрешение технического противоречия.

5. Современная ТРИЗ превращается в ТРТС – теорию развития технических систем.

 Эти утверждения должны были конкретизировать материалистическую диалектику применительно к предмету ТРИЗ: «процессу решения изобретательской задачи». Вместе с тем достаточно очевидно, что решение творческой задачи в любой области деятельности – это тоже изобретение, хотя и патентуемое. Разница лишь в системах, где возникают задачи. И, стало быть, при некоторой «доводке» ТРИЗ должна быть применима и к системам другой природы.

Но получилось так, что Альтшуллер Г.С. вместе с соавторами ТРИЗ руководствовались сугубо идеалистической «диалектической логикой» вместо заявленного диалектического материализма. Поэтому в прикладной части ТРИЗ без труда можно найти отчётливые следы идеалистических философий. В частности, античного материализма Демокрита (обособленность движущихся материальных частиц), идеалистической диалектики Г. Гегеля (в отношении терминологии) и диалектической логики Ильенкова Э.В. (по сути) [1], эволюционной теории Ч. Дарвина (обобщившего учения предшественников от Анаксимандра и Эразма до Ж. Кондорсе и Т. Мальтуса) и панпсихизма персоналистов, А. Уайтхеда и др. Это выразилось в представлении о независимых от человека путях развития техники и т.н. «диалектического противоречия» как объективной, якобы присущего технике и природе вообще, и т.д. [6]. А идеализм никогда не мог решить ни одной практической задачи. Поэтому естественные неопределённости алгоритмов ТРИЗ были дополнены идеалистическими неопределённостями.

Приведённые утверждения оснований ТРИЗ [2] и по содержанию можно назвать аксиоматикой лишь условно. Преимущественно они описывают направления разработки способов применения диамата (в тогдашнем понимании) к решению задач в технической области. Об этом говорилось в первых статьях Альтшуллера Г.С. [3, 4].

Беда в том, что в понятие «материалистическая диалектика» (диалектический материализм) разные люди вкладывают разный смысл (наследие уже отвергнутой диалектической логики). В частности, сводя её к противоречиям [5]. Но противоречие – это только признак непонимания обстоятельств возникновения задачи и поэтому на деле применяются в ТРИЗ как способ уточнения задачи.

А с «технической системой» того хуже: каким-либо образом упорядоченную (обычно – в представлении человека) кучу «объектов» (в просторечии – «железяк»). При этом не учитывается, что «железяки» и их кучи – только средства создания и обеспечения технологических процессов, образующие систему только при их протекании. Хотя инженеры-технологи или экономисты, приучены работать с процессами и создаваемыми ими системами. Они стихийные приверженцы диалектического материализма, да не подозревающие этого. И они понимают, что «объект» – это условность, под которой понимается совокупность отличительных признаков каким-либо образом объединяемой совокупности наблюдаемых процессов. Кто из них не приучен, тот обречён на ошибки, будучи не способным на достаточную полноту и ясность действительности. К сожалению, к числу неприученных относятся инженеры-конструкторы.

Без знания и понимания аксиоматики нет и понимания ТРИЗ, без чего она и её прикладные технологии превращаются в груду утверждений, с которой каждый волен делать всё, что ему заблагорассудится. К примеру, наблюдаются неуклюжие попытки «изобретения» своих «ТРИЗ». Неуклюжих потому, что в обоснование не приводится ничего, кроме заурядного «я так думаю», «я верю только собственным выводам», «система – это то, что я считаю системой» и т.п. Такая неопределённость препятствует развитию прикладных технологий ТРИЗ.

Пяти утверждений [2] недостаточно, чтобы поднять ТРИЗ до уровня Общей теории сильного мышления (ОТСМ). Иначе не удастся распространить её применимость не только на задачи высшего уровня [10], но и за пределы техники вообще. А это возможно только на основе аксиоматики, предельно соответствующей ЛДМ и, разумеется, современному уровню естествознания. Поэтому излагаемая ниже аксиоматика уточнена этой логикой и называется «Аксиоматика ТРИЗ-ОТСМ».

Необходимо понимать невозможность существования неизменной аксиоматики любой теории. ТРИЗ-ОТСМ – не исключение. Аксиоматику, а затем и теорию необходимо развивать в соответствии новыми данными науки. Которая, кстати говоря, тоже не имеет «забронзовевших» аксиом и даже терминов. К примеру, если термин «материализм», отражающий сугубо научный подход к изучению Природы, уже тысячелетия остаётся основой науки, то его содержание меняется. Ныне он совсем не соответствует атомистическим представлениям древних греков. Но углубление в тонкости физических и философских теорий – пока за пределами ТРИЗ-ОТСМ и её Аксиоматики по причине отсутствия достаточных сведений. Хотя иногда и приходится углубляться, во избежание заблуждений. Не зря бывалые люди советовали «никогда не делать то, что может сделать специалист».

Таким образом, ТРИЗ-ОТСМ:

1) опирается на естествознание и его достижения,

2) на соответствующую им ЛДМ,

3) не является ни тем, ни другим,

4) но использует их в меру своих возможностей.

Остальное – лишь следствия.

ТРИЗ – это не столько новый подход к решению технических задач, сколько представление материалистической, рационалистической философии, приспособленное к особенностям мировосприятия, прежде всего, инженеров.

Отличить очередную «ТРИЗ» от оригинала не так уж и затруднительно. Альтшуллер Г.С. выстраивал ТРИЗ опираясь на научный метод, на материализм (диалектический), который в наиболее явном виде представлен в физике. Поэтому можно спокойно отметать без рассмотрения теории и способы решения изобретательских задач, если они явно или косвенно опираются на антинаучные утверждения (к примеру, что энергия или сознание – сущности).

 

 

Аксиоматика

 

Аксиоматика ТРИЗ-ОТСМ – это набор утверждений, принимаемых в качестве бесспорных начал (оснований) для дальнейших логических рассуждений в области развития и применения ТРИЗ-ОТСМ. Её аксиомы при необходимости могут быть заимствованы из других областей знания, где они признаны доказанными или приняты как аксиомы.

Истинность аксиоматики и построенной на ней теории подтверждается отсутствием отрицающих её фактов, а не подтверждающих.

В отличие от формальной логики, современная научная аксиоматика (следовательно, и аксиоматика ТРИЗ-ОТСМ), представляет собой утверждения, принимаемые в качестве истинных только в рамках своей области применения, а вне её – только рабочими предположениями в областях, откуда они были заимствованы. Научной эта аксиоматика называется по той причине, что формальная логика Аристотеля была разработана для юридических и политических нужд. Вершиной её применения в иных областях стали «Начала» Евклида. Однако её ограниченность была установлена ещё Аристоном Хиоским, учеником Зенона-стоика. Много позднее – Леонардо да Винчи и Галилеем. С тех пор для нужд науки и техники начали разрабатывать иные логики на иных аксиоматиках (Декарт, Паскаль, Лейбниц и другие). Более того, по мере вытеснения идеализма из области исследования мышления, формальная логика начала замещаться на ЛДМ и в этой области. Вместе с тем, она сохраняет свою полезность, будучи набором правил, обеспечивающих однозначность и одинаковость понимания передаваемых сообщений.

 

Аксиома 1. Философской основой ТРИЗ является логика диалектического материализма.

Пояснения. Эта логика (ЛДМ) представляет собой отражение в мозге внешних закономерностей [1]. Полнота отражения зависит от врождённой и приобретённой способности мозга учитывать их при выработке сигналов, управляющих поведением человека. Поэтому ЛДМ опирается на естественные закономерности природы и развивается в соответствии с новейшими достижениями науки.

 

Аксиома 2. Физической основой ТРИЗ является диалектический материализм.

Пояснения. 2.1. Обычно диалектический материализм представляют через три закона (утверждения!): «Единство и борьба противоположностей», «Отрицание отрицания» и «Переход количества в качество». Однако эти краткие обобщения тогдашних достижений науки даны на тогдашнем же языке философов и плохо применимы для современных задач во всех областях жизнедеятельности человека. Главный недостаток этих законов – следы идеализма Г. Гегеля. Кроме того, не учтены отсутствовавшие тогда теория систем и другие достижения нынешней науки об устройстве Вселенной. К сожалению, всё знакомство с этой основой ТРИЗ ограничивается только этими древними утверждениями и в дальнейшем никак не сказывается на мировосприятии [1].

2.2. Главное утверждение современного диалектического материализма: существует только движущаяся материя, изотропная во всей Вселенной. Всё взаимосвязано, взаимопроникаемо и взаимозависимо, а значит всё влияет друг на друга! Всё является частью одного целого.

2.2.1. Движущаяся материя – это наиболее общее название всего существующего («сущего»). Она неразрывна и проявляется в восприятии человека как изменения (процессы), события и явления, порядок (упорядоченность) и беспорядок (хаос), размеры и время (расстояния), обособленность (отличие признаков) и среда (общность признаков).

2.2.2. Человек, будучи частью движущейся материи, не может представить её сколько-нибудь целостно и поэтому воспринимает только её области, различающиеся некоторыми признаками, как условно отдельные объекты. Сделанные человеком средства наблюдения усиливают его возможности различения, не более того.

2.2.3. Наиболее общим признаком движения материи и её переходов (процессов) из одних форм в другие является энергия – скалярная физическая величина. Наиболее общие, повторяющиеся отношения между событиями в зависимости от вида восприятия (и, соответственно, способа измерения) воспринимаются как пространство и время. Это положение отражено в LT-системе единиц измерения. Остальные физические единицы измерения производны от метра и секунды. Из-за неразрывности движущейся материи её стороны взаимозависимы, что представляется как «законы», показывающие связь их признаков через меры (количественно).

2.3. Из представления о неразрывности движущейся материи следует, что всё – процессы (изменения, движение). Они изменчивы и конечны.

В каждом процессе происходит колебание величины определённости-неопределённости (устойчивости) его режима, зависящей от столь же изменчивых факторов (других процессов).

При достижении критической величины определённости/неопределённости своего режима каждый процесс вступает в точку бифуркации (область хаоса), в которой происходит его видоизменение (событие) в последующий процесс (процессы). Из чего следует неустойчивость организации систем, мерцающими по своей природе.

2.4. Совокупность взаимозависимых процессов образует одну большую систему – бесконечную изотропную Вселенную, организованную подобно фракталу. Фрактальность воспринимается человеком как законы природы, приводящие к возникновению сходных (закономерных), на самых низких уровнях – одинаковых образований на разных уровнях организации движущейся материи.

Каждая её область (выделяемая человеком по любым признакам) – часть система. В любой точке Вселенной движущаяся материя – и ламинарная, и турбулентная (хаотичная) на разных уровнях своей организации. Из чего следует, что всё – процессы, и всё – системы.

 

Следствие 1 Аксиомы 2. Мышление есть форма от­ра­же­ния действительности мозгом.

Пояснения. Это утверждение Ф. Энгельса было теоретически рассмотрено и обосновано [14]. Но «без познания механизма этого отражения (механизма работы мозга) повисает в воздухе знаменитый философский вопрос о первичности материи или сознания» [15]. Естественно, сугубо материалистического механизма.

 

Следствие 2 Аксиомы 2. Логика и интуиция – это частные виды отражения воспринимаемых закономерностей. Отражение всегда искажено и неполно, а искажение и неполнота – индивидуальны.

ЛДМ применительно к общественно-технической (технологической) среде означает необходимость исследования существования в этой среде закономерностей развития систем как совокупностей взаимосвязанных процессов: общественных и технологических. И далее – создания алгоритмов исследования возникновения и решения задач.

 

Следствие 3 Аксиомы 2. Приспособление человека к окружающей среде происходит путём создания и развития общественно-технических (технологических) систем как средств и способов обеспечения собственного гомеостаза путём создания энтростата и гомеостатического ансамбля.

Энтростат человека – это, прежде всего, природная среда, приспосабливаемая для нужд человека. По мере естественной эволюции из совокупности людей постепенно образовался над-биологический организм как способ организация совместной жизнедеятельности, повышающий возможности для выживания. Событием, запускающим процессы создания и развития энтростата, является нарушение или угроза нарушения гомеостаза, для устранения которых недостаточна естественная скорость биологической и общественной эволюции.

 

Следствие 4 Аксиомы 2. Обстоятельства в области возникновения задачи – это область неопределённости (хаоса)

Неопределённость снижается с построения предварительной физической модели этого явления. Далее строится системно-процессная модель совокупности обстоятельств, прямо или косвенно связанных с нежелательным явлением и выявляется причинно-следственная последовательность. Она позволяет постепенно уточнить предварительную. Физическую модель и тем самым ещё более снизить неопределённость. И так уточнять до тех пор, пока не будет выявлен источник нежелательного эффекта и возникнет возможность построить модель задачи.

 

Следствие 5 аксиомы 2. Изотропность Вселенной проявляется и в общественно-технической (технологической) области, где организации (системы) фрактально в той или иной мере повторяются во всех частных случаях.

Выявляемые и применяемые человеком «законы» и «закономерности» в общественно-технической (технологической) области только на поверхностный взгляд кажутся разными. Поэтому, если их выявлено более одного, то «общепринятые» определения понятий «закон» и «закономерность» ошибочны или недостаточно точны. Тем самым порождается хаос и искусственное разделение единого как в науке, так и во всё остальном.

 Хаос устраняется посредством разработки наиболее общего определения понятия и общего же способа его преобразования для применения в частных случаях преобразования процессов и систем. Это касается «веполей», «приёмов», «линий», «ресурсов» и всего остального.

 

Следствие 6 Аксиомы 2. Закон — это описание взаимообусловленности (взаимозависимости) явлений, отражающее отношение между причиной и следствием. Описание содержит отношения (обычно количественные) между явлениями (событиями) и особенности их проявлений в той или иной среде.

Среда – это всё, что находится за пределами рассматриваемого явления и влиянием чего можно пренебречь. Повторяющиеся следствия образуют закономерность. Закон проявляется через закономерность, а она – через существование системы [9]. Представление о существовании закономерностей применительно к развитию техники стало главным утверждением классической ТРИЗ. В дальнейшем предметом исследования должны быть технологии (технологические процессы), а не техника, которая является средством обеспечения протекания процессов.

 

 

Заключение

 

Аксиоматика ТРИЗ-ОТСМ разработана взамен утверждений Альтшуллера Г.С. в «Справке ТРИЗ-88». Сравнить по пунктам из затруднительно, но можно отметить главное отличие: эти утверждения вместе с дальнейшими построениями, начиная от АРИЗ, опираются на представления античных материалистов и идеалистическую логику, а не на заявленный диалектический материализм [1]. Впрочем, судя по всему, на них всё равно мало кто обращает внимание и тем более – вникает в смысл.

К сожалению, «из всего, на что смотрим, мы видим лишь то, что привыкли или хотим видеть». Следовательно, и приведённую здесь Аксиоматику каждый будет понимать по-своему; старые навыки уходят не просто. Поэтому придётся поработать над собой.

Данное утверждение особенно важно для преподавателей: мировосприятие слушателей обычно далеко от научного. Из чего следует задача перевода Аксиоматики в доступные образы без потери смысла. К примеру, на уроках астрономии не говорят школьникам, что Солнце – это лучезарный Гелиос в огненной колеснице. Им объясняют, что это лишь ближайшая звезда. А Гелиоса могут упомянуть как наивные представления древних греков. Хотя уже сейчас уже наше представление о Солнце подвергается сомнению [11].

Надо также учитывать зависимость развития мышления человека от образования и рода занятий. Беда в том, что хотя человек рождается с врождённым диалектическим мышлением, необходимость в дальнейшем передать ему огромный объём знаний за короткое время приводит их сжатию в набор из отдельных, малосвязанных обрывков. Вследствие этого диалектическое мировосприятие подавляется. Однако некоторые профессии принуждают восстанавливать природные диалектические способности. Скажем, инженеры-технологи принуждаются к этому работой с технологическими процессами, экономисты – с процессами оборота капитала в денежном выражении, управляющие предприятиями – с процессами оборота капитала во всех его формах и с многолетней длительностью цикла, строители – процессами строительства…

К сожалению, ТРИЗ создавалась преимущественно для инженеров-конструкторов как авторов изобретений, что и отражено в аббревиатуре. А механика (одна из старейших наук!) уже давно доведена до состояния набора готовых технических решений (теория машин и механизмов, Артоболевский, Анурьев, сопромат и т.д.). Это естественно сводит деятельность конструктора к комбинаторике и мозговому штурму, обычно не требующим диалектики. Поэтому разработанные на основе ТРИЗ способы решения технических задач (АРИЗ, ПУТП и другие) в действительности призваны только сужать область перебора.

Приводимая Аксиоматика достаточно надёжна для построения если пока ещё не ОТСМ, то по-настоящему диалектической, материалистической ТРИЗ, как доработанной классики – вполне. По мере освоения она постепенно создаёт возможность научиться видеть то, на что смотришь, во всех областях жизнедеятельности человека. На её основе можно создавать более совершенные и приближённые к естественному мышлению способы решения задач. В пределе – сразу видеть обстоятельства возникновения задачи, её суть и решение.

Более того, появление Аксиоматики наконец-то создаёт возможность разработки полноценного учебника. В пределе – для преобразования всей организации обучения новых поколения страны.

 

 

Королёв В.А.

Киев

24.07.2019 г.

 

Аннотация. Квалификация специалиста по ТРИЗ предопределяется уровнем знания и понимания её аксиоматики. А она определена Альтшуллером Г.С. как диалектический материализм. Соответственно, развитие ТРИЗ возможно только как следствие развития науки. Отказ от этой основы означает переход от ТРИЗ к идеалистической теории. Без жёсткого следования собственной аксиоматике нельзя ни планомерно развивать ТРИЗ вместе с её приложениями с уходом от привычного метода проб и ошибок, ни расширять её применение на другие области деятельности человека, ни толком обучать людей. Статья разработана в обоснование соответствующего раздела Справочника терминов ТРИЗ-ОТСМ.

Предыдущие версии Аксиоматики с момента публикации Аксиоматики ТРИЗ-ОТСМ (8) утрачивают свою силу, сохраняя свою ценность только как предмет исторического интереса.

 

Литература:

1. С144 Королёв В.А. «ТРИЗ – прикладной диамат», 2019г., 10 с., http://triz.org.ua/works/wx13.html

2. Альтшуллер Г.С. «Справка «ТРИЗ-88»

3. Альтшуллер Г.С., Шапиро Р.Б. «О психологии изобретательского творчества», ж. «Вопросы психологии» № 6, 1956г., стр. 37-49.

4. Альтшуллер Г., Шапиро Р. «Изгнание шестикрылого Серафима. Как мы встретились с шестикрылым Серафимом», Журнал "Изобретатель и рационализатор", 1959, № 10.

5. Королёв В.А. С134. «Противоположность» в физике и её роль в законах материалистической диалектики (О заблуждениях стихийного диалектика - 3)» (8 с., 2016г.), http://triz.org.ua/works/wx07.html.

6. Б.И. Голдовский «Некоторые комментарии к эвристическим возможностям противоречия в технической системе», 2015 г., http://www.metodolog.ru/node/1949.

7. Главное отличие материализма от идеализма: с позиций материализма сознание вторично по отношению к материи и поэтому может быть познано наукой. Именно поэтому ТРИЗ-ОТСМ будет оставаться научной теорией ровно до тех пор, пока опирается на диамат и его логику.

8. Эпитет «диалектический» означает, что материализм воспринял современные достижения науки и отказался от понимания материи как множества обособленных аристотелевских тел в пользу неразрывной движущейся материи. Аристотелевское понимание материи даёт хотя и приближённый, но вполне приемлемый для практики результат. Для многих, но не для всех. Поэтому для АРИЗ оказалось необходимым перейти к действиям с процессами и системами без античных «свойств». Интересующимся современными философскими баталиями в отношении аксиом диамата лучше обратиться к первоисточникам. Много интересного в изобилующей цитатами статье Вейника В.А. «Первичная аксиоматика материализма»: http://elib.org.ua/philosophy/ua_readme.php?archive=&id=1200798366&start_from=&subaction=showfull&ucat=1.

9. Королёв В.А. С133. «Закон суров – 3» (43 с., 2016г.). http://www.triz.org.ua/works/wx06.html

10. Г.С. Альтшуллер, Б.Л. Злотин, В.И. Филатов «Профессия – поиск нового (функционально-стоимостной анализ и теория решения изобретательских задач как система выявления резервов экономии)», Кишинёв, Картя Молдовеняскэ, 1985 г. стр. 14-16.

11. В. Минковский «Физика Вселенной. Новые теории и гипотезы», 60 с., Одесса, 2016г., http://magru.net/pubs/7241/#59.

12. Ф. Энгельс, в кн.: Маркс К. и Энгельс Ф., Соч., 2 изд., т. 20, стр. 554-555.

13. Ленин В. И., Поли. собр. соч., 5 изд., т. 29, стр. 172.

14. Королёв В.А. С125 «Мышление как форма отражения» (16 с., 2015 г., http://www.triz.org.ua/works/ws83.html).

15. Руслан Хазарзар «Скептический взгляд на диалектический материализм» http://khazarzar.skeptik.net/books/kh/causa.htm).
© 1998-2019 Владимир Королёв
\end{document}
