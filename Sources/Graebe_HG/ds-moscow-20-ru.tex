\documentclass[11pt,a4paper]{article}
\usepackage{od}
\usepackage[ngerman,main=russian]{babel}

\title{Об эволюции технических и общих систем\\[1em] \Large Комментарий к
  материалам Московской ТРИЗ конференции\\ \emph{Модели представления и линии
    развития технических систем}\\ 28.-29. Ноября 2020 г.}

\author{Hans-Gert Gräbe, Leipzig}

\date{Версия от 27 ноября 2020 г.}
\begin{document}
\maketitle

В этом году в рамках Лейпцигского ТРИЗ-семинара мы интенсивно занимались
вопросами развития технических и общих систем, см. например
\begin{itemize}[noitemsep]
\item \url{https://wumm-project.github.io/TTS}
\item \cite{Graebe2020a}, \cite{Graebe2020b}, \cite{Graebe2020c},
  \cite{Graebe2020d}
\end{itemize}

Один из центральных вопросов, который также в
докладах\footnote{См. \url{https://yadi.sk/d/x1q4pYiPhtBxHw}.}  Московской
ТРИЗ конференции не получил адекватного внимания, это вопрос о том, по каким
линиям непрерывности группируются реальные технические системы.  Без ответа на
этот вопрос все рассуждения о S-образных кривых, путей развития и обсуждение
линий эволюции и т.п. имеют только очень ограниченный смысл.

Из стандартной работы MATRIZ \cite{TESE2018} можно извлекать, хотя это там не
явно сформулируется, что там такие линии непрерывности строятся на основе
общих каталогов продуктов технических систем, взятые из рыночной окружающей
среды.  Но это ограничит концепцию технической системы к таким \emph{маленьким
  техническим системам} (МТС) и \emph{крупные технические системы} (КТС),
такие как Минский Тракторный Завод (как эволюция частного) или автозаводы
вообще (как эволюция обобщенного) не рассматриваются.

В \cite{Graebe2020b} развит и обоснован тезис о том, что такие КТС с одной
стороны уникальны -- таким образом концепт прослеживания линий непрерывности
для таких КТС станет более простой -- но, с другой стороны, они развиваются
частично по собственным законам, для чего требуется модифицировать известные
законы для МТС или заменить их полностью.

Эта мысль адекватно не рассмотрена ни в одном из представленных статьей.
Поэтому я хотел бы вынести это на обсуждение.

\begin{thebibliography}{xxx}
\bibitem{Graebe2020a} Hans-Gert Gräbe.  TRIZ und Transformationen
  sozio-technischer und sozio-ökologischer Systeme (ТРИЗ и трансформации
  социо-технических и социо-экологических систем). LIFIS Online,
  27.06.2020.\\ English Translation, June 2020, submitted to TRIZ
  Review.\\ Русский постер для MATRIZ Online Forum August 2020 см.
  \\ \url{https://hg-graebe.de/EigeneTexte/MOF-20-ru.pdf}.
\bibitem{Graebe2020b} Hans-Gert Gräbe. Die Menschen und ihre Technischen
  Systeme (Люди и их технические системы). LIFIS Online, 19.05.2020.
  \url{doi:10.14625/graebe_20200519}
\bibitem{Graebe2020c} Hans-Gert Gräbe.  \foreignlanguage{russian}{Человек и
  его технические системы}.  TRIZ Developer Summit 2020.
  \url{https://hg-graebe.de/EigeneTexte/TDS-2020.pdf}
\bibitem{Graebe2020d} Hans-Gert Gräbe.  Technical Systems and Purposes
  (Технические системы и цели). Материалы первой Немецкой TRIZ Conference
  Online 2020. Публикуется в издательстве Springer.
\bibitem{TESE2018} Alexander Lyubomirskiy, Simon Litvin, Sergey Ikovenko,
  Christian M. Thurnes, Robert Adunka (2018). Trends of Engineering System
  Evolution. Sulzbach-Rosenberg.  ISBN 978-3-00-059846-3.
\end{thebibliography}
\end{document}

