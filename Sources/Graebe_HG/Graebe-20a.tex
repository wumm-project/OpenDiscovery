\documentclass[12pt,a4paper]{article}
\usepackage{od}
\usepackage[T1,T2A]{fontenc}
\usepackage[utf8]{inputenc}
\usepackage{wrapfig}
\usepackage[main=ngerman,russian]{babel}
\usepackage{tikz}
\usetikzlibrary{arrows.meta, positioning}

\title{TRIZ und Transformationen in sozio-\\ technischen und
  sozio-ökologischen Systemen} 
\author{Hans-Gert Gräbe, Leipzig}
%\def\theauthor{Hans-Gert Gräbe}
\date{27.\,06.\,2020}
\begin{document}
\maketitle

\begin{abstract}
  In diesem Aufsatz werden systemische Transitionskonzepte (großer)
  technischer Systeme, die insbesondere in den TRIZ-Theorien zur Evolution
  ingenieur-technischer Systeme eine Rolle spielen, mit Transitionskonzepten
  zum nachhaltigen Management ökologischer Systeme verglichen. Die Ähnlichkeit
  der Problemlagen resultiert vor allem daraus, dass in beiden Bereichen
  \emph{bereits existierende} Systeme weiterentwickelt werden sollen, wobei
  widersprüchliche, interessengeleitete Anforderungen in ein
  \emph{funktionierendes} „Weltmodell“ zu transformieren sind, um dieses dann
  \emph{praktisch zu implementieren}.

  Die Überlegungen setzen auf früheren eigenen Arbeiten auf, in denen bereits
  ein detailliertes Konzept sozio-technischer Systeme entwickelt wurde,
  differenzieren aber stärker zwischen Prozessen der Entscheidungsvorbereitung
  und der Entscheidungsfindung. Dabei wird die ingenieur-technische Qualität
  auch von Managementprozessen unterstrichen und auf dieser Basis ein
  einheitlicher TRIZ-methodischer Zugang zu den Prozessen der
  Entscheidungsvorbereitung und der Entscheidungsfindung vorgeschlagen, der
  Anschlussfähig"|keit auch in komplexeren Modellierungen zu vermitteln
  vermag.
\end{abstract}

\section{Einleitung}

TRIZ ist eine systematische Innovationsmethodik, die widersprüchliche
Anforderungssituationen durch definierte Abstraktionsmuster (Prinzipien,
Standards, Trends) in größere Kontexte einbettet und dort das Potenzial von
Analogielösungen nutzt, um Transitionspfade der untersuchten defizitären
technischen Systeme hin zu Systemen ohne die identifizierten Widersprüche zu
bestimmen mit dem Anspruch, dass diese Transition dann auch unter den
gegebenen \emph{realen} Bedingungen umgesetzt werden können.

Ähnliche Fragen werden im Rahmen von Nachhaltigkeitsdebatten auch für
sozio-ökolo"|gische und sozio-kulturelle Systeme untersucht. Es gibt hierzu
eine Vielzahl von Arbeiten. Ich beziehe mich hier auf die zusammenfassenden
Argumentationen in \cite{Foxon2009} und \cite{Geels2007} sowie die
fundamentale Arbeit \cite{Holling2001}, in der grundlegende Konzepte für
Wandlungsprozesse als Ausbruch aus und Rückkehr zu Fließgleichgewichten
entwickelt werden. Alle drei Ansätze sind sehr komplex und können hier nicht
im Detail dargestellt werden. Es wird also vorausgesetzt, dass der Leserin
oder dem Leser diese drei Arbeiten wie auch die Grundzüge der TRIZ-Methodik
hinreichend bekannt sind.

Allgemein wird davon ausgegangen, dass sich sozio-ökologische von
sozio-technischen Systemen vor allem darin unterscheiden, dass sie nicht
„zweckgerichtet“ konstruiert worden sind, sondern auf „natürliche Weise“
funktionieren.  Eine Synopse entsprechender Untersuchungen \cite{Graebe2020a},
die wir im Rahmen eines \emph{Seminars zur Systemwissenschaft} erstellt haben,
zeigt allerdings, dass diese Annahme aus folgenden Gründen nicht trägt:
\begin{itemize}
\item[1.] Das \emph{Interesse} an Widersprüchen in sozio-ökologischen Systemen
  ist wesentlich durch Zwecke und Interessen menschlichen Handelns bestimmt,
  an denen sich auch die Lösungs"|vorschläge für Systemtransitionen
  orientieren.
\item[2.] Die betrachteten sozio-ökologischen Systeme sind seit mehreren
  tausend Jahren bereits durch menschliches Handeln überformt. Dieser
  sozio-kulturelle Charakter jener Systeme, der in einschlägigen Arbeiten
  weitgehend ignoriert und auf das Studium historischer Bewirtschaftungspraxen
  von Infrastrukturen reduziert wird, rückt jene Systeme in die Nähe
  technischer Systeme in dem Sinne, dass auch jene durch eine Symbiose aus
  Beschreibungs- und Vollzugform charakterisiert sind.
\item[3.] Die vorgeschlagenen Transitionskonzepte haben klar technischen
  Charakter in dem Sinne, dass sozio-kulturelle Prozesse mit Methoden
  \emph{gestaltet} werden sollen, die aus inge"|nieur-technischen Ansätzen
  weitgehend übernommen werden, auch wenn die Differenz zwischen
  \emph{begründeten Erwartungen} und \emph{erfahrenen Ergebnissen} dabei eher
  an die Kinderstube des Industriezeitalters erinnert. 
\end{itemize}
\enlargethispage{1em}
Auch in den TRIZ-Anwendungen ist in den letzten 20 Jahren eine deutliche
Schwerpunktverschiebung zu verzeichnen. Während Theorie und Praxis der TRIZ zu
Altschullers Zeiten stark durch kleinteilige technische Erfindungen und die
gesellschaftliche Unterschätzung systematischer Innovationsmethodiken geprägt
war (in Ost wie West), hat sich dies in führenden Industrieländern inzwischen
geändert. Systematische Innovationsmethodiken spielen in großen Unternehmen im
Zuge eines Innovationsmanagements eine zunehmend wichtige Rolle.
Reifegradmodelle wie CMMI\footnote{CMMI steht für \emph{Capability Maturity
    Model Integration} und ist eine weit verbreitete Familie von
  Referenzmodellen zur Bewertung der Qualitätsfähigkeit von (u.a.)
  IT-Unternehmen. Im europäischen Kontext spielen daneben SPICE (insbesondere
  Automotive SPICE) sowie der internationale Standards ISO/IEC 15504 eine
  Rolle, diese sind aber vergleichbar strukturiert.} zeigen auf, dass hierfür
Beschreibungsformen und Unternehmens\emph{modelle} eine zentrale Rolle
spielen.  Dazu muss das Unternehmen seine eigenen Prozesse zunächst mit
Beschreibungsformen untersetzen (CMMI-Stufe 2 „managed“), diese durch
Standardisierung sprachlich gestaltbar machen (CMMI-Stufe 3 „defined“) und
schließlich durch strukturierte Datenerhebung auf die eigenen Praxen
rückkoppeln (CMMI-Stufe 4 „quantitatively managed“). Erst auf dieser Basis
sind Optimierungen und technologische Wandlungsprozesse (CMMI-Stufe 5
„optimizing“) als Transitionsprozesse strukturiert gestaltbar.

Damit ändert sich der in TRIZ schon immer schwach untersetzte Begriff eines
\emph{technischen Systems} aber grundlegend von einem primär am
Konsumgütermarkt orientierten Verständnis der Verbesserung und Evolution
typähnlicher Artefakte, wie sie noch prägend für die Mehrzahl der Beispiele
etwa in \cite{TESE2018} sind, hin zur Transition technischer Großsysteme, die
nur als Unikate existieren, siehe \cite{Graebe2020b}, und damit dem Charakter
sozio-ökologischer Systeme näher stehen als den technischen Artefakten eines
Massenmarkts.  Mehr noch, wesentliche Widersprüche, die sich in
sozio-ökologischen Systemen gerade zwischen langwelligen „natürlichen
Prozessen“ und kurzwelligen sozio-ökonomischen Rationalitäten ergeben,
reproduzieren sich als Widersprüche zwischen der investiven und operativen
Dimension jener sozio-ökonomischen Prozesse, zwischen der Notwendigkeit
technischer Innovation als langfristiger unternehmerischer Überlebensbedingung
und der kurzfristigen Notwendigkeit, dafür im operativen Geschäft das nötige
„Kleingeld“ zu verdienen.  Diese Internalisierung externer gesellschaftlicher
Widersprüche in die Entfaltung interner Unternehmenslogiken ist ein
wesentlicher Treiber von Entwicklungen, die heute unter der Überschrift
\emph{TRIZ und Business} laufen.  In den spezifischen Handlungsbedingungen der
DDR-Erfinderschulen der 1980er Jahre sind diese Entwicklungen bereits präsent,
und so kann es nicht überraschen, dass diese Erfahrungen auch in technologisch
höher entwickelten sozialistischen Staaten wie der damaligen Tschechoslowakei
auf starke Resonanz stießen.  Siehe dazu \cite{Graebe2019a}.

Es ist deshalb an der Zeit, diese Parallelen zwischen den Herausforderungen
moderner Unternehmensentwicklungen und den Herausforderungen
sozio-okölogischer Transformationsprozesse genauer zu analysieren.  TRIZ als
systematische Innovationsmethodik kann nach meiner Überzeugung zu beidem einen
Beitrag leisten und so eine Brücke bauen zwischen einem
Nachhaltigkeitsdiskurs\footnote{Dieser ist um den Begriff „Bildung für
  nachhaltige Entwicklung“ (BNE) zentriert. }, der Ziele schärft, ohne
realistisch Mittel zu benennen, und einem Industriediskurs\footnote{Dieser ist
  um den Begriff „MINT“ zentriert, was für Mathematik, Informatik,
  Naturwissenschaften, Technik steht. In dieser deutschen Version steht die
  Technik nur am Rande, in der englischen Version „STEM“ -- Science,
  Technology, Engineering, Mathematics -- nimmt sie einen deutlich
  prominenteren Platz ein und ist mit den beiden Begriffen „Technology“ und
  „Engineering“ auch in ihren Dimensionen als Beschreibungs- und Vollzugsform
  klarer angesprochen.}, der auf die Entwicklung der Mittel und
Qualifikationen („uns gehen die Fachkräfte
aus“\footnote{\url{https://www.bmwi.de/Redaktion/DE/Dossier/fachkraeftesicherung.html}})
fokussiert, ohne klare gesellschaftlich übergreifende Zielkorridore zu
formulieren\footnote{Das „MINT-Meter“ der wesentlich von Unternehmerverbänden
  ins Leben gerufenen Initiative „MINT -- Zukunft schaffen“ wurde inzwischen
  jedenfalls wieder eingestellt, wie überhaupt die Finanzierung jener
  Querschnittsaktivität nicht von langer Dauer war, nachdem sich die Telekom
  als Hauptinitiator jener um 2010 herum gestarteten Aktivitäten inzwischen
  selbst in schwereren Fahrwassern bewegt.
  \url{https://mintzukunftschaffen.de/category/mint-meter/mint-luecke/}}.

In diesem Aufsatz sind diesbezügliche Betrachtungen zusammengefasst, die sich
aus der Gegenüberstellung von Transitionskonzepten der TRIZ und denen aus
Transitions- und Resilienzkonzepten sozio-okölogischer und sozio-technischer
Systeme im Nachhaltigkeitsdiskurs ergeben haben.  

\section{Zum Systembegriff}

Betrieb und Nutzung technischer Systeme ist heute ein zentrales Element Welt
verändern"|der menschlicher Praxen. Dafür ist planmäßiges und abgestimmtes
arbeitsteiliges Handeln erforderlich, denn das Nutzen eines Systems setzt
dessen Betrieb voraus.  Umgekehrt ist es wenig sinnvoll, ein System zu
betreiben, das nicht genutzt wird. Eng verbunden mit dieser aus der Informatik
gut bekannten Unterscheidung von Definition und Aufruf einer Funktion ist die
Unterscheidung von Designzeit und Laufzeit, die im realweltlichen
arbeitsteiligen Einsatz technischer Systeme noch größere Bedeutung hat --
während der Designzeit wird das prinzipielle kooperative Zusammenwirken
\emph{geplant}, während der Laufzeit \emph{der Plan ausgeführt}. Für
technische Systeme sind dazu deren interpersonal als \emph{begründete
  Erwartungen} kommunizierten \emph{Beschreibungsformen} und die in
\emph{erfahrenen Ergebnissen} resultierenden \emph{Vollzugsformen} zu
unterscheiden.  Die weitere Argumentation in diesem Abschnitt rekapituliert
\cite{Graebe2020b} und ist dort detaillierter ausgeführt.

Neben der Beschreibungs- und Vollzugsdimension spielt für technische Systeme
auch der \emph{Aspekt der Wiederverwendung} eine große Rolle.  Dies gilt,
zumindest auf der artefaktischen Ebene, allerdings \emph{nicht} für die
meisten technischen Großsysteme -- diese sind \emph{Unikate}, auch wenn bei
deren Montage standardisierte Komponenten verbaut werden. Auch die Mehrzahl
der Informatiker ist mit der Erstellung solcher Unikate befasst, denn die
IT-Systeme, die derartige Anlagen steuern, sind ebenfalls Unikate.  In dieser
Arbeit konzentrieren wir uns besonders auf diese technischen Großsysteme und
deren Parallelen zu Gestaltungsfragen sozio-ökologischer Systeme.

Die Besonderheiten eines technischen Systems liegen damit vor allem im Bereich
des \emph{Zusammenspiels der Komponenten}, bei denen ebenfalls zwischen der
Beschreibungsform (der Modellierung) und der Vollzugsform (dem Betrieb im
Kontext der verschiedenen technischen Großsysteme) unterschieden werden muss.
Während in der Planungs- und Modellierungsphase noch viele Freiheiten für
Änderungen offen bleiben, ist die Vollzugsform durch deutlich höhere
Inflexibilität gekennzeichnet.  Obwohl auch hier die Welt komplizierter ist
als in einer solchen Dichotomie einzufangen -- wer mag schon einen Plan
ändern, der von den hohen Chefs bereits abgesegnet wurde --, soll im Weiteren
mit dieser begriff"|lichen Reduzierung gearbeitet werden.

Damit sind wesentliche Elemente zusammengetragen, die hier als Grundlage für
den \emph{Begriff eines technischen Systems} dienen, der in einem
planerisch-realweltlichen Kontext vierfach überladen ist:
\newpage
\begin{itemize}
\item [1.] als realweltliches Unikat (z.B. als Produkt, auch wenn das Unikat
  ein Service ist),
\item [2.] als Beschreibung dieses realweltlichen Unikats (z.B. in der Form
  einer speziellen Produktkonfiguration)
\end{itemize}
und für in größerer Stückzahl hergestellte Komponenten auch noch
\begin{itemize}
\item [3.] als Beschreibung des Designs des System-Templates (Produkt-Design)
  sowie
\item [4.] als Beschreibung und Betrieb der Auslieferungs- und
  Betriebsstrukturen der nach diesem Template gefertigten realweltlichen
  Unikate (basierend auf Produktions-, Qualitäts"|sicherungs-, Auslieferungs-,
  Betriebs- und Wartungsplänen).
\end{itemize}

\emph{Technische Systeme} sind in einem solchen Kontext Systeme, auf deren
Gestaltung und Nutzung kooperativ und arbeitsteilig agierende Menschen
Einfluss nehmen, wobei \emph{vorgefundene} technische Systeme auf
Beschreibungsebene durch eine \emph{Spezifikation} ihrer Schnittstellen und
auf Vollzugsebene durch die \emph{Gewähr spezifikationskonformen Betriebs}
normativ charakterisiert sind.

Ähnliches gilt auch für die Beschreibungsformen „natürlicher“ Systeme, die in
einschlägigen Arbeiten ebenfalls in einer strukturierten Form als
\emph{Systeme von Systemen} modelliert werden, als Systeme, welche aus
Komponenten bestehen, die ihrerseits Systeme sind, deren \emph{Funktionieren}
(sowohl im funktionalen als auch im operativen Sinn) für die aktuell
betrachtete Systemebene aber vorausgesetzt wird.

Dem (allgemeineren) Begriff eines Systems kommt in einem solchen Verständnis
die epistemische Funktion der (funktionalen) „Reduktion auf das Wesentliche“
zu. Diese Reduktion erfolgt in folgenden drei Dimensionen
\cite[S. 18]{Graebe2020a}
\begin{itemize}
\item [(1)] Abgrenzung des Systems nach außen gegen eine \emph{Umwelt},
  Reduktion dieser Beziehungen auf Input/Output-Beziehungen und garantierten
  Durchsatz.
\item [(2)] Abgrenzung des Systems nach innen durch Zusammenfassen von
  Teilbereichen als \emph{Komponenten}, deren Funktionieren auf eine
  „Verhaltenssteuerung“ über Input/Out"|put-Beziehungen reduziert wird.
\item [(3)] Reduktion der Beziehungen im System selbst auf „kausal
  wesentliche“ Beziehungen.
\end{itemize}
Weiter wird ebenda festgestellt, dass einer solchen reduktiven
Beschreibungsleistung vorgefundene (explizite oder implizite)
Beschreibungsleistungen vorgängig sind:
\begin{enumerate}
\item[(1)] Eine wenigstens vage Vorstellung über die (funktionierenden)
  Input/Output-Leistun"|gen der Umgebung.
\item[(2)] Eine deutliche Vorstellung über das innere Funktionieren der
  Komponenten (über die reine Spezifikation hinaus).
\item[(3)] Eine wenigstens vage Vorstellung über Kausalitäten im System
  selbst, also eine der detaillierten Modellierung vorgängige, bereits
  vorgefundene Vorstellung von Kausalität im gegebenen Kontext.
\end{enumerate}
Die Punkte (1) und (2) können ihrerseits in systemtheoretischen Ansätzen für
die Beschreibung der „Umwelt“ sowie der Komponenten (als Untersysteme)
entwickelt werden, womit die Beschreibung von \emph{Koevolutionsszenarien}
wichtig wird, die ihrerseits für die Vertiefung des Verständnisses von Punkt
(3) relevant sind.

Damit ist der Systembegriff \emph{strukturell} hinreichend umrissen, der im
Weiteren sowohl der Beschrei"|bungs- als auch der Vollzugsform zu Grunde
gelegt wird.

\section{Systemdynamik}

Es bleibt die \emph{prozessuale} Dimension jenes Begriffs genauer zu umreißen,
wie sie auch in der mathematischen Theorie Dynamischer Systeme eine Rolle
spielt. Der prinzipiell reduktionistische Charakter der Beschreibungsform
zwingt dazu, eine Differenz zwischen Theorie und Praxis einzubauen als
Differenz zwischen theoretischer Vorhersage $v(t)$ und praktischer Entwicklung
$p(t)$ der Prozesse selbst.  Beides kann nur in der \emph{Beschreibungsform}
verglichen werden, in der sich auf diese Weise $v(t)$ als \emph{begründete
  Erwartungen} und $p(t)$ als die -- auf die Beschreibungsform reduzierten --
\emph{erfahrenen Ergebnisse} im \emph{Modell} treffen.  Wir gehen dabei wie in
der Theorie Dynamischer Systeme üblich von einem Phasenraum $\Phi$ aus, in dem
sich die beiden Prozesse $v:T\longrightarrow\Phi$ und $p:T\longrightarrow\Phi$
zeitlich entwickeln, wobei wir $\Phi$ als metrischen Raum voraussetzen, um
eine Aussage über die \emph{Größe der Abweichung} $d(t)=dist(v(t),p(t))$
zwischen Vorhersage und realer Entwicklung treffen zu können.

\begin{wrapfigure}[17]{r}{.35\textwidth}
\begin{center}
  \begin{tikzpicture}[scale=.48,line width=1pt]
    \node (A0) at (0,10) {};
    \node (A4) at (3,8.5) {};
    \node (A1) at (5,7) {};
    \node (A2) at (0,5) {};
    \node (A3) at (7,0) {};
    \node (A5) at (7,6) {};
    \node (A6) at (6,.5) {};
    \draw plot [smooth] coordinates {(A0) (A4) (A1) (A2) (A6) (A3)};
    \node[draw=red] at (4.6,9.6) [rectangle] {$r$-Phase};
    \draw[<->] (2.3,8.1) -- (3.1,7.7) ;
    \node[draw=red] at (7,8.3) [rectangle] {$K$-Phase};
    \draw[<->] (5.4,7.3) -- (6.2,7.3) ;
    \node[draw=red] at (9.5,6) [rectangle] {$\Omega$-Phase};
    \draw[->] (6.6,6.9) -- (7.2,6.5) ;
    \node[draw=red] at (9,4) [rectangle] {$\alpha$-Phase};
    \draw[->] (6.4,4.4) -- (6.3,3.8) ;
    \draw[<->] (5.2,.3) -- (6.2,-.2) ;
    \draw[fill=green] (6,.5) circle (6pt) ;
    \draw[->,dashed] plot [smooth] coordinates {(A1) (A5) (A6)};
    \draw[fill=green] (0,10) circle (6pt) ;
    \draw[fill=green] (3,8.5) circle (6pt) ;
    \draw[fill=green] (5,7) circle (6pt) ;
    \draw[fill=green] (7,6) circle (6pt) ;
    \node[draw=red,fill=white] at (4.5,1.7) [rectangle] {neue $r$-Phase};
  \end{tikzpicture}\\[1em]
  Attraktor-Interpretation von Hollings Phasenmodell
\end{center}
\end{wrapfigure}

Wir gehen weiter davon aus, dass $v(t)$ durch \emph{Bewegungsgleichungen}
beschrieben werden kann, die annähernd ein zeitliches \emph{Fortschreiten} des
Prozesses darstellen und deren Lösungen sich in der Nähe eines
Fließgleichgewichts (Attraktors) stabilisieren.  Aus der Theorie Dynamischer
Systeme ist bekannt, dass die geometrische Form jener Attraktoren selbst für
einfache dynamische Systeme hinreichend kompliziert werden kann.

Wir gehen in Analogie zu Holling in \cite{Holling2001} auch davon aus, dass
sich die Systemdynamik $p(t)$ durch die Wirkung entsprechender Rückstellkräfte
in der Regel in der Nähe eines solchen Attraktors bewegt und damit die
\emph{Störung} $d(t)$ klein bleibt, so lange Spielraum auf dem Attraktor
selbst gegeben ist (Hollings $r$-Phase). Dieses Entwicklungspotenzial
erschöpft sich, wenn das System in ein lokales Extremum des Attraktors
hineinläuft -- auf Störungen kann nur noch mit Rückkehr zum selben
Referenzpunkt auf dem Attraktor reagiert werden (Hollings $K$-Phase). Damit
schaukeln sich Störungen auf, der Systemzustand entfernt sich weiter vom
Attraktor, die Nahwirkung der Rückstellkräfte versagt und das System begibt
sich auf die „Suche“ nach einem neuen, oft weiter entfernten Referenzpunkt auf
dem Attraktor (Hollings $\Omega$-Phase). Auf diesem entfernten neuen
Referenzpunkt ist ein Umbau der Systemdynamik entsprechend den neuen
Parametern erforderlich (Hollings $\alpha$-Phase), um dann wieder in eine
längere stabile Entwicklungsphase (Hollings nächste $r$-Phase) einzutreten.


  Kern der Problematik systemischer Transistionskonzepte ist die Frage, in
welchem Umfang sich derartige Umbauprozesse von Systemen in Kausal-Netzen
miteinander verbundener Systeme fortpflanzen, wobei jenes Netz von Systemen
aus einer doppelten Reduktion der realweltlichen Totalität entspringt -- nicht
nur einer Reduktion der Komplexität der Beschreibungsform, sondern auch einer
Strukturierung der Vollzugsform, die wir entsprechend den Vorgaben der
Beschreibungsform, der begründeten Erwartungen, \emph{versuchen}, im
kooperativen Handeln gemeinsam zu gestalten.


\section{Transitionspfade}

In \cite{Geels2007} werden eine Reihe von Transitionspfadtypen beschrieben,
die in Umbauphasen von Systemen beschritten werden. Dies kann als Versuch
gewertet werden, etwas Struktur in die in \cite{Holling2001} weitgehend
unverstandene $\Omega$-$\alpha$-Umbauphase zu bringen. Auch \cite{Geels2007}
bleibt dabei weitgehend auf einer phänomenologischen Ebene stehen und
entwickelt wenig Konzeptionelles, gesellschaftliche, ökonomische und
technische Entwicklungen zusammen zu denken. Der Aufsatz geht auch nicht so
weit wie die TRIZ Evolutionsforschung, hier Gesetze oder wenigstens Muster
\emph{explizit} zu formulieren.

In dem im letzten Abschnitt entwickelten Verständnis ist die Notwendigkeit zum
Systemumbau dadurch gegeben, dass die \emph{lokalen} Entwicklungsmöglichkeiten
auf dem Systemattraktor ausgeschöpft sind, weil sich das System durch ständig
fortschreitende „Idealisierung“ in ein lokales Extremum des Attraktors
manövriert hat (Hollings $K$-Phase), in dem sich externe Störungen
aufschaukeln und das System in einen instabilen Zustand treiben (Hollings
$\Omega$-Phase), aus dem durch Umorganisation (Hollings $\alpha$-Phase) ein
neuer, vom ursprünglichen weit entfernter Referenzpunkt auf dem
Systemattraktor eingenommen wird.

Ein solcher Systemumbau übt einen größeren Stress auf die mit dem System
verbundenen weiteren Systeme (Komponenten im System, Geschwisterkomponenten im
Obersystem, allgemeine „unsystematische“ Beziehungen zu anderen Systemen) aus.
In diesem Sinne migrieren systemische Umbauprozesse längs der kausalen
Systembeziehungen mehr oder weniger weit durch das Netzwerk der Systeme.

Umgekehrt resultiert der Störungsstress aus anderen, mit dem System kausal
verbundenen Systemen, wobei in den klassischen Ansätzen die Bindungen System
-- Obersystem (bzw. System -- „Umwelt“) sowie System -- Komponente in der
Regel separat von allgemeinen Bindungen (etwa zwischen den Komponenten
innerhalb eines Systems bzw. -- dasselbe Bild auf einer anderen
Betrachtungsebene -- zwischen Teilsystemen eines Obersystems) betrachtet
werden. In \cite{Graebe2020a} hatten wir bereits festgestellt, dass eine
solche „Spezialbetrachtung“ einer Mikro- und Makroevolution nur bei
Beziehungen zwischen Systemen sinnvoll ist, die sich auf deutlich
verschiedenen Eigenzeitskalen bewegen: Für das „schnellere“ System kann das
langsamere in erster Näherung als statisch betrachtet werden, für das
„langsamere“ das schnellere als weitgehend störungsfrei und damit
deterministisch oder wenigstens stochastisch, da sich die Störungen des
schnellen Systems auf der Zeitskala des langsameren weitgehend ausmitteln.

Wir folgen auch hier einem Kausalmodell, in dem die System-Obersystem-Relation
nicht herausgehoben ist, sondern durch ein Netzwerk kausaler Abhängigkeiten
als gerichteter Graph ersetzt wird. Dies vereinfacht insbesondere den Prozess
des Trimmens (TRIZ-Trend 3 in \cite{TESE2018}), ersetzt aber das \emph{eine}
Obersystem durch die Möglichkeit, \emph{mehrere} kausal vorgängige Systeme als
„Obersysteme“ zu identifizieren und entsprechende \emph{Zweck}-Relationen zu
postulieren.  Wenn wir im Weiteren dennoch von System-Obersystem-Rela"|tionen
sprechen, dann stets in dem Sinne, dass wir \emph{eine} dieser kausalen
Zweckbindungen herausnehmen und separat betrachten. 

Betrachten wir aus einer solchen Perspektive die Argumente aus
\cite{Geels2007} und \cite{Holling2001}, so fällt zunächst der stark
agentenbasierte Ansatz der ersteren Arbeit auf. Agenten gibt es auch bei
Holling, siehe etwa \cite[Tab. 2]{Holling2001}, doch setzt \cite{Geels2007}
mit „Agentur“, „Regime“, „Organisation“ und „Institution“ den Fokus deutlich
anders.  Mit allen vier Begriffen, die weitgehend synonym verwendet werden,
wird auf die Ablauforganisation und nicht die Aufbauorganisation von Systemen
verwiesen, ohne allerdings die betrachteten Systeme in jedem Fall genau zu
umreißen. Eher ergeben sich das System und seine Grenzen in den von uns
identifizierten drei (oder vier) Reduktionsdimensionen von
Beschreibungskomplexität „von selbst“ aus der Bewegung heraus.

In einem solchen „panta rhei“ Ansatz werden \cite[S. 401]{Geels2007}
Störungsquelle und Ort des Umbaus differenziert, was mit den oben noch einmal
entwickelten eigenen Modellansätzen gut harmoniert. Die auf dieser Basis
zunächst entwickelte Typologie \cite[Fig. 2]{Geels2007} entspringt allerdings
einer Empirie, die sich nur schwer auf unseren Modellansatz abbilden lässt,
was dann auch später \cite[S. 402]{Geels2007} eingeräumt wird: „Die
empirischen Ebenen sind nicht dieselben wie die analytischen Ebenen im MLP“,
dem im sozio-ökologischen Kontext weit verbreiteten und in \cite{Geels2007}
verwendeten Ansatz einer \emph{Multi-Ebenen-Perspektive}.

Die weiter ins Feld geführten „Organisationsebenen“ -- individuell,
Teilorganisation, Organisation, Population von Organisationen,
organisationales Feld, Gesellschaft, Weltsystem -- konzentrieren sich, wenn
dies mit dem Systembegriff relatiert wird, vor allem auf die
institutionalisierten Strukturen der \emph{Aufbauorganisation} der jeweiligen
Systeme (etwa das „System Gesellschaft“) samt ihrer Luhmannschen „Codes“, in
denen jene Systeme überhaupt sprachlich \emph{in der Lage sind}, über
Störungen zu kommunizieren und wenigstens grob zu entscheiden, ob man es mit
einem „inkrementellen, radikalen, systemischen or techno-ökonomischen“ Typ von
Störung aka „Innovation“ zu tun hat und wie darauf typangemessen zu reagieren
ist.

Wenn eine „Verknüpfung multipler Entwicklungen“ \cite[3.2.]{Geels2007}
bedeutsam ist, so wird die These von der Quelle der Störung in einem
Einzelsystem schon fragil, wenn sich jene Störung im Netzwerk der Systeme
wellenförmig fortpflanzt und so kaum noch zu unterscheiden ist, ob jene
„Welle“ von einer punktförmigen Quelle ausgelöst wurde oder ein emergentes
Phänomen des Netzwerks ist (das ja selbst auch wieder als System betrachtet
werden kann) als resonante Antwort auf eine externe Störung. Dass gerade in
Zeiten tiefgreifender technologischer Umbrüche derartige emergenten Phänomene
in komplexen hierarchisch aufgebauten organisationalen Netzwerken nicht außer
Betracht bleiben können, ist ebenso klar wie theoretisch schwierig zu fassen.

Erschwerend kommt hinzu, dass in derartigen Transitionen drei Sphären
wesentlich interagieren:
\begin{itemize}
\item[1.] Die Sphäre der Beschreibungsformen (das gesellschaftlich verfügbare
  Verfahrenswissen),
\item[2.] Die Sphäre der real existierenden, in Systemen strukturierten
  Wirklichkeit (die institutionalisierten Verfahrensweisen) und
\item[3.] Die kooperativen Subjekte (mit ihrem „privaten“ Verfahrenskönnen).
\end{itemize}
Zwischen den Sphären 1 und 2 bestehen \emph{kausale}
$m:n$-Beziehungen\footnote{Beschreibungsformen orientieren sich am Prinzip der
  \emph{Einheit in der Vielfalt}, die Vollzugsformen bündeln Vielfalt solcher
  analytischer Einheiten und gewinnen damit \emph{Vielfalt aus Einheit}
  zurück.  Ich komme am Ende dieses Aufsatzes auf diese Frage zurück.}, durch
Sphäre 3 werden diese Beziehungen \emph{praktisch} vermittelt.

Die drei „Regelungsarten“ (\cite[3.3.]{Geels2007} -- der Begriff „Institution“
wird in \cite{Geels2007} bewusst abgewählt -- ebenda, S. 403, Fußnote 1), über
welche eine solche Vermittlung in einem „Agentenmodell“ läuft, werden als
Basis einer gemeinsamen „Weltinterpretation“ konkreter kooperativer Subjekte
identifiziert, die sich im \emph{Handeln} jener Strukturen
(„Regelanwendungen“, „Regeln beschränken nicht nur, sondern ermöglichen auch“)
bewähren müssen und befestigt werden.  Dies sind die Formen, in denen die
\emph{Pragmatik} zwischen den Sphären 1 und 2 vermittelt und damit
\emph{realweltliche Begriffsbildungsprozesse} induziert werden bis hin zur
„Konzeptualisierung soziotechnischer Landschaften, die einen externen Kontext
formen, den Akteure kurzfristig nicht beeinflussen können“.

Damit werden die Argumentationen in \cite[Fig. 4]{Geels2007} in ihrem
absoluten Anspruch einer „Änderung der Umwelt“ fragwürdig, da Einträge wie
„niedrig“ und „hoch“ \cite[Tabelle 1]{Geels2007} nur gegen klare Etalongrößen
Sinn ergeben, hier also implizit Eigenzeiten und Eigenräume eines Obersystems
als Referenz dienen (bzw., wenn man sich wie ebenda allein an der
Ablauforganisation interagierender Systeme orientiert, ein solches
Referenzsystem erst noch identifiziert werden muss). Dass jenes „environmental
system“ seit wenigstens 10\,000 Jahren als kulturell überformt betrachtet
werden muss, sei nur in Parenthese angemerkt. Eine solche Einhegung wird dann
mit den Begriffen \emph{Frame} und \emph{Closure} \cite[S. 405]{Geels2007}
auch versucht, jedoch auf einem recht simplen Niveau unmittelbar
transformierender Wirkung differierender Wachstumsraten, siehe auch den
TRIZ-Trend 9 der ungleichen Entwicklung von Systemkomponenten in
\cite{TESE2018}.  In anderen Beispielen wird jedoch gezeigt, dass
Ungleichheiten in der Ressourcenverfügung von Akteuren auch oft eingesetzt
werden, um anstehende Transitionen \emph{zu verhindern}.  Der emergente Effekt
ist dann mglw. eine sinkende Performanz des Gesamtsystems.  Selbst der
beschriebene Wettbewerb auf der Basis differierender Wachstumsraten kann auf
der Emergenzebene des Gesamtsystems gegenteilig wirken, wie etwa Marx mit
seinem Gesetz der fallenden Profitrate argumentiert (egal, ob dieses Gesetz
nun wirklich wirkt oder in einem dissipativen Systemkontext die Argumente
anders zu bedenken sind).

Damit lassen sich die sechs Transitionsmuster P0 bis P5 aus \cite{Geels2007}
wie folgt auf Hollings Modell adaptiver Zyklen abbilden:

\paragraph{P0:}
Das System ist in der $r$-Phase und kann den Veränderungsdruck aus einer
seiner Komponenten („kein erxterner Druck aus der Landschaft“) absorbieren.
Dasselbe bleibt richtig, wenn der Druck „von außen“ (also von anderen
Systemen) kommt und nicht zu groß wird.

\paragraph{P1:}
Druck von „außen“, kein Druck aus den Komponenten, System beim Verlassen oder
jenseits der $K$-Phase.  Das System kann nur durch Reorganisation der
Beziehungen reagieren.  Die Autoren sind weitgehend ratlos, vermischen
allerdings auch zwei Modi:
\begin{itemize}
\item[1.] Das System ist bereits in der $\alpha$-Phase eigener Umbauprozesse.
\item[2.] Das System ist im Übergang in die $\Omega$-Phase.
\end{itemize}
Das Beispiel (Dänische Hygiene-Transition; es geht um die flächendeckende
Ablösung von Klärgruben durch den Ausbau einer Kanalisation) ist klar eines
für die Dynamik in der $\Omega$-Phase, dem auf der TRIZ-Seite ein Übergang von
einer S-Kurve auf eine andere entspricht. Wie das geht, versteht man dort
allerdings auch nicht.  Das System wird reorganisiert, die Funktion nach außen
bleibt erhalten bzw. wird verbessert.

\paragraph{P2:}
Das System wird zerlegt, seine Komponenten anders reorganisiert.  Als
typisches Phäno"|men wird „Vakuum“ diagnostiziert, wie es auch als Machtvakuum
beim Zerfall des Ostblocks zu beobachten war. Das im Text angegebene Beispiel
berücksichtigt nicht, dass sich die neuen Bedingungen (Automobil ersetzt
Transport durch Pferde) bereits länger auch strukturell in den Subsystemen --
„im Schoße der alten Gesellschaft“ -- herausgebildet haben. Dabei bleibt die
Kondratjew-Wellen-Dynamik um 1890 unberücksichtigt, mit der ein ganzes
\emph{Bündel neuer Technologien} auf elektrischer und chemischer Basis zum
Durchbruch kommt.

\paragraph{P3:}
Der Druck kommt nicht aus der Umgebung, sondern von einzelnen Komponenten. Das
System kann sich selbst so reorganisieren, dass die für die reorganisierten
Komponenten erforderlichen neuen äußeren Bedingungen sichergestellt werden,
ohne die Funktionalität des Systems nach außen aufzugeben.  Das
Erklärungspotenzial ist dünn, „lawinenartige Umbauprozesse“ und „disruptive
Wechsel“ als “Druck aus der Landschaft“ existieren erstens dauernd als
„Störungen“ und sind zweitens hier nicht kausal, wenn auch möglicherweise
triggernd. Im Beispiel bleibt die Wirkung der Kondratjew-Welle um 1890
ebenfalls unberücksichtigt.  Ebenso werden für solche Transitionen typische
„Marktbereinigungen“ nicht besprochen, da das produktive Ausrollen der neuen
Technologien in größerem Umfang auch größere Mengen vorgeschossenen Kapitals
erfordert, was zur Masseninsolvenz kapitalschwacher Akteure führt.

\paragraph{P4:}
Komponenten in $\Omega$-Phase treffen auf ein System in $\alpha$-Phase.
Eigentlich wird die Transition aber aus einer kausal tiefer liegenden
Technologieebene getriggert, die Auswirkungen auf \emph{viele} Komponenten hat
und diese in $\Omega$-Phase bringt, was jedoch vom System in $\alpha$-Phase
(und damit in besonders flexibler $r$-Phase) aufgefangen werden kann. So auch
das Beispiel, in dem der Übergang der amerikanischen Wirtschaft von einer
Dominanz komplex organisierter mittelständischer Firmen zu einem System der
Massenproduktion beschrieben wird, das durch tieferliegende Umbauprozesse wie
Arbeitsteilung, Mechanisierung und Maschineneinsatz getrieben wurde, nach
denen mittelständische Firmen in zentralen Bereichen einfach nicht mehr
wettbewerbsfähig waren.

\paragraph{P5:}
Im Gegensatz zu P4 lassen sich die Änderungen \emph{nicht} im System auffangen
und werden weitergeleitet. Damit werden auch die Beziehungen des Systems nach
außen instabil.  Die Autoren sind ziemlich ratlos („eine Folge von
Umbaupfaden“) und haben auch kein Beispiel zur Hand.

Generell wird angemerkt, dass derart komplexe Prozesse nicht nur nicht
monokausal erklärt werden können, sondern auch die Variablen in mathematischen
Beschreibungsmodellen sich nicht in abhängige und unabhängige unterteilen
lassen.  Deshalb könne man nur von \emph{Entwicklungsmustern} sprechen.  Die
weiter referenzierten Prozess-Theorien blenden mit einer Fokussierung auf
Ereignisketten in zeitlicher und kausaler Verkettung allerdings
\emph{strukturelle Momente} weitgehend aus, die sich mit fortgeschrittenen
mathematischen Methoden durchaus auch in komplexer strukturierten Phasenräumen
noch gewinnen lassen.

Giddens' Ansatz der „Regeln als Strukturen, die rekursiv von Akteuren
reproduziert (genutzt, verändert) werden“ -- \cite[S. 415]{Geels2007} mit
Verweis auf \cite{Giddens1984} -- weist in eine Richtung, in der solche
strukturellen Erkenntnisse mit Beschreibungen von Handlungsvollzugsformen
konkreter kooperativer Subjekte auf verschiedenen Abstraktionsebenen zu
kombinieren wären, verlangt aber zugleich eine deutlich weitergehende
Dynamisierung auch in der Beschreibungsform, um die damit verbundenen
nichtlinearen Rückkopplungseffekte sprachlich zu fassen.

\section{Adaptives und transitionales Management}

Die im letzten Abschnitt diskutierten Transitionspfade haben ein wesentliches
epistemisches Problem -- das Problem des äußeren („göttlichen“) Standpunkts,
von dem aus Beschreibungsformen entwickelt werden, um Einfluss auf
realweltliche Wandlungsprozesse zu gewinnen.

\cite{Foxon2009} schlägt hier einen komplett anderen Zugang vor, indem diese
Beschreibungs- und Analyseformen von den beteiligten Akteuren (mit
methodischer Unterstützung) selbst entwickelt werden. Der Zugang folgt dennoch
klassischen TRIZ-Methodiken der Modellierung, indem zunächst ein Obersystem
als Kontext der Bestimmung der Zwecke des untersuchten Systems identifiziert
wird, um dann das System selbst genauer zu modellieren. Jene Modellierung wird
aber nicht als externer Prozess verstanden, sondern als Konsensfindung
gemeinsamer Beschreibungsformen der Stakeholder selbst, ohne welche
kooperatives Agieren nicht möglich ist (siehe das Konzertbeispiel in
\cite{Graebe2020b}).  Dieser Modellierungsprozess wird damit zugleich zum
\emph{politischen} Prozess, da als Ergebnis nicht nur anerkannte
Beschreibungsformen erwartet werden, sondern \emph{institutionalisierte
  Verfahrensweisen}. Ersteres (anerkannte Beschreibungsformen) ist zweiterem
allerdings vorgängig in dem Sinn, dass widersprüchliche Anforderungen zunächst
artikuliert werden müssen, ehe diese Widersprüche gelöst werden können. Dies
entspricht aber auch den zwei Phasen des TRIZ-Prozesses.

In einer solchen Modellierung sind zwei dialektische Prinzipien bereits
eingebaut
\begin{itemize}
\item[(A)] die dynamische Weiterentwicklung des Modells selbst längs der
  Differenzen zwischen begründeten Erwartungen und erfahrenen Ergebnissen der
  Vollzugsform -- unter Einbeziehung einer möglichst breiten
  Stakeholder-Landschaft (TRIZ-Trend der Vollständigkeit der Teile des
  Systems) und
\item[(B)] die Weiterentwicklung der Zwecke im Obersystem, in dem das System
  selbst als Komponente („Stakeholder“) erscheint und dort über seine
  spezifizierte Schnittstelle seinen Beitrag dazu in der Vollzugform
  einbringen kann.
\end{itemize}
Ersteres ist Schwerpunkt des Ansatzes \emph{Adaptives Management}, zweiteres
des Ansatzes \emph{Transitionales Management}.  In beiden Fällen ist die
Weiterentwicklung der Beschreibungsform Teil der Vollzugsform.

Damit ist \cite{Foxon2009} in gewissem Sinne orthogonal zu \cite{Geels2007},
indem \emph{das Innere} einer Transitionsphase in ein methodisches Gerüst
gebracht wird.  Es steht natürlich sofort die Frage, für welche der
Transitionstypen in \cite{Geels2007} dieses methodische Gerüst brauchbar ist
oder ob auch hier wiederum ein Konzept als „Allzweckwaffe“ vorgeschlagen wird.

Beide Ansätze unterscheiden sich weiter in der Strategie der
Komplexitätsreduktion.  Wäh"|rend adaptives Management eine Vielzahl
\emph{verschiedener} funktionaler Parameter in der konkreten Ausprägung im
lokalen Kontext eines \emph{Unikats} betrachtet, erfolgt die Reduktion auf der
Ebene des \emph{transitionalen Managements} auf der Basis eines
\emph{funktionalen Prinzips}, nach dem \emph{gleichartige} funktionale
Parameter gebündelt werden (etwa „Energieversorgung der Zukunft“,
„Wasserreinhaltung“, „Biodiversität“), um dieses Prinzip genauer und besser zu
verstehen. Während zweiteres also mehr der Devise „global denken“ folgt, steht
ersteres in der Perspektive „lokal handeln“.

Ein solches Phänomen der verschiedenen Bündelung hatten wir bereits oben im
Kausalverhältnis der Sphären 1 und 2 (der Beschreibungsformen und der
systemisch strukturierten Wirklichkeit) angetroffen. Dieses Phänomen ist auch
aus der Komponententechnologie \cite{Szyperski2002} gut bekannt -- der
\emph{Zuschnitt} von Komponenten erfolgt unter Bündelung \emph{gleichartiger}
Anforderungen aus \emph{verschiedenen} Quellen, der \emph{Einsatz} von
Komponenten erfolgt durch Bündelung \emph{verschiedenartiger} Funktionalitäten
im \emph{gleichen} Zielsystem.  \cite{Szyperski2002} zeigt, dass dies bis hin
zur Ausdifferenzierung von Berufsbildern verfolgt werden kann --
Komponentenentwickler erscheinen im „design for component“ als
Fachspezialisten, Komponentenmonteure im „design from component“ als
Generalisten.

Auch dies hat sein Analogon in der TRIZ-Methodik, wo „global denken“ den
Schritt von der abstrakten Problemstellung zur abstrakten Lösung markiert, die
man im besten Fall bereits als „technische Komponente“ (nach Deployment und
Konfiguration) in konkrete Lösungen einbauen kann, in den meisten Fällen aber
noch eine klare Konkretisierung auf die komplexe und einzigartige
\emph{realweltliche} Problemsituation erforderlich ist.  Wir haben also auch
auf dieser Ebene dieselbe Unterscheidung wie die zwischen Komponentenbauern
(„design for component“) und Industrieanlagenbauern („design from component“)
im Technikbereich.

Der Aufsatz bricht damit eine Lanze für die Koevolution von Beschreibungsform
und Vollzugsform in kooperativen Zusammenhängen. Beides ist nicht
widerspruchsfrei, allerdings kann versucht werden, artikulierte Widersprüche
mit entsprechenden Transitionsstrategien im Netz der Systeme bewusst an eine
solche Stelle zu verschieben, wo sie gelöst werden können.

\section{Transformationsszenarien und TRIZ}

Es bleibt genauer zu verstehen, wie Transformationsszenarien im Kontext der
TRIZ-Metho"|dik konzeptualisiert werden.  Zunächst ist dazu zu bemerken, dass
das Transformationskonzept in der TRIZ eine relativ zentrale Rolle spielt,
denn die Lösung einer widersprüchlichen Anforderungssituation, die sich in
einem systemischen Kontext ergeben hat, besteht in einer geeigneten
Transformation dieses systemischen Kontexts in einen Zustand, in dem der
Widerspruch aufgelöst ist.  Die TRIZ-Methodik hilft dabei, einen solchen
Transformationspfad auf systematische Weise zu finden.

Dieser Ansatz unterscheidet sich in zwei Dimensionen wesentlich von den bisher
betrachteten:
\begin{itemize}
\item[1.] Es geht um die \emph{praktische} Vollzugsdimension einer solchen
  Transformation.
\item[2.] Der Zugang ist problemgetrieben und nicht analysegetrieben.
\end{itemize}
Letzteres (die Analyse) beginnt mit dem Thema „TRIZ und Business“ (wieder)
eine größere Rolle zu spielen, indem praktische Transitionserfahrungen
analytisch aufgearbeitet und systematisiert werden. Damit nähert sich die
TRIZ-Welt der Transitionsforschung in sozio-ökologi"|schen Systemen weiter an,
wobei weiterhin ein wesentlicher Unterschied im Theorie-Empirie-Verhältnis
zwischen beiden Communities besteht.

\cite{Mann2019} ist ein Versuch, auf der Seite der TRIZ-Welt etwas
Theorieboden zu gewinnen.  Der TRIZ-Gegenstand wird zunächst wie folgt
charakterisiert: „TRIZ is essentially a distillation of the 'first principles'
of problem solving. It was originally developed for complicated technical
problem and opportunity situations and, through ARIZ, has been deeply
optimized for such roles.  Increasingly, however, the world has become
dominated by complex, non-technical situations, and in these environments many
of the tools, methods and processes of traditional TRIZ become highly
inappropriate.“ Weiter heißt es auf Seite 2 „Traditional TRIZ was very much
focused on technical problems.  And moreover, the large majority of these
technical problems turned out to be complicated. And so traditional TRIZ
worked. In today’s massively inter-connected world, however, it is
increasingly rare that we find ourselves able to ‘merely’ focus on just the
technical problem“.  Damit werden die Problemlösekapazitäten von TRIZ als
erfinderisches Wirken in \emph{jungen} Technologien noch einigermaßen korrekt
beschrieben. Dies gilt allerdings schon nicht mehr für die meisten der
heutigen TRIZ-Praxen, die sich auf Problemlösungen (auch ingenieur-technischer
Art) in \emph{funktionierenden unternehmerischen Kontexten} beziehen und damit
neben der Lösung des technischen Problems auch die Implementierung dieser
Lösung in den unternehmerischen Kontext im Auge haben müssen. Damit werden
Systeme aber zu sozio-technischen Systemen, denn Zwecke, Ziele,
Business-Strategien und Interessen geraten ins Blickfeld.  Eine solche
Erweiterung des Gesichtsfelds von rein ingenenieur-technischen zu
sozio-technischen Fragestellungen war auch schon Thema der
DDR-Erfinderschulen, die (u.a.) Probleme des massiven
COCOM-Technologieboykotts und entsprechende Importablösungen zu lösen hatten
\cite{Graebe2019a}.  Derartige Fragen stehen auch heute im Zentrum wichtiger
TRIZ-Anwendungen, nicht zuletzt im Kontext der Patentumgehung.

Allerdings steht die Frage, ob D. Mann mit seiner eigenen Charakterisierung
der TRIZ-Methodik als „Grundprinzipien des Problemlösens“ richtig liegt oder
ob sich diese „Grundprinzipien“ -- selbst in den theoretischen Grundlagen der
TRIZ-Methodik -- nicht doch über \emph{mehrere} Ebenen der Abstraktion
erstrecken, auch wenn dies in den Texten zur theoretischen Fundierung der
TRIZ-Methodik nur selten genauer ausgeleuchtet wird.

Weiter stellt sich die Frage, ob nicht auch im Management-Kontext
\emph{Techniken des Problemlösens}, oder anders -- institutionalisierte
Verfahrensweisen --, im selben Umfang eine Rolle spielen wie beim Lösen rein
ingenieur-technischer Probleme. In strukturierten Kontexten läuft die
Bestellung der nächsten Stahllieferung samt Rechnungslegung und Fakturierung
sicher genauso ARIZ-artig ab wie eine ingenieur-technische Entscheidung. Es
gibt also wenig Grund, wie in \cite{Mann2019} Managemententscheidungen per se
der Kategorie \emph{kompliziert} oder gar \emph{komplex} zuzuordnen.  Ich
komme darauf weiter unten zurück.
\newpage

Mit dem Bezug auf eine „Theorie komplexer adaptiver Systeme“ wird der
theoretische Bogen zu \cite{Foxon2009} geschlagen, auch wenn die referenzierte
theoretische Basis mit \cite{Snowden2007} dünn ist.  Der Titel jener Referenz
fokussiert auf „das Fällen von Entscheidungen durch Leiter“ und nicht wie
\cite{Foxon2009} auf partizipative Entscheidungsprozesse (AM) oder
Transitionsmanagement (TM).  Auch das wird weiter unten noch genauer
diskutiert.

Schauen wir uns die Argumente in \cite{Mann2019} im Einzelnen an.  Zunächst
wird am Beispiel von Spulenentwicklungen gezeigt, dass auch in der TRIZ-Welt
Analogielösungen an konkrete Parameterbereiche gebunden sind, deren Grenzen
„disruptive“ Lösungen einfordern, die nur durch Übergang zu anderen
physikalisch-technischen Prinzipien möglich sind. Wir finden also auch in
diesem Bereich die $r$-, $K$-, $\Omega$- und $\alpha$-Phasen aus
\cite{Holling2001}, wobei TRIZ vor allem in der Bewältigung von Übergängen
seine analytischen Stärken ausspielt, in denen wohlfeile Kontexte zu
transzendieren sind. TRIZ bietet hierfür ein größeres Arsenal von abstrakten
Trends, Mustern und Standards an, um Kontexte gezielt zu vergrößern und in
diesem größeren Kontext Transitionspfade zu identifizieren.

Wie bereits weiter oben in der Diskussion um \cite{Geels2007} und
\cite{Foxon2009} steht dabei die Frage, wie allgemeingültig derartige Trends,
Muster und Standards sind.  TRIZ erhebt hier einen sehr universalistischen
Anspruch, der seine historischen Gründe haben mag (siehe dazu
\cite{Gerovitch1996}), aber praktisch nicht zu rechtfertigen ist. Eine
\emph{methodische Kontextualisierung} der TRIZ-Methodik (wann greifen welche
Methoden) ist also angezeigt, und in genau diese Richtung argumentiert
\cite{Mann2019}. Das dort entwickelte Modell ist sehr einfach und stellt
„Komplexität“ von System und Umwelt auf einer vierstufigen Skala ins
Verhältnis, was hier sofort präzisierend als Verhältnis von System und
Obersystem gefasst werden soll. Mit der „Ashby line“ wird dabei ein
spezifisches Komplexitätskonzept aufgerufen, das wir in \cite{Graebe2020a} als
problematisch identifiziert hatten, da es auf reine Kanalkapazitäten setzt und
intelligente Kompressions- und Dekompressionstechniken nicht berücksichtigt.

Gleichwohl können die vier Stadien „einfach“, „kompliziert“, „komplex“ und
„chaotisch“ durchaus verwendet werden, um die Kopplung von
Strukturierungsprozessen in System und Obersystem zu besprechen.  Der Hinweis
„natürliche Kräfte wirken gegen Resilienz“ \cite[Fig. 3]{Mann2019} entspricht
dem Übergang von der $r$- in die $K$-Phase in \cite{Holling2001} und wird auch
ähnlich begründet: Eine \emph{junge} Technologie ist zunächst wenig verstanden
und deshalb „komplex“. Im Zuge der weiteren Entwicklung werden nicht nur die
Beschreibungsformen präziser, sondern auch die institutionalisierten
Verfahrensweisen. Damit werden typische Einsatzszenarien in typischen
Kontexten einfacher, der Gebrauch der Technologie ist nur noch „kompliziert“.
Mit der Weiterentwicklung zu einer \emph{reifen} Technologie differenziert
sich diese Gebrauchsfähigkeit weiter aus und (dies fällt bei D. Mann in Fig. 3
allerdings unter den Tisch) die eine komplizierte Technologie spaltet sich in
eine Vielzahl verschiedener einfacherer technologischer Lösungen für
verschiedene spezifischere Anwendungskontexte.

Als „Quer“-Tendenz (horizontal in Fig. 3) wird der „2. Hauptsatz der
Thermodynamik“ bemüht, um zu begründen, dass sich realweltliche
Kontextualisierungen ändern und damit früher passfähige Lösungen nicht mehr
passen.  Darauf ist angemessen durch Gegenstrategien (Fig. 4) zu reagieren.
Das „Chaos der Welt“, das hier über den 2. Hauptsatz in die Betrachtungen
eingeführt wurde, seine Quelle aber in der reduktiven Qualität der
Beschreibungsform hat, ist selbst strukturiert und rührt (u.a.) aus
Transitionsprozessen an anderen Stellen der „Welt der Systeme“ her mit
unterschiedlicher Anschlussfähigkeit an die im System selbst anstehenden
Transitionen, wie in der Typologie in \cite{Geels2007} genauer entfaltet.

Die „horizontalen Gegenstrategien“ aus Fig. 4 einer Kontextaufspaltung und die
“vertikalen Strategien“ aus Fig. 3 einer weiteren Vereinfachung und
Standardisierung stehen in engem Bezug zueinander und sind eigentlich nur in
ihrer Gemeinsamkeit aus Vereinfachung der Beschreibungsform (Fig. 3) und
Spezialisierung der Vollzugsform (Fig. 4) als sich gegenseitig bedingend
verständlich. Die „vertikalen Gegenstrategien“ aus Fig. 4 entsprechen dem
TRIZ-Trend 4 des „Übergangs zum Makrolevel“ \cite{TESE2018} und damit der
Stabilisierung der Rahmenbedingungen der Vollzugsdimension. Beides
(Diversifizierung im System und Stabilisierung der Rahmenbedingungen) hatten
als wichtige Resilienz-Strategien auch bisher eine Rolle gespielt, um
Transitionen in der Welt der Systeme lokal einzugrenzen. Diversifizierung
bedeutet dabei, das System gegen Änderungen des Kontexts robuster zu machen
und damit Umbauprozesse im Obersystem besser auszuhalten.  Stabilisierung der
Rahmenbedingungen bedeutet den Übergang auf die nächste Abstraktionsebene, auf
der die \emph{Beziehungen} zwischen System und Obersystem(en) zum Gegenstand
systemischer Gestaltung werden. Eine solche Perspektive bleibt komplett
außerhalb des Betrachtungshorizonts von \cite{Mann2019}.  Allerdings wird
„Trend 4“ auch in \cite{TESE2018} anders verstanden.

\section{Management von Transformationen}

M. Rubin (private Kommunikation) betont, dass es aus Sicht der TRIZ-Theorie
„wesentlich und offensichtich“ sei, zwischen \emph{technischen Systemen} und
\emph{sozio-technischen Systemen} zu unterscheiden:
\begin{quote}
  „Bei der Betrachtung eines technischen Systems werden alle bestehenden
  Verbindungen (soziale, wirtschaftliche, politische, Marketing usw.) im
  System ausgeblendet, mit Ausnahme von Objekten und Verbindungen technischer
  Art. Diese äußeren (menschlichen, kulturellen) Verbindungen können durch
  zusätz"|liche Anforderungen oder Beschränkungen an die technischen Objekte
  ersetzt werden.\par Bei der Betrachtung von Systemen als sozio-technische
  werden zusammen mit den technischen Objekten und Zusammenhängen auch soziale
  berücksichtigt.  So werden bei der TRIZ-Analyse etwa von
  Produktionsbetrieben nicht nur das technische System (Maschinen und
  Ausrüstungen) betrachtet, sondern die Fabrik als sozio-technisches Objekt:
  das System der Aufträge und das Marketing, die Personalpolitik, die
  finanziellen und ökonomischen Prozesse, die Systeme der Entscheidungsfindung
  usw.  Es ist offensichtlich, dass dies den Gegenstand der Betrachtung und
  die Instrumente seiner Untersuchung grundlegend verändert.
\end{quote}
\newpage

Diese Position fixiert auf gewisse Weise gängige TRIZ-Praxen als
Beratungsservice: Am Ende der Auseinandersetzung mit einer widersprüchlichen
Anforderungssituation steht ein Bündel von (technischen)
Lösungs\emph{vorschlägen} durch den TRIZ-methodisch geschulte Berater als
\emph{Auftragnehmer}, aus denen einer durch den \emph{Auftraggeber} nach
sozio-technischen Kriterien auszuwählen und praktisch zu implementieren ist,
siehe hierzu im Detail auch \cite{Kozhemyako2019}.  Es stellt sich natürlich
die Frage, wie die hier eingezogenen institutionellen Grenzen auf die Qualität
dieses Entscheidungsprozesses selbst zurückwirken.

In \cite{Snowden2007} werden jene Prozesse von der „anderen Seite“ der
Managementprozesse selbst betrachtet und ein eigenes Modell strukturierten
Vorgehens entwickelt. Derartige Manage"|ment-\emph{Techniken} zeigen die große
Nähe jener Vorgehensweisen zu ingenieur-technischen, was aber nicht weiter
überraschen kann, da strukturierte Vorgehensweisen nicht mit dem Verlassen
eines technischen Bereichs im engeren Sinne enden, wenn man nicht der
neoliberalen Mär von der „unsichtbaren Hand des Marktes“ aufsitzt. Die
Argumente gehen dabei deutlich über \cite{Mann2019}, aber auch
\cite{Foxon2009} und \cite{Geels2007} hinaus, da sich \cite{Snowden2007} nicht
so sehr für die analytische Dimension der \emph{Entscheidungsvorbereitung},
sondern für die prozessuale Dimension der \emph{Entscheidungsfindung}, für ein
„Framework der Entscheidungsfindung“ interessiert. Die vier Systemklassen
einfach, kompliziert, komplex und chaotisch werden dabei verwendet, um
Entscheidungsfindungsprozesse vor allem nach der Qualität der verfügbaren
\emph{Entscheidungsgrundlagen} zu klassifizieren.

Rubins Begriff eines sozio-technischen Systems korrespondiert zu diesem
\emph{System der Entscheidungsfindung}, in das neben rein technischen
Argumenten eine Vielzahl anderer Argumente eingeht, die gegeneinander
abgewogen werden müssen. Dieses System der Entscheidungsfindung bündelt die
oft widersprüchlichen Aussagen und Anforderungen aus verschiedenen anderen
Systemen, das technische System im engeren Sinne von Rubin engeschlossen.
Diese „anderen“ Systeme treten dabei aber sowohl als \emph{Obersysteme} als
auch als \emph{Komponenten} in Erscheinung, wie dies auch schon Anton
Kozhemyako in \cite{Kozhemyako2019} mit anderen Begriff"|lichkeiten
thematisiert hat.  Obersysteme sind sie insoweit, als deren Logik der Logik
der Entscheidungsfindung kausal vorgängig ist, Komponente sind sie insoweit,
als die widersprüchlichen Beziehungen zwischen diesen einzelnen Logiken im
Prozess der Entscheidungsfindung zu thematisieren sind.

Im Sinne unseres Systembegriffs ist dabei das \emph{System der
  Entscheidungsfindung} (SEF) von den verschiedenen \emph{Systemen der
  Entscheidungsvorbereitung} (SEV) zu separieren, um die erforderliche
Komplexitätsreduktion zu erreichen. Das SEF greift dabei auf die Ergebnisse
der SEV über deren Schnittstellen zu und muss die komprimierte Qualität dieser
widersprüchlichen Informationen systemisch prozessieren.  In einem solchen
Verständnis ist Rubins Unterscheidung zwischen technischem und
sozio-technischem System in der Tat „wesentlich und offensichtlich“.
Allerdings werden im sozio-technischen SEF nicht „zusammen mit den technischen
Objekten und Zusammenhänge auch soziale berücksichtigt“, sondern jene
„technischen Objekte und Zusammenhänge“ allein über die \emph{Schnittstelle}
des technischen Systems importiert, mit dem dieses als Komponente im SEF
präsent ist. An dieser Stelle wird die Unterscheidung zwischen einem
immersiven und einem submersiven Systembegriff wesentlich -- das Obersystem
ist nicht durch \emph{mehr Beziehungen} charakterisiert, sondern durch eine
\emph{andere Richtung der Komplexitätsreduktion auf „das Wesentliche“}.  Siehe
dazu \cite{Graebe2020a} im Detail.  

In \cite{Snowden2007} werden hierfür methodische Ratschläge unterbreitet, die
sich allein an der Wahrnahme eines Grades von Widersprüchlichkeit der Signale
aus den Komponenten orientieren. Die Situation ist „einfach“, wenn die
Beschreibungsformen in den Komponenten so weit harmonieren, dass nur
„wahrnehmen, kategorisieren, antworten“ erforderlich ist. Die Situation ist
„kompliziert“, wenn die „Experten“ aus den Komponenten ihre widersprüchlichen
Positionen klar formulieren können und „mindestens eine richtige Antwort
existiert“.  Gefahren werden im „geschlossenen Denken“ einer routinierten
Behandlung und damit Unterschätzung solcher Widersprüche gesehen, als zu
beachtender Ansatz wird „neue Gedanken und Lösungen von anderen begrüßen“
(also kurz: Brainstorming) empfohlen.  Die Situation ist „komplex“, wenn die
Entscheidung im SEF selbst herausgefiltert und formuliert werden muss, sich
die Entscheidung als „emergentes Phänomen“ erst aus der Zusammenschau der
Komponenten ergibt, die \emph{mehr} als die Summe der Teile ist.

\cite{Snowden2007} lässt sich also auch deutlich anders interpretieren als in
\cite{Mann2019}. Eine solche Interpretation eröffnet den Zugang zu einem
besseren Verständnis des Zusammenhangs zwischen den \emph{technischen
  Analyseprozessen} einer klassischen TRIZ und den \emph{unternehmerischen
  Entscheidungsprozessen}, welche zur praktischen Implementierung einer Lösung
des untersuchten Problems zu treffen sind. Diese stehen in \cite{Snowden2007}
aber nicht einmal nebeneinander, da im SEF die systemischen
Entscheidungsprozesse allein auf dem Input der SEV aufsetzen, der in das SEF
über die entsprechenden Schnittstellen der Nachbarsysteme als
\emph{Komponenten des SEF} importiert wird, und im besten Fall ein iteratives
Entscheidungsfindungsmodell eingesetzt wird, das partielle Lösungen über
dieselben Schnittstellen an die Nachbarsysteme zurückspielt, um die partielle
Lösung in den Logiken der SEV noch einmal genauer zu inspizieren und
entsprechende Einwände über die Schnittstelle ins SEF zurückzuspielen.  Das
SEF nimmt damit eine scheinbare Obersystem-Rolle ein, aber nur aus der
internen Sicht des SEF selbst, denn eine solche Koordinierung funktioniert nur
dann, wenn die Systeme im Netz der SEV entsprechend \emph{funktional
  disponiert} sind, um hier anschlussfähig zu sein, wenn also die
koordinierende Anfrage des SEF auf eine Funktion im Nachbarsystem trifft, die
eine Antwort zu erzeugen in der Lage ist. In jedem der Nachbarsysteme des
SEV-Netwzerks ist das SEF seinerseits als Komponente präsent, die Input in
wohldefiniertem Format liefert und Output in ebenso wohldefiniertem Format
erwartet.

Ein wirkliches Obersystem ergibt sich erst aus einer systemischen Betrachtung
der \emph{Beziehungen} zwischen den Systemen des SEV-Netzwerks. Dabei ist aber
die nächste Ebene einer epistemischen Schichtenarchitektur zu erklimmen, in
der nicht das konkrete Problemlösen in diesem konkreten Netzwerk der SEV zu
\emph{prozessieren}, sondern eine größere Anzahl derartiger Problemlösungen zu
\emph{analysieren} ist.  Dieser Prozess der Sprachschöpfung, der in
\cite{Graebe2020b} exemplarisch am Konzertbeispiel dargestellt ist, geht
deutlich über alle hier bisher besprochenen Ansätze hinaus. 

\section{TRIZ und die Entwicklung Technischer Systeme}

Wie harmonieren die hier entwickelten Begriffe \emph{System} und
\emph{technisches System} mit Systembegriffen, die im TRIZ-Umfeld verwendet
werden?  \cite{TESE2018} ist hierfür eine gute Referenz, da die
zusammenfassende Darstellung der „Entwicklungstrends von
inge"|nieur-technischen Systemen“ den Status eines „durch die MATRIZ
autorisierten Lehrbuchs“ hat. Es wird dort explizit der Begriff
\emph{engineering system} gegenüber dem in der sonstigen TRIZ-Literatur,
besonders auch der russisch-sprachigen, üblichen Begriff des \emph{technischen
  Systems} verwendet.

Allerdings finden sich weder in \cite{TESE2018} noch in den anderen Referenzen
genauere Begriffsdefinitionen, was unter einem \emph{technischen System} zu
verstehen sei.  In allen Quellen wird auf die Anschauung verwiesen, wobei die
Facebookdiskussion \cite{Graebe2019b} gezeigt hat, dass diese „Anschauung“
einen weiten Bereich möglicher Interpretation überdeckt.  Allerdings kommt
selbst in jenen Betrachtungen die oben aufgeworfene Frage nicht vor, ob
\emph{Managementtechniken} auch in Systemen technischer Art erfasst werden
können oder hier mit anderen Begriff"|lichkeiten zu operieren sei. Der Rückzug
auf „ingenieur-technische Systeme“ wie in \cite{TESE2018} verschiebt das
Problem nur zur Frage, ob modernes Management- und Verwaltungshandeln nicht
auch eine Ingenieurstätigkeit sei.  Von den Anforderungen an spezifische
Kenntnisse theoretischer Grundlagen, institutionalisierter Verfahrensweisen
und algorithmischer Vorgehensweisen sind diese Tätigkeitsprofile jedenfalls
von Ingenieurstätigkeiten in größeren Unternehmen kaum zu unterscheiden.

Explizite systemtheoretische Ansätze im TRIZ-Umfeld verweisen auf komplexe
Wurzeln in Moskauer philosophischen Kreisen der 1960er bis 1980er Jahre, siehe
hierzu etwa \cite{Kozhemyako2019} und das Gutachten von M. Rubin zu dieser
Arbeit. Davon war offensichtlich auch Altschuller beeinflusst, als er 1984 die
in \cite{TESE2018} referenzierte Liste von acht Gesetzen der Entwicklung
technischer Systeme
\begin{itemize}\itemsep0pt
\item[1.] Gesetz der Vollständigkeit der Teile eines Systems
\item[2.] Gesetz der „Energieleitfähigkeit“ eines Systems
\item[3.] Gesetz der Harmonisierung der Rhythmen der Systemteile
\item[4.] Gesetz der wachsenden Idealität
\item[5.] Gesetz der ungleichmäßigen Entwicklung der Systemteile
\item[6.] Gesetz des Übergangs zum Obersystem
\item[7.] Gesetz des Übergangs von der Makro- zur Mikroebene
\item[8.] Gesetz der wachsenden Stoff-Feld-Interaktionen
\end{itemize}
formulierte. Bereits an dieser Stelle gehen die Darstellungen in
\cite{TESE2018} und \cite{Rubin2019} auseinander. Rubin bezieht sich auf eine
Liste von neun Gesetzen, die Altschuller 1977 in Baku veröffentlicht hat und
das weitere
\begin{itemize}\itemsep0pt
\item[9.] Gesetz der Dynamisierung starrer technischer Systeme
\end{itemize}
enthält, was auch im TRIZ-Prinzip 15 gelistet ist.  Die Abgrenzung von
Gesetzen, Trends, Standards und Prinzipien ist in der TRIZ generell
problematisch.

\cite{Goldovsky1983} scheint eine wichtige Referenz zu sein, welche die
Verbindung zwischen den Ansätzen eines „Schöpfertums als exakter Wissenschaft“
(Altschuller) und philosophischen Überle"|gungen herstellt. In jenen Arbeiten
wird der Gesetzesbegriff strapaziert, um systemische Entwicklungslinien auf
verschiedenen Abstraktionsebenen zu charakterisieren, und es wird der Begriff
\emph{technisches System} in den komplexeren Kontext der Entwicklung
\emph{allgemeiner Systeme} eingebettet.  Auf die Frage, ob es sich um Gesetze
oder eher um Trends oder gar nur um Entwicklungsmuster handelt, soll hier
nicht eingegangen werden.

Sowohl \cite{Goldovsky1983} als auch \cite{Rubin2019} bleiben eine genauere
Fassung auch des allgemeinen Systembegriffs schuldig. Goldovsky thematisiert
eine Hierarchisierung der dort formulierten Gesetze in
\begin{itemize}\itemsep0pt
\item[1.] Grundlegende Entwicklungsmuster
\item[2.] Methodologische Muster der Entwicklung technischer Systeme
\item[3.] Gesetzmäßigkeiten der Herstellung arbeitsfähiger technischer Systeme
\item[4.] Gesetzmäßigkeiten funktioneller Transformationen technischer Systeme
\item[5.] Gesetzmäßigkeiten struktureller Transformationen technischer Systeme
\item[6.] Muster der Transformation der Systemzusammensetzung
\end{itemize}
wobei die formulierten Punkte eher einen metaphysischen Charakter der
Kontextualisierung einer Betrachtungsperspektive haben, und somit doch zur
Schärfung der Begriff"|lichkeit eines „technischen Systems“ beitragen,
insbesondere durch die „methodologischen Muster“ 2.1-2.4.

Diese Hierarchisierung reflektiert in gewisser Weise die Komplexität von
Systemtransformationen und reicht von
\begin{itemize}\itemsep0pt
\item[1.] grundsätzlichen Epistemiken von Beschreibungsformen über
\item[2.] Anforderungen an Beschreibungsformen (an die Modellierung)
  technischer Systeme,
\item[3.] Anforderungen an die Verbindung von Beschreibungs- und
  Vollzugsformen technischer Systeme (Betriebsbedingungen in gegebenem
  Kontext),
\item[4.] Anforderungen an die Lösung von Widersprüchen durch funktionale
  Reorganisation (bei unveränderten Komponenten),
\item[5.] Anforderungen an die Lösung von Widersprüchen durch strukturelle
  Reorganisation (auch die Komponenten werden verändert) bis hin zu
\item[6.] Anforderungen an systemische Reorganisation.
\end{itemize}
Sie deckt damit einen Teil der systemischen Reorganisationserfordernisse ab,
die in \cite{Geels2007} identifiziert werden. Es bleibt weiter auszuloten,
welche tieferliegenden Erkenntnisse aus diesen eher metaphysisch formulierten
Mustern zur Bewältigung \emph{realer} Transitionserfordernisse zu gewinnen
sind.

Altschuller selbst teilt seine Gesetze in statische (1-3), kinematische (4-6)
und dynamische (7-8) und postuliert die Gültigkeit der statischen und
kinematischen Gesetze für die Entwicklung auch allgemeiner Systeme, während er
die dynamischen Gesetze 7-8 als zeit- und domänenspezifisch ansieht.  Diese
Überlegungen werden in \cite{Rubin2019} weiter detailliert. Wie in
\cite{TESE2018} werden die Gesetze in eine baumartige Kausalstruktur gebracht
(präziser: in die Struktur eines gerichteten azyklischen Graphen) und in einem
zweiten Schritt die Verbindung zu den TRIZ-Standards hergestellt, die als
operationale Ausprägung der jeweiligen Gesetze in der TRIZ-Methodik betrachtet
werden. Von dort wird der Bogen weiter zu ARIZ und der Algorithmisierung der
Methodik geschlagen.

Sowohl die Auswahl der Gesetze als auch die genaue Ausgestaltung der kausalen
Beziehungen unterscheiden sich zwischen der Darstellung von Lyubomirsky und
Litvin selbst in \cite[S. 6]{TESE2018}, Rubins Darstellung der Gesetze nach
Lyubomirsky und Litvin \cite[Abb. 1]{Rubin2019} und der eigenen Darstellung
\cite[Abb. 2]{Rubin2019}. Rubin diskutiert weiter die Verbindung dieser
Gesetze zu einer allgemeinen Systemtheorie, für die er 12 Gesetze in 4 Blöcken
formuliert, was weiter zu analysieren bleibt.

Gleichwohl bleibt die Frage, ob ein derartiger Zugang eines \emph{one size
  fits all} allgemeiner Entwicklungsmuster von Systemen gerechtfertigt ist
oder nicht auch hier über eine differenziertere \emph{Methodik der Anwendung
  der TRIZ-Methodik} nachzudenken ist, dabei weitgehend offen. 

\section{Zusammenfassung und Ausblick}

Bleibt die abschließende Frage: Wie weit trägt ein systemtheoretischer Ansatz
überhaupt? Wir hatten eingangs festgestellt, dass es nicht \emph{den}
systemtheoretischen Ansatz gibt, sondern wir mit einem ganzen Universum
aufeinander bezogener Ansätze konfrontiert sind, was zum Begriff der
\emph{Systemwissenschaft} im Titel unseres Seminars \cite{Graebe2020a} Anlass
gab.  \cite{Ropohl2009} lotet dieses Problem weiter aus und identifiziert drei
wesentlich verschiedene Ansätze
\begin{itemize}
\item[1.] das funktionale Konzept eines Systems als „Black Box“,
\item[2.] das strukturelle Konzept der Modellierung von Wechselwirkungen
  zwischen Komponenten und
\item[3.] das hierarchische Konzept einer System-Umwelt-Beziehung.
\end{itemize}
Das hier entwickelte Konzept geht mit der Betrachtung der Einheit von
Beschreibungs- und Vollzugsform einen deutlichen Schritt weiter. Die drei von
Ropohl unterschiedenen An"|sätze werden als drei Reduktionsdimensionen von
Beschreibungsformen identifiziert, die in unserem Systembegriff
\emph{gleichzeitig} wirken. Dabei bekommen insbesondere die unspezifischen
Begriffe „Umwelt“ und „Obersystem“ eine genauere Fassung: Umwelt kann in
diesem Beschreibungsansatz nur selbst wieder als System und damit nicht als
Totalität einfließen.  Allerdings kann ein System in einem solchen Verständnis
auf \emph{mehrere} Obersysteme bezogen sein, womit die
System-Obersystem-Beziehung ihren exklusiven Charakter unter den systemischen
Nachbarschaftsbeziehungen verliert. Auf der anderen Seite ist zwischen
Modellierung und Metamodellierung zu unterscheiden, wobei letztere regelmäßig
bedeutsam wird, wenn es um die systemische Fassung von Beschreibungsformen der
\emph{Beziehungen} zwischen Systemen geht.

Letzteres gibt Anlass zu einer Stratifizierung der Wirklichkeit längs der
Begriffsbildungs"|\mbox{niveaus} der Beschreibungsformen, die als prägend für
hoch technisierte Gesellschaften gelten kann.  Diese
Beschreibungsstratifizierung als spezifische Form der Komplexitäts"|reduktion
(„Fiktion“ in \cite{Graebe2020b}) findet ihre Entsprechung in technischen
Schichtenarchitekturen wie etwa im OSI-7-Schichten-Modell.

Systemische Betrachtungen identifizieren auf der Beschreibungsebene Einheit in
der Vielfalt, aus der in der Vollzugsform wieder Vielfalt zurückgewonnen
werden muss. Menschen sind hier zugleich Subjekt und Objekt von Handeln.  Die
damit verbundenen Widersprüche sind im Prinzip bewusst gestaltbar, enthalten
aber einen weiteren Stolperstein -- Selbstbezüglichkeit. Hier ist
Systemtheorie überfordert und muss in eine Gesellschaftstheorie eingebettet
werden.  \cite{Foxon2009} hatte mit dem partizipativen Ansatz eines adaptiven
Managements in einem Multi-Stakeholder-Kontext eine wichtige Form einer
solchen Einbettung aufgezeigt, die aber mit Managementansätzen wie
\cite{Snowden2007} (und in weiterem Sinne auch \cite{TESE2018}) wieder
aufgeweicht werden.  Systemtheorie bleibt ein wichtiges \emph{Instrument des
  Handelns} in einem solchen Kontext, wenn sie auf vier wesentliche Punkte
ausgerichtet wird:
\begin{itemize}\itemsep0pt
\item[1.] Theoriegeladenheit,
\item[2.] Bewältigung des Ebenenproblems von Beschreibungsformen und
  Begriffsbildungsprozessen,
\item[3.] Bewältigung des Durchsatzproblems: Durchsatz bestimmt das
  Innenverständnis des Systems, das „kooperative Weltbild“, wie in
  \cite{Graebe2020a} genauer entwickelt,
\item[4.] Ausrichtung auf Transition und Transformation, Resilienz und
  Nachhaltigkeit, Dynamik aller Komponenten und Beziehungen.
\end{itemize}

\begin{thebibliography}{xxx}
\bibitem{Foxon2009} Timothy J. Foxon, Mark S. Reed, Lindsay C. Stringer
  (2009). Governing long‐term social–ecological change: what can the adaptive
  management and transition management approaches learn from each other?
  Environmental Policy and Governance, 19 (1),
  3--20. \url{https://doi.org/10.1002/eet.496}
\bibitem{Geels2007} Frank W. Geels, Johan Schot (2007).  Typology of
  Sociotechnical Transition Pathways. In: Research Policy 36 (2007),
  399–417.\\ \url{https://doi.org/10.1016/j.respol.2007.01.003}
\bibitem{Gerovitch1996} Slava Gerovitch (1996). Perestroika of the History of
  Technology and Science in the USSR: Changes in the Discourse. Technology and
  Culture, Vol. 37.1, S. 102--134.
\bibitem{Giddens1984} A. Giddens (1984). The Constitution of Society: Outline
  of the Theory of Structuration. University of California Press, Berkeley.
\bibitem{Goldovsky1983} B.I. Goldovsky (1983).
  \foreignlanguage{russian}{Система закономерностей построения и развития
    технических систем} (System der Gesetzmäßigkeiten des Aufbaus und der
  Entwicklung technischer Systeme).
  \url{https://wumm-project.github.io/Texts.html}
\bibitem{Graebe2019a} Hans-Gert Gräbe (2019).
  \foreignlanguage{russian}{Наследие Движения Школ Изобретателeй в ГДР и
    Развитиe ТРИЗ} (Das Erbe der Erfinderschulbewegung in der DDR und die
  Entwicklung der TRIZ). Erschienen im Online-Protokollband des TRIZ Summit
  2019 Minsk.
\bibitem{Graebe2019b} Hans-Gert Gräbe (2019).  Aufzeichnung einer Diskussion
  über TRIZ und Systemdenken in meinem Open Discovery Blog.\\
  \url{https://wumm-project.github.io/2019-08-07}.
\bibitem{Graebe2020a} Hans-Gert Gräbe (2020). Reader zum
  16. Interdisziplinären Gespräch \emph{Das Konzept Resilienz als emergente
    Eigenschaft in offenen Systemen} am 7.2.2020 in Leipzig.
  \url{http://mint-leipzig.de/2020-02-07/Reader.pdf}.
\bibitem{Graebe2020b} Hans-Gert Gräbe (2020). Die Menschen und ihre
  Technischen Systeme.  LIFIS Online 19.05.2020.  DOI:
  \url{10.14625/graebe_20200519}.
\bibitem{Holling2001} Crawford S. Holling (2001). Understanding the Complexity
  of Economic, Ecological, and Social Systems. In: Ecosystems (2001) 4,
  390–405.
\bibitem{Kozhemyako2019} Anton Kozhemyako (2019).
  \foreignlanguage{russian}{Особенности применения ТРИЗ для решения
    органи\-зационно-управленческих задач: схематизация изобретательской
    ситуации и работа с противоречиями} (Besonderheiten der Anwendung der TRIZ
  bei der Lösung von Organisations- und Managementaufgaben: Schematisierung
  der erfinderischen Situation und die Arbeit mit Widersprüchen).
  Dissertation, eingereicht bei der MATRIZ zur Erlangung des Titels eines TRIZ
  Masters.\\  \url{https://matriz.org/kozhemyako/}
\bibitem{TESE2018} Alexander Lyubomirskiy, Simon Litvin u.a.  (2018). Trends
  of Engineering System Evolution. Sulzbach-Rosenberg.  ISBN
  978-3-00-059846-3.
\bibitem{Mann2019} Darrell Mann (2019).  Systematic innovation in complex
  environments. Proceedings of the TRIZ Summit 2019
  Minsk.\\ \url{https://triz-summit.ru/file.php/id/f304797-file-original.pdf}
\bibitem{Ropohl2009} Günter Ropohl (2009). Allgemeine Technologie: eine
  Systemtheorie der Technik.  KIT Scientific Publishing.
\bibitem{Rubin2019} Michail S. Rubin (2019).  \foreignlanguage{russian}{О
  связи комплекса законов развития систем с ЗРТС} (Zur Verbindung des
  Komplexes der Gesetze der Systementwicklung mit den Gesetzen der Entwicklung
  technischer Systeme). Manuskript, November 2019.
\bibitem{Snowden2007} David J. Snowden, Mary E. Boone (2007).  A Leader’s
  Framework for Decision Making.  Harvard Business Review, November 2007.
\bibitem{Szyperski2002} Clemens Szyperski (2002). Component Software: Beyond
  Object-Oriented Programming. ISBN: 978-0-321-75302-1.
\end{thebibliography}
%\ccnotiz
\end{document}

