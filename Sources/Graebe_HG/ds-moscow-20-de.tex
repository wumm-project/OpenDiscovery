\documentclass[11pt,a4paper]{article}
\usepackage{od}
\usepackage[main=ngerman,russian]{babel}

\title{Zur Evolution Technischer und Allgemeiner Systeme\\[1em] \Large
  Wortmeldung zum\\ Moskauer Workshop 28.-29. November 2020}

\author{Hans-Gert Gräbe, Leipzig}

\date{Version vom 27. November 2020}
\begin{document}
\maketitle

Wir haben uns in diesem Jahr im Rahmen des Leipziger TRIZ-Seminars intensiv
mit Fragen der Entwicklung und Evolution technischer uns allgemeiner Systeme
befasst, siehe etwa
\begin{itemize}[noitemsep]
\item \url{https://wumm-project.github.io/TTS}
\item \cite{Graebe2020a}, \cite{Graebe2020b}, \cite{Graebe2020c},
  \cite{Graebe2020d}
\end{itemize}

Eine der zentralen Fragen, die auch in den für den Moskauer Workshop
vorgelegten Vorträgen\footnote{Siehe \url{https://yadi.sk/d/x1q4pYiPhtBxHw}.}
nicht hinreichend beantwortet wird, ist die nach den Kontinuitätslinien, nach
denen reale technische Systeme angeordnet werden, um überhaupt über so etwas
wie S-Kurven, Entwicklungspfade oder Evolution zu argumentieren.

Im MATRIZ-Standardwerk \cite{TESE2018} ist deutlich, wenn auch nicht
expliziert, dass solche Kontinuitätslinien auf der Basis gängiger
Produktkataloge marktgängiger technischer Systeme erstellt werden.  Damit wird
aber der Begriff des technischen Systems auf solche \emph{Kleinsysteme} (TKS)
eingeschränkt und \emph{technische Großsysteme} (TGS) wie etwa das Minsker
Traktorenwerk (als Konkretum) oder Autofabriken überhaupt (als
Generalisierung) nicht erfasst.

In \cite{Graebe2020b} wird die These entwickelt und begründet, dass solche TGS
einerseits Unikate sind -- womit die Verfolgung von Kontinuitätslinien
eindeutiger wird --, sich aber andererseits nach teilweise eigenen Gesetzen
entwickeln und hierfür die bekannten Gesetze für TKS modifiziert oder ganz
ersetzt werden müssen.

Dieser Gedanke ist in keinem der vorliegenden Beiträge angemessen aufgenommen
worden. Ich möchte ihn deshalb zur Diskussion stellen.

\begin{thebibliography}{xxx}
\bibitem{Graebe2020a} Hans-Gert Gräbe.  TRIZ und Transformationen
  sozio-technischer und sozio-ökologischer Systeme. LIFIS Online,
  27.06.2020.\\ English Translation, June 2020, submitted to TRIZ
  Review.\\ Russian Poster for MATRIZ Online Forum August 2020 at
  \\ \url{https://hg-graebe.de/EigeneTexte/MOF-20-ru.pdf}.
\bibitem{Graebe2020b} Hans-Gert Gräbe. Die Menschen und ihre Technischen
  Systeme. LIFIS Online, 19. Mai 2020. \url{doi:10.14625/graebe_20200519} 
\bibitem{Graebe2020c} Hans-Gert Gräbe.  \foreignlanguage{russian}{Человек и
  его технические системы}.  TRIZ Developer Summit 2020.
  \url{https://hg-graebe.de/EigeneTexte/TDS-2020.pdf}
\bibitem{Graebe2020d} Hans-Gert Gräbe.  Technical Systems and
  Purposes. Materialien der Ersten Deutschen TRIZ Konferenz Online 2020.
  Erscheint bei Springer.
\bibitem{TESE2018} Alexander Lyubomirskiy, Simon Litvin, Sergey Ikovenko,
  Christian M. Thurnes, Robert Adunka (2018). Trends of Engineering System
  Evolution. Sulzbach-Rosenberg.  ISBN 978-3-00-059846-3.
\end{thebibliography}
\end{document}

