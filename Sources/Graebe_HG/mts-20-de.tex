\documentclass[11pt,a4paper]{article}
\usepackage{a4wide,url}
\usepackage[T1,T2A]{fontenc}
\usepackage[utf8]{inputenc}
\usepackage[main=ngerman,russian]{babel}
\usepackage{tikz}
\usetikzlibrary{arrows.meta, positioning}

\newcommand{\bbox}[2]{\parbox{#1cm}{\small\centering #2}}

\parindent0cm
\parskip3pt

\title{Die Menschen und ihre Technischen Systeme} 
\author{Hans-Gert Gräbe, Leipzig}
\date{Version vom 13. Mai 2020}
\begin{document}
\maketitle\vfill
\tableofcontents
\pagebreak
\begin{flushright}
  Die Philosophen haben die Welt nur verschieden interpretiert;\\ es kömmt
  darauf an sie zu verändern.\\ Karl Marx. 11. Feuerbachthese
\end{flushright}

\section{Einführung}

Das Ganze ist mehr als die Summe seiner Teile. In \cite{Graebe2020} ist
ausgeführt, dass es sogar sehr viel mehr als diese Summe ist, denn in den
System\emph{beziehungen} multiplizieren sich die Teile und addieren sich nicht
nur.  Deshalb ist ein submersiver Systembegriff auch deutlich besser geeignet,
die Vielfalt der Welt begriff"|lich zu erfassen als der auch in der
TRIZ-Theorie verbreitete immersive Systembegriff.

Die begriff"|liche Fundierung der eigenen Theorie wird im TRIZ Body of
Knowledge \cite{TBK-2007} nur halbherzig betrieben\footnote{Das mag als hartes
  Urteil erscheinen, stellen sich die Autoren doch selbst die Aufgabe „a.
  Define TRIZ as a Theory — this should indicate TRIZ functions as well as
  current and potential scope of applicability; b.  Name the starting points
  (in other words, postulates) on which TRIZ as a Theory is based; c.  Provide
  a structure of TRIZ as a system, by associating postulates with functions;
  d.  Assess the consistency of this method. Leading TRIZ experts should all
  agree upon this method and verify that everything concurrently satisfying
  items a, b, and c refers to TRIZ.“ Doch der einzige Satz in jenem
  Basisdokument zu dem hier besprochenen Thema ist folgender: „TRIZ methods
  lean upon the objective trends of evolution of systems (predominantly,
  engineering systems)“.}. Insbesondere in der Frage, was denn ein
\emph{technisches System} sei, wird auf die Anschauung verwiesen -- jeder
wisse doch, worüber hier die Rede sei. Die Vielfalt der Verständnisse wurd in
einer Facebook-Diskussion \cite{Graebe2019b} im August 2019 deutlich.  Das ist
natürlich keine wissenschaftliche Arbeitsgrundlage.

In diesem Aufsatz wird eine Annäherung an den Begriff \emph{technisches
  System} versucht und gefragt, ob ein solcher Begriff überhaupt trägt, um die
\emph{Welt der technischen Systeme} genauer zu analysieren.  Die wenig
überraschende Antwort lautet \emph{nein}, denn das Ganze ist eben auch hier
mehr als die Summe seiner Teile.

Relationale Verhältnisse in der \emph{Welt der technischen Systeme} werden
eher im Begriff des \emph{technischen Prinzips} sichtbar als im Begriff des
\emph{technischen Systems}. Insofern ist der Zugang in \cite{Shpakovsky2010}
deutlich besser geeignet, die Evolution in der Welt der technischen Systeme zu
beschreiben als der Zugang in \cite{TESE2018}.  Der Begriff \emph{Prinzip} ist
dabei nicht als TRIZ-Prinzip misszuverstehen, denn jene unglückliche englische
und deutsche Übersetzung des russischen Originals
„\emph{\foreignlanguage{russian}{Приём}}“ ist besser mit Vorgehensweise oder
Designmuster übersetzt.

\section{Theorie und Praxis}

Anliegen dieses Aufsatzes ist es, einen Beitrag zur theoretischen Fundierung
der TRIZ zu leisten. Theoretische Fundierungen erfolgreicher Praxen haben eine
gewisse Eigenart. Auf einer wissenschaftlichen Tagung der Astrophysiker wurde
ein Vortragender nach seiner viel beachteten Videosimulation der
Gravitationswellenausstrahlung bei der Kollision von zwei schwarzen Löchern
gefragt, ob bekannt sei, ob die simulierten Gleichungen überhaupt eine Lösung
besäßen.  Nein, das sei nicht bekannt.

Wie wichtig ist es, diese Lücke im Fundament der Theorie zu schließen? Oder
sollten lieber Mittel auf die Detektion realer Gravitationswellen konzentriert
werden?  Was würde aber in jenen riesigen und kostspieligen unterirdischen
Tanks anderes detektiert als spezielle Muster von Lichtblitzen, die im selben
Theoriegebäude als Zeichen der Wirkung von Gravitationswellen angesehen
würden?

Was ist das Tun von Mathematikern und Philosophen wert, die mit ihren
Anforderungen an die Rigorosität von Argumentationen der Praxis der Physiker
und Ingenieure ständig hoffnungslos hinterherlaufen? Braucht es die
Mathematiker, wenn die Physiker zwei divergente Reihen voneinander
subtrahieren und am Ende das Ergebnis stolz verkünden: „$-\frac{\pi}{3}$“?
Wieso, fragt der Mathematiker erstaunt. „Aus physikalischen Gründen“.  Näher
erläutert sieht der Mathematiker, dass dazu die Reihen umsortiert werden
müssten.  Aber ist das immer statthaft? Nein, weiß der Mathematiker aus
anderen Praxen.  Darf man das also hier?

Interessant wird es, wenn der Mathematiker mit der Antwort „nicht immer“
kommt. Die vorher unkonditionierte Praxis der Physiker wird damit
kontextualisiert, die Physiker beginnen zu fragen, was jenseits dieses
Kontextes passiert, und finden neue, praxistaugliche Effekte.  Das Fundament
wurde verstärkt, das Haus weiter ausgebaut.

Schlimmer wird es, wenn die „Fundamentalisten“ feststellen, dass das Fundament
nicht trägt -- die Erde steht nicht im Mittelpunkt der Welt, es gibt keinen
Äther, die Sache mit dem Phlogiston usw.  Dann muss das Gebäude abgetragen und
ein neues errichtet werden.  Auch ein solcher „Paradigmenwechsel“ (T.S. Kuhn)
ist allerdings kein Abriss, sondern eine Transformation, die nicht alle alten
Erkenntnisse über Zusammenhänge entwertet, sondern viele von ihnen nur auf das
neue Fundament stellt und entsprechend anpasst.

\section{Gesetze der Evolution technischer Systeme?}

Ausgangspunkt dieser Untersuchungen war eine Debatte mit den Organisatoren des
TRIZ-Cups 2019/20 über die Gültigkeit eines „Gesetzes der Verdrängung des
Menschen aus technischen Systemen“, das in einer späteren Fassung der
Ausschreibung als „Trend“ bezeichnet wurde. Unter den acht Gesetzen der
Entwicklung technischer Systeme, die Altschuller 1979 selbst formulierte
\cite[S. 2]{TESE2018}, kommt ein solcher Ansatz nicht vor, auch nicht in der
Auf"|listung von fünf Gesetzen und zehn Tendenzen in
\cite[S. 148\,ff.]{KS2017}.  

Systematisierungen von „Gesetzen der Evolution technischer Systeme“ oder zum
„technology forecasting“ sind aber in der TRIZ-Literatur weit verbreitet. Sie
sind auch Teil der verschiedenen Versionen eines „TRIZ Body of Knowlegde“,
etwa \cite{TBK-2007}. Meine weiteren Ausführungen beziehen sich auf
\cite{TESE2018} als Referenz, da hier von einflussreichen TRIZ-Theoretikern
mit der Autorität der MATRIZ im Rücken ein aktueller Zusammenschnitt der
Debatten um „Trends of Engineering Systems Evolution“ gegeben wird. Einen
deutlich anderen Ansatz, die Betrachtung der Evolution einzelner Funktionen
und nicht kompletter technischer Systeme, schlägt N. Shpakovsky in
\cite{Shpakovsky2010} mit seinem Konzept der „Evolutionsbäume“ vor.

Auch in der marxistischen Literatur wird ein solcher Herauslösungsprozess des
Menschen aus produktiven Prozessen thematisiert und an vielen Stellen als
unausweichlich charakterisiert.  So entwickelt Marx selbst im
„Maschinenfragment“ \cite[S. 570 ff.]{MEW42} -- einem frühen Rohentwurf der
eigenen ökonomischen Theorie -- die Vision einer Gesellschaft, in welcher der
„gesellschaftliche Stoffwechsel“ \cite[S. 37]{MEW23} auf eine Weise
organisiert ist, dass
\begin{quote}
  es nicht mehr der Arbeiter [ist], der modifizierten Naturgegenstand als
  Mittelglied zwischen das Objekt und sich einschiebt; sondern den
  Naturprozess, den er in einen industriellen umwandelt, er als Mittel
  zwischen sich und die unorganische Natur [schiebt], deren er sich
  bemeistert \cite[S. 572]{MEW42},
\end{quote}
und stellt weiter dar, dass die Entwicklung der Produktivkräfte
\emph{notwendig} auf eine solche Weise der Organisation des gesellschaftlichen
Stoffwechsels zusteuert.
\begin{quote}
  In den Produktionsprozess des Kapitals aufgenommen, durchläuft das
  Arbeitsmittel aber verschiedene Metamorphosen, deren letzte die
  \emph{Maschine} ist oder vielmehr ein \emph{automatisches System der
    Maschinerie} (System der Maschinerie; das \emph{automatische} ist nur die
  vollendetste adäquateste Form derselben und verwandelt die Maschinerie erst
  in ein System), in Bewegung gesetzt durch einen Automaten, bewegende Kraft,
  die sich selbst bewegt; dieser Automat bestehend aus zahlreichen mechanischen
  und intellektuellen Organen, sodass die Arbeiter selbst nur als bewusste
  Glieder desselben bestimmt sind. \cite[S. 584]{MEW42}
\end{quote}
Dieser Gedanke sei allerdings weitgehend singulär und im übrigen Marxschen
Werk nirgends ausgearbeitet, so \cite{Goldberg2016}. Zumindest auf das
Interesse an technischen Systemen und Entwicklungen trifft das allerdings
nicht zu, hierzu finden sich viele Stellen im Werk dieser Klassiker. Engels'
Interesse besonders an militär-historischen Entwicklungen ist weithin bekannt,
in \cite{MEW15} wird die Evolution des gezogenen Gewehrs auf eine Weise
analysiert, die es mit jeder TRIZ-Analyse aufnehmen kann, etwa in der
Herausarbeitung des sozio-technischen Widerspruchs \cite[S. 199]{MEW15}:
\begin{quote}
  Unter diesen Umständen ergab sich folgende dringende Aufgabe: eine
  Feuerwaffe zu erfinden, welche die Schussweite und die Genauigkeit der
  Büchse mit der Schnelligkeit und Leichtigkeit des Ladens und mit der Länge
  des Laufs der glattläufigen Muskete vereint, eine Waffe also, die zugleich
  Büchse und Nahkampfwaffe ist, welche man jedem Infanteristen in die Hand
  geben kann.

  So sehen wir also, dass gerade durch die Einführung des Kampfes in
  aufgelöster Ordnung in die moderne Taktik sich die Forderung nach einer
  solchen verbesserten Kriegswaffe erhob. [\ldots] Fast alle Verbesserungen,
  die an Handfeuerwaffen seit 1828 vorgenommen wurden, dienten diesem Zweck.
\end{quote}
\cite{Goldberg2016} untersucht allerdings nicht diese Frage, sondern aktuelle
Kapitalkonzentrationsprozesse, die erforderlich sind, um die immer
kostspieligeren technischen Großprojekte zu finanzieren.  In diesem Kontext
wird neuerdings viel über \emph{Plattformkapitalismus} geschrieben.  Der
unmittelbare Zusammenhang zwischen technologischen und finanziellen
Möglichkeiten ganzer Staaten im Kontext einer „wealth of nations“ ist
hinreichend bekannt. Siehe etwa \cite{Barkleit2000}, wo die technologischen
und ökonomischen Interdependenzen des letztlich gescheiterten
DDR-Mikroelektronikprojekts der 1980er Jahre genauer analysiert werden. Im
TRIZ-Umfeld spielen derartige Überlegungen eine eher marginale Rolle.

Technikoptimistischen Sichten der „Verdrängung des Menschen aus technischen
Systeme“ steht die Position aus dem Kybernetikdiskurs der 1960er bis 1980er
Jahre entgegen \cite[S. 10]{KFK2000}:
\begin{quote}
  Welche Stellung hat der Mensch im hochkomplexen informations-technologischen
  System? Unsere Antwort auf die Frage war immer: Der Mensch ist die einzig
  kreative Produktivkraft, er muss Subjekt der Entwicklung sein und bleiben.
  Daher ist das Konzept der Vollautomatisierung, nach dem der Mensch
  schrittweise aus dem Prozess eliminiert werden soll, verfehlt!
\end{quote}

Die Probleme eines solchen „Konzepts der Vollautomatisierung“, einer Welt der
„in Bewegung gesetzten Automaten“ werden mittlerweile in einer ökologischen
Krise planetaren Ausmaßes sichtbar. Die Verdrängungsthese selbst wird dabei
als direkte Gefährdung wahrgenommen, was hier seinerseits als \emph{These}
explizit formuliert werden soll:
\begin{quote}\it
  \textbf{These 1:} Die (scheinbare) Verdrängung des Menschen aus technischen
  Systemen weist auf eine existenziell gefährliche, unterkomplexe Wahrnahme
  dieser technischen Systeme hin.
\end{quote}
Das TRIZ-Prinzip 11 der \emph{Prävention} weist auf Handlungsbedarf in dieser
Richtung hin. Der Anwendungskontext dieses Prinzips wird in \cite{TT} wie
folgt umrissen: „Der Einsatz des Prinzips ist besonders in solchen Fällen
wichtig, in denen das System nicht über ein ausreichendes Maß an
Zuverlässigkeit verfügt“, um dann festzustellen, dass dem eigentlich
strukturell abgeholfen werden könne -- „Notfallsituationen können vermieden
werden, indem der Prozess zuverlässig gemacht wird.“ Dem stehen allerdings
gewichtige Gründe entgegen -- „was die technischen Systeme, die ihn
durchführen, erheblich verkompliziert oder verteuert. Dies ist kostspielig und
oft prinzipiell unmöglich. Mit anderen Worten: Notfälle sind
unvermeidlich“. Der Optimismus der Autoren, dass „zusätzliche Rettungs- und
Notfallsysteme [\ldots] in das Hauptsystem“ eingebaut werden, die allerdings
„nicht am Hauptsystem teilnehmen, sondern erst in einer Gefahrensituation zu
arbeiten beginnen“, erscheint mit Blick auf Kosten reduzierende Designpraxen
und die allgemeine Ausrichtung der anderen TRIZ-Prinzipien auf
Effizienzgewinne als Widerspruch in der TRIZ-Methodik selbst.  Dahinter mag
sich \emph{auch} ein prinzipielles Problem menschlichen Handelns verbergen --
niemand ist unfehlbar --, allerdings zeigt die Aussage, „das Prinzip kann dort
angewendet werden, wo die Zuverlässigkeit des Systems offensichtlich
unzureichend ist und ein Weg zur Erhöhung der Zuverlässigkeit auf das
notwendige Niveau nicht möglich ist“ (ebenda), dass der Einbau solcher
„Fehler“ in technische Systeme -- in Kenntnis derselben -- auf breiter Front
billigend in Kauf genommen wird.

Dies ist allerdings in keiner Weise mehr ein technisches Problem.  Harrisburg,
Tschernobyl, Fukushima oder der Klimawandel sind genü"|gend Fingerzeige, um
sich mit diesen Positionen genauer zu befassen.

Wir zeigen im Weiteren, dass die 10 in \cite{TESE2018} als „Trends“
präsentierten „Gesetze der Entwicklung technischer Systeme“ in Wirklichkeit
Design Pattern in den ingenieur-technischen Praxen des Entwurfs sowie der
Anpassung und Verbesserung technischer Systeme sind und sich somit
\emph{unmittelbar} auf sozio-technische Praxen beziehen. Dabei beziehen sie
sich auf ein \emph{spezifisches} begriff"|liches Abstraktionsniveau der sich
insgesamt in Widersprüchen entwickelnden Beschreibungsformen von
Welt. Insbesondere stehen die „Trends“ im Widerspruch zu Entwicklungslinien,
die sich auf anderen Abstraktionsniveaus abzeichnen. Kurz, das oben in These
und Gegenthese entfaltete ambivalente Verhältnis zu einer „Verdrängung des
Menschen aus technischen Systemen“ ist kein Alleinstellungsmerkmal nur für
diesen Trend, sondern trifft in ähnlicher Weise auch auf die anderen Trends
zu.

\section{Technik und Welt verändernde Praxen}

Betrieb und Nutzung technischer Systeme ist heute ein zentrales Element Welt
ver"|ändernder menschlicher Praxen. Dafür ist planmäßiges und abgestimmtes
arbeitsteiliges Handeln erforderlich, denn das Nutzen eines Systems setzt
dessen Betrieb voraus.  Umgekehrt ist es wenig sinnvoll, ein System zu
betreiben, das nicht genutzt wird. In der Informatik ist dieser Zusammenhang
zwischen Definition und Aufruf einer Funktion gut bekannt -- der Aufruf einer
Funktion, die noch nicht definiert wurde, führt zu einem Laufzeitfehler; die
Definition einer Funktion, die nie aufgerufen wird, weist auf einen
Designfehler hin.

Eng verbunden mit der informatischen Unterscheidung von Definition und Aufruf
einer Funktion ist die Unterscheidung von Designzeit und Laufzeit.  Eine
solche Unterscheidung hat im realweltlichen arbeitsteiligen Einsatz
technischer Systeme noch größere Bedeutung -- während der Designzeit wird das
prinzipielle kooperative Zusammenwirken \emph{geplant}, während der Laufzeit
\emph{der Plan ausgeführt}. Für technische Systeme sind also zusätzlich deren
interpersonal als \emph{begründete Erwartungen} kommunizierten
\emph{Beschreibungsformen} und die in \emph{erfahrenen Ergebnissen}
resultierenden \emph{Vollzugsformen} zu unterscheiden.

Marx \cite[S. 193]{MEW23} merkt dazu an:
\begin{quote}
  Eine Spinne verrichtet Operationen, die denen des Webers ähneln, und eine
  Biene beschämt durch den Bau ihrer Wachszellen manchen menschlichen
  Baumeister. Was aber von vornherein den schlechtesten Baumeister vor der
  besten Biene auszeichnet, ist, dass er die Zelle in seinem Kopf gebaut hat,
  bevor er sie in Wachs baut. Am Ende des Arbeitsprozesses kommt ein Resultat
  heraus, das beim Beginn desselben schon in der Vorstellung des Arbeiters,
  also schon ideell vorhanden war.
\end{quote}
So einfach ist es allerdings nicht, wie das folgende Beispiel einer
Konzertauf"|führung zeigt. Dieser die Zuhörer erfreuenden Vollzugsform geht
die Erarbeitung der Beschreibungsform, die Verständigung über die genaue
Interpretation des aufzuführenden Werks, voraus. Diese Verständigung auf einen
\emph{gemeinsamen Plan} ist selbst ein voraussetzungsreicher praktischer
Prozess.  Die Voraussetzungen resultieren aus vorgängigen Praxen -- etwa dem
\emph{privaten Verfahrenskönnen} der einzelnen Musiker in der Beherrschung
ihrer Instrumente sowie dem Vorliegen der Partitur als etablierter
Beschreibungsform des aufzuführenden Konzertstücks.  Wenn Alexander Shelley am
14. Oktober 2018 im Leipziger Gewandhaus ohne diese Partitur von Mozarts
Klavierkonzert KV 491 ans Dirigentenpult tritt, so wird deutlich, dass jene
Beschreibungsform allenfalls das Rohmaterial liefert, auf dessen Basis sich
Dirigent und Orchester in den vorausgegangenen Proben auf eine situativ
konkrete Beschreibungsform als Basis der nun zur Auf"|führung gelangenden
Vollzugsform geeinigt haben. Mehr noch weisen die opulenten Gesten des
Dirigenten in Richtung Orchester darauf hin, dass in diesen Proben auch
\emph{Sprache} generiert wurde, um die Ergebnisse längerer
Verständigungsprozesse in eine kompakte Form zu fassen, die den zeitkritischen
Tempi der Vollzugsform gewachsen ist.  Den Rahmen einfacher
ingenieur-technischer „Baumeisterarbeit“ sprengt Gabriela Montero, die
Solistin jenes Abends, mit ihrer Zugabe: Das Publikum wird aufgefordert, eine
Melodie vorzugeben, woraus die Virtuosin eine Improvisation als Vollzugsform
entwickelt, zu der es keine interpersonal kommunizierbare Beschreibungsform
gibt, wenn man einmal von den Tonaufzeichungen jenes Gewandhausabends und den
Berichten der begeisterten Hörerschaft absieht.  Dass auch hierfür technische
Meisterschaft erforderlich war, steht außer Frage.

Das Verhältnis der Menschen zu ihren technischen Systemen ist also komplex und
nur in einer dialektischen Perspektive der Weiterentwicklung bereits
vorgefundener technischer Systeme zu fassen, wenn man sich nicht unentrinnbar
in unfruchtbare Henne-Ei-Debatten verstricken will.  Das relativiert aber
auch die Marxsche Forderung an die Philosophen, denn deren Interpretationen
sind die Differenzen zwischen den begründeten Erwartungen und den erfahrenen
Ergebnissen früherer Praxen vorgängig. Ob es ausreicht, diese Differenzen auf
der Ebene der Techniker, der Ingenieure oder der Fachwissenschaftler zu
besprechen oder eine Intervention der „interpretierenden Philosophen“ als
eigenständige Reflexionsdimension von Bedeutung ist, mag an dieser Stelle
offen bleiben.

\section{Systeme und Komponenten}

Neben der Beschreibungs- und Vollzugsdimension spielt für technische Systeme
auch der \emph{Aspekt der Wiederverwendung} eine große Rolle.  Dies gilt,
zumindest auf der artefaktischen Ebene, allerdings \emph{nicht} für die
meisten technischen Großsysteme -- diese sind \emph{Unikate}, auch wenn bei
deren Montage standardisierte Komponenten verbaut werden. Auch die Mehrzahl
der Informatiker ist mit der Erstellung solcher Unikate befasst, denn die
IT-Systeme, die derartige Anlagen steuern, sind ebenfalls Unikate.  Dasselbe
gilt auch für die Ämter, Behörden und öffentlichen Einrichtungen. So ist zum
Beispiel die Leipziger Stadtverwaltung aktuell damit befasst, ihre
Verwaltungsprozesse zu „digitalisieren“, was unter Führung des Dezernats
Allgemeine Verwaltung und zusammen mit dem städtischen IT-Dienstleister Lecos
erfolgt. Im Industriesektor ist deshalb deutlich zwischen Werkzeugmaschinenbau
und Industrieanlagenbau -- zwischen Ausrüstern sowie Planern und „Baumeistern“
entsprechender Unikate -- zu unterscheiden, auch wenn dies in einschlägigen
Statistiken \cite{VDMA2019} zum \emph{Maschinen- und Anlagenbau}
zusammengefasst wird.

Die Besonderheiten eines technischen Systems liegen damit vor allem im Bereich
des \emph{Zusammenspiels der Komponenten}. So unterscheiden sich
beispielsweise die Produktionsleitsysteme verschiedener BMW-Werke deutlich
voneinander \cite{Kropik2009}. Die Werke wurden zu verschiedenen Zeiten nach
dem jeweiligen Stand der Technik und dem sich ebenfalls verändernden
Geschäftsmodell des Unternehmens konzipiert. Einmal in die Welt gesetzt, sind
derartige technischen Großsysteme nur noch bedingt modifizierbar und werden
deshalb nach Ablauf entsprechender Amortisationsfristen auch konsequent außer
Betrieb gestellt. Gleichwohl spielt der Aspekt der Wiederverwendung auch bei
solch unterschiedlichen technischen Systemen eine Rolle, verschiebt sich aber
von der unmittelbaren Ebene der technischen Artefakte auf höhere Ebenen der
Abstraktion in der Beschreibungsdimension.

Damit sind wesentliche Elemente zusammengetragen, die eine erste Annäherung an
den \emph{Begriff eines technischen Systems} erlauben.  Der Begriff ist in
einem planerisch-realweltlichen Kontext vierfach überladen
\begin{itemize}
\item [1.] als realweltliches Unikat (z.B. als Produkt, auch wenn das Unikat
  ein Service ist),
\item [2.] als Beschreibung dieses realweltlichen Unikats (z.B. in der Form
  einer speziellen Produktkonfiguration)
\end{itemize}
und für in größerer Stückzahl hergestellte Komponenten auch noch
\begin{itemize}
\item [3.] als Beschreibung des Designs des System-Templates (Produkt-Design)
  sowie
\item [4.] als Beschreibung und Betrieb der Auslieferungs- und
  Betriebsstrukturen der nach diesem Template gefertigten realweltlichen
  Unikate (als Produktions-, Qualitätssicherungs-, Auslieferungs-, Betriebs-
  und Wartungspläne).
\end{itemize}
Besonders Punkt 4 spielt im TRIZ-Kontext kaum eine Rolle, obwohl davon
auszugehen ist, dass weder im privaten noch im unternehmerischen Umfeld
technische Produkte nachhaltig nachgefragt werden, für die absehbar
unzureichender Service angeboten wird.

Als Grundlage für einen derart abgrenzenden Systembegriff soll im Weiteren der
submersiv gefasste Begriff offener Systeme der Theorie dynamischer Systeme
\cite{Bertalanffy1950} verwendet werden, der
\begin{itemize}
\item [1.] eine innere Abgrenzung gegen vorgefundene Systeme (Komponenten), 
\item [2.] eine äußere Abgrenzung und funktional determinierte Einbettung in
  eine (funktionierende) Umwelt sowie
\item [3.] einen (funktionierenden) externen Durchsatz postuliert, der zu
  innerer Strukturbildung führt und damit die Leistungsfähigkeit des Systems
  bestimmt,
\end{itemize}
und seine Fruchtbarkeit für eine Behandlung mit mathematischen Instrumenten
seither vielfach unter Beweis gestellt hat.  

\emph{Technische Systeme} sind in einem solchen Kontext Systeme, auf deren
Gestaltung kooperativ und arbeitsteilig agierende Menschen Einfluss nehmen,
wobei \emph{vorgefundene} technische Systeme auf Beschreibungsebene durch eine
\emph{Spezifikation} ihrer Schnittstellen und auf Vollzugsebene durch die
\emph{Gewähr spezifikationskonformen Betriebs} normativ charakterisiert sind.

Wir bewegen uns dabei klar im Bereich der Standard-TRIZ-Terminologie eines
\emph{Systems von Systemen} -- ein technisches System besteht aus Komponenten,
die ihrerseits technische Systeme sind, deren \emph{Funktionieren} (sowohl im
funktionalen als auch im operativen Sinn) für die aktuell betrachtete
Systemebene vorausgesetzt wird.

Dem Begriff eines technischen Systems kommt damit die epistemische Funktion
der (funktionalen) „Reduktion auf das Wesentliche“ zu.  Einstein wird der
Ausspruch zugeschrieben „make it as simple as possible but not simpler“. Das
\emph{Gesetz der Vollständigkeit eines Systems} bringt genau diesen Gedanken
zum Ausdruck, allerdings tritt dieser dabei nicht als \emph{Gesetz}, sondern
als ingenieur-technische \emph{Modellierungsdirektive} in Erscheinung.  Die
scheinbare „Naturgesetzlichkeit“ der beobachteten Dynamik ist also wesentlich
an \emph{vernünftiges} (im Sinne von \cite{Vernadsky2001}) \emph{menschliches
  Agieren} gebunden.

Mit einem Ansatz der „Reduktion auf das Wesentliche“ sowie der „Gewähr
spezifikationskonformen Betriebs“ sind in diese Begriffsbildung inhärent
menschliche Praxen eingebaut, aus denen heraus die Begriffe „wesentlich“,
„Gewähr“ und „Betrieb“ überhaupt erst sinnvoll gefüllt werden können.  Eine
Unterscheidung zwischen technischen und sozio-technischen Systemen, die für
M. Rubin „offensichtlich und wesentlich“ (private Kommunikation) ist, wird
damit problematisch. Wesentliche Begriffe aus dem sozial determinierten
Praxisverhältnis von Menschen wie Ziel, Nutzen, Gewährleistung und
Verantwortung sind fest in die Begriffsgenerierungsprozesse der Beschreibung
konkreter technischer Systeme eingebaut und finden in den konkreten
gesellschaftlichen Setzungen eines primär rechtsförmig konstituierten
bürgerlichen Systems ihre „natürliche“ Fortsetzung.

\section{Die Welt der Technischen Systeme. Basics}

In der TRIZ-Literatur spielen solche begriff"|lichen Fundierungen kaum eine
Rolle.  Einschlägige Lehrbücher wie etwa \cite{KS2017} betrachten den Begriff
des \emph{technischen Systems} als intuitiv gegeben, der sich aus einer
„industriellen Praxis“ heraus \cite[S. 2]{KS2017} von selbst versteht, während
andere Begriffe wie „Prozess“, „Produkt“, „Dienstleistung“, „Ressourcen“ und
„Effekte“ \cite[S. 6--10]{KS2017} genauer eingeführt werden. Selbst die
ausführliche Beschreibung der „Evolution technischer Systeme“ in 5 Gesetzen
und 11 Trends \cite[Kap. 4.8]{KS2017} basiert allein auf der lapidaren
Feststellung „Die Existenz technischer Evolution ist eine zentrale Erkenntnis
der TRIZ“.  Auch \cite{TESE2018} bleibt in dieser Frage vage; im Vorwort von
B. Zlotin heißt es allein zum \emph{Zweck} von Betrachtungen der Evolution
ingenieur-technischer Systeme „humanity can achieve practically any realistic
goal, but certain priorities must be set to ensure the greatest possible
impact on the economy and human life. [\ldots] The powers of contemporary
science and technology as well as financial investment should be applied to
carefully selected and formulated objectives.“

Es ist natürlich möglich, in einem diskursiven Rahmen die verbale Fassung
eines Begriffs offen zu lassen und auf andere Weise -- etwa durch den Bezug
auf gemeinsame Praxen oder durch den „gewöhnlichen Gebrauch“ -- die Konvergenz
der Begriffsverwendung zu erreichen.  Ein solches Grundmuster wird im
TRIZ-Kontext für den Begriff \emph{technisches System} besonders auch in
\cite{TESE2018} angewendet, indem der Begriff durch eine Vielzahl von
Beispielen in Kombination mit den Begriffen „Muster“ und „Evolution“
illustriert, die genaue Fassung aber dem geneigten Leser überlassen wird.  Der
dort mittlerweile erfolgte Rückzug auf Begriffe wie „Muster“ oder „Trend“
gegenüber dem schärferen und wissenschaftspraktisch vorbelegten Begriff
„Gesetz“ unterstützt das Anliegen der Autoren von \cite{TESE2018}, empirische
Erfahrung zu systematisieren, verweist aber zugleich auf das schwache
theoretische Fundament eines solchen Systematisierungsanliegens.  Das weite
Spektrum praktisch kursierender Präzisierungen eines derart im Ungewissen
gelassenen Begriffs wurde in einer Facebook-Diskussion \cite{Graebe2019b} im
August 2019 deutlich. Für ein genaueres Abwägen der Argumente zu oben
formulierter These und Gegenthese ist ein solches Fundament allerdings nicht
ausreichend.

Wie kann der Begriff eines \emph{technischen Systems} also weiter geschärft
werden?  In unserem Seminar \cite{Graebe2020} haben wir „den Systembegriff als
Beschreibungsfokussierung identifiziert, mit der konkrete Phänomene durch
\emph{Reduktion auf das Wesentliche} [\ldots] einer Beschreibung zugänglich
werden.“  Die Reduktion richtet sich auf folgende drei Dimensionen
\cite[S. 18]{Graebe2020} 
\begin{itemize}
\item [(1)] Abgrenzung des Systems nach außen gegen eine \emph{Umwelt},
  Reduktion dieser Beziehungen auf Input/Output-Beziehungen und garantierten
  Durchsatz.
\item [(2)] Abgrenzung des Systems nach innen durch Zusammenfassen von
  Teilbereichen als \emph{Komponenten}, deren Funktionieren auf eine
  „Verhaltenssteuerung“ über Input/Output-Bezie"|hungen reduziert wird.
\item [(3)] Reduktion der Beziehungen im System selbst auf „kausal
  wesentliche“ Beziehungen.
\end{itemize}
Weiter wird ebenda festgestellt, dass -- ähnlich wie im Konzertbeispiel --
einer solchen reduktiven Beschreibungsleistung vorgefundene (explizite oder
implizite) Beschreibungsleistungen vorgängig sind:
\begin{enumerate}
\item[(1)] Eine wenigstens vage Vorstellung über die (funktionierenden)
  Input/Output-Leistungen der Umgebung.
\item[(2)] Eine deutliche Vorstellung über das innere Funktionieren der
  Komponenten (über die reine Spezifikation hinaus).
\item[(3)] Eine wenigstens vage Vorstellung über Kausalitäten im System
  selbst, also eine der detaillierten Modellierung vorgängige, bereits
  vorgefundene Vorstellung von Kausalität im gegebenen Kontext.
\end{enumerate}
Die Punkte (1) und (2) können ihrerseits in systemtheoretischen Ansätzen für
die Beschreibung der „Umwelt“\footnote{Hierfür ist allerdings die Abgrenzung
  eines oder mehrerer Obersysteme in einer noch umfassenderen „Umwelt“
  erforderlich.}  sowie der Komponenten (als Untersysteme) entwickelt werden,
womit die Beschreibung von \emph{Koevolutionsszenarien} wichtig wird, die
ihrerseits für die Vertiefung des Verständnisses von Punkt (3) relevant sind.

Dabei ist der Fokus zunächst auf ein genaueres Verständnis des Begriffs
\emph{System} gerichtet, der als Reduktion von Komplexität in den drei oben
angeführten Dimensionen betrachtet wird. Da in diesem Verständnis Komponenten
eines Systems selbst wieder Systeme sind, liegt auch im allgemeinen Fall die
Betrachtung eines Systems als „System von Systemen“ nahe, wie es in
\cite{Holling2000} thematisiert ist.  Wesentliches Reduktionskriterium für
Beziehungen zwischen Komponenten sind in solchen Systemen \emph{spezifische
  Eigenzeiten und Eigenräume} wie in den Abbildungen 1--3 in
\cite{Holling2000} dargestellt ist, die auch in den TRIZ-Prinzipien 18
\emph{Ausnutzung mechanischer Schwingungen}, 19 \emph{periodische Wirkung}, 23
\emph{Rückkopplung} und 25 \emph{Selbstbedienung} eine Rolle spielen.

Die Beschreibung von Planung, Entwurf und Verbesserung technischer Systeme
geht in einem solchen Ansatz von der Leistungsfähigkeit bereits vorhandener
technischer Systeme aus, die sowohl in (2) als Komponenten als auch -- aus der
Sicht eines Systems im Obersystem -- in (3) als benachbarte Systeme zu
berücksichtigen sind.

Ingenieur-technische Praxen bewegen sich damit in einer \emph{Welt technischer
  Systeme}. Aus der konkreten Beschreibungsperspektive eines Systems sind
andere Systeme als Komponenten oder Nachbarsysteme allein in ihrer
\emph{Spezifikation} wichtig. Eine solche Reduktion auf das Wesentliche
erscheint praktisch als verkürzte Sprechweise über eine gesellschaftliche
Normalität, was ich kurz als \emph{Fiktion} bezeichne.  Diese Fiktion kann und
wird im täglichen Sprachgebrauch so lange aufrecht erhalten, so lange die
gesellschaftlichen Umstände die Aufrechterhaltung der daran gebundenen
gesellschaftlichen Normalität garantieren können, so lange also der
\emph{Betrieb der entsprechenden Infrastrukturen} gewährleistet ist.
Technische Systeme sind damit wenig"|stens in ihrer Vollzugsdimension
\emph{immer} sozio-technische Systeme.

Ein Ausblenden dieser sozialen Zusammenhänge kann sich also allenfalls auf die
\emph{Planung} derartiger Systeme sowie deren artefaktische Daseinsdimension
beziehen, die den \emph{Betrieb} der erforderlichen Infrastruktur ausblendet
oder in ein Obersystem verschiebt.  Letzteres ist aber unzweckmäßig, da das
Beheben von Problemen im Betrieb eines Systems Kenntnisse über dessen
Funktionieren nicht nur auf der Ebene der Spezifikation, sondern auch auf der
Ebene der Implementierung erfordert.

Eine solche Engführung des Begriffs \emph{technisches System} resultiert
möglicherweise aus spezifischen Praxen der Vermittlung und Weiterentwicklung
von TRIZ-Grundlagen, da TRIZ-Praktiker zu derartigen Auseinandersetzungen um
Fundierungen der von ihnen angewendeten Theorien ein entspanntes bis
ignorantes Verhältnis an den Tag legen. Für eine Theorie der Evolution
technischer Systeme ist aber eine noch weitergehende Engführung und
Abstraktion des Begriffs erforderlich, da sich die bisherige Begriffsgenese
ausschließlich an der zu einem gegebenen Zeitpunkt \emph{vorgefundenen}
Landschaft technischer Systeme orientiert.

\section{Zum Evolutionsbegriff. Die sozio-ökonomische Dimension} 

Um evolutionäre Aspekte zu thematisieren, ist eine Zuordnung von zu
verschiedenen Zeiten existierenden technischen Systemen zu
\emph{Entwicklungslinien} erforderlich. Hier gehen \cite{Shpakovsky2010} und
\cite{TESE2018} deutlich verschiedene Wege.

In \cite{TESE2018} wird \emph{Evolution}, wie V. Souchkov im Vorwort
\cite[S. IX]{TESE2018} feststellt, als „innovative development“ verstanden,
„since -- in contrast to nature -- craftsmen and engineers make decisions
based on logic, previous experience, and knowledge of basic principles rather
than chance.“ Die Konzentration auf „craftsmen and engineers“ weist noch
einmal auf die Engführung der Praxen hin, aus denen die Systematisierung in
\cite{TESE2018} abgeleitet wurde.

Mögliche Zugänge zur Einbettung des bisher entwickelten Begriffs in
historische Geneseprozesse könnten sich an der wissenschaftshistorischen
Betrachtung \cite{Weller2008} der Entwicklung der
\emph{Automatisierungstechnik} als eines der wichtigsten interdisziplinären
technikwissenschaftlichen Bereiche und damit \emph{aus der Perspektive der
  Praxen des Industrieanlagenbaus} oder aber der industriehistorischen
Untersuchung der Genese der \emph{Praxen der Produktion} realweltlicher
technischer Systeme orientieren, in denen Produktlinien \cite{Pohl2005},
Produktionsnetzwerke \cite{Friedli2013} oder neuerdings auch technische
Ökosysteme \cite{Graebe2018} eine zentrale Rolle spielen. Mit dem Bereich des
\emph{Systems Engineering} existiert zudem ein aus der Informatik
hervorgegangenes umfassendes technikwissenschaftliches Forschungsgebiet mit
vergleichbaren Fragestellungen, dessen Grundlagen in einer internationalen
Norm ISO 15288 \emph{Systems and Software Engineering} dokumentiert sind.

Der Zugang in \cite{TESE2018} ist allerdings ein anderer -- zur
Identifizierung von Entwicklungslinien wird der Begriff \emph{technisches
  System} zwischen „technology push“ und „market pull“ als „simple means for
understanding the advancement of man-made systems“ eingebettet
\cite[S. 1]{TESE2018}. Der Bezug auf den noch ungenaueren Begriff „man-made
systems“ wird im Weiteren genauer erläutert: Innovation als „improvement of
already-existing systems“ wird durch das Fortschreiben wissenschaftlicher
Erkenntnis angetrieben, aus der heraus neue Systeme, Produkte und
Dienstleistungen entstehen, die von einem „market pull, the second trigger for
innovation“ einem Formungs- und Ausleseprozess unterworfen sind, „that
stimulates the development of a system by meeting the needs of that system's
users“.  Die genaue Ausformung dieses nicht von ingenieur-technischen, sondern
von innovations-unternehmerischen Praxen getriebenen Ansatzes wird in
\cite[Kap. 3]{TESE2018} deutlich.  Die Gründe für den universalistischen
Anstrich des Vortrags der Erfahrungen hat S. Gerovitch in \cite{Gerovitch1996}
hinreichend genau analysiert, so dass dies hier unberücksichtigt bleiben und
auf die nüchterne Feststellung reduziert werden kann, dass sich die Grundlagen
dieser impliziten Begriffsbildungsprozesse im Rahmen des sozio-technischen
ökonomischen Systems einer kapitalistischen Wirtschaftsordnung als Obersystem
bewegen (westlicher Prägung füge ich hinzu, da die Übertragbarkeit auf stärker
autokratisch geprägte Wirtschaftsordnungen wie etwa in China oder Russland
zusätzliche Betrachtungen erfordert).  In Wirklichkeit ist die
Kontextualisierung noch enger gezogen, wie die Analyse der Beispiele zeigt --
eine Unterscheidung zwischen Industrieanlagenbau, Werkzeugmaschinenbau und
Konsumgüterproduktion, wie sie etwa in volkswirtschaftlichen Analysen üblich
ist, wird nicht vorgenommen, gleichwohl grundsätzlich die Perspektive einer an
einem größeren Markt orientierten Produktgängigkeit der untersuchten
\emph{technischen Systeme} eingenommen.  Der Unikat-Charakter der
überwiegenden Mehrzahl technischer Großsysteme und damit die Praxen des
Industrieanlagenbaus bleiben damit unberücksichtigt.

Damit ist der Gegenstandsbereich der technischen Systeme hinreichend umrissen,
deren Evolution in \cite{TESE2018} untersucht wird. Zugleich wird M. Rubins
Position verständlich, dass in einem solchen Kontext die Unterscheidung von
technischen und sozio-technischen Systemen „offensichtlich und wesentlich“
ist, was M. Rubin (private Kommunikation) mit Verweis auf \cite{Rubin2007} und
\cite{Rubin2010} selbst wie folgt umreißt:
\begin{quote}
  Bei der Betrachtung eines technischen Systems berücksichtigen wir keine
  anderen bestehenden Beziehungen (soziale, politische, wirtschaftliche,
  Marketing usw.) im System, mit Ausnahme von Objekten und Beziehungen
  technischer Natur. Diese externen (menschlichen, kulturellen) Beziehungen
  können durch zusätzliche Anforderungen oder Einschränkungen an technische
  Objekte ersetzt werden.  Bei der Betrachtung von Systemen als
  sozio-technisch werden eine Reihe technischer Objekte und Zusammenhänge
  berücksichtigt, beispielsweise wenn die TRIZ-Analyse von
  Produktionsunternehmen nicht nur als technisches System (Maschinen, Geräte),
  sondern die Fabrik als sozio-technisches Objekt betrachtet wird:
  Bestellsystem und Marketing, Personalpolitik, Finanzen und die
  wirtschaftliche Lage des Unternehmens, Systeme der Entscheidungsfindung usw.
  Offensichtlich verändert dies den Gegenstand der Überlegungen und die
  Untersuchungsinstrumente grundlegend.
\end{quote}
Natürlich bedürfen die Begriffe „technisches Objekt“, „technischer
Zusammenhang“ und „Beziehung technischer Natur“, mit denen hier über die
Grenzen zwischen technischen und sozio-technischen Systemen hinweg vermittelt
werden soll, weiterer Präzisierung.

Kehren wir jedoch zu \cite{TESE2018} zurück und untersuchen genauer, auf
welcher Aggregationsgrundlage die „Evolution ingenieur-technischer Systeme“
untersucht wird.  Da wir inzwischen auch ein Obersystem identifiziert haben,
gegen welches die Ausführungen relatiert werden können, können wir die
TRIZ-Methodik selbst zur Rekonstruktion der Modellierung und damit zur Analyse
der begriffstheoretischen Fundierung von \cite{TESE2018} einsetzen.

Ausgangspunkt ist das sozio-ökonomische System einer industriellen
Produktionsweise. „Der Reichtum dieser Gesellschaften“, so beginnt Marx seine
Analyse eines solchen sozio-ökonomi"|schen Systems in \cite{MEW23}, „erscheint
als eine 'ungeheure Warensammlung', die einzelne Ware als seine
Elementarform. Unsere Untersuchung beginnt daher mit der Analyse der Ware.“
Auch wir starten mit diesem Begriff als einer hochgradigen Abstraktion.  Marx'
Arbeitswertheorie abstrahiert im Begriff der \emph{Ware} bekanntlich von
sämtlichen qualitativen Eigenschaften außer der einen, Produkt menschlicher
Arbeit zu sein\footnote{Marx folgt dabei der Begriffsentwicklungsmethodik von
  Hegel. Rainer Thiel \cite[S. 190]{Thiel2007} schreibt dazu: „Hegels
  Begriffsentwicklung beginnt mit dem 'Sein'. Und was tut der Dialektiker
  Hegel? Er entwickelt – mit der Sturheit eines Schelms, wie ein Computer –
  den Inhalt des 'reinen Seins': 'Sein, reines Sein, -- ohne alle weitere
  Bestimmung. In seiner unbestimmten Unmittelbarkeit ist es nur sich selbst
  gleich und auch nicht ungleich gegen Anderes, hat keine Verschiedenheit
  innerhalb seiner, noch nach außen. Durch irgendeine Bestimmung oder Inhalt,
  der in ihm unterschieden, oder wodurch es als unterschieden von einem Andern
  gesetzt würde, würde es nicht in seiner Reinheit festgehalten. Es ist die
  reine Unbestimmtheit und Leere. – Es ist nichts in ihm anzuschauen, wenn von
  Anschauen hier gesprochen werden kann; oder es ist nur dies reine, leere
  Anschauen selbst. Es ist ebensowenig etwas in ihm zu denken, oder es ist
  ebenso nur dies leere Denken. Das Sein, das unbestimmte, unmittelbare, ist
  in der Tat Nichts, und nicht mehr noch weniger als Nichts'.“ Im Gegensatz zu
  dieser reinen Gedankenakrobatik bezieht sich Marx aber immer auf die diesen
  Begriffsentwicklungen vorgängigen \emph{gesellschaftlichen Praxen},
  vgl. etwa die Feuerbachthesen 1--3, \cite{MEW3}.}.  Erst auf einer solchen
Abstraktionsebene werden konkrete Waren global austauschbar und konstituieren
damit einen globalen Markt als \emph{Verhältnis} -- Feld in der
TRIZ-Terminologie -- zwischen diesen Warenkonkreta, den \emph{Tauschwert}.

Darum geht es in \cite{TESE2018} allerdings nicht, sondern um funktionale
Qualitäten konkreter Warengruppen wie Waschmaschinen oder Federhalter. Das
allgemeine Konkurrenzverhältnis abstrakter Waren zerfällt dabei in konkretere
Konkurrenzverhältnisse einzelner Warengruppen auf Einzelmärkten, die in
\cite{TESE2018} als „market pull“ die Hauptfunktion des Werkzeugs „Markt“
sind, welches die Objekte „engineering systems“ zu „nützlichen Produkten“
umformt -- ich folge dabei den TRIZ-Begriff"|lichkeiten von \cite{TT}.  Mit
der Marktgängigkeit von Produkten ist eine erste Struktureinheit im Obersystem
identifiziert -- konkrete Märkte, auf denen \emph{konkrete} Waren mit
\emph{spezifischen} funktionalen Eigenschaften -- \emph{Gebrauchswerten} in
der Terminologie von Marx -- miteinander im Wettbewerb stehen.  Die
\emph{Funktion} Konkurrenzverhältnis des \emph{Werkzeugs} Markt wird in
\cite{TESE2018} seinerseits als \emph{Werkzeug} mit Technologie formender
Funktionalität betrachtet.  Dieser Gedanke soll nun genauer entwickelt werden.

Jede Ware ist „ein Ganzes vieler Eigenschaften und kann daher nach
verschiedenen Seiten nützlich sein“ \cite[S. 49]{MEW23}. Jede konkrete Ware
ist damit selbst ein technisches System im oben entwickelten Verständnis, wenn
sie als durch ihre Spezifikation gegebenes Ensemble „nützlicher“
Funktionalitäten betrachtet wird. Dieses Ensemble nützlicher Eigenschaften
bestimmt aber auch die Möglichkeiten und Grenzen der Substituierbarkeit von
Waren im gesamtgesellschaftlichen technologischen Prozess\footnote{Diese
  Grenzen sind allerdings fließend, wie A. Kuryan in der Diskussion
  \cite{Graebe2019b} am Beispiel eines Hammers demonstriert, der zum
  Offenhalten einer Balkontür eingesetzt wird (\foreignlanguage{russian}{„Если
    ты решил с помощью молотка подпевать дверь на балконе, чтобы она не
    закрывалась, то ты создал решение.“}). Die Bedeutung derartiger
  Grenzüberschreitungen für die TRIZ-Analyse der Evolution technischer Systeme
  bedarf einer genaueren Untersuchung, die hier nicht geleistet werden kann.}.
Jene Grenzen führen zu einer Stratifizierung „des Markts“ in konkrete Märkte
für konkrete Warengruppen. Im sozio-ökonomischen Obersystem haben wir damit
zwischen dem Werkzeug-Template „Markt“ und konkreten realweltlichen
Ausprägungen dieses Werkzeugs zu unterscheiden. Diese realweltliche Struktur
der \emph{Technologiemärkte} ist der in \cite{TESE2018} ausgeführten
S-Kurven-Analyse vorgängig und wird dort implizit als gegeben vorausgesetzt.
Jeder solche Technologiemarkt ist durch ein spezifisches Bündel technischer
Funkionalitäten charakterisiert, wobei \cite{TESE2018} mit dem Ansatz des MPV
(main parameter of value) postuliert, dass sich ein solcher Markt um einen
speziellen technischen Parameter herum gruppiert, der für die Wertschöpfung
von besonderer Bedeutung ist.

Damit ist aber das Erfordernis einer weiteren Abstraktion verbunden, denn
konkrete Waren, im obigen Sinne als technische Systeme, als \emph{konkrete}
Bündel technischer Funktionalitäten verstanden, sind prinzipiell geeignet, auf
\emph{mehreren} derartigen Technologiemärkten gehandelt zu werden und werden
dies praktisch auch. Ein solcher Technologiemarkt wird auch weniger durch die
auf ihm gehandelten Waren bestimmt als durch die diese Waren produzierenden
Unternehmen. Darum geht es den Autoren von \cite{TESE2018} auch, wenn sie
Altschullers S-Kurven-Analyse zu einer \emph{pragmatischen S-Kurven-Analyse}
weiterentwickeln.  Damit verschiebt sich aber das in \cite{TESE2018}
aufgerufene Abstraktionserfordernis von einem MPV als eigenständigem
Charakteristikum zur \emph{unternehmerischen Fähigkeit}, technische Artefakte
mit diesem MPV in angemessenem Preis-Leistungs-Verhältnis \emph{zu
  produzieren}.  Damit wird auch deutlich, dass sich auf jenen
Technologiemärkten zwar „Produkte voneinander unabhängig betriebner
Privatarbeiten“ begegnen, die Produzenten aber \emph{nicht} „erst in
gesellschaftlichen Kontakt treten durch den Austausch ihrer Arbeitsprodukte“
\cite[S. 87]{MEW23} in dem Sinne, dass das Spannungsverhältnis zwischen
begründeten Erwartungen und erfahrenen Ergebnissen der Konditionen
\emph{früheren} Austauschs ihrer Arbeitsprodukte die Dynamik jenes
Technologiemarkts bestimmt.  

Gegenstand technologischer Evolution sind damit aber diese technologischen
Produktionsbedingungen selbst. Dies wird auch in \cite{TESE2018} so gesehen,
denn die im Buch beschriebenen Handlungsoptionen beziehen sich auf die
Organisation entsprechender Innovationsprozesse in Unternehmen. Damit kann
aber das Obersystem in unserer TRIZ-Analyse der begriff"|lichen Grundlagen von
\cite{TESE2018} weiter eingeschränkt werden auf die strategischen
Führungsstrukturen von Unternehmen, in denen die Innovationsprozesse praktisch
gestaltet werden. Auch hierbei haben wir es mit der Dualität von
System-Template -- gängigen gesellschaftlichen Verfahrensweisen zur
Organisation von Innovationsprozessen -- und konkreten realweltlichen
System"|ausprägungen in den einzelnen Unternehmen zu tun. Die
\emph{Hauptfunktion} jener Strukturen im Unternehmen ist die Organisation des
Innovationsprozesses in enger Verbindung mit der allgemeinen
Geschäftsstrategie.  Dieser Prozess selbst wird vom strategischen Management
entschieden und verantwortet, das hierzu die \emph{widersprüchlichen
  Anforderungen} verschiedener Unternehmensteile (R\&D, Vertrieb, Finanzen,
Controlling, SCM, CRM) unter einen Hut zu bringen hat. Die in \cite{TESE2018}
zusammengetragenen Empfehlungen sind \emph{ein} Aspekt in diesem komplexen
Abwägungsprozess. Eine Methodik zwischen „technology push“ und „market pull“
ist dabei mit Blick etwa auf die Ausführungen in \cite{Preez2006} eher auf dem
Niveau der 1960er Jahre anzusiedeln. Ebenda wird in Fig.~3 mit „state of the
art in science and technology“ neben den „needs of society and marketplace“
ein weiteres Obersystem in Stellung gebracht, das auch bei Patenterteilungen
mit den Begriffen „Stand der Technik“ und „Erfindungshöhe“ eine wichtige Rolle
spielt.

\begin{figure}  
\begin{center}
  \begin{tikzpicture}[
      >={Triangle[length=0pt 6,width=0pt 5]},
      rounded corners=2pt,line width=1pt] 
  \node[draw=green] at (0,2) [circle] (A0)
       {\bbox{2}{Idea Generation}};
  \node[draw=red] at (3,2) [rectangle] (A1)
       {\bbox{1.8}{Develop- ment}};
  \node[draw=red] at (6,2) [rectangle] (A2)
       {\bbox{1.8}{Manufac- turing}};
  \node[draw=red] at (9,2) [rectangle] (A3)
       {\bbox{1.8}{Marketing and Sales}};
  \node[draw=green] at (12,2) [circle] (A4)
       {\bbox{2}{Commercial Product}};
  \node[draw=green] at (6,4.3) [rectangle] (A5)
       {\bbox{8}{State of the art\\[12pt] in science and technology}}; 
  \node[draw=green] at (6,-.3) [rectangle] (A6)
       {\bbox{8}{Needs of society\\[12pt] and the marketplace}};   
  \draw[color=red] (1,.3) node
       {\bbox{2}{\footnotesize \sc{Market Pull}}};
  \draw[color=red] (11.3,3.6) node
       {\bbox{2.3}{\footnotesize \sc{Technology Push}}};
  \draw[color=red] (-.5,4.4) node {\bbox{2}{New\\ Technology}};
  \draw[color=red] (-.5,-.5) node {\bbox{2}{New\\ Ideas}};
  
  \draw[<->] (1.2,2) -- (2,2) ;
  \draw[<->] (4,2) -- (5,2) ;
  \draw[<->] (7,2) -- (8,2) ;
  \draw[<->] (10,2) -- (10.8,2) ;
  \draw[<->] (3,2.5) -- (3,3.5) ;
  \draw[<->] (6,2.5) -- (6,3.5) ;
  \draw[<->] (9,2.5) -- (9,3.5) ;
  \draw[<->] (3,1.5) -- (3,0.5) ;
  \draw[<->] (6,1.5) -- (6,0.5) ;
  \draw[<->] (9,1.5) -- (9,0.5) ;
  \draw[<->] (-.4,4) -- (-.4,3.3) ;
  \draw[<->] (.4,4.5) -- (1.4,4.5) ;
  \draw[<->] (-.4,0) -- (-.4,0.7) ;
  \draw[<->] (.4,-.5) -- (1.4,-.5) ;
  \draw[->,dashed] (1,3) arc(120:60:10) ;
  \draw[<-,dashed] (1,1) arc(-120:-60:10) ;
\end{tikzpicture}
\end{center}
\caption{Eine Reproduktion von \cite[Fig. 3]{Preez2006}, die dort von
  \cite{Galanakis2006} übernommen ist.}
\end{figure}

Wir haben damit bereits \emph{drei} sozio-technische Obersysteme (Ökonomie,
Innovationsmanagement, Wissenschaft und Technologie) mit jeweils eigenen
Begriff"|lichkeiten, Strukturen, Komponenten, Beschreibungs- und
Vollzugsformen identifiziert, die in der einen oder anderen Weise zur in
\cite{TESE2018} behandelten Thematik in Bezug stehen, ohne damit der uns
interessierenden Präzisierung des Begriffs \emph{technisches System} --
besonders auch in Abgrenzung zum Begriff \emph{sozio-technisches System} --
näher gekommen zu sein. Die TRIZ-Methodik empfiehlt in einem solchen Fall, den
Zugang von einer anderen Seite neu zu versuchen, vorher aber die Erkenntnisse
aus dem Fehlversuch zu fixieren.  Diesbezüglich sind drei Aspekte interessant:

\paragraph{1)}
Der Anspruch von \cite{TESE2018}, (auch) einen innovationsmethodischen Beitrag
zu leisten, setzt auf weitgehend überholten innovationsmethodischen Konzepten
auf, die dort entwickelte „pragmatische S-Kurven-Analyse“ muss ihre
Passfähigkeit zu moderneren innovationsmethodischen Konzepten erst noch unter
Beweis stellen.

\paragraph{2)}
Die Argumentation in \cite{TESE2018} nimmt impliziten Bezug auf gängige
innovationsmethodische Theorieansätze, wie sie in \cite{Preez2006}
systematisch dargestellt sind.  Insbesondere \cite[Fig. 3]{Preez2006} geht gut
als Diagramm einer Root Cause Analysis durch mit \emph{Idea Generation} (A)
und \emph{Commercial Product} (B) als die beiden Hauptkomponenten und weiteren
Hilfskomponenten (Development, Manufacturing, Marketing and Sales).  Es gibt
zwei Root Cause Pfeile -- $(A) \longrightarrow (B)$, vermittelt durch „State
of the art in science and technology“, und $(B) \longrightarrow (A)$,
vermittelt durch „Needs of society and the marketplace“. Eine TRIZ Root Cause
Analysis, wenigstens in der in \cite[Kap. 4.7]{KS2017} beschriebenen Weise,
kann mit derartigen \emph{Rückkopplungsschleifen} nicht arbeiten, da diese
keine „Wurzel“ haben. Das wichtige TRIZ-Prinzip 25 der \emph{Rückkopplung} ist
an dieser Stelle nicht anschlussfähig.  Eine zweite terminologische Feinheit
ist anzumerken: „push“ und „pull“ beziehen sich im englischen
Sprachgebrauch\footnote{Dasselbe gilt für die informatische Semantik der
  Operationen \texttt{push} und \texttt{pull}, siehe etwa
  \url{https://de.wikipedia.org/wiki/Stapelspeicher}. } auf verschiedene Enden
des Kausalpfeils. Im Sinne des TRIZ-Funktionsmodells \cite[Kap. 4.4]{KS2017}
ist „push“ eine Funktion des Werkzeugs, „pull“ eine Funktion des Zielobjekts,
eines „aktiven Objekts“ im Sinne einer „inversion of control“\footnote{Dazu
  etwa \url{https://de.wikipedia.org/wiki/Inversion_of_Control}.}, etwa nach
dem TRIZ-Prinzip 13 der \emph{Funktionsumkehr} zu verstehen.  Im Diagramm
\cite[Fig. 3]{Preez2006} werden diese beiden Begriffe offensichtlich auf die
Komponente \emph{Commercial Product} bezogen.  Für \emph{Idea Generation} als
Bezugskomponente dieses weitgehend symmetrischen Arrangements drehen sich die
Begriffe um -- push wird zum pull und umgekehrt.

\paragraph{3)}
Wir haben \emph{mehrere} Obersysteme identifiziert, womit noch einmal deutlich
wird, dass der Begriff \emph{Obersystem} nicht immersiv zu denken ist und
nicht mit dem Begriff \emph{Umwelt} verwechselt werden darf. Obersysteme sind
spezifische Nachbarsysteme mit eigener Sprache und Logik. Die Beziehung
Obersystem -- System ist dieselbe wie die Beziehung System -- Komponente und
oben hinreichend genau beschrieben: Es handelt sich um zwei verschiedene
Betrachtungsperspektiven auf die „Totalität der Welt“ mit zwei verschiedenen
Begriffen des \emph{Wesentlichen} und damit aus zwei verschiedenen
Reduktionsperspektiven. Ins Obersystem geht das System allein durch seine
Spezifikation (Beschreibungsdimension) und das Versprechen
spezifikationskonformer Leistung (Vollzugsdimension) ein. Im Konzertbeispiel
wird aus dem \emph{gegebenen} Können der Musiker auf der Ebene des Orchesters
die Interpretation des Musikstücks „produziert“.  Die
Leistungs\emph{fähigkeit} der Teilsysteme ist dabei dem
Leistungs\emph{potenzial} des Obersystems vorgängig -- die Interpretation von
Mozarts Klavierkonzert KV 491, die Alexander Shelley mit dem Leipziger
Gewandhausorchester und Gabriela Montero als Solistin vorgelegt hat, wäre mit
einem Laienorchester nicht möglich gewesen. Aus der Perspektive des Systems
agiert das Obersystem ebenfalls funktional: Die Schnittstelle des Systems
definiert in der Beschreibungsdimension eine erwartete Input- oder
Durchsatzleistung in Quantität, Qualität und Struktur, die dem Funktionieren
des Systems (zur Laufzeit) vorgängig ist, das Obersystem stellt in der
Vollzugsdimension diese Voraussetzungen sicher. Eine solche Strukturierung
bringt die aus der Mathematik bekannten deduktiven Wenn-Dann-Beziehungen mit
realweltlichen Prozessen in Verbindung, wobei der modus ponens in
realweltlichen Systemen eine deutlich komplexere Semantik hat als in der
Mathematik, wo dieser bekanntlich ausschließlich als logische Klammer eine
Rolle spielt, wenn aus kleineren Wenn-Dann-Beziehungen komplexere solche
Beziehungen (Lemmata, Sätze, Theoreme bis zum Beweis eines der
Millenium-Probleme) regelgerecht deduziert werden sollen.  Auch hier hat
dieser Reduktionsmechanismus einen klaren Zweck -- arbeitsteilige Reduktion
von Komplexität.

\section{Zum Evolutionsbegriff. Die ideengeschichtliche Dimension}

Kehren wir jedoch zu \cite{TESE2018} und unserer Frage nach der Fundierung des
Begriffs \emph{technisches System} zurück. Dem Rat der TRIZ-Methodik folgend,
es mit einem anderen Ansatz zu versuchen, wenden wir uns dem „technology push“
\cite[S. 2]{TESE2018} zu, „that creates new systems, products, and services,
that are not yet required by the market“, um aus dieser Perspektive die
systemischen Begriff"|lichkeiten zu entwickeln.  Zunächst sind Zweck und
Nutzen eines solchen Systems zu bestimmen.

Aus der Perspektive des „Coupling Model of Innovation“ in
\cite[Fig. 3]{Preez2006} ist dieses System als „Ideengenerator“ die sprudelnde
Quelle, deren Produkte das formende Flussbett des „state of the art in science
and technology“ durchfließen, um sich an dessen Ende durch einen „technology
push“ in „commercial products“ zu verwandeln.  Die „Ideenquelle“ wird (ebenda)
durch einen „market pull“ angetrieben, der aus den „needs of society and the
marketplace“ gespeist wird. Wie oben bereits erläutert sind diese Termini mit
der TRIZ-Welt von „Werkzeug“, „Aktion“, „Objekt“ und „Produkt“ sowie dem
Konzept der Systemtransformation etwa auch nach \cite{TT} relatierbar, wären
als solche aber nicht Teil der Analyse des uns interessierenden Systems,
sondern des Obersystems „Innovationsmanagement“. In der Tat erfüllt dieses
Begriffs"|universum dessen Reduktionserfordernis „auf das Wesentliche“, in
welches die „Ideenquelle“ als Komponente allein mit ihrer funktionalen
Spezifikation eingeht, von deren genauem inneren Funktionieren beim
Innovationsmanagement -- jedenfalls im Verständis von \cite{Preez2006} --
abstrahiert wird.

Das ist in \cite{TESE2018} natürlich anders, denn mit der Untersuchung der
Evolution ingenieur-technischer Systeme soll ja gerade die ideengeschichtliche
Dimension derartiger Prozesse untersucht werden.  Zu diesem Zweck heißt es
bereits in der Überschrift von Kapitel 1 „Technology Pull: Beyond Technology
Push and Market Pull“.  Wie ist das zu verstehen? Wird der Fokus auf die
Komponente \emph{Idea Generation} gerichtet? Dann wären in der Tat die
Begriffe „push“ und „pull“ gegenüber \cite{Preez2006} zu vertauschen -- aber
von „market push“ ist keine Rede.  Stattdessen wird in den weiteren
Ausführungen in \cite{TESE2018} deutlich, dass in der Tat ein \emph{Umdrehen}
des oberen Pfeils gemeint ist, die Komponente \emph{Idea Generation} des
Obersystems (!) Innovationsmanagement Input aus \emph{zwei} Richtungen
bekommt, die beide vom \emph{Commercial Product} als Quelle -- also Werkzeug
im TRIZ-Verständnis -- ausgehen.

So deutlich wie hier formuliert wird in \cite{TESE2018} allerdings auf die
innovationstheoretischen Vorstellungen von \cite{Preez2006} nicht Bezug
genommen.  Wir folgen dennoch einer solchen weiteren Modellreduktion und
wollen den Leser dabei zugleich von einem Dilemma befreien -- dem
Zirkelschluss zwischen den Komponenten \emph{Idea Generation} und
\emph{Commercial Product}.  Dazu lassen wir die beiden nun auf \emph{Idea
  Generation} gerichteten Pfeile in den Komponenten „Wissenschaft und Technik“
und „Bedürfnisse/Markt“ starten und überlassen das \emph{Commercial Product}
der Teilkomponente \emph{Marketing and Sales} der Komponente
\emph{Unternehmen}.

Unsere anstehende TRIZ-Analyse der inneren Struktur und Funktionsweise des
Systems mit dem provisorischen Namen „Ideengenerator“ ist damit auf die
Erforschung der Beziehung zwischen den drei \emph{Komponenten}\footnote{Oder
  „Nachbarsysteme“?  -- Nein. Nachbarsysteme wären alle vier nur aus der
  Perspektive eines Obersystems.  Die früher als Obersystem identifizierten
  Systeme erscheinen nun -- in temporärer Ermangelung eines anderen
  TRIZ-Begriffs, im System selbst als Komponenten. Eine Lösung des Rätsels
  wird weiter unten vorgeschlagen. } „Bedarf“, „Technologie“ und „Unternehmen“
auszurichten, die -- nach dem oben entwickelten Systembegriff -- mit ihren
\emph{gegebenen} funktionalen Spezifika Input, Output und Durchsatz die innere
Struktur des zu beschreibenden Systems bestimmen.  Ein solcher Ansatz passt
deutlich besser auf die Ausführungen in \cite[Kap. 1]{TESE2018}, denn die
Autoren betonen „parameters of the technology of the system are linked to
parameters of the market“ -- der Einfluss der beiden Komponenten wird also als
nicht unabhängig voneinander betrachtet, sondern lässt sich auf der Ebene von
Parameterkopplungen fassen (so jedenfalls die Autoren).

Wir sehen an diesem Zugang ein grundsätzliches Verfahren, ferne Komponenten in
ein System zu integrieren: die beiden relationalen -- also feldartigen im
Sinne einer Stoff-Feld-Modellierung wie in \cite[Kap. 4.9]{KS2017} genauer
entwickelt -- Beziehungen $(B)\stackrel{F_1}{\longrightarrow} (A)$
(„technology pull“) und $(B) \stackrel{F_2}{\longrightarrow} (A)$ („market
pull“) im Obersystem \emph{Innovationsmanagement} erscheinen im System (A) als
Komponenten $(F_1)$ und $(F_2)$, zwischen denen ein Feld
$(F_1)\stackrel{B}{\longleftrightarrow}(F_2)$ postuliert wird, um die für das
System \emph{wesentlichen} Beziehungen zwischen $(F_1)$ und $(F_2)$ zu
modellieren, die sich aus der Existenz der (fernen, da nicht ins System
übernommenen) Komponente (B) im Obersystem ergeben.  Einen solchen für
Modellreduktionen wichtigen Standard, der allerdings in den 76 TRIZ-Standards
fehlt, hatte ich in \cite{Graebe2019a} als \emph{Stoff-Feld-Swap} bezeichnet.
So ist wohl auch M. Rubin zu verstehen, wenn er anmerkt, dass „externe
(menschliche, kulturelle) Beziehungen durch zusätzliche Anforderungen oder
Einschränkungen an technische Objekte ersetzt“ und damit in technische Systeme
importiert werden können, ohne diese zu einem sozio-technischen System
„aufzublasen“.

Doch kehren wir zur Rekonstruktion des Begriffssystems von \cite{TESE2018}
zurück.  Da es nicht über"|haupt um die Evolution technischer Systeme geht,
sondern um die sich aus der TRIZ-Theorie ergebenden Schlussfolgerungen für
eine solche Evolution, ist -- dem TRIZ-Standard 1.1.1 \emph{Vervollständige
  ein unvollständiges Stoff-Feld-Modell} folgend -- als weitere Komponente der
\emph{TRIZ-Theoriekörper} zu berücksichtigen. Auch dieser Korpus ist als
Komponente zu modellieren, da dieser in \cite{TESE2018} in seiner Gesamtheit
als nicht weiter zu hinterfragende Einheit auftrtt, also allein mit seiner
Input/Output-Leistung in die Systemmodellierung eingeht.  Außerdem
unterscheiden sich die MPV beider Dimensionen deutlich -- während im
TRIZ-Theoriekörper das \emph{Lösen von Problemen} im Fokus steht, geht es in
\cite{TESE2018} um die \emph{Evolution in der „Welt der technischen Systeme“}.

Wir befinden uns mit der bisherigen Begriffsbildung auf der Abstraktionsebene
des Technikbegriffs des VDI (Verein Deutscher Ingenieure -- der deutschen
Standesorganisation der Ingenieure), der in der VDI-Richtlinie 3780 den
Technikbegriff in den folgenden drei Dimensionen fasst:
\begin{itemize}
\item Menge der nutzenorientierten, künstlichen, gegenständlichen Gebilde
  (Artefakte oder Sachsysteme);
\item Menge menschlicher Handlungen und Einrichtungen, in denen Sachsysteme
  entstehen und
\item Menge menschlicher Handlungen, in denen Sachsysteme verwendet werden.
\end{itemize}
Die Definition vermeidet den Begriff „technisches System“ zugunsten von
„nutzenorientierten, künstlichen, gegenständlichen Gebilden“. Ähnlich
\cite{Shpakovsky2003} mit dem Begriff des \emph{technischen Objekts}, der dort
dem Begriff des \emph{technischen Systems} vorgängig ist. In der hier
vorgelegten VDI-Definition werden neben der artefaktischen Dimension auch noch
„Sachsysteme“ einbezogen werden -- neben der Maschine also auch noch die
Maschinerie\footnote{Marx geht hier noch weiter: Das „\emph{automatische
    System der Maschinerie} [\ldots] verwandelt die Maschinerie erst in ein
  System“ \cite[S. 584]{MEW42}.} und damit die Unikate technischer
Großsysteme.  Mit der unmittelbaren Verknüpfung der „gegenständlichen
Sachsysteme“ und den Bedingungen und Folgen ihrer Entstehung sowie Verwendung
wird einer Unterscheidung zwischen technischen und sozio-technischen Systemen
bereits im Grundansatz widersprochen.

\section{Was sind Komponenten?}

Auf diese Frage hat \cite{Szyperski2002} eine einfache Antwort: „Components
are for Composition“.  Diese Definition folgt dem oben entwickelten
Verständnis, dass Systeme aus \emph{bereits vorhandenen} Komponenten
zusammengesetzt werden, wobei neben OTS-Komponenten (off the shelf) auch
selbst entwickelte Komponenten eingesetzt werden können. Allerdings müssen vor
der Integration zum (lauf"|fähigen) Gesamtsystem diese Entwicklungen
abgeschlossen sein. Im Wasserfallmodell der Softwareentwicklung wird diese
sequenzielle Vorgehensweise expliziert, agile Vorgehensmodelle sind flexibler,
erfordern zur frühzeitigen prototypischen Demonstration von
Teilfunktionalitäten aber auch Platzhalterkomponenten -- Mock-Komponenten und
provisorische Oberflächen.

In \cite{Szyperski2002} zerfällt mit diesem Verständnis die Welt der
Produktion technischer Systeme in zwei Teilwelten -- „design to component“ und
„design from component“. Ersteres ist das Gebiet der Komponentenentwickler mit
dem Fokus, Komponenten mit einer speziellen Fachfunktion („core concern“ --
dies entspricht dem MPV in \cite{TESE2018}) zu entwickeln. Neben dieser
Fachfunktion muss die Komponente aber noch eine größere Menge von
Hilfsfunktionen (Logging, Datensicherheit, Zugangsmanagement,
Druckeransteuerung usw. -- die „cross cutting concerns“) erfüllen, die auf
etablierte Konzepte (Beschreibungsdimension) und andere zu integrierende
Komponenten (Vollzugdimension) \emph{anderer technischer Prinzipien und
  Systeme} zurückgreifen.  Komponenten sind in einem solchen Verständnis stets
\emph{Bündel von Funktionen}, die zugleich Verfahrenswissen aus
\emph{mehreren} Bereichen bündeln.  All diese Beschreibungsformen muss der
Komponentenentwickler wenigstens in der Abstraktion ihrer \emph{Spezifikation}
beherrschen, um nützliche Komponenten zu bauen.  Zweiteres ist das Gebiet der
Komponentenassembler. Diese bauen (vorher zu entwerfende) Systeme aus
verfügbaren Komponenten zusammen, entwickeln oder modifizieren weitere
Hilfsfunktionalitäten (den „glue code“), integrieren und testen das
Gesamtsystem, bevor es beim Kunden zum Einsatz kommt.

Die Schnittstelle zwischen beiden Professionen bildet das verwendete
\emph{Komponenten-Frame"|work} wie etwa Spring Boot, das nicht nur durch
Normungen und Standardisierungen das prinzipielle Zusammenwirken der
Komponenten auf einer höheren Ebene der Abstraktion beschreibt
(Beschreibungsdimension), sondern auch von verschiedenen Anbietern als
Laufzeitsystem für Komponenten (Vollzugsdimension) zur Verfügung gestellt
wird. Derartige Laufzeitsysteme -- sicher ebenfalls technische Systeme --
haben eine Spezifik: sie werden \emph{gemeinsam} von Anbieter und Kunde
betrieben, was eine Koordination der sozio-technischen Begleitprozesse auf
hohem Niveau erfordert. In der Informatik hat sich dabei ein System von
Serviceleveln bewährt, die vertraglich vereinbart werden und die Verantwortung
zwischen Anbieter und Kunde verteilen. Bei drei Serviceleveln liegt gewöhnlich
die Verantwortung für die Auswahl und Schulung des Personals für die
Fachanwendung (Level 1) komplett beim Kunden, die Konfiguration und
Rekonfiguration des Laufzeitsystems beim Kunden (Level 2) wird entweder vom
Anbieter (im Rahmen eines „Produktlinien-Managements“) oder einer
spezialisierten Fachabteilung des Kunden übernommen, Wartungen, Updates des
Systems sowie Integration oder Reintegration neuer Komponenten (Level 3) liegt
in den Händen des Anbieters. In einem solchen System liegt eine arbeitsteilige
Situation vor -- der Anbieter ist für die Qualität der Funktionalität
zuständig, der Kunde für die Qualität der Daten. Außerdem verantwortet der
Kunde die Funktionen und Fehlfunktionen des Systems vor der Allgemeinheit und
muss sich entsprechende Schäden, die durch den Betrieb des Systems verursacht
wurden, im Rahmen eines Anscheinsbeweises rechtlich zuordnen lassen.  Die
operative (technische) Qualität des realweltlichen Systems wird in diesem
sozio-technischen Verhältnis von \emph{beiden} Parteien gleichermaßen
beeinflusst, so dass eine sinnvolle Trennung von technischer und
sozio-technischer Ebene \emph{praktisch} unzweckmäßig ist und so auch nicht
erfolgt.

Im Arbeitsprozess werden dabei zu bearbeitende Objekte (in diesem Fall Daten)
zwischen den verschiedenen Komponenten -- hier nicht nur Funktions- sondern
auch Verantwortungseinheiten -- ausgetauscht, was insbesondere im Falle
unzureichender Qualität der Objektbearbeitung zu weiteren Widersprüchen führt.
Für C. Szyperski \cite{Szyperski2002} sind diese allein sozio-technisch
begründeten Widersprüche der Ausgangspunkt für eine deutliche Abgrenzung der
Begriffe \emph{Komponente} und \emph{Objekt}, in der die im Kontext der
objekt-orientierten Programmierung erfolgte Zusammenführung von Funktion und
Verhalten in Objektbegriff wieder zurückgenommen und die Rolle von Objekten
konsequent auf das TRIZ-Prinzip 24 des \emph{Vermittlers} beschränkt wird.
Ich komme weiter unten auf diese Problematik zurück.

\section{Normierung und Standardisierung}

Dieses Vorgehen in der Softwarebranche ist auch in vielen
ingenieur-technischen Anwendungen präsent. „Baukastensysteme“ sind weit
verbreitet und erlauben es, realweltliche technische Unikat-Systeme auf
dieselbe Weise zu entwerfen wie im Konzertbeispiel erläutert. Während dort
aber auf das \emph{private Verfahrenskönnen} der Musiker als Voraussetzung
abgestellt wurde, werden hier die \emph{Logik der Fachanwendung} als „core
concerns“ der Komponenten mit der \emph{Logik der Vernetzung} der
Infrastruktur als „cross cutting concerns“ zusammengeführt. Beide Logiken sind
orthogonal zueinander, womit die Trends 4.2 „of increasing system
completeness“ und 4.4 „of transition to the supersystem“ einander praktisch
entgegenwirken. 
\begin{quote}\it 
  \textbf{These 2:} Ein besseres beschreibungstechnisches Verständnis der
  Infrastrukturanforderungen miteinander agierender Komponenten (Übergang zum
  Obersystem) führt zu einer \emph{Abschwächung} der Anforderungen an die
  Vollständigkeit der einzelnen Komponenten.
\end{quote}
Insbesondere die zur Begründung von Trend 4.2 in \cite{TESE2018} angeführte
Hierachisierung in „operating agent“ (als Kernfunktion), „transmission“
(Unterstützung durch ein Arbeitsmittel), „energy source“ (Einsatz von
Naturkräften) und „control system“ (Einsatz von Steuerung als prozesshaftem
technischen Teilsystem und Quelle des „Trends der Dynamisierung“) sind von
diesen Entwicklungen betroffen, wie ein Besuch im Baumarkt unmittelbar zeigt
-- die Maschinensysteme namhafter Hersteller konzentrieren sich auf die
Bereitstellung der Energie, über entsprechende APIs (etwa Klett-, Schraub-
oder Klickverschlüsse auf mechanischer Ebene) können passende Werkzeuge mit
der Energiemaschine gekoppelt werden\footnote{Wobei durch Fortschritte der
  Materialwissenschaften insbesondere mit Klettverschlüssen eine massive
  Rückkehr zu \emph{mechanischen} Kopplungsprinzipien entgegen dem
  TRIZ-Prinzip 28 des \emph{Austauschs mechanischer Wirkschemata} zu
  verzeichnen ist.}, wobei je nach Geschäftsstrategie der namhaften Hersteller
der jeweilige Technologie-Teilmarkt „passender Gerätschaften“ monopolisiert
oder auch für weniger namhafte Hersteller von passenden Arbeitsmitteln
geöffnet ist. In beiden Fällen spielen \emph{Normierung und Standardisierung}
in dieser „Welt der technischen Systeme“, also inhärent sozio-technische
Prozesse, eine deutlich größere Rolle als die Weiterentwicklungen der rein
technischen Artefakte. Gleiches gilt auf der Ebene der Kontrollsysteme, wo
programmierbare Universalsteuerungen wie die UVR 1611 der Firma Technische
Alternative das „Herz“ vieler technischer Regelungen im Smart-Home-Bereich
bilden.

Ein solcher Normierungsprozess öffnet zugleich ökonomische Skaleneffekte für
Standardkomponenten, d.h. für Umsetzungskonzepte, die sich bereits in Richtung
„idealer Endresultate“ etabliert haben. Die Skaleneffekte wirken sich
kosten\emph{senkend} pro Einzelstück aus und verschieben damit die Leitwirkung
vom Wettbewerb um die bessere \emph{technische} Lösung zum Wettbewerb um die
kostengünstigere \emph{ökonomische} Produktion.  Die S-Kurve endet also nicht
unbedingt -- und wohl auch eher selten -- mit der Außerdienststellung im
Stadium 4 \cite[S. 38]{TESE2018}, sondern geht auf dem Höhepunkt ausgereifter
\emph{technischer} Qualität (einschließlich Normierung und Standardisierung)
in eine Phase der \emph{allgemeinen Verfügbarkeit} über, in der die
\emph{immer geringeren} ökonomischen Aufwendungen für die Verfügbarkeit dieses
„Stands der Technik“ die Leitfunktion der weiteren Entwicklung übernehmen.

Der Trend 4.1 „of increasing (technical) value“ schlägt dabei in einen Trend
„of decreasing economic value“ um, oder -- um es in ökonomischen Termini
auszudrücken -- der vorher durch die Nachfrage getriebene Markt geht in einen
vom Angebot getriebenen Markt über: Derselbe (reife) Gebrauchswert hat einen
immer geringeren Tauschwert.  Damit geht der Wert der „Idealität“
\cite[Kap. 4.1.1]{KS2017} in der Tat durch die Decke, aber als Folge eines
\emph{ökonomischen} Gesetzes.  Dies korrespondiert zum TRIZ-Prinzip 17 des
\emph{Übergangs zu anderen Dimensionen}.
\begin{quote}\it
  \textbf{These 3:} Der (technische) Trend 4.1 „of increasing (technical)
  value“ schlägt im Stadium 3 der S-Kurven-Entwicklung um in einen
  (ökonomischen) „Trend of decreasing (economic) value“.
\end{quote}
Damit wechselt im Stadium 3 in der Produktion gängiger Werkzeuge und
Standardkomponenten die Leitfunktion (MPV) der weiteren Entwicklung von den
technischen Triebkräften zu den ökonomischen. Diesen Prozess der
„Commodification“ hat F. Naetar in \cite{Naetar2005} hinreichend beschrieben;
das Thema muss hier also nicht vertieft werden. Diese Entwicklung ist
allerdings selbst widersprüchlich, wie in der marxistischen Literatur am
Phänomen der \emph{tendenziell fallenden Profitrate} diskutiert wird:
Geringere Produktionskosten durch technischen Fortschritt eines Produzenten
erhöhen dessen Profitrate im Vergleich zu den Konkurrenten. Der Marktpreis
(„decreasing economic value“) wirkt allerdings regulierend und senkt
perspektivisch die Profitrate der Wettbewerber, die diesen technischen
Fortschritt nicht oder zu spät implementieren.

Das TRIZ-Prinzip 17 des \emph{Übergangs zu anderen Dimensionen}, auf das oben
Bezug genommen wurde, erscheint hier allerdings -- im Gegensatz zur Lesart in
der Komponente TRIZ-Theoriekörper -- nicht als \emph{abstraktes Designmuster},
sondern als \emph{abstraktes Evolutionsmuster}, also nicht als Mittel der
aktiven Beeinflussung eines Problemlöseprozesses, sondern als
passiv-beobachtendes Beschreibungsmuster realweltlicher Entwicklungen.  In
diesem Sinne kann aber auch \emph{jedes andere} der TRIZ-Prinzipien sowie auch
jeder der TRIZ-Standards als abstraktes Evolutionsmuster formuliert
werden. Umgekehrt erscheinen die Evolutionstrends in der Komponente
TRIZ-Theoriekörper als weitere abstrakte Designmuster, die neben die
„Prinzipien“ und die „Standards“ treten.
\begin{quote}\it
  \textbf{These 4:} Jedes der TRIZ-Prinzipien und jeder der TRIZ-Standards
  kann auch überzeugend als „Trend der Evolution technischer Systeme“
  formuliert werden und umgekehrt.
\end{quote}
Die Hierarchie der Evolutionsmuster gibt damit insbesondere Anlass zu einer
„Hierarchie der Lösungsprinzipien“ \cite[Kap. 3]{Zobel2016}, wie Dietmar Zobel
bereits vor über 10 Jahren vorgeschlagen hat, siehe auch \cite{Zobel2020}.
Damit wird zugleich die Bedeutung der „Matrix“ entwertet. Leonid Shub
\cite{Shub2006} weist darauf hin, dass dies auch Altschuller bereits 1985
engeren Vertrauten gegenüber geäußert habe.  M. Rubin schlägt in
\cite{Rubin2019} in diesem Sinne den Bogen von Entwicklungsgesetzen zu
TRIZ-Standards und weiter zur Algorithmisierung von Vorgehensweisen in
erfinderischen Praxen im ARIS, was genauer zu untersuchen bleibt.

Mit dem Verzicht auf die Betrachtung sozio-ökonomischer Zusammenhänge bleibt
die Bedeutung von Prozessen der Normierung und Standardisierung in
\cite{TESE2018} allerdings ausgeblendet. Damit versperren sich die Autoren
aber selbst den Blick in eine lebendige Welt technischer Systeme in einem
fortgeschrittenen Zustand der Evolution. So zeichnet sich das technische
System der \emph{Schraubverbindungen} durch eine Massenproduktion genormter
Maschinenschrauben und Holzschrauben aus, was nun genauer ausgeführt werden
soll.

Für die Herstellung von Maschinenschrauben ist hohe Präzision und Stimmigkeit
von Durchmesser und Anstellwinkel der Gewinde erforderlich, damit diese mit
den Gegenstücken zusammenpassen.  Diese Präzision wird nicht nur durch eine
industrielle Herstellungsweise erreicht, sondern für Spezialanwendungen auch
mit entsprechenden Werkzeugen -- etwa einem Gewindeschneider.  Mit Schlitz-,
Kreuzschlitz-, Sechskant-, Senkkopf-, Inbus- usw. -schrauben gibt es ein
großes Sortiment vorgefertigter Lösungen für verschiedene Einsatzszenarien
(TRIZ-Prinzip~3 der \emph{lokalen Qualität}), dazu entsprechende Werkzeuge:
Schraubenschlüssel, Steckschlüssel, Schraubendreher, Inbus-Schlüssel
usw. (noch einmal TRIZ-Prinzip 3), sowohl als Einzelwerkzeuge wie auch als
Einsätze für den Akkuschrauber als Energiemaschine (TRIZ-Prinzip 1 der
\emph{Zerlegung}, TRIZ-Standard 3.1 \emph{Übergang zu einem Bi-System}).
Biegsame Schraubendreher\footnote{Amazon bietet ein solches 31-teiliges Set
  der Firma Lotex GmbH für 20,99 Euro an.} (zusammen mit dem Akkuschrauber
TRIZ-Standard 3.1 \emph{Übergang zu einem Poly-System}) können verwendet
werden, um Schraubverbindungen auch an schwer zugänglichen Stellen einzusetzen
usw.  Diese Werkzeuge werden auch von Industrierobotern eingesetzt (hier ist
der TRIZ-Standard 3 \emph{Übergang zu einem Obersystem} zweimal anzuwenden,
denn die Industrieroboter sind Komponenten im Ober-Ober-System).

Die Welt der Holzschrauben vermeidet das Zwei-Komponenten-System (noch einmal
TRIZ-Standard 3.1: Schraube und Mutter), indem der Halt im zu bearbeitenden
Material selbst gesucht wird (Trend 4.6 \emph{of increasing degree of
  trimming} -- wieso gehört diesTrimmen als zentrale TRIZ-Methode weder zu den
„Prinzipien“ noch zu den „Standards“?), entweder durch Vorbohren (TRIZ-Prinzip
10 \emph{der vorherigen Wirkung}) oder durch eine selbstschneidende Schraube
(TRIZ-Prinzip 25 der \emph{Selbstbedienung} oder auch wieder Trend 4.6 des
\emph{Trimmens}). Leider bieten manche Materialien diesen Halt nicht, es
kommen zusätzlich \emph{Dübel} zum Einsatz (TRIZ-Nicht-Trend des
\emph{Anti-Trimmens}), inzwischen eine eigene Welt technischer Lösungen, die
das Herz jedes TRIZ-Praktikers höher schlagen lässt.  Und da haben wir noch
nicht über spezielle Anwendungen von Schraubverbindungen wie in der Chirurgie
gesprochen, wo wesentliche Parameter an Material und Zuverlässigkeit aus den
Bedingungen des Obersystems zu sehr speziellen Systemlösungen führen.

Ich habe diese Welt so ausführlich beschrieben, um drei Aspekte zu
verdeutlichen:
\begin{itemize}
\item [1.] Es ist eine Welt technischer Systeme, in der Prinzipien des
  Problemlösens auf TRIZ-Basis eine wichtige Rolle spielen.
\item [2.] Das strukturierende Moment in jener Welt sind nicht die technischen
  Systeme, sondern die \emph{technischen Prinzipien}.
\item [3.] Die 10 „Trends“ sind wenig hilfreich, um sich in dieser Welt
  hochvolatiler Anforderungssituationen zurecht zu finden, weil stets nach
  \emph{konkreten} Lösungen in \emph{konkreten} Kontextualisierungen gefragt
  wird, und dabei nicht die „Evolution einzelner technischer Systeme“ eine
  Rolle spielt, sondern ein globaler \emph{Stand der Technik}, in dem sich die
  „Evolution der Welt der technischen Systeme“ als Ganzes spiegelt. 
\end{itemize}

Nun soll an dieser Stelle nicht das Kind mit dem Bade ausgeschüttet werden,
denn die Beobachtungen des letzten Abschnitts haben sich aus der Inspektion
eines Bereichs der Evolution technischer Systeme ergeben, der in
\cite{TESE2018} ausgeblendet bleibt.  Insofern hat das Ergebnis unserer
Untersuchungen den Charakter einer \emph{partiellen TRIZ-Lösung} und es bleibt
die Frage zu beantworten, in welcher Kontextualisierung sich die
Argumentationen in \cite{TESE2018} bewegen. Wir hatten gesehen, dass die
Autoren -- im Gegensatz zu N. Shpakovsky \cite{Shpakovsky2010} -- die
Entwicklungslinien \emph{nicht} um technische Prinzipien herum bauen, sondern
nur Entwicklungsbereiche betrachten, wo eine solche Ablösung technischer
Prinzipien von technischen Systemen noch nicht erfolgt ist.  Wir
diagnostizieren hier widersprüchliche Beschreibungsformen von zwei klar
voneinander getrennten Welten, so dass sich die Frage aufdrängt, ob dieser
Widerspruch mit dem TRIZ-Prinzip 36 der \emph{Anwendung von Phasenübergängen}
zu lösen ist, der hinter dem Wechsel der Leitfunktion von einer technischen zu
einer ökonomischen Dimension zu vermuten ist.

Vor einer vertiefenden Betrachtung dieser Frage ziehen wir hier zunächst
wieder ein Zwischenfazit.  Auf der Suche nach Strukturierungsprinzipien, die
einer Theorie der Evolution technischer Systeme zu Grunde gelegt werden
könnten, haben wir Normierungen und Standardisierungen näher analysiert. Wir
haben herausgearbeitet, dass diese eine zentrale Rolle spielen in einem
Transformationsprozess in der Welt technischer Systeme selbst -- dem Übergang
von einer primären Problemhaftigkeit einer „jungen“ Technologie zur
allgemeinen Verfügbarkeit einer „reifen“ Technologie.  Mit einem solchen
Phasenübergang ist zugleich ein allgemeines Agens der Entwicklung technischer
Systeme aufgedeckt, das in \cite{TESE2018} aus offensichtlich struturellen
Gründen keine Rolle spielt -- der Erfahrungshintergrund von \cite{TESE2018}
sind, wie von weiten Teilen der TRIZ, die erfinderischen Praxen \emph{vor}
diesem Phasenübergang, die sich an Patenten und der Weiterentwicklung des
„Stands der Technik“ orientieren.  Die zweite Phase aber, der flächendeckende
Betrieb einer allgemein verfügbaren Technologie, ist ebenfalls voller
Widersprüche und Gegenstand der Praxen einer neuen Generation von TRIZniks,
die viel enger mit den unmittelbaren Erfordernissen einer technisierten
\emph{Produktion} verbunden sind.  Vom Grundsatz her geht es dabei um die
\emph{Aufrechterhaltung einer gesellschaftlichen Normalität} als Grundlage der
„Fiktionen“ der „Normalbürger“ über das Funktionieren ihrer technischen
Umwelt.  Beides -- „Fiktionen“ und „Normalbürger“ -- steht hier in Quotes, da
sich dahinter komplizierte Prozesse der Komplexitätsreduktion von
Beschreibungsdimensionen verbergen, die in einem gesellschaftlichen
Synchronisationsverhältnis stehen, das sich parallel zur Entwicklung
technischer Systeme entwickelt.
\begin{quote}\it
  \textbf{These 5:} Die Evolution technischer Systeme bleibt ohne die
  Einbeziehung der Evolution alltäglicher reduktionistischer Vorstellungen
  über das Funktionieren von Technologien des Alltags unverständlich.   
\end{quote}

Die Betrachtungsperspektive in \cite{Shpakovsky2010} der \emph{Verbesserung
  vorgefundener} technischer Systeme, um Erfahrungen aus der Beratung großer
Produktionsbetriebe wie SAMSUNG genauer zu analysieren, nimmt die Phase 2 der
allgemeinen Verfügbarkeit einer Technologie stärker in den Blick und kommt
dabei auch zu einem anderen Verständnis der Entwicklung technischer Systeme
als \cite{TESE2018}.

\section{Zwecke}

In den letzten zwei Abschnitten haben wir uns der Frage nach dem Begriff
\emph{Technisches System} von zwei Seiten her zu nähern versucht. Die eine
Richtung startete beim Bedarf des sozio-ökonomischen Systems nach einem
\emph{Ideengenerator} und war schnell bei den Problemlösepraxen der
TRIZ-Praktiker angekommen. Der andere Weg führte über Normierungen und
Standardisierungen in eine Welt der Allgemeinverfügbarkeit technischer
Systeme, in der es keine Probleme mehr zu geben scheint, denn alles
funktioniert hervorragend -- wenigstens so lange die mit der Nutzung
verbundene \emph{Fiktion} gesellschaftlich aufrecht erhalten werden kann.

Aus beiden Richtungen landen wir in einer \emph{Welt technischer Systeme} oder
vielleicht auch nur technischer Artefakte, die Jürgen Mittelstraß in einem
umstrittenen Aufsatz \cite{Mittelstrass2011} als „schöne neue Leonardowelt“
bezeichnet hat.  Vielleicht sind es aber auch keine „Artefakte“ sondern eher
„technische Objekte“, die N. Shpakovsky in \cite{Shpakovsky2003} als
Ausgangspunkt für die Frage nimmt, ob denn jede Ansammlung technischer Objekte
bereits als technisches System bezeichnet werden kann oder welche zusätzlichen
Anforderungen hierfür zu stellen sind.  Dieselbe Frage muss sich die
VDI-Definition gefallen lassen.

Shpakovskys Antwort lautet, dass sich technische Systeme durch einen
\emph{wohldefinierten Zweck} auszeichnen, der \emph{von außen} vorgegeben ist
und den sie erfüllen müssen.  Der hier entwickelte Systembegriff ist für eine
solche Sichtweise gut geeignet, denn mit der \emph{Spezifikation} der
Komponente können Zwecke gut und genau formuliert werden. Weniger klar ist
zunächst, \emph{woher} diese Zwecke kommen.  Die TRIZ-Antwort ist eindeutig:
„Aus dem Obersystem“. Allerdings hatten wird in unserer Analyse festgestellt,
dass es durchaus \emph{mehrere} Obersysteme zu einem System geben kann und der
Begriff eines \emph{benachbarten} Systems der Situation eher angemessen ist.
Denkt man die Welt der technischen Systeme zunächst ohne Hierarchisierungen,
so finden wir eine Welt von \emph{Beziehungen} zwischen technischen Systemen
vor, aus denen heraus \emph{Zwecke} erklärt werden können: \emph{Jenes}
technische System ist entwickelt worden, weil \emph{dieses} dessen nützliche
Funktion als Betriebserfordernis benötigt. Der Zweck \emph{jenes} Systems ist
es also, \emph{diesem} zu dienen.  Technische Objekte \emph{bündeln} auf diese
Weise Funktionen und Dienste verschiedener Komponenten, um selbst Dienste
anzubieten.  Auch an dieser Stelle bietet sich wieder ein Substanz-Feld-Swap
an -- das Relationale, die einzelne Funktion, den einzelnen Dienst als
Substanz zu betrachten und die Funktionen bündelnden technischen Objekte als
Relationen, als Vermittler zwischen diesen Funktionen.  \emph{Zwecke} sind in
einem solchen Verständnis aus menschlichen Praxen erwachsende
\emph{Anforderungen}, nach denen solche Funktionsbündel zusammengestellt
werden.  \emph{Entwicklungslinien} solcher als Funktionsbündel entstandener
technischer Systeme beginnen bei einfachen Kompositionsprinzipien -- zu einer
Schraube eine passende Mutter finden --, reichen über bewährte
Verfahrensweisen -- was ist beim Streichen eines Fensters zu beachten? Welche
Farben auswählen? Welche Pinsel? Wie den Untergrund vorbereiten? Wie
streichen? -- bis hin zu höheren Abstraktionsformen wie etwa der Form, die
benötigt wird, um eine größere Menge von Ziegelsteinen aus Lehm zu formen,
diese zu brennen und dann aus ihnen ein ganzes Haus zu errichten.

Diese Zwecke strukturierenden Abstraktionen sind allerdings nicht willkürlich,
sondern folgen ihrerseits \emph{Zwecken zweiter Ordnung}.  Szyperski
\cite[S. 139\,ff.]{Szyperski2002} identifiziert unter der Überschrift „Aspekte
der Skalierung und Granularität“ eine längere Liste solcher „Zweck-Muster“,
nach denen \emph{Funktionen} zu Einheiten gebündelt -- also als Komponenten
zugeschnitten -- werden: Als
\begin{itemize}\itemsep0pt
\item Einheit der Abstraktion („design expertise embodied ready for use“)
\item Einheit der Abrechnung (Einheit der Kostenüberwachung)
\item Einheit der Analyse (Einheit der Fehlersuche)
\item Einheit der Übersetzung (Einheit eines Transformationsprozesses)
\item Einheit der Auslieferung (Einheit des Transports)
\item Einheit der Entpackung (Bausatz samt Montageanleitung)
\item Einheit der Disputation (Einheit der Auseinandersetzung um
  Verantwortlichkeiten) 
\item Einheit der Erweiterung 
\item Einheit des Fehlereinschlusses (fault containment)
\item Einheit der Instanziierung
\item Einheit der Installation
\item Einheit des Ladens usw.
\end{itemize}
Diese \emph{Zwecke zweiter Ordnung} sind ihrerseits nicht unabhängig
voneinander, wie \cite[S. 145]{Szyperski2002} bemerkt -- eine Einheit der
Analyse kann nicht sinnvoll in \emph{mehrere} Einheiten der Erweiterung
aufgebrochen werden.

Diese vielfältigen \emph{Praxen des Komponentenzuschnitts} stehen ihrerseits
nicht losgelöst voneinander, sondern konstituieren eigene Welten praktischer
Interaktionen und Erfahrungen. Im \emph{Komponentenframework} werden diese
Erfahrungen normiert und standardisiert.  Die Allgegenwart und Bequemheit der
Nutzung technischer Objekte reproduziert sich damit auf der Ebene
ingenieur-technischer Tätigkeit, auf der es aber wiederum nicht die Prinzipien
selbst, sondern die \emph{Passfähigkeit der Prinzipien} ist, auf die es
ankommt.  Auch diese Passfähigkeit fällt nicht vom Himmel, sondern ist
ihrerseits ein Resultat \emph{vernünftiger menschlicher Praxen}.

Die Welt der technischen Systeme ist damit in eine Welt der Beziehungen
zwischen technischen Systemen eingebettet, in der sich komplexe
sozio-technische Beziehungen spiegeln, die von konkreten Zwecken getrieben
werden. Diese Zwecke sind ihrerseits mannigfach aufeinander bezogen, und es
ist diese Beziehungsstruktur, die Gegenstand einer Strukturierung durch
„Zwecke zweiter Ordnung“ ist.  Dass damit das Ende der Fahnenstange noch nicht
erreicht ist, sondern Komponentenframeworks selbst durch übergreifende
\emph{Entwurfsmuster} \cite{Gamma1995} und Prozess"|templates wie „dependency
injection“ oder „inversion of control“ strukturiert werden usw., muss hier
sicher nicht im Detail erörtert werden.

Wir haben in diesem Abschnitt gesehen, dass es zur Untersuchung evolutionärer
Aspekte wichtiger sein könnte, die Welt der \emph{Beziehungen} zwischen
technischen Systemen in Augenschein zu nehmen als die Welt der technischen
Systeme selbst. In unserem systemtheoretischen Ansatz wird eine solche
Beziehung als Beziehung zwischen Komponenten eines Systems primär
\emph{funktional} betrachtet als Spezifikation in der Beschreibungsdimension
und als Versprechen garantierter spezifikationskonformer Leistung in der
Vollzugsdimension. Gegenstand der TRIZ-\emph{Methodik} ist die Transformation
des einen in das andere. Sie ist dabei selbst nur Beschreibungsform, die den
Domänenexperten hilft, diese Transformation \emph{praktisch} zu organisieren.
Funktionen erscheinen in diesem Transformationsprozess in drei Modi:
\emph{vor} der Transformation als \emph{Zweck}, als etwas, das man gern in der
Welt haben würde, \emph{in} der Transformation als Implementierung und
\emph{nach} der Transformation als \emph{Dienst}, als realisiertes
Versprechen. 
\begin{quote}\it
  \textbf{These 6:} Technische Systeme erscheinen als Funktionsbündel, deren
  einzelnes Element -- die Funktion -- in drei verschiedenen Modi: als Zweck,
  als Implementierung und als Dienst.
\end{quote}
Wir haben zugleich herausgearbeitet, dass sich derartige
Transformationsprozesse auch auf höheren Ebenen der Abstraktion in der Welt
technischer Systeme vollziehen. Standardisierungen auf höherer
Abstraktionsebene richten sich allerdings nicht so sehr auf Wiederverwendung
(reuse) als auf gemeinsame Nutzung (sharing). In \cite[Kap.9 ]{Szyperski2002}
werden unter der Überschrift „Pattern, Frameworks, Architectures“
(informatische) Beispiele einer solchen \emph{gemeinsamen Verwendung} genannt:
\begin{itemize}\itemsep0pt
\item sharing consistency -- programming languages
\item sharing concrete solution fragments -- libraries
\item sharing individual contracts -- interfaces
\item sharing individual interaction fragments -- messages and protocols
\item sharing individual interaction architectures -- patterns
\item sharing architectures -- frameworks
\item sharing overall structure -- system architectures
\item system of subsystems -- framework hierarchies
\end{itemize}

\section{Schichtenarchitekturen}

Im letzten Abschnitt hatten wir herausgearbeitet, das Zwecke,
Implementierungen und Dienste nicht nur auf der unmittelbaren technischen
Ebene bedeutsam sind, sondern auch \emph{Zwecke höherer Ordnung} auf die
Lebenswirklichkeit strukturierenden Einfluss haben, sich in
\emph{Verfahrensweisen} implementieren und auf diese Weise zu
\emph{referenzierbaren Diensten} werden.

Ein Beispiel hierfür sind die Trends der Evolution ingenieur-technischer
Systeme \cite{TESE2018} selbst, die mit der Autorität der MATRIZ als
internationaler TRIZ-Organisation in die entsprechenden Zertifikatsprozesse
integriert und so zu einem gewissen Gemeingut werden.  Noch viel deutlicher
wird dies allerdings in Komponentenframeworks der Softwarebranche, wo die
Entwicklung einer Lösung unter Verwendung eines solchen Frameworks bereits die
Modellierung vorstrukturiert. Kooperative Zusammenhänge, die sich in solche
gemeinsamen Abstraktions"|strukturen einbetten, haben es in derartigen weithin
bekannten und akzeptierten Kontexten um ein Vielfaches leichter, sich zu
verständigen.  Normierung und Standardisierung vereinfacht also nicht nur die
Bündelung auf der Ebene technischer Strukturen, sondern auch auf höheren
Abstraktionsebenen menschlicher Gestaltungspraxen.

Normierung und Standardisierung sind das zentrale Mittel, um Vorgehensweisen
als \emph{Stand der Technik} zu befestigen und auf diese Weise
\emph{Fiktionen} gesellschaftlicher Normalität zu generieren, auf die
komplexere technische Prozesse mit höheren Abstraktionen aufsetzen können. Das
TRIZ-Prinzip 7 \emph{Matrjoschka} ist nur eine unterkomplexe Formulierung der
damit verbundenen begriff"|lichen Anforderungen, denn wie in den Proben im
Konzertbeispiel ist das entsprechende Zusammenspiel technischer Prinzipien und
technischer Abstraktionen höherer Ordnung, das mit der Standardisierung
erreicht werden soll, primär eines der Herstellung der Passfähigkeit der
begleitenden Beschreibungsdimensionen, also des Generierens von
\emph{Sprache}, in welcher die neuen Semantiken ausgedrückt werden können. Die
\emph{begriff"|liche} Harmonisierung der Standards als \emph{Semantik} ist der
\emph{technischen} Normierung als \emph{Syntax} vorgängig, aber erst beide
zusammen können in einer \emph{sozio-technischen Verfahrensweise} als
Grundlage für eine neue \emph{Fiktion} verdichtet werden. Erst \emph{nach}
Durchlaufen eines solchen Prozesses können etwa Schraubverbindungen im Sinne
einer verkürzten Sprechweise als \emph{gesellschaftliche Normalität} ihre
Wirkung entfalten, mit der bereits Kleinkinder mit ihren ersten Lego-Baukästen
vertraut gemacht werden.  Damit wird zugleich deutlich, dass dieses
\emph{Schöpfen von Sprache} nicht willkürlich ist, sondern sich an den
\emph{Pragmatiken} (Praxen) gesellschaftlicher Herausforderungen orientiert.

Derartige \emph{Schichtenarchitekturen} spielen allerdings nicht nur auf der
Seite der Beschreibungs- und Verständigungsdimension eine entscheidende Rolle
bei der Reduktion von Komplexität, die zum Erklimmen immer höherer
Abstraktionsebenen technischer Praxen erforderlich ist, sondern auch in der
Vollzugsdimension, was nun am Beispiel des OSI-7-Schichten-Modells näher
erläutert werden soll. Dieses Modell beschreibt das Funktionieren heutiger
Internetverbindungen und hat sich -- seit den ersten ARPA-Net-Versuchen 1974
-- in etwa 40 Jahren \emph{technischer Evolution} zur Grundlage unserer
heutigen \emph{digitalen Praxen} mit ihrer Fiktion der allgemeinen
Ende-zu-Ende-Verbindbarkeit entwickelt\footnote{Dass dies eine \emph{Fiktion}
  ist, weiß nicht nur das Viertel der Weltbevölkerung, das hinter der „Great
  Chinese Firewall“ lebt, sondern wurde auch im „Arabischen Frühling“ 2012 mit
  der Komplettabschaltung des Netzes in Ägypten durch die damaligen Machthaber
  deutlich. Dummerweise traf jene Abschaltung auch die Wirtschaft und wurde
  deshalb nach 10 Tagen zurückgenommen. Seither haben sich die Techniken der
  punktuellen Ausblendung einzelner Teilnehmer und Teilnehmergruppen aus
  dieser Fiktion aber weiter verfeinert wie auch die Techniken des Widerstands
  dagegen.}.

Die unterste Ebene des OSI-Protokollstacks dient der Herstellung der
\emph{Fiktion}, dass Bitfolgen verschickt werden.  Dabei ist der Widerspruch
zwischen der Verschiedenheit der physikalischen Trägermedien und der
Einheitlichkeit der erforderlichen Abstraktion zu lösen. Die API-Spezifikation
ist denkbar einfach -- der Zustandsraum besteht aus Worten über einem Alphabet
aus zwei Buchstaben, die Folge der Buchstaben wird durch einen zeitlichen
Pulsbetrieb (TRIZ-Prinzip 19 der periodischen Wirkung) dargestellt, womit sich
die \emph{Pragmatik} auf die einfache Frage der Darstellung von zwei
verschiedenen Zuständen reduziert, die durch das allgemeine Prinzip des
„Triggern“ umgesetzt wird\footnote{Trigger spielen in vielen technischen
  Prozessen als Schwellwerte, die Events auslösen, eine wichtige Rolle. Ihr
  bewusster Einsatz wirkt sich in nachgeordneten Systemen wie ein (technisch
  geschaffener) \emph{Phasenübergang} aus.  Wie ordnet sich ein solches weit
  verbreitetes Verfahren in die TRIZ-Systematik ein?}, indem über das Setzen
von Schwellwerten analoge Prozesse „digitalisiert“ werden.  In den jeweiligen
Implementierungen des Standards werden damit die verschiedenen Prinzipien der
physikalischen Trägermedien in einer einheitlichen \emph{Semantik}
zusammengefasst und als API-Spezifikation formalisiert, damit die nächste
Ebene des Protokollstacks auf diese \emph{formalisierte Semantik} als
\emph{Syntax} zugreifen kann. Ein wesentliches Charakteristikum ist dabei die
\emph{Taktrate}, also die Geschwindigkeit, in welcher der jeweilige Knoten die
Bitfolgen „produziert“.  Neue Trägermedien (wie etwa WLAN) lassen sich in eine
solche Architektur leicht einbauen, denn sie müssen nur die vorgegebene
Spezifikation \emph{implementieren}.

Die zweite Ebene des Protokollstacks geht von der Fiktion der Bitfolgen aus
und löst ein elementares Problem (Pragmatik) der Übertragung solcher Bitfolgen
zwischen zwei Knoten des Netzes.  Der dabei zu lösende Widerspruch ist der
zwischen differierenden Taktraten der beiden Knoten -- das Problem der
Übertragungsgeschwindigkeit.  Dazu werden die Bitfolgen in \emph{Frames} als
Übertragungseinheiten konstanter Größe unterteilt und deren korrekte
Übertragung einschließlich Fehlerbehandlung auf dieser Ebene sichergestellt.
Die Semantik der sicheren Übertragung führt dazu, dass die Fiktion der
Bitfolgen durch die Fiktion der von Knoten zu Knoten übertragenen Frame-Folgen
als syntaktische Basis für die nächste Ebene des Protokollstacks abgelöst
wird.  Der Begriff der Taktrate spielt nun keine Rolle mehr. 

Die dritte Ebene des Protokollstacks organisiert auf dieser Basis das Routing
und Weiterleiten über größere Entfernungen.  Der zu lösende Widerspruch
besteht darin, dass das Ziel der Übertragung bekannt ist, nicht aber der Weg
dorthin.  Die hierfür geschaffene neue Sprachform sind \emph{Datenpakete} als
weitere Abstraktion, die auf der sicheren Übertragung der Frames von Knoten zu
Knoten aufsetzt und die Paketübertragung über weitere Entfernungen
organisiert.  Während auf der Frame-Ebene die beteiligten Knoten bekannt und
Teil der Input-Spezifikation der Schnittstelle sind, ist nun das Problem
(Pragmatik) der Identifizierung entsprechender Wege zum Zielknoten zu lösen,
wofür entsprechende Routingalgorithmen und Routingprotokolle als
\emph{Sprache} zu vereinbaren waren.  Dabei ist die \emph{Verfügbarkeit der
  Sprache der Frames} eine essentielle Voraussetzung für das Entwickeln der
neuen Sprachformen, denn dabei sind vielfältige Informationen zwischen
benachbarten Knoten auszutauschen und komplexe Informationen lokal zu
aggregieren.  Während die Lösungen der Probleme auf den ersten beiden Ebenen
des Protokollstacks nur \emph{lokale} sozio-technische Abstimmungen erfordern,
ist für diese dritte Ebene eine \emph{globale} Abstimmung erforderlich, denn
die zu entwickelnde Sprache müssen \emph{alle} Knoten im Internet sprechen,
wenn die Paketvermittlung global funktionieren soll. Die weitgehend
anarchistische, aber auch heute noch gut funktionierende Lösung dieser
sozialen Aufgabe durch die Internetpioniere ist bekannt -- die Gründung der
IANA als „technische Weltregierung des Internets“ im Jahre 1988.  Auch ein
Designfehler der damaligen Lösung konnte inzwischen erfolgreich beseitigt
werden -- die (aus damaliger Sicht extrem großzügige) Abschätzung der
erforderlichen Internetadressen mit $2^{32}$, also etwa 4 Milliarden, die dem
Protokoll ipv4 zu Grunde liegt, wurde und wird durch ipv6 mit einem Adressraum
von $2^{128}\approx 3.4\cdot 10^{38}$ ersetzt.

Damit ist zugleich ein Mechanismus der \emph{Höherentwicklung} technischer
Systeme angedeutet, der in den Evolutionsbetrachtungen weder in
\cite{TESE2018} noch in \cite{Shpakovsky2010} eine Rolle spielt.

Wir sehen an diesem Beispiel zugleich, wie die „Herausnahme des Menschen aus
technischen Systemen“ funktioniert, denn das Internet ist ja ein
fortgeschrittenes technisches System, das komplett ohne Zutun des Menschen
funktioniert. Komplett?  Das bezieht sich allein auf die operative Ebene der
Vollzugdimension. Die Herstellung der erforderlichen Ausrüstung, der Betrieb
der Infrastruktur als Service für die vergesellschaftete Menschheit, die
„Aufrechterhaltung der Fiktion“, Reparatur und Planung von Erweiterungen sind
ohne aktiv tätige Menschen nicht denkbar. Diese zerfallen allerdings, bezogen
auf eine konkrete Technologie, in zwei Gruppen -- die große Gruppe der
„Normalos“ und die kleine Gruppe der „Experten“. Erstere können mit der
Beschreibungsform auf der Eebene der Fiktion leben, um mit begründeten
Erwartungen verantwortungsbeladen am gesellschaftlichen Leben teilzunehmen.
Die kleine Gruppe der Experten in jener Technologie kennt sich auch mit der
Implementierung aus. In ihrem \emph{privaten Verfahrenskönnen} spiegelt sich
die Arbeitsteiligkeit der modernen Industriegesellschaft.  Allerdings ist
dieser Faden dünn und kann reißen, besonders unter sozio-ökonomisch prekären
Bedingungen, wie wir von Tschernobyl wie auch Fukushima wissen. In meiner
These~1 kommt diese Fragilität moderner Technik-Semiotik zum Ausdruck. Wichtig
ist nicht so sehr die Beschreibungsdimension, als vielmehr deren Verankerung
als \emph{privates Verfahrenskönnen} konkreter Menschen in der
Vollzugsdimension.
\begin{quote}\it
  \textbf{These 7:} Jeder „Experte“ ist „Normalo“ in Bezug auf fast alle
  Technologien. 
\end{quote}
Mit dem bisher entwickelten Ansatz bewegen wir uns im Kontext des
Technikbegriffs, der in meiner Vorlesung genauer entwickelt wird und drei
Dimensionen umfasst -- das \emph{gesellschaftlich verfügbare Verfahrenswissen}
(Stand der Wissenschaft), \emph{institutionalisierte Verfahrensweisen} (die
technischen Prinzipien, die in \cite{Shpakovsky2010} als
Strukturierungsgrundlage für Evolutionslinien verwendet werden) und
\emph{privates Verfahrenskönnen} (mindestens einer entsprechend ausgebildeten
Schicht von Ingenieuren und Technikern als „Träger von Vernunft“ im Sinne von
\cite{Vernadsky2001}).

\section{Komponenten und Objekte}

Mit der funktionalen Orientierung auf Zwecke, Implementierungen und Dienste
bleibt das \emph{Target}, das \emph{Objekt} jener funktionalen Einwirkungen
eigentümlich im Dunkeln. In der TRIZ-Terminologie ist vom \emph{bearbeiteten
  Objekt} die Rede, das durch entsprechende funktionale Transformationen durch
ein \emph{Werkzeug} zu einem \emph{nützlichen Produkt} wird. Das Werkzeug,
insoweit es selbst hergestellt werden muss, ist seinerseits \emph{Objekt}, oft
zur Verdeutlichung auch \emph{technisches Objekt}, aber in der
Einsatzperspektive \emph{Werkzeug $\to$ bearbeitet $\to$ Objekt} ist es nur
noch Träger von Funktionalität. Wichtig ist allein \emph{bearbeitet $\to$
  Objekt}, die spezifikationskonforme Erfüllung einer Funktion, weniger das
genaue Instrument. \cite{Shpakovsky2003} fragt sogar, wann eine Sammlung
technischer Objekte zu einem technischen System wird. Bei den dabei
betrachteten technischen Objekten kann es sich aber nur um Werzkeuge oder
Komponenten handeln, denn anderes ist nicht zu einem technischen System
aggregierbar. Kurz, in der TRIZ-Literatur werden die Begriffe Komponenten und
Objekt in einer weitgehend unscharfen Synonymität verwendet.

Szyperskis Buch \cite{Szyperski2002} dagegen fundiert auf einer klaren
Unterscheidung zwischen diesen beiden Begriffen. Eine \emph{Komponente} wird
durch drei Eigenschaften charakterisiert:
\begin{itemize}
\item[(1)] als Einheit der unabhängigen Verteilung,
\item[(2)] als Einheit der Komposition durch Dritte und
\item[(3)] als etwas, das keinen (extern beobachtbaren) Zustand hat.
\end{itemize}
Besonders die dritte Charakterisierung ist für unsere Zwecke interessant, denn
in diesem Verständnis ist eine Komponente ein rein funktionales Konstrukt ohne
„Gedächtnis“, ein über eine Spezifikation definiertes Bündel von Funktionen,
das -- so (2) -- komplett entkontextualisiert und von Dritten
rekontextualisiert werden kann.  Dass ein solcher Transformationsprozess in
übergeordneten sozio-technischen Beziehungen zwischen Anbieter und Nutzer
selbst wieder kontextualisiert ist, hatten wir weiter oben schon
herausgearbeitet.  In diesem Sinne ist -- nicht unerwartet -- Szyperskis
Komponentenbegriff also selbst reduktionistisch.  In den Charakteristika (1)
und (2) wird aber auch in diesem Ansatz deutlich, dass der sozio-technische
Charakter nicht von einem technischen abgetrennt werden kann.

Ein \emph{Objekt} wird in \cite{Szyperski2002} durch folgende drei
Charakteristika gekennzeichnet:
\begin{itemize}
\item [1.] als Einheit der Instanziierung, die eine eindeutige Identität hat, 
\item [2.] als etwas mit extern beobachtbaren Zustand,
\item [3.] das Zustand und Verhalten kapselt.
\end{itemize}
Während es in einem solchen Konzept keinen Sinn hat, mehr als eine Komponente
einer Art im System zu haben (da sie zustandslos sind, könnten wir sie nicht
einmal unterscheiden), ist in den Objekten die gesamte systemische
Individualität gebündelt.

Als klassisches Beispiel wird eine Datenbank angeführt -- der Datenbankserver,
der die komplexe Transaktionsfunktionalität zur Verfügung stellt, ist die
Komponente, die einzelnen Datenbanken die Objekte.  Dabei wird auch die
sozio-ökonomische Stellung des „unabhängigen Dritten“ deutlich. Der Anbieter
der Komponente, der Datenbankhersteller, liefert die Komponente, die
Datenbanksoftware, aus, die als „Einheit der Verteilung“ auf einem Rechner des
Kunden installiert wird. Dieser „unabhängige Dritte“ baut diese Komponente in
seine eigene Middleware-Architektur als „Einheit der Komposition“ ein.  Ihre
Wirkung entfaltet die Komponente aber erst durch die Bearbeitung von Objekten,
der Datenbanken und Datensätze, die der Kunde (!) gestaltet, zur Verfügung
stellt und verantwortet.

Die klare Unterscheidung der Begriffe Komponente und Objakt, die Szyperski
hier vornimmt, zeichnet damit die realweltliche Trennung der
Verantwortungsbereiche von Anbieter und Kunde in einer servicebasierten
Wirtschaft nach: der Anbieter liefert ein \emph{Bündel von Funktionen}, das im
Verantwortungsbereich des Kunden \emph{auf den Objekten des Kunden} seine
Funktionalität entfaltet. Wenn etwas schief geht, dann kann es am Anbieter
gelegen haben -- es lag eine Fehlfunktion vor --, es kann aber auch am Kunden
gelegen haben -- die bearbeiteten Objekte waren fehlerhaft. Diese Fragen
werden heute üblicherweise in Service Level Agreements geregelt, erläutern
aber den Untertitel „Beyond Object-Oriented Programming“ von
\cite{Szyperski2002}.  Die Zusammenführung von Funktion und Attribut in
lokalen Objekten als Kern des OO-Paradigmas wird (wieder) aufgelöst und dem
Objekt eine passive Rolle zugeschrieben, die von Zustand als
Attribut\emph{wert} und Verhalten als \emph{Reaktion} auf funktionale
Einwirkung determiniert wird.

In einem solchen Verständnis sind Objekte Träger des Gedächtnisses in der Welt
der technischen Systeme. Sie sind es, die sich „merken“, was mit ihnen
funktional unternommen wurde.  Während in der TRIZ im Grundzusammenhang
\emph{Werkzeug $\to$ bearbeitet $\to$ Objekt} die einzelne Funktion im
Vordergrund steht, steht das von vielen Komponenten bearbeitete Objekt in
einer Vermittlerrolle zwischen den Komponenten. Wenn wir oben herausgearbeitet
haben, dass die Welt  der \emph{Beziehungen} zwischen Komponenten wichtiger
ist als die Welt der technischen Systeme selbst, so haben wir mit den Objekten
nun auch die \emph{Vermittler} dieser Beziehungen identifiziert.

Auch dabei ist allerdings zu beachten, dass diese Vermittlung selten durch ein
einziges Objekt erfolgt, sondern in den meisten Fällen durch ein ganzes Set
von Objekten. Dies sei  am Beispiel des CORBA-Lizenzdiensts
\cite[S. 242]{Szyperski2002} erläutert:
\begin{itemize}
\item Soll ein Objekt einen Dienst (im Weiteren das \emph{Dienstnutzerobjekt})
  nutzen, so kann es sein, dass es hierfür eine Lizenz benötigt.
\item Dafür muss der Dienstanbieter eine solche Lizenz ausstellen, ein zweites
  Objekt, das \emph{Lizenddienstobjekt}.
\item Diese Lizenz ist eine auf den Lizenznehmer (das Dienstnutzerobjekt)
  zugeschnittene Spezialisierung eines allgemeinen Lizenzverfahrens, verfügbar
  als drittes Objekt, als \emph{Lizenzverfahrensobjekt}.  Dieses Objekt wurde
  bei der Registrierung der Lizenz des Dienstanbieters angelegt. Der Nutzung
  der Lizenz in der Vollzugsdimension geht das Anlegen der Lizenz in der
  Beschreibungsdimension voraus.
\item Die Lizenzdienst-Factory, ein zweiter Dienst, ist nun in der Lage, auf
  die Anfrage des potenziellen Dienstnutzers als viertes Objekt ein
  \emph{Lizenzobjekt} zu generieren, das seinerseits den Prozess der Nutzung
  der Lizenz durch das Dienstnutzerobjekt überwacht:
  \begin{itemize}
  \item Es nimmt das Lizenzverfahrensobjekt und das Dienstnutzerobjekt und
    erzeugt daraus das Lizenzdienstobjekt.
  \item Es versetzt den Dienst in den für die Nutzung durch das
    Dienstnutzerobjekt erforderlichen Zustand (etwa Demomodus, Abrechnung),
    was mit dem Anlegen weiterer Objekte verbunden ist, denn nur Objekte
    können Zustände speichern. 
  \item Es erlaubt dem Dienstnutzerobjekt, den Dienst zu nutzen und überwacht
    diese Nutzung (speichert etwa die Nutzungszeit, um diese später an den
    Abrechnungsdienst zu übergeben).
  \item Es wartet, bis die Dienstnutzung abgeschlossen ist (dies kann aktiv
    durch das Dienstnutzerobjekt getriggert werden, aber auch -- etwa durch
    Zeitablauf -- vom Lizenzobjekt) und löst die Gesamtstruktur am Ende auf,
    ohne zu vergessen, die Abrechnungsobjekte an den Abrechnungsdienst
    weiterzugeben.  
  \end{itemize}
\end{itemize}

\section{Zusammenfassung}

Mit dem Begriff des \emph{technischen Systems} dreht sich der ganze
TRIZ-Theoriekorpus um einen in der TRIZ-Literatur wenig präzisierten Begriff,
der als weitgehend aus der Anschauung verständlich postuliert wird. Mit den 40
TRIZ-Prinzipien, den 76 TRIZ-Standards und den (in \cite{TESE2018}) 10
TRIZ-Evolutionstrends wird dabei ein Universum theoretischer Reflexion
praktischer Erfahrung mit tendenziell universalistischem Anspruch aufgespannt.
Die Überverallgemeine"|rung eigener praktischer Erfahrungen und die damit
verbundene Dekontextualisierung wird in der Gemeinde der TRIZ-Praktiker
allerdings kaum als Problem wahrgenommen -- die in einer \emph{TRIZ-Theorie
  des Lebens schöpferischer Persönlichkeiten} explizierten anarchistischen
Grundmuster immunisieren offensichtlich gegen die ideologischen Wirkungen
derartiger Überverallgemeinerungen.

In diesem Aufsatz wurde der Versuch unternommen, die Tragfähigkeit des
Begriffs \emph{technisches System} für diesen Theoriekontext genauer
auszuleuchten und diesen auch mit Ansätzen aus benachbarten Theoriegebäuden zu
relatieren.  Es stellt sich heraus, dass eine Konzentration auf die
artefaktische Dimension von Technik, wie sie dem Begriff  \emph{technisches
  System} inhärent ist, den Blick auf wesentliche relationale Phänomene in
einer \emph{Welt der technischen Systeme} verstellt und der in
\cite{Shpakovsky2010} verwendete Begriff des \emph{technischen Prinzips} für
die Analyse relationaler Phänomene besser geeignet ist. Denn das Ganze ist
\emph{mehr} als die Summe seiner Teile.

\begin{thebibliography}{xxx}
\bibitem{Barkleit2000} Gerhard Barkleit (2000). Mikroelektronik in der
  DDR. Dresden, 2000.
\bibitem{Bertalanffy1950} Ludwig von Bertalanffy (1950). An outline of General
  System Theory. The British Journal for the Philosophy of Science, vol. I.2,
  134–165.
\bibitem{MEW15} Friedrich Engels (MEW 15). Die Geschichte des gezogenen
  Gewehrs.  MEW 15, S. 195--226. Dietz Verlag, Berlin.
\bibitem{Friedli2013} Thomas Friedli, Stefan Thomas, Andreas Mundt (2013).
  Management globaler Produktionsnetzwerke. Strategie – Konfiguration –
  Koordination. Hanser, München. ISBN 978-3-446-43449-3
\bibitem{Gamma1995} Erich Gamma, Richard Helm, Ralph Johnson, John Vlissides
  (1995). Design Patterns: Elements of Reusable Object-Oriented Software.
  Addison-Wesley. ISBN 978-0-201-63361-0.
\bibitem{KFK2000} Klaus Fuchs-Kittowski (2000).  Wissens-Ko-Produktion.
  Verarbeitung, Verteilung und Entstehung von Informationen in
  kreativ-lernenden Organisationen.\\ In: Fuchs-Kittowski u.a.
  (Hrsg.). Organisationsinformatik und Digitale Bibliothek in der
  Wissenschaft. Wissenschaftsforschung, Jahrbuch 2000. Gesellschaft für
  Wissenschaftsforschung, Berlin.
  \url{http://www.wissenschaftsforschung.de/JB00_9-88.pdf}
\bibitem{Galanakis2006} Kostas Galanakis (2006).  Innovation process. Make
  sense using systems thinking.  In: Technovation Volume 26, Issue 11,
  p. 1222--1232.
\bibitem{Gerovitch1996} Slava Gerovitch (1996). Perestroika of the History of
  Technology and Science in the USSR: Changes in the Discourse. Technology and
  Culture, Vol. 37.1, S. 102--134.
\bibitem{Goldberg2016} Jörg Goldberg, André Leisewitz (2016). Umbruch der
  globalen Konzernstrukturen.\\ Z 108, S. 8--19.
\bibitem{Graebe2018} Hans-Gert Gräbe (2018).  12. Interdisziplinäres Gespräch
  \emph{Nachhaltigkeit und technische Ökosysteme}. Leipzig, 02.02.2018. 
  \url{http://mint-leipzig.de/2018-02-02.html}.
\bibitem{Graebe2019a} Hans-Gert Gräbe (2019a).
  \foreignlanguage{russian}{Наследие Движения Школ Изобретателeй в ГДР и
    Развитиe ТРИЗ} (Das Erbe der Erfinderschulbewegung in der DDR und die
  Entwicklung der TRIZ). Erschienen im Online-Protokollband des TRIZ Summit
  2019 Minsk.
\bibitem{Graebe2019b} Hans-Gert Gräbe (2019b).  Bericht zu einer Diskussion
  über TRIZ und Systemdenken in meinem Open Discovery Blog.
  \url{https://wumm-project.github.io/2019-08-07}.
\bibitem{Graebe2020} Hans-Gert Gräbe (2020). Reader zum 16. Interdisziplinären
  Gespräch \emph{Das Konzept Resilienz als emergente Eigenschaft in offenen
    Systemen} am 7.2.2020 in Leipzig.
  \url{http://mint-leipzig.de/2020-02-07/Reader.pdf}.
\bibitem{Holling2000} Crawford S. Holling (2000). Understanding the Complexity
  of Economic, Ecological, and Social Systems. In: Ecosystems (2001) 4,
  390–405.
\bibitem{KS2017} Karl Koltze, Valeri Souchkov (2017).  Systematische
  Innovation.\\ Hanser, München. Zweite Auf"|lage. ISBN 978-3-446-45127-8.
\bibitem{Kropik2009} Markus Kropik (2009). Produktionsleitsysteme in der
    Automobilfertigung. Springer, Dordrecht.\\ ISBN 978-3-540-88991-5.
\bibitem{TBK-2007} S. Litvin, V. Petrov, M. Rubin (2007). TRIZ Body of
  Knowledge. \\ \url{https://triz-summit.ru/en/203941}.
\bibitem{TESE2018} Alexander Lyubomirskiy, Simon Litvin, Sergey Ikovenko,
  Christian M. Thurnes, Robert Adunka (2018). Trends of Engineering System
  Evolution. Sulzbach-Rosenberg.  ISBN 978-3-00-059846-3.
\bibitem{MEW3} Karl Marx (MEW 3).  Thesen über Feuerbach. MEW 3, S. 533--535.
  Dietz Verlag, Berlin.
\bibitem{MEW23} Karl Marx (MEW 23). Das Kapital, Band 1. MEW 23. Dietz Verlag,
  Berlin.
\bibitem{MEW42} Karl Marx (MEW 42). Grundrisse der Kritik der politischen
  Ökonomie.  MEW 42. Dietz Verlag, Berlin.
\bibitem{Mittelstrass2011} Jürgen Mittelstraß (2011).  Schöne neue
  Leonardo-Welt.  Frankfurter Allgemeine Zeitung, 25. Juli 2011, S. 7.
\bibitem{Naetar2005} Franz Naetar (2005). „Commodification“, Wertgesetz und
  immaterielle Arbeit. Grundrisse 14, S. 6--19.
\bibitem{Pohl2005} Klaus Pohl, Günter Böckle, Frank J. van der Linden (2005).
  Software Product Line Engineering. Foundations, Principles and Techniques.
  Springer. ISBN 978-3-540-28901-2
\bibitem{Preez2006} Niek D Du Preez, Louis Louw, Heinz Essmann (2006). An
  innovation process model for improving innovation capability.  Journal of
  high technology management research, vol 17, 1--24.
\bibitem{Rubin2007} Michail S. Rubin (2007). \foreignlanguage{russian}{О
  выборе задач в социально-технических системах} (Über die Wahl von Aufgaben
  in sozial-technischen Systemen). In: \foreignlanguage{russian}{ТРИЗ Анализ.
    Методы исследования проблемных ситуаций и выявления инновационных задач}.
  (TRIZ-Analyse. Methoden zur Untersuchung von Problemsituationen und zur
  Identifizierung innovativer Aufgaben). Hrsg. von S.S. Litvin, V.M. Petrov,
  M.S. Rubin. \foreignlanguage{russian}{Библиотека Саммита Разработчиков
    ТРИЗ}, Moskau. S. 35--46.
  \url{https://www.trizland.ru/trizba/pdf-books/TRIZ-summit2007.pdf}.
\bibitem{Rubin2010} Michail S. Rubin (2010).
  \foreignlanguage{russian}{Филогенез социокультурных систем. Секреты развития
    цивилизаций}.  (Phylogenese soziokultureller Systeme. Geheimnisse der
  Zivilisationsentwicklung).
  \url{http://www.temm.ru/en/section.php?docId=4472}.
\bibitem{Rubin2019} Michail S. Rubin (2019).  \foreignlanguage{russian}{О
  связи комплекса законов развития систем с ЗРТС} (Zur Verbindung des
  Komplexes der Gesetze der Systementwicklung mit den Gesetzen der Entwicklung
  technischer Systeme). Manuskript, November 2019.
\bibitem{Shpakovsky2003} Nikolay Shpakovsky (2003).
  \foreignlanguage{russian}{Человек и Техническая Система} (Der Mensch und das
  technische System).
  \url{https://wumm-project.github.io/Texts/Shpakovsky-mts-ru.pdf}
\bibitem{Shpakovsky2010} Nikolay Shpakovsky (2010).  Tree of Technology
  Еvolution. Forum, Moscow.
\bibitem{Shub2006} Leonid Shub (2006). \foreignlanguage{russian}{Осторожно!
  Таблица технических противоречий}. (Vorsicht! Die Widerspruchstabelle).
  \url{http://metodolog.ru/conference.html}. Siehe auch ders. Vorsicht
  Widerspruchsmatrix, Kurzfassung in Deutsch.
  \url{https://wumm-project.github.io/Texts/Shub-2006.pdf}.
\bibitem{Szyperski2002} Clemens Szyperski (2002). Component Software: Beyond
  Object-Oriented Programming. ISBN: 978-0-321-75302-1.
\bibitem{TT} Target Invention (2020). TRIZ Trainer.
  \url{https://triztrainer.ru}.
\bibitem{Thiel2007} Rainer Thiel (2007). Zur Lehrbarkeit dialektischen Denkens
  – Chance der Philosophie, Mathematik und Kybernetik helfen. In: Klaus
  Fuchs-Kittowski, Rainer E. Zimmermann (Hrsg.). Kybernetik, evolutionäre
  Systemtheorie und Dialektik. Trafo Verlag, Berlin 2012, ISBN:
  978-3-89626-919-5, S. 185--202
\bibitem{VDMA2019} VDMA. Maschinenbau in Zahl und Bild 2019. 
\bibitem{Vernadsky2001} Vladimir I. Vernadsky (1997, Original 1936--38).
  Scientific Thought as a Planetary Phenomenon.
  \url{https://wumm-project.github.io/Texts.html}
\bibitem{Weller2008} Wolfgang Weller (2008). Automatisierungstechnik im
  Überblick. Was ist, was kann Automatisierungstechnik? Beuth, Berlin. ISBN
  978-3-410-16760-0.
\bibitem{Zobel2016} Dietmar Zobel, Rainer Hartmann (2016). Erfindungsmuster.
  2. Auf"|lage.  Expert Verlag, Renningen.
\bibitem{Zobel2020} Dietmar Zobel (2020). Beiträge zur Weiterentwicklung der
  TRIZ.\\  LIFIS Online 19.01.2020. DOI: \url{10.14625/zobel_20200119}
\end{thebibliography}
\end{document}

