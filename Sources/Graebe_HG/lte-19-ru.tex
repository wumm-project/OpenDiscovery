\documentclass[11pt,a4paper]{article}
\usepackage{a4wide,url}
\usepackage[russian]{babel}
\usepackage[utf8]{inputenc}

\parindent0cm
\parskip3pt

\title{О Развитии Технических Систем} 
\author{Hans-Gert Gräbe, Leipzig}
\date{Версия от 11 декабря 2019 г.}
\begin{document}
\maketitle

\section{Предварительные замечания}

Поводом этого маленького наброска на тему «Законы и тенденции развития
технических систем», которые обсуждаются в разных вариантах в контексте ТРИЗ,
была версия этих законов и тенденций в заданиях Кубка
ТРИЗ\footnote{\raggedright Задачи Кубка ТРИЗ
  \url{https://triz-summit.ru/contest/cup-tds-2019-2020/contest-2019-2020/},
  см. дополнительно \url{https://wumm-project.github.io/Upload/lte.pdf}. }
2019/2020 года.

В этих заданиях числится также «закон вытеснения человека» (первая версия),
который в более поздней версии стал «тенденцией вытеснения человека из ТС».

Такой тенденции нет ни в списке Альтшуллера из восьми законов\footnote{Как они
  перечислятся например в английской википедии
  \url{https://en.wikipedia.org/wiki/TRIZ}} ни в списке пяти законов и десяти
тенденций в (Кольце/Сучков)\footnote{Karl Koltze, Valeri Souchkov.
  Systematische Innovation. ISBN 978-3-446-45127-8.}. Найдется такая тенденция
однако в выданнной в 2018 году и МАТРИЗом уполномоченной версии \emph{Тенденций
  развития инженерных систем} (TESE)\footnote{A. Lyubomirskiy, S. Litvin,
  S. Ikovenko, C.M. Thurnes, R. Adunka. Trends of Engineering System
  Evolution. Sulzbach-Rosenberg 2018.  ISBN 978-3-00-059846-3.}.

Конечно, это поднимает вопрос о контекстуальных предположениях, которые ведут
к таким разным позициям. В марксистской литературе такой процесс вытеснения
также рассматривается. В «Машинном Фрагменте» (MEW 42, стр. 570 и
след.)\footnote{Здесь и в следующих ссылках цитаты по немецкой версии
  произведений Маркса/Энгельса, которая широко используется в немецкоязычных
  странах.} -- ранний набросок его экономической теории -- Маркс развивает
видение общества, в котором «социальный метаболизм» (MEW 23, стр. 37)
организован таким образом, что
\begin{quote}
  это уже не рабочий, который ставит модифицированный природный продукт между
  объект и себя; но естественный процесс, который он превращает в
  индустриальный, он ставит как средство между себя и неорганическую природу,
  которую он осваивает. (MEW 42, стр. 592)
\end{quote}
Маркс далее обсуждает, что развитие производительных сил \emph{необходимо}
движется к такому образу организации социального метаболизма.
\begin{quote}
  Поглощенный в процесс производства капитала, рабочий инструмент проходит
  через различные метаморфозы, чья последняя это \emph{машина} или, скорее,
  \emph{автоматическая система машин} (система машин, \emph{автоматическое}
  является только наиболее полной и адекватной форме того же, и преобразует
  машины в систему), приводимая в движение автоматом, движущей силой, который
  движется сам по себе; этот автомат состоит из многочисленных механических и
  интеллектуальных органов, так что сами рабочие определяются только как
  сознательные участники этого же автомата. (MEW 42, стр. 584)
\end{quote}
Эта мысль во многом сингулярна в работах Маркса и нигде более подробно не
разработана\footnote{Так утверждается по крайней мере в статье Jörg Goldberg,
  André Leisewitz: Umbruch der globalen Konzernstrukturen. Z 108 (Dezember
  2016), S. 8--19.}.

Проблемы таких «приводимых в движение автоматов» видны сейчас в экологическом
кризисе планетарного масштаба, поэтому вопрос является уместным, не является
ли предполагаемая «тенденция вытеснения человека из технической чистемы»
фундаментальной теоретической ошибки в предложенном комплексе законов и
направлений развития технических систем.

Такая тенденция вытеснения также противоречит основным положениям немецкой
кибернетической школы, как я писал в записке от 8.11.2019 авторам задач кубка
ТРИЗ.
\begin{quote}
  «Закон вытеснения человека» отсутствует в списке (Кольце/Сучков) и я
  совершенно не согласен, что это закон развития технических систем. По
  крайней мере в немецкой литературе такие воросы обсуждаются с 80-х годов.
  Например, Клаус Фукс-Киттовский подчеркивает в сводке его
  работ\footnote{\url{http://www.informatik.uni-leipzig.de/~graebe/Texte/Fuchs-02.pdf}}
  \begin{quote}
    Наш ответ на этот вопрос всегда был: человек единственная творческая
    производительная сила, он должен быть и оставаться субъектом
    развития. Следовательно концепция полной автоматизации, согласно которой
    человек должен постепенно элиминироваться из процесса, пропущена!
  \end{quote}  
  Замена человека как закон технического развития коренится в очень странном
  понимании термина \emph{техника}, которое забывает очевидное -- нет
  \emph{технических систем}, но только \emph{техносоциальные системы}.
\end{quote}

Михаил Рубин разъяснил свою позицию в личном письме от 10.11.2019 следующим
образом:
\begin{quote}
  Для этого требуется отдельная дискуссия. Мы ссылаемся на работу Любомирского
  и Литвина, в которой говорится о вытеснении человека из технической системы.
  Мы согласны с тем, что это явление не закон, а тенденция, которая происходит
  в рамках другого закона: повышения уровня автономности систем. Обновленная
  система законов и тенденций добавлена в файл, который приложен к этому
  письму.  Вы абсолютно правы в том, что технические системы не являются
  самостоятельными в своем развитии и более общими являются
  социально-технические системы. Законы развития социально-технических систем
  отличаются от законов развития технических систем. Для чисто технических
  систем действительно можно наблюдать тенденцию постепенного исключения
  участия человека. Весто весельной лодки появляется лодка с мотором. Вся
  промышленная революция XVII века была построена на вытеснении человека
  двигателями и машинами. Следующая технологическая революция также была
  связана с вытеснением человека из области управления за счет автоматизации и
  компьютеров. Это совсем не означает, что из техники, как
  социально-технической системы вытесняется человек. Наоборот, человек
  остается главным источником требований для технических систем. Но выполнение
  этих требований все в большей степени происходит без участия человека. Эта
  тенденция характерна и для кинематографа, как технической системы. Понятно,
  что ни из процесса создания кинопроизведений, как произведений искусства, ни
  из процесса потребления продуктов кинематографа человек не вытесняется, он
  остается центром всех этих процессов. 
\end{quote}
В этом контексте у меня возникает ряд вопросов, на которые в первом обсуждении
на Facebook\footnote{\url{https://www.facebook.com/groups/111602085556371}} я
не получил удовлетворительного ответа.
\begin{enumerate}
\item Что такое \emph{техническая система} в отличие от
  \emph{социально-технической системы}?
\item Как понять концепцию \emph{развитие технических систем}?  Стоит ли
  рассматривать развитие отдельных технических систем или рассмотрение их
  развития осмысленным способом требует их рассмотрение только в совокупности
  или даже в более общих социальных структурах?
\item В каком отношении стоит \emph{человек} к отдельным техническим системам
  и к совокупности его технических творений? В какой степени при обсуждении
  этого вопроса надо различать \emph{человек как родовое существо} (имеющее
  процессуальное знание), отдельные люди как действующие \emph{актеры в
    отношениях между целями и средствами} (носители процессуального умения) и
  кооперативные акторы как \emph{операторы отдельных технических систем}
  (институционализированные процессуальные процедуры)?
\end{enumerate}

Такие вопросы возникают, в частности, при изучении социально-экологических
систем, в которые действие технических систем очевидно встроено. Такие воросы
обсуждаются например в текстах Элинор Остром\footnote{Anderies, John M., Marco
  A. Janssen, Elinor Ostrom (2004).  Framework to Analyze the Robustness of
  Social-ecological Systems from an Institutional Perspective. In: Ecology and
  Society 9 (1), 18. -- Ostrom, Elinor (2007). A diagnostic approach for going
  beyond panaceas.  Proceedings of the national Academy of sciences, 104(39),
  15181--15187.} и в нашем Лейпцигском
семинаре\footnote{\url{https://github.com/wumm-project/Leipzig-Seminar}.}.

\section{Что такое технические системы?}

\subsection{Некоторые предварительные соображения}

Подавляющее большинство технических систем созданных человеком уникальные.
Отрасль, которая занимается производством таких уникальных предметов
называется \emph{строительство крупного промышленного оборудования}. Также
большинство компьютерных специалистов имеет дело с созданием таких уникальных
предметов, потому что ИТ-системы, управляющие такими системами, также
уникальны. То же самое относится и к офисам, правительству и общественным
институтам. Например Лейпцигская городская администрация в настоящее время
занята трансформацией своих административных процессов на «компютерный лад».
Этот процесс возглавляет Департамент общего управления и осуществляется
совместно с муниципальным поставщиком ИТ-услуг Lecos.

Конечно, велосипед не постоянно переизобретается -- компонентные технологии
составляют основу каждой инженерно-технической работы, а также информатика
после кризиса программного обеспечения, который длился более 25
лет\footnote{Patrick Hamilton (2008). Wege aus der Softwarekrise:
  Verbesserungen bei der Softwareentwicklung.  ISBN 978-3-540-72869-6.}
перешел к методологии разработки на основе компонентов\footnote{Clemens
  Szyperski (2002). Component Software: Beyond Object-Oriented
  Programming. ISBN: 978-0-321-75302-1.}.  В этом контексте стали также
раздифференцироваться профессиональные профилы компьютерных специалистов в
разработчики компонентов («дизайн для компонентов») и сборщики компонентов
(«дизайн из компонент»). Первые разрабатывают компоненты для большого рынка,
вторые продолжают разрабатывать большие уникальные устройства («системы» также в
терминологии компьютерных специалистов), но уже используя те компоненты.

Мы ясно находимся в области стандартной терминологии ТРИЗ \emph{системы
  систем} -- техническая система состоит из компонентов, которые, в свою
очередь, являются техническими системами, которые должны бесперебойно
\emph{работать} (как в функциональном так и в оперативном смысле) в
рассматриваемой на настоящем уровне системе. Понятие технической системы имеет
таким образом ясную эпистемическую функцию «сокращения к существенному».
Эйнштейну приписывают высказывание «сделай просто как можно, но не проще».
\emph{Закон полноты системы} точно выражает эту идею, но действует она здесь
не как \emph{закон}, а как \emph{директива моделирования}.

Термин \emph{техническая система} в таком контексте планово-реального мира
четыре раза перегружен
\begin{itemize}
\item [1.] как уникат в реалном мир,
\item [2.] как описание этого униката,
\end{itemize}
а также для компонентов, производимых в больших количествах
\begin{itemize}
\item [3.] как описание дизайна шаблона системы
\item [4.] как описание и работа в реальном мире доставочной и
  эксплуатационной структуры, которая готовит и обеспечвает надежную работу
  сделанных по этому шаблону уникатов в реальном мире.
\end{itemize}

Особенно последний момент, отношения между компонентом как концепция и
реальними экземплярами компоненты является сложным, потому что
производственные структуры производства и использования этих экземпляров
обычно распадаются, экземпляры после производства отправляются на их место
назначения работы, где их надо готовить к их конкретному использованию и потом
вмонтировать. В теории \emph{Component Software} при такой подготовке различают
три этапа \emph{deploy, install, configure}.

\subsection{Комментарий Н. Шпаковского, 8.12.2019}

Законы и линии развития активно применяются при решении ситуационных и
прогнозных задач. Там есть про систему, но очень мало, и понимал я так очень
давно.
 
Я часто думаю о понятии «техническая система» в последнее время. Это понятие —
важная часть процесса решения задачи по нашему подходу. Не нахожу ничего
неправильного в подходе German VDI\footnote{ Я написал об этом в Facebook:
  Центральный вопрос для меня -- что такое техническая система. В большинстве,
  как и в контексте ТРИЗ, этот термин используется без всякого сомнения в
  множественном числе. Оправдано ли это? VDI -- Verein Deutscher Ingenieure --
  профессиональная организация немецких инженеров, определяет в своей
  директиве VDI 3780 понятие \emph{техника}. Авторы директивы нерешительно в
  этом вопросе и пишут о «множестве систем». Техника берется в трех измерениях
  \begin{itemize}
  \item множество ориетированных на пользу, искусственных, предметных
    образований (Артефакты или вещественные системы);
  \item множество человеческих действий и учреждений, в которых возникают
    вещественные системы и
  \item множество человеческих действий и учреждений, в которых используют
    вещественные системы.
  \end{itemize}}, всё правильно, всё на свете можно представить как системы. Любая
система может быть представлена как «система систем», просто мы выбираем один
какой-то уровень и говорим — это система.  Тогда сразу появляется возможность
сказать, что есть надсистемы, что есть подсистемы.

У тебя есть конкретный вопрос — какая разница между понятиями «система —
подсистемы» и «система — компоненты». Всё просто — компонент есть еще не
систематизированная часть системы, потенциальная подсистема.

\begin{quote}
  Примечание HGG: Но это противоречит пониманию компонентной технологии,
  согласно которой компоненты на этапе строительства системы уже должны
  существовать.
\end{quote}

Что до понятия «техническая система», то оно страшно запутано в ТРИЗ.
Технической системой называют совокупность каких-то механизмов, дающую новое
качество, например, автомобиль, ручка, часы. И технической системой называют
систему для выполнения какой-то функции, например, перевозки груза, куда кроме
автомобиля входит еще много чего. Это полбеды, проблема в том, что эти
определения смело смешивают, от чего получается страшная путаница.  Включать
или не включать водителя в состав автомобиля? А что делать с бензином?  А
воздух — часть автомобиля или нет?  Люди на этой путанице хорошо живут, строят
целые теории и проводят семинары, усиливая эту путаницу.
 
Для себя я разделяю 
\begin{enumerate}
\item техническую систему (систематизированный технический объект, машина на
  складе),
\item функционирующую систему (то, что в патенте называют «машина в работе»),
\item полезную техническую систему (то, что дает полезный продукт).
\end{enumerate}

Конечно, слово «техническая» тут многое запутывает, но при таком раскладе
лучше понимать так, что техническая — это система, имеющая отношение к технике
(инженерии) или к технике выполнения какого-то полезного действия. А лучше
выкинуть это слово совсем.  Самой важной, самой полезной для решения задачи
является полезная система. На таком уровне слово «техническая» теряет смысл,
поскольку это может быть и электрик, вкрутивший лампочку, и вышедший на орбиту
космический корабль, и адвокат, и компьютерная программа. Основной критерий —
дает ли это полезный результат или это «нелетающий самолет Можайского».

\end{document}
