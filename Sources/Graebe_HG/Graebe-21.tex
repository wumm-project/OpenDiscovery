\documentclass[12pt,a4paper]{article}
\usepackage{od}
\usepackage[utf8]{inputenc}
\usepackage[english]{babel}
\setlist{noitemsep}

\title{Towards a Linked Open Data TRIZ Ontology} 
\author{Hans-Gert Gräbe, Leipzig}
\def\theauthor{Hans-Gert Gräbe}
%\date{13.\,08.\,2021}
\begin{document}
\maketitle
\begin{abstract}
  We discuss some basic aspects of TRIZ ontology modelling that have emerged
  from our practical work on semantic concepts and with semantic tools in the
  context of the TRIZ Ontology Project. Particular emphasis is placed on the
  distinction between a conceptual and a methodological level of modelling.
  Concepts and tools are first to be described, designed and modeled before
  they can be used in a methodological way.
  
  The paper is not about research but about a practical contribution to
  enhance the common public TRIZ research infrastructure with new tools and
  data based on modern Semantic Web concepts. All code and data is publicly
  available at github \cite{25} and waits for new combatants.\bigskip

  \textbf{Keywords:} Linked Open Data, TRIZ Ontology, RDF, Semantic
  Technologies
\end{abstract}

\section*{1 Aim of this Paper}

In this paper we report on the efforts within the WUMM project \cite{26} to
apply semantic technologies in the TRIZ domain to relevant methodological as
well as social aspects and processes.

The first activities were aimed at building up a semantically supported
\emph{TRIZ Social Network}, in which information about important TRIZ
activities (conferences, events, presentations, other TRIZ-relevant
activities) is collected. A second focus was on the semantic preparation of
the TRIZ Body of Knowledge \cite{14,15} in a multilingual version.

Recently we refocused our activities in favor of a contribution to the TRIZ
Ontology Project (TOP) launched at the TRIZ Developer Summit 2019 in Minsk.
TOP attempts to model and map the landscape of TRIZ concepts across the whole
spectrum of different TRIZ areas. In addition to a common view on fundamental
concepts of TRIZ problem solving, it also focuses on areas such as
\begin{itemize}
 \item laws of evolution of technical and general systems (ZRTS and ZRS),
 \item development of creative imagination (RTV),
 \item theory of the development of innovative personalities (TRTL),
 \item TRIZ history,
 \item TRIZ application areas etc.
\end{itemize}

Thus this project is strongly in the spirit of development of a general LOD
World\footnote{LOD is the abbreviation for Linked Open Data, a world of
  interlinked data and “worlds of concepts” steadily growing during the last
  15 years. See \url{https://lod-cloud.net/}.} and its efforts on
terminological, taxonomic and notational standardisation of conceptual worlds
in precisely definable areas such as FOAF \cite{7}, SKOS \cite{19,20}, ORG
\cite{17}. However, TOP does not yet adequately exploit the experience gained
in such more general projects how socio-cultural coordination processes can be
organised forming an Open Culture environment.

The TOP project is chosen as a reference project because it is one of the
first efforts \emph{across} TRIZ schools to apply semantic concepts to and
within TRIZ. In section 2, this background is presented in more detail. In
section 3 we compare their semantic approaches with modern developments in the
rapidly changing field of semantic technologies. In section 4, we discuss in
more detail the concept of a \emph{system}, which is central to TRIZ, and
shortcomings of various modelling approaches. Such connections between models
and concepts in well-defined conceptual worlds form the socio-cultural core of
semantic technologies \cite{9}. Finally, in section 5, the central
meta-concepts of our ontologisation are presented on this modelling basis as
an example.

As an infrastructure project, WUMM is designed for participation and
collaboration of interested parties. Section 5 is essential to understand the
basic approach if you plan to delve deeper into individual WUMM sub-projects
to which you would like to contribute. There is no space in this paper to go
into more detail on such individual sub-projects. The interested reader is
referred to \cite{10}.  The materials are openly available in our github
infrastructure \cite{25} in machine-readable RDF format. The concepts can be
evaluated on the basis of first practical results, especially our multilingual
and provenance-aware combined glossary, in a prototypical web implementation
\cite{27}.

\section*{2 Background}

In 2019, a group of TRIZ specialists around A.G. Kuryan and M.S. Rubin
launched the \emph{TRIZ Developer Summit Ontology Project} (TOP). It aims at
nothing less than to collect detailed information of the status quo of the
whole TRIZ theory corpus in an ontological mapping. The work is a natural
continuation of earlier efforts by other authors \cite{14,15} to outline a
\emph{TRIZ Body of Knowledge}. While the latter focused on a guide through the
literature, the TOP activities are concerned with the identification of
essential concepts and relationships between these concepts using a modern
semantic approach.

The status of the TOP project was presented at the TRIZ Developer Summits in
2019 and 2020 and fixed in two publications \cite{11,12}. In a webinar
series\footnote{See \url{https://wumm-project.github.io/OntologyWebinar} for
  links to the presentations and an English summary of the talks and
  discussions.} first approaches of a detailed modelling of several TRIZ
sub-areas were presented. The project operates its own
website\footnote{\url{https://triz-summit.ru/onto_triz/}} on which
consolidated results are published.

The main results so far have been a mapping of the "continents" of the TRIZ
world as a \emph{Top Level Ontology} as well as a (still developing) division
of that world into \emph{Ontomaps} as specifically defined areas, which are to
be modeled in more detail. Moreover, a \emph{thesaurus} of about 500 terms as
essential TRIZ concepts has been identified, which are also to be defined in
more detail. V. Souchkov’s glossary in its version 1.2 \cite{21} serves as
basis for this work. In the meantime a first list of 100 terms \cite{23} has
been published on the TOP website.

The efforts differ significantly from earlier approaches to develop a TRIZ
ontology \cite{3, 4, 5, 6, 31, 32}. In those earlier works, the focus was
rather on models of elements of a concrete TRIZ problem solving strategy based
on IDM. The model was mainly used to develop corresponding tools, e.g.
\cite{5}, or processual elements in flow charts, e.g. \cite{4}. The efforts
were not directed towards a community infrastructure effort and – different to
the WUMM project – the material was not released to the public domain.

\section*{3 Ontology Modelling Basics}

The basis for these and the more recent modelling in the TOP project is the
OWL ontology. Although being very powerful, OWL has several disadvantages. OWL
was designed as a unified tool for multiple tasks. The associated high
complexity proved counterproductive, as it leads to algorithmically unsolvable
tasks in sufficiently meaningful contexts. It has proven successful not to
fully formalise meanings and to use different concepts and tools for different
aspects.

Limits for the cardinality of attribute values of a predicate, which are
required to validate data and implement a web interface, are expressed in more
recent developments of the Semantic Web on the basis of SHACL \cite{18}. The
inference possibilities of OWL that go beyond this are hardly used in
practice. The modelling restrictions of weaker OWL variants as OWL-DL do not
meet the requirements of real-world modelling even of \emph{structural}
relationships in TRIZ.

Our approach therefore returns to RDF as modelling base and consistently
relies on the SKOS ontology as a lightweight modelling framework for
structural relationships in conceptual systems. Such a restriction is
reasonable also according to general insights into the development of
conceptual systems in a first stage of ontological modelling. On this basis we
model structural aspects and relationships between TRIZ concepts and tools.
\emph{Processual} relationships as in \cite{4} or questions of an
implementation of web interfaces as in \cite{5} are initially not covered.
Note that comprehensive experience from other application areas is available
using SHACL especially for the second question.

The main disadvantage of the TOP approach so far is the inconsistent use of
semantic means. They play a role in background and internal processes within
the TOP team, but even a clear namespace concept for URIs\footnote{URI –
  Unique Resource Identifier, one of the basic RDF concepts. This string is
  the digital identity of a concept and allows to add independently
  information about “the same thing” in a distributed environment.}, the
public availability of the results in an RDF store or at least as files in a
relevant format and even a SPARQL endpoint for querying the concepts – all
this is missing.

Such an infrastructure was developed and set up in the context of the WUMM
project \cite{26}. The data is publicly available at github \cite{25} and
forms the basis for a prototypical web site \cite{27} that uses simple
semantic tools to present different facets of the data. Via a SPARQL endpoint
\cite{29} experts can make their own complex queries to the data set.

This technical basis is the starting point for our contribution to the TOP
project as \emph{WUMM TRIZ Ontology Companion Project} (WOP) \cite{30}. This
project accompanies the TOP activities in order
\begin{enumerate}
\item to carry out a remodelling according to semantic standards, 
\item to enhance the material multilingually and
\item to compile a Linked Open Data infrastructure on this basis,
\end{enumerate}
and thus to improve the basis for the necessary social coordination processes.
Note that WOP is not an integral part of the TOP activities.

In addition to our own modelling (so far of the TRIZ Principles, the TRIZ
Inventive Standards and the TRIZ Business Standards), the \emph{Top Level
  Ontology} and the division into \emph{Ontomaps} are available in this
format. The work on a \emph{thesaurus} as well as the presentation of
different approaches to a common glossary is actively accompanied with own
efforts, including RDF versions of glossaries developed by Lippert/Cloutier
\cite{13}, Matvienko \cite{16} and in the VDI norm 4521 \cite{24}. Different
explanations of the same term can coexist within our system since the
provenance of the definitions of different TRIZ schools is stored. This
aspect, together with the focus on multilingualism based on relevant RDF
concepts, are essential add ons of the WOP approach to TOP.

In the following we explain the basic modelling and semantic assumptions,
concepts and settings of the WOP approach in more detail.

\section*{4 Modelling a TRIZ Ontology}

\subsection*{4.1 TRIZ and the World of (Technical) Systems}

Main TRIZ concepts revolve around the central notion of a \emph{system}, its
planning, creation, operation, maintenance, further development,
etc. Following the widely accepted understanding of that concept in the TRIZ
community, TOP defines this notion as follows:
\begin{quote}
  A \emph{system} is a set of elements in relationship and connection with
  each other, which forms a certain integrity, unity. The need to use the term
  “system” arises when it is necessary to emphasize that something is large,
  complex, not fully immediately understandable, yet whole, unified. In
  contrast to the notions of “set” and “aggregate“, the concept of a system
  emphasizes order, integrity, regularities of construction, functioning and
  development. The notion of system is part of the system and functional
  approach, and is used in the system operator.
\end{quote}
Usually, however, the definition of a system refers to the concept of a
\emph{component}, as in Souchkov’s glossary \cite{21}:
\begin{quote}  
  \emph{Technical System:} A number of components (material objects) that were
  consciously combined to a system by establishing specific interactions
  between the components. A technical system is designed, developed,
  manufactured, and assigned to perform a controllable main useful function or
  a number of functions within a particular context. A technical system can
  include subsystems which can be considered as separate technical systems.

  \emph{Component:} A material object (substance, field, or substance-field
  combination) that constitutes a part of a technical system or its
  supersystem. A component might represent both a single object and a group of
  objects.
\end{quote}
In both concepts a system is essentially a collection of components that
interact in a specific way to produce the characteristic functionalities of
the system. The subsystems referred to as components provide own functions,
but the functionalities of the system do not result from a simple addition of
the functions of the components, but as an emergent system property from their
interaction. For the modelling of systems, their \emph{structural}
organisation and their \emph{processual} organisation are equally important.
The systemic approach is thus self-similar and fractal; the terms “system” and
“component” are largely used synonymously depending on the respective
modelling focus.

In TRIZ, an \emph{engineering problem} is always conceptualised as the design
of a new system or the improvement of an existing one. The design of a new
system can be considered as a special case of further development, since also
in this case concepts of a model of the “system as it is” do exist, how vague
they may be.

The delimitation of meaningful systems as modelling units has many facets and
points of view, see for example \cite[section 8]{22}. In the TRIZ concepts, a
certain functional completeness plays a major role in this delimitation, even
if a defined throughput of energy, material and information is required to
operate the system.

For a system, its \emph{design} and \emph{operation} have to be distinguished,
as explained in \cite{8} in more detail. This also applies to
\emph{components} of a system. In the white-box analysis of a system, its
components are considered as viable black-boxes, which are characterised in
the design dimension by a \emph{specification of their functionality} and in
the operational dimension by its \emph{guaranteed specification compliant
  operation}, provided that the operating conditions (in particular the
throughput of material, energy and information required for its operation) are
ensured within the system. The description of these operating conditions is
part of the specification, which thus consists of an input and an output part
(also referred to as import and export interfaces in Computer Science).

The components thus constitute a \emph{world of technical systems} in the
sense of the explanations in \cite{8}, to which we refer for further details
of this conceptualisation.

\subsection*{4.2 Abstraction Levels of Modelling}

An ontology is about “modelling of models”, because the clarification of terms
and concepts of an ontology is intended to be practically used in real-world
modelling contexts. This “modelling of models” references a typical
engineering context, in which the modelling of real systems plays a central
role and serves as basis of further planned action (including project
planning, implementation, operation, maintenance, further development of the
system).

In this process, several levels of abstraction are to be distinguished.

\paragraph{0.}
The level of the \emph{real-world system} as the target of the engineering
task. This level is only \emph{practically} accessible. The model to be
developed at level 1 must be appropriate to cover all problems arising in the
process of development and use of the real system and to express its inherent
contradictory character.

This contradictory nature of the system can be formulated only in language
terms, i.e. on the model level and \emph{applying} the concepts available
there.  These concepts must therefore not only be able to describe the
structure of the system itself, but have also to cover a description of the
necessary aspects of its operation.

\paragraph{1.}
The level of \emph{modelling the real-world system}. The worlds of
\emph{several} conceptual systems often meet in the modelling of a real-world
system with its \emph{core and cross-cutting concerns} (concepts known from
Software Engineering \cite{22}). In addition to the methodological dimension
of a TRIZ ontology, this regularly includes the conceptual world of a
domain-specific technical ontology and possibly other conceptual worlds such
as a company-internal compliance etc.

These different conceptual systems (ontologies) provide the language means,
concepts (RDF subjects) and properties (RDF predicates), which are to be
\emph{applied} at this level. This level is also the \emph{level of
  methodological practice}.

\paragraph{2.}
The \emph{level of the meta-model} is the actual (TRIZ) ontology level (and
also of domain-specific ontologies) on which the systemic concepts are
\emph{defined}.

These definitions are processed \emph{applying} the methodological concepts
whose linguistic means are made available on meta-level 2.

\paragraph{3.}
The \emph{modelling meta-level 2} is the level at which the methodological
concepts are \emph{defined}.

\subsection*{4.3 The TOP Concept of a System}

A central concept in TOP modelling is the distinction between the stages of
\begin{itemize}
\item[(1)] the system as it is,
\item[(2)] the TRIZ model of the system as it is,
\item[(3)] the TRIZ model of the system as required, and
\item[(4)] the system as required.
\end{itemize}
The TOP glossary \cite{23} explains the differences as follows
\begin{itemize}
\item[(1)] The \emph{system as it is} is a system in its original state before
  it is analysed and transformed into a new “system as it is”.
\item[(2)] The \emph{TRIZ model of the system as it is} is formed from the
  “system as it is” by means of various TRIZ models: component-structural and
  functional models, su-field or ele-field models, description of
  contradictions or of typical conflict schemes, etc. Depending on the chosen
  model type, the model will be transformed into the “TRIZ model of the system
  as required”.
\item[(3)] The \emph{TRIZ model of the system as required} is formed from the
  “TRIZ model of the system as it is” by procedures which correspond to the
  selected model transformation method (functional, su-field, ele-field,
  solution of the contradictions in requirements and properties, etc.). The
  transition is performed along the line 
  \begin{center}
    System as it is $\to$ TRIZ model of the system as it is\\ $\to$ TRIZ model
    of the system as required $\to$ System as required
  \end{center}
\item[(4)] The \emph{system as required} is a system derived from the “system
  as it is” through a transformation, based on the “model of the system as
  required”.
\end{itemize}

In stage (1) “system” can only mean a \emph{model of the system} in which, in
addition to the ontology of the TRIZ methodology, domain-specific ontologies
play a central role. A system can only be described via a model.

In stage (2) the “TRIZ model of the system as it is” is derived by application
of specific structural TRIZ concepts and instruments. How is this to be
understood? Is the (model of the) “system as it is” initially a
domain-specific model that is to be enriched by an appropriate TRIZ model in
this stage (2)? Such an understanding would contradict TRIZ modelling
practices, which methodically are to be applied already in the creation of the
domain model. For example, modelling the specific application at stage (1) the
schema of a \emph{minimal technical system} as template is to be filled in to
get the problem-specific model. According to the hill schema, in stage (2)
rather the specific TRIZ structure of the problem-specific model is to be
determined. This requires to strengthen the domain-specific model from stage
(1) in a targeted manner at points to be identified (operative zone and
operative time).

This TRIZ model as a \emph{prototypical abstraction} of the problem-specific
model of the real-world problem determines at the same time the \emph{abstract
  TRIZ tools} to be applied at stage (3) and thus provides the context for the
transition to the abstract solution model “on the top of the hill”, the
\emph{TRIZ model of the system as required}. In the end, this abstract
solution model has to be “rolled down the hill” to obtain the
(problem-specific model of the) “system as required” in stage (4).

The TRIZ model is thus a \emph{context} for all four stages of real-world
modelling. In this way the notion of TRIZ model is also explained in
\cite{23}:
\begin{quote}  
  A \emph{TRIZ model} is a schematic notation of a gradual transition from the
  problem to TRIZ model of the problem, then to TRIZ model of the solution and
  then to the solution itself; or from the system to TRIZ model of the system,
  then to TRIZ model of the new system and then to actual change of the system
  (“system as required”). The TRIZ model includes the basic components of
  inventive thinking: analysis, synthesis, evaluation.
\end{quote}
Hence a \emph{TRIZ model} is the common (developing during the stages)
abstract TRIZ context of the four model stages described above, including the
modelling process itself. However, these four stages all refer to abstraction
level 1 models of a real-world system in the meaning developed in section 4.2;
no distinction is made between application of concepts from level 2 of a
\emph{TRIZ ontology of tools} (present at level 1 as concept instances) and
level 3 of a \emph{TRIZ ontology of methods} (present at level 1 only in a
methodological-practical way).

What does this mean for the scope of a TRIZ ontology? The modelling of any
system starts at stage (1) with a problem-specific model of the “system as it
is” based on domain-specific concepts. If this modelling is practically
performed on the methodological basis of TRIZ principles, the domain-specific
system of concepts must be enriched with TRIZ methodological concepts such as
MPV, conflicting pairs, operative zone and operative time, etc. Stage (2)
requires a special abstraction from domain-specific concepts for the
extraction of abstract TRIZ patterns as a “TRIZ model of the system as it is”
(TRIZ task model) according to the hill schema. Hence the modelling of the
real-world system requires the domain-specific concepts of the
problem-specific model to be compatible with the requirements of TRIZ
modelling. Both ontologies – the domain-specific and the TRIZ ontology – have
a similar relationship of the specific to the general and thus stand in a
relationship of mutual complementarity of their modelling languages at the
level of the problem-specific model.

It is obvius, however, that the (problem and domain specific) “model of the
system as it is” (MSI), the abstract “TRIZ model of the system as it is”
(TSI), the resulting “TRIZ model of the system as required” (TRIZ solution
model, TSO) and finally the (again problem and domain specific) “model of the
system as required” (MSO) call up largely the same language constructs from
the point of view of a TRIZ ontology and are thus four instances of the
(developing through the four stages) model of the real-world system related by
consecutive instance transformations. These instance transformations can be
characterised as follows:
\begin{itemize}
\item \textbf{MSI $\to$ TSI}: Consolidation and refinement of TRIZ-relevant
  concepts in the MSI.
\item \textbf{TSI $\to$ TSO}: Description of an abstract transformation and
  execution of the parts of the transformation that are possible at this
  level, i.e. without interaction with the domain-specific parts of the model.
\item \textbf{TSO $\to$ MSO}: Detailing the model, completion and execution of the
  domain-specific part of the transformation.
\end{itemize}
For the level 2 ontology (it answers the question “Which TRIZ tools are
available and how do they relate to each other?”), the distinction between
these four system models is therefore not relevant. Corresponding language
tools are only required at level 3, when it comes to the \emph{description} of
the application of the TRIZ methodology itself.

Concerning the balance between the new and the old, as suggested by relevant
methodologies for the further development of conceptual systems, we see the
requirement to clearly distinguish between these two levels of ontologisation
and limit our ontological modelling to level 2.

\section*{5 Basics of the WUMM Ontology Project}

\subsection*{5.1 SKOS Basics}

The SKOS ontology \cite{19,20} allows to express concepts and their relations
in a lightweight way. The class \texttt{skos:Concept} and the predicates
\texttt{skos:narrower}, \texttt{skos:broader} and \texttt{skos:related} are
used for this purpose. The first two predicates describe hierarchical
relationships between concepts, the third one is used for non-hierarchical
relationships.

Relationships between concepts can be of very different nature.  Hierarchical
relationships, for example, can model (transitive) subconcept relationships in
taxonomies as well as whole-part relationships, which are inherently
non-transitive when concepts of different qualities are related. Both types of
conceptualisation have an intentional as well as an extensional aspect – the
new units of meaning, especially their emergent properties, can neither be
adequately described by mere enumeration of their subconcepts nor by the
“legitimate interpretation of sense” of the intentions of their constitution
in the meaning developed by Berger/Luckmann in \cite{2}. In the SKOS primer
\cite{20} these modelling aspects are described in more detail, especially the
modelling of class-instance and whole-part relationships. We follow the
recommendation in \cite[sect. 4.7]{20} and introduce subpredicates of the
generic SKOS predicates listed above for different modelling contexts.  More
detailed modelling rules for such contexts are described and discussed below.

The WUMM project aims to model a unified level 2 space of TRIZ concepts
without “concepts of concepts”. Hence we limit the concepts used from the SKOS
universe to those described above and do not use further SKOS aggregation
concepts such as \texttt{skos:Concept\-Scheme}, \texttt{skos:Collection},
\texttt{skos:OrderedCollection} etc. The aggregation of different concepts in
collections (assignment to TRIZ generations \cite[Table 1]{12}, in concept
classes Basic, Model, Rule and TermGroup \cite[Fig. 4]{12} or Categories in
\cite{21}) is realised via special predicates.

SKOS provides an initial descriptive framework for conceptualisations. We use
the following concepts (K) from the SKOS ontology \cite{19}
\begin{itemize}
\item \texttt{skos:Concept}, \texttt{skos:prefLabel}, \texttt{skos:altLabel} –
  concept naming
\item \texttt{skos:definition}, \texttt{skos:example}, \texttt{skos:note} –
  concept properties
\item \texttt{skos:narrower}, \texttt{skos:broader}, \texttt{skos:related} –
  concept relations.
\end{itemize}
For the meaning and usage of the different SKOS concepts, we refer to
\cite{19} and the explanations below.

\subsection*{5.2 URIs and Namespaces}

The allocation of meaningful URIs is one of the central problems of
transferring the existing stock of data on TRIZ concepts, since the individual
glossary entries in the existing TOP sources are identified solely by their
labels. This applies even to the OSA platform\footnote{The OSA platform is
  used as an TOP internal ontology editor, see
  \url{https://wumm-project.github.io/TOP} for more information about the
  platform, its odds and evens.} since the URIs assigned there (both for the
nodes and the edges of the RDF graph) are not publicly visible.

For a concise concept of URIs we first define \emph{namespaces} which
correspond to the different modelling contexts. Since \emph{ontology
  modelling} has the basic purpose to be applied in problem-specific
\emph{modellings of real-world systems}, at least these two modelling contexts
have to be distinguished. In our application the modelling context of a
(prototypical) real-world system is present only in examples which practically
demonstrate the effect of ontology modelling decisions. At the level of
ontology modelling, we further distinguish between the parts of the concepts
that are largely uncontroversial\footnote{These are mainly the concepts to be
  included in a glossary. We assign URIs of a \texttt{skos:Concept} to them
  and model their names as \texttt{skos:prefLabel}.} and the parts of the
concept for which special conceptual approaches have been developed within the
WUMM Ontology Project (WOP). For these different abstraction layers we use the
following namespaces:
\begin{itemize}
\item \texttt{ex:} – the namespace of a problem-specific model of a special
  real system.
\item \texttt{tc:} – the namespace of the TRIZ concepts (RDF subjects).
\item \texttt{od:} – the namespace of WUMM’s own concepts (RDF predicates,
  general concepts).
\end{itemize}

\subsection*{5.3 Provenance of Explanations}

Another problem of this ontological modelling is the representation of the
provenance of the individual explanations. For this purpose the SKOS concepts
listed under (K) are replaced for each individual source by subconcepts in the
namespace \texttt{od:} in order to separate the “worlds” of the individual
TRIZ schools. The same applies to the use of provenance-dependent subclasses
of \texttt{skos:Concept}.

Such notational variations are for example
\begin{itemize}
\item \texttt{skos:Concept} $\to$ \texttt{od:GSAThesaurusEntry},
  \texttt{od:VDIGlossaryEntry} \ldots
\item \texttt{skos:definition} $\to$ \texttt{od:SouchkovDefinition},
  \texttt{od:VDIGlossaryDefinition} \ldots
\item \texttt{skos:example} $\to$ \texttt{od:VDIGlossaryExample} \ldots
\end{itemize}
etc. Here \texttt{GSAThesaurus} stands for the thesaurus published on the
Altshuller website \cite{1}, \texttt{VDIGlossary} for the VDI glossary
\cite{24} and \texttt{SouchkovDefinition} for V. Souchkov's glossary
\cite{21}. All these data were available or provided in a machine-readable
format, transformed into suitable RDF formats and stored as open source both
as files in the github repo \emph{RDFData} at \cite{25} and in our RDF Store
\cite{28}. See the RDF data itself, which can also be queried via our SPARQL
Endpoint \cite{29}.

This is used to build a \emph{combined glossary} where definitions from
different TRIZ schools of the same concept co-exist. This is implemented
prototypically\footnote{See \url{http://wumm.uni-leipzig.de/ontology.php}.} in
such a way, that for each concept represented by a URI, a link displays all
RDF triples in which this concept occurs as a subject or object. Further links
in this representation can be used to navigate in the entire RDF graph.

\section*{6 Conclusion}

We presented in this paper some of our experiences within the WUMM project
\cite{26} as a contribution to an Open TRIZ Research Infrastructure based on
modern Semantic Web technologies.

The theoretical explanations focus on a critical consideration of the
modelling of a TRIZ system concept as proposed in the context of the TOP
project. We justify in more detail why, in our view, it makes sense to
distinguish between the \emph{conceptual level of model structures} and the
level of \emph{methodological concepts} of the \emph{application} of model
structures in the course of modelling a TRIZ ontology.

This distinction is in the core of the WUMM project, which initially
concentrates on the conceptual-taxonomic level of TRIZ. Collecting
conceptual-taxonomic data we use the features of RDF to preserve the
provenance of different interpretations in a machine-readable dataset. Thus we
avoid to take our own position in the dispute over the exact meaning of
individual glossary terms. This lays the foundation for a (potentially)
broader process of understanding and standardisation. Moreover, RDF's
multilinguality concepts can be used to support such a process also across
different languages. Of course, this raises additional hurdles of
cross-cultural understanding also at the semantic level. The WUMM project uses
the technical infrastructure of github which is well suited to offer a
practical technical basis also for such a socio-cultural communication
process.

This paper is not a research paper, but takes the opportunity to report to the
audience of the \emph{TRIZ Future Conference} as the leading annual conference
in the field of systematic innovation methodologies on the status of our
research infrastructure project.  

\section*{7 Postscript}

The paper was accepted by the reviewers for presentation at the \emph{TRIZ
  Future Conference 2021}, but it does not meet the "novelty" criteria for a
paper to be included in the Conference Proceedings, as 63\% of the material
presented here\footnote{According to the analysis of the chairs of the
  conference.} can also be found on the pages of the WUMM project (in
particular in various preprint publications) and hence „is not new“.  Such
rules massively hinder the further development of scientific ideas and call
into question the discursive character of scientific work. \emph{LIFIS-Online}
is a scientific journal that stands on clearly different positions. Hence this
survey is published in this journal.

\begin{thebibliography}{xxx}
\bibitem{1} Altshuller Web Site. Basic TRIZ Terms.\\
  \url{https://www.altshuller.ru/thesaur/thesaur.asp}
\bibitem{2} Berger, P.L., Luckmann, T. (1966). The Social Construction of
  Reality: A Treatise in the Sociology of Knowledge. Anchor Books.
\bibitem{3} Bultey, A., de Beuvron, F.d.B., Rousselot, F. (2007). A
  Substance-field Ontology to Support the TRIZ Thinking Approach.
  International Journal of Computer Applications in Technology 30 (1),
  113-124.\\  \url{https://doi.org/10.1504/IJCAT.2007.015702}.
\bibitem{4} Bultey, A., Yan, W., Zanni, C. (2015). A Proposal of a Systematic
  and Consistent Substance-field Analysis. Procedia Engineering 131,
  701-710. \\ \url{https://doi.org/10.1016/j.proeng.2015.12.357}.
\bibitem{5} Cavallucci, C., Rousselot, F., Zanni, C. (2011). An Ontology for
  TRIZ.  Proc. TRIZ Future Conference 2009. Procedia Engineering 9,
  251–260.\\ \url{https://doi.org/10.1016/j.proeng.2011.03.116}.
\bibitem{6} Dubois, S., Lutz, P., Rousselot, F., Vieux, G. (2007). A Model for
  Problem Representation at Various Generic Levels to Assist Inventive Design.
  International Journal of Computer Applications in Technology 30 (1),
  105-112.\\  \url{https://doi.org/10.1504/IJCAT.2007.015701}.
\bibitem{7} FOAF Vocabulary Specification 0.99. W3C Namespace Document.
  Version of 14 January 2014. \url{http://xmlns.com/foaf/spec/}
\bibitem{8} Gräbe, H.-G. (2020). Human and Their Technical Systems. In
  Proceedings of the TRIZ Future Conference 2020. Springer, Cham,
  399-410.\\ \url{https://doi.org/10.1007/978-3-030-61295-5_30}
\bibitem{9} Gräbe, H.-G. (2021). Technical Systems and Their Purposes. In
  TRIZ-Anwendertag 2020. Springer Vieweg, 1-13.
  \url{https://doi.org/10.1007/978-3-662-63073-0_1}.
\bibitem{10} Gräbe, H.-G. (2021). About the WUMM modelling concepts of a TRIZ
  ontology. Manuscript 2021. Online available at
  \url{https://wumm-project.github.io/Texts/WOP-Basics.pdf}
\bibitem{11} Kuryan, A., Souchkov, V., Kucharavy, D.: Towards Ontology of
  TRIZ (2019). Proceedings of the TRIZ Developers Summit 2019, Minsk.\\
  \url{https://triz-summit.ru/confer/tds-2019/}. 
\bibitem{12} Kuryan, A., Rubin, M., Shchedrin, N., Eckardt, O., Rubina, N.
  (2020).  TRIZ Onto\-logy. Current State and Perspectives. TRIZ Developers
  Summit 2020 (in Russian).\\ \url{https://triz-summit.ru/confer/tds-2020/}. 
\bibitem{13} Lippert, K., Cloutier, R. (2019). TRIZ for Digital Systems
  Engineering: New Characteristics and Principles Redefined. Systems 2019,
  7 (3), 39.\\  \url{https://doi.org/10.3390/systems7030039}.
\bibitem{14} Litvin, S., Petrov, V., Rubin, M. (2007). TRIZ Body of
  Knowledge.\\ \url{https://triz-summit.ru/en/203941.}
\bibitem{15} Litvin, S., Petrov, V., Rubin, M., Fey, V. (2012). TRIZ Body of
  Knowledge.\\  \url{https://matriz.org}
\bibitem{16} Matvienko, N.N. (1991). TRIZ Terms: A Problem Book (in Russian).
  Vladivostok, 1991.
\bibitem{17} The Organization Ontology. W3C Recommendation. Version of 16
  January 2014.  \url{https://www.w3.org/TR/vocab-org/}
\bibitem{18} Shapes Constraint Language (SHACL). W3C Recommendation. Version
  of 20 July 2017. \url{https://www.w3.org/TR/shacl/}
\bibitem{19} SKOS – The Simple Knowledge Organization System.\\
  \url{https://www.w3.org/TR/skos-reference/.}
\bibitem{20} SKOS Simple Knowledge Organization System Primer.\\
  \url{https://www.w3.org/TR/2009/NOTE-skos-primer-20090818/.}
\bibitem{21} Souchkov, V. (2018). Glossary of TRIZ and TRIZ-related terms,
  version 1.2. The International TRIZ Association, MATRIZ.
  \url{https://matriz.org}
\bibitem{22} Szyperski, C. (2002). Component Software: Beyond Object-Oriented
  Programming. Addison Wesley.
\bibitem{23} TRIZ 100 Glossary (2021). A Short Glossary of Key TRIZ Concepts
  and Terms (in Russian).  \url{https://triz-summit.ru/onto_triz/100/.}
\bibitem{24} VDI-Norm 4521 Blatt 1 (2016). Inventive Problem Solving with TRIZ
  – Basics and Terminology (in German). April 2016.
\bibitem{25} The WUMM github Repositories.\\
  \url{https://github.com/wumm-project.}
\bibitem{26} The github Pages of the WUMM Project.\\
  \url{https://wumm-project.github.io/.}
\bibitem{27} The Web Demonstration Pages of the WUMM Project.\\
  \url{http://wumm.uni-leipzig.de.}
\bibitem{28} The RDF Store of the WUMM Project.\\
  \url{http://wumm.uni-leipzig.de/rdf.}
\bibitem{29} The SPARQL Endpoint of the WUMM Project.\\
  \url{http://wumm.uni-leipzig.de:8891/sparql.}
\bibitem{30} The WUMM TOP Companion Project.\\
  \url{https://wumm-project.github.io/Ontology.html}
\bibitem{31} Zanni-Merk, C., Rousselot, F., Cavallucci, D. (2009). An
  Ontological Basis for Inventive Design. Computers in Industry, 60 (8),
  563-574.
\bibitem{32} Zanni-Merk, C., de Beuvron, F.d.B., Rousselot, F., Yan, W.
  (2013).  A Formal Ontology for a Generalized Inventive Design Methodology.
  Applied Ontology, 8 (4), 231-273.  \url{https://doi.org/10.3233/AO-140128}.
\end{thebibliography}

%\ccnotice
\end{document}

