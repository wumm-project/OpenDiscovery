\documentclass[11pt,a4paper]{article}
\usepackage{a4wide,url}
\usepackage[T1,T2A]{fontenc}
\usepackage[utf8]{inputenc}
\usepackage[main=ngerman,russian]{babel}

\parindent0cm
\parskip3pt

\title{Der Mensch und seine Technischen Systeme} 
\author{Hans-Gert Gräbe, Leipzig}
\date{Version vom 25. April 2020}
\begin{document}
\maketitle

\begin{flushright}
  Die Philosophen haben die Welt\\ nur verschieden interpretiert;\\ es kömmt
  darauf an sie zu verändern.\\ Karl Marx. 11. Feuerbachthese
\end{flushright}
\section{Vorbemerkungen}

Ausgangspunkt dieser Überlegungen war eine Debatte mit den Organisatoren des
TRIZ-Cups 2019/20 über die Gültigkeit eines „Gesetzes der Verdrängung des
Menschen aus technischen Systemen“, das in einer späteren Fassung der
Ausschreibung als „Trend“ bezeichnet wurde. Unter den acht Gesetzen der
Entwicklung technischer Systeme, die Altschuller 1979 selbst formulierte
\cite[S. 2]{TESE2018}, kommt ein solcher Ansatz nicht vor, auch nicht in der
Auf"|listung von fünf Gesetzen und zehn Tendenzen in
\cite[S. 148\,ff.]{KS2017}.  Einen deutlich anderen Ansatz, die Betrachtung
der Evolution einzelner Funktionen und nicht kompletter technischer Systeme,
schlägt N. Shpakovsky in \cite{Shpakovsky2010} mit seinem Konzept der
„Evolutionsbäume“ vor.

Systematisierungen von „Gesetzen der Evolution technischer Systeme“ oder zum
„technology forecasting“ sind aber in der TRIZ-Literatur weit verbreitet. Sie
sind auch Teil der verschiedenen Versionen eines „TRIZ Body of Knowlegde“,
etwa \cite{TBK-2007}. Meine weiteren Ausführungen beziehen sich auf
\cite{TESE2018} als Referenz, da hier von einflussreichen TRIZ-Theoretikern
mit der Autorität der MATRIZ im Rücken ein aktueller Zusammenschnitt der
Debatten um „Trends of Engineering Systems Evolution“ gegeben wird.

Auch in der marxistischen Literatur wird ein solcher Herauslösungsprozess des
Menschen aus produktiven Prozessen thematisiert und an vielen Stellen als
unausweichlich charakterisiert.  So entwickelt Marx selbst im
„Maschinenfragment“ \cite[S. 570 ff.]{MEW42} -- einem frühen Rohentwurf der
eigenen ökonomischen Theorie -- die Vision einer Gesellschaft, in welcher der
„gesellschaftliche Stoffwechsel“ \cite[S. 37]{MEW23} auf eine Weise
organisiert ist, dass
\begin{quote}
  es nicht mehr der Arbeiter [ist], der modifizierten Naturgegenstand als
  Mittelglied zwischen das Objekt und sich einschiebt; sondern den
  Naturprozess, den er in einen industriellen umwandelt, er als Mittel
  zwischen sich und die unorganische Natur [schiebt], deren er sich
  bemeistert \cite[S. 572]{MEW42},
\end{quote}
und stellt weiter dar, dass die Entwicklung der Produktivkräfte
\emph{notwendig} auf eine solche Weise der Organisation des gesellschaftlichen
Stoffwechsels zusteuert.
\begin{quote}
  In den Produktionsprozess des Kapitals aufgenommen, durchläuft das
  Arbeitsmittel aber verschiedene Metamorphosen, deren letzte die
  \emph{Maschine} ist oder vielmehr ein \emph{automatisches System der
    Maschinerie} (System der Maschinerie; das \emph{automatische} ist nur die
  vollendetste adäquateste Form derselben und verwandelt die Maschinerie erst
  in ein System), in Bewegung gesetzt durch einen Automaten, bewegende Kraft,
  die sich selbst bewegt; dieser Automat bestehend aus zahlreichen mechanischen
  und intellektuellen Organen, sodass die Arbeiter selbst nur als bewusste
  Glieder desselben bestimmt sind. \cite[S. 584]{MEW42}
\end{quote}
Dieser Gedanke sei allerdings weitgehend singulär und im übrigen Marxschen
Werk nirgends ausgearbeitet, so \cite{Goldberg2016}. Zumindest auf das
Interesse an technischen Systemen und Entwicklungen trifft das allerdings
nicht zu, hierzu finden sich viele Stellen im Werk dieser Klassiker. Engels'
Interesse besonders an militär-historischen Entwicklungen ist weithin bekannt,
in \cite{MEW15} wird die Evolution des gezogenen Gewehrs auf eine Weise
analysiert, die es mit jeder TRIZ-Analyse aufnehmen kann, etwa in der
Herausarbeitung des sozio-technischen Widerspruchs \cite[S. 199]{MEW15}:
\begin{quote}
  Unter diesen Umständen ergab sich folgende dringende Aufgabe: eine
  Feuerwaffe zu erfinden, welche die Schussweite und die Genauigkeit der
  Büchse mit der Schnelligkeit und Leichtigkeit des Ladens und mit der Länge
  des Laufs der glattläufigen Muskete vereint, eine Waffe also, die zugleich
  Büchse und Nahkampfwaffe ist, welche man jedem Infanteristen in die Hand
  geben kann.

  So sehen wir also, dass gerade durch die Einführung des Kampfes in
  aufgelöster Ordnung in die moderne Taktik sich die Forderung nach einer
  solchen verbesserten Kriegswaffe erhob. [\ldots] Fast alle Verbesserungen,
  die an Handfeuerwaffen seit 1828 vorgenommen wurden, dienten diesem Zweck.
\end{quote}
\cite{Goldberg2016} untersucht allerdings nicht diese Frage, sondern aktuelle
Kapitalkonzentrationsprozesse, die erforderlich sind, um die immer
kostspieligeren technischen Großprojekte zu finanzieren.  In diesem Kontext
wird neuerdings viel über \emph{Plattformkapitalismus} geschrieben.  Der
unmittelbare Zusammenhang zwischen technologischen und finanziellen
Mögilchkeiten ganzer Staaten im Kontext einer „wealth of nations“ ist
hinreichend bekannt. Siehe etwa \cite{Barkleit2000}, wo die technologischen
und ökonomischen Interdependenzen des letztlich gescheiterten
DDR-Mikroelektronikprojekts der 1980er Jahre genauer analysiert werden. Im
TRIZ-Umfeld spielen derartige Überlegungen eine eher marginale Rolle.

Technikoptimistischen Sichten der „Verdrängung des Menschen aus technischen
Systeme“ steht die Position aus dem Kybernetikdiskurs der 1960er bis 1980er
Jahre entgegen \cite[S. 10]{KFK2000}:
\begin{quote}
  Welche Stellung hat der Mensch im hochkomplexen informations-technologischen
  System? Unsere Antwort auf die Frage war immer: Der Mensch ist die einzig
  kreative Produktivkraft, er muss Subjekt der Entwicklung sein und bleiben.
  Daher ist das Konzept der Vollautomatisierung, nach dem der Mensch
  schrittweise aus dem Prozess eliminiert werden soll, verfehlt!
\end{quote}
Die Probleme eines solchen „Konzepts der Vollautomatisierung“, einer Welt der
„in Bewegung gesetzten Automaten“ werden mittlerweile in einer ökologischen
Krise planetaren Ausmaßes sichtbar. Die Verdrängungsthese selbst wird dabei
als direkte Gefährung wahrgenommen, was hier als \emph{Gegenthese} explizit
formuliert werden soll:
\begin{quote}\it
  \textbf{Gegenthese 1:} Die (scheinbare) Verdrängung des Menschen aus
  technischen Systemen weist auf eine existenziell gefährliche, unterkomplexe
  Fehlwahrnahme dieser technischen Systeme hin.
\end{quote}
Das Altschullersche „Prinzip 11 der Prävention“ (verschämt auch als „Prinzip
des untergelegten Kissens“ bezeichnet) weist auf Handlungsbedarf in dieser
Richtung hin, der Anwendungskontext dieses Prinzips wird in \cite{TT} wie
folgt umrissen: „Der Einsatz des Prinzips ist besonders in solchen Fällen
wichtig, in denen das System nicht über ein ausreichendes Maß an
Zuverlässigkeit verfügt“, um dann festzustellen, dass dem eigentlich
strukturell abgeholfen werden könne -- „Notfallsituationen können vermieden
werden, indem der Prozess zuverlässig gemacht wird.“ Dem stehen allerdings
gewichtige Gründe entgegen -- „was die technischen Systeme, die ihn
durchführen, erheblich verkompliziert oder verteuert. Dies ist kostspielig und
oft prinzipiell unmöglich. Mit anderen Worten: Notfälle sind unvermeidlich“.
Der Optimismus der Autoren, dass „zusätzliche Rettungs- und Notfallsysteme
[\ldots] in das Hauptsystem“ eingebaut werden, die allerdings „nicht am
Hauptsystem teilnehmen, sondern erst in einer Gefahrensituation zu arbeiten
beginnen“, erscheint mit Blick auf Kosten reduzierende Designpraxen und die
allgemeine Ausrichtung der anderen TRIZ-Prinzipien auf Effizienzgewinne als
Widerspruch in der TRIZ-Methodik selbst. Die Aussage, „das Prinzip kann dort
angewendet werden, wo die Zuverlässigkeit des Systems offensichtlich
unzureichend ist und ein Weg zur Erhöhung der Zuverlässigkeit auf das
notwendige Niveau nicht möglich ist“ (ebenda) zeigt, dass der Einbau solcher
Defizite in technische Systeme -- in Kenntnis derselben -- auf breiter Front
billigend in Kauf genommen wird.

Dies ist allerdings in keiner Weise mehr ein technisches Problem.  Harrisburg,
Tschernobyl, Fukushima, das Bienensterben \cite{Jacobasch2019} oder der
Klimawandel sind genü"|gend Fingerzeige, um sich mit diesen Positionen genauer
zu befassen.

Wir zeigen in diesem Aufsatz, dass die 10 nunmehr als „Trends“ präsentierten
„Gesetze der Entwicklung technischer Systeme“ in Wirklichkeit Design Pattern
in den ingenieur-technischen Praxen des Entwurfs sowie der Anpassung und
Verbesserung technischer Systeme sind und sich somit \emph{unmittelbar} auf
sozio-technische Praxen beziehen. Dabei beziehen sie sich auf ein
\emph{spezifisches} begriff"|liches Abstraktionsniveau der sich insgesamt in
Widersprüchen entwickelnden Beschreibungsformen von Welt. Insbesondere stehen
die „Trends“ damit im Widerspruch zu Entwicklungslinien, die sich auf anderen
Abstraktionsniveaus abzeichnen. Kurz, das oben in These und Gegenthese
entfaltete ambivalente Verhältnis zu einer „Verdrängung des Menschen aus
technischen Systemen“ ist kein Alleinstellungsmerkmal nur für diesen Trend,
sondern trifft in ähnlicher Weise auch auf die anderen 9 Trends zu.

TRIZ ist eine gute Methodik, um diese Widersprüche zu analysieren. Dazu darf
man sich aber nicht auf das Memorieren dieser Trends beschränken.  

\section{Technik und Welt verändernde Praxen}

Betrieb und Nutzung technischer Systeme ist heute sicher ein zentrales Element
Welt ver"|ändernder menschlicher Praxen. Dafür ist planmäßiges und abgestimmtes
arbeitsteiliges Handeln erforderlich, denn das Nutzen eines Systems setzt
dessen Betrieb voraus.  Umgekehrt ist es wenig sinnvoll, ein System zu
betreiben, das nicht genutzt wird. In der Informatik ist dieser Zusammenhang
zwischen Definition und Aufruf einer Funktion gut bekannt -- der Aufruf einer
Funktion, die noch nicht definiert wurde, führt zu einem Laufzeitfehler; die
Definition einer Funktion, die nie aufgerufen wird, weist auf einen
Designfehler hin.

Eng verbunden mit der informatischen Unterscheidung von Definition und Aufruf
einer Funktion ist die Unterscheidung von Designzeit und Laufzeit.  Eine
solche Unterscheidung hat im realweltlichen arbeitsteiligen Einsatz
technischer Systeme noch größere Bedeutung -- während der Designzeit wird das
prinzipielle kooperative Zusammenwirken \emph{geplant}, während der Laufzeit
\emph{der Plan ausgeführt}. Für technische Systeme sind also zusätzlich deren
interpersonal als \emph{begründete Erwartungen} kommunizierten
\emph{Beschreibungsformen} und die in \emph{erfahrenen Ergebnissen}
resultierenden \emph{Vollzugsformen} zu unterscheiden.

Marx \cite[S. 193]{MEW23} merkt dazu an:
\begin{quote}
  Eine Spinne verrichtet Operationen, die denen des Webers ähneln, und eine
  Biene beschämt durch den Bau ihrer Wachszellen manchen menschlichen
  Baumeister. Was aber von vornherein den schlechtesten Baumeister vor der
  besten Biene auszeichnet, ist, dass er die Zelle in seinem Kopf gebaut hat,
  bevor er sie in Wachs baut. Am Ende des Arbeitsprozesses kommt ein Resultat
  heraus, das beim Beginn desselben schon in der Vorstellung des Arbeiters,
  also schon ideell vorhanden war.
\end{quote}
So einfach ist es allerdings nicht, wie das folgende Beispiel einer
Konzertauf"|führung zeigt. Dieser die Zuhörer erfreuenden Vollzugsform geht
die Erarbeitung der Beschreibungsform, die Verständigung über die genaue
Interpretation des aufzuführenden Werks, voraus. Diese Verständigung auf einen
\emph{gemeinsamen Plan} ist selbst ein voraussetzungsreicher praktischer
Prozess.  Die Voraussetzungen resultieren aus vorgängigen Praxen -- etwa dem
\emph{privaten Verfahrenskönnen} der einzelnen Musiker in der Beherrschung
ihrer Instrumente sowie dem Vorliegen der Partitur als etablierter
Beschreibungsform des aufzuführenden Konzertstücks.  Wenn Alexander Shelley am
14. Oktober 2018 im Leipziger Gewandhaus ohne diese Partitur von Mozarts
Klavierkonzert KV 491 ans Dirigentenpult tritt, so wird deutlich, dass jene
Beschreibungsform allenfalls das Rohmaterial liefert, auf dessen Basis sich
Dirigent und Orchester in den vorausgegangenen Proben auf eine situativ
konkrete Beschreibungsform als Basis der nun zur Auf"|führung gelangenden
Vollzugsform geeinigt haben. Mehr noch weisen die opulenten Gesten des
Dirigenten in Richtung Orchester darauf hin, dass in diesen Proben auch
\emph{Sprache} generiert wurde, um die Ergebnisse längerer
Verständigungsprozesse in eine kompakte Form zu fassen, die den zeitkritischen
Tempi der Vollzugsform gewachsen ist.  Den Rahmen einfacher
ingenieur-technischer „Baumeisterarbeit“ sprengt Gabriela Montero, die
Solistin jenes Abends, mit ihrer Zugabe: Das Publikum wird aufgefordert, eine
Melodie vorzugeben, woraus die Virtuosin eine Improvisation als Vollzugsform
entwickelt, zu der es keine interpersonal kommunizierbare Beschreibungsform
gibt, wenn man einmal von den Tonaufzeichungen jenes Gewandhausabends und den
Berichten der begeisterten Hörerschaft absieht.  Dass auch hierfür technische
Meisterschaft erforderlich war, steht außer Frage.

Das Verhältnis der Menschen zu ihren technischen Systemen ist also komplex und
nur in einer dialektischen Perspektive der Weiterentwicklung bereits
vorgefundener technischer Systeme zu fassen, wenn man sich nicht unentrinnbar
in unfruchtbaren Henne-Ei-Debatten verstricken will.  Das relativiert aber
auch die Marxsche Forderung an die Philosophen, denn deren Interpretationen
sind die Differenzen zwischen den begründeten Erwartungen und den erfahrenen
Ergebnissen früherer Praxen vorgängig. Ob es ausreicht, diese Differenzen auf
der Ebene der Techniker, der Ingenieure oder der Fachwissenschaftler zu
besprechen oder eine Intervention der „interpretierenden Philosophen“ als
eigenständige Reflexionsdimension von Bedeutung ist, mag an dieser Stelle
offen bleiben.

\section{Systeme und Komponenten}

Neben der Beschreibungs- und Vollzugsdimension spielt für technische Systeme
auch der \emph{Aspekt der Wiederverwendung} eine große Rolle.  Dies gilt,
zumindest auf der artefaktischen Ebene, allerdings \emph{nicht} für die
meisten technischen Großsysteme -- diese sind \emph{Unikate}, auch wenn bei
deren Montage standardisierte Komponenten verbaut werden. Auch die Mehrzahl
der Informatiker ist mit der Erstellung solcher Unikate befasst, denn die
IT-Systeme, die derartige Anlagen steuern, sind ebenfalls Unikate.  Dasselbe
gilt auch für die Ämter, Behörden und öffentlichen Einrichtungen. So ist zum
Beispiel die Leipziger Stadtverwaltung aktuell damit befasst, ihre
Verwaltungsprozesse zu „digitalisieren“, was unter Führung des Dezernats
Allgemeine Verwaltung und zusammen mit dem städtischen IT-Dienstleister Lecos
erfolgt. Im Industriesektor ist deshalb deutlich zwischen Werkzeugmaschinenbau
und Industrieanlagenbau -- zwischen Ausrüstern sowie Planern und „Baumeistern“
entsprechender Unikate -- zu unterscheiden, auch wenn dies in einschlägigen
Statistiken \cite{VDMA2019} zum \emph{Maschinen- und Anlagenbau}
zusammengefasst wird.

Die Besonderheiten eines technischen Systems liegen damit vor allem im Bereich
des \emph{Zusammenspiels der Komponenten}. So unterscheiden sich
beispielsweise die Produktionsleitsysteme verschiedener BMW-Werke deutlich
voneinander \cite{Kropik2009}. Die Werke wurden zu verschiedenen Zeiten nach
dem jeweiligen Stand der Technik und dem sich ebenfalls verändernden
Geschäftsmodell des Unternehmens konzipiert. Einmal in die Welt gesetzt, sind
derartige technischen Großsysteme nur noch bedingt modifizierbar und werden
deshalb nach Ablauf entsprechender Amortisationsfristen auch konsequent außer
Betrieb gestellt. Gleichwohl spielt der Aspekt der Wiederverwendung auch bei
solch unterschiedlichen technischen Systemen eine Rolle, verschiebt sich aber
von der unmittelbaren Ebene der technischen Artefakte auf höhere Ebenen der
Abstraktion in der Beschreibungsdimension.

Damit sind wesentliche Elemente zusammengetragen, die eine erste Annäherung an
den \emph{Begriff eines technischen Systems} erlauben.  Der Begriff ist in
einem planerisch-realweltlichen Kontext vierfach überladen
\begin{itemize}
\item [1.] als realweltliches Unikat (z.B. als Produkt, auch wenn das Unikat
  ein Service ist),
\item [2.] als Beschreibung dieses realweltlichen Unikats (z.B. in der Form
  einer speziellen Produktkonfiguration)
\end{itemize}
und für in größerer Stückzahl hergestellte Komponenten auch noch
\begin{itemize}
\item [3.] als Beschreibung des Designs des System-Templates (Produkt-Design)
  sowie
\item [4.] als Beschreibung und Betrieb der Auslieferungs- und
  Betriebsstrukturen der nach diesem Template gefertigten realweltlichen
  Unikate (als Produktions-, Qualitätssicherungs-, Auslieferungs-, Betriebs-
  und Wartungspläne).
\end{itemize}
Besonders Punkt 4 spielt im TRIZ-Kontext kaum eine Rolle, obwohl davon
auszugehen ist, dass weder im privaten noch im unternehmerischen Umfeld
technische Produkte nachhaltig nachgefragt werden, für die absehbar
unzureichender Service angeboten wird. Insbesondere kann das TRIZ-Prinzip 27
der \emph{billigen Kurzlebigkeit anstelle teurer Langlebigkeit} nur als
Kompromiss zwischen Produzent und Nutzer funktionieren, ist also
sozio-technisch eingebettet.

Als Grundlage für einen derart abgrenzenden Systembegriff soll im Weiteren der
submersiv gefasste Begriff offener Systeme der Theorie dynamischer Systeme
\cite{Bertalanffy1950} verwendet werden, der
\begin{itemize}
\item [1.] eine innere Abgrenzung gegen vorgefundene Systeme (Komponenten), 
\item [2.] eine äußere Abgrenzung und funktional determinierte Einbettung in
  eine (funktionierende) Umwelt sowie
\item [3.] einen (funktionierenden) externen Durchsatz postuliert, der zu
  innerer Strukturbildung führt und damit die Leistungsfähigkeit des Systems
  bestimmt,
\end{itemize}
und seine Fruchtbarkeit für eine Behandlung mit mathematischen Instrumenten
seither vielfach unter Beweis gestellt hat.  

\emph{Technische Systeme} sind in einem solchen Kontext Systeme, auf deren
Gestaltung kooperativ und arbeitsteilig agierende Menschen Einfluss nehmen,
wobei \emph{vorgefundene} technische Systeme auf Beschreibungsebene durch eine
\emph{Spezifikation} ihrer Schnittstellen und auf Vollzugsebene durch die
\emph{Gewähr spezifikationskonformen Betriebs} normativ charakterisiert sind.

Wir bewegen uns dabei klar im Bereich der Standard-TRIZ-Terminologie eines
\emph{Systems von Systemen} -- ein technisches System besteht aus Komponenten,
die ihrerseits technische Systeme sind, deren \emph{Funktionieren} (sowohl im
funktionalen als auch im operativen Sinn) für die aktuell betrachtete
Systemebene vorausgesetzt wird. Die Rolle des Begriffs \emph{Objekt} und
dessen Abgrenzung gegen den Begriff \emph{Komponente} besprechen wir weiter
unten. 

Der Begriff eines technischen Systems hat damit eine klar epistemische
Funktion der (funktionalen) „Reduktion auf das Wesentliche“.  Einstein wird
der Ausspruch zugeschrieben „make it as simple as possible but not
simpler“. Das \emph{Gesetz der Vollständigkeit eines Systems} bringt genau
diesen Gedanken zum Ausdruck, allerdings tritt dieser dabei nicht als
\emph{Gesetz}, sondern als ingenieur-technische \emph{Modellierungsdirektive}
in Erscheinung.  Die scheinbare „Naturgesetzlichkeit“ der beobachteten Dynamik
ist also wesentlich an \emph{vernünftiges} (im Sinne von \cite{Vernadsky1997})
\emph{menschliches Agieren} gebunden.

Mit einem Ansatz der „Reduktion auf das Wesentliche“ sowie der „Gewähr
spezifikationskonformen Betriebs“ sind in diese Begriffsbildung inhärent
menschliche Praxen eingebaut, aus denen heraus die Begriffe „wesentlich“,
„Gewähr“ und „Betrieb“ überhaupt erst sinnvoll gefüllt werden können.  Eine
Unterscheidung zwischen technischen und sozio-technischen Systemen, die für
M. Rubin „offensichtlich und wesentlich“ (private Kommunikation) ist, wird
damit problematisch. Wesentliche Begriffe aus dem sozial determinierten
Praxisverhältnis von Menschen wie Ziel, Nutzen, Gewährleistung und
Verantwortung sind fest in die Begriffsgenerierungsprozesse der Beschreibung
konkreter technischer Systeme eingebaut und finden in den konkreten
gesellschaftlichen Setzungen eines primär rechtsförmig konstituierten
bürgerlichen Systems ihre „natürliche“ Fortsetzung.

\section{Die Welt der Technischen Systeme. Basics}

In der TRIZ-Literatur spielen solche begriff"|lichen Fundierungen kaum eine
Rolle.  Einschlägige Lehrbücher wie etwa \cite{KS2017} betrachten den Begriff
des \emph{technischen Systems} als intuitiv gegeben, der sich aus einer
„industriellen Praxis“ heraus \cite[S. 2]{KS2017} von selbst versteht, während
andere Begriffe wie „Prozess“, „Produkt“, „Dienstleistung“, „Ressourcen“ und
„Effekte“ \cite[S. 6--10]{KS2017} genauer eingeführt werden. Selbst die
ausführliche Beschreibung der „Evolution technischer Systeme“ in 5 Gesetzen
und 11 Trends \cite[Kap. 4.8]{KS2017} basiert allein auf der lapidaren
Feststellung „Die Existenz technischer Evolution ist eine zentrale Erkenntnis
der TRIZ“.  Auch \cite{TESE2018} bleibt in dieser Frage vage; im Vorwort von
B. Zlotin heißt es allein zum \emph{Zweck} von Betrachtungen der Evolution
ingenieur-technischer Systeme „humanity can achieve practically any realistic
goal, but certain priorities must be set to ensure the greatest possible
impact on the economy and human life. [\ldots] The powers of contemporary
science and technology as well as financial investment should be applied to
carefully selected and formulated objectives.“

Es ist natürlich möglich, in einem diskursiven Rahmen die verbale Fassung
eines Begriffs offen zu lassen und auf andere Weise -- etwa durch den Bezug
auf gemeinsame Praxen oder durch den „gewöhnlichen Gebrauch“ -- die Konvergenz
der Begriffsverwendung zu erreichen.  Ein solches Grundmuster wird im
TRIZ-Kontext für den Begriff \emph{technisches System} besonders auch in
\cite{TESE2018} angewendet, indem der Begriff durch eine Vielzahl von
Beispielen in Kombination mit den Begriffen „Muster“ und „Evolution“
illustriert, die genaue Fassung aber dem geneigten Leser überlassen wird.  Der
dort mittlerweile erfolgte Rückzug auf Begriffe wie „Muster“ oder „Trend“
gegenüber dem schärferen und wissenschaftspraktisch vorbelegten Begriff
„Gesetz“ unterstützt das Anliegen der Autoren von \cite{TESE2018}, empirische
Erfahrung zu systematisieren, verweist aber zugleich auf das schwache
theoretische Fundament eines solchen Systematisierungsanliegens.  Das weite
Spektrum praktisch kursierender Präzisierungen eines derart im Ungewissen
gelassenen Begriffs wurde in einer Facebook-Diskussion \cite{Graebe2019b} im
August 2019 deutlich. Für ein genaueres Abwägen der Argumente zu oben
formulierter These und Gegenthese ist ein solches Fundament allerdings nicht
ausreichend.

Wie kann der Begriff eines \emph{technischen Systems} also weiter geschärft
werden?  In unserem Seminar \cite{Graebe2020} haben wir „den Systembegriff als
Beschreibungsfokussierung identifiziert, mit der konkrete Phänomene durch
\emph{Reduktion auf das Wesentliche} [\ldots] einer Beschreibung zugänglich
werden.“  Die Reduktion richtet sich auf folgende drei Dimensionen
\cite[S. 18]{Graebe2020} 
\begin{itemize}
\item [(1)] Abgrenzung des Systems nach außen gegen eine \emph{Umwelt},
  Reduktion dieser Beziehungen auf Input/Output-Beziehungen und garantierten
  Durchsatz.
\item [(2)] Abgrenzung des Systems nach innen durch Zusammenfassen von
  Teilbereichen als \emph{Komponenten}, deren Funktionieren auf eine
  „Verhaltenssteuerung“ über Input/Output-Bezie"|hungen reduziert wird.
\item [(3)] Reduktion der Beziehungen im System selbst auf „kausal
  wesentliche“ Beziehungen.
\end{itemize}
Weiter wird ebenda festgestellt, dass -- ähnlich wie im Konzertbeispiel --
einer solchen reduktiven Beschreibungsleistung vorgefundene (explizite oder
implizite) Beschreibungsleistungen vorgängig sind:
\begin{enumerate}
\item[(1)] Eine wenigstens vage Vorstellung über die (funktionierenden)
  Input/Output-Leistungen der Umgebung.
\item[(2)] Eine deutliche Vorstellung über das innere Funktionieren der
  Komponenten (über die reine Spezifikation hinaus).
\item[(3)] Eine wenigstens vage Vorstellung über Kausalitäten im System
  selbst, also eine der detaillierten Modellierung vorgängige, bereits
  vorgefundene Vorstellung von Kausalität im gegebenen Kontext.
\end{enumerate}
Die Punkte (1) und (2) können ihrerseits in systemtheoretischen Ansätzen für
die Beschreibung der „Umwelt“ (hierfür ist allerdings die Abgrenzung eines
oder mehrerer Obersysteme in einer noch umfassenderen „Umwelt“ erforderlich)
sowie der Komponenten (als Untersysteme) entwickelt werden, womit die
Beschreibung von \emph{Koevolutionsszenarien} wichtig wird, die ihrerseits für
die Vertiefung des Verständnisses von Punkt (3) relevant sind.

Dabei ist der Fokus zunächst auf ein genaueres Verständnis des Begriffs
\emph{System} gerichtet, der als Reduktion von Komplexität in den drei oben
angeführten Richtungen betrachtet wird. Da in diesem Verständnis Komponenten
eines Systems selbst wieder Systeme sind, liegt auch im allgemeinen Fall die
Betrachtung eines Systems als „System von Systemen“ nahe, wie es in
\cite{Holling2000} thematisiert ist.  Wesentliches Reduktionskriterium für
Beziehungen zwischen Komponenten sind in solchen Systemen \emph{spezifische
  Eigenzeiten und Eigenräume} wie in den Abbildungen 1--3 in
\cite{Holling2000} dargestellt ist, die auch in den TRIZ-Prinzipien 18
\emph{Ausnutzung mechanischer Schwingungen}, 19 \emph{periodische Wirkung}, 23
\emph{Rückkopplung} und 25 \emph{Selbstbedienung} eine Rolle spielen.

Die Beschreibung von Planung, Entwurf und Verbesserung technischer Systeme
geht in einem solchen Ansatz von der Leistungsfähigkeit bereits vorhandener
technischer Systeme aus, die sowohl in (2) als Komponenten als auch -- aus der
Sicht eines Systems im Obersystem -- in (3) als benachbarte Systeme zu
berücksichtigen sind.

Ingenieur-technische Praxen bewegen sich damit in einer \emph{Welt von
  technischen Systemen}. Aus der konkreten Beschreibungsperspektive eines
Systems sind andere Systeme als Komponenten oder Nachbarsysteme allein in
ihrer \emph{Spezifikation} wichtig. Eine solche Reduktion auf das Wesentliche
erscheint praktisch als verkürzte Sprechweise über eine gesellschaftliche
Normalität, was ich kurz als \emph{Fiktion} bezeichne.  Diese Fiktion kann und
wird im täglichen Sprachgebrauch so lange aufrecht erhalten, so lange die
gesellschaftlichen Umstände die Aufrechterhaltung der daran gebundenen
gesellschaftlichen Normalität garantieren können, so lange also der
\emph{Betrieb der entsprechenden Infrastrukturen} gewährleistet ist.
Technische Systeme sind damit wenig"|stens in ihrer Vollzugsdimension
\emph{immer} sozio-technische Systeme.

Ein Ausblenden dieser sozialen Zusammenhänge kann sich also allenfalls auf die
\emph{Planung} derartiger Systeme sowie deren artefaktische Daseinsdimension
beziehen, die den \emph{Betrieb} der erforderlichen Infrastruktur ausblendet
oder in ein Obersystem verschiebt.  Letzteres ist aber unzweckmäßig, da das
Beheben von Problemen im Betrieb eines Systems Kenntnisse über dessen
Funktionieren nicht nur auf der Ebene der Spezifikation, sondern auch auf der
Ebene der Implementierung erfordern.

Eine solche Engführung des Begriffs \emph{technisches System} resultiert
möglicherweise aus spezifischen Praxen der Vermittlung und Weiterentwicklung
von TRIZ-Grundlagen, da TRIZ-Praktiker zu derartigen Auseinandersetzungen um
Fundierungen der von ihnen angewendeten Theorien ein entspanntes bis
ignorantes Verhältnis an den Tag legen. Für eine Theorie der Evolution
technischer Systeme ist aber eine noch weitergehende Engführung und
Abstraktion des Begriffs erforderlich, da sich die bisherige Begriffsgenese
ausschließlich an der zu einem gegebenen Zeitpunkt \emph{vorgefundenen}
Landschaft technischer Systeme orientiert.

\section{Zum Evolutionsbegriff in \cite{TESE2018}.\\ Die sozio-ökonomische
  Dimension} 

Um evolutionäre Aspekte zu thematisieren, ist eine Zuordnung von zu
verschiedenen Zeiten existierenden technischen Systemen zu
\emph{Entwicklungslinien} erforderlich. Hier gehen \cite{Shpakovsky2010} und
\cite{TESE2018} deutlich verschiedene Wege.

In \cite{TESE2018} wird \emph{Evolution}, wie V. Souchkov im Vorwort
\cite[S. IX]{TESE2018} feststellt, als „innovative development“ verstanden,
„since -- in contrast to nature -- craftsmen and engineers make decisions
based on logic, previous experience, and knowledge of basic principles rather
than chance.“ Die Konzentration auf „craftsmen and engineers“ weist noch
einmal auf die Engführung der Praxen hin, aus denen die Systematisierung in
\cite{TESE2018} abgeleitet wurde.

Mögliche Zugänge zur Einbettung des bisher entwickelten Begriffs in
historische Geneseprozesse könnten sich an der wissenschaftshistorischen
Betrachtung \cite{Weller2008} der Entwicklung der
\emph{Automatisierungstechnik} als eines der wichtigsten interdisziplinären
technikwissenschaftlichen Bereiche und damit \emph{aus der Perspektive der
  Praxen des Industrieanlagenbaus} oder der industriehistorischen Untersuchung
der Genese der \emph{Praxen der Produktion} realweltlicher technischer Systeme
orientieren, in denen Produktlinien \cite{Pohl2005}, Produktionsnetzwerke
\cite{Friedli2013} oder neuerdings auch technische Ökosysteme
\cite{Graebe2018} eine zentrale Rolle spielen. Mit dem Bereich des
\emph{Systems Engineering} existiert zudem ein aus der Informatik
hervorgegangenes umfassendes technikwissenschaftliches Forschungsgebiet mit
vergleichbaren Fragestellungen, dessen Grundlagen in einer internationalen
Norm ISO 15288 \emph{Systems and Software Engineering} dokumentiert sind.

Der Zugang in \cite{TESE2018} ist allerdings ein anderer -- zur
Identifizierung von Entwicklungslinien wird der Begriff \emph{technisches
  System} zwischen „technology push“ und „market pull“ als „simple means for
understanding the advancement of man-made systems“ eingebettet
\cite[S. 1]{TESE2018}. Der Bezug auf den noch ungenaueren Begriff „man-made
systems“ (nicht „men-made systems“!) wird im Weiteren genauer erläutert:
Innovation als „improvement of already-existing systems“ wird durch das
Fortschreiben wissenschaftlicher Erkenntnis angetrieben, aus der heraus neue
Systeme, Produkte und Dienstleistungen entstehen, die von einem „market pull,
the second trigger for innovation“ einem Formungs- und Ausleseprozess
unterworfen sind, „that stimulates the development of a system by meeting the
needs of that system's users“.  Die genaue Ausformung dieses nicht von
ingenieur-technischen, sondern von innovations-unternehmerischen Praxen
getriebenen Ansatzes wird in \cite[Kap. 3]{TESE2018} deutlich.  Die Gründe für
den universalistischen Anstrich des Vortrags der Erfahrungen hat S. Gerovitch
in \cite{Gerovitch1996} hinreichend genau analysiert, so dass dies hier
unberücksichtigt bleiben und auf die nüchterne Feststellung reduziert werden
kann, dass sich die Grundlagen dieser impliziten Begriffsbildungsprozesse im
Rahmen des sozio-technischen ökonomischen Systems einer kapitalistischen
Wirtschaftsordnung als Obersystem bewegen (westlicher Prägung füge ich hinzu,
da die Übertragbarkeit auf stärker autokratisch geprägte Wirtschaftsordnungen
wie etwa in China oder Russland zusätzliche Betrachtungen erfordern).  In
Wirklichkeit ist die Kontextualisierung noch enger gezogen, wie die Analyse
der Beispiele zeigt -- eine Unterscheidung zwischen Industrieanlagenbau,
Werkzeugmaschinenbau und Konsumgüterproduktion, wie sie etwa in
volkswirtschaftlichen Analysen üblich ist, wird nicht vorgenommen, gleichwohl
grundsätzlich die Perspektive einer am größeren Markt orientierten
Produktgängigkeit der untersuchten \emph{technischen Systeme} eingenommen.
Der Unikat-Charakter der überwiegenden Mehrzahl technischer Großsysteme und
damit die Praxen des Industrieanlagenbaus bleiben damit unberücksichtigt.

Damit ist der Gegenstandsbereich der technischen Systeme hinreichend umrissen,
deren „Evolution“ in \cite{TESE2018} untersucht wird. Zugleich wird M. Rubins
Position verständlich, dass in einem solchen Kontext die Unterscheidung von
technischen und sozio-technischen Systemen „offensichtlich und wesentlich“
ist, was M. Rubin (private Kommunikation) mit Verweis auf \cite{Rubin2007} und
\cite{Rubin2010} auch selbst klar sieht:
\begin{quote}
  Bei der Betrachtung eines technischen Systems berücksichtigen wir keine
  anderen bestehenden Beziehungen (soziale, politische, wirtschaftliche,
  Marketing usw.) im System, mit Ausnahme von Objekten und Beziehungen
  technischer Natur. Diese externen (menschlichen, kulturellen) Beziehungen
  können durch zusätzliche Anforderungen oder Einschränkungen an technische
  Objekte ersetzt werden.  Bei der Betrachtung von Systemen als
  sozio-technisch werden eine Reihe technischer Objekte und Zusammenhänge
  berücksichtigt, beispielsweise wenn die TRIZ-Analyse von
  Produktionsunternehmen nicht nur als technisches System (Maschinen, Geräte),
  sondern die Fabrik als sozio-technisches Objekt betrachtet wird:
  Bestellsystem und Marketing, Personalpolitik, Finanzen und die
  wirtschaftliche Lage des Unternehmens, Systeme der Entscheidungsfindung usw.
  Offensichtlich verändert dies den Gegenstand der Überlegungen und die
  Untersuchungsinstrumente grundlegend.
\end{quote}
Natürlich bedürfen die Begriffe „technisches Objekt“, „technischer
Zusammenhang“ und „Beziehung technischer Natur“, mit denen hier über die
Grenzen zwischen technischen und sozio-technischen Systemen hinweg vermittelt
werden soll, weiterer Präzisierung.

Kehren wir jedoch zu \cite{TESE2018} zurück und untersuchen genauer, auf
welcher Aggregationsgrundlage die „Evolution ingenieur-technischer Systeme“
untersucht wird.  Da wir inzwischen auch ein Obersystem identifiziert haben,
gegen welches die Ausführungen relatiert werden können, können wir die
TRIZ-Methodik selbst zur Rekonstruktion der Modellierung und damit zur Analyse
der begriffstheoretischen Fundierung von \cite{TESE2018} einsetzen.

Ausgangspunkt ist das sozio-ökonomische System einer industriellen
Produktionsweise. „Der Reichtum dieser Gesellschaften“, so beginnt Marx seine
Analyse eines solchen sozio-ökonomi"|schen Systems in \cite{MEW23}, „erscheint
als eine 'ungeheure Warensammlung', die einzelne Ware als seine
Elementarform. Unsere Untersuchung beginnt daher mit der Analyse der Ware.“
Auch wir starten mit diesem Begriff als einer hochgradigen Abstraktion.  Marx'
Arbeitswertheorie abstrahiert im Begriff der \emph{Ware} bekanntlich von
sämtlichen qualitativen Eigenschaften außer der einen, Produkt menschlicher
Arbeit zu sein\footnote{Marx folgt dabei der Begriffsentwicklungsmethodik von
  Hegel. Rainer Thiel \cite[S. 190]{Thiel2007} schreibt dazu: „Hegels
  Begriffsentwicklung beginnt mit dem 'Sein'. Und was tut der Dialektiker
  Hegel? Er entwickelt – mit der Sturheit eines Schelms, wie ein Computer –
  den Inhalt des 'reinen Seins': 'Sein, reines Sein, -- ohne alle weitere
  Bestimmung. In seiner unbestimmten Unmittelbarkeit ist es nur sich selbst
  gleich und auch nicht ungleich gegen Anderes, hat keine Verschiedenheit
  innerhalb seiner, noch nach außen. Durch irgendeine Bestimmung oder Inhalt,
  der in ihm unterschieden, oder wodurch es als unterschieden von einem Andern
  gesetzt würde, würde es nicht in seiner Reinheit festgehalten. Es ist die
  reine Unbestimmtheit und Leere. – Es ist nichts in ihm anzuschauen, wenn von
  Anschauen hier gesprochen werden kann; oder es ist nur dies reine, leere
  Anschauen selbst. Es ist ebensowenig etwas in ihm zu denken, oder es ist
  ebenso nur dies leere Denken. Das Sein, das unbestimmte, unmittelbare, ist
  in der Tat Nichts, und nicht mehr noch weniger als Nichts'.“ Im Gegensatz zu
  dieser reinen Gedankenakrobatik bezieht sich Marx aber immer auf die diesen
  Begriffsentwicklungen vorgängigen \emph{gesellschaftlichen Praxen},
  vgl. etwa die Feuerbachthesen 1--3, \cite{MEW3}.}.  Erst auf einer solchen
Abstraktionsebene werden konkrete Waren global austauschbar und konstituieren
damit einen globalen Markt als \emph{Verhältnis} -- Feld in der
TRIZ-Terminologie -- zwischen diesen Warenkonkreta, den \emph{Tauschwert}.

Darum geht es in \cite{TESE2018} allerdings nicht, sondern um funktionale
Qualitäten konkreter Warengruppen wie Waschmaschinen oder Federhalter. Das
allgemeine Konkurrenzverhältnis abstrakter Waren zerfällt dabei in konkretere
Konkurrenzverhältnisse einzelner Warengruppen auf Einzelmärkten, die in
\cite{TESE2018} als „market pull“ die Hauptfunktion des Werkzeugs „Markt“
sind, welche die Objekte „engineering systems“ zu „nützlichen Produkten“
umformen -- ich folge dabie argumentativ den TRIZ-Begriff"|lichkeiten von
\cite{TT}.  Mit der Marktgängigkeit von Produkten ist eine erste
Struktureinheit im Obersystem identifiziert -- konkrete Märkte, auf denen
\emph{konkrete} Waren mit \emph{spezifischen} funktionalen Eigenschaften --
\emph{Gebrauchswerten} in der Terminologie von Marx -- miteinander im
Wettbewerb stehen.  Die \emph{Funktion} Konkurrenzverhältnis des
\emph{Werkzeugs} Markt wird in \cite{TESE2018} seinerseits als \emph{Werkzeug}
mit Technologie formender Funktionalität betrachtet.  Dieser Gedanke soll nun
genauer entwickelt werden.

Jede Ware ist „ein Ganzes vieler Eigenschaften und kann daher nach
verschiedenen Seiten nützlich sein“ \cite[S. 49]{MEW23}. Jede konkrete Ware
ist damit selbst ein technisches System im oben entwickelten Verständnis, wenn
sie als durch ihre Spezifikation gegebenes Ensemble „nützlicher“
Funktionalitäten betrachtet wird. Dieses Ensemble nützlicher Eigenschaften
bestimmt aber auch die Möglichkeiten und Grenzen der Substituierbarkeit von
Waren im gesamtgesellschaftlichen technologischen Prozess\footnote{Diese
  Grenzen sind allerdings fließend, wie A. Kuryan in der Diskussion
  \cite{Graebe2019b} am Beispiel eines Hammers demonstriert, der zum
  Offenhalten einer Balkontür eingesetzt wird (\foreignlanguage{russian}{„Если
    ты решил с помощью молотка подпевать дверь на балконе, чтобы она не
    закрывалась, то ты создал решение.“}). Die Bedeutung derartiger
  Grenzüberschreitungen für die TRIZ-Analyse der Evolution technischer Systeme
  bedarf einer genaueren Untersuchung, die hier nicht geleistet werden kann.}.
Jene Grenzen führen zu einer Stratifizierung „des Markts“ in konkrete Märkte
für konkrete Warengruppen. Im sozio-ökonomischen Obersystem haben wir damit
zwischen dem Werkzeug-Template „Markt“ und konkreten realweltlichen
Ausprägungen dieses Werkzeugs zu unterscheiden. Diese realweltliche Struktur
der \emph{Technologiemärkte} ist der in \cite{TESE2018} ausgeführten
S-Kurven-Analyse vorgängig und wird dort implizit als gegeben vorausgesetzt.
Jeder solche Technologiemarkt ist durch ein spezifisches Bündel technischer
Funkionalitäten charakterisiert, wobei \cite{TESE2018} mit dem Ansatz des MPV
(main parameter of value) postuliert, dass sich ein solcher Markt um einen
speziellen technischen Parameter herum gruppiert, der für die Wertschöpfung
von besonderer Bedeutung ist.

Damit ist aber das Erfordernis einer weiteren Abstraktion verbunden, denn
konkrete Waren, im obigen Sinne als technische Systeme, als \emph{konkrete}
Bündel technischer Funktionalitäten verstanden, sind prinzipiell geeignet, auf
\emph{mehreren} derartigen Technologiemärkten gehandelt zu werden und werden
dies praktisch auch. Ein solcher Technologiemarkt wird auch weniger durch die
auf ihm gehandelten Waren bestimmt als durch die diese Waren produzierenden
Unternehmen. Darum geht es den Autoren von \cite{TESE2018} auch, wenn sie
Altschullers „S-curve analysis“ zu einer „pragmatic S-curve analysis“
weiterentwickeln.  Damit verschiebt sich aber das in \cite{TESE2018}
aufgerufene Abstraktionserfordernis von einem MPV als eigenständigem
Charakteristikum zur \emph{unternehmerischen Fähigkeit}, technische Artefakte
mit diesem MPV in angemessenem Preis-Leistungs-Verhältnis \emph{zu
  produzieren}.  Damit wird auch deutlich, dass sich auf jenen
Technologiemärkten zwar „Produkte voneinander unabhängig betriebner
Privatarbeiten“ begegnen, die Produzenten aber \emph{nicht} „erst in
gesellschaftlichen Kontakt treten durch den Austausch ihrer Arbeitsprodukte“
\cite[S. 87]{MEW23} in dem Sinne, dass das Spannungsverhältnis zwischen
begründeten Erwartungen und erfahrenen Ergebnissen der Konditionen
\emph{früheren} Austauschs ihrer Arbeitsprodukte die Dynamik jenes
Technologiemarkts bestimmt.  

Gegenstand technologischer Evolution sind damit aber diese technologischen
Produktionsbedingungen selbst. Dies wird auch in \cite{TESE2018} so gesehen,
denn die im Buch beschriebenen Handlungsoptionen beziehen sich auf die
Organisation entsprechender Innovationsprozesse in Unternehmen. Damit kann
aber das Obersystem in unserer TRIZ-Analyse der begriff"|lichen Grundlagen von
\cite{TESE2018} weiter eingeschränkt werden auf die strategischen
Führungsstrukturen von Unternehmen, in denen die Innovationsprozesse praktisch
gestaltet werden. Auch hierbei haben wir es mit der Dualität von
System-Template -- gängigen gesellschaftlichen Verfahrensweisen zur
Organisation von Innovationsprozessen -- und konkreten realweltlichen
System"|ausprägungen in den einzelnen Unternehmen zu tun. Die
\emph{Hauptfunktion} jener Strukturen im Unternehmen ist die Organisation des
Innovationsprozesses in enger Verbindung mit der allgemeinen
Geschäftsstrategie.  Dieser Prozess selbst wird vom strategischen Management
entschieden und verantwortet, das hierzu die \emph{widersprüchlichen
  Anforderungen} verschiedener Unternehmensteile (R\&D, Vertrieb, Finanzen,
Controlling, SCM, CRM) unter einen Hut zu bringen hat. Die in \cite{TESE2018}
zusammengetragenen Empfehlungen sind \emph{ein} Aspekt in diesem komplexen
Abwägungsprozess, eine Methodik zwischen „technology push“ und „market pull“
ist mit Blick etwa auf die Ausführungen in \cite{Preez2006}, wo es zentral um
das Konzept eines „Fugle Innovation Process“ geht, eher auf dem Niveau der
1960er Jahre anzusiedeln. Ebenda wird in Fig.~3 mit „state of the art in
science and technology“ neben den „needs of society and marketplace“ ein
weiteres Obersystem in Stellung gebracht, das auch  bei Patenterteilungen mit
den Begriffen „Stand der Technik“ und „Erfindungshöhe“ eine wichtige Rolle
spielt.

Wir haben damit bereits \emph{drei} sozio-technische Obersysteme (Ökonomie,
Innovationsmanagement, Wissenschaft und Technologie) mit jeweils eigenen
Begrifflichkeiten, Strukturen, Komponenten, Beschreibungs- und Vollzugsformen
identifiziert, die in der einen oder anderen Weise zur in \cite{TESE2018}
behandelten Thematik in Bezug stehen, ohne damit der uns interessierenden
Präzisierung des Begriffs \emph{technisches System} -- besonders auch in
Abgrenzung zum Begriff \emph{sozio-technisches System} -- näher gekommen zu
sein. Die TRIZ-Methodik empfiehlt in einem solchen Fall, den Zugang von einer
anderen Seite neu zu versuchen, vorher aber die Erkenntnisse aus dem
Fehlversuch zu fixieren.  Diesbezüglich sind drei Aspekte interessant:

\paragraph{1)}
Der Anspruch von \cite{TESE2018}, (auch) einen innovationsmethodischen Beitrag
zu leisten, setzt auf weitgehend überholten innovationsmethodischen Konzepten
auf, eine „pragmatic S-curve analysis“ muss ihre Passfähigkeit zu moderneren
innovationsmethodischen Konzepten erst noch unter Beweis stellen.

\paragraph{2)}
Die Argumentation in \cite{TESE2018} nimmt impliziten Bezug auf gängige
innovationsmethodische Theorieansätze, wie sie in \cite{Preez2006}
systematisch dargestellt sind.  Insbesondere \cite[Fig. 3]{Preez2006} geht gut
als Diagramm einer Root Cause Analysis durch mit \emph{Idea Generation} (A)
und \emph{Commercial Product} (B) als die beiden Hauptkomponenten und weiteren
Hilfskomponenten (Development, Manufacturing, Marketing and Sales).  Es gibt
zwei Root Cause Pfeile -- $(A) \longrightarrow (B)$, vermittelt durch „State
of the art in science and technology“, und $(B) \longrightarrow (A)$,
vermittelt durch „Needs of society and the marketplace“. Eine TRIZ Root Cause
Analysis, wenigstens in der in \cite[Kap. 4.7]{KS2017} beschriebenen Weise,
kann mit derartigen \emph{Rückkopplungsschleifen} nicht arbeiten, da diese
keine „Wurzel“ haben. Das wichtige TRIZ-Prinzip 25 der \emph{Rückkopplung} ist
an dieser Stelle nicht anschlussfähig.  Eine zweite terminologische Feinheit
ist anzumerken: „push“ und „pull“ beziehen sich im englischen Sprachgebrauch
auf verschiedene Enden des Kausalpfeils. Im Sinne des TRIZ-Funktionsmodells
\cite[Kap. 4.4]{KS2017} ist „push“ eine Funktion des Werkzeugs, „pull“ eine
Funktion des Zielobjekts, etwa nach dem TRIZ-Prinzip 13 der
\emph{Funktionsumkehr} zu verstehen.  Im Diagramm \cite[Fig. 3]{Preez2006}
werden diese beiden Begriffe offensichtlich in Bezug auf die Komponente
\emph{Commercial Product} als Zielobjekt verwendet.  Für das Zielobjekt
\emph{Idea Generation} dieses weitgehend symmetrischen Arrangements drehen
sich die Begriffe um -- push wird zum pull und umgekehrt.

\paragraph{3)}
Wir haben \emph{mehrere} Obersysteme identifiziert, womit noch einmal deutlich
wird, dass der Begriff \emph{Obersystem} nicht immersiv zu denken ist und
nicht mit dem Begriff \emph{Umwelt} verwechselt werden darf. Obersysteme sind
spezifische Nachbarsysteme mit eigener Sprache und Logik. Die Beziehung
Obersystem -- System ist dieselbe wie die Beziehung System -- Komponente und
oben hinreichend genau beschrieben: Es handelt sich um zwei verschiedene
Betrachtungsperspektiven auf die „Totalität der Welt“ mit zwei verschiedenen
Begriffen des \emph{Wesentlichen} und damit aus zwei verschiedenen
Reduktionsperspektiven. Ins Obersystem geht das System allein durch seine
Spezifikation (Beschreibungsdimension) und das Versprechen
spezifikationskonformer Leistung (Vollzugsdimension) ein. Im Konzertbeispiel
wird aus dem \emph{gegebenen} Können der Musiker auf der Ebene des Orchesters
die Interpretation des Musikstücks „produziert“.  Die
Leistungs\emph{fähigkeit} der Teilsysteme ist dabei dem
Leistungs\emph{potenzial} des Obersystems vorgängig -- die Interpretation von
Mozarts Klavierkonzert KV 491, die Alexander Shelley mit dem Leipziger
Gewandhausorchester und Gabriela Montero als Solistin vorgelegt hat, wäre mit
einem Laienorchester nicht möglich gewesen. Aus der Perspektive des Systems
agiert das Obersystem ebenfalls funktional: Die Schnittstelle des Systems
definiert in der Beschreibungsdimension eine erwartete Input- oder
Durchsatzleistung in Quantität, Qualität und Struktur, die dem Funktionieren
des Systems (zur Laufzeit) vorgängig ist, das Obersystem stellt in der
Vollzugsdimension diese Voraussetzungen sicher. Eine solche Strukturierung
bringt die aus der Mathematik bekannten deduktiven Wenn-Dann-Beziehungen mit
realweltlichen Prozessen in Verbindung, wobei der modus ponens in
realweltlichen Systemen eine deutlich komplexere Semantik hat als in der
Mathematik, wo dieser bekanntlich ausschließlich als logische Klammer eine
Rolle spielt, wenn aus kleineren Wenn-Dann-Beziehungen komplexere solche
Beziehungen (Lemmata, Sätze, Theoreme bis zum Beweis eines der
Millenium-Probleme) regelgerecht deduziert werden sollen.  Auch hier hat
dieser Reduktionsmechanismus einen klaren Zweck -- arbeitsteilige Reduktion
von Komplexität.

\section{Zum Evolutionsbegriff in \cite{TESE2018}.\\ Die ideengeschichtliche
  Dimension}

Kehren wir jedoch zu \cite{TESE2018} und unserer Frage nach der Fundierung des
Begriffs \emph{technisches System} zurück. Dem Rat der TRIZ-Methodik folgend,
es mit einem anderen Ansatz zu versuchen, wenden wir uns dem „technology push“
\cite[S. 2]{TESE2018} zu, „that creates new systems, products, and services“
(HGG -- indeed „creates“?) „that are not yet required by the market“, um aus
dieser Perspektive die systemischen Begriff"|lichkeiten zu entwickeln.
Zunächst sind Zweck und Nutzen eines solchen Systems zu bestimmen.

Aus der Perspektive des „Coupling Model of Innovation“ in
\cite[Fig. 3]{Preez2006} ist dieses System als „Ideengenerator“ die sprudelnde
Quelle, deren Produkte das formende Flussbett des „state of the art in science
and technology“ durchfließen, um sich an dessen Ende durch einen „technology
push“ in „commercial products“ zu verwandeln.  Die „Ideenquelle“ wird (ebenda)
durch einen „market pull“ angetrieben, der aus den „needs of society and the
marketplace“ gespeist wird. Wie oben bereits erläutert sind diese Termini mit
der TRIZ-Welt von „Werkzeug“, „Aktion“, „Objekt“ und „Produkt“ sowie dem
Konzept der Systemtransformation etwa auch nach \cite{TT} relatierbar, wären
als solche aber nicht Teil der Analyse des uns interessierenden Systems,
sondern des Obersystems „Innovationsmanagement“. In der Tat erfüllt dieses
Begriffs"|universum dessen Reduktionserfordernis „auf das Wesentliche“, in
welches die „Ideenquelle“ als Komponente allein mit ihrer funktionalen
Spezifikation eingeht, von deren genauem inneren Funktionieren beim
Innovationsmanagement -- jedenfalls im Verständis von \cite{Preez2006} --
abstrahiert wird.

Das ist in \cite{TESE2018} natürlich anders, denn mit der Untersuchung der
Evolution ingenieur-technischer Systeme soll ja gerade die ideengeschichtliche
Dimension derartiger Prozesse untersucht werden.  Zu diesem Zweck heißt es
bereits in der Überschrift von Kapitel 1 „Technology Pull: Beyond Technology
Push and Market Pull“.  Wie ist das zu verstehen? Wird der Fokus auf die
Komponente \emph{Idea Generation} gerichtet? Dann wären in der Tat die
Begriffe „push“ und „pull“ gegenüber \cite{Preez2006} zu vertauschen -- aber
von „market push“ ist keine Rede.  Stattdessen wird in den weiteren
Ausführungen in \cite{TESE2018} deutlich, dass in der Tat ein \emph{Umdrehen}
des oberen Pfeils gemeint ist, die Komponente \emph{Idea Generation} des
Obersystems (!) Innovationsmanagement Input aus \emph{zwei} Richtungen
bekommt, die beide vom \emph{Commercial Product} als Werkzeug ausgehen.

So deutlich wie hier formuliert wird in \cite{TESE2018} allerdings auf die
innovationstheoretischen Vorstellungen von \cite{Preez2006} nicht Bezug
genommen.  Wir folgen einer solchen weiteren Modellreduktion und wollen den
Leser dabei zugleich von einem Dilemma befreien -- dem Zirkelschluss zwischen
den Komponenten \emph{Idea Generation} und \emph{Commercial Product}.  Dazu
lassen wir die beiden nun auf \emph{Idea Generation} gerichteten Pfeile in den
Komponenten „Wissenschaft und Technik“ und „Bedürfnisse/Markt“ starten und
überlassen das \emph{Commercial Product} der Teilkomponente \emph{Marketing
  and Sales} der Komponente \emph{Unternehmen}.

Unsere anstehende TRIZ-Analyse der inneren Struktur und Funktionsweise des
Systems mit dem provisorischen Namen „Ideengenerator“ ist auf die Erforschung
der Beziehung zwischen den drei \emph{Komponenten}\footnote{Oder
  „Nachbarsysteme“?  -- Nein. Nachbarsysteme wären alle vier nur aus der
  Perspektive eines -- bisher allerdings fiktiven -- Obersystems.} „Bedarf“,
„Technologie“ und „Unternehmen“ auszurichten, die -- nach dem oben
entwickelten Systembegriff -- mit ihren \emph{gegebenen} funktionalen
Spezifika Input, Output und Durchsatz die innere Struktur des zu
beschreibenden Systems bestimmen.  Ein solcher Ansatz passt deutlich besser
auf die Ausführungen in \cite[Kap. 1]{TESE2018}, denn die Autoren betonen
„parameters of the technology of the system are linked to parameters of the
market“ -- der Einfluss der beiden Komponenten wird also als nicht unabhängig
voneinander betrachtet, sondern lässt sich auf der Ebene von
Parameterkopplungen fassen (so jedenfalls die Autoren).

Wir sehen an diesem Zugang ein grundsätzliches Verfahren, Zirkel in
Kausalketten aufzulösen: die beiden relationalen -- also feldartigen im Sinne
einer Stoff-Feld-Modellierung wie in \cite[Kap. 4.9]{KS2017} genauer
entwickelt -- Beziehungen $(B)\stackrel{F_1}{\longrightarrow} (A)$
(„technology pull“) und $(B) \stackrel{F_2}{\longrightarrow} (A)$ („market
pull“) im Obersystem \emph{Innovationsmanagement} erscheinen im System (A) als
Komponenten $(F_1)$ und $(F_2)$, zwischen denen ein Feld
$(F_1)\stackrel{B}{\longleftrightarrow}(F_2)$ postuliert wird, um die für das
System \emph{wesentlichen} Beziehungen zwischen $(F_1)$ und $(F_2)$ zu
modellieren, die sich aus der Existenz der (fernen, da nicht ins System
übernommenen) Komponente (B) im Obersystem ergeben.  Einen solchen für den
Umgang mit zyklischen Kausalitätsverhältnissen wichtigen Standard, der
allerdings in den 76 TRIZ-Standards fehlt, hatte ich in \cite{Graebe2019a} als
\emph{Stoff-Feld-Swap} bezeichnet.  So ist wohl auch M. Rubin zu verstehen,
wenn er anmerkt, dass „externe (menschliche, kulturelle) Beziehungen durch
zusätzliche Anforderungen oder Einschränkungen an technische Objekte ersetzt“
und damit in technische Systeme importiert werden können, ohne diese zu einem
sozio-technischen System „aufzublasen“.

Doch kehren wir zur Rekonstruktion des Begriffssystems von \cite{TESE2018}
zurück.  Da es nicht über"|haupt um die Evolution technischer Systeme geht,
sondern um die sich aus der TRIZ-Theorie ergebenden Schlussfolgerungen für
eine solche Evolution, ist -- dem TRIZ-Standard 1.1.1 \emph{Vervollständige
  ein unvollständiges Stoff-Feld-Modell} folgend -- als weitere Komponente der
\emph{TRIZ-Theoriekörper} zu berücksichtigen. Auch dieser Korpus ist als
Komponente zu modellieren, da dieser in \cite{TESE2018} in seiner Gesamtheit
als nicht weiter zu hinterfragende Einheit auftrtt, also allein mit seiner
Input/Output-Leistung in die Systemmodellierung eingeht.  Außerdem
unterscheiden sich die MPV beider Dimensionen deutlich -- während im
TRIZ-Theoriekörper das \emph{Lösen von Problemen} im Fokus steht, geht es in
\cite{TESE2018} um die \emph{Evolution in der „Welt der technischen Systeme“}.

Diese Welt der technischen Systeme spielt in allen 4 Komponenten eine Rolle
und steht so in der spezifischen Vermittlerrolle eines \emph{Objekts}.  Was
unterscheidet Komponenten von Objekten? Ehe wir uns dieser Frage zuwenden, sei
allerdings festgehalten, dass wir uns mit der bisherigen Begriffsbildung auf
der Abstraktionsebene des Technikbegriffs des VDI (Verein Deutscher Ingenieure
-- der deutschen Standesorganisation der Ingenieure) befinden, der in der
VDI-Richtlinie 3780 den Technikbegriff in den folgenden drei Dimensionen
fasst:
\begin{itemize}
\item Menge der nutzenorientierten, künstlichen, gegenständlichen Gebilde
  (Artefakte oder Sachsysteme);
\item Menge menschlicher Handlungen und Einrichtungen, in denen Sachsysteme
  entstehen und
\item Menge menschlicher Handlungen, in denen Sachsysteme verwendet werden.
\end{itemize}
Die Definition vermeidet den Begriff „technisches System“ zugunsten von
„nutzenorientierten, künstlichen, gegenständlichen Gebilden“, wobei neben der
artefaktischen Dimension auch noch „Sachsysteme“ einbezogen werden -- neben
der Maschine also auch noch die Maschinerie\footnote{Marx geht hier noch
  weiter: Das „\emph{automatische System der Maschinerie} [\ldots] verwandelt
  die Maschinerie erst in ein System“ \cite[S. 584]{MEW42}.} und damit die
Unikate technischer Großsysteme --, allerdings wird der These, dass eine
Unterscheidung zwischen technischen und sozio-technischen Systemen
„offensichtlich und wesentlich“ sei, mit der unmittelbaren Verknüpfung der
„gegenständlichen Sachsysteme“ und den Bedingungen und Folgen ihrer Entstehung
sowie Verwendung bereits im Grundansatz widersprochen.

\section{Was sind Komponenten?}

Auf diese Frage hat \cite{Szyperski2002} eine einfache Antwort: „Components
are for Composition“.  Diese Definition folgt dem oben entwickelten
Verständnis, dass Systeme aus \emph{bereits vorhandenen} Komponenten
zusammengesetzt werden, wobei neben OTS-Komponenten (off the shelf) auch
selbst entwickelte Komponenten eingesetzt werden können. Allerdings müssen vor
der Integration zum (lauf"|fähigen) Gesamtsystem diese Entwicklungen
abgeschlossen sein. Im Wasserfallmodell der Softwareentwicklung wird diese
sequenzielle Vorgehensweise expliziert, agile Vorgehensmodelle sind flexibler,
erfordern zur frühzeitigen prototypischen Demonstration von
Teilfunktionalitäten aber auch Platzhalterkomponenten -- Mock-Komponenten und
provisorische Oberflächen.

In \cite{Szyperski2002} zerfällt mit diesem Verständnis die Welt der
Produktion technischer Systeme in zwei Teilwelten -- „design to component“ und
„design from component“. Ersteres ist das Gebiet der Komponentenentwickler mit
dem Fokus, Komponenten mit einer speziellen Fachfunktion („core concern“ --
dies entspricht dem MPV in \cite{TESE2018}) zu entwickeln. Neben dieser
Fachfunktion muss die Komponente aber noch eine größere Menge von
Hilfsfunktionen (Logging, Datensicherheit, Zugangsmanagement,
Druckeransteuerung usw. -- die „cross cutting concerns“) erfüllen, die auf
etablierte Konzepte (Beschreibungsdimension) und andere zu integrierende
Komponenten (Vollzugdimension) zurückgreifen.  All diese Beschreibungsformen
muss der Komponentenentwickler beherrschen, um nützliche Komponenten zu bauen.
Zweiteres ist das Gebiet der Komponentenassembler. Diese bauen (vorher zu
entwerfende) Systeme aus verfügbaren Komponenten zusammen, entwickeln oder
modifizieren weitere Hilfsfunktionalitäten (den „glue code“), integrieren und
testen das Gesamtsystem, bevor es beim Kunden zum Einsatz kommt.

Die Schnittstelle zwischen beiden Professionen bildet das verwendete
\emph{Komponenten-Frame"|work} wie etwa Spring Boot, das nicht nur durch
Normungen und Standardisierungen das prinzipielle Zusammenwirken der
Komponenten auf einer höheren Ebene der Abstraktion beschreibt
(Beschreibungsdimension), sondern auch von verschiedenen Anbietern als
Laufzeitsystem für Komponenten (Vollzugsdimension) zur Verfügung gestellt
wird. Derartige Laufzeitsysteme -- sicher ebenfalls technische Systeme --
haben eine Spezifik: sie werden \emph{gemeinsam} von Anbieter und Kunde
betrieben, was eine Koordination der sozio-technischen Begleitprozesse (die
nach M. Rubin \emph{nicht} zum technischen System zu rechnen sind) auf hohem
Niveau erfordert. In der Informatik hat sich dabei ein System von
Serviceleveln bewährt, die vertraglich vereinbart werden und die Verantwortung
zwischen Anbieter und Kunde verteilen. Bei drei Serviceleveln liegt gewöhnlich
die Verantwortung für die Auswahl und Schulung des Personals für die
Fachanwendung (Level 1) komplett beim Kunden, die Konfiguration und
Rekonfiguration des Laufzeitsystems beim Kunden (Level 2) wird entweder vom
Anbieter (im Rahmen eines „Produktlinien-Managements“) oder einer
spezialisierten Fachabteilung des Kunden übernommen, Wartungen, Updates des
Systems sowie Integration oder Reintegration neuer Komponenten (Level 3) liegt
in den Händen des Anbieters. In einem solchen System liegt eine arbeitsteilige
Situation vor -- der Anbieter ist für die Qualität der Funktionalität
zuständig, der Kunde für die Qualität der Daten. Außerdem verantwortet der
Kunde die Funktionen und Fehlfunktionen des Systems vor der Allgemeinheit und
muss sich entsprechende Schäden, die durch den Betrieb des Systems verursacht
wurden, im Rahmen eines Anscheinsbeweises zunächst rechtlich zuordnen lassen.
Die operative (technische) Qualität des realweltlichen Systems wird in diesem
sozio-technischen Verhältnis von \emph{beiden} Parteien gleichermaßen
beeinflusst, so dass eine sinnvolle Trennung von technischer und
sozio-technischer Ebene \emph{praktisch} unzweckmäßig ist und so auch nicht
erfolgt.

\section{Normierung und Standardisierung}

Dieses Vorgehen in der Softwarebranche ist auch in vielen
ingenieur-technischen Anwendungen präsent. „Baukastensysteme“ sind weit
verbreitet und erlauben es, realweltliche technische Unikat-Systeme auf
dieselbe Weise zu entwerfen wie im Konzertbeispiel erläutert. Während dort
aber auf das \emph{private Verfahrenskönnen} der Musiker als Voraussetzung
abgestellt wurde, werden hier die \emph{Logik der Fachanwendung} als „core
concerns“ der Komponenten mit der \emph{Logik der Vernetzung} der
Infrastruktur als „cross cutting concerns“ zusammengeführt. Beide Logiken sind
orthogonal zueinander, was die Trends 4.2 „of increasing system completeness“
und 4.4 „of transition to the supersystem“ in ihrer separaten Betrachtung in
\cite{TESE2018} entwertet und zu folgender weiteren \emph{Gegenthese} Anlass
gibt:
\begin{quote}\it 
  \textbf{Gegenthese 2:} Ein besseres beschreibungstechnisches Verständnis der
  Infrastrukturanforderungen miteinander agierender Komponenten (Übergang zum
  Obersystem) führt zu einer \emph{Abschwächung} der Anforderungen an die
  Vollständigkeit der einzelnen Komponenten.
\end{quote}
Insbesondere die zur Begründung von Trend 4.2 in \cite{TESE2018} angeführte
Hierachisierung in „operating agent“ (als Kernfunktion), „transmission“
(Unterstützung durch ein Arbeitsmittel), „energy source“ (Einsatz von
Naturkräften) und „control system“ (Einsatz von Steuerung als prozesshaftem
technischen Teilsystem und Quelle des „Trends der Dynamisierung“) sind von
diesen Entwicklungen affektiert, wie ein Besuch im Baumarkt unmittelbar zeigt
-- die Maschinensysteme namhafter Hersteller konzentrieren sich auf die
Bereitstellung der Energie, über entsprechende APIs (etwa Klett-, Schraub-
oder Klickverschlüsse auf mechanischer Ebene) können passende Werkzeuge mit
der Energiemaschine gekoppelt werden\footnote{Wobei durch Fortschritte der
  Materialwissenschaften insbesondere mit Klettverschlüssen eine massive
  Rückkehr zu \emph{mechanischen} Kopplungsprinzipien entgegen dem
  TRIZ-Prinzip 28 des \emph{Austauschs mechanischer Wirkschemata} zu
  verzeichnen ist.}, wobei je nach Geschäftsstrategie der namhaften Hersteller
der jeweilige Technologie-Teilmarkt „passender Gerätschaften“ monopolisiert
oder auch für weniger namhafte Hersteller von passenden Arbeitsmitteln
geöffnet ist. In beiden Fällen spielen \emph{Normierung und Standardisierung}
in dieser „Welt der technischen Systeme“, also inhärent sozio-technische
Prozesse, eine deutlich größere Rolle als die Weiterentwicklungen der rein
technischen Artefakte. Gleiches gilt auf der Ebene der Kontrollsysteme, wo
programmierbare Universalsteuerungen wie die UVR 1611 der Firma Technische
Alternative das „Herz“ vieler technischer Regelungen im Smart-Home-Bereich
bilden.

Ein solcher Normierungsprozess öffnet zugleich ökonomische Skaleneffekte für
Standardkomponenten, d.h. für sich in Richtung „idealer Endresultate“
etablierende Umsetzungskonzepte, die sich kosten\emph{senkend} pro Einzelstück
auswirken.  Die S-Kurve endet also nicht unbedingt -- und wohl auch eher
selten -- mit der Außerdienststellung im Stage 4 \cite[S. 38]{TESE2018},
sondern geht auf dem Höhepunkt ausgereifter \emph{technischer} Qualität
(einschließlich Normierung und Standardisierung) in eine Phase der
\emph{Ubiquität} über, in der die \emph{immer geringeren} ökonomischen
Aufwendungen für die Verfügbarkeit dieses „Stands der Technik“ die
Leitfunktion der weiteren Entwicklung übernehmen.

Der Trend 4.1 „of increasing (technical) value“ schlägt dabei in einen Trend
„of decreasing economic value“ um, oder -- um es mit ökonomischen Termini
auszudrücken -- der vorher durch die Nachfrage getriebene Markt geht in einen
vom Angebot getriebenen Markt über: Derselbe (reife) Gebrauchswert hat einen
immer geringeren Tauschwert.  Damit geht der Wert der „Idealität“
\cite[Kap. 4.1.1]{KS2017} in der Tat durch die Decke, aber als Folge eines
\emph{ökonomischen} Gesetzes.  Dies korrespondiert zum TRIZ-Prinzip 17 des
\emph{Übergangs zu anderen Dimensionen} und soll hier als weitere Gegenthese
festgehalten werden:
\begin{quote}\it
  \textbf{Gegenthese 3:} Der (technische) Trend 4.1 „of increasing (technical)
  value“ schlägt im Stage 3 der S-Kurven-Entwicklung um in einen
  (ökonomischen) „Trend of decreasing (economic) value“.
\end{quote}
Damit wechselt im Stage 3 in der Produktion gängiger Werkzeuge und
Standardkomponenten die Leitfunktion (MPV) der weiteren Entwicklung von den
technischen Triebkräften zu den ökonomischen. Diesen Prozess der
„Commodification“ hat F. Naetar in \cite{Naetar2005} hinreichend beschrieben;
das Thema muss hier also nicht vertieft werden. Diese Entwicklung ist
allerdings selbst widersprüchlich, wie in der marxistischen Literatur am
Phänomen der \emph{tendenziell fallenden Profitrate} diskutiert wird:
Geringere Produktionskosten durch technischen Fortschritt eines Produzenten
erhöhen dessen Profitrate im Vergleich zu den Konkurrenten. Der Marktpreis
(„decreasing economic value“) wirkt allerdings regulierend und senkt
perspektivisch die Profitrate der Wettbewerber, die diesen technischen
Fortschritt nicht implementieren.

Das TRIZ-Prinzip 17 des \emph{Übergangs zu anderen Dimensionen}, auf das oben
Bezug genommen wurde, erscheint hier allerdings -- im Gegensatz zur Lesart in
der Komponente TRIZ-Theoriekörper -- nicht als \emph{abstraktes Designmuster},
sondern als \emph{abstraktes Evolutionsmuster}, also nicht als Mittel der
aktiven Beeinflussung eines Problemlöseprozesses, sondern als
passiv-beobachtendes Beschreibungsmuster realweltlicher Entwicklungen.  In
diesem Sinne kann aber auch \emph{jedes andere} der TRIZ-Prinzipien sowie auch
jeder der TRIZ-Standards als abstraktes Evolutionsmuster formuliert
werden. Umgekehrt erscheinen die Evolutionstrends in der Komponente
TRIZ-Theoriekörper als weitere abstrakte Designmuster, die neben die
„Prinzipien“ und die „Standards“ treten.
\begin{quote}\it
  \textbf{Gegenthese 4:} Jedes der TRIZ-Prinzipien und jeder der
  TRIZ-Standards kann auch überzeugend als „Trend der Evolution technischer
  Systeme“ formuliert werden und umgekehrt. 
\end{quote}
Die Hierarchie der Evolutionsmuster gibt damit insbesondere Anlass zu einer
„Hierarchie der Lösungsprinzipien“ \cite[Kap. 3]{Zobel2016}, wie Dietmar Zobel
bereits vor über 10 Jahren vorgeschlagen hat, siehe auch \cite{Zobel2020}.
Damit wird zugleich die Bedeutung der „Matrix“ entwertet. Leonid Shub
\cite{Shub2006} weist darauf hin, dass dies auch Altshuller bereits 1985
engeren Vertrauten gegenüber so geäußert habe.

Mit dem Verzicht auf die Betrachtung sozio-ökonomischer Zusammenhänge bleibt
die Bedeutung von Prozessen der Normierung und Standardisierung in
\cite{TESE2018} ausgeblendet. Damit versperren sich die Autoren aber selbst
den Blick in eine lebendige Welt technischer Systeme in einem
fortgeschrittenen Zustand der Evolution. So zeichnet sich das technische
System der \emph{Schraubverbindungen} durch eine Massenproduktion genormter
Maschinenschrauben und Holzschrauben aus.

Für die Herstellung von Maschinenschrauben ist hohe Präzision und Stimmigkeit
von Durchmesser und Anstellwinkel der Gewinde erforderlich, damit diese mit
den Gegenstücken zusammenpassen.  Diese Präzision wird nicht nur durch eine
industrielle Herstellungsweise erreicht, sondern für Spezialanwendungen auch
mit entsprechenden Werkzeugen -- einem Gewindeschneider.  Mit Schlitz-,
Kreuzschlitz-, Sechskant-, Senkkopf-, Inbus- usw. -schrauben gibt es ein
großes Sortiment vorgefertigter Lösungen für verschiedene Einsatzszenarien
(TRIZ-Prinzip 3 der \emph{lokalen Qualität}), dazu entsprechende Werkzeuge:
Schraubenschlüssel, Steckschlüssel, Schraubendreher, Inbus-Schlüssel
usw. (noch einmal TRIZ-Prinzip 3), sowohl als Einzelwerkzeuge wie auch als
Einsätze für den Akkuschrauber als Energiemaschine (TRIZ-Prinzip 1 der
\emph{Zerlegung}, TRIZ-Standard 3.1 \emph{Übergang zu einem Bi-System}).
Biegsame Schraubendreher\footnote{Amazon bietet ein solches 31-teiliges Set
  der Firma Lotex GmbH für 20,99 Euro an.} (zusammen mit dem Akkuschrauber
TRIZ-Standard 3.1 \emph{Übergang zu einem Poly-System}) können verwendet
werden, um Schraubverbindungen auch an schwer zugänglichen Stellen einzusetzen
usw.  Diese Werkzeuge werden auch von Industrierobotern eingesetzt
(TRIZ-Standard 3 \emph{Übergang zu einem Obersystem} zweimal angewendet, denn
die Industrieroboter sind Komponenten im Ober-Ober-System).

Die Welt der Holzschrauben vermeidet das Zwei-Komponenten-System (noch einmal
TRIZ-Standard 3.1: Schraube und Mutter), indem der Halt im zu bearbeitenden
Material selbst gesucht wird (Trend 4.6 \emph{of increasing degree of
  trimming} -- wieso gehört diese zentrale TRIZ-Methode weder zu den
„Prinzipien“ noch zu den „Standards“?), entweder durch Vorbohren (TRIZ-Prinzip
10 \emph{der vorherigen Wirkung}) oder durch eine selbstschneidende Schraube
(TRIZ-Prinzip 25 der \emph{Selbstbedienung} oder auch wieder Trend 4.6 des
\emph{Trimmens}). Leider bieten manche Materialien diesen Halt nicht, es
kommen \emph{Dübel} zum Einsatz (TRIZ-Nicht-Trend des \emph{Anti-Trimmens}),
inzwischen eine eigene Welt technischer Lösungen, die das Herz jedes
TRIZ-Praktikers höher schlagen lässt.  Und da haben wir noch nicht über
spezielle Anwendungen von Schraubverbindungen wie in der Chirurgie gesprochen,
wo wesentliche Parameter an Material und Zuverlässigkeit aus den Bedingungen
des Obersystems zu sehr speziellen Systemlösungen führen.

Ich habe diese Welt so ausführlich beschrieben, um drei Aspekte zu
verdeutlichen:
\begin{itemize}
\item [1)] Es ist eine Welt technischer Systeme, in der Prinzipien des
  Problemlösens auf TRIZ-Basis eine wichtige Rolle spielen.
\item [2)] Das strukturierende Moment in jener Welt sind nicht die technischen
  Systeme, sondern das \emph{technische Prinzip}.
\item [3)] Die 10 „Trends“ sind wenig hilfreich, um sich in dieser Welt
  hochvolatiler Anforderungssituationen zurecht zu finden, weil stets nach
  \emph{konkreten} Lösungen in \emph{konkreten} Kontextualisierungen gefragt
  wird, und dabei nicht die „Evolution einzelner technischer Systeme“ eine
  Rolle spielt, sondern ein globaler \emph{Stand der Technik}, in dem sich die
  „Evolution der Welt der technischen Systeme“ als Ganzes spiegelt. 
\end{itemize}
Karl Steinbuch nimmt diese Perspektive einer Evolution der Welt der
technischen Systeme als Ganzes bereits vor 50 Jahren ein, wenn er in
\cite[S. 7]{Steinbuch1966} schreibt: „Wo sich das geschichtliche Interesse
jedoch der Naturwissenschaft und der Technik zuwendet, kann die Realität des
Fortschritts nicht geleugnet werden. Man kann hier den Fortschritt präzise
erklären: Er besteht darin, dass im fortgeschritteneren Zustand nicht nur die
früheren Einsichten vorhanden sind und die früheren technischen Leistungen
vollbracht werden können, sondern darüber hinaus auch noch neue, zusätzliche.
In der Geschichte der Naturwissenschaft und Technik ist der Fortschritt nicht
eine bestreitbare Fiktion, sondern die Vermehrung registrierbarer Leistungen.“
Einer solchen Fortschrittsgläubigkeit ist zwar ebenfalls zu widersprechen
(siehe dazu \cite{Graebe2012}), das Argument zeigt aber noch einmal, dass der
Ansatz des VDI den \emph{realen} Kausalitäten der Evolution technischer
Systeme näher steht als der Ansatz der Autoren von \cite{TESE2018}.

Nun soll an dieser Stelle nicht das Kind mit dem Bade ausgeschüttet werden,
denn die „Weisheiten“ des letzten Abschnitts haben sich aus der Inspektion
eines Bereichs der Evolution technischer Systeme ergeben, die in
\cite{TESE2018} nicht untersucht werden.  Insofern hat unser Ergebnis den
Charakter einer \emph{partiellen TRIZ-Lösung} und es bleibt die Frage zu
beantworten -- worüber argumentieren die Autoren von \cite{TESE2018}? Wir
hatten gesehen, dass die Autoren -- im Gegensatz zu N. Shpakovsky
\cite{Shpakovsky2010} -- die Entwicklungslinien \emph{nicht} um technische
Prinzipien herum bauen, sondern nur Entwicklungsbereiche betrachten, wo eine
solche Ablösung technischer Prinzipien von technischen Systemen noch nicht
erfolgt ist.  Wir diagnostizieren hier widersprüchliche Beschreibungsformen
von zwei klar voneinander getrennten Welten, so dass sich die Frage aufdrängt,
ob dieser Widerspruch mit dem TRIZ-Prinzip 36 der \emph{Anwendung von
  Phasenübergängen} zu lösen ist, der hinter dem Wechsel der Leitfunktion von
einer technischen zu einer ökonomischen Dimension zu vermuten ist. Dies muss
hier als Andeutung stehen bleiben, da entsprechende Überlegungen den Rahmen
dieses Aufsatzes massiv sprengen würden.

\begin{thebibliography}{xxx}
\bibitem{Barkleit2000} G. Barkleit (2000). Mikroelektronik in der
  DDR. Dresden, 2000.
\bibitem{Bertalanffy1950} Ludwig von Bertalanffy (1950). An outline of General
  System Theory. The British Journal for the Philosophy of Science, vol. I.2,
  134–165.
\bibitem{MEW15} Friedrich Engels (MEW 15). Die Geschichte des gezogenen
  Gewehrs.  MEW 15, S. 195--226. Dietz Verlag, Berlin.
\bibitem{Friedli2013} Thomas Friedli, Stefan Thomas, Andreas Mundt (2013).
  Management globaler Produktionsnetzwerke. Strategie – Konfiguration –
  Koordination. Hanser, München. ISBN: 978-3-446-43449-3
\bibitem{KFK2000} Klaus Fuchs-Kittowski (2000).
  Wissens-Ko-ProduktionVerarbeitung, Verteilung und Entstehung von
  Informationen in kreativ-lernenden Organisationen.\\ In: Fuchs-Kittowski
  u.a. (Hrsg.). Organisationsinformatik und Digitale Bibliothek in der
  Wissenschaft. Wissenschaftsforschung, Jahrbuch 2000. Gesellschaft für
  Wissenschaftsforschung, Berlin.
  \url{http://www.wissenschaftsforschung.de/JB00_9-88.pdf}
\bibitem{Gerovitch1996} Slava Gerovitch (1996). Perestroika of the History of
  Technology and Science in the USSR: Changes in the Discourse. Technology and
  Culture, Vol. 37.1, S. 102--134.
\bibitem{Goldberg2016} Jörg Goldberg, André Leisewitz (2016). Umbruch der
  globalen Konzernstrukturen.\\ Z 108, S. 8--19.
\bibitem{Graebe2012} Hans-Gert Gräbe (2012). Wie geht Fortschritt? LIFIS
  online, 12.11.2012.
\bibitem{Graebe2018} Hans-Gert Gräbe (2018).  12. Interdisziplinäres Gespräch
  \emph{Nachhaltigkeit und technische Ökosysteme}. Leipzig, 02.02.2018. 
  \url{http://mint-leipzig.de/2018-02-02.html}.
\bibitem{Graebe2019a} Hans-Gert Gräbe (2019a).
  \foreignlanguage{russian}{Наследие Движения Школ Изобретателeй в ГДР и
    Развитиe ТРИЗ.}  Erschienen im Online-Protokollband des TRIZ Summit 2019
  Minsk.
\bibitem{Graebe2019b} Hans-Gert Gräbe (2019b).  A discussion about TRIZ
    and system thinking reported in my Open Discovery Blog.
    \url{https://wumm-project.github.io/2019-08-07}.
\bibitem{Graebe2020} Hans-Gert Gräbe (2020). Reader zum 16. Interdisziplinären
  Gespräch \emph{Das Konzept Resilienz als emergente Eigenschaft in offenen
    Systemen} am 7.2.2020 in Leipzig.
  \url{http://mint-leipzig.de/2020-02-07/Reader.pdf}.
\bibitem{Holling2000} C.S. Holling (2000). Understanding the Complexity of
  Economic, Ecological, and Social Systems. In: Ecosystems (2001) 4, 390–405.
\bibitem{Jacobasch2019} Gisela Jacobasch (2019). Bienensterben -- Ursachen und
  Folgen.  Leibniz Online 37 (2019).
  \url{https://leibnizsozietaet.de/bienensterben-ursachen-und-folgen/}
\bibitem{KS2017} Karl Koltze, Valeri Souchkov (2017).  Systematische
  Innovation.\\ Hanser, München. Zweite Auf"|lage. ISBN 978-3-446-45127-8.
\bibitem{Kropik2009} Markus Kropik (2009). Produktionsleitsysteme in der
    Automobilfertigung. Springer, Dordrecht.\\ ISBN 978-3-540-88991-5.
\bibitem{TBK-2007} S. Litvin, V. Petrov, M. Rubin (2007). TRIZ Body of
  Knowledge. \\ \url{https://triz-summit.ru/en/203941}.
\bibitem{TESE2018} Alexander Lyubomirskiy, Simon Litvin, Sergey Ikovenko,
  Christian M. Thurnes, Robert Adunka (2018). Trends of Engineering System
  Evolution. Eigenverlag, Sulzbach-Rosenberg.  ISBN 978-3-00-059846-3.
\bibitem{MEW3} Karl Marx (MEW 3).  Thesen über Feuerbach. MEW 3, S. 533--535.
  Dietz Verlag, Berlin.
\bibitem{MEW23} Karl Marx (MEW 23). Das Kapital, Band 1. MEW 23. Dietz Verlag,
  Berlin.
\bibitem{MEW42} Karl Marx (MEW 42). Grundrisse der Kritik der politischen
  Ökonomie.  MEW 42. Dietz Verlag, Berlin.
\bibitem{Naetar2005} Franz Naetar (2005). „Commodification“, Wertgesetz und
  immaterielle Arbeit. Grundrisse 14, S. 6--19.
\bibitem{Pohl2005} Klaus Pohl, Günter Böckle, Frank J. van der Linden (2005).
  Software Product Line Engineering. Foundations, Principles and Techniques.
  Springer. ISBN 978-3-540-28901-2
\bibitem{Preez2006} Niek D Du Preez, Louis Louw, Heinz Essmann (2006). An
  innovation process model for improving innovation capability.  Journal of
  high technology management research, vol 17, 1--24.
\bibitem{Rubin2007} Michail S. Rubin (2007). \foreignlanguage{russian}{О
  выборе задач в социально-технических системах}. (Über die Wahl von Aufgaben
  in sozial-technischen Systemen). In: \foreignlanguage{russian}{ТРИЗ Анализ.
    Методы исследования проблемных ситуаций и выявления инновационных
    задач}. (TRIZ-Analyse. Methoden zur Untersuchung von Problemsituationen
  und zur Identifizierung innovativer Aufgaben). Hrsg. von S.S. Litvin,
  V.M. Petrov, M.S. Rubin. \foreignlanguage{russian}{Библиотека Саммита
    Разработчиков ТРИЗ}, Moskau. S. 35--46.
  \url{https://www.trizland.ru/trizba/pdf-books/TRIZ-summit2007.pdf}.
\bibitem{Rubin2010} Michail S. Rubin (2010).
  \foreignlanguage{russian}{Филогенез социокультурных систем. Секреты развития
    цивилизаций}.  (Phylogenese soziokultureller Systeme. Geheimnisse der
  Zivilisationsentwicklung).
  \url{http://www.temm.ru/en/section.php?docId=4472}.
\bibitem{Shpakovsky2010} Nikolay Shpakovsky (2010).  Tree of Technology
  Еvolution. Forum, Moscow.
\bibitem{Shub2006} Leonid Shub (2006). \foreignlanguage{russian}{Осторожно!
  Таблица технических противоречий}. (Vorsicht! Die Widerspruchstabelle).
  \url{http://metodolog.ru/conference.html}. Siehe auch ders. Vorsicht
  Widerspruchsmatrix, Kurzfassung in Deutsch.
  \url{https://wumm-project.github.io/Texts/Shub-2006.pdf}.
\bibitem{Steinbuch1966} Karl Steinbuch (1966). Die informierte Gesellschaft.
  Deutsche Verlags-Anstalt, Stuttgart.
\bibitem{Szyperski2002} Clemens Szyperski (2002). Component Software: Beyond
  Object-Oriented Programming. ISBN: 978-0-321-75302-1.
\bibitem{TT} Target Invention (2020). TRIZ Trainer.
  \url{https://triztrainer.ru}.
\bibitem{Thiel2007} Rainer Thiel (2007). Zur Lehrbarkeit dialektischen Denkens
  – Chance der Philosophie, Mathematik und Kybernetik helfen. In: Klaus
  Fuchs-Kittowski, Rainer E. Zimmermann (Hrsg.). Kybernetik, evolutionäre
  Systemtheorie und Dialektik. Trafo Verlag, Berlin 2012, ISBN:
  978-3-89626-919-5, S. 185--202
\bibitem{VDMA2019} VDMA. Maschinenbau in Zahl und Bild 2019. 
\bibitem{Vernadsky1997} Vladimir I. Vernadsky (1997, Original 1936--38).
  Scientific Thought as a Planetary Phenomenon.
  \url{https://wumm-project.github.io/Texts.html}
\bibitem{Weller2008} W. Weller (2008). Automatisierungstechnik im
  Überblick. Was ist, was kann Automatisierungstechnik? Beuth, Berlin. ISBN
  978-3-410-16760-0.
\bibitem{Zobel2016} D. Zobel, R. Hartmann (2016). Erfindungsmuster.
  2. Auflage.  Expert, Renningen.
\bibitem{Zobel2020} D. Zobel (2020). Beiträge zur Weiterentwicklung der TRIZ.
  LIFIS Online 19.01.2020. DOI: \url{10.14625/zobel_20200119}
\end{thebibliography}
\end{document}


Mehr noch lehrt die Theorie Dynamischer Systeme, dass die Kopplung zwischen
Systemen nicht so sehr allein von Durchsatzraten bestimmt wird, sondern in
ihrer Wirkung stark von zeitlichen Regimes in Form von Resonanzen und
Dissonanzen bestimmt sein können.  Damit kann das Zusammenspiel von System und
Systemkomponenten stark von der Verschränkung von Mikro- und Makrodynamik auf
kurzwelligen (Komponenten) und langwelligen (System) Skalen abhängen.  Ein
wesentliches viertes Charakteristikum \emph{autonom funktionierender
  technischer Systeme} ist eine Entkopplung dieser Systemdynamiken, im
Kontrast etwa zum TRIZ-Prinzip 19 der periodischen Wirkung, das auf die
Ausnutzung entsprechender Kopplungsphänomene gerichtet ist.
