\documentclass[11pt,a4paper]{article}
\usepackage{a4wide,url}
\usepackage[T1,T2A]{fontenc}
\usepackage[utf8]{inputenc}
\usepackage[main=ngerman,russian]{babel}

\parindent0cm
\parskip3pt

\title{Der Mensch und seine Technischen Systeme} 
\author{Hans-Gert Gräbe, Leipzig}
\date{Version vom 19. April 2020}
\begin{document}
\maketitle

\begin{flushright}
  Die Philosophen haben die Welt\\ nur verschieden interpretiert;\\ es kömmt
  darauf an sie zu verändern.\\ Karl Marx. 11. Feuerbachthese
\end{flushright}
\section{Vorbemerkungen}

Ausgangspunkt dieser Überlegungen war eine Debatte mit den Organisatoren des
TRIZ-Cups 2019/20 über die Gültigkeit eines „Gesetzes der Verdrängung des
Menschen aus technischen Systemen“, das in einer späteren Fassung der
Ausschreibung als „Trend“ bezeichnet wurde. Unter den acht Gesetzen der
Entwicklung technischer Systeme, die Altschuller 1979 selbst formulierte
\cite[S. 2]{TESE2018}, kommt ein solcher Ansatz nicht vor, auch nicht in der
Auf"|listung von fünf Gesetzen und zehn Tendenzen in
\cite[S. 148\,ff.]{KoltzeSouchkov2017}. Einen deutlich anderen Ansatz, die
Betrachtung der Evolution einzelner Funktionen und nicht kompletter
technischer Systeme, schlägt N. Shpakovsky in \cite{Shpakovsky2010} mit seinem
Konzept der „Evolutionsbäume“ vor. 

Systematisierungen von „Gesetzen der Evolution technischer Systeme“ oder zum
„technolgy forecasting“ sind aber in der TRIZ-Literatur weit verbreitet. Sie
sind auch Teil der verschiedenen Versionen eines „TRIZ Body of Knowlegde“,
etwa \cite{TBK-2007}. Meine weiteren Ausführungen beziehen sich auf
\cite{TESE2018} als Referenz, da hier von einflussreichen TRIZ-Theoretikern
mit der Autorität der MATRIZ im Rücken ein aktueller Zusammenschnitt der
Debatten um „Trends of Engineering Systems Evolution“ gegeben wird.

Auch in der marxistischen Literatur wird ein solcher Herauslösungsprozess des
Menschen aus produktiven Prozessen thematisiert und an vielen Stellen als
unausweichlich charakterisiert.  So entwickelt Marx selbst im
„Maschinenfragment“ \cite[S. 570 ff.]{MEW42} -- einem frühen Rohentwurf der
eigenen ökonomischen Theorie -- die Vision einer Gesellschaft, in welcher der
„gesellschaftliche Stoffwechsel“ \cite[S. 37]{MEW23} auf eine Weise
organisiert ist, dass
\begin{quote}
  es nicht mehr der Arbeiter [ist], der modifizierten Naturgegenstand als
  Mittelglied zwischen das Objekt und sich einschiebt; sondern den
  Naturprozess, den er in einen industriellen umwandelt, er als Mittel
  zwischen sich und die unorganische Natur [schiebt], deren er sich
  bemeistert \cite[S. 572]{MEW42},
\end{quote}
und stellt weiter dar, dass die Entwicklung der Produktivkräfte
\emph{notwendig} auf eine solche Weise der Organisation des gesellschaftlichen
Stoffwechsels zusteuert.
\begin{quote}
  In den Produktionsprozess des Kapitals aufgenommen, durchläuft das
  Arbeitsmittel aber verschiedene Metamorphosen, deren letzte die
  \emph{Maschine} ist oder vielmehr ein \emph{automatisches System der
    Maschinerie} (System der Maschinerie; das \emph{automatische} ist nur die
  vollendetste adäquateste Form derselben und verwandelt die Maschinerie erst
  in ein System), in Bewegung gesetzt durch einen Automaten, bewegende Kraft,
  die sich selbst bewegt; dieser Automat bestehend aus zahlreichen mechanischen
  und intellektuellen Organen, sodass die Arbeiter selbst nur als bewusste
  Glieder desselben bestimmt sind. \cite[S. 584]{MEW42}
\end{quote}
Dieser Gedanke sei allerdings weitgehend singulär und im übrigen Marxschen
Werk nirgends ausgearbeitet, so \cite{Goldberg2016}. Zumindest auf das
Interesse an technischen Systemen und Entwicklungen trifft das allerdings
nicht zu, hierzu finden sich viele Stellen im Werk dieser Klassiker. Engels'
Interesse besonders an militär-historischen Entwicklungen ist weithin bekannt,
in \cite{MEW15} etwa wird die Evolution des gezogenen Gewehrs auf eine Weise
analysiert, die es mit jeder TRIZ-Analyse aufnehmen kann, etwa in der
Herausarbeitung des sozio-technischen Widerspruchs \cite[S. 199]{MEW15}:
\begin{quote}
  Unter diesen Umständen ergab sich folgende dringende Aufgabe: eine
  Feuerwaffe zu erfinden, welche die Schussweite und die Genauigkeit der
  Büchse mit der Schnelligkeit und Leichtigkeit des Ladens und mit der Länge
  des Laufs der glattläufigen Muskete vereint, eine Waffe also, die zugleich
  Büchse und Nahkampfwaffe ist, welche man jedem Infanteristen in die Hand
  geben kann.

  So sehen wir also, dass gerade durch die Einführung des Kampfes in
  aufgelöster Ordnung in die moderne Taktik sich die Forderung nach einer
  solchen verbesserten Kriegswaffe erhob. [\ldots] Fast alle Verbesserungen,
  die an Handfeuerwaffen seit 1828 vorgenommen wurden, dienten diesem Zweck.
\end{quote}
\cite{Goldberg2016} untersucht allerdings nicht diese Frage, sondern aktuelle
Kapitalkonzentrationsprozesse, die erforderlich sind, um die immer
kostspieligeren technischen Großprojekte zu finanzieren.  In diesem Kontext
wird neuerdings viel über \emph{Plattformkapitalismus} geschrieben.  Der
unmittelbare Zusammenhang zwischen technologischen und finanziellen
Möglchkeiten ganzer Staaten im Kontext einer „wealth of nations“ ist
hinreichend bekannt. \cite{Barkleit2000}  analysiert die technologischen und
ökonomischen Interdependenzen des historischen Prozesses des letztlich
gescheiterten DDR-Mikroelektronikprojekts der 1980er Jahre. Im TRIZ-Umfeld
spielen derartige Überlegungen eine eher marginale Rolle.

Technikoptimistischen Sichten der „Vrdrängung des Menschen aus technischen
Systeme“ steht die Position aus dem Kybernetikdiskurs der 1960er bis 1980er
Jahre entgegen \cite[S. 10]{KFK2000}:
\begin{quote}
  Welche Stellung hat der Mensch im hochkomplexen informations-technologischen
  System? Unsere Antwort auf die Frage war immer: Der Mensch ist die einzig
  kreative Produktivkraft, er muss Subjekt der Entwicklung sein und bleiben.
  Daher ist das Konzept der Vollautomatisierung, nach dem der Mensch
  schrittweise aus dem Prozess eliminiert werden soll, verfehlt!
\end{quote}
Die Probleme eines solchen „Konzepts der Vollautomatisierung“, einer Welt der
„in Bewegung gesetzten Automaten“ werden mittlerweile in einer ökologischen
Krise planetaren Ausmaßes sichtbar. Die Verdrängungsthese selbst wird dabei
als direkte Gefährung wahrgenommen, was hier als \emph{Gegenthese} explizit
formuliert werden soll:
\begin{quote}
  Die (scheinbare) Verdrängung des Menschen aus technischen Systemen weist auf
  eine existenziell gefährliche, unterkomplexe Fehlwahrnahme dieser
  technischen Systeme hin.
\end{quote}
Das Altschullersche „Prinzip 11 der Prävention“ (verschämt auch als „Prinzip
des untergelegten Kissens“ bezeichnet) weist auf Handlungsbedarf in dieser
Richtung hin, der Anwendungskontext dieses Prinzips wird in \cite{TT} wie
folgt umrissen: „Der Einsatz des Prinzips ist besonders in solchen Fällen
wichtig, in denen das System nicht über ein ausreichendes Maß an
Zuverlässigkeit verfügt“, um dann festzustellen, dass dem eigentlich
strukturell abgeholfen werden könne -- „Notfallsituationen können vermieden
werden, indem der Prozess zuverlässig gemacht wird.“ Dem stehen allerdings
gewichtige Gründe entgegen -- „was die technischen Systeme, die ihn
durchführen, erheblich verkompliziert oder verteuert. Dies ist kostspielig und
oft prinzipiell unmöglich. Mit anderen Worten: Notfälle sind unvermeidlich“.
Der Optimismus der Autoren, dass „zusätzliche Rettungs- und Notfallsysteme
[\ldots] in das Hauptsystem“ eingebaut werden, die allerdings „nicht am
Hauptsystem teilnehmen, sondern erst in einer Gefahrensituation zu arbeiten
beginnen“, erscheint mit Blick auf Kosten reduzierende Designpraxen und die
allgemeine Ausrichtung der anderen TRIZ-Prinzipien auf Effizienzgewinne als
Widerspruch in der TRIZ-Methodik selbst. Die Aussage, „das Prinzip kann dort
angewendet werden, wo die Zuverlässigkeit des Systems offensichtlich
unzureichend ist und ein Weg zur Erhöhung der Zuverlässigkeit auf das
notwendige Niveau nicht möglich ist“ (ebenda) zeigt, dass der Einbau solcher
Defizite in technische Systeme -- in Kenntnis derselben -- auf breiter Front
billigend in Kauf genommen wird.

Dies ist allerdings in keiner Weise mehr ein technisches Problem.  Harrisburg,
Tschernobyl, Fukushima, das Bienensterben \cite{Jacobasch2019} oder der
Klimawandel sind genü"|gend Fingerzeige, um sich mit diesen Positionen genauer
zu befassen.

Wir zeigen in diesem Aufsatz, dass die 10 nunmehr als „Trends“ präsentierten
„Gesetze der Entwicklung technischer Systeme“ in Wirklichkeit Design Pattern
in den ingenieur-technischen Praxen des Entwurfs sowie der Anpassung und
Verbesserung technischer Ssysteme sind und sich somit \emph{unmittelbar} auf
sozio-technische Praxen beziehen. Dabei beziehen sie sich auf ein
\emph{spezifisches} begriffliches Abstraktionsniveau der sich insgesamt in
Widersprüchen entwickelnden Beschreibungsformen von Welt. Insbesondere stehen
die „Trends“ damit im Widerspruch zu Entwicklungslinien, die sich auf anderen
Abstraktionsniveaus abzeichnen. Kurz, das oben in These und Gegenthese
entfaltete ambivalente Verhältnis zu einer „Verdrängung des Menschen aus
technischen Systemen“ ist kein Alleinstellungsmerkmal nur für diesen Trend
sondern trifft in ähnlicher Weise auch auf die anderen 9 Trends zu.

TRIZ ist eine gute Methodik, um diese Widersprüche zu analysieren. Dazu darf
man sich aber nicht auf das Memorieren dieser Trends beschränken.  

\section{Technik und Welt verändernde Praxen}

Betrieb und Nutzung technischer Systeme ist heute sicher ein zentrales Element
Welt ver"|ändernder menschlicher Praxen. Dafür ist planmäßiges und abgestimmtes
arbeitsteiliges Handeln erforderlich, denn das Nutzen eines Systems setzt
dessen Betrieb voraus.  Umgekehrt ist es wenig sinnvoll, ein System zu
betreiben, das nicht genutzt wird. In der Informatik ist dieser Zusammenhang
zwischen Definition und Aufruf einer Funktion gut bekannt -- der Aufruf einer
Funktion, die noch nicht definiert wurde, führt zu einem Laufzeitfehler; die
Definition einer Funktion, die nie aufgerufen wird, weist auf einen
Designfehler hin.

Eng verbunden mit der informatischen Unterscheidung von Definition und Aufruf
einer Funktion ist die Unterscheidung von Designzeit und Laufzeit.  Eine
solche Unterscheidung hat im realweltlichen arbeitsteiligen Einsatz
technischer Systeme noch größere Bedeutung -- während der Designzeit wird das
prinzipielle kooperative Zusammenwirken \emph{geplant}, während der Laufzeit
\emph{der Plan ausgeführt}. Für technische Systeme sind also zusätzlich deren
interpersonal als \emph{begründete Erwartungen} kommunizierten
\emph{Beschreibungsformen} und die in \emph{erfahrenen Ergebnissen}
resultierenden \emph{Vollzugsformen} zu unterscheiden.

Marx \cite[S. 193]{MEW23} merkt dazu an:
\begin{quote}
  Eine Spinne verrichtet Operationen, die denen des Webers ähneln, und eine
  Biene beschämt durch den Bau ihrer Wachszellen manchen menschlichen
  Baumeister. Was aber von vornherein den schlechtesten Baumeister vor der
  besten Biene auszeichnet, ist, dass er die Zelle in seinem Kopf gebaut hat,
  bevor er sie in Wachs baut. Am Ende des Arbeitsprozesses kommt ein Resultat
  heraus, das beim Beginn desselben schon in der Vorstellung des Arbeiters,
  also schon ideell vorhanden war.
\end{quote}
So einfach ist es allerdings nicht, wie das folgende Beispiel einer
Konzertauf"|führung zeigt. Dieser die Zuhörer erfreuenden Vollzugsform geht
die Erarbeitung der Beschreibungsform, die Verständigung über die genaue
Interpretation des aufzuführenden Werks, voraus. Diese Verständigung auf einen
\emph{gemeinsamen Plan} ist selbst ein voraussetzungsreicher praktischer
Prozess.  Die Voraussetzungen resultieren aus vorgängigen Praxen -- etwa dem
\emph{privaten Verfahrenskönnen} der einzelnen Musiker in der Beherrschung
ihrer Instrumente sowie dem Vorliegen der Partitur als etablierter
Beschreibungsform des aufzuführenden Konzertstücks.  Wenn Alexander Shelley am
14. Oktober 2018 im Leipziger Gewandhaus ohne diese Partitur von Mozarts
Klavierkonzert KV 491 ans Dirigentenpult tritt, so wird deutlich, dass jene
Beschreibungsform allenfalls das Rohmaterial liefert, auf dessen Basis sich
Dirigent und Orchester in den vorausgegangenen Proben auf eine situativ
konkrete Beschreibungsform als Basis der nun zur Auf"|führung gelangenden
Vollzugsform geeinigt haben. Mehr noch weisen die opulenten Gesten des
Dirigenten in Richtung Orchester darauf hin, dass in diesen Proben auch
\emph{Sprache} generiert wurde, um die Ergebnisse längerer
Verständigungsprozesse in eine kompakte Form zu fassen, die den zeitkritischen
Tempi der Vollzugsform gewachsen ist.  Den Rahmen einfacher
ingenieur-technischer „Baumeisterarbeit“ sprengt Gabriela Montero, die
Solistin jenes Abends, mit ihrer Zugabe: Das Publikum wird aufgefordert, eine
Melodie vorzugeben, woraus die Virtuosin eine Improvisation als Vollzugsform
entwickelt, zu der es keine interpersonal kommunizierbare Beschreibungsform
gibt, wenn man einmal von den Tonaufzeichungen jenes Gewandhausabends und den
Berichten der begeisterten Hörerschaft absieht.  Dass auch hierfür technische
Meisterschaft erforderlich war, steht außer Frage.

Das Verhältnis der Menschen zu ihren technischen Systemen ist also komplex und
nur in einer dialektischen Perspektive der Weiterentwicklung bereits
vorgefundener technischer Systeme zu fassen, wenn man sich nicht unentrinnbar
in unfruchtbare Henne-Ei-Debatten verstricken will.  Das relativiert aber auch
die Marxsche Forderung an die Philosophen, denn deren Interpretationen sind
die Differenzen zwischen den begründeten Erwartungen und den erfahrenen
Ergebnissen früherer Praxen vorgängig. Ob es ausreicht, diese Differenzen auf
der Ebene der Techniker, der Ingenieure oder der Fachwissenschaftler zu
besprechen oder eine Intervention der „interpretierenden Philosophen“ als
eigenständige Reflexionsdimension von Bedeutung ist, mag an dieser Stelle
offen bleiben. 

\section{Systeme und Komponenten}

Neben der Beschreibungs- und Vollzugsdimension spielt für technische Systeme
auch der \emph{Aspekt der Wiederverwendung} eine große Rolle.  Dies gilt,
zumindest auf der artefaktischen Ebene, allerdings \emph{nicht} für die
meisten technischen Großsysteme -- diese sind \emph{Unikate}, auch wenn bei
deren Montage standardisierte Komponenten verbaut werden. Auch die Mehrzahl
der Informatiker ist mit der Erstellung solcher Unikate befasst, denn die
IT-Systeme, die derartige Anlagen steuern, sind ebenfalls Unikate.  Dasselbe
gilt auch für die Ämter, Behörden und öffentlichen Einrichtungen. So ist zum
Beispiel die Leipziger Stadtverwaltung aktuell damit befasst, ihre
Verwaltungsprozesse zu „digitalisieren“, was unter Führung des Dezernats
Allgemeine Verwaltung und zusammen mit dem städtischen IT-Dienstleister Lecos
erfolgt. Im Industriesektor ist deshalb deutlich zwischen Werkzeugmaschinenbau
und Industrieanlagenbau -- zwischen Ausrüstern sowie Planern und „Baumeistern“
entsprechender Unikate -- zu unterscheiden, auch wenn dies in einschlägigen
Statistiken \cite{VDMA2019} zum \emph{Maschinen- und Anlagenbau}
zusammengefasst wird.

Die Besonderheiten eines technischen Systems liegen damit vor allem im Bereich
des \emph{Zusammenspiels der Komponenten}. So unterscheiden sich
beispielsweise die Produktionsleitsysteme verschiedener BMW-Werke deutlich
voneinander \cite{Kropik2009}. Die Werke wurden zu verschiedenen Zeiten nach
dem jeweiligen Stand der Technik und dem sich ebenfalls verändernden
Geschäftsmodell des Unternehmens konzipiert. Einmal in die Welt gesetzt, sind
derartige technischen Großsysteme nur noch bedingt modifizierbar und werden
deshalb nach Ablauf entsprechender Amortisationsfristen auch konsequent außer
Betrieb gestellt. Gleichwohl spielt der Aspekt der Wiederverwendung auch bei
solch unterschiedlichen technischen Systemen eine Rolle, verschiebt sich aber
von der unmittelbaren Ebene der technischen Artefakte auf höhere Ebenen der
Abstraktion in der Beschreibungsdimension.

Damit sind wesentliche Elemente zusammengetragen, die eine erste Annäherung an
den \emph{Begriff eines technischen Systems} erlauben.  Der Begriff ist in
einem planerisch-realweltlichen Kontext vierfach überladen
\begin{itemize}
\item [1.] als realweltliches Unikat (z.B. als Produkt, auch wenn das Unikat
  ein Service ist),
\item [2.] als Beschreibung dieses realweltlichen Unikats (z.B. in der Form
  einer speziellen Produktkonfiguration)
\end{itemize}
und für in größerer Stückzahl hergestellte Komponenten auch noch
\begin{itemize}
\item [3.] als Beschreibung des Designs des System-Templates (Produkt-Design)
  sowie
\item [4.] als Beschreibung und Betrieb der Auslieferungs- und
  Betriebsstrukturen der nach diesem Template gefertigten realweltlichen
  Unikate (als Produktions-, Qualitätssicherungs-, Auslieferungs-, Betriebs-
  und Wartungspläne).
\end{itemize}
Als Grundlage für einen derart abgrenzenden Systembegriff soll im Weiteren der
submersiv gefasste Begriff offener Systeme der Theorie dynamischer Systeme
\cite{Bertalanffy1950} verwendet werden, der
\begin{itemize}
\item [1.] eine innere Abgrenzung gegen vorgefundene Systeme (Komponenten), 
\item [2.] eine äußere Abgrenzung und funktional determinierte Einbettung in
  eine (funktionierende) Umwelt sowie
\item [3.] einen (funktionierenden) externen Durchsatz postuliert, der zu
  innerer Strukturbildung führt und damit die Leistungsfähigkeit des Systems
  bestimmt,
\end{itemize}
und seine Fruchtbarkeit für eine Behandlung mit mathematischen Instrumenten
seither vielfach unter Beweis gestellt hat.  

\emph{Technische Systeme} sind in einem solchen Kontext Systeme, auf deren
Gestaltung kooperativ und arbeitsteilig agierende Menschen Einfluss nehmen,
wobei \emph{vorgefundene} technische Systeme auf Beschreibungsebene durch eine
\emph{Spezifikation} ihrer Schnittstellen und auf Vollzugsebene durch die
\emph{Gewähr spezifikationskonformen Betriebs} normativ charakterisiert sind.

Wir bewegen uns dabei klar im Bereich der Standard-TRIZ-Terminologie eines
\emph{Systems von Systemen} -- ein technisches System besteht aus Komponenten,
die ihrerseits technische Systeme sind, deren \emph{Funktionieren} (sowohl im
funktionalen als auch im operativen Sinn) für die aktuell betrachtete
Systemebene vorausgesetzt wird. Die Rolle des Begriffs \emph{Objekt} und
dessen Abgrenzung gegen den Begriff \emph{Komponente} besprechen wir weiter
unten. 

Der Begriff eines technischen Systems hat damit eine klar epistemische
Funktion der (funktionalen) „Reduktion auf das Wesentliche“.  Einstein wird
der Ausspruch zugeschrieben „make it as simple as possible but not
simpler“. Das \emph{Gesetz der Vollständigkeit eines Systems} bringt genau
diesen Gedanken zum Ausdruck, allerdings tritt dieser dabei nicht als
\emph{Gesetz}, sondern als ingenieur-technische \emph{Modellierungsdirektive}
in Erscheinung.  Die scheinbare „Naturgesetzlichkeit“ der beobachteten Dynamik
ist also wesentlich an \emph{vernünftiges} (im Sinne von \cite{Vernadsky1997})
\emph{menschliches Agieren} gebunden.

Mit einem Ansatz der „Reduktion auf das Wesentliche“ sowie der „Gewähr
spezifikationskonformen Betriebs“ sind in diese Begriffsbildung inhärent
menschliche Praxen eingebaut, aus denen heraus die Begriffe „wesentlich“,
„Gewähr“ und „Betrieb“ überhaupt erst sinnvoll gefüllt werden können.  Eine
Unterscheidung zwischen technischen und sozio-technischen Systemen, die für
M. Rubin „offensichtlich und wesentlich“ (private Kommunikation) ist, wird
damit problematisch. Wesentliche Begriffe aus dem sozial determinierten
Praxisverhältnis von Menschen wie Ziel, Nutzen, Gewährleistung und
Verantwortung sind fest in die Begriffsgenerierungsprozesse der Beschreibung
konkreter technischer Systeme eingebaut und finden in den konkreten
gesellschaftlichen Setzungen eines primär rechtsförmig konstituierten
bürgerlichen Systems ihre „natürliche“ Fortsetzung.

\section{Die Welt der Technischen Systeme. Basics}

In der TRIZ-Literatur spielen solche begriff"|lichen Fundierungen kaum eine
Rolle.  Einschlägige Lehrbücher wie etwa \cite{KoltzeSouchkov2017} betrachten
den Begriff des \emph{technischen Systems} als intuitiv gegeben, der sich aus
einer „industriellen Praxis“ heraus \cite[S. 2]{KoltzeSouchkov2017} von selbst
versteht, während andere Begriffe wie „Prozess“, „Produkt“, „Dienstleistung“,
„Ressourcen“ und „Effekte“ \cite[S. 6--10]{KoltzeSouchkov2017} genauer
eingeführt werden. Selbst die ausführliche Beschreibung der „Evolution
technischer Systeme“ in 5 Gesetzen und 11 Trends
\cite[Kap. 4.8]{KoltzeSouchkov2017} basiert allein auf der lapidaren
Feststellung „Die Existenz technischer Evolution ist eine zentrale Erkenntnis
der TRIZ“.  Auch \cite{TESE2018} bleibt in dieser Frage vage; im Vorwort von
B. Zlotin heißt es allein zum \emph{Zweck} von Betrachtungen der Evolution
ingenieur-technischer Systeme „humanity can achieve practically any realistic
goal, but certain priorities must be set to ensure the greatest possible
impact on the economy and human life. [\ldots] The powers of contemporary
science and technology as well as financial investment should be applied to
carefully selected and formulated objectives.“

Es ist natürlich möglich, in einem diskursiven Rahmen die verbale Fassung
eines Begriffs offen zu lassen und auf andere Weise -- etwa durch den Bezug
auf gemeinsame Praxen oder durch den „gewöhnlichen Gebrauch“ -- die Konvergenz
der Begriffsverwendung zu erreichen.  Ein solches Grundmuster wird im
TRIZ-Kontext für den Begriff \emph{technisches System} besonders auch in
\cite{TESE2018} angewendet, indem der Begriff durch eine Vielzahl von
Beispielen in Kombination mit den Begriffen „Muster“ und „Evolution“
illustriert, die genaue Fassung aber dem geneigten Leser überlassen wird.  Der
dort mittlerweile erfolgte Rückzug auf Begriffe wie „Muster“ oder „Trend“
gegenüber dem schärferen und wissenschaftspraktisch vorbelegten Begriff
„Gesetz“ unterstützt das Anliegen der Autoren von \cite{TESE2018}, empirische
Erfahrung zu systematisieren, verweist aber zugleich auf das schwache
theoretische Fundament eines solchen Systematisierungsanliegens.  Das weite
Spektrum praktisch kursierender Präzisierungen eines derart im Ungewissen
gelassenen Begriffs wurde in einer Facebook-Diskussion \cite{Graebe2019} im
August 2019 deutlich. Für ein genaueres Abwägen der Argumente zu oben
formulierter These und Gegenthese ist ein solches Fundament allerdings nicht
ausreichend.

Wie kann der Begriff eines \emph{technischen Systems} also weiter geschärft
werden?  In unserem Seminar \cite{Graebe2020} haben wir „den Systembegriff als
Beschreibungsfokussierung identifiziert, mit der konkrete Phänomene durch
\emph{Reduktion auf das Wesentliche} [\ldots] einer Beschreibung zugänglich
werden.“  Die Reduktion richtet sich auf folgende drei Dimensionen
\cite[S. 18]{Graebe2020} 
\begin{itemize}
\item [(1)] Abgrenzung des Systems nach außen gegen eine \emph{Umwelt},
  Reduktion dieser Beziehungen auf Input/Output-Beziehungen und garantierten
  Durchsatz.
\item [(2)] Abgrenzung des Systems nach innen durch Zusammenfassen von
  Teilbereichen als \emph{Komponenten}, deren Funktionieren auf eine
  „Verhaltenssteuerung“ über Input/Output-Bezie"|hungen reduziert wird.
\item [(3)] Reduktion der Beziehungen im System selbst auf „kausal
  wesentliche“ Beziehungen.
\end{itemize}
Weiter wird ebenda festgestellt, dass -- ähnlich wie im Konzertbeispiel --
einer solchen reduktiven Beschreibungsleistung vorgefundene (explizite oder
implizite) Beschreibungsleistungen vorgängig sind:
\begin{enumerate}
\item[(1)] Eine wenigstens vage Vorstellung über die (funktionierenden)
  Input/Output-Leistungen der Umgebung.
\item[(2)] Eine deutliche Vorstellung über das innere Funktionieren der
  Komponenten (über die reine Spezifikation hinaus).
\item[(3)] Eine wenigstens vage Vorstellung über Kausalitäten im System
  selbst, also eine der detaillierten Modellierung vorgängige, bereits
  vorgefundene Vorstellung von Kausalität im gegebenen Kontext.
\end{enumerate}
Die Punkte (1) und (2) können ihrerseits in systemtheoretischen Ansätzen für
die Beschreibung der „Umwelt“ (hierfür ist allerdings die Abgrenzung eines
oder mehrerer Obersysteme in einer noch umfassenderen „Umwelt“ erforderlich)
sowie der Komponenten (als Untersysteme) entwickelt werden, womit die
Beschreibung von \emph{Koevolutionsszenarien} wichtig wird, die ihrerseits für
die Vertiefung des Verständnisses von Punkt (3) relevant sind.

Dabei ist der Fokus zunächst auf ein genaueres Verständnis des Begriffs
\emph{System} gerichtet, der als Reduktion von Komplexität in den drei oben
angeführten Richtungen betrachtet wird. Da in diesem Verständnis Komponenten
eines Systems selbst wieder Systeme sind, liegt auch im allgemeinen Fall die
Betrachtung eines Systems als „System von Systemen“ nahe, wie es in
\cite{Holling2000} thematisiert ist.  Wesentliches Reduktionskriterium für
Beziehungen zwischen Komponenten sind in solchen Systemen spezifische
Eigenzeiten und Eigenräume wie in den Abbildungen 1--3 in \cite{Holling2000}
dargestellt ist, die auch in den TRIZ-Prinzipien 18 „Ausnutzung mechanischer
Schwingungen“, 19 „periodische Wirkung“, 23 „Rückkopplung“ und 25
„Selbstbedienung“ eine Rolle spielen. 

Die Beschreibung von Planung, Entwurf und Verbesserung \emph{technischer
  Systeme} geht in einem solchen Ansatz von der Leistungsfähig"|keit bereits
vorhandener technischer Systeme aus, die sowohl in (2) als Komponenten als
auch -- aus der Sicht eines Systems im Obersystem -- in (3) als benachbarte
Systeme zu berücksichtigen sind.

Ingenieur-technische Praxen bewegen sich damit in einer \emph{Welt von
  technischen Systemen}. Aus der konkreten Beschreibungsperspektive eines
Systems sind andere Systeme als Komponenten oder Nachbarsysteme allein in
ihrer \emph{Spezifikation} wichtig. Eine solche Reduktion auf das Wesentliche
erscheint praktisch als verkürzte Sprechweise über eine gesellschaftliche
Normalität, was ich kurz als \emph{Fiktion} bezeichne.  Diese Fiktion kann und
wird im täglichen Sprachgebrauch so lange aufrecht erhalten, so lange die
gesellschaftlichen Umstände die Aufrechterhaltung der daran gebundenen
gesellschaftlichen Normalität garantieren können, so lange also der
\emph{Betrieb der entsprechenden Infrastrukturen} gewährleistet ist.
Technische Systeme sind damit wenig"|stens in ihrer Vollzugsdimension
\emph{immer} sozio-technische Systeme.

Ein Ausblenden dieser sozialen Zusammenhänge kann sich also allenfalls auf die
\emph{Planung} derartiger Systeme sowie deren artefaktische Daseinsdimension
beziehen, die den \emph{Betrieb} der erforderlichen Infrastruktur ausblendet
oder in ein Obersystem verschiebt.  Letzteres ist aber unzweckmäßig, da das
Beheben von Problemen im Betrieb eines Systems Kenntnisse über dessen
Funktionieren nicht nur auf der Ebene der Spezifikation, sondern auch auf der
Ebene der Implementierung erfordern.

Eine solche Engführung des Begriffs \emph{technisches System} resultiert
möglicherweise aus spezifischen Praxen der Vermittlung und Weiterentwicklung
von TRIZ-Grundlagen, da TRIZ-Praktiker zu derartigen Auseinandersetzungen um
Fundierungen der von ihnen angewendeten Theorien ein entspanntes bis
ignorantes Verhältnis an den Tag legen. Für eine Theorie der Evolution
technischer Systeme ist aber eine noch weitergehende Engführung und
Abstraktion des Begriffs erforderlich, da sich die bisherige Begriffsgenese
ausschließlich an der zu einem gegebenen Zeitpunkt \emph{vorgefundenen}
Landschaft technischer Systeme orientiert.

\section{Zum Evolutionsbegriff in \cite{TESE2018}.\\ Die sozio-ökonomische
  Dimension} 

Um evolutionäre Aspekte zu thematisieren, ist eine Zuordnung von zu
verschiedenen Zeiten existierenden technischen Systemen zu
\emph{Entwicklungslinien} erforderlich. Hier gehen \cite{Shpakovsky2010} und
\cite{TESE2018} deutlich verschiedene Wege.

In \cite{TESE2018} wird \emph{Evolution}, wie V. Souchkov im Vorwort
\cite[S. IX]{TESE2018} feststellt, als „innovative development“ verstanden,
„since -- in contrast to nature -- craftsmen and engineers make decisions
based on logic, previous experience, and knowledge of basic principles rather
than chance.“ Die Konzentration auf „craftsmen and engineers“ weist noch
einmal auf die Engführung der Praxen hin, aus denen die Systematisierung in
\cite{TESE2018} abgeleitet wurde.

Andere mögliche Zugänge zur Einbettung des bisher entwickelten Begriffs in
historische Geneseprozesse hätten sich an der wissenschaftshistorischen
Betrachtung \cite{Weller2008} der Entwicklung der
\emph{Automatisierungstechnik} als eines der wichtigsten interdisziplinären
technikwissenschaftlichen Bereiche und damit \emph{aus der Perspektive der
  Praxen des Industrieanlagenbaus} oder der industriehistorischen Untersuchung
der Genese der \emph{Praxen der Produktion} realweltlicher technischer Systeme
orientieren können, in denen Produktlinien \cite{Pohl2005},
Produktionsnetzwerke \cite{Friedli2013} oder neuerdings auch technische
Ökosysteme \cite{Graebe2018} eine zentrale Rolle spielen. Mit dem Bereich des
\emph{Systems Engineering} existiert zudem ein aus der Informatik
hervorgegangenes umfassendes technikwissenschaftliches Forschungsgebiet mit
vergleichbaren Fragestellungen, dessen Grundlagen in einer Norm
ISO/IEC/IEEE-15288 \emph{Systems and Software Engineering} dokumentiert sind.

Der Zugang in \cite{TESE2018} ist allerdings ein anderer -- zur
Identifizierung von Entwicklungslinien wird der Begriff \emph{technisches
  System} zwischen „technology push“ und „market pull“ als „simple means for
understanding the advancement of man-made systems“ eingebettet
\cite[S. 1]{TESE2018}. Der Bezug auf den noch ungenaueren Begriff „man-made
system“ (nicht „men-made system“!) wird im Weiteren genauer erläutert:
Innovation als „improvement of already-existing systems“ wird durch das
Fortschreiben wissenschaftlicher Erkenntnis angetrieben, aus der heraus neue
Systeme (!), Produkte und Dienstleistungen entstehen, die von einem „market
pull, the second trigger for innovation“ einem Formungs- und Ausleseprozess
unterworfen sind, „that stimulates the development of a system by meeting the
needs of that system's users“.  Die genaue Ausformung dieses nicht von
ingenieur-technischen, sondern von innovations-unternehmerischen Praxen
getriebenen Ansatzes wird in \cite[Kap. 3]{TESE2018} deutlich.  Die Gründe für
den universalistischen Anstrich des Vortrags der Erfahrungen hat S. Gerovitch
in \cite{Gerovitch1996} hinreichend genau analysiert, so dass dies hier
unberücksichtigt bleiben und auf die nüchterne Feststellung reduziert werden
kann, dass sich die Grundlagen dieser impliziten Begriffsbildungsprozesse im
Rahmen des sozio-technische ökonomische System einer kapitalistischen
Wirtschaftsordnung als Obersystem bewegen (westlicher Prägung füge ich hinzu,
da die Übertragbarkeit auf stärker autokratisch geprägte Wirtschaftsordnungen
wie etwa in China oder Russland zusätzliche Betrachtungen erfordert hätte).
In Wirklichkeit ist die Kontextualisierung noch enger gezogen, wie die Analyse
der Beispiele zeigt -- eine Unterscheidung zwischen Industrieanlagenbau,
Werkzeugmaschinenbau und Konsumgüterproduktion, wie sie etwa in
volkswirtschaftlichen Analysen üblich ist, wird nicht vorgenommen, gleichwohl
grundsätzlich die Perspektive einer am größeren Markt orientierten
Produktgängigkeit der untersuchten \emph{technischen Systeme} eingenommen.

Damit ist der Gegenstandsbereich der technischen Systeme hinreichend umrissen,
deren „Evolution“ in \cite{TESE2018} untersucht wird. Zugleich wird M. Rubins
Position verständlich, dass in einem solchen Kontext die Unterscheidung von
technischen und sozio-technischen Systemen „offensichtlich und wesentlich“
ist, was M. Rubin (private Kommunikation) mit Verweis auf \cite{Rubin2007} und
\cite{Rubin2010} auch selbst klar sieht:
\begin{quote}
  „Bei der Betrachtung eines technischen Systems berücksichtigen wir keine
  anderen bestehenden Beziehungen (soziale, politische, wirtschaftliche,
  Marketing usw.) im System, mit Ausnahme von Objekten und Beziehungen
  technischer Natur. Diese externen (menschlichen, kulturellen) Beziehungen
  können durch zusätzliche Anforderungen oder Einschränkungen an technische
  Objekte ersetzt werden.  Bei der Betrachtung von Systemen als
  sozio-technisch werden eine Reihe technischer Objekte und Zusammenhänge
  berücksichtigt, beispielsweise wenn die TRIZ-Analyse von
  Produktionsunternehmen nicht nur als technisches System (Maschinen, Geräte),
  sondern die Fabrik als sozio-technisches Objekt betrachtet wird:
  Bestellsystem und Marketing, Personalpolitik, Finanzen und die
  wirtschaftliche Lage des Unternehmens, Systeme der Entscheidungsfindung usw.
  Offensichtlich verändert dies den Gegenstand der Überlegungen und die
  Untersuchungsinstrumente grundlegend.“
\end{quote}
Natürlich bedürfen die Begriffe „technisches Objekt“, „technischer
Zusammenhang“ und „Beziehung technischer Natur“, mit denen hier über die
Grenzen zwischen technischen und sozio-technischen Systemen hinweg vermittelt
werden soll, weiterer Präzisierung.

Kehren wir jedoch zu \cite{TESE2018} zurück und untersuchen genauer, auf
welcher Aggregationsgrundlage die „Evolution ingenieur-technischer Systeme“
untersucht wird.  Da wir inzwischen auch ein Obersystem identifiziert haben,
gegen welches die Ausführungen relatiert werden können, wollen wir nun die
TRIZ-Methodik selbst zur Rekonstruktion der Modellierung und damit zur Analyse
der begriffstheoretischen Fundierung von \cite{TESE2018} einsetzen.

Ausgangspunkt ist das sozio-ökonomische System einer industriellen
Produktionsweise. „Der Reichtum dieser Gesellschaften“, so beginnt Marx seine
Analyse eines solchen sozio-ökonomi"|schen Systems in \cite{MEW23}, „erscheint
als eine 'ungeheure Warensammlung', die einzelne Ware als seine
Elementarform. Unsere Untersuchung beginnt daher mit der Analyse der Ware.“
Auch wir starten mit diesem Begriff als einer hochgradigen Abstraktion.  Marx'
Arbeitswertheorie abstrahiert im Begriff der \emph{Ware} bekanntlich von
sämtlichen qualitativen Eigenschaften\footnote{Marx folgt in dieser Methodik
  der Begriffsentwicklung Hegel. Rainer Thiel \cite[S. 190]{Thiel2007}
  schreibt dazu: „Hegels Begriffsentwicklung beginnt mit dem 'Sein'. Und was
  tut der Dialektiker Hegel? Er entwickelt – mit der Sturheit eines Schelms,
  wie ein Computer – den Inhalt des 'reinen Seins': 'Sein, reines Sein, --
  ohne alle weitere Bestimmung. In seiner unbestimmten Unmittelbarkeit ist es
  nur sich selbst gleich und auch nicht ungleich gegen Anderes, hat keine
  Verschiedenheit innerhalb seiner, noch nach außen. Durch irgendeine
  Bestimmung oder Inhalt, der in ihm unterschieden, oder wodurch es als
  unterschieden von einem Andern gesetzt würde, würde es nicht in seiner
  Reinheit festgehalten. Es ist die reine Unbestimmtheit und Leere. – Es ist
  nichts in ihm anzuschauen, wenn von Anschauen hier gesprochen werden kann;
  oder es ist nur dies reine, leere Anschauen selbst. Es ist ebensowenig etwas
  in ihm zu denken, oder es ist ebenso nur dies leere Denken. Das Sein, das
  unbestimmte, unmittelbare, ist in der Tat Nichts, und nicht mehr noch
  weniger als Nichts'.“ } außer der einen, Produkt menschlicher Arbeit zu
sein. Erst auf einere solchen Abstraktionsebene werden konkrete Waren global
austauschbar und konstituieren damit einen globalen Markt als
\emph{Verhältnis} -- Feld in der TRIZ-Terminologie -- zwischen diesen
Warenkonkreta, den \emph{Tauschwert}.

Darum geht es in \cite{TESE2018} allerdings nicht, sondern um funktionale
Qualitäten konkreter Warengruppen wie Waschmaschinen oder Federhalter. Das
allgemeine Konkurrenzverhältnis abstrakter Waren zerfällt dabei in konkretere
Konkurrenzverhältnisse einzelner Warengruppen auf Einzelmärkten, die in
\cite{TESE2018} als „market pull“ die Hauptfunktion des Werkzeugs „Markt“
sind, welche die Objekte „engineering systems“ zu „nützlichen Produkten“
umformen -- den TRIZ-Begriff"|lichkeiten von \cite{TT} folgend. Mit der
Marktgängigkeit von Produkten ist eine erste Struktureinheit im Obersystem
identifiziert -- konkrete Märkte, auf denen \emph{konkrete} Waren mit
\emph{spezifischen} funktionalen Eigenschaften -- \emph{Gebrauchswerten} in
der Terminologie von Marx -- miteinander im Wettbewerb stehen.  Die
\emph{Funktion} Konkurrenzverhältnis des \emph{Werkzeugs} Markt wird in
\cite{TESE2018} seinerseits als \emph{Werkzeug} mit Technologie formender
Funktionalität betrachtet.  Dieser Gedanke soll nun genauer entwickelt werden.

Jede Ware ist „ein Ganzes vieler Eigenschaften und kann daher nach
verschiedenen Seiten nützlich sein“ \cite[S. 49]{MEW23}. Jede konkrete Ware
ist damit selbst ein technisches System im oben entwickelten Verständnis, wenn
sie als durch ihre Spezifikation gegebenes Ensemble „nützlicher“
Funktionalitäten betrachtet wird. Dieses Ensemble nützlicher Eigenschaften
bestimmt aber auch die Möglichkeiten und Grenzen der Substituierbarkeit von
Waren im gesamtgesellschaftlichen technologischen Prozess. Jene Grenzen führen
zu einer Stratifizierung „des Markts“ in konkrete Märkte für konkrete
Warengruppen. Im sozio-ökonomischen Obersystem haben wir damit zwischen dem
Werkzeug-Template „Markt“ und konkreten realweltlichen Ausprägungen dieses
Werkzeugs zu unterscheiden. Diese realweltliche Struktur der
\emph{Technologiemärkte} ist der in \cite{TESE2018} ausgeführten
S-Kurven-Analyse vorgängig und wird dort implizit als gegeben vorausgesetzt.
Jeder solche Technologiemarkt ist durch ein spezifisches Bündel technischer
Funkionalitäten charakterisiert, wobei \cite{TESE2018} mit dem Ansatz des MPV
(main parameter of value) postuliert, dass sich ein solcher Markt um einen
speziellen technischen Parameter herum gruppiert, der für die Wertschöpfung
von besonderer Bedeutung ist.

Damit ist aber das Erfordernis einer weiteren Abstraktion verbunden, denn
konkrete Waren, im obigen Sinne als technische Systeme, als \emph{konkrete}
Bündel technischer Funktionalitäten verstanden, sind prinzipiell geeignet, auf
\emph{mehreren} derartigen Technologiemärkten gehandelt zu werden und werden
dies praktisch auch. Ein solcher Technologiemarkt wird auch weniger durch die
auf ihm gehandelten Waren bestimmt als durch die diese Waren produzierenden
Unternehmen. Darum geht es den Autoren von \cite{TESE2018} auch, wenn sie
Altschullers „S-curve analysis“ zu einer „pragmatic S-curve analysis“
weiterentwickeln.  Damit verschiebt sich aber das in \cite{TESE2018}
aufgerufene Abstraktionserfordernis von einem MPV als eigenständigem
Charakteristikum zur \emph{unternehmerischen Fähigkeit}, technische Artefakte
mit diesem MPV in angemessenem Preis-Leistungs-Verhältnis \emph{zu
  produzieren}.  Damit wird auch deutlich, dass sich auf jenen
Technologiemärkten zwar „Produkte voneinander unabhängig betriebner
Privatarbeiten“ begegnen, die Produzenten aber \emph{nicht} „erst in
gesellschaftlichen Kontakt treten durch den Austausch ihrer Arbeitsprodukte“
\cite[S. 87]{MEW23} in dem Sinne, dass das Spannungsverhältnis zwischen
begründeten Erwartungen und erfahrenen Ergebnissen der Konditionen
\emph{früheren} Austauschs ihrer Arbeitsprodukte die Dynamik jenes
Technologiemarkts bestimmt.  

Gegenstand technologischer Evolution sind damit aber diese technologischen
Produktionsbedingungen selbst. Dies wird auch in \cite{TESE2018} so gesehen,
denn die im Buch beschriebenen Handlungsoptionen beziehen sich auf die
Organisation entsprechender Innovationsprozesse in Unternehmen. Damit kann
aber das Obersystem in unserer TRIZ-Analyse der begriff"|lichen Grundlagen von
\cite{TESE2018} weiter eingeschränkt werden auf die strategischen
Führungsstrukturen von Unternehmen, in denen die Innovationsprozesse praktisch
gestaltet werden. Auch hierbei haben wir es mit der Dualität von
System-Template -- gängigen gesellschaftlichen Verfahrensweisen zur
Organisation von Innovationsprozessen -- und konkreten realweltlichen
System"|ausprägungen in den einzelnen Unternehmen zu tun. Die
\emph{Hauptfunktion} jener Strukturen im Unternehmen ist die Organisation des
Innovationsprozesses in enger Verbindung mit der allgemeinen
Geschäftsstrategie.  Dieser Prozess selbst wird vom strategischen Management
entschieden und verantwortet, das hierzu die \emph{widersprüchlichen
  Anforderungen} verschiedener Unternehmensteile (R\&D, Vertrieb, Finanzen,
Controlling, SCM, CRM) unter einen Hut zu bringen hat. Die in \cite{TESE2018}
zusammengetragenen Empfehlungen sind \emph{ein} Aspekt in diesem komplexen
Abwägungsprozess, eine Methodik zwischen „technology push“ und „market pull“
ist mit Blick etwa auf die Ausführungen in \cite{Preez2006}, wo es zentral um
das Konzept eines „Fugle Innovation Process“ geht, eher auf dem Niveau der
1960er Jahre anzusiedeln. Ebenda wird in fig.~3 mit „state of the art in
science and technology“ neben den „needs of society and marketplace“ ein
weiteres Obersystem in Stellung gebracht, das auch  bei Patenterteilungen mit
den Begriffen „Stand der Technik“ und „Erfindungshöhe“ eine wichtige Rolle
spielt.

Wir haben damit bereits \emph{drei} sozio-technische Obersysteme (Ökonomie,
Innovationsmanagement, Wissenschaft und Technologie) mit jeweils eigenen
Strukturen, Komponenten, Beschrei"|bungs- und Vollzugsformen identifiziert, die
in der einen oder anderen Weise zur in \cite{TESE2018} behandelten Thematik in
Bezug stehen, ohne damit der uns interessierenden Präzisierung des Begriffs
\emph{technisches System} -- besonders auch in Abgrenzung zum Begriff
\emph{sozio-technisches System} -- näher gekommen zu sein. Die TRIZ-Methodik
empfiehlt in einem solchen Fall, den Zugang von einer anderen Seite neu zu
versuchen, vorher aber die Erkenntnisse aus dem Fehlversuch zu fixieren.
Diesbezüglich sind zwei Aspekte interessant:

\paragraph{1)}
Der Anspruch von \cite{TESE2018}, (auch) einen innovationsmethodischen Beitrag
zu leisten, setzt auf weitgehend überholten innovationsmethodischen Konzepten
auf, eine „pragmatic S-curve analysis“ muss ihre Passfähigkeit zu moderneren
innovationsmethodischen Konzepten erst noch zeigen.

\paragraph{2)}
Wir haben \emph{mehrere} Obersysteme identifiziert, womit noch einmal deutlich
wird, dass der Begriff \emph{Obersystem} nicht immersiv zu denken ist und
nicht mit dem Begriff \emph{Umwelt} verwechselt werden darf. Obersysteme sind
spezifische Nachbarsysteme mit eigener Sprache und Logik. Die Beziehung
Obersystem -- System ist dieselbe wie die Beziehung System -- Komponente und
oben hinreichend genau beschrieben: Es handelt sich um zwei verschiedene
Betrachtungsperspektiven auf die „Totalität der Welt“ mit zwei verschiedenen
Begriffen des \emph{Wesentlichen} und damit aus zwei verschiedenen
Reduktionsperspektiven. Ins Obersystem geht das System allein durch seine
Spezifikation (Beschreibungsdimension) und das Versprechen
spezifikationskonformer Leistung (Vollzugsdimension) ein. Wie im
Konzertbeispiel wird aus dem \emph{gegebenen} Können der Musiker auf der Ebene
des Orchesters die Interpretation des Musikstücks „produziert“.  Die
Leistungs\emph{fähigkeit} der Teilsysteme bestimmt dabei wesentlich das
Leistungs\emph{potenzial} des Obersystems -- die Interpretation von Mozarts
Klavierkonzert KV 491, die Alexander Shelley mit dem Leipziger
Gewandhausorchester und Gabriela Montero als Solistin vorgelegt hat, wäre mit
einem Laienorchester nicht möglich gewesen. Aus der Perspektive des Systems
agiert das Obersystem ebenfalls funktional: Die Schnittstelle des Systems
definiert in der Beschreibungsdimension eine erwartete Input- oder
Durchsatzleistung in Quantität, Qualität und Struktur, die dem Funktionieren
des Systems (zur Laufzeit) vorgängig ist, das Obersystem stellt in der
Vollzugsdimension diese Voraussetzungen sicher. Eine solche Strukturierung
bringt die aus der Mathematik bekannten deduktiven Wenn-Dann-Beziehungen mit
realweltlichen Prozessen in Verbindung, wobei der modus ponens in
realweltlichen Systemen eine deutlich komplexere Semantik hat als in der
Mathematik, wo dieser bekanntlich ausschließlich als logische Klammer eine
Rolle spielt, wenn aus kleineren Wenn-Dann-Beziehungen komplexere solche
Beziehungen (Lemmata, Sätze, Theoreme bis zum Beweis eines der
Millenium-Probleme) regelgerecht deduziert werden sollen.  Auch hier hat
dieser Reduktionsmechanismus einen klaren Zweck -- arbeitsteilige Reduktion
von Komplexität.

\section{Zum Evolutionsbegriff in \cite{TESE2018}.\\ Die ideengeschichtliche
  Dimension}

Kehren wir jedoch zu \cite{TESE2018} und unserer Frage nach der Fundierung des
Begriffs \emph{technisches System} zurück. Dem Rat der TRIZ-Methodik folgend,
es mit einem anderen Ansatz zu versuchen, wenden wir uns dem „technology push“
\cite[S. 2]{TESE2018} zu, „that creates new systems, products, and services“
(HGG -- indeed „creates“?) „that are not yet required by the market“, um aus
dieser Perspektive ein System zu entwickeln. Zunächst sind Zweck und Nutzen
eines solchen Systems zu bestimmen.

Aus der Perspektive des „Coupling Model of Innovation“ in
\cite[fig. 3]{Preez2006} ist dieses System als „Ideengenerator“ die sprudelnde
Quelle, deren Produkte das formende Flussbett des „state of the art in science
and technology“ durchfließen, um sich an dessen Ende durch einen „technology
push“ in „commercial products“ zu verwandeln.  Die „Ideenquelle“ wird (ebenda)
durch einen „market pull“ angetrieben, der aus den „needs of society and the
marketplace“ gespeist wird.  Alle diese Termini sind deutlich mit den
TRIZ-Abstrakta „Werkzeug“, „Aktion“, „Objekt“ und „Produkt“ sowie dem Konzept
der Systemtransformation etwa nach \cite{TT} relatierbar, wären als solche
aber nicht Teil der Analyse des uns interessierenden Systems, sondern des
Obersystems „Innovationsmanagement“. In der Tat erfüllt dieses
Begriffs"|universum dessen Reduktionserfordernis „auf das Wesentliche“, in
welches die „Ideenquelle“ als Komponente allein mit ihrer funktionalen
Spezifikation eingeht, von deren genauem inneren Funktionieren abstrahiert
wird.

Das ist in \cite{TESE2018} natürlich anders, denn mit der Untersuchung der
Evolution ingenieur-technischer Systeme soll ja gerade die ideengeschichtliche
Dimension derartiger Prozesse untersucht werden.  Zu diesem Zweck wird ebenda
auch bereits in der Überschrift von Kapitel 1 ein wesentlicher Pfeil in
\cite[fig. 3]{Preez2006} umgedreht: „Technology Pull: Beyond Technology Push
and Market Pull“.  In der Theorie des Obersystems „Innovationsmanagement“
würde dann das System, das wir im Weiteren mit dem provisorischen Namen
„Ideengenerator“ markieren wollen, aus den beiden Systemen „Bedarf“ (als
„market pull“) und „Wissenschaft und Technik“ (als „technology pull“)
gespeist, um seinerseits das System „Unternehmen“ über dessen Teilsystem
„Development“ mit Input zu versorgen. Aus der Perspektive dieses Systems
„Ideengenerator“ sind die Beschreibungsverhältnisse umzukehren, denn Outputs
des Obersystems in die Komponente werden Inputs der Komponente und umgekehrt.
Genauer ist die in unserer TRIZ-Analyse anstehende Erforschung der inneren
Struktur und Funktionsweise des Systems „Ideengenerator“ auf die Erforschung
der Beziehung zwischen den drei \emph{Komponenten}\footnote{Oder
  „Nachbarsysteme“? -- Nein. Nachbarsysteme wären alle vier nur aus der
  Perspektive eines -- bisher allerdings fiktiven -- Obersystems.} „Bedarf“,
„Technologie“ und „Unternehmen“ auszurichten, die -- nach dem oben
entwickelten Systembegriff -- mit ihren \emph{gegebenen} funktionalen
Spezifika Input, Output und Durchsatz die innere Struktur des Systems
„Ideengenerator“ bestimmen. So weit eine erste Analyse des Vorgehens der
Autoren von \cite{TESE2018} aus dieser neuen Perspektive. Sie passt deutlich
besser auf die Ausführungen in \cite[Kap. 1]{TESE2018} als der Ansatz über
Innovationsmanagement, denn die Autoren betonen „parameters of the technology
of the system are linked to parameters of the market“ -- der Einfluss der
beiden Komponenten wird also als nicht unabhängig voneinander betrachtet,
sondern lässt sich auf der Ebene von Parameterkopplungen fassen (so jedenfalls
die Autoren).

\section{Component Software. Funktion und Verhalten}

\section{Komponenten und Objekte. Zustände}

Der Zusammenhang zwischen einer Komponenten als Konzept und den realweltlich
verbauten Komponenteninstanzen, ist komplex, da die produktiven Strukturen der
Herstellung und des Einsatzes dieser Komponenteninstanzen gewöhnlich
auseinanderfallen, die Komponenteninstanzen nach der Herstellung also
verschickt und an ihrem Einsatzort für den konkreten Gebrauch vorbereitet und
verbaut werden müssen. In der Theorie einer \emph{Software aus Komponenten}
werden dabei die drei Phasen \emph{deploy, install, configure} deutlich
unterschieden.

\section{Bisheriger Text}

HGG: Die Ersetzung des Menschen als Gesetz der technischen Entwicklung wurzelt
in einem sehr merkwürdigen Verständnis des Begriffs \emph{Technik}, welches
das Offensichtliche vergisst -- es gibt keine \emph{technischen Systeme},
sondern nur \emph{technosoziale Systeme}.

Michail S. Rubin präzisierte in einer PM vom 10.11.2019 seine Position wie
folgt:
\begin{quote}
  Dies erfordert eine gesonderte Diskussion. Wir verweisen auf die Arbeit von
  Lubomirsky und Litvin, die sich auf die Verdrängung des Menschen aus
  technischen System bezieht.  Wir sind uns einig, dass dieses Phänomen kein
  Gesetz ist, sondern ein Trend, der einem anderen Gesetz folgt: dem Gesetz
  der Erhöhung der Autonomie von Systemen.  Wir haben die Liste von Gesetzen
  und Trends in der Ausschreibung entsprechend modifiziert. Sie haben absolut
  Recht, dass technische Systeme nicht unabhängig sind in ihrer Entwicklung
  und allgemeiner sozio-technische Systeme betrachtet werden müssen. Gesetze
  der Entwicklung sozio-technischer Systeme unterscheiden sich aber von den
  Gesetzen der Entwicklung technischer Systeme. Für rein technische Systeme
  kann wirklich der Trend der schrittweisen Herauslösung menschlicher
  Beteiligung beobachtet werden. Statt eines Ruderbootes erscheint ein Boot
  mit einem Motor. Die ganze industrielle Revolution des 17. Jahrhunderts ist
  auf der Verdrängung des Menschen durch Motoren und Maschinen aufgebaut. Die
  nächste technologische Revolution ist mit der Verdrängung des Menschen aus
  dem Bereich der Kontrolle durch Automatisierung und Computer verbunden. Das
  heißt aber nicht, dass aus technologischer Sicht der Mensch aus dem
  sozio-technischen System verdrängt wird. Im Gegenteil, der Mensch bleibt die
  Hauptanforderungsquelle für technische Systeme. Aber diese Anforderungen
  werden zunehmend ohne menschliches Eingreifen erfüllt. Dieser Trend ist auch
  charakteristisch für das Kino als technisches System\footnote{Das Thema der
    Aufgaben des TRIZ-Cups.}. Es ist klar, dass der Mensch weder aus dem
  Prozess der Schaffung von Filmwerken, noch von Kunstwerken, noch aus dem
  Prozess des Konsums von Kinoprodukten herausgedrängt wird, er bleibt das
  Zentrum all dieser Prozesse.
\end{quote}

\section{Was sind technische Systeme?}


\subsection{Kommentar von Nikolay Shpakovski, 8.12.2019}

Gesetze und Entwicklungslinien werden aktiv bei der Lösung von situativen und
prognostischen Aufgaben eingesetzt. Es geht um das System, aber sehr wenig,
und das habe ich schon lange verstanden.

In letzter Zeit denke ich oft an das Konzept des „technischen Systems“. Dieses
Konzept ist ein wichtiger Teil des Prozesses zur Lösung von Problemen nach
unserem Ansatz. Ich finde nichts Falsches am Ansatz des VDI\footnote{Auf
  Facebook schrieb ich dazu: Als zentrale Frage steht für mich, was überhaupt
  ein „Technisches System“ ist. Ist dieser Begriff in der Mehrzahl, wie im
  TRIZ-Kontext wie selbstverständlich gebraucht, überhaupt sinnvoll
  verwendbar? Der VDI -- Verein Deutscher Ingenieure -- als
  Standesorganisation, der in der VDI-Richtlinie 3780 den Technikbegriff
  normiert, ist in dieser Frage uneins, indem er von einer „Menge von
  Systemen“ spricht und Technik in folgenden drei Dimensionen betrachtet: 
  \begin{itemize}
  \item Menge der nutzenorientierten, künstlichen, gegenständlichen Gebilde
    (Artefakte oder Sachsysteme);
  \item Menge menschlicher Handlungen und Einrichtungen, in denen Sachsysteme
    entstehen und
  \item Menge menschlicher Handlungen, in denen Sachsysteme verwendet werden.
  \end{itemize}}, alles stimmt, alles auf der Welt kann als System betrachtet
werden.  Jedes System kann als „System von Systemen“ dargestellt werden, wir
wählen einfach irgendeine Ebene aus und sagen -- das ist ein System.  Dann
ergibt sich sofort die Möglichkeit zu sagen, dass es Obersysteme und
Subsysteme gibt.

Du hast eine konkrete Frage gestellt - was ist der Unterschied zwischen den
Konzepten „System -- Subsysteme“ und „System -- Komponenten“. Es ist einfach
-- die Komponente ist ein noch nicht systematisierter Teil des Systems, ein
potenzielles Teilsystem.
\begin{quote}
  Anmerkung HGG: Das widerspricht aber dem Verständnis der
  Komponententechnologie, nach dem die Komponenten zur Bauzeit des Systems,
  also \emph{vor} dessen Betrieb vorhanden sein müssen.
\end{quote}

Das Konzept des „technischen Systems“ ist in der TRIZ schrecklich verstrickt.
Als technisches System wird eine Reihe von Mechanismen betrachtet, die eine
neue Qualität ergeben, zum Beispiel ein Auto, ein Stift, eine Uhr. Als
technisches System wird ein System zur Durchführung einiger Funktionen
bezeichnet, beispielsweise zum Transport von Gütern, wozu außer dem Auto noch
viel mehr gehört. Das ist nicht schlimm, das Problem ist, dass diese
Definitionen kühn vermischt werden, was zu schrecklicher Verwirrung führt.
Den Fahrer in das System Auto einbeziehen oder nicht? Was ist mit Benzin? Ist
Luft ein Teil des Autos oder nicht? Menschen leben mit diesen Verwirrungen
gut, bauen ganze Theorien und führen Seminare durch, was diese Verwirrungen
nur noch verstärkt.

Für mich unterscheide ich
\begin{enumerate}
\item ein technisches System (systematisiertes technisches Objekt, eine
  Maschine auf dem Lager),
\item ein funktionierendes System (was im Patent als „Maschine in Arbeit“
  bezeichnet wird),
\item ein nützliches technisches System (das, was ein nützliches Produkt
  herstellt).
\end{enumerate}

Natürlich verwirrt das Wort „technisch“ hier viel, aber in dieser Situation
ist das so zu verstehen, dass ein technisches System ein System ist, das Bezug
zur Technik (Ingenieurwesen) hat oder zur Technik des Durchführens irgendeiner
nützlichen Handlung. Wirf besser dieses Wort komplett weg. Das Wichtigste, das
Nützlichste zur Lösung des Problems ist ein nützliches System. Auf dieser
Ebene verliert das Wort „technisch“ seine Bedeutung, weil es ein Elektriker
sein kann, der eine Glühbirne einsetzt oder ein Raumschiff geht in die
Umlaufbahn oder ein Anwalt oder ein Computerprogramm. Das Hauptkriterium ist,
ob dies ein nützliches Ergebnis ergibt oder es sich um „Mozhaiskis
nicht-fliegendes Flugzeug“ handelt\footnote{Ein im russischen Kontext
  berühmtes Beispiel ähnlich dem „Schneider von Ulm“ im Deutschen, siehe
  \url{https://de.wikipedia.org/wiki/Geschichte_der_Luftfahrt}.}?

\begin{thebibliography}{xxx}
\bibitem{Barkleit2000} G. Barkleit (2000). Mikroelektronik in der
  DDR. Dresden, 2000.
\bibitem{Bertalanffy1950} Ludwig von Bertalanffy (1950). An outline of General
  System Theory. The British Journal for the Philosophy of Science, vol. I.2,
  134–165.
\bibitem{Friedli2013} Thomas Friedli, Stefan Thomas, Andreas Mundt (2013).
  Management globaler Produktionsnetzwerke. Strategie – Konfiguration –
  Koordination. Hanser, München. ISBN: 978-3-446-43449-3
\bibitem{KFK2000} Klaus Fuchs-Kittowski (2000).
  Wissens-Ko-ProduktionVerarbeitung, Verteilung und Entstehung von
  Informationen in kreativ-lernenden Organisationen.\\ In: Fuchs-Kittowski
  u.a. (Hrsg.). Organisationsinformatik und Digitale Bibliothek in der
  Wissenschaft. Wissenschaftsforschung, Jahrbuch 2000. Gesellschaft für
  Wissenschaftsforschung, Berlin.
  \url{http://www.wissenschaftsforschung.de/JB00_9-88.pdf}
\bibitem{Gerovitch1996} Slava Gerovitch (1996). Perestroika of the History of
  Technology and Science in the USSR: Changes in the Discourse. Technology and
  Culture, Vol. 37.1, S. 102--134.
\bibitem{Goldberg2016} Jörg Goldberg, André Leisewitz (2016). Umbruch der
  globalen Konzernstrukturen.\\ Z 108, S. 8--19.
\bibitem{Graebe2018} Hans-Gert Gräbe (2018).  12. Interdisziplinäres Gespräch
  \emph{Nachhaltigkeit und technische Ökosysteme}. Leipzig, 02.02.2018. 
    \url{http://mint-leipzig.de/2018-02-02.html}.
\bibitem{Graebe2019} Hans-Gert Gräbe (2019).  A discussion about TRIZ
    and system thinking reported in my Open Discovery Blog.
    \url{https://wumm-project.github.io/2019-08-07}.
\bibitem{Graebe2020} Hans-Gert Gräbe (2020). Reader zum 16. Interdisziplinären
  Gespräch \emph{Das Konzept Resilienz als emergente Eigenschaft in offenen
    Systemen} am 7.2.2020 in Leipzig.
  \url{http://mint-leipzig.de/2020-02-07/Reader.pdf}.
\bibitem{Holling2000} C.S. Holling (2000). Understanding the Complexity of
  Economic, Ecological, and Social Systems. In: Ecosystems (2001) 4, 390–405.
\bibitem{Jacobasch2019} Gisela Jacobasch (2019). Bienensterben -- Ursachen und
  Folgen.  Leibniz Online 37 (2019).
  \url{https://leibnizsozietaet.de/bienensterben-ursachen-und-folgen/}
\bibitem{KoltzeSouchkov2017} Karl Koltze, Valeri Souchkov (2017).
  Systematische Innovation.\\ Hanser, München. Zweite Auf"|lage. ISBN
  978-3-446-45127-8.
\bibitem{Kropik2009} Markus Kropik (2009). Produktionsleitsysteme in der
    Automobilfertigung. Springer, Dordrecht.\\ ISBN 978-3-540-88991-5.
\bibitem{TBK-2007} S. Litvin, V. Petrov, M. Rubin (2007). TRIZ Body of
  Knowledge. \\ \url{https://triz-summit.ru/en/203941}.
\bibitem{TESE2018} Alexander Lyubomirskiy, Simon Litvin, Sergey Ikovenko,
  Christian M. Thurnes, Robert Adunka (2018). Trends of Engineering System
  Evolution. Eigenverlag, Sulzbach-Rosenberg.  ISBN 978-3-00-059846-3.
\bibitem{MEW15} Friedrich Engels (MEW 15). Die Geschichte des gezogenen
  Gewehrs.  MEW 15, S. 195--226. Dietz Verlag, Berlin.
\bibitem{MEW23} Karl Marx (MEW 23). Das Kapital, Band 1. MEW 23. Dietz Verlag,
  Berlin.
\bibitem{MEW42} Karl Marx (MEW 42). Grundrisse der Kritik der politischen
  Ökonomie.  MEW 42. Dietz Verlag, Berlin.
\bibitem{Pohl2005} Klaus Pohl, Günter Böckle, Frank J. van der Linden (2005).
  Software Product Line Engineering. Foundations, Principles and Techniques.
  Springer. ISBN 978-3-540-28901-2
\bibitem{Preez2006} Niek D Du Preez, Louis Louw, Heinz Essmann (2006). An
  innovation process model for improving innovation capability.  Journal of
  high technology management research, vol 17, 1--24.
\bibitem{Rubin2007} Michail S. Rubin (2007). \foreignlanguage{russian}{О
  выборе задач в социально-технических системах}. (Über die Wahl von Aufgaben
  in sozial-technischen Systemen). In: \foreignlanguage{russian}{ТРИЗ Анализ.
    Методы исследования проблемных ситуаций и выявления инновационных
    задач}. (TRIZ-Analyse. Methoden zur Untersuchung von Problemsituationen
  und zur Identifizierung innovativer Aufgaben). Hrsg. von S.S. Litvin,
  V.M. Petrov, M.S. Rubin. \foreignlanguage{russian}{Библиотека Саммита
    Разработчиков ТРИЗ}, Moskau. S. 35--46.
  \url{https://www.trizland.ru/trizba/pdf-books/TRIZ-summit2007.pdf}.
\bibitem{Rubin2010} Michail S. Rubin (2010).
  \foreignlanguage{russian}{Филогенез социокультурных систем. Секреты развития
    цивилизаций}.  (Phylogenese soziokultureller Systeme. Geheimnisse der
  Zivilisationsentwicklung).
  \url{http://www.temm.ru/en/section.php?docId=4472}.
\bibitem{Shpakovsky2010} Nikolay Shpakovsky (2010).  Tree of Technology
  Еvolution. Forum, Moscow.
\bibitem{Szyperski2002} Clemens Szyperski (2002). Component Software: Beyond
  Object-Oriented Programming. ISBN: 978-0-321-75302-1.
\bibitem{TT} Target Invention (2020). TRIZ Trainer.
  \url{https://triztrainer.ru}.
\bibitem{Thiel2007} Rainer Thiel (2007). Zur Lehrbarkeit dialektischen Denkens
  – Chance der Philosophie, Mathematik und Kybernetik helfen. In: Klaus
  Fuchs-Kittowski, Rainer E. Zimmermann (Hrsg.). Kybernetik, evolutionäre
  Systemtheorie und Dialektik. Trafo Verlag, Berlin 2012, ISBN:
  978-3-89626-919-5, S. 185--202
\bibitem{VDMA2019} VDMA. Maschinenbau in Zahl und Bild 2019. 
\bibitem{Vernadsky1997} Vladimir I. Vernadsky (1997, Original 1936--38).
  Scientific Thought as a Planetary Phenomenon.
  \url{https://wumm-project.github.io/Texts.html}
\bibitem{Weller2008} W. Weller (2008). Automatisierungstechnik im
  Überblick. Was ist, was kann Automatisierungstechnik? Beuth, Berlin. ISBN
  978-3-410-16760-0.

\end{thebibliography}
\end{document}


Mehr noch lehrt die Theorie Dynamischer Systeme, dass die Kopplung zwischen
Systemen nicht so sehr allein von Durchsatzraten bestimmt wird, sondern in
ihrer Wirkung stark von zeitlichen Regimes in Form von Resonanzen und
Dissonanzen bestimmt sein können.  Damit kann das Zusammenspiel von System und
Systemkomponenten stark von der Verschränkung von Mikro- und Makrodynamik auf
kurzwelligen (Komponenten) und langwelligen (System) Skalen abhängen.  Ein
wesentliches viertes Charakteristikum \emph{autonom funktionierender
  technischer Systeme} ist eine Entkopplung dieser Systemdynamiken, im
Kontrast etwa zum TRIZ-Prinzip 19 der periodischen Wirkung, das auf die
Ausnutzung entsprechender Kopplungsphänomene gerichtet ist.
