\documentclass[11pt,a4paper]{article}
\usepackage{od}
%\usepackage[T1,T2A]{fontenc}
\usepackage[utf8]{inputenc}
\usepackage[main=english,russian]{babel}

\title{Men and their Technical Systems} 
\author{Hans-Gert Gräbe, Leipzig}
\date{Version of May 20, 2020}
\begin{document}
\maketitle

\begin{abstract}
  With the concept of a \emph{technical system} the whole TRIZ theory corpus
  revolves around a term that is not very precisely defined in the TRIZ
  literature, but left to a ``common sense''.  In this paper an attempt is
  made to determine how far a notion of \emph{technical system} takes in this
  theoretical context and how it can be related with approaches from
  neighbouring theory corpusses.  It turns out that focusing on an artifact
  dimension of technology, as defined by the term \emph{technical system},
  blocks the view on essential \emph{relational phenomena} that are inherent
  to a \emph{world of technical systems} and a notion of \emph{technical
    principle} is more suitable for the analysis of such relational phenomena.
\end{abstract}

\section{Introduction}

The whole is more than the sum of its parts. In \cite{Graebe2020a} I show that
it is even much more than this sum, because in the system's \emph{relations}
the parts multiply and do not just add up.  In the TRIZ Body of Knowledge
\cite{TBK-2007} the conceptual foundation of the own theory is developed only
half-heartedly. Especially for the question, what is a \emph{technical
  system}, the ``commom sense'' is addressed. The diversity of this ``common
sense'' was visible in a Facebook discussion \cite{Graebe2019} in August
2019.  This is of course no basis for scientific work.

In this paper we develop an approximation to the notion \emph{technical
  system} and ask whether such a term carries at all, in order to analyse the
\emph{world of technical systems} more closely.  The rarely surprising answer
is \emph{no}, because the whole is more than sum of its parts.  Relational
conditions in the \emph{world of technical systems} become rather visible in
the notion of \emph{technical principle} than in the concept \emph{technical
  system}. To that extent, the approach in \cite{Shpakovsky2010} is much
better suited to analyse evolution in the world of technical systems as the
approach in the ``official MATRIZ document'' \cite{TESE2018}.  The term
\emph{principle} is not to be misunderstood here as TRIZ principle, because
the unfortunate English and German translation of the Russian original
\foreignlanguage{russian}{``Приём''} would be better changed to ``approach''
or ``design pattern''.

For a long version with a more detailed argumentation (in German) we refer to
\cite{Graebe2020b}.

\section{Laws of Evolution of Technical Systems}

Laws of evolution of technical systems in different versions are one of the
pillars of a ``TRIZ Body of Knowlegde'' \cite{TBK-2007} and regularly contain
a \emph{law of displacement of humans from technical systems}.  In this paper
I refer to \cite{TESE2018} as reference, where influential TRIZ theorists with
the authority of MATRIZ compiled a systematization of the current state of
debates on such ``Trends of Engineering Systems Evolution'' (TESE).

The opposite view was formulated in the cybernetics discourse of the 1960s to
1980s \cite[p. 10]{KFK2000}: `` What is the position of man in the highly
complex information-techno\-logical system? Our answer to that question was
always: Man is the only creative productive force, it must be and remain the
\emph{subject} of development.  Therefore, the concept of full automation,
according to which the human is to be gradually eliminated from the process,
misses the point!''

The problems of such a ``concept of full automation'', a world of
``automatically moving machines'' meanwhile are becoming visible in an
ecological crisis of planetary proportions. The displacement thesis itself is
perceived as a direct threat, that can be formulated as thesis itself:
\begin{quote}\it
  \textbf{Thesis 1:} An (apparent) displacement of man from technical systems
  points to an under-complex, existentially dangerous misperception of the
  technical systems under consideration.
\end{quote}

However, this is no longer a technical problem in any way. Harrisburg,
Chernobyl, Fukushima or the climate change request a further examintion of
such contradictory positions.  The ``trends (or laws) of development of
technical systems'' refer to a \emph{specific} conceptual level of abstraction
of descriptions of a world that developes in contradictory forms.  In
particular the ``trends'' are in contradiction to lines of development
extracted on other levels of abstraction.  In short, the ambivalent
relationship to a ``displacement of humans from technical systems'' developed
above in thesis and counterthesis is in no way special just for this trend,
but applies in a similar way to the other nine trends.  TRIZ is a good
methodology to analyze such contradictions if one does not stop at memorizing
these trends only.

\section{Technology and World Changing Practices}

Operation and use of technical systems is certainly today a central element of
world-changing human practices. For this purpose planned and coordinated
action based on a division of labour is required, because using the benefit of
a system requires its operation. Conversely, it makes little sense to operate
a system that is not being used. In computer science this connection is well
known as the connection between defining and calling a function -- calling a
function that has not yet been defined causes a runtime error; the definition
of a function that is never called points to a design error.

Closely linked to this distinction between definition and call of a function
is the distinction between design time and runtime. Such a distinction is even
more important in the real world of using technical systems based on division
of labour -- during the design phase the cooperative interactions are
\emph{planned in principle}, during the runtime \emph{the plan is executed}.
Hence for technical systems one has to distinguish interpersonally
communicated \emph{justified expectations} as \emph{forms of description} and
the \emph{experienced results} as \emph{forms of performance}.

This is not a simple task as the following example of a concert performance
shows. The form of performance, which pleases the listeners, is preceded by
the description form, the agreement on the exact interpretation of the work to
be performed. This agreement on a \emph{joint plan} is itself a
precondition-rich practical process. The requirements result from previous
practices -- such as the \emph{private procedural skills} of the individual
musicians in the mastery of their instruments and the existence of the score
as an established form of description of the concert piece to be
performed. When Alexander Shelley on October 14, 2018 at Leipzig Gewandhaus
went without this score of Mozart's Piano Concerto KV 491 to the director's
podium, it becomes clear that this form of description provided at most the
raw material basis for director and orchestra in the preceding rehearsals to
agree upon a situation-specific special form of description of the performance
form. Even more, the opulent gestures of the director towards the orchestra
show that during these rehearsals also \emph{language} was generated to
transform the results of longer processes of understanding into a compact form
that meets the time-critical requirements of the tempi of performance.  The
mere ``engineering'' dimension was transcended by Gabriela Montero, the
soloist of that evening, with her encore: the audience is asked to sing a
melody, out of which the virtuoso develops an improvization as a form of
performance to which there is no interpersonal communicable description form,
beside the sound recordings of that Gewandhaus evening and the reports from
the enthusiastic listeners. That also here technical mastery was necessary, is
beyond question.

The relationship between men and their technical systems is therefore complex
and can be grasped only in a dialectical perspective of further development of
already existing technical systems, not to inescapably tangle up in unfruitful
hen-egg debates. 

\section{Systems and Components}

In addition to the dimensions of description and performance, for technical
systems the \emph{aspect of reuse} also plays a major role. This applies, at
least on the artifact level, but \emph{not} to the most large technical
systems -- these are \emph{unique specimens}, even though assembled using
standardized components. Also the majority the computer scientist is concerned
with the creation of such unique specimens, because the IT systems that
control such systems are also unique. 

The special features of a technical system result therefore mainly from the
\emph{interplay of components}. For example, the production control systems of
various BMW plants differ significantly \cite{Kropik2009}. The plants were
built at different times taking into account the respective state of the art
and the likewise changing business model of the company. Once such large
technical systems are released they can only be modified to a limited extent
and are therefore, after the corresponding amortization periods, also
consistently decommissioned. Nevertheless, the aspect of reuse also plays a
role in such very different technical systems, but is shifted from the
immediate level of technical artifacts to higher levels of abstraction.

Hence the \emph{concept of a technical system} rooted in a planning and
real-world context has four dimensions
\begin{itemize}[noitemsep]
\item [1.] as a real-world unique specimen (e.g. as a product or a service),
\item [2.] as a description of this real-world unique specimen (e.g. in the
  form of a special product configuration)
\end{itemize}
and for components produced in larger quantities also
\begin{itemize}[noitemsep]
\item [3.] as description of the design of the system template (product
  design) and
\item [4.] as description and operation of the delivery and operating
  structures of the real-world unique specimens of this system produced
  according to this template (as production, quality assurance, delivery,
  operational and maintenance plans).
\end{itemize}
Point 4 in particular hardly plays a role in the TRIZ context, although
neither in the private nor in the business environment technical products are
sustainably demanded for which foreseeable inadequate service is offered.

As a basis for such a delimiting system concept, the submersive concept of
open systems from the theory of dynamical systems \cite{Bertalanffy1950} is
used, which postulates
\begin{itemize}[noitemsep]
\item [1.] an inner demarcation against existing systems (components),
\item [2.] an outer boundary and functionally determined embedding in a
  (functioning) environment and
\item [3.] a (functioning) external throughput that leads to dynamic internal
  structure formation as source of the performance of the system.
\end{itemize}

\emph{Technical systems} in such a setting are systems whose design is
influenced by cooperatively acting people based on the division of labour,
where \emph{existing} technical systems are normatively characterized at
description level by a \emph{specification} of their interfaces and at
performance level by the \emph{guarantee of specification-compliant
  operation}.

We are clearly within the range of standard TRIZ terminology of a
\emph{systems of systems} -- a technical system consists of components, which
in turn are technical systems, whose \emph{functioning} (both in functional
and operational sense) is assumed for the currently considered system level.

The concept of a technical system thus has a clearly epistemic function of
(functional) ``reduction to the essential''. To Einstein the recommendation is
attributed ``to make it as simple as possible but not simpler''. The \emph{law
  of completeness of a system} expresses exactly this thought, however, not as
a \emph{law}, but as an engineering \emph{modeling directive}. The apparent
``natural law'' of the observed dynamics is therefore essentially based on 
\emph{reasonable human action}.

In an approach of ``reduction to the essential'' and ``guaranteed
specification-compliant operation'' human practices are inherently built in,
since only in such a context the terms ``essential'', ``guarantee'' and
``operation'' can be filled in a meaningful way. These essential terms from
the socially determined practical relationship of people are deeply rooted in
the concept generation processes of descriptions of special technical systems
and find their ``natural'' continuation in the special social settings of a
legally constituted societal system.

\section{The World of Technical Systems. Basics}

In the TRIZ literature such conceptual foundations hardly play a role.
Relevant textbooks such as \cite{KS2017} consider the term ``technical
system'' as intuitively given from ``industrial practices'' \cite
[p. 2]{KS2017}, while other terms such as ``process'', ``product'',
``service'', ``resources'' and ``effects'' \cite[p. 6--10]{KS2017} are
carefully introduced. Even the detailed description of the ``evolution of
technical systems'' in 5 laws and 11 trends \cite[Section 4.8]{KS2017} is
based solely on the succinct statement ``The existence of technical evolution
is a central insight of the TRIZ''. 

How the concept of a \emph{technical system} can be further sharpened?  In
\cite{Graebe2020a} we identified ``the system concept as descriptional
focusing to make real-world phenomena accessible for description by
\emph{reduction to the essentials}.'' Such a reduction focuses on the
following three dimensions \cite[p. 18]{Graebe2020a}
\begin{itemize}[noitemsep]
\item [(1)] Outer demarcation of the system against an \emph{environment},
  reduction of these relationships to input/output relationships and
  guaranteed throughput.
\item [(2)] Inner demarcation of the system by combining subareas to
  \emph{components}, whose functioning is reduced to ``behavioural control''
  via input/output relations.
\item [(3)] Reduction of the relations in the system itself to ``causally
  essential'' relationships.
\end{itemize}
Further, it is stated that -- similar to the concert example -- such a
reductive description (explicitly or implicitly) rests on prior descriptions:
\begin{enumerate}
\item[(1)] An at least vague idea about the (working) input/output services of
  the environment.
\item[(2)] A clear idea of the inner workings of the components (beyond the
  pure specification).
\item[(3)] An at least vague idea about causalities in the system itself, that
  precedes the detailed modelling.
\end{enumerate}

The description of planning, design and improvement of technical systems in
such an approach is based on the performance of already existing technical
systems, which are present both in (2) as components and -- from the point of
view of a system in the supersystem -- in (3) as adjacent systems.  Thus
engineering practices are embedded into a \emph{world of technical systems}.
From the special descriptive perspective of a system the components or
neighbouring systems are given with their \emph{specification} only. Such a
\emph{reduction to the essential} appears practically as a shortened way of
reasoning about social normality, what I call \emph{fiction} for short. This
fiction can and does work in daily language use as long as the social
circumstances are in operation, that guarantee the maintenance of the social
normality, i.e., as long as the \emph{operation of the corresponding
  infrastructure} is guaranteed.  Hence technical systems are -- at least in
their performance dimension -- \emph{always} socio-technical systems.

\section{Engineering Systems and Socio-Economic Evolution}

Evolution of engineering systems, as V. Souchkov states in the preface
\cite[p. IX]{TESE2018}, should be considered as ``innovative development since
-- in contrast to nature -- craftsmen and engineers make decisions based on
logic, previous experience, and knowledge of basic principles rather than
chance.'' The concentration on ``craftsmen and engineers'' points to narrow
practices, from which the systematization in \cite{TESE2018} is drawn.

To identify lines of development, in \cite{TESE2018} the term \emph{technical
  system}\footnote{We do not distinguish between the newly introduced term
  ``engineering system'' and this old term being in use together with its
  abbreviation TS in the TRIZ literature for many years.} is embedded between
``technology push'' and ``market pull'' as ``simple means for understanding
the advancement of man-made systems'' \cite[p. 1]{TESE2018}. The reference to
the even more vague term ``man-made systems'' (not men-made systems!) is
explained afterwards in more detail: innovation as an ``improvement of
already-existing systems'' is supported by the advance of scientific
knowledge, from which new systems, products and services are created which are
driven by a ``market pull, the second trigger for innovation'' as a shaping
selection process, ``that stimulates the development of a system by meeting
the needs of that system's users''. The exact form of this approach, driven
not from engineering, but from innovation-entrepreneurial practices, becomes
clear in \cite[Chapter 3]{TESE2018}. The reasons for the universalistic touch
of the findings were analyzed in \cite{Gerovitch1996} in sufficient detail, so
that this analysis can be skipped here and reduced to a sober statement that
the foundations of these implicit conceptualizations are located in the
framework of the economic system of a capitalist society as supersystem (I
add: of western type, since the transferability to more autocratic economic
systems as in China or Russia, for example, require additional
considerations).  In reality, the conceptualization is even more tightly
drawn, as the analysis of the examples shows that a distinction between
industrial plant construction, mechanical engineering and consumer goods
production, as common in economic analysis, is not carried out, but
nevertheless prevails the perspective of a larger market oriented company that
estimates the product compatibility of technical systems.  The unique specimen
character of the vast majority of large technical systems and thus the
practices of industrial plant construction are not taken into account.

This renders the subject area of technical systems (``engineering systems'' in
\cite{TESE2018}) sufficiently clear, whose ``evolution'' is examined.  But
what is the aggregation principle used to examine such an evolution? Now that
we have identified a supersystem, we can use the TRIZ methodology itself to
reconstruct the modeling in \cite{TESE2018} and analyze its conceptual
foundation.

The starting point is the socio-economic (super)system of an industrial mode
of production. ``The wealth of these societies,'' Marx begins his analysis of
such a socio-economic system in \cite{MEW23}, ``appears as a 'vast collection
of goods', and individual goods are their elementary form. Our investigation
therefore begins with the analysis of the goods.''  We also start with this
term as a high level abstraction. As well known Marx' labour value theory
abstracted in the concept of \emph{good} from all qualitative characteristics
other than the one, to be a product of human labour. Only on such a level of
abstraction, special goods become globally exchangeable and constitute a
global market as a \emph{relation} -- field in TRIZ terminology -- between
these special commodities, the \emph{exchange value}.

However, that is not what \cite{TESE2018} is about, it is about functional
qualities of specific product groups such as washing machines or fountain
pens. The general competitive relationship between abstract goods is broken
down into more specific competitive conditions of individual product groups on
individual markets, and in \cite{TESE2018} the ``market pull'' is the main
function of the tool ``market'', which transforms the objects ``engineering
systems'' into ``useful products'' -- I use the TRIZ terminology of \cite{TT}.
With the marketability of products we identified a first structural unit in
the supersystem -- special markets in which \emph{special} goods with
\emph{specific} functional characteristics -- \emph{use value} in Marx's
terminology -- are competing with each other.

The use value of a good is characterized by a bundle of specific ``useful''
functionalities, i.e. by the property of a good to be a specific technical
component in the sense developed so far.  This ensemble of useful features
determines the possibilities and limits of the interchangeability of goods in
the overall societal technological process\footnote{These limits are, however,
  fluid, as A. Kuryan pointed out in the discussion \cite{Graebe2019} on the
  example of a hammer which is used keeping a balcony door open. The
  importance of such boundary crossings for the TRIZ analysis of the evolution
  of technical systems requires a more detailed examination, which cannot be
  given here.}.  Such borders lead to a stratification of ``the market'' into
special \emph{technology markets} for specific product groups with different
MPV (main parameter of value) -- according to \cite{TESE2018} a central
characteristic of such markets.

Such a technology market is less likely determined by the goods traded on it,
as by the companies producing these goods.  But this shifts the focus from an
MPV as an independent characteristic of goods to the \emph{business ability}
to \emph{produce} technical artifacts with this MPV in a reasonable
price-performance ratio.  Hence, on these technology markets meet producers
updating their \emph{prior experience} on the contradictions between justified
expectations and experienced results in the exchange of their work products.
This drives the dynamics of such a technology market.

Hence technical evolution should consider these technological production
conditions, too. This is also recognized in \cite{TESE2018}, because the
options for action described in the book refer to the organization of
corresponding innovation processes within companies. Hence a \emph{second}
supersystem pops up in our TRIZ analysis -- the management structures of
companies that are responsible for the innovation process. Again we have to
take into account the duality of system template -- common social practices of
the organization of innovation processes as discussed in \cite{Preez2006} --
and special real-world systems of innovation in the individual companies. The
\emph{main function} of those structures is the organization of innovation
processes in close connection with the general business strategy. This process
is contradictory by itself, since it has to take into account the
\emph{contradictory requirements} of different parts of the company (R\&D,
sales, finance, controlling, SCM, CRM). The recommendations compiled in
\cite{TESE2018} are \emph{one} aspect in this complex balancing process, a
methodology between ``technology push'' and ``market pull'' is, based on the
analysis in \cite{Preez2006}, rather on the level of the 1960s. In
\cite[Fig.~3]{Preez2006} with ``state of the art in science and technology''
next to the ``needs of society and marketplace'' a \emph{third} supersystem
pops up, that is relevant also for patent grants and the concepts \emph{state
  of the art} and \emph{level of invention}.  Thus we already identified
\emph{three} socio-technical supersystems (economy, innovation management,
science and technology), each with its own terms, structures, components,
forms of description and implementation, which, in one way or another, are
related to the evolution of technical systems.

The existence of \emph{multiple} supersystems clearly indicates that the term
\emph{supersystem} should not be confused with the term \emph{environment}.
Supersystems are specific systems with their own language and logic. The
relationship supersystem -- system is similar to the relationship system --
component: they constitute two different perspectives of perception on the
``totality of the world'' with two different understandings of the
\emph{essential} and thus from two different reduction perspectives.  From the
perspective of the system the supersystem also acts functionally: the
description of the system's interface specifies an input or throughput in
terms of quantity, quality and structure, which is required by the system to
function at runtime, the supersystem guarantees to fulfill these requirements
in the performance dimension.  Hence from the system's perspective \emph{a
  supersystem is nothing more than a special kind of component}, a
\emph{neighbouring component}.

\section{Normalization and Standardization}

One of the main roads of technical development is concerned with normalization
and standardization as a prerequisite for \emph{modularization}.
Modularization is an important -- if not the most important at all --
engineering technology that drives the evolution of technical systems.
Modular systems are widely used and make it possible to create unique
technical real-world specimen in the same way as explained in the concert
example. While there the \emph{private procedural skills} of the musicians
were an essential prerequisite, now the \emph{logic of the business
  application} appers as ``core concern'' of the components and the
\emph{logic of networking} of the infrastructure as ``cross cutting
concerns''. Both logics are orthogonal to each other, which devaluates the
trends 4.2 ``of increasing system completeness'' and 4.4 ``of transition to
the supersystem'' in their separate consideration.  This suggests the
following second thesis:
\begin{quote}\it 
  \textbf{Thesis 2:} A better descriptive understanding of the infrastructure
  requirements of interacting components (transition to the supersystem) leads
  to an \emph{attenuation} of the requirements for completeness of the
  individual components.
\end{quote}

In particular, the arguments in \cite{TESE2018} for trend 4.2 to justify the
hierarchization into ``operating agent'' (as core function), ``transmission''
(support by a working tool), ``energy source'' (use of forces of nature) and
``control system'' (use of -- nowadays mainly digital -- control elements) are
affected by these developments, as a visit to a DIY store immediately shows --
the machine systems of reputable manufacturers concentrate on provision of
energy, via appropriate APIs (such as velcro, screw or click fasteners on the
mechanical level) suitable tools can be joined with the energy
machine\footnote{Progress in material sciences, in particular with hook and
  loop fasteners, led to a massive return to \emph{mechanical} coupling
  principles in contrast to the TRIZ principle 28 of \emph{replacement of
    mechanical schemes}.}, where relational effects as normalization and
standardization in this \emph{world of technical systems} play a much greater
role than the further development of the technical artifacts only.

Standardization also opens up economies of scale for standard components,
i.e. for concepts near to the ``ideal end results''.  Economies of scale lead
to \emph{decreasing} cost per individual item and thus move the guiding
principle of competition from the \emph{better technical} solution to the
\emph{cheaper economic} production. So the S-curve does not necessarily end --
and probably rarely does -- with the decommissioning in stage 4
\cite[p. 38]{TESE2018}, but turns at the height of mature \emph{technical}
quality (including normalization and standardization) into the direction of
\emph{ubiquity}, in which the \emph{ever-less} economic expenditures for the
availability of this ``state of the art'' take on a leading role in further
development.

The trend 4.1 ``of increasing (technical) value'' thus turns to a trend ``of
decreasing economic value'', or -- in economic terms -- the market previously
driven by demand is turning to a supply-driven market: The same (mature) use
value has ever lower exchange value. Thus the value of ``ideality''
\cite[chapter 4.1.1]{KS2017} indeed goes beyond any limits, but as a
consequence of an \emph{economic} law. This corresponds to TRIZ Principle 17
of \emph{transition to other dimensions} and can be fixed as a third thesis:
\begin{quote}\it
 \textbf{Thesis 3:} The (technical) trend 4.1 ``of increasing (technical)
 value'' turns in Stage 3 of the S-curve development into an (economic)
 ``trend of decreasing (economic) value''.
\end{quote}
This means that in stage 3 the leading function (MPV) of the further
development of the production of common tools and standard components turns
from the technical driving forces to the economic ones. This process of
``commodification'' is sufficiently described in \cite{Naetar2005}, hence
there is no need to delve into the subject here. 

TRIZ principle 17 of \emph{transition to other dimensions} appears in the
above argumentation -- in contrast to the context of the TTB component -- not
as \emph{abstract design pattern}, but as \emph{abstract evolution pattern},
since it does not act as a means of active influence on a problem-solving
process, but as a description pattern of passively observed real-world
developments. In this sense, however, \emph{every other} of the TRIZ
principles as well as the TRIZ standards can be formulated as an abstract
evolution pattern. Conversely, the trends of evolution appear in the TTB
component as further abstract design patterns that can be used in addition to
the ``principles'' and the ``standards''. Although this is not new to
experienced TRIZ practitioners, see \cite{Shub2006}, I formulate this
observation as another thesis:
\begin{quote}\it
 \textbf{Thesis 4:} Each of the TRIZ principles and each of TRIZ standards can
 convincingly be formulated as a ``trend of evolution of technical systems''
 and vice versa.
\end{quote}
The hierarchy of evolution patterns thus gives cause to develop a ``hierarchy
of TRIZ principles'' \cite[chapter 3]{Zobel2016}, as proposed by Dietmar Zobel
more than 10 years ago, see also \cite{Zobel2020}.  The approach of M. Rubin
in \cite{Rubin2019} to systematize the connections between such
hierarchizations remains to be investigated further.

Normalization and standardization heavily influences the evolution in the
world of technical systems in an advanced state.  We demonstrate these effects
in the world of \emph{bolted joints} with machine screws and wood screws.

For the production of machine screws, high precision and coherence of diameter
and angle of attack of the threads is required to ensure that they fit with
the counterparts. This precision can be reached not only in an industrial
production mode, but -- for special applications -- also with appropriate
private tools -- e.g., a thread tap. With slotted, crosshead, hexagonal,
countersunk head, socket head etc. screws there is a wide range of ready for
use solutions for different application scenarios (TRIZ Principle 3 of
\emph{local quality}), and corresponding tools: ordinary wrenches, socket
wrenches, screwdrivers, Allen keys and so on (once again TRIZ principle 3),
both as individual tools and inserts for the cordless screwdriver as an energy
machine (TRIZ principle 1 of \emph{decomposition}, TRIZ Standard 3.1
\emph{transition to a bi-system}).  Flexible connections\footnote{Amazon
  offers such a 31-piece set of the company Lotex GmbH for 20.99 Euro.}
(together with the cordless screwdriver TRIZ standard 3.1 \emph{transition to
  a poly system}) can be used to apply screw connections even in places that
are difficult to access and so on. These tools are also used by industrial
robots (TRIZ standard 3 \emph{transition to a supersystem} applied twice,
because the industrial robots are components in the super-super-system).

The world of wood screws avoids the two-component system (once again TRIZ
standard 3.1: bolt and nut), in that the hold in the material itself is sought
(trend 4.6 \emph{of increasing degree of trimming} -- why this central TRIZ
method is neither part of the ``principles'' nor the ``standards''?), either
by pre-drilling (TRIZ principle 10 \emph{of previous action}) or by a
self-tapping screw (TRIZ principle 25 of the \emph{self-service} or again
trend 4.6 of \emph{trimming}). Unfortunately some materials do not offer this
grip, thus \emph{anchors} were invented (TRIZ non-trend of
\emph{anti-trimming}\footnote{This is a subtle point, since this trend is
  called ``\foreignlanguage{russian}{закон развертывания — свертывания}'' in
  Russian \cite[2.1.6]{TBK-2007}, but from this bidirectional mode only one
  direction survived in the English (and German) translation.}), a world of
technical solutions that are at the heart of every TRIZ practitioner.  We
didn't touch yet special applications of screw connections as in surgery,
where essential parameters of material and reliability determined by the
conditions of the supersystem lead to very special system solutions.

I have described this world in so much detail to clarify three aspects:
\begin{itemize}[noitemsep]
\item [1)] It is a world of technical systems in which principles of problem
  solving based on TRIZ play an important role.
\item [2)] The structuring moment in that world are not the technical systems,
  but the \emph{technical principles}.
\item [3)] The 10 ``trends'' in a decontextualized fashion are not very
  helpful to determine your way through highly volatile requirement
  situations, if you seek for \emph{special} solutions in \emph{special}
  contextualizations.
\end{itemize}
In such a world the ``evolution of individual technical systems'' is of minor
interest compared to a global \emph{evolution of technology}, i.e. the
``evolution of the world of technical systems'' as a whole.

\section{Summary}

With the concept of a \emph{technical system} the whole TRIZ theoretical body
revolves around a term that is not precisely defined in the TRIZ literature,
but left to a ``common sense''. The 40 TRIZ principles, the 76 TRIZ standards
and the (in \cite{TESE2018}) 10 TRIZ trends of evolution constitute a universe
of theoretical reflections of practical inventory experience with a tendency
to universalism.  Nevertheless overarching generalizations of practical
experience and the resulting decontextualization in the TRIZ theoretical body
are hardly perceived as a problem in the TRIZ community.

In this paper an attempt was made to determine how far a notion of
\emph{technical system} takes in this theoretical context and to relate this
with approaches from neighbouring theory corpusses.  It turns out that
focusing on an artifact dimension of technology, as inherent to the term
\emph{technical system}, blocks the view on essential \emph{relational}
phenomena in the \emph{world of technical systems}. A notion of
\emph{technical principle} as used in \cite{Shpakovsky2010} is better suited
for the analysis of relational phenomena in that world. Once more, the whole
is \emph{more} than the sum of its parts.

\begin{thebibliography}{xxx}
\bibitem{Bertalanffy1950} Ludwig von Bertalanffy (1950). An outline of General
  System Theory. The British Journal for the Philosophy of Science, vol. I.2,
  134–165.
\bibitem{KFK2000} Klaus Fuchs-Kittowski (2000).  Wissens-Ko-Produktion.
  Verarbeitung, Verteilung und Entstehung von Informationen in
  kreativ-lernenden Organisationen (Knowledge co-production. Processing,
  distribution and creation of information in creative learning
  organizations). In: Fuchs-Kittowski et al. (eds.). Organisationsinformatik
  und Digitale Bibliothek in der Wissenschaft (Organisational Informatics and
  Digital Library in the Science). Wissenschaftsforschung (Science Research),
  Yearbook 2000. Gesellschaft für Wissenschaftsforschung, Berlin.\\
  \url{http://www.wissenschaftsforschung.de/JB00_9-88.pdf}
\bibitem{Gerovitch1996} Slava Gerovitch (1996). Perestroika of the History of
  Technology and Science in the USSR: Changes in the Discourse. Technology and
  Culture, Vol. 37.1, pp. 102--134.
\bibitem{Graebe2019} Hans-Gert Gräbe (2019). A discussion about TRIZ and
  system thinking reported in my Open Discovery Blog.
  \url{https://wumm-project.github.io/2019-08-07}. 
\bibitem{Graebe2020a} Hans-Gert Gräbe (2020). Reader for the 16th
  Interdisciplinary Discussion \emph{The concept of resilience as an emergent
    characteristic in open systems} on 7.2.2020 in Leipzig (in German).
  \url{http://mint-leipzig.de/2020-02-07/Reader.pdf}.
\bibitem{Graebe2020b} Hans-Gert Gräbe (2020). Die Menschen und ihre
  Technischen Systeme (Men and their technical systems). LIFIS Online
  05/19/202020.  DOI \url{10.14625/graebe_20200519}.
\bibitem{KS2017} Karl Koltze, Valeri Souchkov (2017). Systematische Innovation
  (Systematic Innovation). Hanser, Munich. Second edition. ISBN
  978-3-446-45127-8.
\bibitem{Kropik2009} Markus Kropik (2009). Produktionsleitsysteme in der
  Automobilfertigung (Production control systems in the automobile
  manufacturing). Springer, Dordrecht. ISBN 978-3-540-88991-5.
\bibitem{TBK-2007} S. Litvin, V. Petrov, M. Rubin (2007). TRIZ Body of
  Knowledge. \\ \url{https://triz-summit.ru/en/203941}.
\bibitem{TESE2018} Alexander Lyubomirskiy, Simon Litvin et al. (2018). Trends
  of Engineering System Evolution. Sulzbach-Rosenberg. ISBN 978-3-00-059846-3.
\bibitem{MEW23} Karl Marx (MEW 23). Das Kapital, volume 1. MEW 23. Dietz
  Verlag, Berlin.
\bibitem{Naetar2005} Franz Naetar (2005). Commodification, Wertgesetz und
  immaterielle Arbeit (Commodification, law of values and immaterial labour).
  Grundrisse 14, p. 6--19.
\bibitem{Preez2006} Niek D Du Preez, Louis Louw, Heinz Essmann (2006). To
  innovation process model for improving innovation capability. Journal of
  high technology management research, vol 17, 1--24.
\bibitem{Rubin2019} Michail S. Rubin (2019).  \foreignlanguage{russian}{О
  связи комплекса законов развития систем с ЗРТС} (On the connection between
  laws of development of general systems and laws of development of technical
  systems). Manuscript, November 2019.
\bibitem{Shpakovsky2010} Nikolay Shpakovsky (2010). Tree of Technology
  Evolution. Forum, Moscow.
\bibitem{Shub2006} Leonid Shub (2006). \foreignlanguage{russian}{Осторожно!
  Таблица технических противоречий}. (Caution! The contradiction table).
  \url{http://metodolog.ru/conference.html}. 
\bibitem{TT} Target Invention (2020). TRIZ Trainer.
  \url{https://triztrainer.ru}.
\bibitem{VDMA2019} VDMA. Maschinenbau in Zahl und Bild 2019 (Machine-building
  industry in numbers and picture 2019).
\bibitem{Zobel2016} Dietmar Zobel, Rainer Hartmann (2016). Erfindungsmuster
  (Pattern of Invention).  Second edition. Expert, Renningen.
\bibitem{Zobel2020} Dietmar Zobel (2020). Beiträge zur Weiterentwicklung der
  TRIZ (Contributions to the further development of TRIZ).  LIFIS Online
  01/19/2020.\\ DOI \url{10.14625/zobel_20200119}
\end{thebibliography}
\end{document}

