\documentclass[11pt,a4paper]{article}
\usepackage{od}
\usepackage[english]{babel}
\usepackage[utf8]{inputenc}

\newenvironment{example}{\begin{quote} \textbf{Example:}\par }{\end{quote}}

\title{How invent.\\ Thoughts on the methodology of scientific work}

\author{Altshuller, Genrich Saulovich}
\date{1960}

\begin{document}
\maketitle

\begin{quote}
  The russian original of this text was taken from
  \url{https://www.altshuller.ru/triz/investigations1.asp} and translated by
  Hans-Gert Gr\"abe.  The translation can be reused under the terms of the
  CC-BY Creative Commons License.
\end{quote}

In 1960, I sketched theses on the method of “discovery.“ These are sketches
for myself to streamline the initial ideas about technology of scientific
creativity.  The outline remained unfinished, distracted by work on ARIZ.
Almost 20 years have passed, but it seems to me that these thoughts are not
outdated and can be used by TRIZ teachers in the classroom on scientific
creativity.

In the works and articles of G. Filkovsky and I. Kondrakov, technology of
creativity is considered in solving problems of “discovering new theories“.
Meanwhile, there are “discoveries of new phenomena“ (discovery of
radioactivity, etc.). Therefore useful up the presentation of the “theory of
the discovery of new theories” to give an overview -- this will help outline.
\begin{flushright}
G. Altshuller, 1979-08-22
\end{flushright}

\section*{A. What is the difficulty of resolving the problem}

\paragraph{1.}
Difficulties in creating a methodology of scientific work -- compared with the
methodology inventions -- consist in the fact that circumstances play a big
role here, complicating the main circuit. Such circumstances exist in
inventive creativity. But in discovery they are present in great
concentration. it no longer reservations or exceptions to fairly universal
rules, but constantly operating “distorting factors“ that must always be taken
into account.

\paragraph{2.}
The first such factor is the historical development of scientific methods. In
modern inventions still coexist peacefully at tricks that arose at different
stages development. In discovery -- a more rigid change of methods. Therefore,
it is necessary to carefully argue with examples taken from different eras.

\paragraph{3.}
The second factor is the uneven development of scientific methods in different
fields of science.  This phenomenon is also observed in invention, but in
science it is expressed more strongly.

\paragraph{4.}
The third factor is the relatively large role of the case.

\section*{B. Classification of Openings}

\paragraph{5.}
All discoveries are divided into two groups, so significantly different in all
factors that one wonders how they are united in one word.  The first group is
discoveries consisting in the establishment of a new phenomenon.

\begin{example}
  \begin{itemize}
  \item discovery of X-rays by X-rays,
  \item discovery of superconductivity by Kemmerling-Onnes.
  \end{itemize}
\end{example}

\paragraph{6.}
I would call this group of discoveries DETECTION, for the essence of the new
phenomenon does not open at all; a new phenomenon is just first discovered.

A subgroup of discoveries consisting in establishing a concrete fact. These
are geographical discoveries. This also includes most discoveries of
observational astronomy, geological discoveries (new deposits - denia), the
discovery of new species of plants and animals. For this subgroup it is
characteristic that its discoveries (in contrast to the discovery of
phenomena) require no explanation.

It is interesting to note that this is spontaneously reflected in terminology:
discoveries phenomena (as well as the opening of the second group, which is
discussed below) sometimes called scientific discoveries, involuntarily
emphasizing their difference from simple establishing a specific fact.

In the future, we will consider in the first group only the main core -
discovery of new phenomena.

\paragraph{7.}
The second group -- discoveries, consisting in the establishment of patterns.
In this case new phenomena do not open, the discovery is manifested in the
explanation of already known phenomena whose essence was previously
incomprehensible or did not fit into the existing explanations.

\begin{example}
  \begin{itemize}
  \item explanation of the photoelectric effect by Einstein, 
  \item explanation of the evolution of plants and animals in the struggle for
    existence.
  \end{itemize}
\end{example}
\paragraph{8.}
By no means should the first group be identified with the experimental ones,
and the second with theoretical discoveries. On the contrary, in each group
there are discoveries made experimentally, and there are discoveries made
theoretically. So, in the first group: experimental discovery of radioactivity
and theoretical discovery of electro- magnetic oscillations by Maxwell. In the
second group: experimental discovery Coulomb's law and the theoretical
discovery of the relationship between mass and inertia.

\paragraph{9.}
If we give a simplified scheme, then we can say this: the discovery of the
phenomenon is the new quality QUALITY of matter, the discovery of regularity
is the establishment QUANTITATIVE RELATIONS. Even simpler and rougher: in the
first case the result of creativity is new information, in the second -- a new
formula.

\paragraph{10.}
We must immediately stipulate that there is no blind wall between the two
groups. Wherein historically, the “wall power” has been decreasing all the
time.  At present, often in a single work, an open phenomenon and an immediate
explanation (sometimes vice versa: a hypothesis and the following hypothetical
discoveries -- predictions -- of new phenomena).

\paragraph{11.}
However, a very tangible difference has been preserved to this day. It used to
be, as indicated, was even more pronounced. Hence the most important
consequence: existed (and in still largely preserved) two groups of
scientists, two significantly different types discoverers. The first type is
scientists who discovered new phenomena. The second is scientists, established
new patterns.

\paragraph{12.}
Scientists of the second type are much higher (in terms of creativity) of
scientists of the first type.  One can discover the phenomenon by chance. It
is possible and not entirely by chance, but still on free market, trying to
discover the phenomenon of five cents, you discover something worthwhile One
hundred rubles. The discovery of new laws requires -- overwhelmingly - In most
cases, focused efforts.

\paragraph{13.}
In this regard, it is interesting to make the following observation. The fate
of scientists of the first type like a lightning fast, but a single flash of a
new star: twentieth asterisk magnitude suddenly turns into a star of the first
magnitude ... And soon again will return It shifts to its former
appearance. It’s understandable: the ability to make the discovery of one new
appearance does not mean the ability to make another discovery.

\begin{example}
  Galvani discovery, Michelson discovery.
\end{example}

On the contrary, scientists of the second type, as a rule, equally fruitfully
work in different (sometimes very distant) areas. Examples: Einstein (first
the photoelectric effect, then the theory of relativity), Faraday
(electromagnetism and chemistry), Schmidt (higher algebra and cosmogony),
Mendeleev, Pavlov (first physiology of food- jam, then -- the study of higher
nervous activity).

\section*{B. Initial Construction of the Method}

\paragraph{14.}
The presence of two groups of discoveries means that when creating a MODERN
methodology discovery should be based on the presence of TWO GROUPS
METHODOLOGY-SKILLS, very different from each other.

\paragraph{15.}
The purpose of the first group of methods is to lead to the discovery of a new
phenomena. Hence, the association in this group of specific search methods.
The purpose of the second group is the discovery of new patterns. Hence the
union various combination techniques aimed at opening a new combination (i.e.,
a new explanation) of already known phenomena.

\paragraph{16.}
The task of the elementary discovery technique is to study separately separate
techniques in each group of methods. In other words: give a number of specific
and sound recommendations for making discoveries.  The ultimate goal of the
methodology is to establish patterns in historical development methods of
discovery and explore the mechanism of mutual understanding of methods in
different fields of science. In the same way, the elementary technique of
invention (if to it add the study of historical development and mutual
influence) goes into theory development of technology.

\section*{D. Basic Receptions for Opening New Phenomena}

\paragraph{17.}
The simplest technique -- historically very important -- consists
(paradoxically simplicity) is to pay attention to already known and different
countries -- ness of the phenomenon. Briefly -- look for an anomaly. At first
glance, no one will pass past a strange phenomenon. But during the 11th --
19th centuries this happened all the time.  In a number of branches of
science, this situation has been preserved to a certain extent in our days.

\begin{example}
  At the end of the 18th century, Cavendish, exploring the air, discovered a
  certain, with nothing connecting part. This fact was well known among
  scientists, considered an anomaly, but did not attract attention. Only a
  century later, Ramsay, continuing this study, discovered argon (and the
  phenomenon of chemical inertness at all). It can be argued that with a focus
  on the found Cavendish anomalies could have discovered inert gases (and the
  phenomenon of chemical inertia) 60-80 years earlier.
\end{example}
\paragraph{18.}
A somewhat more complicated trick is paying attention to white spots within
already known phenomena. In this case, anomalies are also sought -- within the
limits already studied range of temperatures, pressures, distances, speeds,
etc.

\begin{example}
  In 1772, in Bonn, a book by Tipius “Contemplation of Nature“ was published,
  in which attention was paid to the correct increase in the distances between
  the planets and Sun and space between Mars and Jupiter. Then another
  scientist -- Bods - stated that there should be an unknown planet in place
  of the gap. She was found by one of the astronomers who responded to the
  call “search.“
\end{example}
\paragraph{19.}
The next most difficult trick is to pay attention to white spots outside known
phenomena.

\begin{example}
  Bridgman's research into previously unavailable high pressures at led to the
  discovery of new ice modifications.
\end{example}

\paragraph{20.}
Next in complexity is the assessment of known phenomena from a new point of
view.

\begin{example}
  Quote from an article in the Literary Gazette of February 6, 1960: “18 years
  ago to Professor M. Volsky (co-dispatcher) was contacted by a doctor
  working on his dissertation, asking battle to calculate the perimeter of the
  trachea, ellipsoidal in cross section. “Having made the calculations, Volsky
  recommended a number of other calculations and calculate the loss air
  pressure during its movement. A doctor who has done enough work in the field
  physiology of breathing, sincerely admitted that with such a question to him
  never had to meet. Then professor Volsky made the calculations himself and
  came to completely unexpected conclusions. He proved theoretically that the
  old concept of breathing is against the laws of physics. For millennia,
  there is no air in the pleural cavity. The whole theory was built on this.
  And the conclusions of Volsky testified: there is air in the pleural cavity.
  Soon this was confirmed experimentally.
\end{example}
\paragraph{21.}
The discovery of new phenomena by combining the old. In other words:
phenomenon A in itself is known, phenomenon B is also known, the discovery
consists in the fact that phenomenon B, consisting in the relationship of A
and B.
\begin{example}
  The frequency of sunspots has long been known; the frequency of phenomena in
  ionosphere -- too; the discovery was that a mutually connections between
  sunspot activity and ionosphere functions.  There may be more complex
  options for discoveries: the formula A + B gives a new phenomenon B, then B
  + known G gives a new phenomenon D.
\end{example}

\begin{example}
  The periodicity in solar activity is known, the periodicity in clumping
  colloids too. First, a relationship was established between these phenomena.
  Then the resulting new phenomenon was associated with the well-known
  phenomenon, consisting in that the human body is a colloidal system. As a
  result, the phenomenon of mutual the connection of some processes in the
  body with the frequency of sunspots.
\end{example}

\paragraph{22.}
Reverse reception: investigation of phenomenon A in order to establish that it
is a co- the combination of two previously unknown phenomena B and B.

\begin{example}
  At first, radioactive radiation was generally known, then using magnetic
  field -- found that the rays of radium -- a combination of three different
  rays.  So discovered the phenomenon of alpha, beta and gamma radioactivity.
\end{example}
\paragraph{23.}
Other schemes:
\begin{itemize}
\item By analogy. There is a group of phenomena and, for example, there is
  another more or less similar pressing on her for the second group of
  phenomena; then we can expect that phenomenon A in the first group
  corresponds to the still unknown phenomenon A1 in the second group.
\item Question self-evident and universally recognized phenomena. On each It’s
  useful to verify the stage of development of the experimental technique ny
  phenomena.
\item Exclusion of a non-universal phenomenon. Suppose phenomenon A combines
  well a number of factors, but does not explain any one fact. Then it makes
  sense to abandon phenomenon A or replace it with particular
  phenomena. Wherein the existence of boundaries between particular phenomena
  is a new phenomenon in itself.
\item Search among mutually contradictory phenomena. Such inconsistency is far
  not always obvious.
\end{itemize}

\paragraph{24.}
Probably this list can be replenished. The challenge is to give a clear
scheme of receptions. Perhaps some of them are special cases of others,
more general tricks.

\section*{D. Reception Regulations}

\paragraph{25.}
There may be two cases: either the pattern is established for the first time
(for example, Kepler’s establishment of the laws of planetary motion), or the
task is to overcome difficulties (explain contradictions, exceptions) that the
previous theory was nullified.

\paragraph{26.}
In the first case (historically, the development of any exact science begins
with it) The simplest trick is this: you need to increase the number of known
facts until the pattern manifests itself. This, incidentally, is an elementary
method of scientific research today.

\paragraph{27.}
In the second case, a characteristic trick is that hypothetical phenomena that
remove difficulties.

\begin{example}
  And before Mendeleev there were various systems of periodization of
  elements.  However, all these systems ran into difficulties, the main of
  which consisted of in that the periodicity evident at the beginning of a
  series of elements was then violated.  Mendeleev eliminated this difficulty
  by introducing hypothetical elements. it there was a “stretch” that removed
  the difficulty and allowed for the first time to hold the principle
  periodicity throughout the whole series of elements.  Roughly speaking, in
  this case, to open it is necessary to raise the question: “What unknown
  phenomena or facts must be taken as reliable in order to remove difficulties
  theory? “- and answer this question.  It is important to consider the
  psychological moment. When some kind of theory that has long served
  faithfully, suddenly begins to stall, the vast majority of scholars It is
  possible to explain heretical facts without changing the theory. Great
  scientists are great that they leave the hypnotizing effect of the theory
  and boldly change it, they fret, not worrying that these prohibitions seem
  arbitrary at first glance extremely unlikely or inexplicable (e.g., final
  velocity Sveta).  Often the most difficult thing is to recognize the
  leakyness of the old theory and the need for new assumptions. If this is
  recognized, the assumptions themselves are sometimes quite difficult to
  find.
\end{example}
\paragraph{28.}
Every theory is mortal. Therefore, in the period of maturity, theories must be
concentrated efforts not only (and in the period of old theory and not so
much) to apply it to explanation of new groups of phenomena, but also on the
study of weaknesses. Simply put: necessary to develop a theory not where it is
strong, but where its weakness is felt. All prep university admissions, all
graduate studies, the vast majority of candidates works, in general a
significant part of scientific work is based on the application of substantial
theory to particular problems (for example, the application of mechanical
prin- circuits to the problems of chemistry). We must look for those cases
when the theory is just not applicable.  neem. We must not expand and
strengthen the foundations of the natural sciences, humane tare and social
theories (the result of such work at best is hardening theories), and look for
cracks.

\paragraph{29.}
The pace of development of science is largely determined by a combination of
various factors, which are far from always possible to influence with the goal
of afterburner. However it is possible it is reasonably argued that the
average rate of development of science could be noticeable above, if the work
on finding cracks in the “well derived“ theories today.

It should be noted that in those branches of science that are developing now
especially rapidly, they do just that. For example, in nuclear physics the
whole story of the last three decades -- this is the story of seeking “but“
and the desire to introduce new theoretical parcels explaining these “buts.“
Of course, at the same time using “established” principles to solve various
problems, for example to explain the mechanism of chem. reactions. On the
other hand, in the branches of science, relatively slow, there is a different
distribution of forces: the main efforts sent to the application of existing
theories to explain new groups phenomena. This is the case in physiology.
After Pavlov, the vast majority The work consisted in the fact that the
strengths of Pavlovian theory extended to explanation of new phenomena. And
one should energetically look for weak points of learning Pavlova.

Exaggerating somewhat, we can say: study not the laws, but the exceptions to
them: namely new discoveries are hiding there.

\paragraph{30.}
Returning to the methods of searching for new patterns, it should be noted
characteristic technique: transferring methods and apparatus of one area to
another area.

\begin{example}
  The creation of cybernetics (it would be interesting to transfer, for example,
  “methods and apparatus“ of music in geology or nuclear physics).
\end{example}
\paragraph{31.}
This technique is a special case of this technique: the spread of methods and
apparatus (as well as phenomena and facts) to a wider area.  The following
special case also adjoins here: the proclamation of a well-known, but
considered limited phenomenon, as a universal law.

\begin{example}
  Attraction -- as a more or less particular phenomenon -- was widely known
  until Newton. But Newton proclaimed the universality of gravity (this led to
  do that the planets are attracted by the Sun). And then from the previously
  installed Kepler’s third law directly follows the formula of attraction.
\end{example}

\paragraph{32.}
The opposite method is applied less often: a new pattern is established by
limitations of the previously considered universal principle.

\paragraph{33.}
Another trick: a return to the old theories on a new basis (“alchemical“
transformations of elements by nuclear physics methods).

\section*{E. General Considerations}

\paragraph{34.}
The development of technology gives science ever more accurate machines and
instruments for measuring.  Many discoveries are directly an obvious
consequence of the new -- more accurate -- measurement methods.

\paragraph{35.}
All the techniques described above are elements of scientific work. Cannot be
identified discovery discovery with research. Research is a collection of
searches for discovery, mechanical accumulation of facts, refinement of
measurements, accounting for new ones -- made others -- discoveries,
establishing the relationship between all of the above and Sofsky
installations. Thus, research is a much broader concept, than opening.
\vfill

\ccnotice
\end{document}
