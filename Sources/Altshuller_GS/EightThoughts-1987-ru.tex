\documentclass[11pt,a4paper]{article}
\usepackage{od}
\usepackage[russian]{babel}
\usepackage[utf8]{inputenc}

\title{Восемь Мыслей о Природе и Технике}

\author{G.S. Altshuller, M.S. Rubin}
\date{1987}

\begin{document}
\maketitle
\begin{quote}
  Source: \url{https://scientifically.info/publ/7-1-0-214}
\end{quote}

\section*{Что будет после окончательной победы}

\begin{flushright}
  Не будем, однако, слишком обольщаться нашими победами\\ над природой.
  За каждую такую победу она нам мстит.\\
  Ф. Энгельс. Диалектика природы
\end{flushright}

Есть три основных типа разрушающего воздействия современной технической
цивилизации на природу:

\paragraph{1. Преступное разрушение природы.}
Наиболее откровенная форма уничтожения природной среды. Например, поджоги
леса: в пожарах гибнут сотни тысяч гектаров леса. Сброс отходов с танкеров --
в открытом море, тайком. Сброс -- в реки и озера -- отходов
нефтеперерабатывающей и химической промышленности. Выброс вредных газов в
атмосферу -- вопреки всем санитарным нормам.

Недопустимость преступного разрушения природы в какой-то мере осознана
обществом.  Законы, защищающие природу от варварского истребления, постоянно
ужесточаются.

Здесь есть резервы для защиты природы: в принципе, в любой день могут быть
введены еще более суровые законы и налажен еще более строгий контроль за их
выполнением.

\paragraph{2. «Законное» разрушение природы.}
Законы позволяют разрушать природу в определенных, якобы безопасных для
природы, пределах. Через каждые 10—15 лет выясняется, что пределы эти надо
резко ужесточить: нормы пересматривают, делают более жесткими, но в
большинстве случаев бывает уже поздно ... Казалось бы, надо сразу ввести очень
жесткие нормы. Но это разрушило бы основы технической цивилизации. Так, чтобы
ликвидировать фото-химический смог в Лос-Анджелесе, надо запретить
автомобильное движение. Кто пойдет на это? ... «Законное» разрушение природы
продиктовано экономической целесообразностью. Изменить понятие
«целесообразности» трудно: надо изменить представление о человеческих
ценностях. Пока в споре «автомобиль в центре города или лес на окраине
города», безусловно, побеждает автомобиль ...

Разумеется, есть и такое «законное» разрушение, которое не диктуется железной
экономической необходимостью. Такова ситуация с целлюлозными предприятиями на
Байкале. Площадь усыхающих лесов в районе Байкала составляет сейчас
полмиллиона гектаров, гибнет рыба, изменяется состав воды ... Получение
какого-то дополнительного количества целлюлозы перевешивает -- как фактор
экономической «целесообразности» -- ценность уникального природного региона.

Иногда «законное» разрушение природы идет не непосредственно, а по цепочке.
Закон не запрещает строить танкеры все большего водоизмещения. Но большой
танкер -- это много нефти, сосредоточенной на одном корабле. А море остается
морем -- со всеми его опасностями, и если гибель небольшого танкера -- опасная
авария, то гибель супертанкера, перевозящего полмиллиона или миллион тонн
нефти, -- это катастрофа планетарного масштаба.

Бурно развивается авиация: растет число самолетов, увеличивается мощность
двигателей и высота полетов. В атмосферу -- на «законном» основании --
выбрасывается все большее количество вредных газов. Закон не видит нарастающей
опасности разрушения озонного слоя в атмосфере. Между тем, озон защищает все
живое на Земле от губительных ультрафиолетовых лучей.

Нарастает мощность лазерных устройств -- закон пока не задумывается над
возможными последствиями воздействия мощных лазерных лучей на атмосферу ...

Законы стремятся не задеть интересы экономики. Законы не заглядывают в
будущее. Этим объясняется все более мощное «законное» разрушение природы.

Здесь есть определенные резервы. Законодательство может быть более суровым и
более дальновидным. Но резервы эти не слишком велики: нельзя заметно
притормозить экономическое развитие и научно-технический прогресс.

\paragraph{3. Необходимое вытеснение природы.}
Численность населения на планете быстро увеличивается. Нужны новые города,
новые заводы и фабрики, новые дороги ... Нужно новое место для технического
мира -- взять это место неоткуда -- можно только отнять его у природы.

Предположим, искоренено преступное разрушение природы, изданы мудрые и
дальновидные законы, положившие конец хищническому развитию экономики, нет
откровенно преступного истребления природы и нет «узаконенного хищничества».
Все равно техника будет стремительно вытеснять природу: нужно место для
увеличивающегося населения, нужно место для техники, обеспечивающей высокий
уровень благосостояния всему быстро растущему населению планеты.

Предположим невероятное: уже сегодня введены эффективные меры уменьшения
темпов роста населения Земли. В самом идеальном случае эти меры скажутся через
три-четыре поколения. А этого времени сверхдостаточно для практически полного
вытеснения природы техникой.

\paragraph{Мысль 1.}
Сегодня еще существует шаткое равновесие природы и техники, но потенциально
природа обречена; она неизбежно будет вытеснена стремительно растущей техникой
-- даже если хищническое истребление природы (незаконное и «законное») будет
прекращено.

Мысль о том, что природа даже в самом идеальном случае неизбежно будет
вытеснена техникой, встречает сильное психологическое сопротивление. «Этого не
может быть, потому что этого не может быть никогда ...»

Наиболее распространенный довод: на нашей планете еще много свободного места.
Действительно, города, производственные предприятия, дороги занимают всего 3.2
процента земельного фонда нашей планеты; пашни и сады -- 10.6; пастбища --
23.2; водохранилища, реки и озера -- 2.4 процента. Итого -- 39.4 процента
имеющегося фонда. Как будто не так и много -- меньше половины. Но что
представляют собой оставшиеся земли? Ледники, пески и земли, испорченные
человеком, -- 15 процентов; леса -- 30; пустыни -- 6.9; болота -- 3; тундра --
5.5 процента.

Земля поделена без остатка! Уже прекратился рост площадей, отведенных под
пашню. Площадь лесов ежегодно уменьшается на 1.7 процента (то есть на 0.5
процента от всего фонда). Это катастрофические темпы: не будет лесов и океанов
-- не будет кислорода в атмосфере. Осваивать пустыни? Очень дорогое и
медленное дело! Вот Каракумский канал. Бетонное русло строить было дорого --
канал имеет земляное русло, через которое теряется 17 процентов воды. 170
тысяч литров воды в секунду! Поднимается уровень грунтовых вод, образуются
соляные озера ... Последствия предсказуемы только в одном: ничего хорошего
ждать не приходится. Осушение болот? Нарушается экологическое равновесие,
исчезают многие виды растительности, вымирают некоторые виды животных ...
Необходимо все -- и пустыни, и болота, и леса. То, что можно было взять у
природы, в основном уже взято ...

Другой довод: техника, развиваясь, стремится к миниатюризации, современные ЭВМ
в тысячи раз компактнее ЭВМ первого поколения. Да, рабочие элементы
современных машин становятся компактнее: увеличивается производительность на
единицу веса и объема. Но именно это создает условия для взрывного роста
массовости машин: в тысячу раз меньше объем, но зато в тысячу раз больше число
рабочих элементов, и в тысячу раз больше места на Земле занимает тот или иной
вид машин, и тем больше пространства нужно для производства и обслуживания
микротехники.

Еще один довод: технику можно вывести в космос ... Напрасная надежда! Выход в
космос требует особо интенсивного расширения производственных площадей на
Земле: нужны новые добывающие, перерабатывающие, машиностроительные
предприятия. Нужны новые города, дороги, космодромы ...

\paragraph{Природа обречена.}
При самом бережном отношении она все равно будет вытеснена техникой. Даже если
мы попытаемся затормозить развитие техники, тормозной путь окажется слишком
длинным.

Через три-четыре поколения человечеству предстоит жить в мире, в котором
природа будет на задворках. Леса пройдут стадии заповедников, потом парков,
потом садов и превратятся в чахлые скверики. Пашни станут полутеплицами.
Атмосфера загрязнится до недопустимых для человека норм ... Может быть, это
произойдет не за три-четыре, а за пять-шесть поколений -- какая разница?!
Важно другое: это неизбежно произойдет, это произойдет неотвратимо, даже при
самом бережном отношении к природе, произойдет -- потому, что это уже
запрограммировано. Мы не успеем сменить стиль жизни, не сумеем понять, что
«природные ценности» несоизмеримо выше «автомобильных ценностей». У нас не
осталось времени, чтобы перестроиться и спасти природу.

Но есть -- еще есть! -- время, чтобы взглянуть правде в глаза и подготовиться
к жизни в новом техническом мире

До сих пор техника имела дело, так сказать, с бесприродными «микромирами».
Искусственные, технические миры создавались в ограниченном пространстве: в
подводных лодках, в кабинах самолетов, на космических аппаратах, в какой-то
мере -- в производственных и жилых помещениях. В основном же цивилизация была
природно-технической. Природа не исключалась, она работала вместе с техникой
(и наоборот — техника вместе с природой).  Тем не менее, в развитии техники
очень важное значение имели задачи, выдвигаемые для создания и
совершенствования бесприродных «микромиров». Они были одним из главных
приводов технического прогресса. Создание и функционирование большого
бесприродного технического мира потребуют решения множества технических задач.
Нужды нового мира на долгое время станут основным фактором, определяющим
будущий технический прогресс.

\paragraph{Мысль 2.}
Проектирование бесприродного технического мира (БТМ) позволит заранее выявить
задачи, жизненно важные для существования и развития цивилизации, и
своевременно подготовиться к их решению. Таким образом, проектирование БТМ
даст стержневую линию не только для социального, но и для технического
прогнозирования.

Сейчас мы щедро оплачиваем исполнение наших желаний валютой природных
ценностей. Захотели обзавестись миллионами автомобилей -- пожалуйста! Заняли
автодорогами тысячи и тысячи километров природного простора, изломали природу
нефтедобычей и нефтепереработкой ... Захотели издавать несметное количество
печатных изданий -- пожалуйста! Пустили под топор леса -- источник кислорода
...

Перенести такой мир в бесприродные условия невозможно: нечем платить. БТМ
должен быть основан на иных принципах.

Мысль о неизбежности мира без живой природы пугает наше воображение. Но
отключим на время эмоции и попытаемся трезво оценить возможность создания БТМ.

Принципиальная осуществимость БТМ зависит, прежде всего, от возможности или
невозможности техническими средствами сделать то, что природа делает
«автоматически» и «бесплатно»: обеспечить человечество кислородом, питьевой
водой, пищей, энергией, материалами ... Перечень даров природы бесконечен.
Природа «автоматически» и «бесплатно» снабжает человечество оптимальными
факторами существования: силой тяжести, атмосферным давлением, освещением,
температурой и влажностью воздуха.

Природа неутомимо уничтожает отходы. Обеспечивает ритмику: смену времен года,
суточный цикл, биоритмы и т. д. Дает надежный защитный комплекс: защиту от
радиации, вредных излучений, перегрева и переохлаждения ...

В рамках этой статьи мы затронем только вопрос о техногенном обеспечении трех
наиболее важных функций природы -- снабжения человечества кислородом, пресной
водой, питанием.

«Проектирование БТМ», «жизнеобеспечение в БТМ» -- при постановке задач в таком
виде может создаться впечатление, что БТМ это нечто, что можно начать и
кончить строить, между тем мы уже живем в БТМ.

Мы практически не бываем на открытом воздухе: дом, метро, автобус, цех или
другое рабочее помещение, магазины, театры, спортивные залы. Мы не пьем
ключевой воды, редки в нашем рационе биологически чистые продукты ...

Это первая, начальная фаза БТМ, когда среда обитания в значительной мере уже
бесприродна, но жизнеобеспечение еще основано на природных системах.

Следующая фаза -- промежуточная: часть функций жизнеобеспечения будет
выполняться искусственно, а часть -- с использованием природных процессов. При
этом «искусственная» часть будет постоянно возрастать.

Наконец, заключительная фаза: идеальный БТМ -- мир, в котором степень
независимости от природы (точнее: от того, что к этому времени останется от
природы) очень высока (порядка 90 процентов) и продолжает увеличиваться.

Создание БТМ -- долгий процесс, включающий существенно разные фазы. Полный
(идеальный) БТМ отделен от нас, живущих в эпоху «раннего» БТМ, долгими
столетиями. Но первоначальные, прикидочные расчеты по жизнеобеспечению
человечества целесообразнее относить к полному БТМ -- процесс его формирования
может оказаться быстро ускоряющимся. Еще одно предварительное -- перед
расчетами -- соображение. По прогнозам ООН к 2080 году население Земли
стабилизируется и составит не менее 8 миллиардов человек. К этому времени
мощность всех энергетических установок будет составлять примерно $7\cdot
10^{10}$ киловатт. Исходя из этих данных, мы и будем вести расчеты.

\paragraph{Обеспечение кислородом.}
Для дыхания одному человеку требуется 550--600 литров (0.83 килограмма)
кислорода в сутки.

Всему человечеству на весь 2080-й год понадобится $1.6\cdot 10^{15}$ литра, а
на нужды техники (при современной структуре потребления кислорода) --
$6-9\cdot 10^{17}$ литров. При получении кислорода из загрязненного воздуха
глубоким охлаждением затрачивается $0.0004-0.0016$ киловатт-часа на каждый
литр кислорода. Для всего человечества это составит $1.9\cdot 10^9$ киловатта в
год, или 0.27 процента вырабатываемой во всем мире энергии.

Чтобы обеспечить замкнутый цикл, необходимо получать кислород из выделяемого
при дыхании углекислого газа. При разложении углекислого газа электролизом с
применением твердых электролитов на каждый литр кислорода, получаемый в
течение часа, требуется установка мощностью $6-8$ ватт. На каждого человека
необходима установка мощностью 150 ватт, на все человечество -- $1.2\cdot 10^9$
киловатта, или 1.7 процента вырабатываемой в 2080 году энергии.

Обеспечение человечества кислородом в БТМ -- относительно несложная задача,
если речь идет только о дыхании. Иное дело -- искусственное обеспечение
кислородом техники: тут требуется вдвое больше энергии, чем ее будет
вырабатываться во всем мире.  Техника должна стать бескислородной. Прежде
всего, это означает отказ от сжигания угля, нефтепродуктов и газа. Нужен поток
новых изобретений по переходу на бескислородные процессы. Сегодня такие
изобретения невыгодны. Но создавать и разрабатывать их надо именно сегодня.
Завтра будет поздно.

\paragraph{Обеспечение водой.}
Норма потребления воды на одного человека в сутки 2.5 литра, в условиях
пустынь до 10 литров.

На бытовые нужды в крупных городах норма примерно 500 литров. С учетом
расходов на промышленность на каждого человека приходится до 6500 литров воды
в сутки.  При длительных полетах на космических кораблях и орбитальных
станциях суточная норма потребления 2.2-2.5 литра на человека, кроме того, на
санитарно-гигиенические нужды расходуется от 6 до 25 литров воды в сутки.
Известно много физических, химических, электрохимических и биологических
методов получения воды опреснением морской воды или регенерацией ее из отходов
жизнедеятельности людей, использованных санитарно-гигиенических или
технических вод. Расход энергии при этом может составить 8-10 киловатт-часов
на кубометр воды.

\begin{center}
  \newcommand{\abox}[1]{\parbox{2.5cm}{\vskip2pt\raggedright #1\vskip2pt}}
  \begin{tabular}{|c|c|c|c|c|}\hline
    \abox{Назначение} & \abox{Суточная норма на человека (литр)} &
    \abox{Расход на все человечество за год (литр)} & \abox{Мощность установок
      при расходе энергии 0.01 кВт-час на 1 литр} & \abox{Количество энергии в
      \% от всей получаемой в 2080 г.} \\\hline
    \abox{на физиологические нужды}& 3 & $8.78\cdot 10^{12}$ & $1\cdot 10^7$ &
    0.014 \\\hline 
    \abox{на бытовые нужды}& $150\cdots500$ & $0.438\cdots1.46\cdot 10^{15}$ & 
    $5\cdots17\cdot 10^8$ & $0.7\cdots2.4$\\\hline 
    \abox{на промышленность и сельское хозяйство}& 6500 & $1.9\cdot 10^{16}$ &
    $2.2\cdot 10^{10}$ & 30.9\\\hline 
  \end{tabular}
\end{center}

В таблице приведены данные о расходе энергии при различных вариантах
потребления воды. Космонавты обходятся 28 литрами воды в день и меньше,
поэтому можно рассчитывать на снижение расхода воды на бытовые нужды по
крайней мере до 150 литров на человека в день. Ни один из вариантов
удовлетворения потребностей в воде не вызывает особой тревоги -- за
исключением, конечно, нужд промышленности и сельского хозяйства: здесь
необходима интенсивная перестройка на безводную технологию,

Общие расходы на водообеспечение, по всей видимости, не превысят 10—12
процентов от вырабатываемой энергии, из которых на жизнеобеспечение приходится
только 0.014 процента. В настоящее время водоснабжение отнимает 0.7 процента
всей вырабатываемой энергии.

\paragraph{Обеспечение питанием.}
Энергетическая ценность питания одного человека должна составлять 3000
килокалорий в сутки. Всему человечеству ежегодно требуются $1.16\cdot 10^9$
киловатт «пищевой» энергии. Для определения общей потребности в энергии
необходимо знать к.п.д. системы по производству пищевых продуктов. При
использовании природных систем (собирательство, охота) затрачивается в
несколько раз меньше мускульной энергии, чем содержится в добытой пище, но
этот способ требует в 20\,000 раз больших площадей и в 33 раза больше
человеко-часов рабочего времени, чем современная технология производства
питания. Экономия площади и время, техника снижает к.п.д. получения пищи. В
Англии, например, на 1 килокалорию затраченной технической энергии приходится
только 0.4 килокалории произведенных продуктов.

В производстве питания наблюдаются две противоположные тенденции. С одной
стороны, постоянно снижается природно-ресурсный потенциал, что приводит к
уменьшению к.п.д. С другой -- совершенствуется и упрощается технология, что
повышает отдачу от затраченной в производстве питания энергии. Широкое
распространение, например, получает сейчас новая технология в производстве
питания, при которой полностью исключается этап животноводства -- растительный
белок искусственно превращается в равноценный животному.

Пока не известна технология, которая позволяла бы обеспечить замкнутый цикл
воспроизводства продуктов питания без элементов природных систем. На ближайшее
будущее тенденции развития будут состоять в использовании более простых
природных систем с их симбиозом с технологическими процессами. Вместо
экосистем используются их фрагменты, вместо животных -- растения, вместо
растений -- клетчатка и бактерии. Одновременно развиваются технологии
искусственного синтеза пищи. В любом случае к.п.д. производства продуктов
питания не будет ниже (не должен быть ниже!) 3-4 процентов. При этом для
обеспечения питанием человечества потребуется $2.9\cdot 10^{10}$ киловатт или
около 40 процентов всей вырабатываемой энергии. Сейчас сельское хозяйство
потребляет 10 процентов энергии.

В сущности, обеспечение БТМ основными продуктами, необходимыми для поддержания
жизни, почти всецело зависит от «энергетической платы». Даже современный
уровень техники гарантирует энергетику, необходимую и достаточную для
постройки БТМ вместимостью в 8 миллиардов человек. Населению БТМ придется
отказаться, конечно, от автотранспорта и авиатранспорта (в их современных
формах, связанных с чудовищным расходом кислорода, воды и ценных продуктов),
но жить в БТМ -- дышать, пить, питаться -- будет возможно.

\paragraph{Мысль 3.}
Технически (энергетически) создание БТМ осуществимо уже на современном уровне
техники. Это отчасти печальный вывод. Ибо нет самого сильного фактора, который
бы сдерживал вымирание природного мира. Как ни грустно, без природы можно
выжить, построив БТМ. И природу быстро добьют ...

Наши выкладки — «первоприкидочные» оценки «самого-самого» минимума. Между тем,
природный мир построен с колоссальной избыточностью. Эта избыточность с точки
зрения, так сказать, самого природного мира обеспечивает ему высокую
надежность. А с точки зрения человека, и если так можно выразиться, с точки
зрения науки и техники, избыточность дает возможность познания и развития.
Исследование и перестройка мира связаны с ошибками. В мире без избыточности
такие ошибки означали бы катастрофу. Миру природному или бесприродному — нужна
избыточность.

Один пример: судьба Кара-Богаз-Гола.  В 1982 году 200-метровый пролив между
Кара-Богаз-Голом и Каспием перегородила дамба. Такое решение было принято
из-за обмеления Каспия якобы в результате огромного водоотбора на Волге.
Позже, когда соли Кара-Богаза из ценного сырья превратились в разносимый по
огромным площадям яд, оказалось, что потери от высыхания Кара-Богаза выше
потерь от обмеления Каспия.  Попутно выяснилось, что уровень Каспия не
понижается, а повышается: главным фактором, определяющим его уровень,
оказались тектонические процессы его дна, а не испарение воды Кара-Богаз-Гола
и не водоотбор на Волге ... В БТМ, не имеющем избыточности, такой «прокол»
означал бы глобальную катастрофу или, во всяком случае, чрезвычайное бедствие.
В мире с высокой избыточностью легко переносятся и значительно большие
«проколы» ...

Проектирование БТМ с достаточно высокой степенью избыточности в первом
приближении сводится к созданию запасов энергии и ограниченного числа наиболее
важных веществ. Такая задача не выходит за рамки реальности. Причем «запасы»
вовсе не означают «неиспользуемые запасы».

Зимой высокогорное село Куруш в Дагестане полностью оторвано от цивилизации. С
древних времен заборы рядом с домами делают там из кизяка. Если к концу зимы
топлива не хватает, заборы постепенно разбираются на отопление. Летом запасы
восстанавливаются. В 1962 году аналогичное решение выдвинула американская
фирма «Грумман Эйркрафт»: изготавливать часть внутренних перегородок
космического корабля «Аполлон-С» из прессованной пищевой смеси. Другой вариант
создания запасов -- мебель, содержащая бертолетовую соль. При сгорании такой
мебели выделяется кислород.

\paragraph{Мысль 4.}
В принципе можно построить БТМ с высокой избыточностью (БТМ-ВИ). Для этого
потребуется сделать и реализовать множество новых изобретений. Необходима
будет также более высокая степень осмотрительности при исследовании и
перестройке мира. Учитывая ускоряющиеся темпы развития науки и техники,
следует предполагать, что возможность создания БТМ-ВИ появится уже через
80—100 лет. В том, что можно на какое-то время создать БТМ или даже БТМ-ВИ,
еще, как говорится, мало радости. Мир нужен человеку на очень долгие времена,
практически навечно. Человек должен чувствовать: мир будет всегда. Только в
вечном мире появятся стимулы к продолжению прогрессирующей эстафеты поколений,
к сохранению и развитию цивилизации.

Никакими техническими средствами нельзя обеспечить вечность БТМ. Это проблема
по преимуществу социальная, ведь речь идет о строительстве МИРА, а не
благоустроенной и долговременной клетки.

\paragraph{Мысль 5.}
Социально-устойчивый и развивающийся БТМ (СУР-БТМ, СУР-БТМ-ВИ) должен быть
миром, неисчерпаемым для познания. Красота этого мира тоже должна быть
неисчерпаемой. Только такой мир будет вечным.

Обеспечить неисчерпаемость познания сравнительно легко, например,
исследованиями микромира (глубин вещества) и макро-мира (Вселенной).
Неизмеримо труднее создать мир, неисчерпаемый по красоте. «Запасы красоты» —
по этому «показателю» техника сильнее всего уступает природе. Технические
системы, как правило, приобретают самоценную красоту лишь на последнем этапе
своего существования (такова, например, красота чайных клиперов или
деревянного зодчества). Обычно «техническая красота» — это красота
узкофункциональная (обтекаемая форма скоростных транспортных средств) или
подражающая природе.

В 1982 году, когда начиналась работа по теме «БТМ», мы обошли бакинские
магазины электротоваров. В продаже были 20 типов электрокаминов. 18 из них
наивно и грубо имитировали «горение дров». Два электрокамина (самые дешевые)
были узкофункциональны -- они представляли собой примитивные
электронагреватели. Даже в патентной литературе не оказалось ни единого
изобретения, в котором «обыгрывалась» бы красота физических эффектов, присущих
только технике и не используемых природой.

Человек появился и развивался в мире, исключительно благоприятном по
познаваемости и красоте. Это одна из главных причин быстрого «очеловечивания
человека». Но столь же быстро (даже намного быстрее!) пойдет обратный процесс,
если исчерпаются таинственность и красота мира.

Природа имеет колоссальный «запас красоты». В БТМ такой запас создать не
удастся. Неисчерпаемость красоты в БТМ может быть обеспечена только
возможностью ее постоянного возникновения и развития.

Поясним эту мысль. До музыки был только природный шум: свист ветра, звуки
леса, пение птиц, ритм прибоя ... Музыка начиналась с подражания голосам
природы. Но очень скоро звукоподражание выросло в МУЗЫКУ. Это единственный
бесспорный случай, когда «техническая красота» (в смысле «красота, которая
создана искусственно») сильнее и неисчерпаемее «природной красоты».

\paragraph{Мысль 6.}
Создание СУР-БТМ и СУР-БТМ-ВИ немыслимо без многих новых социально-технических
изобретений типа «от шума к музыке». Решение этих сложнейших суперзадач
требует огромного расхода сил и времени. Поэтому начинать надо сегодня. Завтра
будет поздно. В природном мире можно было мыслить методом проб и ошибок.
Колоссальная избыточность природы покрывала издержки от ошибок, прощала
медлительность и неэффективность решения задач.

\paragraph{Мысль 7.}
При построении СУР-БТМ и СУР-БТМ-ВИ и для жизни в этих мирах необходимо иное
мышление -- эффективное, исключающее крупные просчеты, учитывающее диалектику
стремительно развивающегося мира. Отдаленным прототипом такого мышления можно
считать теорию решения изобретательских задач (ТРИЗ). Точнее -- общие принципы
сильного мышления, заложенные в ТРИЗ

Мы живем в мире, где главное -- материальное потребление. За столетие такой
мир съел половину природы и четыре пятых ее красоты (цифры примерные -- речь
идет о порядке величин).

\paragraph{Мысль 8.}
В БТМ неизбежно придется отказаться от материально-потребительского образа
жизни, от материального потребления как главной жизненной ценности. Главным
вектором БТМ должно стать творчество, направленное на углубление и расширение
познания и на обогащение красоты мира. В творчество будет вовлечена большая
часть населения БТМ. Потребуется развитая система воспитания творческого
мировоззрения и обучения творческой технологии мышления. Дальний прототип
такой системы -- нынешние занятия по ТРИЗ.

Полагалось бы закончить эту проблемную статью традиционным обращением к
читателю: авторы будут благодарны за критику и замечания. Но если говорить
откровенно, мы знаем замечания и возражения. И отношение наше к ним скорее
грустное, чем благодарное. Люди привыкли (и это естественно) к природному
миру.  Мысль о неизбежности превращения этого мира в БТМ вызывает крайне
резкие отрицательные эмоции. Список возражений, которые мы выслушали, пока шла
работа над статьей, занял бы не один десяток страниц. Возражения эти основаны
на эмоциях (или преимущественно на эмоциях) — и потому неопровержимы. Ну что
можно ответить на такой довод: «Все это чушь! Я не представляю, как можно жить
в мире, где не будет неба, моря, леса, животных ...». Или на такое
соображение: «Люди не могут жить без войн и столкновений. А в БТМ любой
военный конфликт будет означать конец света ...».

Первое время мы пытались противопоставить эмоциям логику, разум, расчеты. Мы
доказывали, что переход в БТМ уже идет и обратного пути нет. Мы осторожно
намекали, что наши далекие предки тоже не представляли себе жизни без пещер и
мамонтов. Мы проводили аналогии между БТМ и кораблем: на кораблях не бывает
междоусобных войн — иначе мореплавание было бы невозможно ... Все это
оставалось пустым звуком. Оппоненты упорно твердили: этого не может быть,
потому что этого не было никогда!

Эмоции невозможно переспорить, перекрыть, остановить. Им надо открыть «зеленый
свет», пусть свободно выплескиваются. Столкнувшись с невероятным (с тем, что
кажется невероятным), человек должен сначала тысячу (или десять тысяч) раз
воскликнуть: «Этого не может быть!», чтобы потом, в конце концов, задуматься:
«А что если? ...».

Итак, авторы заранее благодарны за критику и замечания. И обещают
неукоснительно учесть все возражения.

\end{document}
