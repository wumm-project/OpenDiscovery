\documentclass[11pt,a4paper]{article}
\usepackage{od}
\usepackage[russian]{babel}
\usepackage[utf8]{inputenc}

\title{Теория Решения Изобретательских Задач. Справка «ТРИЗ-88»}

\author{Altshuller, Genrich Saulovich}
\date{1988}

\begin{document}
\maketitle
\tableofcontents

\section{Наука изобретать}

\subsection{Метод проб и ошибок -- катастрофически плохая технология
  творчества} 

Изобретательство -- древнейшее занятие человека. С изобретением первых орудий
труда и начинается история человека. За многие тысячи лет, прошедшие с тех
пор, все изменилось, неизменной осталась только технология создания новых
изобретений -- метод проб и ошибок: «А что, если сделать так?.. Ах, не
получается?.. Ну, тогда можно попробовать сделать вот так...» Эта технология
творчества предельно неэффективна в условиях современной НТР.

В СССР ежегодно выполняется около 150\,000 научно-исследовательских
разработок.  Приблизительно две трети их прерываются на стадии эксперимента
или испытания нового образца. Огромные средства оказываются затраченными
впустую.  Из 50\,000 разработок, что доходят до стадии внедрения, лишь тысяча
находит более или менее широкое внедрение (»Социалистическая индустрия» от
26.06.82 г.). Таким образом, из 150 000 разработок жизненными оказываются
только 1\,000, т.е. менее 0,7\%!

Представьте себе аэропорт, в котором из 150 ежедневно взлетающих самолетов
поднимается только один, а остальные разбиваются при разбеге и взлете. Или же
представьте строительную организацию, у которой из 150 домов обваливаются в
процессе постройки 100, а в 49-ти домах пригодны только отдельные квартиры, и
лишь один (!) дом может быть полностью заселен. Таков по эффективности метод
проб и ошибок -- самая расточительная из всех технологий. Применение этого
метода в современном промышленном обществе неизбежно приводит к разорению
общества, к упадку темпов его прогресса, к застою экономики и производства.

М.С. Горбачев в докладе на пленуме ЦК КПСС 25 июня 1987 г. сказал: «Нельзя
успешно двигаться вперед методом проб и ошибок, это дорого обходится обществу.
Искусство политического руководства требует умения выявлять и эффективно
разрешать противоречия...»

Конечно, речь в докладе идет о политике, но политика базируется на экономике,
а экономика -- на творческом решении задач. К методу проб и ошибок привыкли,
слова «творчество» и «перебор вариантов» стали синонимами. Упорство в переборе
вариантов рассматривают как доблесть. Вот строки из обычного очерка об
изобретателях: «Шли к решению проблемы почти на ощупь, перебрали множество
теорий, в конце каждой из которых стояло: нуждается в практической
проверке. Поставили тысячи экспериментов только для того, чтобы убедиться:
пошли не туда. Испытали десятки конструкций приборов, перепаяли сотни метров
проволоки и извели не поддающееся учету количество кинопленки»
(Е. Марголин. Как падают яблоки. Изд. «Лиесма», Рига, 1976, с. 8).

За не решенные вовремя изобретательские задачи расплачиваться приходится не
только недополученными прибылями, но и жизнями людей. Потери времени, сил и
жизней из-за несовершенства метода проб и ошибок страшнее потерь от чумы,
землетрясений и наводнений.

\subsection{Методы активизации перебора вариантов -- путь в тупик}

Иногда пытаются модернизировать метод проб и ошибок или интенсивнее его
использовать. Такова, например, японская практика. Ее сущность: все служащие
все время должны перебирать всевозможные варианты решений. На прогулке, дома,
во время еды -- всегда! Тосабуро Наката приучил себя перебирать варианты в
туалете (чтобы не пропадало время) и через два года изобрел шариковую ручку,
став национальным героем...

Главный недостаток метода проб и ошибок -- это, во-первых, медленное
генерирование новых идей, а во-вторых, отсутствие защиты от психологической
инерции (т.е. выдвижение идей тривиальных, обыденных, неоригинальных). С 20-х
годов нашего столетия в разных странах стали появляться методы активизации
перебора вариантов. Один из наиболее распространенных методов такого рода --
мозговой штурм. Решение задачи проходит в два этапа. На первом этапе
(генерирование идей) запрещена всякая критика, поощряются «дикие», явно
неосуществимые, даже фантастические предложения (чтобы по возможности
устранить психологическую инерцию). На втором этапе эксперты критически
оценивают результаты штурма, пытаясь отобрать рациональные идеи.

Другой метод -- морфологический анализ. Суть его состоит в построении таблиц,
которые должны охватить все мыслимые варианты. Например, требуется предложить
новую упаковку для изделий. Если на одной оси записать, скажем, двадцать видов
материала (металл, дерево, картон и т.д.), а на другой -- двадцать видов формы
(сплошная жесткая упаковка, сплошная гибкая упаковка, рейчатая упаковка,
сетчатая и т.д.), получится таблица, включающая 400 сочетаний, каждое из
которых соответствует одному варианту. Можно ввести и другие оси,
неограниченно наращивая число полученных вариантов. А затем в безграничном
море этих вариантов -- в основном, «пустых» -- надо найти несколько разумных
идей.

Есть и другие методы активизации перебора вариантов, например синектика, метод
фокальных объектов, метод контрольных вопросов и пр. Все эти методы обладают
общими, принципиально непреодолимыми, недостатками:
\begin{itemize}
\item[а)] нет механизма для составления списка всех возможных вариантов (а
  значит, нет гарантии выхода на самые выгодные, экономичные решения),
\item[б)] нет объективных критериев отбора лучших вариантов: предложения
  оцениваются специалистами, и выбирают они, естественно, то, что подсказывает
  им здравый смысл (т.е. психологическая инерция): генерирование нетривиальных
  идей сводится на нет тривиальным отбором.
\end{itemize}

Причина неэффективности подобных методов в том, что они не меняют сути старой
технологии перебора вариантов, сам этот перебор. Нужен принципиально новый
инструмент творчества, а не «косметический» ремонт старого.

Методы активизации хороши при решении простых задач и неэффективны для задач
сложных, а таких задач в современной изобретательской практике большинство.
Именно от решения сложных задач зависят темпы прогресса.

Со времени своего появления эти методы активизации не претерпели существенных
изменений, это означает, что выбран неверный путь, ведущий в тупик. Нужна иная
-- более эффективная -- технология решения изобретательских задач.

\subsection{Что такое ТРИЗ?}

В 1946 году в СССР началась работа над созданием научной технологии
творчества.  Новая технология получила название ТРИЗ -- теория решения
изобретательских задач. Первая публикация по ТРИЗ относится к 1956 году
(7). Дальнейшее развитие отражено в книгах (8-12, 14-16) и в материалах,
регулярно публиковавшихся журналом «Техника и наука» в 1979-1983 г.г. (13).

Отечественная теория решения изобретательских задач принципиально отличается
от метода проб и ошибок и всех его модификаций, основная идея ТРИЗ:
технические системы возникают и развиваются не «как попало», а по определенным
законам: эти законы можно познать и использовать для сознательного -- без
множества «пустых» проб -- решения изобретательских задач. ТРИЗ превращает
производство новых технических идей в точную науку. Решение изобретательских
задач -- вместо поисков вслепую -- строится на системе логических операций.

Теоретичекой основой ТРИЗ являются законы развития технических систем. Прежде
всего, это законы материалистической диалектики. Используются также некоторые
аналоги биологических законов, ряд законов выявлен изучением исторических
тенденций развития техники, широко применяются общие законы развития систем.
Законы проверены, уточнены, детализированы, а иногда и выявлены путем анализа
больших массивов патентной информации по сильным решениям (десятки и сотни
тысяч отобранных патентов и авторских свидетельств).

Весь инструментарий ТРИЗ, включая фонды физических, химических, геометрических
эффектов, также выявлялся и развивался на основе изучения больших массивов
патентной информации, вообще, каждое нововведение в ТРИЗ проходит тщательную
проверку и корректировку на патентных и историко-технических материалах. В
этом смысле ТРИЗ можно считать обобщением сильных сторон творческого опыта
многих поколений изобретателей: отбираются и исследуются сильные решения,
критически изучаются решения слабые и ошибочные.

Главный закон развития технических систем -- стремление к увеличению степени
идеальности: идеальная техническая система -- когда системы нет, а ее функция
выполняется. Пытаясь обычными (уже известными) путями повысить идеальность
технической системы, мы улучшаем один показатель (например, уменьшаем вес
транспортного средства) за счет ухудшения других показателей (например,
снижается прочность). Конструктор ищет компромиссное решение -- оптимальное в
каждом конкретном случае. Изобретатель должен сломать компромисс: улучшить
один показатель, не ухудшая других. Поэтому в наиболее распространенном случае
процесс решения изобретательских задач можно рассматривать как выявление,
анализ и разрешение технического противоречия.

Основным рабочим механизмом совершенствования ТС и синтеза новых ТС в ТРИЗ
служит алгоритм решения изобретательских задач (АРИЗ) и сиcтема
изобретательских стандартов.

Решение задач по АРИЗ идет без множества «пустых» проб. Планомерно, шаг за
шагом, по четким правилам корректируют первоначальную формулировку задачи,
строят модель задачи, определяют имеющиеся вещественно-полевые ресурсы (ВПР),
составляют идеальный конечный результат (ИКР), выявляют и анализируют
физические противоречия, прилагают к задаче операторы необычных, смелых,
дерзких преобразований, специальными приемами гасят психологическую инерцию и
форсируют воображение.

Сходные противоречия разрешают однотипными приемами, наиболее сильные приемы
-- комплексные (сочетания нескольких приемов, часто -- сочетания приемов с
физ-, хим-, геомэффектами). Самые сильные комплексные приемы образуют систему
стандартов -- аппарат ТРИЗ для решения типовых изобретательских задач. Следует
подчеркнуть, что стандартные задачи стандартны только с позиций ТРИЗ;
изобретатель, незнакомый с ТРИЗ, воспринимает такие задачи как нетипичные,
сложные. Стандарты могут быть использованы для решения задач, сложных даже с
позиций ТРИЗ; такие задачи решаются сочетанием нескольких стандартов.

Важное значение имеет в ТРИЗ упорядоченный и постоянно пополняемый
информационный фонд: указатели применения физических, химических и
геометрических эффектов, банк типовых приемов устранения технических и
физических противоречий. Этот фонд -- операционная основа всех инструментов
ТРИЗ.

Особый раздел ТРИЗ -- курс развития творческого воображения (РТВ). В этом
курсе, в основном, на нетехнических примерах отрабатывается умение применять
операторы ТРИЗ. Курс РТВ расшатывает привычные представления об объектах,
ломает жесткие стереотипы.

Знание законов развития ТС позволяет решать не только имеющиеся
изобретательские задачи, но и прогнозировать появление новых задач. Результаты
такого прогнозирования значительно точнее, чем полученные с помощью
субъективных методов, например экспертными оценками. ТРИЗ стремится к
планомерной эволюции ТС. Таким образом, современная ТРИЗ превращается в ТРТС
-- теорию развития технических систем.

ТРИЗ возникла в технике, потому что здесь был мощный патентный фонд,
послуживший фундаментом теории. Но, помимо технических, существуют и другие
системы: научные, художественные, социальные и т.д. Развитие всех систем
подчинено сходным закономерностям, поэтому многие идеи и механизмы ТРИЗ могут
быть использованы при построении теорий решения нетехнических творческих
задач. Такая работа ведется (23). В частности, с помощью механизмов,
используемых в ТРИЗ, была открыта ветроэнергетика растений (24) и объяснены
парадоксы, связанные с эффектом Рассела (25).

Аппарат теории решения изобретательских задач постоянно проверяется,
корректируется и совершенствуется в ходе практического применения. Ежегодно в
сотнях школ и курсов ТРИЗ слушатели решают множество учебных и неучебных
(новых производственных) задач. Анализ письменных работ позволяет объективно
определять причины ошибок: совершены ли они по вине преподавателя, по вине
слушателя или имеет место сбой того или иного инструмента ТРИЗ. Накопленная
информация тщательно изучается, это позволяет быстро развивать методику
обучения ТРИЗ и саму теорию.

\section{ТРИЗ успешно работает}

\subsection{Эффективность}

До 70-х годов обучение ТРИЗ велось преимущественно на экспериментальных
семинарах, с 1970 года обучение сосредоточивается в постоянно действующих
учебных центрах: народных университетах научно-технического творчества
(Ленинград, Днепропетровск, Петрозаводск), общественных институтах и школах
изобретательского творчества (Кишинев, Минск, Новосибирск, Ангарск,
Владивосток), учебу организуют также центры НТТМ, различные министерства,
ведомства, предприятия. Занятия ведутся в институтах патентоведения, в ряде
отраслевых институтов повышения квалификации (ИПК). В 1980 году в ИПК
Минэлектротехпрома впервые начата подготовка специалистов по ТРИЗ для
постоянной работы в подразделениях функционально-стоимостного анализа (ФСА).

Об эффективности обучения ТРИЗ можно судить на примере Днепропетровского
народного университета научно-технического творчества: с 1972 года по 1982 год
университет сделал 9 выпусков, слушатели получили -- к 1982 году -- 350
авторских свидетельств на изобретения, экономия от внедрения новых технических
решений составляет десятки миллионов рублей (см. В. Некрылов, В. Каленик. 9
выпусков, 500 слушателей. -- Журнал «Техника и наука», 1982, N 1, с. 24).

Еще пример: «За период обучения (два года) 30 выпускников, используя при
решении своих практических задач полученные в школе знания, подали 103 заявки
на предполагаемые изобретения, получили 65 положительных решений и 38
авторских свидетельств. По предварительным данным, некоторые из внедренных
изобретений и 99 рацпредложений дали экономический эффект почти в полмиллиона
рублей, но самое главное достижение -- формирование специалиста с активной
творческой позицией, понимающего шире и глубже стоящие перед ним задачи,
такого специалиста, который сейчас остро необходим для выполнения планов,
намеченных в постановлении «О мерах по ускорению научно-технического прогресса
в народном хозяйстве» (Калужская газета «Знамя» за 26.10.83 -- О школе ТРИЗ в
научном городке Обнинске).

Из статьи «Изобретайте, вы талантливы» В. Митрофанова, С. Литвина,
А. Сушанского в газете «Вечерний Ленинград» за 11.10.80: «Вот несколько
конкретных примеров. Ведущий конструктор ВПТИЭНЕРГОМАШ Ю.Г. Ермаков до
обучения ТРИЗ подал около 40 заявок и получил только 5 авторских свидетельств,
после обучения он подал в в течение двух лет 42 заявки и получил 24
свидетельства на изобретения; заместитель заведующего кафедрой высшего
инженерного морского училища им. адмирала С.О. Макарова А.В. Смыков до
знакомства с методикой технического творчества подал 2 заявки и получил 2
положительных ответа, после окончания нашего университета подал 11 заявок и
получил 11 авторских свидетельств, два из которых патентуются за рубежом и еще
два внедрены с большим экономическим эффектом».

Из письма одесского преподавателя ТРИЗ, к.т.н. С.Д. Тетельбаума (май 1982 г.):
«В течение 9 выпусков ОИП подготовлено по ТРИЗ 190 человек, из них процесс
обучения завершили подачей реальных заявок на изобретения по задачам, решенным
с помощью ТРИЗ, более 170 человек».

Из письма В.М. Жабина, руководителя школы научно-технического творчества на
Красногорском оптико-механическом заводе (январь, 1986 г.):

»Информация по эффективности ТРИЗ:

- изобретения Эфроимсона В.Г. внедрены в киноаппарате «Кварц-8Х», поступившем
в розничную продажу в 1985 г.:

А.с. 475591 -- экономический эффект -- 63.000 руб.

А.с. 476536 -- экономический эффект -- 45.000 руб.

А.с. 890351 -- экономический эффект -- 67.500 руб.

Всего: 175.500 руб.

- изобретения Полиновского В.А., которые он сделал с соавторами, внедрены:

А.с. 733302 -- экономический эффект -- 71.000 руб.

А.с. 733303 -- экономический эффект -- 54.000 руб.

А.с. 733320 -- экономический эффект -- 11.000 руб.

А.с. 1013892 -- экономический эффект -- 50.000 руб.

Всего: 182.250 руб.

- внедрено изобретение Мейтина В.А.:

А.с. 693110 -- экономический эффект -- 70.000 руб.

Всего по школе: 431.750 руб.»

Из справки Сибирского филиала «Оргстройпроект» (Ангарск, 20.01.83 г.): «О
влиянии учебного семинара по ТРИЗ на творческую активность сотрудников.
Сотрудниками предприятия в течение 1982 года оформлено 80 заявок на
предполагаемые изобретения, динамика следующая: с января по октябрь -- в
среднем 5 заявок в месяц, после проведения учебного семинара по ТРИЗ (с 4 по
22 октября, преподаватель т. Г.С. Альтшуллер), количество подаваемых заявок
составило 15 штук в месяц, количество изобретений увеличилось в 1,9 раза».

Из статьи «ТРИЗ и фантастика» В.К. Гребнева, токаря-инструктора объединения
«Турбомоторный завод», председателя заводского совета новаторов, заместителя
председателя областного совета новаторов (в сборнике «Истоки новаторства:
очерки о психологии поиска», выпуск второй, Свердловск,
Сред.-Урал. книжн. изд., 1986 г.): «Признаюсь... решение рационализаторских
задач в последние годы отнимает у меня куда меньше времени, чем прежде,
во-первых, потому что ТРИЗ, которую я раньше не знал, дает достаточно надежные
способы решения технических задач, и, что особенно важно, способы
индивидуальные, в отличие от коллективных приемов мозгового штурма, синектики
и других, во-вторых, в памяти накапливается немалое число хорошо отработанных
комбинаций, которые позволяют быстро решать новые задачи уже старыми для меня
способами».

Из письма изобретателя В.Е. Лукьяненко (Москва) в редакцию журнала «Техника и
наука» (от 21.05.82): «Я работаю инженером, до знакомства с АРИЗ и ТРИЗ имел
одно авторское свидетельство, сейчас, после обучения ТРИЗ, получил 10
авторских свидетельств; все они внедрены, среди них есть изобретения
(а.с. 569782) с суммарным экономическим эффектом более 97 тыс. руб.».

Из статьи заместителя председателя совета молодых ученых и специалистов
Львовского обкома ЛКСМУ Ю. Мариловцева (газета «Комсомольское знамя», Львов,
11.11.79 г.): «... Лучший молодой изобретатель Украины 1978 года и города
Львова 1979 года Олег Копыл убежден, что своими лучшими результатами в
изобретательстве он обязан знанию теории решения изобретательских задач
(ТРИЗ)».

Из письма изобретателя (Ленинград, февраль 1986 г.): «Я -- Энглин Роберт
Кальманович, старший научный сотрудник НПО «Ритм», кандидат технических наук,
изучал ТРИЗ в Ленинградском народном университете научно-технического
творчества в 1978-79 гг. До поступления в университет имел 46 авторских
свидетельств, которые получил за 16 предыдущих лет. После окончания
университета и изучения ТРИЗ за 6 последующих лет стал автором еще 88
изобретений. Все мои 136 изобретений являются служебными и созданы по тематике
выполняемых мною НИР».

В статье полковника В. Колбенкова «Изобретатель лейтенант Федоров» (журнал
«Тыл вооруженных сил» №10-1987г., с. 72-73) рассказывается о молодом военном
инженере, имеющем более двадцати авторских свидетельств. Некоторые из них
зарегистрированы не только у нас в стране, но и за рубежом, в самых технически
развитых государствах. Автор статьи, анализируя обстоятельства становления
изобретателя, пишет: «Первое -- это то, что его отец -- корабел, работающий
инженером-конструктором и имеющий на своем счету более полусотни изобретений.
Он часто рассказывал сыну о том, как вставали перед ним неразрешимые проблемы
-- неразрешимые традиционными способами, как ставил себе задачи, вел поиск и,
наконец, находил оригинальное решение, по сути дела, он учил сына алгоритму
решения изобретательских задач -- дисциплине, которая ныне уже введена в
некоторых вузах».

\subsection{Информация к размышлению (I)}

В декабре 1968 года впервые были организованы занятия с будущими
преподавателями ТРИЗ. Стоили эти занятия около 6000 рублей. В апреле 1969 года
один из слушателей М.И. Шарапов рассказывал в газете «Магнитогорский металл»
об изобретении, сделанном по ТРИЗ. Позже была подсчитана экономия -- 42.000
р. в год только на Магнитогорском комбинате. Это перекрыло расходы на обучение
во всех школах ТРИЗ в течение следующих пяти лет. Между тем у того же Шарапова
к 1977 году было уже свыше 30 авторских свидетельств (см. журнал «Техника и
наука» №4, 1980 г., с. 27). У другого слушателя курсов 1968 года Ю.В. Чиннова
через 10 лет число авторских свидетельств превысило 100. М.И. Шарапов и
Ю.В. Чиннов ныне -- заслуженные изобретатели.

Мы не собирали статистику по всем изобретениям, созданным с помощью ТРИЗ.
Систематически издаются книги, методические пособия, публикуются статьи,
рассказывающие о новейших разработках в ТРИЗ. Сегодня элементы ТРИЗ
используются очень многими изобретателями и рационализаторами, общую отдачу
определить практически невозможно, если же суммировать сведения только по
нескольким главным школам, получится примерно такая картина. За 1972-81 г.г.
через эти школы ТРИЗ прошло примерно 7000 слушателей, поданы почти 11000
заявок, получено свыше 4000 авторских свидетельств (более половины заявок еще
на рассмотрении), экономия от внедрения составляет миллионы рублей, общие
расходы на обучение не превышают ста тысяч.

\subsection{ТРИЗ и ФСА}

По Постановлению ЦК КПСС (»Правда», 1982 г., № 153) идет интенсивная работа по
внедрению функционально-стоимостного анализа (ФСА). На ряде предприятий ТРИЗ
принята в качестве основного инструмента для решения задач, выявленных в ходе
ФСА (19, 20). Это существенно ускоряет процесс распространения и внедрения
ТРИЗ: подразделения ФСА в КБ, НИИ и на заводах становятся школами ТРИЗ и
постоянными «потребителями» теории.

Из статьи зам. министра электротехнической промышленности Ю. Никитина (журнал
«Коммунист», 1982 г., №11, с. 71): «В Ленинградском производственном
объединении «Электросила» при проведении ФСА используют созданную в нашей
стране теорию решения изобретательских задач... В результате анализа изделия
низковольтной аппаратуры -- контактора серии КП-2000 -- было сформулировано
около пятидесяти предложений. Их реализация даст годовую экономию в размере
250 тысяч рублей, позволит сберечь до 450 килограммов серебра».

Ленинградский ЦНТИ выпустил информационный листок (№217-86, УДК
608.1:658.511:005) «Метод проведения функционально-стоимостного анализа с
применением теории решения изобретательских задач» (внедрено в июле 1985 г.).
Материал поступил в ЦНТИ 14 февраля 1986 г. Составители: преподаватели и
разработчики ТРИЗ и ФСА В.М. Герасимов и С.С. Литвин. В информационном листке,
в частности, отмечено: «Внедрение метода ФСА с применением ТРИЗ в ЛПЭО
«Электросила» при производстве электрокипятильников позволило получить годовой
экономический эффект 80.000 р. на 2,5 млн. штук».

Зам. пред. Госплана Латвии В.А. Лейтан и специалист по ФСА и ТРИЗ И.Б. Бухман
в своей брошюре «Применение ФСА и теории решения изобретательских задач в
промышленности Латвийской ССР» (ЛатНИИНТИ, Рига, 1985 г.) пишут: «Опыт
использования ФСА и ТРИЗ в промышленности Латвийской ССР позволяет
констатировать следующее: достигнут определенный прогресс в эффективности
использования ФСА и ТРИЗ. В 1984 году экономический эффект составил 1,5
миллиона рублей...»

В конце ноября 1987 года в Москве состоялся международный семинар:
«Функционально-стоимостный анализ и повышение технико-экономического уровня
изделий». Профессор кафедры конструкций судов А.Л. Васильев опубликовал отчет
об этом семинаре в газете Ленинградского кораблестроительного института «За
кадры верфям» (от 12 января 1988 г.). В отчете, в частности, сказано
следующее: «... ФСА обеспечивает гармоничное сочетание между техникой и
экономическим осмыслением ее функционирования, а ТРИЗ помогает находить именно
новые, нетривиальные решения. Думаю, что такой подход позволит значительно
усилить мощь человеческого интеллекта и обеспечит прорыв на стратегическом
направлении повышения эффективности судостроения».

\section{Культура творческого мышления}

\subsection{ТРИЗ и ТРТЛ}

Каждый инструмент оказывает обратное действие на человека, использующего этот
инструмент. ТРИЗ -- инструмент для тонких, дерзких, высокоорганизованных
мысленных операций. Решение одной задачи еще не меняет стиля мышления, но в
ходе занятий решаются десятки, сотни задач, постепенно мышление
перестраивается: становится более гибким и управляемым.

Вот портрет изобретателя, овладевшего ТРИЗ: «Тесное знакомство с Просяником и
его работой ломает привычное представление о типичных чертах изобретателя
(сколько таких «чудаков» видели мы в кино, литературе, встречали в жизни!) --
упорство, самоуверенность, некоммуникабельность, непрактичность в обычных
житейских делах... Просяник совсем иной. Типичный изобретатель новой формации,
высококвалифицированный специалист по теории изобретательства, по
направленному поиску -- необходимую уверенность он получает от знания
закономерностей развития техники. А вместо тех самых традиционных
«изобретательских качеств» ТРИЗ воспитывает иное -- диалектическое мышление,
способность видеть в любых технических (да и не только технических) системах
противоречия, мешающие развитию, умение устранять эти противоречия, разрешать
на основе системного мышления, способности воспринимать любой предмет, любую
проблему всесторонне, во всем многообразии их связей» (»Социалистическая
индустрия», 18 декабря 1984 г.).

ТРИЗ обеспечивает выход на решение, близкое к идеальному, но творческий
процесс не сводится к одному лишь поиску решения. Необходимо довести найденную
идею до уровня работоспособной и технологичной конструкции, «обжелезить» ее,
добиться как можно более широкого внедрения. А затем -- взяться за решение
новой проблемы. Из практики известно, что средний срок внедрения среднего по
уровню изобретения составляет 7-10 лет. Это значительный отрезок времени в
жизни человека. Борьба за внедрение часто связана с большими личными потерями,
с колоссальными затратами сил и времени, непониманием окружающих,
необходимостью «пробивать» идею. Новатору порой приходится терпеть и
материальные лишения, и отчуждение от родного коллектива. Гораздо спокойнее
жить без творчества, быть «как все», не «фантазировать»... Как заставить
человека выйти из болота обыденности, презреть отчетливо видимые трудности и
вступить в схватку с косностью и консерватизмом?

Общие призывы и лозунги здесь бессильны. Необходимо тщательно, шаг за шагом
готовить человека к предстоящим творческим битвам, к возможным временным
поражениям и неизбежным трудностям. Человек, знающий о подстерегающих в пути
опасностях, сумеет проложить верный, наиболее разумный маршрут.

Для формирования активной творческой позиции нужны как минимум шесть качеств:
\begin{itemize}[noitemsep]
\item[1.] Наличие достойной цели -- новой (или недостигнутой), значительной,
  общественно-полезной.
\item[2.] Умение программировать достижение поставленной цели.
\item[3.] Большая работоспособность по выполнению намеченных планов.
\item[4.] Умение решать творческие задачи в выбранной области, владение
  техникой преодоления противоречий на пути к цели.
\item[5.] Готовность «держать удар»: отстаивать свои идеи, выносить
  непризнание, непонимание.
\item[6.] Результативность: на пути к конечной цели должны регулярно
  вырабатываться промежуточные результаты.
\end{itemize}
Воспитание комплекса творческих качеств -- главная цель жизненной стратегии
творческой личности (ЖСТЛ). Метод построения ЖСТЛ обычный для всех
исследований в ТРИЗ: анализ больших информационных массивов (с целью выявления
общих закономерностей). Изучено свыше тысячи биографий творческих личностей.

Удалось проследить становление и развитие творческой личности на протяжении
всей жизни. На историко-биографических примерах убедительно доказано:
творческий образ жизни доступен каждому, для этого не нужны особые
прирожденные способности или сверхблагоприятные условия. В силах любого
человека выбрать достойную цель и начать планомерную борьбу за ее достижение.

Подробно рассматривая путь к цели, ЖСТЛ дает человеку суммированный жизненный
опыт поколений творцов: предупреждает о типичных опасностях, рекомендует
конкретные методы их преодоления, предсказывает наиболее сильные ходы.

Систематические исследования по ЖСТЛ постепенно формируют новую область знания
-- теорию развития творческой личности (ТРТЛ).

\subsection{ТРИЗ -- рабочий инструмент диалектики}

ТРИЗ использует законы материалистической диалектики для организации
творческой деятельности. Механизмы ТРИЗ позволяют инструментализировать эти
глобальные законы развития в применении к частным задачам изобретательского
творчества. Поэтому теорию решения изобретательских задач иногда называют
прикладной диалектикой. Это определяет отношение обществоведов к ТРИЗ.

В статье «Пути сближения научного и технического творчества (проблема
промежуточных дисциплин)» П.С. Дышлевского, д-ра филос. наук, и Л.В. Яценко,
канд. филос. наук (в сборнике «Фундаментальные исследования и технический
прогресс», изд. «Наука», Новосибирск, 1985 г., с. 80-99), дана следующая
оценка ТРИЗ: «Образование новых концептуальных звеньев научно-технической цепи
ведется у нас в русле так называемой теории решения изобретательских задач
(ТРИЗ). Методологический анализ этих разработок должен способствовать
реализации инструментальной функции естествознания и его сближению с массовым
изобретательством... ТРИЗ формализует наиболее ответственную стадию
научно-технических разработок, на которой происходит диалектическое
взаимодействие фундаментальных и прикладных исследований. Если раньше
вычленение практически полезных фрагментов естественно-научного знания
осуществлялось в каждом конкретном случае стихийно, то ТРИЗ программирует ряд
мыслительных и информационно-знаковых операций, гарантирующих внедрение науки
в конструкторскую практику».

В журнале «Вопросы философии», № 5, 1986 г. опубликованы материалы «круглого
стола», организованного редакцией журнала по проблеме «инженерная деятельность
и наука». Среди высказываний, приводимых в журнальной статье, можно отметить
слова Ф.П. Тарасенко, д.т.н, профессора, зав. кафедрой теоретической
кибернетики Томского Госуниверситета (с. 83): «Блестящим примером
«бескомпьютерной кибернетизации» инженерного творчества является известный
АРИЗ -- алгоритм изобретений Г.С. Альтшуллера, представляющий собой систему
эвристик из изобретательской практики».

В газете «Восход» (г. Арсеньев, Приморского края) от 9 января 1988 г. помещена
статья «Научить творчеству» -- о скорейшем внедрении ТРИЗ в городскую систему
образования и производства. Любопытен комментарий секретаря горкома КПСС
В.Г. Беспалова: «Мы намерены заниматься пропагандой и внедрением обучения
ТРИЗ. В этих целях проведены специальные занятия в школах
партийно-хозяйственного актива и резерва при горкоме КПСС... Больше того,
сейчас разрабатывается комплексная программа «Алгоритм» по созданию в
Арсеньеве системы непрерывного управленческого образования и компьютерного
всеобуча на 1988-90 и до 2000 года, в которых ТРИЗ отводится достойное
место... В передовые центры ТРИЗ в стране будут направлены в ближайшее время
инженеры ведущих предприятий и учителя школ города для стажировки. Уже сейчас
размножаются пособия и материалы по ТРИЗ...»

\section{Признание ТРИЗ}

\subsection{ТРИЗ глазами ученых}

Приведем несколько мнений.

Из статьи «Как учить творчеству?» А. Дюнина, заслуженного деятеля науки и
техники РСФСР, профессора (в газете Новосибирского института инженеров
железнодорожного транспорта «Кадры -- транспорту» от 12 июля 1986 г.):
«Основные идеи ТРИЗ -- стратегия выхода на задачу, строгий анализ задачи,
выявление противоречий, мешающих ее решению, поиск путей их снятия и выход на
сильное решение. Эти идеи плодотворны не только в чисто изобретательской, но и
в широкой научно-исследовательской практике инженеров... Необходимость
обучения студентов методологии научно-технического творчества становится
неотложной».

Из письма В.И. Тихомирова, заслуженного деятеля науки и техники РСФСР, д-ра
экон. наук, профессора (Москва, октябрь 1985 г.): «...ТРИЗ вошла в учебный
план ВТУЗов и одобрена НТ советом Минвуза СССР. Мы, в частности, излагаем ее в
МАИ на всех технических факультетах и пропагандируем ее».

В 1985 году в издательстве «Машиностроение» под редакцией В.И. Тихомирова
выпущено учебное пособие для студентов авиационных специальностей --
«Организация, планирование и управление авиационными научно-производственными
организациями». Пособие содержит (с. 47-55) раздел «Методы организации
творческого поиска», написанный проф. В.И. Тихомировым. Несколько строчек
отведено упоминанию о не-ТРИЗных методах; фактически весь раздел посвящен
ТРИЗ.

Из письма в газету «Известия» В.Н. Шмигальского, д-ра техн. наук, проф.,
зав. кафедрой Симферопольского филиала Днепропетровского
инженерно-строительного института (Симферополь, апрель 1985 г.): «АРИЗ
направляет творческого работника к решению задачи по кратчайшему пути, резко
уменьшая количество перебираемых вариантов и нацеливая на ИКР, позволяет
находить решения задач на очень высоком уровне. Регулярное применение АРИЗ
развивает диалектическое мышление, помогает преодолевать психологические
барьеры при создании новой техники и разработке более совершенной технологии,
обогащает человека пониманием закономерностей развития технических систем.
Основные идеи ТРИЗ могут быть перенесены на другие виды творчества (искусство,
наука), поскольку они также развиваются, преодолевая противоречия».

Из обзора писем в журнале «Техника и наука» № 10, 1982 г., с. 15: «В
Курганском НИИ экспериментальной и клинической ортопедии и травматологии
закончена научно-исследовательская работа «Выявление перспективных направлений
развития и разработки новых технических средств остеосинтеза». Руководил
работой лауреат Ленинской премии Герой социалистического труда
проф. Г.А. Илизаров. В качестве основного методологического инструмента был
использован аппарат ТРИЗ. Это позволило четко выявить перспективное
направление развития технических задач, часть из которых защищена авторскими
свидетельствами на изобретения. По нашему мнению, ТРИЗ является наиболее
перспективной основой для прогнозирования развития технических систем...»

Из заметки «ТРИЗ в Уфимском авиационном» Ю. Гусева, д-ра техн. наук,
заведующего кафедрой промэлектроники, и И. Алексеева, к.т.н. (в журнале
«Техника и наука» № 4, 1983 г., с. 14): «Освоив основы ТРИЗ, студенты, как мы
убедились, начали более осмысленно оценивать закономерности развития техники,
приобрели начальные навыки научно обоснованного решения задач на основе
выявления и разрешения технических противоречий... Когда группа студентов,
прослушавшая 20-ти часовой курс «Основы ТРИЗ», находилась на
конструкторско-технологической практике, каждый внес как минимум одно
рационализаторское предложение. Сейчас у нас нет никаких сомнений в
целесообразности и несомненной пользе обучения студентов основам ТРИЗ».

Из интервью с А.П. Достанко, член-корр. АН БССР, лауреатом премии Минвуза СССР
и Государственной премии БССР, заслуженным изобретателем СССР, проректором
Минского радиотехнического института (статья «Кто, как не автор?» в газете
«Советская Белоруссия» от 24 октября 1987г.): «Творчеству, как это ни
покажется парадоксальным, можно учить. С 1976 г. в нашем институте работает
школа молодого изобретателя, в которой изучается теория решения
изобретательских задач (ТРИЗ). Второй год в Минске под эгидой горкома
комсомола действует молодежная изобретательская школа, занятия в которой ведут
наши выпускники. Опыт показывает, что при достаточном терпении научиться
изобретательству может каждый молодой специалист, студент и даже
школьник. Сейчас более 200 студентов нашего института являются
изобретателями».

В журнале «Вопросы изобретательства» № 11, 1987 г., с. 28-29 А.П. Достанко
привел любопытные цифры: «Десятилетний опыт школы молодого изобретателя при
институте показывает, что примерно лишь 10\% студентов считают себя способными
к созданию нового. Однако после освоения курса теории решения изобретательских
задач каждый слушатель школы подготовлен к сложным проблемам».

В издательстве «Экономика» в 1985 г. вышло второе издание двухтомного
«Справочного пособия директору производственного объединения, предприятия» под
редакцией д-ра экон. наук Д.А. Егиазаряна и д-ра экон. наук А.Д. Шеремета. В
первом томе раздел 13 -- об изобретательстве. Подраздел 13.4.3 -- «технология
изобретательства» -- посвящен ТРИЗ (с. 530-533). Авторы этого раздела
д.т.н. А.В. Проскуряков и д. экон. наук Н.К. Моисеева, кратко упомянув о
зарубежных методах (мозговой штурм и пр.) и отметив, что эти методы не всегда
эффективны при решении задач, пишут: «В СССР разработано и используется другое
методическое направление». Далее изложены основные принципы ТРИЗ. Отмечена
практическая эффективность теории. Заключительный абзац: «ТРИЗ создает основу
для перехода к подлинно коллективному творчеству, способствует повышению
уровня организации творческой деятельности и может рассматриваться как один из
эффективных методов изобретательства» (с. 531).

\subsection{Международное признание ТРИЗ}

В статье «Без учета уроков прошлого» В. Манихина (»Книжное обозрение», 27
марта 1987 г.) рассказывается о ходе подготовки к Московской международной
книжной выставке-ярмарке ММКВЯ-87. В частности, говорится о книгах «Бакинского
автора Г. Альтшуллера по изобретательству, чьими произведениями
заинтересовались издатели из социалистических стран». Действительно, книги и
статьи по ТРИЗ неоднократно издавались в ГДР, Польше, Болгарии, ЧССР,
Вьетнаме, Венгрии. Например, в ГДР только за последнее время вышли переводы
двух книг: «Творчество как точная наука» и «Крылья для Икара». В органе ЦК
СЕПГ журнале «Единство» № 2, 1985 г. -- рецензии на эти книги. В 1986 г. в ГДР
опубликовано второе издание книги «Творчество как точная наука».

Не обошли вниманием ТРИЗ и издатели из капиталистических стран: США, Англии,
Франции, Японии, Швейцарии, Финляндии. Книга «Творчество как точная наука»
издана международным издательством «ГОРДОН ЭНД БРИЧ» на английском языке в
серии по кибернетике (1984 г.). Всесоюзное агентство по охране авторских прав
(ВААП) ведет переговоры с книгоиздателями из ФРГ о переводе на немецкий язык
трех книг по ТРИЗ.

За рубежом не только переводят литературу по ТРИЗ, но и ведут обучение. Так,
например, в Болгарии в 1984 году через курсы ТРИЗ прошло 2000 человек. А в
1985 -- уже 6000. В ГДР ТРИЗ используют в группах ФСА. С 1986 года начато
систематическое обучение ТРИЗ на различных фирмах Финляндии. С 1987 года
начались занятия на курсах ТРИЗ во Вьетнаме. Вот выдержка из письма одного из
вьетнамских специалистов по ТРИЗ Зыонг Суан Бао (г. Ханой, август 1987 г.): «В
течение трех месяцев (апрель, май, июнь) мы с Чаном занимались преподаванием
АРИЗ и ТРИЗ на одном курсе, открытом «Управлением по делам изобретений
Социалистической республики Вьетнам». Это был первый курс по ТРИЗ в Ханое».

\subsection{Информация к размышлению (II)}

У читателя этой справки может сложиться неверное представление о спокойном,
беспрепятственном развитии ТРИЗ. Может показаться, что внедрение идет полным
ходом. На самом деле положение достаточно тревожное. Вот характерный пример: в
СССР систематическое обучение начато в середине 60-х годов, в Болгарии -- с
начала 80-х. Обладая почти двадцатилетним заделом, мы до сегодняшнего времени
не имеем постоянно действующего государственного центра обучения и
исследований по ТРИЗ. В Болгарии такой центр существует уже несколько лет!

Другой пример. В журнале ГДР «Военная техника» №6 1984 (журнал выходит раз в
два месяца) было объявлено о публикации в 1985 году серии статей по ТРИЗ.
Статья заканчивалась так: «Редакция убеждена, что систематическая
популяризация постоянно развивающихся идей Г.С. Альтшуллера будет
способствовать новаторской и изобретательской деятельности в вооруженных
органах». Весь 1985 год шла серия публикаций по ТРИЗ. А в Советском Союзе до
сих пор нет журнала с постоянной рубрикой по ТРИЗ...

Разумеется, это журнал союзнической армии. Но интерес к ТРИЗ, как уже
отмечалось, проявляется не только в социалистических странах. Вот что пишет
специалист по ТРИЗ из Финляндии Калеви Рантенен (г. Турку, апрель 1987 г.):
«Мы начали новый курс в фирме «ПАРТЕК». Осенью начинаем большой курс в
химической компании «КЕМИРА». Кроме того, начинается большой курс в центре
повышения квалификации инженеров и в государственном центре технических
исследований... Хочу еще сообщить, что я получил из Дании письмо от
преподавателя Датского технического высшего учебного заведения. Он занимается
методами разработки новой продукции и хочет наладить контакты с коллегами,
занимающимися ТРИЗ в СССР». По дополнительным сведениям, в процесс обучения
своих специалистов ТРИЗ подключилось еще несколько финских компаний: «ФАЦЕР»,
«НОКИА», «ЭЛЕКТРОЛЮКС».

Конечно, Финляндия и Дания -- маленькие и неагрессивные страны, но вот
экономический журнал на английском языке (28-я страница журнала за
январь-февраль 1987 г.): короткая заметка, пересказывающая большую статью о
ТРИЗ в специальном журнале «Технология». В заметке говорится, что 70 инженеров
и групп людей разных специальностей... используют этот метод для решения
нетривиальных задач по развитию систем. На военных и авиационных предприятиях,
«пользуясь методом проф. Альтшуллера, они точно определяют общие для
существующих двигателей недостатки. Применяя алгоритмы, созданные
проф. Альтшуллером, они затем проводят анализ этих недостатков и применяют
соответствующие законы развития систем для их исправления».

С 1985 года в нашей стране начато государственное (а не общественное, как
раньше) обучение методам технического творчества. В программе, разработанной
Госкомизобретений СССР и ЦС ВОИР, ТРИЗ отведено всего несколько учебных часов.
Вместо глубокого изучения теории решения изобретательских задач слушателей
бегло знакомят с осколками одной из старых модификаций АРИЗ, преподают люди,
не прошедшие качественного обучения ТРИЗ, не имеющие даже представления о
современном состоянии теории и, естественно, не использующие новых разработок.

Между тем хорошо поставленное преподавание ТРИЗ могло бы дать стране многое, в
том числе -- валюту. Вот строки из письма консультанта фирмы «КОДАК» Кеннета
Лампорта, прочитавшего книгу «Творчество как точная наука» (Нью-Йорк, декабрь
1987 г.): «...Есть ли печатные материалы на эту тему, переведенные на
английский? Как я могу получить их? Я также хотел бы узнать, ведется ли в
Советском Союзе или где-либо еще обучение ТРИЗ на английском языке, доступное
жителю США, если да, то было бы очень полезно получить информацию о курсах:
сроки, место, стоимость, обязательные формальности и т.д...»

Разумеется, возникает спасительная мысль: засекретить ТРИЗ. Но ТРИЗ -- наука.
Можно на какое-то время засекретить отдельные разработки, всю ТРИЗ засекретить
невозможно, как нельзя засекретить физику, химию, генетику, кибернетику.
Напротив, нормальное развитие науки требует международных контактов,
международного научного обмена. За рубежом систематически проводятся научные
конференции, семинары по проблемам методологии технического творчества -- ни
разу еще ни один специалист по ТРИЗ не участвовал в этих мероприятиях.

Новое мышление заключается в том, чтобы жить и работать в открытом, общем
мире. А для этого необходимо научиться работать, не халтуря, научиться
выдерживать любую конкуренцию и всегда быть впереди.

\subsection{Информация к размышлению (III)}

В статье «Механизмы перестройки изобретательского дела» В.Г. Лебедева, д-ра
экон. наук, проф. Академии общественных наук при ЦК КПСС, и Г.К. Леванова
(»Вопросы изобретательства» №12-1987 г., с. 3) отмечено: «Эра поиска
рациональных технических решений методом проб и ошибок себя исчерпала, нужны
методы и методики оптимизации выбора окончательных вариантов технических
решений. В нашей стране уже для этого немало сделано, причем на уровне,
опережающем зарубежные страны. Стоит хотя бы напомнить о вкладе в разработку
теории изобретательства таких авторов, как Г.С. Альтшуллер... Однако ни
алгоритм решения изобретательских задач (АРИЗ), ни теория решения
изобретательских задач, разработанные Г.С. Альтшуллером... пока еще не
получили заметного, а тем более массового использования, несмотря на все их
достоинства и преимущества».

Еще одна выдержка -- из книги В.А. Сидорова «Прорыв в будущее» (изд. «Молодая
гвардия», М, с. 209-210): «Созданная в нашей стране ТРИЗ -- теория решения
изобретательских задач -- общепризнана уникальной, наиболее эффективной из
существующих в мире. Как и многое новое, методология изобретательства
рождалась в схватках с консервативными взглядами. Конечно, ушли в прошлое
дискуссии о том, что склонность к изобретательству -- это природный дар,
«божья искра», удел немногих. Но и дальше отрицания этой теории во многих
местах не ушли».

Год за годом, преодолевая недоверие скептиков, организовывались новые школы
ТРИЗ, готовились преподаватели, накапливался опыт обучения, составлялись
наглядные и учебные пособия. Эта колоссальная работа позволяет теперь в
течение короткого времени развернуть массовую и практически эффективную
систему обучения ТРИЗ. Вопрос стоит так: либо использовать огромный потенциал
системы обучения ТРИЗ, либо игнорировать ТРИЗ и работать предельно
неэффективным методом проб и ошибок.

Книги и статьи по ТРИЗ мгновенно и регулярно переводятся за рубежом. Пройдет
5-7 лет, и ТРИЗ получит широкое распространение во многих дружественных и не
очень дружественных странах. Необходимо не потерять время! Не утратить
авангардное положение, которое занимает пока Советский Союз в этой области.

Новая технология творчества -- самая важная из технологий, национальное
богатство страны.

\section{Система обучения ТРИЗ}

\subsection{Как обучают ТРИЗ?}

ТРИЗ -- новая отрасль знания, быстро формирующаяся в отдельную науку. У ТРИЗ
своя область изучения (законы развития технических систем, законы развития
творческой личности), свой метод (анализ больших массивов патентной,
историко-технической и историко-биографической информации), свой язык
(вепольный анализ: технические «реакции» можно записывать так, как реакции
химические), свой информационный фонд (принципы, методы и приемы разрешения
противоречий, указатели применения эффектов).

По идее, обучать ТРИЗ надо с детства. С 1976 года в «Пионерской правде» начат
эксперимент: отрабатывается методика обучения детей. Оформлено это в виде
страницы «Изобретать? Это так просто! Это так сложно!» Вышло более 50-ти
страниц. Первые давали мало откликов -- 300-500 писем. Теперь на каждую
страницу приходит около 10.000 писем. Постепенно мы учимся работать с детьми:
излагаем «кусочки» теории, подбираем интересные задачи и т.д. По материалам
«Пионерской правды» написана книга: Г. Альтов «И тут появился изобретатель».

Аналогичная работа ведется сейчас на страницах кишиневских газет «Юный
ленинец» и «Молодежь Молдавии» преподавателями и разработчиками ТРИЗ
Б. Злотиным и А. Зусман. По материалам этих выпусков и очных занятий со
школьниками ими подготовлена книга «Месяц под звездами фантазии».

Подобные материалы публикуются и в ленинградской газете «Ленинские искры»
преподавателем и разработчиком ТРИЗ И. Викентьевым. Им вместе с соавтором
составлено пособие для «детских» преподавателей ТРИЗ.

Успешно идут занятия с детьми и в других городах: Риге, Норильске, Одессе и
др.

С 1988 года начата работа по ТРИЗ со старшеклассниками и учащимися ПТУ на
страницах журнала «Парус».

Занятия по ТРИЗ идут и в ряде вузов.

И все же основной контингент наших слушателей -- инженеры от 30 до 50 лет. Для
них разработана и испытана гамма учебных программ:
\begin{enumerate}
\item До 40 учебных часов. Такой краткий курс имеет
  ознакомительно-информационный характер. Цель курса -- показать круг
  вопросов, которыми занимается современная ТРИЗ, и привлечь слушателей к
  дальнейшей систематической серьезной учебе. Занятия обычно проходят с
  частичным отрывом от работы по 4-8 часов в неделю. Практикуются и недельные
  семинары с полным отрывом от работы. Организаторам семинара необходимо
  размножить до 100 страниц раздаточных учебных материалов (здесь и далее --
  из расчета на каждого слушателя). Желательно снабдить каждого слушателя
  одной книгой по ТРИЗ.
\item 60-90 учебных часов. Полгода при занятиях раз в неделю или двухнедельный
  семинар с отрывом от работы. Цель -- освоение главных рабочих инструментов
  ТРИЗ. Введение в ТРТЛ. Раздаточные материалы -- объемом до 200
  страниц. Кроме того -- 1-2 книги по ТРИЗ.
\item 120-150 учебных часов. Год при занятиях раз в неделю или месячный
  семинар с отрывом от работы. В последнее время такие программы осуществляют
  в два этапа: организуют двухнедельный семинар, преподаватели оставляют
  слушателям задание на период самоподготовки, а через несколько месяцев с той
  же аудиторией проводят семинар второго цикла. Завершаются занятия выпускной
  работой -- решение производственной задачи. Цель -- прочное освоение
  основных рабочих инструментов ТРИЗ и решение с их помощью хотя бы одной
  практической задачи (с последующим оформлением заявки). Выработка навыков
  творческого мышления, начальная подготовка к преподавательской деятельности
  в ТРИЗ, практика применения элементов ТРТЛ. К занятиям необходимо размножить
  раздаточные материалы объемом до 400 страниц. Плюс две-три книги по ТРИЗ.
\item 220-280 учебных часов. Два года занятий раз в неделю или два месячных
  семинара с отрывом от производства. Цель углубленное освоение современной
  ТРИЗ и решение нескольких производственных задач (с обязательным оформлением
  заявок), приобретение навыков системного творческого мышления, подготовка
  преподавателей и разработчиков ТРИЗ и ТРТЛ, минимальная преподавательская и
  исследовательская практика по ТРИЗ и ТРТЛ. Объем раздаточных материалов --
  около 500 машинописных страниц. Плюс три-четыре книги по ТРИЗ.
\end{enumerate}
Какова отдача от обучения ТРИЗ?

Если вынести за скобки повышение интереса к жизни, расцвет творческой
активности, «омоложение» человека и подсчитать только экономическую
эффективность, то получится примерно следующее. Опыт свидетельствует, что
можно гарантировать -- при обучении группы в 30 человек по программе объемом
150 часов -- как минимум:
\begin{itemize}[noitemsep]
\item Сразу по окончании обучения: 20 технических решений на уровне
  предполагаемых изобретений.
\item Через год после обучения: 15 заявок на изобретения.
\item Через два года: 30 заявок, 5 авт. свидетельств, одно внедренное
  изобретение.
\item Через три года: 40 заявок, 12 авт. свид., 3 внедренных изобретения.
\end{itemize}
Подчеркнем еще раз: подсчет эффективности ведется по гарантированному минимуму.

В стране создана развитая единая общественная система обучения и разработки
ТРИЗ и ТРТЛ: курсы, семинары, народные университеты, общественные институты,
заводские, городские и областные школы. Намечается переход этой общественной
структуры в государственную, причем на основе полного хозрасчета,
самоокупаемости. На таких принципах уже работают школы в ряде городов страны:
в Ангарске, Владивостоке, Кишиневе и др.

\subsection{Где получить консультацию?}

По какой бы программе ни велись занятия, необходимо -- прежде всего --
обеспечить качественный уровень семинара. Сегодня преподавание методологии
технического творчества становится материально прибыльным. Это привлекает к
преподаванию немало случайных для ТРИЗ людей. Спекулируя на возросшем интересе
к техническому творчеству, эти люди лишь создают у слушателей искаженное
представление о предмете, фальсифицируя учебу. Понятно, что ТРИЗ не несет
ответственности за результаты обучения, если преподавателями оказываются
случайные люди.

Во избежание некачественного обучения рекомендуем обращаться в
консультационные центры ТРИЗ по следующим адресам: (Поскольку адреса,
приведенные в справке, в настоящее время не актуальны, этот фрагмент текста
пропущен. -- Прим. администратора сайта.)

Все вопросы, связанные с семинарами, надо решать заранее. С просьбой о
проведении занятий необходимо обращаться не позднее, чем за полгода до
предполагаемого срока. Организаторам приходится вести большую подготовительную
работу. У опытных преподавателей, как правило, программа семинаров расписана
на полгода-год вперед.

\subsection{Литература}

Диалектический материализм -- основа ТРИЗ
\begin{itemize}
\item[1.] Материалистическая диалектика. В 5-ти томах. Под общей редакцией
  Ф.В. Константинова и В.Г. Махарова. Изд. «Мысль», т. 1 (1981), т. 3 (1983)
\item[2.] В.И. Ленин. Об изобретательстве и внедрении научно-технических
  достижений в производство. Политиздат, 1973.
\item[3.] Институт истории, филологии и философии СО АН СССР. Философское
  отделение СССР. Методология и методы технического творчества. Тезисы
  докладов и сообщений к научно-практической конференции 30 июня -- 2 июля
  1984 года. Изд. СО АН СССР. Новосибирск, 1984 г.
\item[4.] П.С. Дышлевый, Л.В. Яценко. Пути сближения научного и технического
  творчества. -- В сборнике «Фундаментальные исследования и технический
  прогресс», Сиб. отд. изд. «Наука», Новосибирск, 1985 г., с. 93-98.
\end{itemize}
К истории развития ТРИЗ
\begin{itemize}
\item[5.] Журнал «Парус» № 1 -- 1988, с. 16-21 (В. Цуриков. Дорога во
  вселенную идей. А. Росин. Как изобрести... себя).
\item[6.] Д. Биленкин. Путь «через невозможное», Тамбовское книжное изд, 1964.
\item[7.] Г.С. Альтшуллер, Р.Б. Шапиро. О психологии изобретательского
  творчества. «Вопросы психологии» № 6 -- 1956 г., с. 37-49.
\item[8.] Г.С. Альтшуллер. Как научиться изобретать. Тамбовское книжное изд,
  1961 г.
\item[9.] Г.С. Альтшуллер. Основы изобретательства. Центрально-Черноземное
  книжное изд, 1964 г.
\item[10.] Г.С. Альтшуллер. Алгоритм изобретений. Изд. «Московский рабочий».
  1-е издание -- 1969 г., 2-е издание -- 1973 г.
\end{itemize}
Основы современной ТРИЗ
\begin{itemize}
\item[11.] Г.С. Альтшуллер. Творчество как точная наука. Изд. «Советское
  радио», Москва, 1979 г.
\item[12.] Г.С. Альтшуллер, А.Б. Селюцкий. Крылья для Икара. Изд. «Карелия»,
  Петрозаводск, 1980 г.
\item[13.] С 1979 по 1983 г.г. материалы по ТРИЗ регулярно публиковались в
  журнале «Техника и наука»:
  \begin{itemize}[noitemsep]
  \item[а)] Изложение и обсуждение основ ТРИЗ: №№ 3-6 и 9-10 за 1979 г., №№ 10
    и 12 за 1980 г.
  \item[б)] Развитие фантазии при обучении ТРИЗ: №№ 5-7 за 1980 г.
  \item[в)] Примеры использования ТРИЗ при решении конкретных задач: № 10 за
    1979 г.; №№ 4 и 9 за 1980 г.; №№ 2 и 10 за 1980 г.
  \item[г)] «Практикум по ТРИЗ»: начиная с № 1 за 1980 г.
  \item[д)] Фрагменты указателя применения физэффектов: №№ 1-9 за 1981 г.; №№
    3-5 за 1982 г.
  \item[е)] Применение химэффектов: № 6 за 1982 г.
  \item[ж)] Применение геомэффектов: № 7 за 1982 г.
  \end{itemize}
\item[14.] Г.С. Альтшуллер. Найти идею. Сиб. отд. изд. «Наука», Новосибирск,
  1986 г.
\item[15.] Сборник «Дерзкие формулы творчества», изд. «Карелия», Петрозаводск,
  1987 г.
\item[16.] Сборник «Нить в лабиринте». Изд. «Карелия», Петрозаводск, 1988 г.
\item[17.] Г.И. Иванов...И начинайте изобретать! Восточно-Сибирское книжное
  изд., Иркутск, 1987 г.
\item[18.] Н.Т. Петрович, В.М. Цуриков. Путь к изобретению. Изд. «Молодая
  гвардия», Москва, 1986 г.
\end{itemize}
ТРИЗ и ФСА
\begin{itemize}
\item[19.] Г.С. Альтшуллер, Б.Л. Злотин, В.И. Филатов. Профессия -- поиск
  нового.  Изд. «Картя Молдовеняскэ», Кишинев, 1985 г.
\item[20.] В.М. Герасимов, С.С. Литвин. Учет закономерностей развития техники
  при проведении функционально-стоимостного анализа технологических процессов.
  -- В сборнике «Практика проведения функционально-стоимостного анализа в
  электротехнической промышленности». Энергоатомиздат, Москва, 1987 г.,
  с. 193-210.
\end{itemize}
Приложение ТРИЗ к решению правовых и научных задач
\begin{itemize}
\item[21.] Р.Б. Шапиро, Г.С. Альтшуллер. О некоторых вопросах советского
  изобретательского права. «Советское государство и право», № 2, 1958 г.,
  с. 35-44.
\item[22.] Г. Альтов, В. Журавлева. Путешествие к эпицентру полемики.
  «Звезда», № 2 -- 1964 г., с. 130-138.
\item[23.] И.М. Кондраков. Алгоритм открытий? -- «Техника и наука», № 11 --
  1979 г.
\item[24.] Г.Г. Головченко. О ветроэнергетике растений. «Физиология растений»,
  1974 г., т. 21, вып. 4. c. 861-863.
\item[25.] В.В. Митрофанов, В.И. Соколов. О природе эффекта Рассела. «Физика
  твердого тела», 1974 г., т. 16, № 8, с. 24-35.
\end{itemize}
Развитие творческих способностей учащихся
\begin{itemize}
\item[26.] Г. Альтов. И тут появился изобретатель. Изд. «Детская литература».
  М, 1984. Второе издание -- 1987.
\item[27.] Б. Злотин, А. Зусман. Месяц под звездами фантазии. Изд. «Лумина»,
  Кишинев, 1988 г.
\end{itemize}
Развитие творческого воображения
\begin{itemize}
\item[28.] Г. Альтов. Фантастика и читатель. Сборник «Проблемы социологии
  печати».  Выпуск 2. 1970 г., Сиб. отдел. изд. «Наука» Новосибирск.
\item[29.] Б. Шевченко. Развитие творческого воображения. Методическое
  руководство. Фрунзенский политехнический институт. 1987 г.
\item[30.] П. Амнуэль. Звездные корабли воображения. Изд. «Знание», М, 1988 г.
\end{itemize}

\subsection{Фонд материалов по ТРИЗ}

В 1987 году Г. Альтшуллером и Л. Кожевниковой в Челябинской областной
универсальной научной библиотеке (ЧОУНБ) основан фонд материалов по ТРИЗ.
Создание фонда вызвано несколькими причинами. Во-первых, начинается массовое
применение ТРИЗ. В стране действует около трехсот школ ТРИЗ (курсов,
семинаров, институтов, групп по изучению ТРИЗ и т.д.). Число новых школ и
преподавателей быстро растет. Конечно, в своих местных библиотеках
преподаватели смогут проработать несколько книг по теории решения
изобретательских задач, но современные знания не исчерпываются этим. Есть ряд
материалов, неизданных большими тиражами: методические пособия, программы
занятий, задачники, сводные картотеки технических решений, картотеки
биографий, исследования по законам развития технических систем и их
механизмам, фотографии с занятий, учебные и информационные плакаты и т.п.
Квалифицированному преподавателю необходимо знать и использовать эти
материалы, собранные в ЧОУНБ, они становятся общедоступными.

Вторая важная причина -- сбор всех разработанных ранее (и создаваемых сейчас)
материалов по ТРИЗ. За долгие годы работы накопился большой массив печатных и
рукописных работ, разрозненных по школам и личным фондам преподавателей и
исследователей. Для учета и классификации материала их следовало собрать в
одном центре. Это позволяет эффективнее идти вперед и в обучении ТРИЗ, и в
развитии теории.

И третье: единый, зарегистрированный фонд необходим для сохранения
национального приоритета и гарантии авторских прав исследователей ТРИЗ.

Более подробную информацию о фонде и правилах работы с его материалами можно
узнать у Л. Кожевниковой (адрес см. в разделе 5.2).

\section{Вместо заключения: учить всех? Да!}

ТРИЗ предназначена для решения конкретных технических задач, но не менее важна
другая функция ТРИЗ: обеспечение такой технологии мышления, которую в старых
терминах мы называем талантом. Сегодня, когда накопился опыт работ многих школ
ТРИЗ, можно уверенно сказать: каждый инженер способен и должен «по науке» (по
ТРИЗ) решать задачи, которые принято считать творческими.

Реально ли каждого научить бегать со скоростью 50 км/час? Да, если «бег»
заменить ездой на автомобиле. Можно ли каждого научить работать быстрее самого
талантливого землекопа? Да, если заменить лопату экскаватором. ТРИЗ дает
инженеру экскаватор -- в этом смысл ТРИЗ, нового инструмента творческой
деятельности.

Неразумно и расточительно решать творческие задачи методом «осенения» в то
время, когда создана научная технология, как неразумно и расточительно
перетаскивать на себе грузы, когда есть поезда, автомобили, самолеты.

Сегодня против ТРИЗ возражают лишь люди, плохо знакомые с современной
ТРИЗ. Эти возражения чаще всего строятся по формуле: «Я ничего не читал про
вашу ТРИЗ, но думаю, что...» и далее следуют «мысли», суть которых сводится к
тому, что надо пользоваться здравым смыслом, смекалкой, народными поговорками,
пословицами, анекдотами... Что же касается ТРИЗ, то изучать ее не следует,
считают «мыслители». Во-первых, говорят они, потому, что ТРИЗ слишком сложна.
Следуя этой логике, не надо заниматься физикой, химией, математикой, биологией
и т.д. (не надо даже учиться разговаривать: по утверждению иностранцев,
русский язык очень сложен в изучении). Если бы человечество слушалось таких
советов, оно бы до сих пор сидело на деревьях, придерживаясь хвостами за
ветви.

Второй аргумент выглядит так: «А кто будет выполнять черновую, нетворческую
работу? Зачем миру столько людей, занимающихся творчеством? Достаточно и тех,
что талантливы от рождения».

Как многое в истории повторяется! Когда-то точно так же возражали против
обучения грамоте: считалось, что быть грамотным -- удел избранных, сегодня у
нас в стране всех учат читать и писать: разве это привело к катастрофе, к
перепроизводству грамотных? ТРИЗ -- азбука талантливого мышления, каждый
человек обязан быть творчески грамотным.

»Вы утверждаете, -- говорят скептики, -- что каждого можно научить изобретать?
Но этого не может быть! А врожденные способности? А гении: ведь они -- от
рождения гении, разве не так?»

Эта позиция обусловлена простыми причинами:
\begin{itemize}
\item[а)] С древнейших времен изменилось все, кроме технологии решения
  творческих задач. К методу проб и ошибок настолько привыкли, что само
  творчество стало отождествляться с решением задач путем перебора вариантов.
\item[б)] ТРИЗ отрицает монополию на творчество, якобы обеспечиваемую
  прирожденными способностями. А за такие монополии и привилегии (родовые,
  имущественные, расовые и т.п.) держатся крепко, отстаивая их всеми
  средствами, включая силу.
\end{itemize}
В бурях великих революций рухнули предрассудки о превосходстве родовитых
людей, об особых способностях людей богатых, но устояло и, пожалуй, даже стало
прочнее представление об исключительности людей, наделенных творческими
способностями. Уже не говорят: он лучше, потому, что богаче. Теперь говорят
иначе: он лучше, потому что у него творческие способности. Только это не
причина неравенства, а следствие. Миф, обосновывающий неравенство ссылкой на
способности, исключительно крепок, но и он неизбежно рухнет.

Люди имеют одинаковое право на счастье, а право это включает, прежде всего,
возможность творчества, развитие -- для творчества -- соответствующих
способностей.  
\end{document}
