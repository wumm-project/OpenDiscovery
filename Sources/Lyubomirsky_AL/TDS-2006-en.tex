\documentclass[a4paper,11pt]{article}
\usepackage{od}
\usepackage[russian,main=english]{babel}

\setlist{noitemsep}

\author{Alex Lubomirsky}
\title{The Law of Improving the Efficiency\\ of Matter, Energy and Information
  Flows} 
\date{TRIZ Developer Summit, 2006}

\begin{document}
\maketitle

\begin{quote}
  Translated by Hans-Gert Gräbe from the Russian Original available at\\
  \url{https://triz-summit.ru/confer/tds-2006/203452/203525/} 
\end{quote}

This paper presents the results of reworking the \emph{Law of Minimum Energy
  Conductivity} discovered by G.S. Altshuller.  The main mechanisms of the law
and its components are shown. The vast majority of the identified mechanisms
are illustrated with examples from various fields of technology.

Keywords: Law, mechanisms, flows, energy, information, conductivity.

\section*{The Formulation of the Law}

The law (\foreignlanguage{russian}{закон}) of development of technical systems
containing flows of substance, energy and information describes the fact that
in the process of development there is an increase of efficiency of use of
these flows.

\section*{Characteristic of the Law}

Historically, G.S. Altshuller firstly formulated the \emph{Law of Minimum
  Energy Conductivity}, which stated that in order to ensure the minimum
viability of a technical system (TS), all the links that transfer energy from
the energy source to the operational zone of the main function (usually
implied by the transmission and working tool) must have some minimum
conductivity. This law belonged to the so-called group of statical laws,
i.e. describing the initial conditions of the system's existence.

Then Igor Gridnev proposed the idea to extend this law to the entire lifetime
of the system. In doing so, he found that during the development of a TS, the
conductivity of its parts that carry energy flows usually increases, and he
identified the mechanisms to ensure this increase in conductivity. This law
was therefore called the \emph{Law of Increased Energy Conductivity}. Quite
quickly it was extended to matter and information flows as well, and the name
was changed again -- now it was called the \emph{Law of Increasing
  Conductivity of Substance, Energy and Information Flows}, and in abbreviated
form -- simply the \emph{Law of Increasing Conductivity of Flows}.

In parallel, an analytical tool based on this law, namely \emph{Flow
  Analysis}, was being developed. Within its framework, special types of flows
were identified -- harmful flows (performing harmful functions in the system
or supersystem) and parasitic flows (all kinds of leaks leading to losses).
Accordingly, it became clear that the development of systems goes not only by
increasing the conductivity of useful flows, but also by reducing the
conductivity of harmful and parasitic flows, for which there are special
mechanisms (often mirror-symmetric to the mechanisms of increasing the
conductivity of useful flows).

Later it was found that the efficiency of systems can be increased by
manipulations with useful flows, which are not connected with the increase of
their conductivity. Moreover, mechanisms have even been found to improve the
efficiency of the systems, leading to a reduction in the intensity of useful
flows and, consequently, reducing the necessary conductivity of the channels
for these flows.

The addition of the newly identified mechanisms forced another change in the
name of the law. Ideally, the name \emph{Flow Optimisation Law} best expresses
the essence of the matter. However, in TRIZ the term 'optimisation' has
negative connotations associated with the fact that when solving problems with
a contradiction, 'optimisation' is the search for a compromise, rightly (in
many cases) rejected as a dead end; the correct approach in this case is the
search for a solution resolving the contradiction. Therefore I had to stop at
a more cumbersome name \emph{Law of increasing efficiency of the use of
  flows}.

\section*{Mechanisms of the Law}

There are three main trends:
\begin{itemize}
\item Increase the positive effect of useful flows
  \begin{itemize}
  \item Increase the conductivity of useful flows
  \item Increase the efficiency of useful flows
  \end{itemize}
\item Reduce the negative effect of harmful flows
  \begin{itemize}
  \item Reduce the conductivity of harmful flows
  \item Reduce the damaging effect of harmful flows
  \end{itemize}
\item Reduce the negative effect of parasitic flows
  \begin{itemize}
  \item Reduce the conductivity of parasitic flows
  \item Reduce the cost of parasitic flows
  \end{itemize}
\end{itemize}
Let us consider each mechanism separately.

\section{Increase of Positive Effect of Useful Flows}

This pattern (\foreignlanguage{russian}{закономерность}) of development of a
TS consists in the fact that in the process of development the positive effect
of useful flows increases.

\subsection{Increase of conductivity of useful flows}

This pattern of development of a TS consists in the fact that in the process
of development the conductivity of useful flows increases.

\subsubsection{Reduction in the number of flow transformations}

This pattern of development of a TS consists in the transition from a flow
with many transformations to a homogeneous flow.

Usually, every transformation of a floa (transfer of substances from one state
to another, the change of types of energy, the change in the way information
is represented) is accompanied by losses and inhibition.  Consequently,
reducing the number of such transformations leads to an increase in
conductivity.  Ideally, there should be no transformations at all, and all
components of flows should immediately have the form necessary for their final
use.

An example is a diesel generator and a fuel cell:

In a diesel generator, the energy flow has the following form:

Chemical energy of the fuel $\to$ thermal energy $\to$ mechanical energy $\to$
electrical energy.

In a fuel cell, the transformation is only one:

Chemical energy of the fuel $\to$ electrical energy.

Consequently, the efficiency of a fuel cell is twice as high.

\subsubsection{Flow conversion}

This pattern of development of a TS consists in the transition from poorly
transmitable flow to well transmitable flow.

If there is considerable resistance to the flow, and the losses involved in
its conversion are relatively small, the flow is converted to the most easily
transferable form.

An example is the communication tube on a ship.

As the size of ships increased, it became increasingly difficult to use such
tubes. Eventually, there was a natural transition to the interphone -- the
sound stream was converted into an easy-to-transmit electrical signal.

\subsubsection{Reducing the length of the flow}

This pattern of development of a TS consists in the transition from a long flow
to a short one.

Usually, many types of losses and resistances to a flow are proportional to
its length. Consequently, flow length must be reduced in order to increase
conductivity. Ideally, the flow should be zero-length, i.e. its components
should appear immediately where they are used.

An example is the drill. In the past, the drill was driven by a stationary
motor through a flexible gear, or even a system of gears.

The long mechanical energy flow imposed limits on the speed of rotation of the
drill, which reduced productivity and increased suffering of the patient. In
today's systems, the source of rotation is located in the housing of the
drill, i.e. the flow length is reduced to almost zero.

As an example from another field, it is a well-known fact to all militaries
that highly stretched communications have an extremely detrimental effect on
troop supply, so reduced communications (i.e. shorter equipment flow lengths)
is a constant headache for strategists.

\subsubsection{Eliminating "grey zones"}

This pattern of development of a TS consists in the transition from a flow
that contains areas in which its behaviour cannot be predicted with sufficient
accuracy to a flow that is free of such areas.

Since flow behaviour in the grey zone cannot be predicted, the parameters of
these areas are usually chosen empirically. It is not always possible to carry
out a sufficient number of experiments and therefore these areas are usually
not sufficiently optimised, resulting in increased losses and resistance.
Consequently, the elimination of grey areas indirectly leads to an increase in
conductivity through better optimisation.

An example is fishing. A school of fish cannot be seen underwater and its
behaviour cannot be accurately predicted, so ensuring a steady flow of fish
from the sea to the fishing vessel is difficult -- causing many unnecessary
costs in unsuccessful net casts and idle crossings.

Sonar has been used to eliminate the grey zone -- it is now possible to see
whether fish are present and to target the trawl.

A grey area is also, for example, the area where the flow of advertising
information interacts with potential consumers, causing advertising costs to
be overspent and ineffective. This area can be eliminated by various means,
such as targeting ads to supposedly homogenous groups, for example, Barbie
dolls are advertised in cartoons targeted at girls, or the charms of military
service in the commercial breaks of martial arts movies.

\subsubsection{Eliminating "bottle necks"}

This pattern of development of a TS consists in the transition from a flow
containing areas of resistance much greater than the path resistance, to a
flow free of such areas.

A "bottle neck" is an area of the flow with sharply increased resistance.
Obviously, eliminating such areas greatly increases conductivity.

An example are filters that prevent allergy-causing particles from reaching
the nasal mucosa.

While good at trapping allergens, they have proven to be a serious obstacle to
air (a typical "bottle neck") -- it is so difficult to breathe through them
that such filters have not been widely used.

Therefore, in line with the trend, our company has developed anti-allergen
nasal inserts working on the cyclone principle.

In these inserts, the swirling air flow causes the particles suspended in the
air to be ejected by centrifugal force to the walls and adhere to the
non-drying glue-coated surface; the air itself flows freely, experiencing
almost no additional resistance. This eliminates the bottle neck.

\subsubsection{Increasing the conductivity of individual flow paths}

This pattern of development of a TS consists of increasing the conductivity of
individual flow links up to a physical limit for a given type of conductor.

Since the flow resistance is strongly dependent on the characteristics of the
conductors, improving them leads to an increase in conductivity. Ideally, the
characteristics should correspond to the physical limit for a given conductor
type.

There are many examples. E.g. roads -- the conductors of wheeled traffic flow
-- have evolved from unpaved country lanes to high-speed multi-lane highways.

Another example is the windings of electrical machinery. Copper is the best
conductor (apart from superconductors), so electrical conductivity cannot be
improved. On the other hand, insulating materials and coating technologies
have been developed that can increase the voltage.

\subsubsection{Increasing the characteristics of a specific flow}

This pattern of development of a TS consists in the transition from large, low
density flows to small, high density flows.

Often the resistance to a flow is independent of the specific characteristics
of the flow. Therefore, it is advantageous to reduce the volume of the flow
while increasing its density in order to increase conductivity. As a result,
more flow can be carried through the same conductor, or the cost of the
conductor can be reduced for the same flow.

An example is transporting gas from the production site to the consumers. To
increase the capacity, gas is compressed at the input of a trunk pipeline by a
compressor, so that significantly more gas flows through a given pipe
cross-section.

\subsubsection{Giving additional functions to the flow}

This pattern of development of a TS consists in the transfer of all or part of
the function of one flow to another.

If a flow additionally takes over the function of another flow, the second
flow becomes unnecessary. Therefore, the total power of the flows in the
system is reduced without affecting performance, and therefore, efficiency
increases.

An example is a carbureted internal combustion engine. It has a flow of
electricity which causes a spark which ignites the fuel-air mixture.

In the transition to the diesel engine, the 'ignite the mixture' function has
been taken over by a flow of mechanical energy that is converted into heat
when the mixture is compressed.

\subsubsection{Useful influence of flows on each other}

Flows of different nature can affect each other in such a way that the
conductivity of the system in relation to them increases.

An example is thermal extrusion. A heat flow acts beneficially on the flow of
the extruded material, increasing the plasticity of the material.

\subsubsection{Beneficial effect of the flow on the conductive path of another
  flow} 

The flow can improve the conductive characteristics of another flow, resulting
in an integral increase in the conductivity of the system.

Example: a well-known physical effect -- cooling of a conductor leads to lower
electrical resistance. Therefore, in developing the concept of hydrogen
economy, it is proposed to combine power lines with pipelines for liquid
hydrogen.  In this case the flow of negative heat from the hydrogen will at
the same time reduce the resistance of the electric cable.

\subsubsection{Using one flow as a carrier of the other}

This pattern of development of a TS consists in the transition from
independent transmission of heterogeneous flows to the carriage of one flow by
another.

Flows of different nature can be used to carry each other: a flow of matter
can carry different kinds of energy, a flow of energy can carry information,
etc.

An example is a two-stroke engine.

In it, unlike a conventional engine, where the flows of fuel and lubricating
oil are separated, oil is injected directly into petrol, i.e. there is a
transfer of one flow by the other.

Another example: In the days when computer technology used punched tape, the
problem arose, how to warn the user that the tape reel was about to run out?
The solution was simple: the last few meters of tape were dyed pink. The rest
is easy. Since the tape is pink, a new reel should be installed. From our
point of view, there is a situation, when the flow of substance (the tape),
which carries the flow of information for the machine, became the carrier of
one more flow of information for the user.

Another example: a strong smelling substance (mercaptan) was added to
household gas. The gas flow now carries with it a signal flow informing people
about a leakage.

\subsubsection{Transmission of several homogeneous flows in one channel}

This pattern of development of a TS consists in the transition from the
transmission of several homogeneous flows in independent channels to their
transmission in a single channel.

Combining several homogeneous flows in a single channel increases the integral
conductivity of the system and reduces the cost of each flow.

An example is multi-channel transmission, where multiple independent flows
of information, separated by carrier frequency, are transmitted simultaneously
over the same telephone wire or optical fibre.

Another example is the transmission of information signals over a lighting
network.

\subsubsection{Flow modification to increase conductivity}

This pattern of development of a TS consists in imparting to a flow a set of
properties which improve its transmission along a given type of path.

Sometimes it is possible to modify the flow in such a way that the resistance
to it is reduced. Such modifications include various ways of reducing the
viscosity of fluids, laminarisation/turbulisation of flows, use of
'transparency windows', etc.

An example is the transition from visual inspection (using a flow of visible
light) to X-rays, for which the human body is much more transparent (the same
flow of electromagnetic radiation, but shifted in frequency), allowed doctors
to look inside the living body without the help of a scalpel and probe.

\subsubsection{Total or partial flow out of the system}

This pattern of development of a TS involves a transition from a flow entirely
within the system to a flow entirely or partially through a path external to
the system.

In some cases it is possible to pass the flow through a supersystem or
environment. This allows the use of external paths with high conductivity, and
also reduces the system requirements and costs of the intra-system path.

An example is the use of a cable TV channel instead of a leased line for a
fast connection between a home computer and an Internet service provider.

Another example is the switch from wired telephone and telegraph to radio.
Transmission of the signal through the environment has reduced the cost of
laying and maintaining communication channels (cables, wires).

\subsection{Increasing the efficiency of useful flows}

This pattern of development of a TS is that in the process of development,
there is an increase in the efficiency of useful flows.

\subsubsection{Elimination of "stagnation zones"}

This pattern of development of a TS consists in the transition from a flow
that contains areas where some part of the flow is for a long time or
permanently stagnant to a flow that is free of such areas.

A "stagnation zone" is an area of a flow where some part of it is for a long
time or permanently stagnant. As a result, the effective flow capacity is
reduced, as if there were leaks, although technically all of it remains in the
system. Consequently, the elimination of "stagnant zones" leads to an increase
in the efficiency of useful flow by increasing the completeness of its
utilisation without increasing the overall capacity.

An example is the problem of cold-starting a car engine.

It is known that up to 70-80\% of engine wear occurs during the so-called
"cold start". The thing is that during engine start in cold weather when
lubricating oil thickens, the oil pump does not have time to deliver it to the
cylinders, and at first the friction in the cylinder-piston pair occurs
without any lubrication, which naturally leads to increased wear. As you can
see, there is a typical "stagnation zone" in the oil flow, which temporarily
occurs during a cold start. Indeed, there is technically enough oil in the
system but it is not being used for its intended purpose because it is stuck
somewhere on the way.

It's clear that one struggles with this phenomenon -- e.g. with special
additives in oil, or just by warming up the engine at idling speed. It looks
like there is no universal solution, but surely sooner or later the trend will
triumph, and it will be possible to start right from the spot without fear for
the engine.

Another example is a road junction. In order to pass one traffic flow, you
have to stop the other. Formally, there is enough space on the road, but in
fact -- behind the junction there is nothing, and in front of it there is a
congestion zone, i.e. a familiar traffic jam. In accordance with the trend,
such zones are eliminated, for example, by multi-level interchanges.

\subsubsection{Transition to impulse actions}

This pattern of development of a TS consists in the transition from constant
flow to pulsed flow (including sign-variable flow).

Often the efficiency of a flow depends mainly on its amplitude value.
Therefore, it is advantageous to switch to a pulsed flow in order to increase
efficiency. The total power of such a flow can be small, because its effective
value is small, but the efficiency is significant, because the pulse amplitude
can be very high. However, larger amplitudes can be achieved more easily in
pulsed mode by storing energy at pauses.

Example: Concrete demolition using a pulsed water jet.

\subsubsection{Use of resonance}

This pattern of development of a TS consists in the transition from a pulsed
(variable) flow with an arbitrary frequency to a flow whose frequency is equal
to the frequency of natural vibrations of the flow source, elements of its
path or the object to which the flow is directed.

In particular, the use of resonance allows selective high-intensity effects at
low total power of the flow.

In contrast to a conventional vibratory conveyor, it provides a significantly
higher output for the same amount of energy and dimensions. This is because
its moving part vibrates at its own vibration frequency, making maximum use of
the drive energy.

\subsubsection{Flow modulation}

This pattern of development of a TS consists in the transition to a flow whose
characteristics change over time in response to changes in the characteristics
of the object to which the flow is directed.

The flow is modulated so that it acts on the object only at those moments in
time when the object is most sensitive to this effect. In doing so, the
efficiency of the flow increases.

An example is such a commonly used thing as an atomic bomb.

It turns out that in order to detonate it, you need to create a certain level
of neutrons in the fissile material. A neutron gun is provided for that
purpose. But it is useless to irradiate uranium -- it is still impossible to
create a neutron flux of the density needed to initiate a reaction. Therefore,
the neutron beam is switched on precisely at the moment when all the
pre-critical parts of the charge are joined together -- precisely when they
are most sensitive to it. This is where it all happens.

\subsubsection{The use of gradients}

This pattern of development of a TS consists in transition from a flow that is
uniformly or randomly distributed in space to a flow whose characteristics are
distributed in space according to the location of the object (parts of the
object, several objects) to which the flow is directed.

Often, a high flow intensity is needed only in a certain area (operational
zone), while costs are determined by the overall intensity. Therefore, it is
advantageous to apply a flow with a gradient -- high intensity in the
operational zone and low intensity throughout the rest of the path to increase
efficiency.

Actually, all cutting and stabbing tools are based on the concentration of
force in a selected area of the workpiece -- with a relatively small overall
force, the stress at the point of contact, which has a very small area,
increases so much that it exceeds the resistance limits of the material.

Another example is glass cutting.

A worker scratches the glass in the right place and then lightly loads it.
This creates such a concentration of stress that the glass fails and breaks
off evenly along the notch.

Another example is the shaped charge.

The shaped charge concentrates most of the blast energy into a very small
area, resulting in extremely high armour piercing efficiency within a very low
overall charge.

\subsubsection{Mixing of several homogeneous flows}

This pattern of development of a TS consists in the transition from a single
strong flow to several weak flows that add up at the right place.

Several weak homogeneous flows can also be used to achieve a local
concentration of flow, which are stacked in the operational zone. For flows
having a wave nature, the phenomenon of interference can be used. Since the
gain in total power is not achieved in this way, it is usually done in cases
where several weak flows are easier to provide than one strong flow.

An example is the multi-oar boat already mentioned on another occasion. Each
rower individually cannot create a large force over a long period of time, but
all together can, simply by folding.

Another example is drying paper. Wet paper is rewound from reel to reel and
the free water is squeezed out with a special roller.

In order to reduce the viscosity of the water and therefore increase the
squeezing efficiency, the paper is heated. To do this, the support drum is
heated from the inside by steam. However, it was found that the contact time
between the paper and the hot drum surface was so short that the water did not
have time to heat up due to the high winding speed. If you increase the
temperature of the drum, the surface layers adjacent to the drum will start to
burn due to the limited thermal conductivity of the paper. In other words,
there is a situation where a strong heat flow cannot be used and a weak heat
flow is not enough. Therefore, in line with the trend, a second heat flow was
introduced -- blowing hot air onto the paper from outside.

\subsubsection{Multiple use of flow (adding a flow to itself)}

This pattern of development of a TS consists in the transition from a strong
flow to a weak flow passing repeatedly through an operational zone.

The total power of the flow can be reduced by allowing a relatively weak flow
to pass repeatedly through the operational zone. This is usually the case when
the strong flow is difficult to create or cannot be fully used in one pass and
the effect may be cumulative.

An example is the coil of an electromagnet.

The required magnetic field strength can, in principle, be obtained with just
a single coil. However, in order to do this, an enormous current would have to
be passed through the coil. Instead, a relatively weak current is used which
is passed through the coil many times, combining the magnetic fields of each
coil to make a single strong field.

\subsubsection{Using two heterogeneous flows to achieve a synergetic effect} 

This pattern of development of a TS consists in the transition from a single
strong flow to two weak heterogeneous flows, the joint use of which leads to
a synergetic effect.

Sometimes, instead of one strong flow, two weak heterogeneous flows can be
used, which have a synergetic effect. This effect consists in the fact that
the result of simultaneous impact of both flows is much greater than the sum
of the results of their separate use. Due to this, weak flows provide high
efficiency of the system at low losses.

An example is the problem of anthrax spore elimination. These spores are
extremely resistant to heat and chemical attack. However, the simultaneous
action of some chemical agents and relatively low heat reliably kills them.
This means that the synergetic effect of two simultaneous currents (heat and
chemicals) is being exploited.

\subsubsection{Pre-saturation of the operative zone with substance, energy and
  information} 

This pattern of development of a TS consists in the transition from a strong
flow to a weak one, acting on an object pre-saturated with the components of
this flow.

Ideally, there should be no flows in the system at all, because any flow leads
to losses and additional load on the system. Complete coagulation of flows can
be achieved by pre-saturating the operational zone with substance, energy, and
information of the required type and quantity. In doing so, a weak initiating
signal is often sufficient to carry out the entire process. If it is not
possible to fully saturate an operational area with everything it needs,
partial saturation may be limited. In this case, it will be possible to switch
to the use of weak flows.

An example of a pre-injection of a substance is sleeping pills.

If they are overdosed, poisoning is possible up to and including death. In
this case, inducing vomiting is often sufficient to save the person.
Therefore, in line with the trend, the following solution has been found -- a
small dose of a vomiting agent has been injected into the pills in advance.
In a normal situation, this has no effect on well-being, but in a significant
overdose, it is triggered before irreparable harm is done to the person.

Another example are the already mentioned land mines. Instead of shelling the
enemy (shelling is the organisation of the flow of certain substances through
the environment), the charges are placed in advance in places where they are
likely to appear.

An example of pre-saturation of an operational area with energy is
self-heating canned food. Now, you don't need a fire or any other external
heat source -- just press the bottom and you get a can of hot coffee.

An example of pre-saturation of an operational zone with information is the
use of code signals. If it is agreed in advance which signal means what (i.e.,
pre-introduce in the OZ the vast majority of information), then any signal
(and, in principle, even absence of a signal!) can carry an almost unlimited
amount of information. For example, submarine commanders have detailed
instructions about how to proceed if, after surfacing, they don't get a
certain signal from the base (absence of signal means base destruction --
here, the submarine will show itself in such a way that the submarine will not
only win!).

\subsubsection{Reducing the intensity of information flows by switching to
  self-regulating processes} 

This pattern of development of a TS consists in the transition from an
externally regulated system -- with a high flow of information between the
control system and the working body -- to a self-regulating system.

Often the flow of information in the system is necessary to control the
processes occurring in it. These flows can be reduced or eliminated altogether
by using self-regulating processes.

An example is a kettle with a whistle.

A whistle is an information signal for a person to drop everything and go to
perform certain actions -- take the kettle off the fire or turn it off. Then
they made a self-switching kettle, in which sits an almost completely coiled
nifty system -- a bimetallic plate. It is a sensor and actuator in one person,
using the energy of its supersystem to work. As a result, the person can go
about their business instead of running around on a whistle like a dog.

\section{Reducing the Negative Effect of Harmful Flows}

This pattern of development of a TS is that in the process of development, the
negative effect of harmful flows is reduced.

\subsection{Reduction of harmful flow conductivity}

This pattern of development of a TS consists in the fact that in the process
of development the conductivity of harmful flows decreases.

\subsubsection{Prevention of harmful flows}

Harmful flow prevention is practically reduced to the prevention or
significant reduction of losses in useful flows. The methods listed above are
used for this purpose.

\subsubsection{Absorption of harmful flows}

This pattern of development of a TS consists in the change from a strong
undesirable flow to a weak (absent) one through its partial or complete
absorption in a path.

In order to absorb the flow, the system resistance must be increased. To do
this, methods inverse to those used for useful flows are usually applied.

\subsubsection{Flow conversion}

This pattern of development of a TS consists in a transition from a harmful
flow that is well transmitted to a flow that is not well transmitted.

Example: one of the components of stealth technology is that a cloaked object
is coated with a substance that converts radio waves into heat.

From the pilot's point of view, the radio waves are a harmful flow. Clearly,
heat travels much less well in the atmosphere than radio waves, so this
conversion has a positive effect.

\subsubsection{Increase in flow length}

This pattern of development of a TS consists in a transition from a short flow
to a long flow.

Usually, many types of losses and resistance to a flow are proportional to its
length. Consequently, to increase the resistance to a harmful flow, its length
must be increased.

Indeed, in full agreement with this trend, sources of unpleasant noise and
smells, as well as harmful emissions, are, in the simplest case, simply kept
away from areas where people are constantly present.

A concrete example is labyrinth sealing.

A labyrinth increases the path length many times and therefore the resistance
to the harmful flow of contaminants from the outside as well as the flow of
oil (parasitic flow) from the inside.

\subsubsection{Introduction of bottle necks}

This pattern of development of a TS consists in a transition from a harmful
flow, free from areas where the resistance is significantly greater than the
flow resistance of the pathway, to a flow containing such areas.

A "bottle neck" is an area of a flow with sharply increased resistance.
Obviously, introducing such areas greatly reduces the conductivity with
respect to unwanted flow.

An example are sunglasses.

\subsubsection{Introduction of "stagnation zones" in the flow path}

This pattern of development of a TS consists in the transition from a harmful
flow, free from areas where some part of it is for a long time or permanently
stagnant, to a flow containing such areas.

A "stagnation zone" is an area of a flow where some part of it is for a long
time or permanently retained. As a result, the effective capacity of the flow
is reduced, although formally all of it remains in the system. Consequently,
the introduction of dead spots results in the actual absorption of harmful
flow into the pathway.

An example is a respirator.

Harmful dust is trapped in the filter (a typical dead zone), technically
staying in the system but causing no harm.

\subsubsection{Reducing the conductivity of individual flow paths}

This pattern of development of a TS consists in the reduction of the
conductivity of individual links of the harmful flow down to zero.

As the resistance to a flow is strongly dependent on the characteristics of
the conductors, their reduction leads to a reduction in conductivity. Ideally,
the conductivity should be zero. A typical example are all kinds of
insulators.

An example is a kitchen potholder.

Why do we use a potholder or simply use a cloth to grab a hot pan? By doing
so, we introduce a link with a rather low thermal conductivity into the
harmful heat flow from the pan to our hand, and the pan is now perfectly safe
to hold.

\subsubsection{Weakening the harmful flow by adding it to itself}

This pattern of development of a TS consists in the transition to a harmful
flow, which is weakened by adding it to itself.

Example: an automobile damper design in which the sound vibrations dampen
themselves.

\subsection{Reducing the damaging power of the harmful flow}

This pattern of development of a TS consists in prevention of undesirable
effects of flows on an object by changing characteristics of the harmful flow
(without changing its power) or object to be damaged.

\subsubsection{Introduction of grey zones}

This pattern of development of a TS consists in a transition from a harmful
flow, free from areas in which its behaviour cannot be predicted with
sufficient accuracy, to a flow containing such areas.

All military cloaking is based on this principle.

The interaction of a flow of bullets/munitions/bombs/rockets with an invisible
enemy cannot be accurately predicted (maybe it is not there at all), so it is
notoriously ineffective, which is what the cloaking experts are trying to
achieve.

\subsubsection{Reducing specific flow characteristics}

This pattern of development of a TS consists in a transition from a small
high-density flow to a large low-density flow.

Example: Electrical equipment intended to be used in environments where there
is a high risk of electric shock (damp, metallic constructions) is designed
for very low voltages of up to 12 volts. The currents are very high and the
cross-section of the cables must be increased, but if a person is exposed to
voltage, he or she will not suffer any damage.

\subsubsection{Avoidance of resonance}

This pattern of development of a TS consists in the transition from an
impulsive (variable) harmful flow with an arbitrary frequency to a flow the
frequency which is far from the eigenfrequency of vibration of the source of
the flow, of elements of its path or the object to which the flow is directed.

An example is the suspension of a car. A car is an oscillating system which is
forced to vibrate at a frequency depending on its speed and the nature of the
roughness of the road. The frequency of this damaging vibration flow is high
and the vehicle's eigenfrequency should therefore be kept as low as possible
by using the softest possible leaf springs.

\subsubsection{Use of gradients}

This pattern of development of a TS consists in the transition from a flow
which is uniformly or randomly distributed in space to a flow whose
characteristics are distributed in space according to the location of the
object (parts of the object, several objects) to which the flow is to be
directed.

Harmful flow shall be redistributed in such a way that in the most vulnerable
places it has a minimum intensity. The total power of the flow is not reduced,
and its harmful effect is reduced.

An example is the positioning of smokestacks in relation to residential areas,
taking into account the wind pattern so that the harmful smoke flow mainly
goes to the sparsely populated area.

Another example, high beam headlights in cars are adjusted so that the head of
the driver of an oncoming car is in the least illuminated zone.

\subsubsection{Adding up the flow with an anti-flow}

A developmental pattern in TS, where the harmful flow is reduced by adding an
anti-flow.

Sometimes it is possible to neutralise the harmful effect of a flow by
combining it with another flow having the opposite set of characteristics.

An example is active armour. A cumulative jet (harmful flow) is
neutralised/dispersed by a counter explosion (anti-flow).

\subsubsection{Modification of a flow in order to reduce its harmful effect}

This pattern of development of a TS consists in giving the flow a set of
properties which reduce its harmful effect.

It is sometimes possible to neutralise the harmful effect of a flow by
modifying it in such a way that it makes the potentially damaged object
insensitive to the flow. The flow remains, but is no longer harmful.

An example is developing a photographic film.

To prevent the film from being exposed, the darkroom is illuminated with red
light (after modification, the flow of light remains, but it is no longer
harmful to the film).

\subsubsection{Modification of a potentially damaged object in order to reduce
  the harmful effect of a flow on it} 

This pattern of development of a TS consists in imparting a set of properties
to the object that is potentially damaged by the harmful flow in order to
reduce the harmful effect.

Sometimes it is possible to neutralise the harmful effect of a flow by
modifying the potentially damaged object to make it insensitive to the flow.
In doing so, the flow remains, but ceases to be harmful.

An example is the use of stainless steel in various products operating in
corrosive environments and all kinds of protective coatings rendering them
insensitive to the damaging effects of that environment.

\subsubsection{Introduction of a second flow correcting the damage from the
  first flow} 

This pattern of development of a TS consists in a transition of the harmful
flow whose harmful effect is corrected by another flow.

If the harmful flow and the potentially damaged object cannot be modified, a
second damage-correcting flow is introduced. The harmful effect remains, but
does not produce visible results.

An example is welding with a non-consumable electrode in an inert gas.

The heat flow from the arc destroys the electrode even when refractory alloys
and various cooling methods are used. One possible solution to this problem is
to introduce small amounts of methane into the inert gas flow. Pyrolysis of
methane causes the formation of a thin film of electrically conductive soot on
the cathode. This film is constantly being destroyed by the heat of the arc,
but is just as constantly being regenerated. As a result, the lifetime of the
electrode is considerably increased.

This also includes the freezing of a film of ice on the leading edge of an
cavitation-damaged hydrofoil, and the setting of a layer of sugar sand on the
walls of the hopper, which is abraded by the same sand.

\subsubsection{Pre-saturation of the potentially damaged object with
  substance, energy and information to neutralise  the harmful flow} 

This pattern of development of a TS consists in the transition to a harmful
flow acting on an object that is pre-saturated with the constituents of a
neutralising flow.

If it is not possible to supply the neutralising flow, the potentially damaged
object shall be pre-saturated with the neutralising components of the harmful
flow.

An example is the wide use of buffer solutions in chemistry: when an excess of
$H^+$ or $OH^-$ ions (harmful flow) occurs, the equilibrium of reactions in
the solution is shifted in such a way as to neutralise this excess and restore
the pH value to the same level. As you can see, the neutralising flow is not
supplied from the outside, but is introduced into the system in advance.

This also includes antiseptics already introduced into the material of
surgical gowns during the manufacturing process.

\subsubsection{Lead the flow out of the system}

This pattern of development of a TS consists in changing from a harmful
flow occurring wholly within the system to a flow occurring wholly or partly
through a pathway external to the system.

In order to eliminate the harmful effect or simply to reduce the load on the
system, the harmful flow is led outside the system. To do this, the
conductivity is increased by all the methods used for useful flows.

Examples include cooling anything that overheats with heat dissipation to the
environment by means of various radiators, disposal of waste water, smoke and
waste, removal of swarf during machining, earthing of electrical equipment,
ensuring that the flow of electricity that has become harmful in an emergency
is discharged into the ground, etc.

\section{Reduction of the negative effects of parasitic flows}

This pattern of development of a TS is that in the process of development the
negative effect of parasitic flows is reduced.

\subsection{Reduction of the conductivity of parasitic flows}

This pattern of development of a TS consists in the fact that in the process
of development the conductivity of parasitic flows decreases.

\subsubsection{Prevention of parasitic flows}

The prevention of a parasitic flow is practically reduced to the prevention or
significant reduction of useful flow losses. To do this, the previously
mentioned methods are used.

\subsubsection{Absorption parasitic flows}

This pattern of development of a TS consists in a transition from a strong
parasitic flow to a weak (absent) parasitic flow by partially or completely
absorbing it in the path.

The same methods are used for this as for harmful flows.

\subsection{Reduction of the cost of parasitic flows}

This pattern of development of a TS is that in the process of development, the
cost of parasitic flows decreases.

\subsubsection{Reuse parasitic flows}

This pattern of development of a TS implements the approach that the process
of development leads to closed parasitic flows returning to the operational
zone for reuse.

Examples are reuse of scrap metal, plastic and paper waste, various
closed-loop technologies, energy recovery, etc.

\section*{Results, Conclusions, Next Steps}

These materials allow to look at the evolution of systems from a slightly
different perspective -- from the perspective of flow optimisation. Some
mechanisms of previously discovered laws find their explanation. For example,
the usually occurring in the first place trimming of transmissions is
explained by the need to reduce the flow length and the number of its
transformations. The analytical tool based on this law -- flow analysis -- is
successfully applied on a regular basis in consultation projects at the
company \emph{Algorithm}.

Therefore, it seems reasonable to recommend this law for general application.

Areas for further research include the following:
\begin{itemize}
\item Refine the identified mechanisms of the law.
\item Find new mechanisms.
\item Identify the sequence of applications of the mechanisms, as at the
  moment they all seem to be applied equally.
\item Find and justify the criteria for choosing one or another mechanism, as
  at present it is not at all clear whether, say, one should try to eliminate
  the "bottle neck" in the flow or simply bypass it.
\end{itemize}
\end{document}
