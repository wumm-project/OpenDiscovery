\documentclass[12pt,a4paper]{article}
\usepackage{lifis}
\usepackage[utf8]{inputenc}

\title{Beiträge zur Weiterentwicklung der TRIZ} 
\author{Dietmar Zobel, Wittenberg}
\date{19.\,01.\,2020}
\def\theauthor{Dietmar Zobel}

\newcommand{\these}[2]{\paragraph{#1:} \emph{#2}\par}

\begin{document}
\maketitle

\section*{Vorbemerkungen des Herausgebers}

G.S. Altschuller hat mit der Ausarbeitung von
TRIZ\footnote{\url{https://de.wikipedia.org/wiki/TRIZ}} als Erfindungsmethodik
den expliziten Anspruch vertreten, „Schöpfertum als exakte Wissenschaft“ (so
der Titel seines zentralen Buchs) zu entwickeln.  Der wissenschaftliche Kern
der Theorie umfasst eine Reihe grundlegender methodischer Ansätze und
Prinzipien, steht aber in engem Zusammenhang und Austausch mit praktischen
Erfahrungen, die zunächst aus einem genauen Studium eines umfangreichen
Patentfonds extrahiert wurden und sich heute (auch) aus den Erfahrungen im
praktischen Beratungseinsatz einer größer werdenden TRIZ-Community speisen.
Den Link zwischen Theorie und Praxis stellen verschiedenen Versionen von ARIZ
her -- einer prozessualen Vorgehensmethodik, mit der die Theorie in einen
Algorithmus transformiert wird, wenn man den Begriff „Algorithmus“ an dieser
Stelle nicht zu eng auslegt.

Einem solchen Herangehen einer „theoria cum praxi“ in der Tradition von
G.\,W.\,Leibniz\footnote{Vgl. etwa Eberhard Knobloch. Theoria cum praxi.
  Leibniz und die Folgen für Wissenschaft und Technik. Studia Leibnitiana,
  Bd. 19, H. 2 (1987), pp. 129--147.} ist ein Institut wie LIFIS bereits vom
Grundansatz her verpflichtet.  Im TRIZ-Umfeld möchten wir insbesondere die
ostdeutschen Erfahrungen in die umfassenderen Diskussionen einbringen, wozu
neben dem Erbe der Erfinderschulen\footnote{Siehe dazu die Aufsätze
  \emph{Hegel, Altschuller, TRIZ. Zehn Anmerkungen}, LIFIS-Online 25.09.2016,
  und \emph{Erfinderschulen – Problemlöse-Workshops. Projekt und Praxis},
  LIFIS-Online 03.07.2016, von Rainer Thiel.} auch das umfangreiche Werk von
\emph{Dietmar Zobel}\footnote{\url{http://www.dietmar-zobel.de/}.} gehört,
welches in seinen erfindungsmethodischen Büchern (Zobel 2007), (Zobel 2009),
(Zobel/Hartmann 2016), (Zobel 2018a) und (Zobel 2018b) dargestellt ist.

Die Arbeiten präsentieren den großen praktischen Erfahrungsschatz des Autors
aus seiner jahrzehntelangen eigenen Erfindertätigkeit vor allem im Umfeld der
chemischen Industrie und der Weitergabe jener Erfahrungen in Schulungen und
Unterrichtungen von Kolleginnen und Kollegen.  Die Erfahrungen erlauben es dem
Autor, sich in das kritische Verhältnis der TRIZ-Gemeinde zu ihren eigenen
Grundlagen mit eigenen Beiträgen einzubringen, wobei „Kritik“ hier als
wissenschaftlich fundierte Kritik -- ein unabdingbares Element einer
lebendigen Theorieentwicklung -- zu verstehen ist.

In dieser kleinen Note stellt der Autor die von ihm vorgeschlagenen und
vorgenommenen Erweiterungen, Modifikationen und Interpretationen der
widerspruchsorientierten Methodik nach Altschuller komprimiert vor.

\begin{flushright}
  Hans-Gert Gräbe, 10. Januar 2020
\end{flushright}

\section*{Beiträge des Autors zur Weiterentwicklung der TRIZ}

Die wesentlichen Beiträge des Autors zur Weiterentwicklung der TRIZ werden
hier in zwölf Punkten komprimiert dargestellt.  Für genauere Ausführungen zu
den einzelnen Punkten wird auf die Buchpublikationen (Zobel 2007), (Zobel
2009), (Zobel/Hartmann 2016), (Zobel 2018a) und (Zobel 2018b) verwiesen.

\these{1}{Es wird eine Hierarchie der Prinzipien zum Lösen Technischer
  Widersprüche vorgeschlagen, die I: Universalprinzipien; II: Minder
  universelle Prinzipien; III: Für bestimmte Fachgebiete nützliche
  Lösungsvorschläge unterscheidet.}

Die Widerspruchs-Matrix nach Altschuller ist – nicht nur nach eigenen
Erfahrungen, sondern z.\,B. auch nach den Untersuchungen von (Möhrle 2003) –
kaum treffsicher. Dies gilt auch für die neuen, 2003 und danach von (Mann
2004) erarbeiteten Versionen. Da aber die Prinzipien zum Lösen Technischer
Widersprüche \emph{als solche} zutreffend und hilfreich sind, halte ich eine
Hierarchie wie oben angegeben für die methodisch bessere Lösung: Zunächst
werden die Universalprinzipien (I) durchgesehen, dann erst die minder
universellen (II), und nur, falls sich bis dahin nichts gefunden hat, die der
Kategorie III.  Meist genügen die Universal-Prinzipien (I), zumal die der
Kategorien II und III, genau betrachtet, fast alle denen der Kategorie I
hierachisch unterzuordnen sind (!).

\these{2}{Es wird eine neue Sicht auf die Umkehr- und Analogieeffekte
  vorgeschlagen. Methodische Defizite sind hier auch bei
  Spitzenwissenschaftlern und berühmten Entdeckern zu finden.}

Die Physikalischen Effekte zählen in der systemschaffenden Phase zu den
wichtigsten Werkzeugen des Erfinders. Sie beschreiben die (nicht
schutzfähigen) \emph{Ursache-Wirkungs-Beziehungen} und geben hochwertige
Anregungen, wie man mit ihrer Hilfe zu (schutzfähigen)
\emph{Mittel-Zweck-Beziehungen} gelangen kann. Zwei spezielle Kategorien sind
von besonderer Bedeutung: die \emph{Umkehreffekte} und die
\emph{Analogieeffekte}.  An eindrucksvollen Beispielen lässt sich belegen,
dass anscheinend auch Spitzenwissenschaftler – hier: \emph{berühmte Entdecker}
– nicht „automatisch“ prüfen, ob es zu dem von ihnen gerade erst entdeckten
neuen Effekt einen \emph{Umkehreffekt} bzw. einen \emph{Analogieeffekt} gibt
oder geben könnte. So ist der merkwürdige Sachverhalt zu beobachten, dass die
Umkehreffekte fast im Regelfalle von anderen Physikern als den Entdeckern des
jeweiligen „Originaleffekts“ gefunden wurden, und dies zudem erst Jahre oder
Jahrzehnte später. Bei den Analogieeffekten ist es ganz ähnlich, jedoch ist
dies (im Unterschied zur Situation bei den Umkehreffekten) einigermaßen
erklärlich: Die besten Analogien finden sich meist weit außerhalb des vom
Entdecker gerade bearbeiteten – und somit auf ihn „hypnotisch“ wirkenden –
Spezialgebietes.

\these{3}{Es wurde die ursprünglich überwiegend maschinentechnische
  Beispielsammlung um Beispiele aus anderen Gebieten, speziell der Chemischen
  Technologie, erweitert.}

In den Print-Veröffentlichungen zum Thema dominieren nicht nur bei
Altschuller, sondern auch bei den aktuellen Autoren noch immer
maschinentechnische und rein physikalische Beispiele. Das wichtige Gebiet der
im Grenzbereich von Physik, Chemie, Maschinenbau und Verfahrenstechnik
liegenden \emph{Chemischen Technologie} ist dort eindeutig unterrepräsentiert.
Anhand eigener erfinderischer Erfahrungen konnten dazu inzwischen viele –
methodisch überzeugende – Beispiele geliefert werden.

\these{4}{TRIZ-Elemente finden sich auch als Elemente übergeordneten Denkens:
  in der Literatur, in Karikaturen, der Werbung und anderen nicht-technischen
  Gebieten.}

In der neuesten TRIZ-Literatur wurde und wird viel zu wenig berücksichtigt,
dass TRIZ im Grunde auf der \emph{Hegel\/}schen Dialektik beruht (\emph{These
  -- Antithese -- Synthese}).  Daraus ergibt sich, dass im Prinzip auf
\emph{allen} Gebieten, auch den eindeutig nicht-technischen, das „TRIZ-Denken“
eine erhebliche Rolle spielen müsste. An Beispielen habe ich belegt, dass dies
der Fall ist – ohne dass dies den jeweiligen Akteuren, insbesondere den
Künstlern, unbedingt bewusst ist. Kreative Lösungen, unabhängig vom
betrachteten Gebiet, sind jedoch immer dann besonders überzeugend bzw.
anregend, wenn sie einen \emph{inneren Widerspruch} und dessen
\emph{überraschende Lösung} erkennen lassen. Mir ist bekannt, dass Literatur
zur TRIZ in Werbung, Personalmanagement und einigen anderen nichttechnischen
Gebieten durchaus existiert, nur fehlt darin eben der prinzipielle Bezug zur
o.\,a. \emph{Hegel\/}schen Dialektik. Wäre dieser berücksichtigt worden, so
hätte man in diesen Publikationen auf die oft geradezu krampfhaft anmutende
„Übersetzung“ der (für die Technik formulierten) Altschuller-Prinzipien in die
jeweilige nicht-technische Fachsprache verzichten können.

\these{5}{Es wird ein neues Gesetz der Entwicklung Technischer Systeme
  vorgeschlagen: „Die Funktionssicherheit eines Systems wird primär nicht
  durch konstruktive Gesichtspunkte, sondern durch die sich aus dem
  Verfahrens-Funktions-Prinzip ergebenden Notwendigkeiten bestimmt.“}

Wenn Konstrukteure eine Aufgabe bekommen, setzten sie sich oftmals sofort an
den Computer und beginnen zu arbeiten, ohne zuvor das zu lösende Problem nach
den Regeln der TRIZ gründlich analysiert zu haben. Sie starten also mit dem
zweiten Schritt vor dem ersten. Die einmal begonnene Konstruktion übt nunmehr
eine geradezu „hypnotische“ Wirkung aus, und es wird nur noch in \emph{dieser}
Richtung weitergearbeitet, auch wenn die voreilig gewählte Art der
Konstruktion nicht optimal geeignet ist. Deshalb ist grundsätzlich
erforderlich, das Problem zunächst unter dem Aspekt der zu gewährleistenden
\emph{Funktion} zu analysieren, und dann erst zu konstruieren. Das mag banal
klingen, aber die Praxis sieht noch weit schlimmer aus: Wenn in einem Konzern
die Designer und Marketingleute mehr Macht als die Konstrukteure haben, was
oftmals der Fall ist, dann wird das so wichtige Funktionale \emph{noch
  weniger} beachtet (\emph{Tucholsky: „Keine Qualität, nur Ausstattung“}).

\these{6}{Es wird das Konzept der Denkfelder und Ideenketten vorgeschlagen als
  systematische Mehrfach-Anwendung ein und desselben Physikalischen Effektes
  für analoge Lösungen auf recht verschiedenen Gebieten. Verbindende
  Gemeinsamkeit ist die Nutzung des \emph{Von-Selbst}-Prinzips.}

Mithilfe eines bestimmten Physikalischen Effekts lassen sich ganz
unterschiedliche (besser: \emph{vermeintlich} unterschiedliche) Aufgaben
lösen.  Das ist unbestritten und gilt, obzwar \emph{expressis verbis} so nicht
formuliert, unter TRIZ-niks als Konsens. Was fehlt, sind überzeugende
Beispiele im Sinne einer „Ideenkette“, etwa so: Ich habe ein erfinderisches
Problem mithilfe eines bestimmten Physikalischen Effekts gelöst und überlege,
welche \emph{weiteren} Probleme (mit denen ich mich momentan gar nicht
befasse) sich nun mithilfe \emph{desselben} Effektes ebenfalls lösen ließen.
Überzeugende „Ideenketten“ dieser Art habe ich in der Literatur bisher nicht
gefunden. Deshalb wurde der Effekt \emph{„Saugende Wirkung einer hängenden
  bzw. langsam herabströmenden Flüssigkeitssäule“} von mir zur Demonstration
benutzt, und die Mehrfachnutzung dieses Effekts für die automatische
\emph{Filtration}, die automatische \emph{Destillation} und die automatische
\emph{Entgasung} beschrieben. Alle drei Lösungen haben sich als schutzfähig
erwiesen und konnten patentiert werden. Sie stellen zugleich einen
überzeugenden Beleg für das nach meiner Auffassung besonders wichtige (sogar
universelle) Altschuller-Prinzip Nr. 25 „Selbstbedienung“ (\emph{„Von
  Selbst“}) dar.

\these{7}{Das Prinzip „Von Selbst“ ist die Hohe Schule des Systematischen
  Erfindens.}

Viele Systeme sind hochkompliziert und müssen dies auch sein, sonst hätten wir
den heutigen Stand wohl nie erreicht. Allerdings muss \emph{per se} kein
System in \emph{allen} seinen Teilen zwingend hochkompliziert sein. Es gibt
immer auch Systemteile, die mit ganz einfachen Mitteln (oder gar nach dem
\emph{Von-Selbst}-Prinzip) funktionieren könnten, falls man sie entsprechend
auslegte. Zudem gibt es nach wie vor Systeme, die in \emph{allen} ihren Teilen
\emph{von selbst} arbeiten, wenn anstelle hoch komplizierter technischer
Mittel \emph{Naturkräfte} wie Gravitation, Magnetismus, Auftrieb, Kohäsion,
Adhäsion etc. eingesetzt werden. Im weitesten Sinne fällt die systematische
Anwendung derartiger Naturkräfte unter das besonders wichtige Universalprinzip
Nr. 25 „Selbstbedienung“ (\emph{„Von Selbst“}). Deshalb ist es nach meiner
Auffasung gerechtfertigt, dieses Prinzip methodisch herauszuheben und
gesondert – sowie mit ausführlichen, überzeugenden Beispielen belegt – zu
behandeln. Im Kapitel 6.9 von (Zobel 2018a) habe ich Einzelheiten dazu unter
der o.\,a. Überschrift erläutert.
\clearpage

\these{8}{Die Bedeutung der Weiterentwicklungen von ARIZ-77 wird überschätzt.}

Heute wird gewöhnlich ARIZ-85B bzw. ARIZ-85C eingesetzt, jedenfalls, wenn es
innerhalb einer mordernen TRIZ-Community um die Erlangung eines höheren Levels
mithilfe einer Belegarbeit geht. In den \emph{allgemein zugänglichen} Quellen
finden sich jedoch so gut wie keine ausführlichen Praxisbeispiele, welche die
nachvollziehbare Bearbeitung eines konkreten Themas betreffen. Eine positive
Ausnahme schien mir das Werk (Koltze/Souchkov 2017) zu sein, bis ich bemerkte,
dass es sich bei dem dort erläuterten Blitzableiter-Beispiel um ein altes
Original-Altschuller-Beispiel handelt, siehe (Altschuller 1984). Von Koltze
und Souchkov wird der ARIZ-85C auf das gleiche Problem angewandt, das von
Altschuller seinerzeit nach dem ARIZ-77 bereits überzeugend bearbeitet worden
war. \emph{Konkrete Unternehmensthemen betreffende Belege} habe ich zu diesem
älteren, nach meiner Auffassung noch immer sehr nützlichen ARIZ-77 ansonsten
nicht gefunden. Zwei eigene Beispiele dazu beschreiben einerseits die Lösung
eines sicherheitstechnischen Problems im Transportwesen, andererseits die
Lösung eines Verfahrensproblems in der Chemischen Technologie. Erstgenanntes
Beispiel führte zu einem Gebrauchsmuster, zweitgenanntes zu einem Patent.

\these{9}{Die Morpholgische Tabelle ist ein zweistufig anwendbares
  Universalwerkzeug und sollte in den ARIZ integriert werden.}

(Zwicky 1959) hat die Morphologie als umfassende, eigenständige Methode
entwickelt.  Heute ist es jedoch üblich geworden, nur \emph{einen} Baustein
seiner Methode, die Morphologische Matrix (\emph{Morphologische Tabelle}) für
sich allein anzuwenden.  Dies geschieht nicht nur ohne Verbindung zur
Gesamt-Morphologie, sondern auch ohne Verbindung zu anderen Methoden. Nach
meiner Auffassung wäre jedoch die Verbindung mit dem ARIZ sinnvoll, und zwar
unter Beachtung des Doppelcharakters der Tabelle. Einerseits liefert sie eine
umfassende Beschreibung des vom Erfinder bearbeiteten Systems (Morphologische
Analyse: \emph{gegebene} Varianten-Kombinationen). Andererseits liefert sie
die Möglichkeit, \emph{ungewöhnliche} Varianten-Kombinationen erkennen und
erfinderisch nutzen zu können. Deshalb habe ich vorgeschlagen, die
Morphologische Tabelle in den ARIZ einzufügen, und zwar an zwei Stellen: an
einer passenden Stelle der systemanalytischen Stufe einerseits, und als
eigenen \emph{Tool} in der systemschaffenden Stufe andererseits. Als
ausführliches Belegbeispiel habe ich eine Morphologische Tabelle samt
Interpretation zum Thema „Luftschiff“ vorgelegt.

\these{10}{Der AZK-Operator nach Altschuller hat eine systemische
  Doppelfunktion.  Dies wird an einem exotischen „Von Selbst“-Beispiel
  erläutert.}

Altschullers Operator \emph{„Abmessungen, Zeit, Kosten“} hat eine
Doppelfunktion.  Einerseits sichert er in einer frühen Phase der
Problembearbeitung ab, dass Extremfälle und Extrembereiche nicht gänzlich
unbeachtet bleiben. So wird die vorzeitige „Kanalisierung“ des Denkens
vermieden, welche eine allzu eingeschränkte Sicht auf den Wirkungsbereich der
angestrebten Erfindung zur Folge hätte (es wäre eine
\emph{„Auswahlerfindung“}).  Andererseits führt die systematische Einbeziehung
der Extrembereiche günstigen Falles zu völlig neuen Aufgaben, deren
Bearbeitung sinnvoll sein kann. Mindestens aber wird unser allgemeiner
Kenntnisstand verbessert und das Blickfeld erweitert.  Gekoppelt mit dem
Altschuller-Prinzip \emph{„Nicht vollständige Lösung“} habe ich entsprechende
Experimente durchgeführt: Anfertigung von Kopien unter Einsatz von
natürlichen, nicht speziell präparierten, faktisch kostenlosen Materialien,
wie Sperrholzverschnitt, Zeitungsrändern und Laubbaumblättern, siehe (Zobel
2018a).

\these{11}{Es wurde erstmals eine Anleitung zum Abfassen von Patentschriften
  unter konsequentem Einsatz der widerspruchsorientierten Nomenklatur
  gegeben.}

Selbst eine hoch schöpferische Lösung erreicht keinen Patentschutz, wenn der
Text der Anmeldung ungeschickt formuliert ist. An einem eigenen Beispiel aus
dem Bereich der Chemischen Technologie habe ich belegt, wie sinnvoll der
Einsatz der Widerspruchsterminologie für eine erfolgreiche Patentanmeldung
sein kann. Eine besondere Rolle spielt dabei die \emph{Darlegung des Wesens
  der Erfindung}. Zunächst ist zu erklären, welche Parameter – und warum –
einander behindern, und somit einer Problemlösung mit \emph{herkömmlichen}
Mitteln im Wege stehen. Sodann ist das daraus resultierende Hindernis auf dem
Weg zum angestrebten Ziel als anscheinend \emph{unlösbarer} Widerspruch zu
formulieren. Der Abschnitt sollte mit einem Standardsatz beendet werden. Er
lautet: \emph{„Vorliegende Erfindung löst diesen Widerspruch“}.

\these{12}{Es wurden TRIZ-basierte Fragen als Instrumente zum Bewerten
  derzeitiger Verfahren und Produkte, zur Beurteilung von Projekten sowie zum
  Bewerten neuer Lösungen entwickelt.}

Es gibt bereits zahlreiche Methoden zum Bewerten von Verfahren und Produkten.
Angewandt werden beispielsweise: Scoring-Modelle, Nutzwertanalyse, Wertanalyse
nach DIN 69910, Gemeinkostenwertanalyse, Entscheidungstabellentechnik und
Risikoanalyse. Diese Methoden beanspruchen für sich, wissenschaftlich zu
arbeiten. In der Praxis fließt jedoch, bewusst oder unbewusst, stets viel
Subjektives in die Bewertung ein. Die Methoden sind kaum geeignet, Vorhaben,
Pläne, Lösungsansätze und Projekte objektiv zu beurteilen. Im Kapitel
„TRIZ-orientiertes Bewerten ersetzt subjektive Einschätzungen“ (Zobel 2009)
habe ich deshalb speziell TRIZ-orientierte, praxistaugliche Bewertungsfragen
zwecks Verbesserung der Objektivität vorgeschlagen, und zwar für die
Kategorien:
\begin{itemize}
\item Bewertung vorhandener bzw. gegebener Produkte, Verfahren oder Systeme,
\item Beurteilung von Plänen bzw. Projekten bzw. Pflichtenheften,
\item Bewertung innovativer Lösungen bzw. neu geschaffener Systeme.
\end{itemize}
\clearpage
\section*{Literatur}

\begin{itemize}
\item Genrich S. Altschuller (1984). Erfinden – Wege zur Lösung Technischer
  Probleme. Verlag Technik, Berlin. Drei Auf"|lagen: 1984, 1986, 1998.
\item Genrich S. Altschuller, Alexander B. Seljuzki (1983). Flügel für
  Ikarus. Verlag MIR Moskau; Urania-Verlag, Leipzig 1983.
\item Karl Koltze, Valeri Souchkov (2017). Systematische Innovation.
  Hanser-Verlag Mün"|chen. 2. Auf"|lage 2017.
\item Darrell Mann (2004).  Hands on Systematic Innovation: For Business and
  Management. 
\item Martin G. Möhrle (2003).  Evaluation of inventive principles and
  contradiction matrix, or: How useful is the Altshullerian theory
  today? 3. Europäischer TRIZ-Kongress, 19.-21.03.2003, Zürich
\item Dietmar Zobel (2007). Kreatives Arbeiten. Expert-Verlag Renningen.
\item Dietmar Zobel (2009). Systematisches Erfinden. Expert-Verlag Renningen.
  5. voll"|ständig überarbeitete und erweiterte Auf"|lage 2009.
\item Dietmar Zobel, Rainer Hartmann (2016). Erfindungsmuster. Expert-Verlag
  Renningen. 2. Auf"|lage 2016.
\item Dietmar Zobel (2018a). TRIZ für alle. Expert-Verlag Renningen.
  4. überarbeitete und erweiterte Auf"|lage 2018.
\item Dietmar Zobel (2018b). Verfahrensentwicklung und Technische Sicherheit
  in der Anorganischen Phosphorchemie. Expert-Verlag Renningen.
  2. überarbeitete und wesentlich erweiterte Auf"|lage 2018.
\item Fritz Zwicky (1959). Morphologische Forschung. Winterthur, 1959.
\end{itemize}

%\ccnotiz
\end{document}
