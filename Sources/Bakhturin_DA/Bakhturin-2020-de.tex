\documentclass[11pt,a4paper]{article}
\usepackage{od}
\usepackage[utf8]{inputenc}
\usepackage[main=german,russian]{babel}

\title{No-TRIZ versions of system notions.\\[1em] \Large Abstract eines
  Vortrags auf der Moskauer TRIZ-Konferenz 2020}

\author{Bakhturin D.A.}
\date{November 2020}

\begin{document}
\maketitle
\tableofcontents

\begin{abstract}
  Im Vortrag werden einige Versionen von Konzeptualisierungen über „Systeme“
  und „technische Systeme“ mit unterschiedlichem Hintergrund der
  intellektuellen Traditionen vorgestellt und kurz charakterisiert. Ein großer
  Teil der vorgestellten Thesen basiert auf Präsentationen von Vertretern der
  erwähnten Schulen im Moskauer TRIZ-Club 2019-2020. Vollständige Versionen
  der Präsentationen (in Russisch) sind auf der Website \texttt{Metodolog.ru}
  in der Rubrik „Materialien des Moskauer TRIZ-Clubs“ verfügbar.
\end{abstract}

\section{Tätigkeits-Natur-Systeme}

Es geht um Ausarbeitungen des Moskauer Methodischen Kreises (MMK), die am
weitesten entwickelte Beschreibung ist im Beitrag von P.G.Shchedrovitsky „Das
Tätigkeits-Natur-System“\footnote{Das russische Original ist
  \foreignlanguage{russian}{Деятельностно-природная система}. Im Deutschen
  gibt es die zwei Begriffe \emph{Aktivität} und \emph{Tätigkeit}, mit denen
  \foreignlanguage{russian}{деятельностно} übersetzt werden kann.
  Möglicherweise ist das Deutsche hier ausdrucksmächtiger.  Andererseits ist
  das russische Wort hier ein Adjektiv, während mir für keines der beiden
  genannten deutschen Substantive eine sinnvoll Adjektivform geläufig ist.  An
  dieser Stelle ist also die russische Sprache ausdrucksmächtiger.  Ich habe
  versucht, je nach Kontext eine angemessene Übersetzung zu finden.  Diese ist
  nicht einheitlich!} zu finden.  Die Bedeutung dieser Version liegt darin,
dass sie auf eine ganz bestimmte -- Marxistische -- Tradition verweist, mit
der man die in der TRIZ verwendete Version tiefer verstehen kann, dass
\emph{Technik = Soziales + Natürliches (\foreignlanguage{russian}{Природное})}
ist.  Es sei darauf hingewiesen, dass in der TRIZ selbst die Idee, was
\emph{das Soziale} sei, faktisch nicht ausgearbeitet worden ist, so dass die
Ersetzung des Begriffs \emph{Soziales} durch \emph{Tätigkeit}\footnote{HGG: es
  wird an dieser Stelle das Adjektiv verwendet.}  durchaus gerechtfertigt
werden kann, besonders auch, da TRIZ und die Ausarbeitungen des MMK gemeinsame
-- marxistische -- Wurzeln haben. Vorstellungen über \emph{Tätigkeit} beziehen
sich einerseits auf Marx' 11. Feuerbach-These („Die Philosophen haben die Welt
auf verschiedene Weise interpretiert, es kommt darauf an, sie zu verändern“)
und ermöglichen es andererseits, eine ganze Reihe von Ausarbeitungen des MMK
über eine allgemeine Theorie der Tätigkeit aufzunehmen, in Fortführung und
Entwicklung einer Marxistischen Tradition. Insbesondere wurde am MMK ein
System von Vorstellungen über \emph{Tätigkeit} geschaffen und ausgearbeitet,
verkörpert im so genannten \emph{Schema der Reproduktion der Tätigkeit und
  Kulturtranslation} (\foreignlanguage{russian}{схема воспроизводства
  Деятельности и трансляции Культуры}), was darauf hinweist, dass die führende
Komponente von Tätigkeit die Norm ist, die in jeder neuen Situation
reproduziert (translatiert) wird.

\paragraph{Beispiel:}
Ist das Funktionieren eines technischen Systems eine „Aktivität“ im Sinne der
MMK-Traditionen?

\paragraph{Antwort:}
„Aktivität“ ist eine rein menschliche Tätigkeit\footnote{Hier wird in der Tat
  zweimal \foreignlanguage{russian}{деятельность} verwendet. Ich habe es
  differenziert übersetzt.}, und ist in dieser Hinsicht ein Begriff kein
Begriff im „Geist des MMK“, sondern im Geiste der Traditionen von Marx und
Engels und stellt eine gewisse Spitze in der Entwicklung sogenannter
„materialistischer Vorstellungen“ dar.

Von dieser Position aus können wir sagen, dass menschliche Tätigkeit eine
umfassende Kategorie ist. Sie expandiert, entwickelt sich, wächst -- unter
anderem auch durch die Schaffung von \emph{technischen Systemen}, die Teile
der Tätigkeit übernehmen, die bisher von einem Menschen ausgeführt wurde. Auf
dieser Grundlage können wir sagen, dass die Menschheit die Tätigkeit ausübt,
die Technischen Systeme (TS) dagegen funktioniert, indem sie bestimmte
Funktionen innerhalb dieser Tätigkeit ausüben.  Das Funktionieren TS ersetzt
Tätigkeit des Menschen.

\paragraph{Frage:}
Wie können wir das Konzept der Tätigkeit in der Analyse TS verwenden?

\paragraph{Antwort:}
Tätigkeit als umfassende Kategorie setzt einen allgemeinen, übergeordneten
Rahmen für das Funktionieren TS. Abhängig von der spezifischen Tätigkeit, in
der ein TS eingebunden ist, werden wir dessen Haupt-, Zusatz- und
Hilfsfunktionen anders definieren. Wenn wir zum Beispiel ein Mikroskop
innerhalb einer Forschungsaktivität nutzen, wird es ein funktionales Porträt
haben, und wenn wir es beim Nüsseknacken einsetzen, ein ganz anderes, obwohl
dabei der Aufbau des Mikroskops aus Komponenten unverändert bleibt.

\section{Das Schema der Gedankentätigkeit
  (\foreignlanguage{russian}{Мыследеятельность})}

Dies ist eine der finalen Entwicklungen des MMK, die in gewisser Weise die
ganze Arbeit zusammenfasst. Das Schema ist die Grundlage der sogenannten
\emph{System-Methodik des Denkens} (SMD-Methodik --
\foreignlanguage{russian}{Системо-Мыследеятельностная Методология}). Grundlage
des Schema ist die Identifizierung von drei Schichten -- Tätigkeit,
Kommunikation und Reines Denken und eine Reihe von Positionen, die mit diesen
Schichten verbunden sind. Die Bedeutung des Schemas besteht darin, dass in ihm 
\begin{itemize}
\item[а)] die führende Rolle der Kommunikationsebene zwischen den Subjekten
  hervorgehoben wird,
\item[b)] das, was allgemein als \emph{Denken} bezeichnet wird, in eine Reihe
  intellektueller Teilprozesse aufgespalten wird, wie z.B. \emph{Verstehen},
  \emph{Reflexion}, \emph{Kommunikation} usw.
\end{itemize}
Das Schema hat große Bedeutung, um Gruppenarbeitsprozesse sicherzustellen, und
auch für die Organisation der pädagogischen Praxis. Historisch gesehen
entstand das Schema in Konfrontation mit Anhängern eines rein „aktivistischen“
Ansatzes, bei es keinen Ort für das Denken gab. Es wurde eine riesige Menge
Arbeit geleistet, um die Kategorien \emph{Denken} und \emph{Tätigkeit}
abzugrenzen und zu verknüpfen (was sind letztlich im Begriff der
\empg{Denktätigkeit} manifestiert). Die scheinbare Selbstverständlichkeit
solcher die Übergänge sollte nicht irreführen. In Russland praktizierende
TRIZ-Spezialisten stoßen ständig auf dieses Problem und stellen fest, dass
unsere „moderne“ Industrie auf die Entwicklung und Umsetzung von
„Arbeitsplänen“ fixiert ist, diese auf Aktivitäten, aber Orte für Denken,
Innovation, TRIZ gibt es in diesen Plänen nicht. Solche Orte müssen
geschaffen, „erkämpft“ werden. Philosophisch gesehen heißt das, wie
G.P.Shchedrovitsky betonte, dass wir „viele denklose Aktivitäten“ ausführen.

\paragraph{Frage:}
Wie lassen sich die Ausarbeitungen des MMK zum Denken im Prozess der Analyse
TS nutzen?

\paragraph {Antwort:}.
Im Anwendungs-Plan ist Folgendes wichtig: Wir, die TRIZ-Spezialisten, erhalten
hauptsächlich Informationen (insbesondere im Anfangsstadium, bei der Analyse
der Situation) von Experten, tätigen Ingenieuren und Leitern, diese aber nicht
direkt, sondern durch Vermittlung eines „Kommunikations“-Prozesses. Das TS
selbst ist uns selten in seiner Materialität zugänglich.  Das Verständnis der
Abhängigkeit des Informationsgehalts von der Position, die ein bestimmter
Kommunikant (Experte, Leiter, Ingenieur) besetzt, hilft bei der Klärung der
lokalen Wahrhaftigkeit dieser Informationen. Zum Beispiel ist es für viele
Manager „in Position“ nicht rentabel, TRIZ einzuführen und anzuwenden,
entsprechend kann die Information, die aus einer solchen Quelle bereitgestellt
wird („aus dieser Position“), in die Irre führen, das Bild verzerren usw.
Eine Möglichkeit, aus einer solchen Situation herauszukommen, ist die
„Repositionierung“ des Kommunikanten, seine Verschiebung auf eine andere
Position, wo seine Ziele die Vermittlung einer anderen Botschaft ist, „aus
einem anderen Blickwinkel“, „in einem anderen Kontext“ usw. Ein Spezialfall
einer solchen „Neupositionierung“ ist genau das TRIZ-Training für alle Arten
von Managern und leitenden Spezialisten. Danach nehmen sie eine andere
Position ein.

\section{Geschichtliche Tradition}
Aus der \emph{Weltorganisation für Systeme und Kybernetik} trat im Moskauer
TRIZ-Club der Philosoph und Methodiker Vyatcheslav Maracha mit diesem Thema
auf. Das Wichtigste aus der Präsentation ist die von V. Maracha verteidigte
Position (Hypothese), dass der systemische Ansatz in seiner modernen Version
eine relativ neue „Erfindung“ ist, die an der Wende vom 19. zum
20. Jahrhundert entstanden ist und sich in der Mitte des 20. Jahrhunderts
ausgeformt hat. Eine besonders bedeutsame Phase war die Veröffentlichung des
Buches \emph{System Engineering} von Hood und McCullough (?).  Die
inhaltsreiche Schlüsselhypothese von V. Maracha ist, dass der Systemansatz den
so genannten \emph{gegenständlichen Ansatz}
(\foreignlanguage{russian}{«предметный подход}) ablöst.  Der „gegenständliche
Ansatz“ ging von der Bildung von Vorstellungskomplexen nach dem Vorbild der
Einzelwissenschaften -- Physik, Chemie usw. -- aus. In einem bestimmten
Entwicklungsstadium sind die „gegenständlichen“ Darstellungen über sich selbst
hinausgewachsen, als multidisziplinäres Wissen aufzutauchen begann
(Elektro-Mechanik, physikalische Chemie usw.). Der Systemansatz hat den
„gegensttändlichen“ Ansatz abgelöst, da er die disparate Konzepte der
„gegenständlichen Ansätze“ unifizieren konnte durch die Anwendung einer
gemeinsamen Ideologie, der „Systemkonzepte“. Aus der Sicht von V. Maracha hat
der systemische Ansatz in der Anfangsphase die in ihn gesetzten Erwartungen
erfüllt, aber heute haben wir auch eine große Anzahl unterschiedlicher
Versionen des Systembegriffs, Modelle, Vorstellungen.  Es vollzieht sich
letztlich dasselbe, was sich zum Ende des „gegenständlichen Ansatzes“ auch
vollzogen hat, nämlich ein Prozess der Aufweichung einheitlicher
Basisvorstellungen, das Auftreten einer großen Zahl spezieller Versionen und
Hypothesen, was die Grundlagen des „systemischen Zugangs“ selbst untergräbt.

\paragraph{Frage:}
Kann man so sagen, dass es so viele Versionen des Systemansatzes gibt, dass
der Begriff „systemischer Zugang“ zunehmen an Schäfre verliert und wir von
dort nichts mehr zu erwarten haben?

\paragraph{Antwort:}
Nun, zum einen nicht allzu viel, wie sich aus den Arbeiten von V. Maracha
selbst ergibt.  Obwohl die TRIZ meiner Meinung nach Systemizität im
Allgemeinen anders versteht, so dass es für uns keinen Sinn macht, sich völlig
in die Logik der „systemischen Bewegung“ einzupassen. Meine These ist, dass
die TRIZ diese Unzulänglichkeiten eines systemischen Ansatzes überwunden hat
und den Übergang zu post-systemischem Denken markiert wie, sich von dort
sozusagen abstößt und zur nächsten S-Form übergeht. Auf welcher Basis?
Erinnern wir uns an die Empfehlungen der Etappen 3-4 der ZRTS
(Entwicklungsgesetze Technischer Systeme) beim Übergang zu einer neuen
S-Kurve. Identifiziere die dem alten Ansatz innewohnenden Widersprüche (welche
die Entwicklung behindern), und finde neue Wege, diese zu lösen.  Die
klassische Systemtheorie versucht, die ganze Welt der Systeme als Objekte zu
erfassen, zu klassifizieren, alle Versionen und alle Verbindungen zu
beschreiben. TRIZ bewegt sich in der Logik „System = Problem/Aufgabe“, kein
„natürlicher“ Gegenstand und kein Objekt. Wir haben eine systemische
Situation, das Problem bzw. die Aufgabe ist systemisch konstruiert. Auf diese
Weise überwinden wir die Grenzen des „offiziellen“ systemischen Ansatzes, der
die ganze Vielfalt der Systembeschreibungen neu organisiert in eine
überschaubare Zahl systemisch beschriebener Probleme oder erfinderischer
Aufgaben.

\section{System Engineering (SE)}
Im Moskauer TRIZ Club berichtete V. Batovrin, einer der Leiter der SE-Bewegung
in der Russischen Föderation, Übersetzer von Schlüsselbüchern über SE ins
Russische, Gründer der ersten russischen Abteilung für SE bei MIREA. In der
heutigen Welt beansprucht das SE, eine „angewandte Systemtheorie" zu sein (so,
wie man über die TRIZ sagt, sie sei eine „angewandte Dialektik“).
G.S. Altshuller unterschied einmal \emph{Ingenieur-Probleme} und
\emph{Ingenieur-Aufgaben}. In dieser Logik kann man sagen, dass SE eine
verallgemeinerte Theorie der Ingenieurtätigkeit ist, der Praxis der Lösung von
Ingenieurproblemen. Als philosophisch-methodologische Grundlage des SE dienen
einige Versionen des „Systemansatzes“.  Zu den wesentlichen Merkmalen des SE, 
die V. Batovrina in seiner Präsentation genannt hat, gehören folgende:
\begin{itemize}
\item[a)] Fehlen eines einheitlichen Begriffs von „System“ und keine
  einheitliche Definition dessen, was SE ist, was es tut und wofür es
  verantwortlich ist.
\item[b)] In der SE-Logik ist für die Systemcharakterisierung das
  Vorhandensein eines „Synergieeffekts“ (wenn die Eigenschaft des Systems
  größer ist als die Summe der Eigenschaften seiner Teile) nicht zwingend
  erforderlich. Ein System kann einfach durch einen Satz von der beteiligten
  Komponenten definiert werden, ohne Bezug auf einen „Systemeffekt“.
\item[c)] Im SE gibt es keine Vorstellungen über Probleme und Hindernisse im
  beim Synthetisieren, Zusammenfügen von Teilen zu einem System (in dieser
  Hinsicht gibt es dort keinen Begriff von einem Ort der Erfindung, einer
  erfinderischen Situation usw.).
\end{itemize}
\paragraph{Beispiel:}
Beim SE ist das System nicht definiert, kann eine Aufzählung von nicht
miteinandere verknüpfter Elemente sein. Ist das bequemer als das, was derzeit
in der TRIZ gilt?

\paragraph{Antwort:}
Nein, das sage ich nicht. Sie haben ein System von Elementen, die auch
miteinander verbunden sind, auch wenn es sich dabei um eine Art formale und
nicht um eine materielle Verbindung handelt.  Die Hauptsache ist, dass das SE
den Begriff der Synergie nicht als die notwendige Qualität des Systems hat, in
der TRIZ wurde das von der Altschuller direkt eingeführt. Für die TRIZ weist
der Synergiefaktor TS auf eine sprunghafte, qualitative Entwicklung hin,
während im SE Entwicklung einfach die Hinzufügung eines weiteren Komponente
zum TS ist, nur als Ergebnis der Ingenieurarbeit. Mit dem Fehlen von
Synergien, imho, hängen auch die Tatsache zusammen, dass es im SE keine
Bgriffe wie Problem, Erfindung, Entwicklung gibt (es gibt keinen Platz für
sie, weil es keine Sprünge und Grenzen gibt).  Es ist definitiv „bequemer“ für
sie, da es alles auf routinemäßige Ingenieursarbeit reduziert, die leicht zu
automatisieren ist. In der TRIZ sind solche Darstellungen nicht möglich, da
damit die Idee eines qualitativen Sprungs in der Entwicklung TS und damit
Erfindertum ausgeklammert wird.

\section{Die Schule des Systemdenkens}
(Anatoly Levenchuk und Co, Bericht im Moskauer TRIZ Club). Die Schule ist eine
die Version von A. Levenchuks systematischem Ansatz, der aus dem SE und
Methoden des MMK gewachsen ist. Von der TRIZ werden regelmäßig Konzepte
übernommen, insbesondere das Konzept des \emph{Obersystems} aus dem
TRIZ-Systemoperator. Berechtigterweise wird auf den transzendentalen, nicht
offensichtlichen, nicht trivialen Charakter des Übergangs durch die
hierarchischen Ebenen „Subsystem“ -- „System“ -- „Obersystem“ hingewiesen.
(ein Umstand, dem in der TRIZ wenig Aufmerksamkeit geschenkt wird).  Das von
A. Levenchuks vorangetriebene Konzept des „System-Engineering-Denkens“
entpuppt sich bei sorgfältiger Prüfung als „systemisch konstruiertes Denken“.
Der Fokus wird von „Systemen“ als extern existierenden „Objekten“ auf die
„systemische Organisation“ von Vorstellungen über Objekte, über die Außenwelt
verschoben (siehe zum Vergleich den Altschuller-Artikel von 1975 -- \emph{Die
  Entwicklung des systemischen Denken ist das finale Ziel des Erlernens von
  ARIZ})\footnote{\url{https://www.altshuller.ru/triz/triz70.asp}}. Ähnlich
Diskussionen sind auch in der TRIZ zu sehen, zum Beispiel über den „objektive
Charakter von Widersprüchen“ („Da ist ein Schornstein und Rauch - wo ist der
Widerspruch?"  V. Mitrofanov), obwohl sich die Schule von A. Levenchuk damit
deutlich mehr beschäftigt.

\section{Zusammenfassung}
Ich denke, es ist nutzlos, wenn die TRIZ versucht, eine eigene „allgemeine
System-Theorie“ zu schaffen.  Ähnliche Prozesse in der Welt sind viel
mächtiger, haben eine lange Geschichte, Tradition, sind eingewoben in die
Kultur usw. Es ist notwendig, einen Dialog mit Vertretern anderer systemischer
Ansätze und Methoden zu organisieren -- um sicherzustellen, dass die TRIZ in
diese globalen Bewegungen eingebunden wird und dabei die Beschäftigung mit der
TRIZ ihren eigenen -- würdigen -- Platz bekommt. Versuche, innerhalb der TRIZ
eine eigene „allgemeine Theorie der Systeme“ zu entwickeln, führt nur zur
Vergrößerung von Autarkie, zu Abgeschlossenheit, zur Isolierung der TRIZ und
behindert damit deren Entwicklung.

Als ein Beispiel derartiger unfruchtbarer Aktionen betrachte ich die im Rat
für Forschung und Entwicklung aufgereufene Diskussion zum Thema „Wie viele
primär nützliche Funktionen (PNF) kann ein technisches System haben“? Als
Ergebnis der Abstimmung im Rat könnte sich durchaus ergeben, dass in der
TRIZ-Welt TS nur \emph{eine} PNF haben können. Aber es genügt, einen Blick auf
die objektive Welt um uns herum zu werfen, um zu sehen, wie schnell der Umfang
der Funktionalität wächst, die durch geschaffene TS realisiert wird
(multifunktionale Komplexe, Lastwagen mit eingebautem Kran, Computer,
Smartphone...).  Und zu Bildungszwecken führen wir eine Reihe von Beispielen
zu diesem Thema auf -- Trimmen, Aufspalten, Linien der Entwicklung TS usw.  Es
ist offensichtlich, dass in der „großen“ Welt der Ingenieure Systeme eine
Menge Hauptfunktionen haben können (und haben!). Wie diese allgemeine
Eigenschaft erhalten und „innen“ die Spezifik der TRIZ einzuschreiben --
leider kann diese Frage im „autarkistischen“ Zugang nicht einmal gestellt
werden. Appelle wie „wie wäre das bequemer“ überführen die Frage dann ganz von
der öffentlichen Agenda in die Diskussion projektinterner Spezifika -- „Ich
fühle mich wohler so, und alle haben dann mehr Respekt vor uns“. Einige der
Ideen des Autors zu diese Frager sind im Text der Präsentation enthalten.

Die Entwicklung von Technologie, Ingenieurwesen und erfinderischer Praxis ist
ein objektiver Prozess, und jede der Komponenten hat ihre eigene, einzigartige
und gleichzeitig mit anderen verbundenen Orte und Funktionen. Der bereits
erwähnte System-Operator soll uns dabei helfen.

\end{document}
