\documentclass[11pt,a4paper]{article}
\usepackage{od}
\usepackage[utf8]{inputenc}
\usepackage[russian]{babel}

\title{Сравнение Функционального анализа Продукта\\ и Функционального анализа
  Технологии,\\ некоторые выводы и рекомендации.\\[1em] \Large Тезисы к
  докладу на Московской ТРИЗ-Конференции 2020.}

\author{Бахтурин Д.А.}
\date{November 2020}

\begin{document}
\maketitle
\tableofcontents

\section{Практическая часть}
\subsection{Ситуация}
В современной ТРИЗ, в т.ч. в стандарте МАТРИЗ 1ого и 2ого уровня выделяют два
вида Функционального анализа – функциональный анализ Продукта (ФАП) входит в
состав стандарта 1ого уровня, и функциональный анализ Технологии (ФАТ),
изучается на 2м уровне по стандартам МАТРИЗ. Данные два вида работ трактуются
и в целом как разные разделы и подходы, а кроме того, они еще разнятся
составом этапов (в ФАТ нет структурного анализа для операций -- есть их
компонентный анализ, и потом сразу функциональное моделирование).

Это создает определенные «внешние» проблемы, так, приходится изучать ФА
дважды, занимать на это время в процессе обучения ТРИЗ. В практическом плане
встают вопросы о том, как и какой из подходов выбирать, на каких основаниях и
по каким признакам, и каких результатов ожидать.

\subsection{Противоречие 1}
При более тщательном рассмотрении выявляются более существенные расхождения
уже теоретического свойства.

Так, в ФАТ даются следующие определения:
\begin{itemize}
\item[а)] задача ФАТ -- «идентифицировать и классифицировать \textbf{функции
  операций} в процессе»;
\item[б)] «процесс – это последовательность \textbf{действий (операций)} над
  объектами».
\end{itemize}
На основе определения б) можно заключить, что «операции» -- этот синоним
«действий», а если объединить эту трактовку с определением а), то получится,
что «задача ФАТ – идентифицировать и классифицировать функции действий в
процессе».

В определениях ФАП, в свою очередь, говорится, что «функция есть действие,
выполняемое одним компонентом (носителем функции) для поддержания и/или
изменения параметра другого компонента (объекта функции)».

Итак, \textbf{противоречие 1} базовых определений ФАТ и ФАП очевидно: в ФАП –
\textbf{«действие» есть «функция»}, а в ФАТ \textbf{«действие» есть
  «операция»}, а некие «функции» еще должны быть определены в ходе собственно
анализа («функции, \textbf{выполняемые посредством} операции»).

\subsection{Противоречие 2}
Продолжение исследований в данном направлении и попытки снятия противоречия 1
приводят к еще более парадоксальной ситуации.

Вдумаемся, что такое «Функциональный анализ \emph{продукта}»,
«\emph{изделия}»? Разве он когда-то существовал, осуществлялся в ТРИЗ? Возьмем
самый известный пример, из «классики»: ФА для мясорубки\footnote{Как он описан
  в работе \cite{Gerasimov1991}.}. Что есть продукт для мясорубки? Фарш, мясо.
Разве фарш и мясо анализируются в этой работе?  Разве для фарша строится
компонентная, структурная, функциональная модели?  Быстрый взгляд на рис 1
показывает, что это не
так\footnote{https://triz-summit.ru/triz/metod/fsa/1991/}. 
\begin{quote}
Пример. В табл. 2 приведено ранжирование функций компонентов ручной мясорубки.

Таблица 2. Ранжирование функций ручной мясорубки

\begin{tabular}{p{4.5cm}p{4.5cm}c}
\textbf{Носитель функции} & \textbf{Функция} & \textbf{Ранг функции}\\
Мясорубка ручная & Измельчать продукт & \textbf{Г} \\
Нож & Тоже & \textbf{О}\\
Шнек & Вращать нож & \textbf{В$^I$}\\
Ручка & Тоже & \textbf{В$^{II}$}\\
Винт & Крепить ручку (к шнеку) & \textbf{В$^{III}$}
\end{tabular}
\end{quote}
Возьмем более современный пример, из практики уважаемой ТРИЗ-консалтинговой
компании, пример, ставший уже почти повсеместно известным образцом ФАП
\begin{center}
  Add picture
\end{center}
Продуктом системы «Подачи краски в ванну» является краска. Но разве мы заняты
функциональным анализом «Краски» в этой работе? Совершенно точно, нет. Прямо
по классике, компонентная, структурная и функциональная модель строятся для
«системы подачи». Но разве «система подачи краски в ванну» является
\emph{продуктом}? Тоже нет, это самый что ни на есть \emph{инструмент},
техническая система, созданная для выполнения нужной нам полезной функции.

Не поможет и замена термина «Продукт» на термин «Изделие». По сути, при любом
названии, когда мы говорим о продукте или изделии, речь идет о некотором
«объекте функции», чье функционирование мы никак не анализируем, берем как
данность, в некоем «страдательном залоге». Заметим, «Продукт» в современных
представлениях прямо относится к надсистеме, те в ТС не
входит\footnote{Справедливости ради надо заметить, что мы можем смещать свое
  внимание с инструмента на продукт, изделие – например, при поиске
  возможностей свертывания, где, по одному из правил, функции можно передавать
  изделию. Но это ничего не меняет. Да, мы можем применить ФА для продукта
  нашей ТС, но тогда это будет опять анализ этого «изделия» как инструмента, в
  т.ч. подразумевающего анализ его функционирования в этой роли. Мы можем,
  например, провести ФА для фарша, но в роли инструмента, а его продуктом
  будет рот или желудок.}.

Итак, \textbf{противоречие 2} состоит в том, что «Продукт» в ФАП – вовсе и не
продукт.

\subsection{Противоречие 3}
Заглянем теперь с другой стороны. А что дает нам выделение операций – в смысле
этапа и процедуры в ФАТ? Рассмотрение сквозь призму «операций» - это что? В
чем специфика? Преимущества такого способа рассмотрения?

Зайдем опять с распространенных примеров. Вот схема свертывания из примера на
применение ФАП, опять же ставшего общеизвестным (примеры взяты из интернета).
\begin{center}
  Add picture
\end{center}
Чем отличаются схемы ТС мясорубки, «ванны с краской» и данной системы сушки
эластомера? Чем функция Шнека «перемещать мясо» отличается от операции
«Горячий воздух нагревает растворитель»? Носитель, объект, функция...
материальные компоненты...

Далее, разве мы свертываем тут какие-то мифические «операции»? Нет, вполне себе
материальные компоненты, в духе ФАП. Котел, газ, воздух, теплообменник. От того, что мы
тут выделяем некоторые «операции» (они же «действия», см определения выше), разве у
этих действий пропадают носители/источники?. От того, что мы их не рассматриваем,
ничего не меняется, они существуют и именно они выполняют
операции/действия/функции.

Без них, носителей, эти процессы/операции/ невозможны, да и свернуть
операцию/действие мы сможем только свернув носитель... Или, может быть, просто
отключим его и оставим стоять в системе просто так («бездействовать»), чтобы
сохранить идею ФА для технологий?

Несколько особняком стоит еще такой, довольно тонкий вопрос -- а как,
собственно мы можем получить схему, совокупность действий/операций, иначе чем
через декомпозицию главной функции, проведение компонентного и структурного
анализа ТС? Просто представить себе функции не удается, потому что каждый раз
мы упираемся в некоторый предел – либо в объект функции, либо в носитель (в
зависимости от направления возможного рассмотрения). Так, если мы идем от
Главной функции к функциям более высокого ранга, вглубь системы, то уже между
функцией 1ого ранга и функцией 2ого ранга должен появиться некоторый
промежуточный элемент, который одновременно носитель Ф1 и объект Ф2. Иначе
функции просто не различатся, нет оснований для их разделения.

С другой стороны -- если в этом состоит сущность задания для ТРИЗ-проекта --
мы можем рассмотреть отдельные выявленные функции в отрыве от их носителя в
действующей ТС (например, при проведении функционально-ориентированного
поиска), но это не более чем частный случай, один из видов задач, базирующийся
на общей методологии ФА, и выполняемый после него, на базе его результатов.

Итак, \textbf{противоречие 3}. В ТС не существует функций без носителей, и,
соответственно, логика «выделения операций» в ФАТ есть просто редукция
обычного функционального анализа, а не какая-то качественно иная процедура.
Соответственно, нет и никакой отдельной специфики «операций» в ФАТ, отличной
от «функций» в ФАП.

\section{Дополнительные сложности, вызванные существующим\\ status quo}

\subsection{Процесс}

В описании подхода ФАТ даются также дополнительные понятия, мало задействованные
в собственно методике работы, но при этом создающие трудности для распространения
ТРИЗ подхода к анализу более сложных систем, или, точнее, к более сложному анализу
систем. Для начала рассмотрим понятие «процесс».

Процесс вводится как «последовательность операций». Если учесть все ранее сказанное
про соотношение «операции» и «функции», то вполне можно сказать, что процесс есть
«последовательность выполняемых функций», и мы получаем прямую отсылку к
функциональным моделям, получаемым в результате ФА. Пример для того же кейса
«подача краски» приведен ниже.
\begin{center}
  Add picture
\end{center}
Что здесь надо бы уточнить? По отношению к анализу ТС речь должна идти не
только и не столько о «последовательности действий», сколько прежде всего об
их «полной совокупности» (включая, конечно, порядок их выполнения), ибо именно
полная совокупность действий и задает и задает технологию, воплощенную в
данной ТС. И как только мы что-то меняем в нашей ТС, то ее функциональный
портрет трансформируется.  В этой связи термина «процесс», определенного как
«совокупность операций» \textbf{оказывается недостаточно}, т.к. мы теряем
возможность обсуждать разные процессы (в т.ч. разные по своей природе),
которые координируются между собой, и продукт, результат получается только при
их совокупном скоординированном исполнении.

В этой связи предлагается ввести/уточнить следующие уточняющие понятия,
которые могут снять указанные затруднения:

\paragraph{Поток}
(сохранить из потокового анализа) – движущийся однородный материальный объект,
движение которого рассматривается как непрерывное (река, электричество при
включенной настольной лампе, проч).

\paragraph{Процесс (моно-процесс)}
– движение дискретных однородных единиц одной природы, которое можно при
определенных условиях рассматривать как поток (движение автомобилей по дороге,
движение конвейера, транспортерной ленты).

\paragraph{Полипроцесс}
– совокупность процессов разной природы в технической системе, реализующих
выполнение ею Главных и дополнительных функций (например, информационный
(моно)процесс порождает сигнал, который запускает (моно)процесс зажигания,
которое в свою очередь запускает процесс горения, которое запускает процесс
движения ракеты).

Собственно различение процесса и полипроцесса позволяет нам увеличить
сложность рассмотрения ТС. В качестве примера можно рассмотреть соединение
классической модели ТС (Источник энергии – трансмиссия – рабочий орган) с
системой управления (вкоторой реализуется процесс другой природы, нежели,
например, энергетический, но именно координация обоих процессов необходима для
выполнения функций ТС). Другим приложением этого понятия может быть анализ ТС
с рассмотрением различных подсистем, находящихся на разных уровнях развития по
S-кривой, или введение модели «поли-противоречий», как некоторого набора из
взаимосвязанных противоречивых требований к развитию.

\subsection{Технологический процесс}
Технологический процесс определяется как процесс, в котором в котором «для
создания готового продукта используются материальные объекты, такие как сырье,
рабочая сила, энергия и оборудование». Ничего собственно ТРИЗовского в этом
определении нет, обычное заимствование из инженерных и экономических
дисциплин. По сути аналогичен определению производственного процесса, про
который говорится, что в его результате происходит «необратимое изменение»
параметров продукта.

С точки зрения подхода ФАТ, мы вновь попадаем в ситуацию избыточных
определений и разно-речивых определений. Если мы используем термин
«технологический процесс», то, значит, говорим о совокупности операций. Но мы
уже выяснили, что операции это то же, что и действия, те. технология есть
совокупность действий. А чем тогда по содержанию это отличается от
функциональной модели, которую мы получаем в результате ФАП? См выше
графическая схема функционирования для ванны с краской – это и есть
«технология», которая реализована в данной ТС,

\section{Предложения}

С учетом изложенного выше предлагается сделать несколько практических
корректировок в существующих разделах стандартов МАТРИЗ – ФАП и ФАТ.
\begin{itemize}
\item[1.] Упразднить разделение на ФАП и ФАТ как искусственное, связанное с
  определенным этапом развития ТРИЗ, и в настоящее время не несущее
  принципиально новых смыслов, значимых в теоретическом и практическом смысле.
  Унифицировать определения понятий «действие», «операция», «функция», «тип
  функции» и т.д., с учетом задачи синергетического объединения сделанных
  наработок по ФАП и ФАТ.
\item[2.] Ввести в оборот (может быть, вернуть) термин «Функциональный Анализ
  Технических систем (ФАТС)», совпадающий по структуре, методике и результатам
  с сегодняшней версией ФАП. Произвести интеграцию в ФАТС результатов и
  наработок по ФАП и ФАТ.
\item[3.] Ввести в оборот понятие «полипроцесс», с учетом предложенных
  уточнений к понятию «процесс»
\item[4.] Ввести понятие «технология» как «полипроцесс», реализованный в
  данной ТС, с учетом возможного ограничения уровней рассмотрения (только
  основные функции, только функции с рангом не более N, и т.д.).
\end{itemize}

\section{Контекст текста}

Данный материал представляет собой продолжение работ, начатых автором в 2019г,
и названных\footnote{С легкой руки А.В.Кудрявцева} «Проект Галилео». В целом
проект посвящен переоформлению, переорганизации концепций и положений
современной ТРИЗ, с учетом фактического состояния дел, которое было
охарактеризовано как «острова в океане». Так метафорически было
охарактеризовано сегодняшнее состояние собственно теории, теоретической части
ТРИЗ, представляющую из себя ряд появлявшихся в той или иной
последовательности разработок, без проведения работы по их системной связки с
уже существующим, интеграции, критическому пересмотру и логической
переупаковке. В докладе «Галилео 1» (доклад на Мос ТРИЗ Конференции 2019)
говорилось о переорганизации, пере-систематизации таких разделов, как
«Технические противоречия» и «Физические противоречия», на основе положений
диалектической логики. В данном тексте («Галилео 2) аналогичная работа начата
по отношению к разделам «Функциональный анализ для продукта» и «Функциональный
анализ для технологий».

Продолжение следует.

\begin{thebibliography}{xxx}
\bibitem{Gerasimov1991} В. М. Герасимов, В. С. Калиш, М. Г. Карпунин,
  А. М. Кузьмин, С. С. Литвин.  ТРИЗ-ФСА -- Методические рекомендации 1991.
  \emph{Основные положения методики проведения функционально-стоимостного
    анализа}.  \url{https://triz-summit.ru/triz/metod/fsa/1991/}
\end{thebibliography}


\end{document}
