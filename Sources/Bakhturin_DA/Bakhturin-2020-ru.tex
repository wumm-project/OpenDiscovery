\documentclass[11pt,a4paper]{article}
\usepackage{od}
\usepackage[utf8]{inputenc}
\usepackage[russian]{babel}

\title{Не-ТРИЗ версии системных представлений.\\[1em] \Large Тезисы к докладу на
  Московской ТРИЗ-Конференции 2020.} 

\author{Бахтурин Д.А.}
\date{November 2020}

\begin{document}
\maketitle
\tableofcontents

\begin{abstract}
  В докладе представлены и кратко охарактеризованы некоторые версии
  представлений о «системах» и «технических системах» из различных
  интеллектуальных традиций. Большая часть представленных тезисов основана на
  докладах упомянутых представителей тех или иных школ, сделанных ими в
  Московском ТРИЗ-Клубе в 2019-2020гг. Полные версии докладов доступны на
  сайте Metodolog.ru в разделе «Материалы Московского ТРИЗ-Клуба».
\end{abstract}

\section{Деятельностно-природная система.}

Разработки Московского методологического кружка (ММК), наиболее развитое
описание в работе П.Г.Щедровицкий «Деятельностно-природная система».
Значимость данной версии в том, что она указывает на вполне определенную –
марксистскую – традицию, с помощью которой можно более глубоко понять
применяемую в ТРИЗ версию, о том, что ТЕХНИКА = Социальное + Природное.
Следует отметить, что в самой ТРИЗ представления о том, что такое
«социальное», фактически не проработана, в этой связи замена понятия
«социальное» на «деятельностное» вполне оправдана, тем более, что ТРИЗ и
разработки ММК имеют общие – марксистские – корни. Представление о
«Деятельности», с одной стороны, отсылает к Марксовым тезисам о Фейербахе
(«философы разными способами пытались объяснить мир, а дело в том, чтобы его
изменить»), а с другой стороны, позволяет заимствовать целый ряд разработок
ММК по общей теории деятельности, в продолжение и развитие марксистской
традиции. В частности, в ММК была создана и проработана система представлений
о Деятельности, воплощенная в т.н. «схеме воспроизводства Деятельности и
трансляции Культуры», указывая, что ведущим компонентом Деятельности является
Норма, воспроизводящаяся (транслируемая) в каждой новой ситуации.

\paragraph{Вопрос:}
Является ли функционирование ТС «деятельностью» в духе традиций ММК?

\paragraph{Ответ:}
«Деятельность» – это исключительно человеческая деятельность, и она в этом
плане не в «духе ММК», а в духе традиций Маркса и Энгельса, и является некой
вершиной в развитии так называемых «материалистических представлений».

С этой позиции можно сказать, что человеческая деятельность -- это объемлющая
категория. Она расширяется, развивается, растет – за счет в том числе и
создания «технических систем», которые берут на себя куски деятельности, ранее
выполнявшиеся человеком. Исходя из этого, можно говорить, Человечество
осуществляет Деятельность, а ТС функционируют, выполняют определенные функции
внутри этой Деятельности.  Функционирование ТС замещает деятельность человека.

\paragraph{Вопрос:}
Как мы можем использовать понятие деятельности при анализе ТС?

\paragraph{Ответ:}
Деятельность как объемлющая категория задает общую, надсистемную рамку для
функционирования ТС. В зависимости от того, в какую конкретно деятельность
включена данная ТС, мы по разному будем определять ее (ТС) главные,
дополнительные и вспомогательные функции. Так, если мы используем микроскоп
внутри исследовательской деятельности, у него будет один функциональный
портрет, а если же внутри работы по колке орехов – то совершенно другой, хотя
при этом компонентный состав микроскопа останется неизменным.

\section{Схема Мыследеятельности.}
Одна из финальных разработок ММК, в определенном смысле подводящая итоги его
работы. Схема лежит в основе т.н. Системо-Мыследеятельностной Методологии
(СМД-методологии). В основе схемы – выделение трех слоев – Деятельности,
Коммуникации и Чистого Мышления, а также наборов позиций, связанных с этими
слоями. Важность схемы в том, что в ней 
\begin{itemize}
\item[а)] выделяется ведущая роль слоя коммуникации между субъектами,
\item[б)] производится расщепление того, что в общем виде называется
  «мышление» на ряд частных интеллектуальных процессов, таких как «понимание»,
  «рефлексия», «коммуникация» и тд.
\end{itemize}
Схема имеет важное значение для обеспечения процессов групповой работы, а
также для организации педагогической практики. Исторически схема возникла в
противостоянии со сторонниками чисто «деятельностного» подхода, в котором не
было места для Мышления. Огромный пласт работ был проделан для того, чтобы
включить и связать категории «Мышление» и «Деятельность» (что, собственно, и
манифестирует термин «Мыследеятельность»). Кажущаяся очевидность подобных
переходов не должна вводить в заблуждение. Практикующие в РФ ТРИЗ-специалисты
постоянно упираются в эту проблему, фиксируя, что наша «современная»
промышленность сфокусирована на разработке и выполнении «планов работ», те на
деятельности, а места для мышления, инноваций, ТРИЗ – в этих планах нет, это
место приходится создавать, «выбивать с боем». В философском плане это и
означает, как говорил Г.П.Щедровицкий, что у нам «много без-мыслительной
деятельности».

\paragraph{Вопрос:}
Как можно использовать наработки ММК по мышлению в процессе анализа работы ТС?

\paragraph{Ответ:}
В прикладном плане важно следующее: мы, ТРИЗ-специалисты, в основном получаем
информацию (в особенности на начальном этапе, при анализе ситуации) от
экспертов, действующих инженеров и руководителей, те не напрямую, а при
посредничестве процесса «коммуникации». Сама ТС редко нам дана в своей
вещности.  Понимание зависимости информационного содержания от позиции,
которую занимает тот или иной коммуникант (эксперт, руководитель, инженер)
помогает уточнить локальную истинность этой информации. Например, многим
руководителям «по позиции» не выгодно внедрять и применять ТРИЗ,
соответственно, информация, которая от них («из этойпозиции») может поступать,
способна вводить в заблуждение, искажать картину и тд.  Вариант выхода из
подобной ситуации – «перепозиционирование», те перемещение данного
коммуниканта в другую позицию, где его целям будет соответствовать сообщение
иной информации, «под другим углом», «в другом контексте» и тд. Частным
случаем такого «перепозиционирования» является как раз обучение ТРИЗ
различного рода руководителей и главных специалистов. Они за счет этого
занимают другую позицию.

\section{Историческая традиция}
«Всемирная организация по системам и кибернетике», в Мос ТРИЗ Клубе выступал с
этой темой философ и методолог Вячеслав Марача. Наиболее важным из доклада
является отстаиваемая В. Марачей позиция (гипотеза) о том, что системный
подход в его современной версии – относительно новое «изобретение», возникшее
на рубеже 19-20 веков, и оформившееся к середине 20в. В частности значимым
этапом является выход книги Гуда и Маккалоу «Системотехника». Ключевая
содержательная гипотеза В.Марачи состоит в том, что системный подход пришел на
смену т.н. «предметному подходу».  Предметный подход предполагал формирование
комплекса представлений по образцу отдельных наук – физики, химии и тд... На
определенном этапе развития «предметные» представления переросли сами себя,
так как стали появляться мульти-предметные знания (электро-механика,
физическая химия и тд). Системный подход пришел на смену «предметному», так
как позволял унифицировать разрозненные представления «предметного подхода» --
за счет применения общей идеологии, «системных представлений». С точки зрения
В. Марачи, на начальном этапе системный подход в целом оправдывал возлагавшиеся
на него ожидания, но в настоящее время мы имеем также большой набор разных
системных версий, моделей, представлений. По сути, происходит то, что на
поздних этапах случилось с «предметным» подходом, а именно идет процесс
размывания унифицированного базового представления, появление большого
количества частных версий и гипотез, что подрывает сами основания «системного
подхода».

\paragraph{Вопрос:}
Можно ли так сказать, что системных версий так много, что произошло размывание
"системного подхода" и мы оттуда вообще уже ничего взять не можем?

\paragraph{Ответ:}
Ну, с одной стороны, не совсем уж так много, как известно из разработок самого
В. Марачи. Хотя в целом, с моей точки зрения, ТРИЗ мыслит системность иначе, и
нам нет смысла полностью «вписываться» в логику, принятую в «системном
движении». Моя версия в том, что ТРИЗ преодолела упомянутые недостатки
системного подхода и знаменует собой переход к пост-системному мышлению, как
бы отталкиваясь от него и переходя на следующую S-образку. За счет чего?
Вспомним рекомендации 3-4 этапа ЗРТС, по переходу на новую S-кривую. Выявить
противоречия, присущие старому подходу (которые препятствуют развитию), и
найти новый способ их разрешения. Так вот, классическая теория систем пытается
охватить весь мир систем как объектов, расклассифицировать, описать все версии
и все связи. ТРИЗ движется в логике «Система = проблема/задача», а не
«натуральный» предмет и не объект. У нас системна ситуация, системно устроена
проблема/задача. Таким образом мы и преодолеваем ограничения «официального»
системного подхода, переорганизуя все многообразие описываемых ими систем в
совокупность небольшого количества системно представляемых
проблем/изобретательских задач.

\section{«Системная инженерия»}
Доклад в Мос ТРИЗ Клубе – В.Батоврин, один из лидеров системно-инженерного
движения в РФ, переводчик ключевых книг по СИ на русский язык, создатель
первой в РФ кафедры Системной инженерии в МИРЭА. В современном мире системная
инженерия (далее СИ) претендует на роль «прикладной теории систем» (по
аналогии, как говорят про ТРИЗ, что это «прикладная диалектика»).  Когда-то
Г.С. Альтшуллер различил «инженерные задачи» и «инженерные проблемы». В этой
логике можно сказать, что системная инженерия – это обобщенная теория
инженерной деятельности, практики решения инженерных задач. В качестве
философско-методологического базиса в СИ принимаются те или иные версии
«системного подхода».  Из значимых характеристик СИ, заявленных в докладе
В. Батоврина, можно выделить 
\begin{itemize}
\item[а)] отсутствие единого понятия «системы» и единого определения того, что
  же такое «системная инженерия», чем она занимается и за что отвечает,
\item[б)] в логике СИ для характеристики систем не является обязательным
  наличие «синергетического эффекта» (когда свойство системы больше суммы
  свойств ее частей), т.е. система может быть просто определена через набор
  входящих компонентов, без указания на «системный эффект»,
\item[в)] в системной инженерии нет представления о проблемах и препятствиях
  при синтезе, сборке частей в систему (в этой связи там нет представлений о
  месте изобретений, об изобретательской ситуации и тд.)
\end{itemize}

\paragraph{Вопрос:}
В системной инженерии система не определяется, может являться перечислением не
связанных друг с другом элементов. Это удобнее, чем то, что в ТРИЗ сейчас?

\paragraph{Ответ:}
Нет, я так не говорю. У них система тоже совокупность связанных друг с другом
элементов, пусть даже это какая-то формальная связь, а не материальная.
Главное, что в СИ не используется представление о синергийности как о
необходимом качестве системы, в ТРИЗ же ГСА вводил это впрямую. Для ТРИЗ
фактор синергийности ТС указывает на скачкообразное, качественное развитие,
тогда как в СИ развитием может считаться просто добавление еще одного
компонента в ТС, просто как результат инженерной работы. С отсутствием
синергийности, имхо, связано также и то, что в СИ нет понятия проблема,
изобретение, развитие (там нет для них места, так как нет скачков, разрывов).
Им это, безусловно, «удобнее», тк сводит все к рутинной инженерной работе, в
тч легко поддающейся автоматизации. В ТРИЗ такие представления невозможны, тк
убивают саму идею скачкообразного качественного развития ТС, собственно
изобретательства.

\section{Школа системного мышления}
(Анатолий Левенчук и Ко, доклад в Мос ТРИЗ Клубе). Школа является некоторой
авторской версией системного подхода А. Левенчука, выросшего из системной
инженерии и методологии ММК. Производятся регулярные заимствования из ТРИЗ, в
частности, понятие «Надсистемы» из ТРИЗ-Системного оператора. Справедливо
указывается на трансцедентальный, неочевидный, нетривиальный характер перехода
по иерархическим уровням «подсистема» -- «система» -- «надсистема»
(обстоятельство, которому мало уделяется внимания в ТРИЗ). Продвигаемое
А. Левенчуком понятие «системоинженерное мышление» при тщательном рассмотрении
оказывается «системно устроенным мышлением» Фокус переносится с «систем» как
внешне существующих «объектов» на «системную организацию» представлений об
объектах, о внешнем мире (см. для сравнения статья ГСА 1975 года – «Развитие
системного мышления – конечная цель обучения
АРИЗу»)\footnote{\url{https://www.altshuller.ru/triz/triz70.asp}}. Аналогичные
дискуссии можно увидеть и в ТРИЗ, например, относительно «объективного
характера противоречий» («Вот стоит труба и дымит -- где тут противоречие?».
В. Митрофанов), хотя в школе А. Левенчука этому уделяют значительно больше
внимания.

\section{Резюме}
Считаю, что ТРИЗ бесполезно пытаться создать свою собственную «общую теорию
систем». Аналогичные процессы в мире значительно мощнее, имеют давнюю
традицию, вплетены в культуру и тд. Необходима организация диалога с
представителями другихсистемных подходов и методологий – именно для того,
чтобы обеспечить вплетение ТРИЗ в эти общемировые движения, и занятие ТРИЗ в
них своего – достойного – места. Попытки развивать внутри ТРИЗ собственную
«общую теорию систем» ведут лишь к увеличению автаркии, замкнутости,
изолированности ТРИЗ, и тем самым препятствуют ее развитию.

Примером такого непродуктивного действия считаю недавно состоявшуюся дискуссию
в Совете по исследованиям и разработкам, по вопросу «Сколько главных функций
может иметь техническая система?» По итогам голосования в Совете вполне могло
бы оказаться, что в ТРИЗ-мире все же ГФ у ТС – \emph{одна}. Но достаточно
взглянуть на объективный мир вокруг – чтобы увидеть, как стремительно растет
объем функционала, реализуемого создаваемыми ТС (многофункциональные
комплексы, грузовик с встроенным краном-манипулятором, компьютер,
смартфон...).  Тем более, что и в учебных целях мы приводим самые разные
примеры на эту тему – свертывание, развертывание, линии ЗРТС и т.д.  Очевидно,
что в «большом» инженерном мире у систем может быть (и есть!) множество
Главных Функций. Как сохранить это общее свойство и прописать «внутри» него
специфику ТРИЗ – этот вопрос, к сожалению, в «автаркическом» подходе даже не
может быть поставлен. Апелляции к тому, «как это было бы удобнее» вообще
переводят обсуждение из общезначимого в некий коммунальный план – «мне так
удобнее, а мне так, мы с тобой уважаемые люди». Некоторые идеи автора по
данному вопросу приведены в тексте доклада.

Развитие техники, инженерии и изобретательской практики – объективный процесс,
и каждая из составляющих имеет свое, уникальное и в то же время связанное с
другими место и функции. Уже упоминавшийся Системный оператор нам в помощь.

\end{document}
