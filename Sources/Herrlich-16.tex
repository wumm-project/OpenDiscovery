\documentclass[11pt,a4paper]{article}
\usepackage{od}
\usepackage[utf8]{inputenc}

\title{Von der Arbeitsgemeinschaft\\ „Erfindertätigkeit und Methodik des
  Erfindens“\\ über die KDT-Erfinderschulen\\ hin zur Deutschen
  Erfinder-Akademie}

\author{Michael Herrlich, Leipzig, und Justus Schollmeyer, Berlin}
\date{20.11.2016} 

\begin{document}
\maketitle
\begin{quote}
  Aus Gesprächen mit Michael Herrlich vom 12.9.2016 und vom 21.11.2016
  zusammengestellt von Justus Schollmeyer als Beitrag zur
  21. Leibniz-Konferenz „Systematisches Erfinden“, 24.--25. November 2016 in
  Lichtenwalde.
\end{quote}

Die Geschichte der Erfinderschulen der Kammer der Technik und damit auch die
der TRIZ in Deutschland ist untrennbar mit dem Namen \emph{Michael Herrlich}
verbunden.  Wer ist dieser Mann?

Michael Herrlich wurde am 9.3.1938 in Leipzig geboren. Sein Großvater Bernhard
Frölich war der Erfinder der Ofenrohrkniebiegemaschine. Schon früh begeisterte
er seinen jungen Enkel, dessen Eltern ein heute 151 Jahre altes
Familienunternehmen namens Leder-Herrlich führten, für Technik. Wegen des
Familienbetriebs galt der junge Michael zu DDR-Zeiten weder als Arbeiter-,
noch als Bauern- oder Intelligenzlerkind, sondern als „Sonstiger“. Daher
konnten seine Zensuren noch so gut sein, er erhielt zunächst weder Oberschul-
noch Studienplatz. Herrlich schrieb an den Staatsrat. Er wies darauf hin, dass
die Verfassung „Sonstige“ mit guten Zensuren nicht von weiterführender Bildung
ausschließe und drohte damit, seiner Tante in West-Berlin von dem Fall zu
berichten.  Seine Tante würde sich wiederum an die Presse wenden, die auf
derartige Fälle regelrecht warte. Herrlich entschied sich ganze dreimal für
diesen Schachzug. Beim ersten Versuch erhielt er seinen Oberschul-, beim
zweiten seinen Studienplatz für Maschinenbau an der TU Dresden bei
Prof. Gottfried Tränkner. Erst der dritte Brief, mit dem Herrlich ein
Stipendium erstreiten wollte, wurde mit der Begründung abgelehnt, dass darauf
auch nach der Verfassung kein Anspruch bestünde. Doch auch hier fand sich bald
eine Lösung. Gottfried Tränkner, der auf Herrlich aufmerksam wurde, hatte
einen Auftrag für ihn: Die Kriegsjahrgänge, die ihr Ingenieur-Studium noch vor
dem Abschluss unterbrechen mussten, sollten es nun in Abendschulen von 16-21
Uhr abschließen können. So begann Herrlich nun im Alter von 19 Jahren gegen
Bezahlung Maschinenbaufächer zu unterrichten.

Zu dieser Zeit wohnte er auf dem Plauenschen Berg in Dresden. Von seinem
Zimmer aus blickte er auf das Institut des berühmten Erfinders Manfred von
Ardenne. Die Wege der beiden sollten sich später noch öfters kreuzen.
Überhaupt war Herrlichs Weg eher untypisch – genau wie das Auto, mit dem er
täglich vom Plauenschen Berg aus zur Uni fuhr. Ein Opel Cabriolet aus den 20er
Jahren stand eines Tages zum Verkauf vor seinem Fenster. Herrlich handelte
schnell, kaufte von seinen ersten zwei Gehältern den Wagen für 1300 Ostmark
und staunte nicht schlecht, als sich herausstellte, dass er mit blankem
Waschbenzin fuhr. Später tauschte er den Oldtimer gegen einen neuen Wartburg
Tourist.

Auch sein Studium zum Verarbeitungsmaschinenkonstrukteur durchlief er
schneller als gedacht, genauer gesagt in der Hälfte der vorgesehenen
Zeit. Grund dafür war die Erfindung eines Schwingförderers für
Montageautomaten. Herrlich präsentierte ihn 1961 in seinem Großen Beleg – dem
heutigen Vordiplom. Als er zur Besprechung der Arbeit in das Büro seines
Professors kam, bat Tränkner ihn aus dem Sekretariat zwei Büroklammern zu
besorgen. Kaum war Herrlich zurück, teilte Tränkner die Arbeit in zwei Teile,
in einen theoretischen Teil, der die Berechnungen enthielt, und in einen
Anwendungsteil, der die Erfindung beschrieb. Den ersten Stapel wertete
Tränkner als Vordiplom, den zweiten als Diplom. Am nächsten Tag fand die
Verteidigung statt und wenige Minuten später war Herrlich diplomiert.

1962 begann Herrlich als stellvertretender Direktor des Instituts für Süß- und
Dauerbackwaren Leipzig zu arbeiten. Derartige Waren wurden damals weltweit
noch manufakturell hergestellt. Herrlichs Erfindungen zur vollmechanisierten
Herstellung von Bonbons, Dragees und Knäckebrot konnten daher weltweit
patentiert und nach Japan, in die BRD und USA in Lizenz vergeben werden. Schon
bald erfüllte Herrlich die Norm als Verdienter Erfinder. Bedingung für die
Normerfüllung waren 20 erteilte Patente, Lizenzvergaben in kapitalistischen
Ländern und ein wirtschaftlicher Nutzen von mindestens zwei Millionen Mark.
Als Herrlich 1972 ausgezeichnet wurde, verfügte er bereits über 44 erteilte
Patente, deren wirtschaftlicher Nutzen auf drei Millionen Mark geschätzt
wurde. Die Auszeichnung war der einzige auch finanziell honorierte Orden in
der DDR-Geschichte. Verdiente Erfinder erhielten jeweils 10\,000 Mark. Bis zur
Wiedervereinigung vergab das DDR-Patentamt den Titel insgesamt 220 Mal. Ihr
jüngster Preisträger war Michael Herrlich.

Schon 1963 hatte Herrlich gemeinsam mit anderen „Verdienten Erfindern“ die
Arbeitsgemeinschaft \emph{Erfindertätigkeit und Methodik des Erfindens} der
Kammer der Technik gegründet. Aus diesem Kreis sollten später die
Erfinderschultrainer hervorgehen.  Der Aktionskreis der AG war zunächst auf
den Bezirk Leipzig beschränkt, breitete sich später über das gesamte
DDR-Gebiet aus und existierte bis zur Wende. In dieser Zeit wurde Herrlich
gemeinsam mit Gerhard Zadek vom DDR-Patentamt von der sowjetischen
„Allunions-Gesellschaft der Erfinder und Rationalisatoren“, kurz: VOIR, nach
Moskau eingeladen. Zur Überraschung der beiden empfing sie ein Diplomat, der
sie über den Diplomatenausgang aus dem Flughafen führte. Er erzählte, dass
Lenin Adliger und promovierter Patentanwalt gewesen sei. Patente seien laut
Lenin der wichtigste Rohstoff jeder Volkswirtschaft – wichtiger als Diamanten
und Juwelen. Aus diesem Grund sei beim Entwurf der Institutionen in der
Sowjetunion darauf geachtet worden, dass das Patentamt nicht nach britischem
Vorbild dem Justizministerium, sondern dem Wirtschaftsministerium unterstand.
Nach diesem Muster wurde auch das Patentamt in Ost-Berlin dem
Wirtschaftsministerium unterstellt; und damit Günter Mittag – dem Gegenspieler
Kurt Hagers (Wissenschaft).

Herrlich und Zadek wurde zusehends klar, dass es sich bei VOIR um eine
vermögende Organisation handelte. Sie erfuhren, dass alle sowjetischen
Betriebe einen gewissen Prozentsatz ihres Umsatzes an die Organisation
abführen mussten und dass VOIR über ca. 40\,000 Angestellte, diverse Institute
und für die Anfertigung von Patenten vorgesehene Räume in Wohnblöcken
verfügte. Zu dieser Zeit richtete VOIR bereits TRIZ-Schulungen aus, sodass
Herrlich und Zadek auf dieser Reise erstmals von Genrich Saulowitsch
Altschullers TRIZ erfuhren. Als Gastgeschenke erhielten sie einen Text
Altschullers. Zurück in Leipzig angekommen, schickten Herrlich und sein Team
einen Anzug zu Altschuller nach Baku. Sie hatten gehört, dass er wegen seiner
jüdischen Abstammung die Stadt nicht verlassen durfte. Das russische
Manuskript wurde dem Germanisten Kurt Willimczik aus Berlin übergeben, der in
Herrlichs AG gekommen war und nun mit der Übersetzung der Materialien
beauftragt wurde. Da die Papierkontingente der Verlage erschöpft waren,
verzögerte sich die Veröffentlichung. Als erster verfügte der Tribüne Verlag
wieder über Papier, sodass der Text „Erfinden – (K)ein
Problem?“\footnote{Original: \foreignlanguage{russian}{Алгоритм изобретения. —
    М.: Московский рабочий. — 1969.}} 1973 schließlich dort als
Propagandamaterial veröffentlicht wurde.

Wie auch andere seiner Kollegen aus dem Kreis der Verdienten Erfinder war
Michael Herrlich parteilos. 1978 schrieb er dem SED-ZK-Sekretär für
Wirtschaft, Günter Mittag, einen Brief, in dem er empfahl, ihm einen
Gewerbeschein zur Erfinderausbildung zu erteilen. Sein Schreiben blieb
unbeantwortet. Ein Jahr später, am 20.12.1979, sendete Herrlich erneut eine
Kopie des Briefes. Nun ging alles sehr schnell. Zwischen Weihnachten und
Neujahr klingelte das Telefon und Herrlich wurde für den 3.1.1980 nach Berlin
geladen. Dort traf er den Präsidenten des Amts für Erfindungs- und
Patentwesen, Prof. Joachim Hemmerling, und den Präsidenten der Kammer der
Technik, Prof. Manfred Schubert. Hemmerling kannte Herrlich bereits, da er ihm
1972 die Auszeichnung zum Verdienten Erfinder samt Prämie überreicht hatte. Da
Herrlich kein Gewerbeschein für Erfinderschulen erteilt werden konnte, wurde
ihm eine Berufung ans Patentamt als Lektor vorgeschlagen und zwar so, dass er
direkt dem Patentamtspräsidenten Hemmerling unterstand. Hemmerling konnte dann
über das Büro Mittag, also über das Wirtschaftsministerium, Erfinderschulen
beim Zentralkomitee bestellen. Zwar hatte die Anstellung am Patentamt den
Nachteil, dass Herrlich selbst fortan nicht mehr erfinden durfte, immerhin
konnte er für sich aber aushandeln, nur maximal einmal im Monat im Amt
anwesend sein zu müssen.

Aufbauend auf seiner AG \emph{Erfindertätigkeit und Methodik des Erfindens}
konnte Herrlich nun damit beginnen, Erfinderschultrainer für die Durchführung
von Erfinderschulen auszubilden. 1988 standen ihm insgesamt 70 trainierte
Erfinderschultrainer zur Verfügung. Bezahlung der Dozenten, Bewerbung der ein-
bis zweiwöchigen Seminare und Anmietung der Tagungsräume übernahm die Kammer
der Technik. Die erste Tagung fand in der sächsischen Schweiz statt. Neben
einigen Stasi- Spitzeln nahm daran auch der Vize-Präsident der KDT,
Prof. Horst Bendix, teil.  Ehrenpräsident der Erfinderschulen war
Prof. Manfred von Ardenne. Herrlich hatte mit dessen Tochter, Beatrice Bettina
Wilhelmine, an der TU Dresden studiert. Sie wiederum erzählte ihrem Vater von
Herrlichs Ausbildungen und Ardenne lud daraufhin jährlich ein fünfzehnköpfiges
Team aus Herrlichs Arbeitsgemeinschaft ein. Die AG selbst tagte zweimal im
Jahr – im Frühjahr und im Herbst in der „saure Gurkenzeit“ auf der Wartburg,
der Hohen Sonne, oder der Veste Wachsenburg. Die KDT stellte jeweils 30 bis 40
Betten.  Während dieser Treffen wurde die Grundsubstanz der Erfinderschulen
entwickelt und zu Papier gebracht. Besonderer Fleiß wurde mit der Einladung zu
Ardenne belohnt. In die Erfinderschulmaterialien flossen neben den
Grundgedanken der TRIZ die Erfahrungen, Überlegungen und Analysen des gesamten
Kreises ein. In Altschuller sieht Herrlich heute den Ersten, der die
glorreiche Idee hatte, dass Patenterteilungen nach Spielregeln erfolgen. Diese
Regeln habe Altschuller versucht durch Patentanalysen offenzulegen.  Aufbauend
auf Altschullers Entdeckungen entwickelten Herrlich und sein Arbeitskreis
weitere eigene Prinzipien.

Seine finanziell lukrativste Erfindung verdankte Michael Herrlich dem Prinzip
der Umkehrung. 1963 traf er auf einer Messe in Düsseldorf den HARIBO-Chef
Dr. Hans Riegel aus Bonn. Er hatte bereits von Herrlich gehört und wollte ihn
sprechen. In den Schlauchbeuteln, die HARIBO zu dieser Zeit verwendete, kam es
von Zeit zu Zeit vor, dass einige Gummibärchen zusammenklebten. Spätestens als
in Altersheimen der Bundesrepublik damit begonnen wurde, gezielt nach
zusammengeklebten Tierchen zu suchen, um anschließend die nahezu
leergegessenen Packungen zu reklamieren, schlug das Problem merklich zu Buche.
Riegel versprach Herrlich eine saftige Belohnung, wenn er einen Weg fände, des
Problems Herr zu werden. Zurück in Leipzig ließ sich Herrlich
Gelatinezuckerwaren anfertigen, nahm den Föhn seiner Frau Renate und trocknete
die Stücke mit warmer trockener Luft. Es sah so aus, als wäre das Problem
gelöst. Als er am nächsten Tag zurückkam, fand er jedoch, dass sämtliche
Stücke zusammenklebten. An der tatsächlichen Lösung arbeitete Herrlich noch
anderthalb Jahre. Ihm, dem Getriebebauer, der mit seiner Spezialisierung zu
den kalten Maschinenbauern gehörte und der somit während seines Studiums nur
wenig mit Thermodynamik zu tun hatte, wurde rückblickend klar, dass ihn ein
Problem der Thermodiffusion beschäftigte. Überall in der Fachliteratur hieß es
damals „porenlose hygroskopische Körper“, also Körper, die die Eigenschaft
haben, Feuchtigkeit aus der Umgebung zu binden, „lassen sich nicht
trocknen“. Genau das war es, was Herrlich mit dem Föhn seiner Frau vergeblich
versucht hatte. Eines Abends traf er während einer Konzertpause im Leipziger
Gewandhaus den für seine Theorie der Thermodiffusion in Gasen und
Flüssigkeiten bekannten Physik-Nobelpreisträger Prof. Gustav Hertz. Herrlich
fragte ihn, ob die Thermodiffusion, also die Bewegung von Teilchen bei
Temperaturgefällen, die in der Regel von warm nach kalt verläuft, auch für
Festkörper gelte. Zu Herrlichs Überraschung erwiderte Hertz, davon keine
Ahnung zu haben. Die Tatsache, dass der Physik-Nobelpreisträger diese
Möglichkeit nicht ausschloss, ließ Herrlich vermuten, dass sich die
Erkenntnis, dass schwingende Moleküle von warm nach kalt wandern, auch auf
Festkörper anwenden ließe. Mit Temperaturunterschieden, dachte sich Herrlich,
sei immer auch Stofftransport verbunden. Der warme Luftstrom des Föhns, der
die Gelatinestücke hätte trocknen sollen, hatte die Feuchtigkeit also nur
vorübergehend ins Innere der Stückchen zurückgetrieben.  Statt warmer Luft,
folgerte Herrlich, war also kalte Luft nötig. Diese Einsicht führte zu einer
ganzen Reihe weiterer Erfindungen und bis zum funktionierenden Prototyp zur
Intervalltrocknung waren weitere Tricks von Nöten. So durfte der Föhn immer
nur 2 bis 3 Grad kälter sein als das Gummitierchen, das schrittweise erhitzt
und dann wieder abgekühlt wurde. Damit die Luftfeuchtigkeit nach der Trocknung
nicht erneut in die Gelatinemasse eindrang, bedurfte es einer abschließenden
Versiegelung mit Bienenwachs. Das Problem war gelöst und Gummibärchen von
HARIBO wurden fortan in der VEB Süßwarenfabrik Wesa Wilkau-Haßlau bei Zwickau
hergestellt. 99 Prozent der produzierten Waren gingen nach Bonn. 1990 wurde
das Werk von der Treuhand für den symbolischen Wert von 1~DM an HARIBO
verkauft.

Bereits 1984 hatten Klaus Busch in Rostock und 1987 Hansjürgen Linde in
Dresden zur TRIZ- und Erfinderschulmethodik promoviert. 1988 reichte nun auch
Michael Herrlich als Externer an der TU Ilmenau seine Dissertation ein. Das
Thema lautete: \emph{„Erfinden als Informationsver\-arbeitungs- und
  -generierungsprozess, dargestellt am eigenen erfinderischen Schaffen und am
  Vorgehen in KDT-Erfinderschulen“}. Kurz vor dem Mauerfall, im Okt. 1989,
wurde Herrlich plötzlich von der Stasi verhaftet und direkt zum Leiter der
Stasi-Abteilung in Leipzig zum Verhör gebracht. Warum er sich dem Klassenfeind
angebiedert habe, wurde er immer wieder gefragt. Herrlich wusste nicht, wie
ihm geschah. Im Laufe des Verhörs stellte sich heraus, dass die Stasi einen
Brief aus Bonn abgefangen hatte, der an Herrlichs Privatadresse gerichtet
war. In dem Brief sei angefragt worden, ob Herrlich nicht auch Erfinderschulen
in Westdeutschland durchführen wolle – Herrlich, der seit 1963 wegen seiner
Erfindertätigkeit als Geheimnisträger nicht mehr in den Westen ausreisen
durfte.

Die Geschichte klärte sich erst nach der Wende auf, als Herrlich erfuhr, was
geschehen war. Während eines Vortragsbesuchs im Deutschen Patent- und
Markenamt in München hatte Ardenne davon berichtet, dass Michael Herrlich
Erfinderschulen durchführte. Das gefiel dem stellvertretenden Generalsekretär
der Bund-Länder-Kommission für Bildungsplanung und Forschungsförderung,
Dr. Matthias Heister. Heister war 1982 Mitbegründer der Deutschen
Aktionsgemeinschaft Bildung-Erfindung-Innovation, kurz \emph{DABEI}. Ziel der
AG war und ist es, dem Abrutschen der BRD als Erfindernation
entgegenzuwirken. Das Prinzip der Erfinderschulen kam Heister wie gerufen. Der
Gewohnheit folgend schickte er seinen Brief über den Stempelautomaten seiner
Behörde, weshalb das an Herrlichs Privatadresse gerichtete Schreiben mit dem
Bundesadler versehen war. Das wiederum rief die Stasi auf den Plan.

So unangenehm die Situation für Herrlich kurz vor dem Mauerfall auch war, so
zuversichtlich stimmte sie ihn bei der Wiedervereinigung. Herrlich wusste,
dass Erfinderschulen gefragt waren. Hemmerling, der Präsident des Patentamts
Ost, übertrug Herrlich noch 1990 die Funktion der Erfinderschulorganisation,
die zuvor der Kammer der Technik und dem Patentamt vorbehalten war. Herrlich
gründete daraufhin die Deutsche Erfinder-Akademie e.V. Leipzig, deren
Präsident er auch heute noch ist.

Matthias Heister, der direkt nach der Wende damit begann, die Erfahrungen der
Erfinderschulen aufzuarbeiten, lud Herrlich 1990 zu einer DABEI-Tagung auf das
Schloss in Saarbrücken ein. Da der als Hauptredner angekündigte Oskar
Lafontaine wegen des an ihm verübten Attentats ausfiel und Herrlich dessen
Vertretung nicht ausstehen konnte, verließ er den Vortrag und steckte sich im
Freien nach alter Gewohnheit eine Zigarre an.  Nach kurzer Zeit sprach ihn ein
unbekannter älterer Herr beim Namen an und stellte sich als Prof. Erich
Häußer, Präsident des Deutschen Patentamts in München, vor. Er hatte Herrlich
anhand seiner Zigarre erkannt. Die Patentamtspräsidenten von West und Ost,
Häußer und Hemmerling, waren Duzfreunde. Sie kannten einander von Treffen der
Weltorganisation für Geistiges Eigentum und hatten, was zu Zeiten des kalten
Kriegs noch niemand wissen durfte, miteinander Brüderschaft getrunken.

Hemmerling hatte Häußer von Herrlich berichtet und ihm ein Exemplar von dessen
Dissertation geschenkt. Häußer, der als Präsident des Patentamts von 1976 bis
1995 nicht müde wurde vor dem technischen Rückstand der BRD zu warnen, setzte
sich mit der Wiedervereinigung für die Einrichtung von Erfinderschulen für
Deutschland ein und auch dafür, dass das gesamtdeutsche Patentamt nach
DDR-Vorbild dem Wirtschaftsministerium unterstellt werde. Sabine
Leutheusser-Schnarrenberg, die 1992 in der Regierung Kohl zur Bundesministerin
der Justiz berufen worden war und der damit auch das Patentamt unterstand,
lehnte das allerdings entschieden ab. Das Verhältnis zwischen ihr und ihrem
Patentamtspräsidenten galt als mindestens angespannt.

Auch der VDI konnte zu diesem Zeitpunkt nicht für das Konzept der
Erfinderschulen gewonnen werden und Michael Herrlichs Möglichkeit sein
Anliegen direkt bei Helmut Kohl vorzutragen, endete im Eklat. Noch zu
DDR-Zeiten hatte Herrlich den promovierten Ingenieur Paul Krüger im Rahmen der
KDT-Erfinderschulen ausgebildet. Für den kurzen Zeitraum von 1993 bis 1994 war
Krüger nun Bundesminister für Forschung und Technologie in der Regierung Kohl.
In dieser Funktion konnte er Herrlich ein Treffen mit dem Bundeskanzler
arrangieren. Ziel des Treffens war es, zu erwirken, dass vom Bund
eingerichtete Erfinderschulen gemeinsam mit Herrlich durchgeführt werden
konnten.  Als es zu dem Treffen kam, habe Kohl trocken gemeint: „Kommt aus dem
Osten, kann ja nichts taugen.“ Daraufhin antwortete Herrlich verärgert: „Ich
darf Sie daran erinnern, dass die friedliche Revolution nicht Ihr Werk ist,
sondern das der 70\,000 aus Leipzig. Und außerdem ist Ihre Frau Leipzigerin“.
Damit war nicht nur die Möglichkeit für Erfinderschulen vom Tisch, sondern
auch Krüger wurde als Minister abgelöst.

Die Versuche, das Konzept der Erfinderschulen nach dem Vorbild der KDT
institutionell anzubinden, verliefen im Sand. Herrlichs Akademie entwickelte
sich dennoch gut. Allein 1992 führte Herrlich etwa 15 Erfinderschulen in den
alten Bundesländern durch, beispielsweise in einem Unternehmen in Kassel, das
Faltenbälge herstellte. Der ehemalige Direktor des Europäischen Patentamtes,
Beath, empfahl Herrlich nach St. Gallen in die Schweiz. 1993/4 veranstaltete
Herrlich Schulungen in der Schweiz und in Österreich.

Laut Herrlich wurden durch seine Akademie bis heute in der Nachwendezeit über
14\,000 Ingenieure, Naturwissenschaftler, Lehrer, aber auch begabte Studenten
und Gymnasiasten erfindermethodisch qualifiziert. Bereits ein Jahr nach den
fünfphasigen Erfinderseminaren meldeten Herrlich zufolge 23\% der Absolventen
niveauvolle Patente an, die zu über 80\% erteilt wurden. Vergleichsweise
würden bei den Anmeldungen im Land insgesamt nur 22,8\% erteilt
werden.

Herrlichs Erfinderseminare mit bis zu 10 Teilnehmern kosten pauschal 6\,000
Euro plus Reise- und Hotelspesen für die Trainer.
\begin{itemize}\itemsep0pt
\item 
Im ersten Zweitagesseminar erhalten die Teilnehmer einen Überblick über die
Erfindermethodik und formulieren die konkreten erfinderischen Ziele ihres
Projekts.
\item 
Die anschließende Selbstarbeitsphase besteht im eigenständigen Patent- und
Literaturstudium.
\item 
Im zweiten Zweitagesseminar werden offene Fragen zur Methodik geklärt und der
Entwurf der Patentschrift angefertigt.
\item 
Anschließend wird ein Rechercheantrag beim Deutschen Patent- und Markenamt
gestellt.
\item 
Das dritte Zweitagesseminar dient schließlich der Korrektur der vom Patentamt
festgestellten Vorbehalte, der Erarbeitung der endgültigen Patentschrift und
der Umsetzungskonzeption.
\end{itemize}


\ccnotice
\end{document}
