\documentclass[11pt,a4paper]{article}
\usepackage{od}
\usepackage[utf8]{inputenc}

\title{About the „Exalted and the Beatiful after the Fall“}

\author{Text compiled and translated by Hans-Gert Gr\"abe, Leipzig}
\date{July 9, 2019}

\begin{document}
\maketitle

The discussion was compiled from a thread in the post
\url{https://www.facebook.com/dmitriy.bakhturin/posts/2837979666275007} in the
Facebook account of Dmitri Bakhturin.

\section*{Background}

to be added. 

\section*{Excerpt from Podorog's Lecture}

to be added.

\section*{The Original Discussion}
Dmitriy Bakhturin Вчера имел честь представлять визионерскую лекцию
В.А.Подороги на Острове 1022. Тема «Общее чувство. Возвышенное. После
падения».  Докладчиком были развернуты важнейшие для современного
интеллектуала понятия и линии их эволюции - Страх и Прекрасное, их взлет и
падение (от Берка к Канту, далее через Бодлера к Адорно и далее). В завершении
-- логично -- зашел разговор о ситуации в «актуальном» (современном) искусстве
(читай -- творчестве, читай далее -- подстрочник -- в проектировании). Базовое
разделение, предложенное ВА -- быть «захваченным искусством» vs «владеть
искусством». Быть играемым или играть...  Спасибо Андрей Силинг (Andrey
Siling), Василий Буров (Vasiliy Burov) за риск и поддержку этого эксперимента,
и огромная благодарность великолепной Ольга Форись (Olga Foris) за
сопровождение и заботу.

Hans-Gert Gräbe Я прослушал это. Почему философы сегодня являются такими
агностиками? Подорога призывает к предсказанию цифрового будущего. Почему он
сам так слабо в этом бизнесе? И он не одинок в этом, в Германии несколько лет
назад было так же спорно с вмешательством Юргена Миттельштасса («Интернет или
прекрасный новый мир Леонардо»). Почему нет анализа, похоже по уровне мышления
как в Георг Лукач: Разрушение разума (со времен до Адорно) -- со всеми
трудностями тоже его подхода? А как насчет современной живописи, например, Нео
Рауха? Google быстро показывает, что я имею в виду. 

Dmitriy Bakhturin Агностика -- это форма знания. И он, в этом смысле,
марксист, указывая на кризис, 4й этап развития. А цифра -- да, не его. Он так
и сказал -- хотя бы книжку какую путную почитать про это..

Hans-Gert Gräbe «Агностицизм - это мировоззрение, которое подчеркивает, в
частности, основные ограничения человеческого знания, понимания и
осознания». Так по крайней мере, немецкая
Википедия\footnote{\url{https://de.wikipedia.org/wiki/Agnostizismus}.}.
Извини, если я не выразил себя достаточно точно. В частности, я не уверен, что
«цифровое будущее» правильный перевод английских терминов digital future,
digital era or digital change. Я согласен, что стоит «хотя бы книжку какую
путную почитать» «про это». История идей, однако, не заканчивается Адорно и
Хабермасом, может быть, стоит также всглянуть в книги Joseph Weizenbaum или
Donna Haraway? Даже почтёному философу?

\section{English Translation}

Dmitriy Bakhturin Yesterday I had the honor to present a visionary lecture by
V.A. Podorogi on the \emph{Island 1022}\footnote{As far as I understand, a
  famous Russian televisor channel promoting intellectual discussions.}. The
theme is \emph{General feeling.  Exalted. After the fall.} The speaker traced
the most important concepts and lines for their modern intellectual evolution
-- Fear and the Beautiful, their rise and fall (from Burke to Kant, further
via Baudelaire to Adorno and beyond). At the end -- logically -- I started
talking about situations in the „current“ (modern) art (read -- creativity,
read hereinafter -- interlinear -- in design). The basic separation proposed
by V.A. Podorogi is to be „captured by art“ vs „own art“. Be playable or play
... Thanks Andrey Siling, Vasily Burov for the risk and support of this
experiment, and many thanks to the great Olga Foris for escort and care.

Hans-Gert Gräbe I listened to this. Why are philosophers today so agnostic?
Podorogi calls for a prediction of the digital future. Why is he himself so
weak in this business? And he is not alone in this, in Germany several years
ago was a controversial about the intervention of Jürgen Mittelstass
(„Internet or Leonardo’s wonderful new world.”) Why there is no analysis,
similar in level of thinking as in Georg Lukácz: Destruction of the mind (a
text from the time before Adorno) -- with all difficulties also in his
approach? What about modern painting, for example, Neo Rauch? Google quickly
shows what I mean.

Dmitriy Bakhturin Agnostic is a form of knowledge. And he, in this sense, is a
Marxist, pointing to the crisis, 4th stage of development. And cipher -- yes,
not his. He said so -- at least a worthy book to read about it ..

Hans-Gert Gräbe "Agnosticism is a worldview that emphasizes in particular, the
main limitations of human knowledge, understanding and awareness." So at least
the German
Wikipedia\footnote{\url{https://de.wikipedia.org/wiki/Agnostizismus}.}.  I'm
sorry if I did not express myself accurately enough. In particular, I’m not
sure if "цифровое будущее" (cipher future) is the right translation of the
English terms digital future, digital era or digital change. I agree that it’s
worth “at least some worthwhile book to read” “about this”. The history of
ideas, however, does not end with Adorno and Habermas, maybe it’s also worth
to take a look at the books of Joseph Weizenbaum or Donna Haraway? Even for
the venerable philosopher?

\ccnotice

\end{document}
