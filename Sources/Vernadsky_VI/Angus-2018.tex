\documentclass[11pt,a4paper]{article}
\usepackage{od}
\usepackage[utf8]{inputenc}
\usepackage[english]{babel}


\title{Vladimir Vernadsky and the Disruption of the Biosphere}

\author{Ian Agnus}

\date{June 5, 2018}

\begin{document}
\maketitle
  \begin{quote}
  Source: \url{https://climateandcapitalism.com}, 2018/06/05.
  \end{quote}

  \begin{quote}\it\bf 
    Virtually unknown in the west, the great Russian geologist and geochemist
    Vladimir Vernadsky, 1863-1945, pioneered scientific study of life’s impact
    on the Earth.
  \end{quote}
As we’ve seen, metabolism, a defining feature of all life, always involves
exchanges with the world outside the organism. Life cannot exist without
ingesting matter and excreting waste. The fact that the Earth is a sphere
surrounded by a vacuum, and that we have access only to its outer few
kilometers, means that the amount of matter available for life to use is
finite, and that life’s wastes have nowhere else to go.

If metabolisms were \emph{linear}, if inputs were simply consumed, the
nutrients needed by living organisms would soon be depleted. Plants could
consume all the carbon dioxide in the atmosphere in about 8\,000 years, and
all the nitrogen in a million years. Life has lasted far longer than that
because its support systems are \emph{circular}. Vast recycling operations
endlessly reprocess and reuse essential elements and compounds. Radical
biologist Barry Commoner described the Biosphere as “a closed, circular
system, [in which] there is no such thing as ‘waste’; everything that is
produced in one part of the cycle ‘goes somewhere’ and is used in a later
step.” \cite[p. 10]{1}

This permanent recycling regime would be impossible if Earth were a totally
closed system — because, as the second law of thermodynamics says, totally
closed systems eventually run down. Entropy (disorder) increases until
everything comes to a stop. Fortunately for us, while the Earth System is
closed to \emph{external matter} apart from occasional meteorites, it is open
to \emph{external energy}\footnote{A physicist would say that the Earth System
  is \emph{closed} but not \emph{isolated}.}. The constant inflow of light and
other radiation from the Sun, combined with the existence of organisms that
can convert solar energy into chemical energy, makes endless recycling — and
thus all life — possible. All biogeochemical cycles are ultimately powered by
photosynthesis.

The nineteenth century scientists who discovered the circular metabolic
processes that make life possible tended to view them as local or regional.
The idea of \emph{global} metabolism wasn’t even considered until the
twentieth century — and even then it was a minority view until almost the
twenty-first.

\section*{Vladimir Vernadsky}

The first scientist to undertake a serious study of the dynamic relationship
between life and the Earth as a whole was the Russian geochemist Vladimir
Ivanovich Vernadsky. Born in 1863 and educated in St. Petersburg, Munich and
Paris, by 1900 he was well-known both as a geologist and as a liberal opponent
of Tsarist autocracy. A founder of the Constitutional Democratic (Kadet) Party
and member of its central committee for many years, he represented the
universities’ constituency in the Duma (Parliament) from 1906 to 1911, when he
resigned to protest government attacks on academic freedom. In 1915, he
founded the Commission for the Study of the Natural Productive Forces of
Russia (KEPS), to identify sources of strategic raw materials: its work
continued under the Soviet government until 1930. Although he opposed the
Bolshevik revolution, he resigned from the Kadets when the party supported
military action against the new government. After the Civil War, he returned
to Petrograd and resumed his position as head of the Academy of Sciences.

In the early 1930s, Vernadsky criticized the government’s takeover of
scientific institutions, and objected to attempts to impose dialectical
materialism as an official and mandatory philosophy. He frequently intervened
privately to aid scholars who faced official censorship or persecution. But
for the most part he refrained from publicly opposing Stalin’s policies, to
avoid endangering his scientific work. He wasn’t a Marxist, but he was a
Russian patriot, eager to contribute to the country’s development, and that
probably saved him from the fate of many other scientists in the purges. As
his biographer notes, “it was not uncommon for Stalinists to worry more about
Marxists with whom they disagreed and whom they distrusted, than they did
about non-Marxists who worked loyally for the regime, did not intrigue, and
were no real threat to Stalin’s position.” \cite[p. 167]{3}

\section*{The Biosphere}

In 1922, while studying and teaching in Paris, Vernadsky wrote “A plea for the
establishment of a biogeochemical laboratory,” and sent it to scientific
bodies in Europe and the United States, hoping to get international funding,
but only the Soviet government responded positively\footnote{His proposal was
  rejected by, among others, the British Association for the Advancement of
  Science, the U.S. National Research Council, and the Carnegie Institution.}.
He established his laboratory — really a small research institute — in
Leningrad in 1926.

Vernadsky’s focus on biogeochemistry — he created both the word and the
science — reflected his conviction that the composition and principal
characteristics of our planet could not be explained by geology and chemistry
alone. “I realized,” he later wrote, “that the basis of geology lies in the
chemical element — in the atom — and that living organisms play a prominent
role, perhaps the leading one, in our natural environment — the biosphere.”
Quoted in \cite[p. 185]{5}.

He summarized his views in 1926 in the pathbreaking book \emph{Biosfera} (The
Biosphere). Geologists had long recognized the existence of three “envelopes”
surrounding the Earth’s crust — atmosphere (air), hydrosphere (water), and
lithosphere (soil and rock). The biosphere was a fourth, “a specific,
life-saturated envelope of the Earth’s crust,” comprising all living matter on
Earth, and all parts of Earth where life exists, from the crust to the upper
atmosphere \cite[p. 91]{6}\footnote{As Vernadsky pointed out, the Austrian
  geologist Edward Suess introduced the word biosphere in his popular 1885
  textbook \emph{The Face of the Earth}.}. His argument was revolutionary in
two major respects: it treated the \emph{entire planet} as an object of study,
and it identified \emph{life itself} as a major factor in creating shaping the
planet.
\begin{quote}
  “No chemical force on Earth is more constant than living organisms taken in
  aggregate, none is more powerful in the long run. The more we learn, the
  more convinced we become that biospheric chemical phenomena never occur
  independent of life.  …

  “Life is, thus, potently and continuously disturbing the chemical inertia on
  the surface of our planet. It creates the colors and forms of nature, the
  associations of animals and plants, and the creative labor of civilized
  humanity. And also becomes a part of the diverse chemical processes of the
  Earth’s crust. There is no substantial chemical equilibrium on the crust in
  which the influence of life is not evident, and in which chemistry does not
  display life’s work.” \cite[p. 56-58]{6}
\end{quote}
He described organisms as “transformers” that use solar energy to power their
metabolic relationships with the rest of the planet. “This transformation of
energy can be considered as a property of living matter, its function in the
biosphere.”
\begin{quote}
  “The radiations that pour upon the Earth cause the biosphere to take on
  properties unknown to lifeless planetary surfaces, and thus transform the
  face of the Earth. Activated by radiation, the matter of the biosphere
  collects and redistributes solar energy, and converts it ultimately into
  free energy capable of doing work on Earth.

  “The outer layer of the Earth must, therefore, not be considered as a region
  of matter alone but also as a region of energy and a source of
  transformation of the planet. To a great extent, exogenous cosmic forces
  shape the face of the Earth, and as a result, the biosphere differs
  historically from other parts of the planet. This biosphere plays an
  extraordinary planetary role.

  “The biosphere is at least as much a \emph{creation of the sun} as a result
  of terrestrial processes.” \cite[p. 59, 44]{6}
\end{quote}
He identified recycling as a central feature of global metabolism.
\begin{quote}
  “The biosphere’s $10^{20}$ to $10^{21}$ grams of living matter is
  incessantly moving, decomposing, and reforming. The chief factor in this
  process is not growth, but multiplication. New generations, born at
  intervals ranging from tens of minutes to hundreds of years, renew the
  substances that have been incorporated into life.

  “Because enormous amounts of living matter are created and decomposed every
  24 hours, the quantity which exists at any moment is but an insignificant
  fraction of the total in a year.“

  “It is hard for the mind to grasp the colossal amounts of living matter that
  are created, and that decomposed, each day, in a vast dynamic equilibrium of
  death, birth, metabolism, and growth.” \cite[p. 72]{6}
\end{quote}
In a 1938 article, he described the intimate connection of living organisms
with their environments through metabolic processes.
\begin{quote}
  “Living organisms are connected with the biosphere through their nutrition,
  breathing, reproduction, metabolism. This connection may be precisely and
  fully expressed quantitatively by the migration of atoms from the biosphere
  to the living organism and back again — the biogenic migration of atoms. …
  There is no natural phenomenon in the biosphere more geologically powerful
  than life… .“

  “Between the living and inert matter of the biosphere, there is a single,
  continuous material and energetic connection, which is continuously
  maintained during the processes of respiration, feeding, and reproduction of
  living matter, and is necessary for its survival: \emph{the biogenic
    migration of atoms} of the chemical elements, from the inert bodies of the
  biosphere into the living natural bodies and back again.” \cite[p. 39,
    50]{10}
\end{quote}
Until his death in 1945, Vernadsky and his co-workers conducted cutting-edge
research on the composition and dynamics of the biosphere. A recently
translated selection of papers he wrote in that period includes articles on
the oxygen and carbon cycles, the organic origins of coal and petroleum, the
sources of atmospheric carbon dioxide, and more. I was particularly struck by
one that showed that “the Earth’s atmosphere itself, consisting primarily of
oxygen, nitrogen, and carbon, is the creation of life” \cite{11}. In these
areas and others, Vernadsky’s work was well ahead of science in other
countries.

\section*{Humans and the Biosphere}

Vernadsky insisted that biogeochemistry was not concerned with life as such.
Science could not explain life, so discussions of it tended to be “permeated
with philosophical and religious concepts alien to science”
\cite[p. 51]{6}\footnote{The science of life has made major advances since
  Vernadsky’s time, but non-scientific influences remain.}.  Nor did
biogeochemistry study individual organisms: that was the domain of biology.
Biogeochemistry addressed \emph{planetary} questions, so its concern was with
the planetary impact of “living matter as a whole — the totality of living
organisms” \cite[p. 58]{6}.

He did not, however, adopt the artificial holism that is sometimes invoked as
an alternative to dualism. As the research topics listed above show, Vernadsky
was fully aware of the need to investigate parts of the biosphere in order to
build a picture of the whole. He was certainly aware that many planetary
cycles can’t be understood without knowledge of the differing metabolisms of
the species involved — for example, his work in the 1930s included
consideration of the different planetary impacts of autotrophs (organisms that
live by photosynthesis) and heterotrophs (organisms that live by directly or
indirectly consuming autotrophs).

Above all, he was very aware of the unique biospheric impact of one particular
species: \emph{homo sapiens}.

Long before he developed his views on the biosphere, Vernadsky’s practical
work as a geologist made him aware of the destructive effects of extractive
industries on the environment. In 1913, for example, after visiting the nickel
and cobalt mines in Sudbury, Ontario, he wrote home to his wife:
\begin{quote}
  “This new technology — American technology — which has given so much to
  mankind, has its dark side. Here we see it in everything: a beautiful land
  has been made ugly, the forest burned out; for tens of miles the land turned
  into a wasteland, all plant life poisoned and burned out, and all of this in
  order to achieve a single goal: the quick mining of nickel” \cite{14}.
\end{quote}
After the revolution, he and two of his former students convinced the
Bolshevik government to ban mining and other commercial activity in a
geologically significant region of the southern Urals. On May 4, 1920, Lenin
signed a decree establishing that area as the first territory anywhere in the
world to be protected for scientific study \cite[p. 29]{15}.

In the 1920s, Vernadsky began to consider whether intelligent matter (humans)
might be overwhelming the impact of the rest of living matter. In his 1926
book \emph{The Biosphere}, he noted that human intelligence had enabled the
species to “reach places that are inaccessible to any other living organisms“,
which made it difficult to determine what the limits of the biosphere might be
\cite[p. 142]{6}\footnote{Since Vernadsky’s time, it has become clear that
  living matter exists virtually everywhere on Earth, including in places that
  humans cannot reach.}. What’s more, humanity was making unprecedented
changes in the “film of life” that covers the land.
\begin{quote}
  “Civilized humanity has introduced changes into the structure of the film on
  land which have no parallel in the hydrosphere. These changes are a new
  phenomenon in geological history, and have chemical effects yet to be
  determined. One of the principal changes is the systematic destruction
  during human history of forests, the most powerful parts of the film”
  \cite[p. 143]{6}.
\end{quote}
More research into biogeochemical cycles made it evident that economic
activity was changing the global metabolism in measurable ways. This passage,
from an essay Vernadsky wrote in the 1930s on the carbon cycle, has a very
modern feel.
\begin{quote}  
  “The release of carbonic acid [carbon dioxide] by Man in the process of his
  technical work is considered biogenic, such as the release occurring in
  factory furnaces, calcinating lime, fermentation, and in many other
  processes.  It is a very interesting and characteristic fact in the history
  of carbon that the quantity of carbonic acid released by mankind in this way
  increases with the progress of civilization. It has already reached such an
  order that it must be taken into account in the geochemical history of the
  biosphere.“

  “Thus, according to A. Krogh’s calculations, the quantity of carbonic acid
  released by the consumption of coal reached $7\cdot 10^8$ tons in 1904, and
  rose to $1\cdot 10^9$ tons in 1919 (F. Clarke). This amounts to as much as
  0.05\% of the entire mass of carbonic acid existing in the atmosphere. Such
  an increase acquires the status of an important geochemical phenomenon. In
  this way, civilized Man breaks the established terrestrial balance. With the
  civilization of \emph{Homo sapiens}, a new geological power has appeared…”
  \cite[p. 185-186]{11}\footnote{Vernadsky was familiar with Arrhenius’ work
    on the greenhouse effect, but wasn’t convinced that changes in CO$_2$
    levels could have major impacts on climate.}.
\end{quote}

\section*{Entering the Noösphere}

In the 1930s, Vernadsky concluded human activity was creating a new planetary
envelope that he dubbed the \emph{Noösphere} (pronounced no-osphere), from
\emph{nous}, the ancient Greek word for \emph{mind} or \emph{intelligence}. He
borrowed the word from Pierre Teilhard de Chardin, a Jesuit priest and
geologist he met in the 1920s in Paris.

That borrowing has been a source of confusion, since the two men defined the
word in radically differently ways. Teilhard, a Catholic mystic, defined the
Noösphere as the spiritual realm that humanity would achieve when it evolved
out of the material world, out of the biosphere — the “omega point” where
humans would meet Christ. Vernadsky, an atheist and materialist (he called
himself a “cosmic realist”) viewed the Noösphere as the part of the Biosphere
that was being physically transformed by human activity. So it’s important,
when the word appears, to determine which version the writer means, or if the
writer is even aware of the deep difference.

Vernadsky’s most complete account of the Noösphere was a chapter in his
unfinished book \emph{Scientific Thought as a Planetary Phenomenon}. The new
envelope, he wrote, began to take form with the invention of agriculture,
which “radically transforms nature … clearing the land from other living
organisms.”
\begin{quote}  
  “You might say that within the last five to seven thousand years the
  continuous creation of the Noösphere has proceeded apace, ever increasing in
  tempo, and that the increase of the cultural biogeochemical energy of
  mankind is advancing steadily without fundamental regression, albeit with
  interruptions continually diminishing in duration. There is a growing
  understanding that this increase has no insurmountable limits, that it is an
  elemental geological process” \cite[p. 27-28]{19}.
\end{quote}
Vernadsky strongly believed in evolution as an inevitable and progressive
advance to a better future, that any negative side-effects caused by the
expansion of the Noösphere would be overcome by human intelligence. It was
already having positive social effects.
\begin{quote}  
  “Profound social changes, giving support to the broad masses, advanced their
  interests into the first rank, and the question of eliminating malnutrition
  and famine, became a realistic option that can no longer be ignored.“

  “The question of a planned unified activity for the mastery of nature and a
  just distribution of wealth associated with a consciousness of the unity and
  equality of all peoples, the unity of the noösphere, became the order of the
  day.” \cite[p. 30]{19}
\end{quote}
In one of his last articles, one of the few published in English during his
lifetime, he wrote that in modern times, human economic activity was literally
changing the chemical composition of the biosphere.
\begin{quote}  
  “That mineralogical rarity, native iron, is now being produced by the
  billions of tons. Native aluminum, which never before existed on our planet,
  is now produced in any quantity. The same is true with regard to the
  countless number of artificial chemical combinations (biogenic ‘cultural’
  minerals) newly created on our planet. The number of such artificial
  minerals is constantly increasing. All of the strategic raw materials belong
  here. Chemically, the face of our planet, the biosphere, is being sharply
  changed by man.” \cite[p. 9]{21}
\end{quote}
He described the Noösphere in terms that sound very like 21st century
discussions of the Anthropocene.
\begin{quote}  
  “Proceeding from the notion of the geological role of man, the geologist
  A. P. Pavlov [1854-1929] in the last years of his life used to speak of the
  anthropogenic era in which we now live … He rightly emphasized that man,
  under our very eyes, is becoming a mighty and ever-growing geological force
  … In the twentieth century, man, for the first time in the history of the
  earth, knew and embraced the whole biosphere, completed the geographic map
  of the planet Earth, and colonized its whole surface.” \cite[p. 8]{21}
\end{quote}
The Noösphere would be “the last of many stages in the evolution of the
biosphere in geological history.” For him, progressive geological evolution
and the democratic fight against Nazi barbarism were related.
\begin{quote}  
  “Now we live in the period of a new geological evolutionary change in the
  biosphere. We are entering the noösphere. This new elemental geological
  process is taking place at a stormy time, in the epoch of a destructive
  world war. But the important fact is that our democratic ideals are in tune
  with the elemental geological processes, with the laws of nature, and with
  the noösphere. Therefore we may face the future with confidence. It is in
  our hands. We will not let it go.” \cite[p. 10]{21}
\end{quote}

\section*{Influence}

There are obvious parallels between Vernadsky’s view that human activity was
transforming the Biosphere into the Noösphere and the current view that human
activity has so changed the Earth System that a new geological epoch has
begun.  His description of humanity’s impact on the biosphere could fit easily
into any modern account of the profound disruption of biogeochemical cycles —
in fact, of metabolic rifts.
\begin{quote}  
  “Man always increases the number of atoms leaving the ancient cycles — the
  geochemical ‘eternal’ cycles. He intensifies the breach of these processes,
  introduces new ones, and interferes with old ones. With Man, an enormous
  geological power has appeared on the surface of our planet. The balance of
  the migrations of elements that had been established in the course of
  geological time is being broken by the reason and activities of Man. At
  present we are changing the thermodynamic equilibrium inside the biosphere
  in this way.” \cite[p. 124]{11}
\end{quote}
We should not overstate the similarities. The research that defines Earth
System science, including studies of global biogeochemical cycles, didn’t even
begin until years after Vernadsky’s death. What’s more, as Clive Hamilton and
Jacques Grinevald point out, he saw the Noösphere as the inevitable and
progressive evolution of the Biosphere, while the Anthropocene represents “a
very unwelcome rupture … a radical breakdown of any idea of advance to a
higher stage.” \cite[p. 9]{25}\footnote{I would add that the Noösphere is a
  region of space, the part of the biosphere changed by humans, while the
  Anthropocene is the time when human influences are dominant.}

More practically, Vernadsky’s influence on the development of Earth System
science was limited because until recently his work was virtually unknown
outside of the Soviet Union. When he died, fewer than half a dozen of his
articles had been translated into English, and only a handful more into French
or German. A full translation of \emph{The Biosphere} wasn’t published until
1997. Even in the Soviet Union, most of his work was unavailable until the
publication of his \emph{Selected Works} in 1967.

In 1970, the influential magazine \emph{Scientific American} published a
special issue on the Biosphere, edited by George Evelyn Hutchinson, a Yale
professor who is often called the father of modern ecology. His introductory
article provided an overview of biospheric science, incorporating recent
advances and fully crediting Vernadsky as originator of the field. He
concluded by arguing that Vernadsky’s positive view of the Noösphere is
difficult to maintain now that growing environmental crises are threatening
the very survival of the Biosphere.

The \emph{Scientific American} article generated new interest in Vernadsky’s
work, but its impact was limited, particularly because so little of his work
is available in languages other than Russian. Perhaps publishers and
translators don’t think his thoroughly interdisciplinary works will sell in
western academia, where geologists study geology and biologists study biology
and the twain never meet. As Jacques Grineveld writes, “The revolutionary
character of the Vernadskian science of the Biosphere was long hidden by the
reductionist, overspecialized and compartmentalized scientific knowledge of
our time.” \cite{26}

Although Vernadsky’s work didn’t directly influence the development of Earth
System science, it remains important as an alternative materialist approach to
understanding the relationships between life and planet. Seven decades after
his death, Vernadsky’s insights into the nature and development of the
biosphere can still illuminate our efforts to understand global metabolism —
and global metabolic rifts.

\begin{thebibliography}{xxx}
\bibitem{1} Barry Commoner. {\em Making Peace With the Planet}. New York: New
  Press, 1992.
\bibitem{3} Kendall E. Bailes. {\em Science and Russian Culture in an Age of
  Revolutions: V.I. Vernadsky and His Scientific School, 1863-1945}.
  Bloomington: Indiana University Press, 1990.
\bibitem{5} Bailes. {\em Science and Russian Culture}. 
\bibitem{6} Vladimir I. Vernadsky. {\em The Biosphere}. Trans. by David
  Langmuir and Mark McMenamin. New York: Springer, 1998 [1926].
\bibitem{10} Jason Ross, ed. {\em 150 Years of Vernadsky, Volume 1: The
  Biosphere}. Leesberg VA: 21st Century Science Associates, 2014.
\bibitem{11} Vladimir I. Vernadsky. {\em Geochemistry and the Biosphere}.
  Ed. Frank B. Salisbury. Santa Fe: Synergetic Press, 2007.
\bibitem{14} Vernadsky to Vernadskaia, May 1913. Quoted in Bailes. {\em
  Science and Russian Culture}, 127. Later observers compared the landscape
  around Sudbury to the surface of the moon.
\bibitem{15} Douglas R. Weiner. {\em Models of Nature: Ecology, Conservation
  and Cultural Revolution in Soviet Russia}. Pittsburgh: University of
  Pittsburgh Press, 1988.
\bibitem{19} Vladimir Vernadsky. {\em The Transition from the Biosphere to the
  Noosphere}. Trans. William Jones. 21st Century, Spring-Summer 2012.
\bibitem{21} Vernadsky. The Biosphere and the Noösphere. {\em American
  Scientist}, January 1945.
\bibitem{25} Clive Hamilton and Jacques Grinevald. “Was the Anthropocene
  anticipated?” {\em The Anthropocene Review}, Vol 2, No. 1, April 2015.
\bibitem{26} Jacques Grinevald. “Introduction: The Invisibility of the
  Vernadskian Revolution.” In \cite[p. 27]{6}.
\end{thebibliography}
	
\section*{Comments}

Gary Severson, June 5, 2018

Thanks for the reply. I basically agree with your points. I would say though,
that it appears that Ray Lindeman did measurements of organic/inorganic energy
exchange before anyone else including Vernadsky proving Vernadsky’s theories
of a wholistic noosphere. I am not a biologist but an amateur historian so you
would know better if my statement about Lindeman being the first to precisely
measure energy exchange is accurate.
    
Ian Angus, June 13, 2018

Lindeman may have been the first in the U.S. to work out the mathematics of
trophic levels, but Vladimir Stanchinskii in the Soviet Union was ahead of him
by some twenty years. Here is the relevant passage from Frank Benjamin
Golley’s fairly definitive, {\em A History of the Ecosystem Concept in
  Ecology}: [link added]

Gary Severson June 14, 2018

Thanks Ian. I just ordered the book. As I mentioned before I lived a couple
miles from Lindeman’s boyhood farm for 15 years here in Minnesota. I had no
idea of his existence at the time. As a Marxist I am very interested in the
early advances in Soviet ecology. Who knew?

Gary Severson June 5, 2018

This article says “In 1970, the influential magazine Scientific American
published a special issue on the Biosphere, edited by George Evelyn
Hutchinson, a Yale professor who is often called the father of modern ecology.
His introductory article provided an overview of biospheric science,
incorporating recent advances and fully crediting Vernadsky as originator of
the field.”

Apparently, the author is unaware that G.E. Hutchinson didn’t only arrive at
Vernadsky’s research in 1970. In fact, by 1942 Hutchinson had already
discovered him. Vernadsky’s son was at Yale U. as a professor and
Dr. Hutchinson had become aware of “Biosphere”. Further, Hutchinson had a
27-year-old graduate doctoral student, Ray Lindeman, that had spent 5 years,
1936-41, measuring the metabolism of a Minnesota lake. These were the first
precise long-term metabolic measurements made by anyone of a natural system.
Lindeman died in 1942 at age 28, of a genetic liver ailment but today his
doctoral dissertation is required reading by for graduate biology students
worldwide. Lindeman grew up on a Minnesota farm in Redwood County about 100
miles southwest of Minneapolis. A new book about the “New Conservation” edited
by Prof. Anthony Amato from Southwest State Univ. in Marshall, Mn. contains a
chapter about Lindeman’s research. The lake he studied in 1936-41, 20 miles
north of the St. Paul campus of the U. M., is now a research center named
after him.

A 2010 biography of Hutchinson by Nancy G. Slack contains many references to
Hutchinson’s grad student, Ray Lindeman.

Ian Angus June 5, 2018

Gary, thank you for this.

I didn’t suggest that Hutchinson only learned of Vernadsky in 1970. In fact,
my article quotes from an article by Vernadsky that his son George translated
and that was published, with an introduction by Hutchinson, in a U.S. journal
in 1944.

If this had been an article about Hutchinson I would have included the
information you have kindly provided. I would also have discussed his earlier
connections with the English socialist scientists JBS Haldane, Joseph Needham
and Lancelot Hogben, who clearly had an influence on his ideas.

Although information about Hutchinson and his student is important in its own
right, it really didn’t fit into an article specifically about Vernadsky and
the Biosphere. It’s good to know that others are addressing related subjects.

\end{document}

