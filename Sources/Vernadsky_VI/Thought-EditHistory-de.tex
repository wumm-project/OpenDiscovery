\documentclass[11pt,a4paper]{article}
\usepackage{od}
\usepackage[utf8]{inputenc}
\usepackage[russian,main=ngerman]{babel}

\title{Wissenschaftliches Denken als planetares Phänomen\\ Zur
  Editionsgeschichte der 1997er Ausgabe}
\author{Übersetzt von Hans-Gert Gräbe, Leipzig}
\date{28. April 2020}
\begin{document}
\maketitle
\begin{quote}
  Original: \foreignlanguage{russian}{Научная мысль как планетное явление}.
  Anmerkung der Herausgeber.\\
  Quelle: \url{http://vernadsky.lib.ru/e-texts/archive/thought.html}
\end{quote}

\section{Inhalt} 

\subsection*{Abteilung 1: Wissenschaftliches Denken und Arbeiten als
  geologische Kraft in der Biosphäre}
\begin{itemize}
\item Kapitel 1: Mensch und Menschheit in der Biosphäre als natürlicher Teil
  ihrer lebenden Materie, als Teil ihrer Organisation. Physikalisch-chemische
  und geometrische Heterogenität der Biosphäre: der grundlegende organisierte
  Unterschied - materiell-energetisch und temporär - zwischen ihrer lebenden
  Materie und ihrer eigenen Substanz des Kosmischen. Evolution der Arten und
  Evolution der Biosphäre. Entdeckung einer neuen geologischen Kraft in der
  Biosphäre - der wissenschaftliche Gedanke der sozialen Menschlichkeit. Seine
  Manifestation ist mit der Eiszeit verbunden, in der wir leben, mit einer der
  wiederkehrenden geologischen Manifestationen in der Geschichte des Planeten,
  die durch ihre Ursache über die Erdkruste hinausgeht.

\item Kapitel 2: Darstellung des historischen Moments als geologischer
  Prozess. Die Entwicklung der Arten lebender Materie und die Entwicklung der
  Biosphäre zur Noosphäre. Diese Entwicklung lässt sich durch den Lauf der
  Weltgeschichte nicht aufhalten. Wissenschaftliches Denken und das
  menschliche Leben als seine Manifestation.

\item Kapitel 3: Die Bewegung des wissenschaftlichen Denkens im zwanzigsten
  Jahrhundert und seine Bedeutung in der geologischen Geschichte der
  Biosphäre. Ihre Hauptmerkmale sind die Explosion der wissenschaftlichen
  Kreativität, das sich wandelnde Verständnis der Grundlagen der Wirklichkeit,
  des Universums und die effektive, soziale Manifestation der Wissenschaft.
\end{itemize}

\subsection*{Abteilung 2: Zur wissenschaftlichen Wahrheit}

\begin{itemize}
\item Kapitel 4.  Die Stellung der Wissenschaft im modernen Staatssystem. 

\item Kapitel 5.  Die Unveränderlichkeit und allgemeine Gültigkeit korrekt
  hergeleiteter wissenschaftlicher Wahrheiten für jeden Menschen, jede
  Philosophie und jede Religion. Die allgemeine Gültigkeit der
  wissenschaftlichen Errungenschaften in ihrem Kompetenzbereich ist der
  Hauptunterschied zu Philosophie und Religion, deren Schlussfolgerungen
  keinen so zwingenden Charakter haben dürfen.
\end{itemize}

\subsection*{Abteilung 3: Neue wissenschaftliche Erkenntnisse und der Übergang
  der Biosphäre in die Noosphäre}

\begin{itemize}
\item Kapitel 6.  Die neuen Herausforderungen des 20. Jahrhunderts - neue
  Wissenschaften. Die Biogeochemie ist untrennbar mit der Biosphäre verbunden.

\item Kapitel 7.  Struktur des wissenschaftlichen Wissens als Ausdruck der
  durch den geologisch neuen Zustand der Biosphäre verursachten Noosphäre. Der
  historische Verlauf der planetarischen Manifestation des Homo sapiens durch
  die Schaffung einer neuen Form der kulturellen biogeochemischen Energie und
  der damit verbundenen Noosphäre.
\end{itemize}

\subsection*{Abteilung 4: Biowissenschaften im wissenschaftlichen
  Wissenssystem} 

\begin{itemize}
\item Kapitel 8: Ist das Leben eine ewige Manifestation der Wirklichkeit oder
  vorübergehend? Die natürlichen Körper der Biosphäre sind lebendig und
  kosmisch. Die komplexen natürlichen Körper der Biosphäre sind
  biokosmisch. Die Linie zwischen dem Lebendigen und dem Kosmischen ist in
  ihnen nicht unterbrochen.

\item Kapitel 9.  Eine biogeochemische Manifestation der unüberwindbaren
  Grenze zwischen den lebenden und kosmischen Naturkörpern der Biosphäre.

\item Kapitel 10.  Die Biowissenschaften sollten den physikalischen und
  chemischen Wissenschaften unter denjenigen, die sich mit der Noosphäre
  befassen, gleichgestellt werden.
\end{itemize}

\section{Anmerkung zur elektronischen Ausgabe}

Digitales Vernadsky-Archiv \url{http://vernadsky.lib.ru}

Diese elektronische Ausgabe von Vernadskys Buch „Wissenschaftliches Denken als
planetares Phänomen“ wurde seit Ende 1999 auf der Basis von
\cite{Vernadsky1991} erstellt.  Die ersten vier Kapitel des Buches wurden bis
April 2000 erstellt und sind in der Maxim-Moschkow-Bibliothek
(\url{http://lib.ru/FILOSOF/WERNADSKIJ}) erhältlich. Diese ersten Kapitel sind
sorgfältig redigiert worden und enthalten hoffentlich nur wenige Druckfehler.

Bis Ende November 2000 waren das fünfte und das sechste Kapitel des Buches
redigiert worden (ebenfalls recht sorgfältig, wenn auch schlechter als die
ersten vier Kapitel) -- sie wurden auf die Server des Elektronischen Archivs
\url{http://vernadsky.lib.ru} hochgeladen, aber nicht an die
Moschkow-Bibliothek geschickt, in der Hoffnung, dass die verbleibenden vier
Kapitel ebenfalls schnell vorbereitet würden.

Leider hat sich die Arbeit an den übrigen Kapiteln aus Zeitmangel immer weiter
in die Länge gezogen, so dass ich mich nun entschlossen habe, eine
elektronische Version dieser Kapitel zu verwenden, die vom Russischen Fonds
für Grundlagenforschung auf der Basis von  \cite{Vernadsky1997} 
(\url{http://elibrary.ru/books/vernadsky/obl.htm}) erstellt wurde. 

Vergleicht man diese beiden Ausgaben, so scheint es mir, dass die frühere
Ausgabe von 1991 noch viel näher am Originaltext von Vernadsky liegt. Die
Ausgabe von 1997 ist voller kleiner redaktioneller Änderungen am Text, die
zwar nirgendwo (wie es scheint) Vernadskys Idee verzerren, aber dennoch seine
Art sich auszudrücken verändern -- und oft stolpert man an solchen Stellen
einfach nur und hat sofort das Gefühl, dass Wladimir Iwanowitsch so nicht
geschrieben hätte. Daher wird es notwendig sein, die Kapitel 7-10 irgendwann
in Übereinstimmung mit der Ausgabe von 1991 zu korrigieren. Es wird auch
notwendig sein, alle Kapitel noch einmal zu redigieren und die verbleibenden
Druckfehler zu korrigieren.

Zu Beginn des Buches (dies ist hier separiert - HGG) bringe ich auch die
Notizen der Herausgeber beider Ausgaben dieses Buches, die über die
Entstehungsgeschichte des Buches berichten, sowie über die schwierige und
recht umstrittene Geschichte seiner Veröffentlichung.

Wenn Sie die elektronische Ausgabe von „Wissenschaftliches Denken als
planetares Phäno"|men“ kommerziell nutzen oder (was großartig wäre ;-) beim
Korrekturlesen helfen möchten, kontaktieren Sie mich unter der auf dem Server
angegebenen Adresse \url{http://vernadsky.lib.ru}.

Achtung: die elektronische Ausgabe ist im \LaTeX-Format vorbereitet -- beim
Korrekturlesen müssen Sie sie korrigieren, nicht die daraus abgeleitete
HTML-Version in der Maxim-Moshkov-Bibliothek.

\begin{flushright}
  Sergey Mingaleyev, 16. Oktober 2001 
\end{flushright}

\section{Vorwort und Anmerkung von A.L. Yanshin}

A.L. Yanshin, Vorsitzender der Kommission der Akademie der Wissenschaften der
UdSSR zur Aufbereitung des wissenschaftlichen Erbe des Akademikers
V. I. Vernadsky

F.T. Yanshina, Direktor und Organisator des Museums des Akademikers
V. I. Vernadsky

Elektronische Version des Vorworts und der Notizen, die für die
Veröffentlichung im Buch \cite{Vernadsky1991} vorbereitet wurden.

\subsection*{Vorwort}

Der Name Vladimir Iwanowitsch Vernadsky ist in unserem Land inzwischen weithin
bekannt. Es gibt keinen einzigen auch nur etwas gebildeten Menschen, der
nicht, wenn nicht die Werke von Vernadsky, dann zumindest zahlreiche Zeitungs-
und Zeitschriftenartikel über ihn und sein Werk gelesen hätte.

Es gibt die Vernadsky-Allee in Moskau. Seinen Namen trägt das Institut für
Geochemie und Analytische Chemie, eines der größten in der Akademie der
Wissenschaften der UdSSR. Beim Präsidium der Akademie der Wissenschaften der
UdSSR gibt es eine Kommission für die Entwicklung des wissenschaftlichen Erbes
des Akademikers Vernadsky, die ihr „Bulletin“ herausgibt. Die Abteilungen
dieser Kommission arbeiten in Leningrad und Kiew. Vernadsky-Stipendien wurden
an den Universitäten von Moskau, Leningrad, Kiew und Simferopol eingerichtet.
Öffentliche Forschungszentren zum Studium des schöpferischen Schaffens dieses
herausragenden Denkers und seiner Anwendung auf die Lösung der heutigen
Probleme gibt es in Odessa, Rostow am Don, Eriwan, Simferopol, Iwanowo und
anderen Städten der UdSSR sowie im Ausland - in Prag, Oldenburg und
Berlin\footnote{Die folgenden sind auch nach V. I. Vernadsky benannt:
  Staatliches Geologisches Museum, Allunions-Volksuniversität für
  Biosphärenwissen, Wissenschaftliche Zentralbibliothek der Akademie der
  Wissenschaften der Ukrainischen SSR, Studentisches Soziologisches Zentrum
  „Noosphäre“, Gipfel im Flusseinzugsgebiet. Podkamennaya Tunguska, ein Krater
  auf der anderen Seite des Mondes, eine Halbinsel in der Ostantarktis in der
  Nähe des Kosmonautenmeeres, ein Berg auf der Insel Paramushir (Kurilen),
  subglaziale Berge in der Ostantarktis, ein Unterwasservulkan im Atlantik,
  eine Mine im Gebiet des Baikalsees, das Mineral „Vernadit“ $[Mn^{4+},
    Fe^{3+}, Ca, NaS( O, OH)_{2n}\cdot H_2O]$, Kieselalge, das
  Forschungsschiff der Akademie der Wissenschaften der ukrainischen SSR
  „Akademiker Vernadsky“, der Dampfer der Kama-Flussschifffahrtsgesellschaft
  „Geologist Vernadsky“, dem Dorf Vernadovka in der Nähe von Simferopol,
  Vernadovka Station der Kasaner Eisenbahn, Station Ich bin die U-Bahnstation
  Vernadsky Prospekt in Moskau, das Biosphärenmuseum in der Leningrader
  Abteilung des Instituts für Geschichte der Naturwissenschaften und
  Technologie der Akademie der Wissenschaften der UdSSR. Denkmal für
  V. I. Vernadsky in Kiew installiert, Gedenktafeln am alten Gebäude der
  Moskauer Staatsuniversität. MV Lomonosov und Vernadsky Avenue in Moskau, das
  Gebäude der Leningrader Staatsuniversität sowie das Gebäude der Kiewer
  Staatsuniversität. T. G. Shevchenko. Für herausragende wissenschaftliche
  Arbeiten auf dem Gebiet der Mineralogie, Geochemie und Kosmochemie werden
  die Akademie der Wissenschaften der UdSSR und die Akademie der
  Wissenschaften der ukrainischen SSR mit Preisen ausgezeichnet.
  V. I. Vernadsky.  Eine nach ihm benannte Goldmedaille wurde von der Akademie
  der Wissenschaften der UdSSR eingerichtet.  }.

Im März 1988 feierten unser Land und das Ausland (in Prag und Berlin) den
125. Jahrestag der Geburt von Vernadsky.

Die Feierlichkeiten haben sich sehr weit entfaltet. Am 15. Januar 1988 wurde
im VDNKh UdSSR eine seinem Werk gewidmete Ausstellung eröffnet. Vom 3. bis
11. März fanden wissenschaftliche Symposien unter Beteiligung ausländischer
Wissenschaftler in verschiedenen Forschungsbereichen V. V. Makarovs statt. Vom
3. bis 11. März fanden in Leningrad, Kiew und Moskau wissenschaftliche
Symposien unter Beteiligung ausländischer Wissenschaftler in verschiedenen
Bereichen der Vernadsky-Forschung statt. An seinem Geburtstag fand am 12. März
in Moskau im Bolschoi-Theater ein feierliches Treffen unter Beteiligung
öffentlicher Organisationen statt. An denselben Tagen fanden getrennte Treffen
und wissenschaftliche Sitzungen in Iwanowo, Odessa, Simferopol und Rostow am
Don statt. Eriwan, Baku, Alma-Ata, Nowosibirsk, Irkutsk und viele andere
wissenschaftliche Zentren des Landes. Auf Initiative ausländischer und
sowjetischer Wissenschaftler wurde beschlossen, die Internationale
Vernadsky-Stiftung zu gründen, um die Übersetzung seiner Werke in andere
Sprachen zu subventionieren, in ausländischen Archiven nach Materialien über
ihn zu suchen und Wissenschaftler aus anderen Ländern für Berichte und
Vorträge über die moderne Entwicklung wissenschaftlicher Probleme in die UdSSR
einzuladen. Fast alle sowjetischen und internationalen Zeitungen und
Zeitschriften veröffentlichten Artikel über ihn und seine vielfältige
wissenschaftliche Arbeit.

Kurz vor dem Jubiläum im Februar 1988 veröffentlichte der Verlag „Nauka“
Bände mit Werken von B. V. Lomonosov. Vernadsky und seine „Briefe von
N.E. Vernadsky“, einschließlich des Buches „Philosophische Gedanken eines
Naturalisten“, in dem als erster Teil des Werkes „Wissenschaftliches Denken
als planetarisches Phänomen“ zum zweiten Mal veröffentlicht wurde, nun aber
mit der Wiederherstellung all jener Notizen, die bei der ersten Ausgabe des
Buches 1977 gemacht wurden, auf das im Archiv erhaltene Original. Das Buch
erschien in einer Auflage von 20.000 Exemplaren. Die gesamte Auflage wurde in
den ersten Tagen nach seinem Erscheinen in den Regalen der Buchhandlungen
gekauft. Der Nauka-Verlag und der Wissenschaftliche Verlagsrat der Akademie
der Wissenschaften der UdSSR erhielten zahlreiche Briefe mit der Bitte, eine
zusätzliche Ausgabe der „Philosophischen Gedanken der Naturforscher“ oder
zumindest deren ersten Teil herauszugeben.

Erschien 1988. „The Naturalist's Philosophical Thoughts“ hat in der Presse ein
breites positives Echo gefunden. So veröffentlichte die Zeitung „Iswestija“
vom 29. September 1988 den Artikel „Der unbekannte Vernadskij“, in dem ihr
Autor F. Lukjanow schrieb:

„Der Name des Akademikers Vladimir Ivanovich Vernadsky (1863-1945) kann einem
unbekannten sowjetischen Leser nicht genannt werden. Dennoch ist er in seiner
Heimat vor allem als Naturwissenschaftler und Wissenschaftshistoriker bekannt
und als Denker und Philosoph fast unbekannt, obwohl sein philosophisches Erbe
seit langem ein anerkanntes Phänomen des europäischen und weltweiten
wissenschaftlichen Denkens ist.

Vernadskys Buch „Die philosophischen Gedanken eines Naturalisten“, das gerade
im Nauka-Verlag erschienen ist, stellt ihn und unser lesendes Publikum endlich
als Philosophen und Denker vor. In der Tat handelt es sich bei diesem Buch um
die erste vollständige Veröffentlichung, ohne Banknoten, der grundlegenden
Werke des russischen Denkers, allen voran des grundlegenden Werkes
„Wissenschaftliches Denken als planetares Phänomen“, das in der Zeit von 1880
bis Anfang der 1940er Jahre entstanden ist und, ob überhaupt nicht
veröffentlicht oder längst zu einer bibliographischen Rarität geworden ist.

In Vorbereitung auf den Druck dieser Ausgabe von Vernadskys Werk
“Wissenschaftliches Denken als planetares Phänomen“ wurde der Text mit Hilfe
der Mitarbeiter des Archivs der Akademie der Wissenschaften der UdSSR erneut
sorgfältig mit dem Manuskript von S.N. Schidowinow abgeglichen, wodurch es
möglich wurde, einige kleinere Ungenauigkeiten, die in früheren Ausgaben
unbemerkt geblieben waren, zu korrigieren sowie den Stil, die Rechtschreibung
und die Interpunktion des Autors wiederherzustellen.

Was bietet das Buch dem Leser? Um diese Frage zu beantworten, ist es
notwendig, kurz auf die Entwicklung von Vernadskys Ideen einzugehen, die in
dieser Arbeit am vollständigsten zum Ausdruck kommen.

Aus Briefen an seine Frau Natalja Egorowna und an einige Wissenschaftler sowie
aus den erhaltenen Tagebüchern von Wladimir Iwanowitsch geht hervor, dass
seine Aufmerksamkeit in jungen Jahren, d.h. am Ende des letzten Jahrhunderts,
durch die zunehmende technische Macht der Menschheit erregt wurde, die vom
Umfang ihrer Tätigkeit her mit den gewaltigsten natürlichen geologischen
Prozessen vergleichbar wurde. Diese Aktivität in physikalischer,
geographischer und chemischer Hinsicht verändert das gesamte Antlitz der Erde,
ihre gesamte Natur irreversibel (Vernadsky hatte den Begriff „Biosphäre“ noch
nicht verwendet).

Solche Gedanken wurden nicht nur bei Vernadsky geboren, und er erwähnt seine
Vorgänger und Zeitgenossen in späteren Werken mit der ihm eigenen
Sensibilität. Im Jahr 1933 schlug der amerikanische Geologe Charles Schuhert
vor, die Neuzeit als den Beginn einer neuen psychozoischen Ära in der
Erdgeschichte zu betrachten und betonte mit diesem Namen die Bedeutung der
geistigen Aktivität des Menschen als geologischer Faktor
\cite[S. 80]{Schuchert1933}. Unser russischer Wissenschaftler A. P. Pawlow,
der 1890 W. I. Vernadsky einlud, Mineralogie an der Moskauer Universität zu
lehren, glaubte auch, dass mit dem Aufkommen des Menschen auf der Erde eine
neue geologische Periode ihrer Geschichte begann, die er als anthropogen (vom
griechischen Wort „anthropos“ - Mensch) zu bezeichnen schlug
\cite{Pavlov1922}. Ende des letzten und Anfang dieses Jahrhunderts gab es
weitere ähnliche Aussagen.

Zusätzlich zu seinen allgemeinen Aussagen begann Vernadsky jedoch eine mühsame
Arbeit, um das Ausmaß menschlicher Aktivitäten zu quantifizieren. Bereits in
den Kursen „Mineralogie“, die jedes Mal mit Ergänzungen während seiner Arbeit
an der Moskauer Universität (von 1891 bis 1912) veröffentlicht wurden, stellte
Vernadsky Mineralien und neue chemische Verbindungen fest, die als Ergebnis
der industriellen Tätigkeit der Menschheit entstanden, und gab erste
Schätzungen des Gesamtvolumens und des Gewichts solcher „anthropogener“
Mineralien ab.

Seit 1908 veröffentlichte er seine „Erfahrungen in der beschreibenden
Mineralogie“ in separaten Ausgaben, die später alle nativen Elemente,
einschließlich gasförmiger, sowie deren Schwefel- und Selenverbindungen
abdeckten. In diesen Ausgaben, die später in den Bänden 2 und 3 der
Ausgewählten Werke von B. gesammelt wurden. I. Vernadsky“ (1955 und 1959),
wenn er fast jedes Mineral oder ihre Gruppe beschreibt, hebt er einen
separaten Absatz „Menschliche Arbeit“ oder „Menschliche Aktivität“ hervor, in
dem er Zahlen über ihre weltweite Produktion und Verarbeitung angibt,
berichtet Daten über den direkten und indirekten Einfluss der menschlichen
Aktivität eines bestimmten Minerals oder einer chemischen Verbindung
(z.B. Schwefelwasserstoff).

In den Jahren 1933 und 1934. Vernadsky veröffentlichte die „Geschichte der
natürlichen Gewässer“ in zwei Büchern, die er als den zweiten Band der
“Geschichte der Mineralien der Erdkruste“ betrachtete. In diesem Werk widmet
er viele Seiten dem bewussten und unbewussten Einfluss menschlicher
Aktivitäten auf die geographische Verteilung und Zusammensetzung aller
Gewässer der Erde. Schon damals kam Vernadsky zu dem Schluss, dass
“jungfräuliche Flüsse rasch verschwinden oder verschwunden sind und durch eine
neue Art von Formationen ersetzt wurden, neue Gewässer, die vorher nicht
existierten. Auf dem riesigen Territorium Eurasiens und während des letzten
Jahrhunderts in Amerika und Australien werden in der gesamten Biosphäre
natürliche Gewässer verarbeitet und gleichzeitig neue kulturelle Flüsse, Seen,
Teiche, Meeresformationen an der Küste und Bodenlösungen geschaffen. „Dieser
Prozess geht in die Tiefe und verändert das Regime des Reservoirwassers in der
Biosphäre und der Stratosphäre. Dieser Prozess geht in die Tiefe, verändert
das Regime des Formationswassers in der Biosphäre und Stratosphäre. Vor
Jahrtausenden gab es eine Veränderung im Oberlauf - Grundwasser, später gab es
eine Veränderung beim Bohren und Erzbergbau des Formationsdruckwassers. Jetzt
betrifft es an einigen Stellen tiefer als zwei Kilometer von der Erdoberfläche
entfernt“. „In der gesamten Biosphäre verschwinden alte Arten von
Oberflächengewässern, Formationsgewässern, Bodengewässern und Quellen und
verändern sich, neue Kulturgewässer entstehen“\cite[S. 85]{Vernadsky1960}.

Parallel zur Untersuchung der Auswirkungen menschlicher Aktivitäten auf die
Veränderung der Natur der Erde begann Vernadsky 1914-1916 mit der Entwicklung
einer Lehre über die Biosphäre - die Hülle der Erde, die konzentrierte
“lebende Materie“ ist. Er mochte keine exzessive Wortschöpfung und die
Schaffung neuer Begriffe, aber er kannte die gesamte wissenschaftliche
Weltliteratur und verwendete deren Terminologie in großem Umfang. Und so war
auch der Begriff „Biosphäre“. Es wurde erstmals 1804 von dem französischen
Wissenschaftler Jean Baptiste Lamarck in seiner Arbeit über Hydrogeologie
verwendet, um die Menge der lebenden Organismen zu bezeichnen, die den Globus
bewohnen. Ende des 19. Jahrhunderts wurde es von dem österreichischen Geologen
Eduard Suess und dem deutschen Wissenschaftler Johan Walter verwendet, aber
wiederum in einer Weise, die dem Verständnis Lamarcks nahe kommt. Vernadsky
gab diesem Begriff eine ganz andere, viel tiefere Bedeutung. Für die
Gesamtheit der Organismen, die die Erde bewohnen, führte er den Begriff
“lebende Materie“ ein, und die Biosphäre begann, die gesamte Umwelt, in der
sich diese lebende Materie befindet, zu nennen. die gesamte Wasserhülle der
Erde, denn lebende Organismen gibt es auch in den tiefsten Tiefen der
Weltmeere, in der unteren Atmosphäre, in der Insekten, Vögel und Menschen
fliegen, und im oberen Teil der harten Schale der Erde - der Lithosphäre, in
der lebende Bakterien im Grundwasser bis zu einer Tiefe von etwa 2 km
zusammentreffen, und der Mensch mit seinen Minen in den goldhaltigen Regionen
Indiens, Südafrikas und Brasiliens ist nun in noch tiefere Tiefen
vorgedrungen, über 3 km. Die Biosphäre hat einen „Lebensfilm“, in dem die
Konzentration von lebender Materie am höchsten ist. Dabei handelt es sich um
die Landoberfläche, den Boden und die oberen Schichten der Weltmeere. Von ihr
aus nimmt die Menge an lebender Materie in der Biosphäre der Erde nach oben
und unten stark ab.

\paragraph{B.} 
Die Ergebnisse seiner Forschung B. Vernadsky hat in zahlreichen Artikeln, in
dem Buch „Biosphäre“, das 1926 zum ersten Mal veröffentlicht und dann mehrmals
neu aufgelegt wurde, und in einem grundlegenden Werk „Chemische
Zusammensetzung der Biosphäre der Erde und ihrer Umwelt“, das erstmals nach
dem Tod des Autors 1965 veröffentlicht wurde, vorgestellt. 3a des letzten
Jahrzehnts erschienen aufgrund der verstärkten Aufmerksamkeit für die Aufgaben
des Naturschutzes in der sowjetischen und ausländischen Presse zahlreiche
Artikel, Broschüren und Bücher, die sich mit Vernadskys Lehren über die
Biosphäre, seiner ausführlichen Darstellung, seinen Kommentaren und leider nur
teilweise mit seiner Entwicklung befassten. Daher erübrigt es sich, im Vorwort
zu diesem Buch ausführlicher darauf einzugehen. Es ist jedoch wichtig zu
betonen, dass Vernadsky die menschliche Aktivität zunächst als einen der
Biosphäre aufgezwungenen Prozess betrachtete, der ihr in seinem Wesen fremd
ist. Es ist anzunehmen, dass ihm diese Idee durch die vom Menschen geschaffene
Natur dieser menschlichen Tätigkeit, die in vielerlei Hinsicht gegen den
natürlichen Ablauf natürlicher Prozesse verstieß und diesen widersprach,
nahegelegt wurde.

Die „überlagerte“, fremde Natur der Natur der industriellen menschlichen
Tätigkeit kann anhand einer Reihe von Aussagen von B. beurteilt werden.
I. Vernadsky auch in seinen Werken der frühen dreißiger Jahre. So schrieb er
in der bereits erwähnten „Geschichte der natürlichen Gewässer“ über vom
Menschen geschaffene feste Mineralien und Gewässer: „Diese neuen chemischen
Verbindungen sind „künstlich“, d.h. unter Mitwirkung des Willens und des
Bewusstseins des Menschen entstanden, und können beim Studium der Geschichte
der natürlichen Körper bisher beiseite gelassen werden
\cite[S. 87]{Vernadsky1960}.

Im letzten Jahrzehnt seines Lebens begann Vernadsky jedoch zu der
unvermeidlichen Schlussfolgerung über die Entwicklung der Biosphäre der Erde
zu kommen, über quantitative und qualitative Veränderungen ihres
Hauptbestandteils - der lebenden Materie, über die Stadien der Entwicklung der
Biosphäre. Eine solche Denkweise führte ihn zu der Schlussfolgerung, dass die
Entstehung des Menschen und die Auswirkungen seiner Aktivitäten auf die
natürliche Umwelt kein Zufall, kein dem natürlichen Lauf der Dinge
„aufgezwungener“ Prozess ist, sondern eine bestimmte natürliche Stufe der
Evolution der Biosphäre. Diese Etappe sollte dazu führen, dass die Biosphäre
der Erde unter dem Einfluss des wissenschaftlichen Denkens und der kollektiven
Arbeit der vereinten Menschheit, die darauf abzielt, all ihre materiellen und
geistigen Bedürfnisse zu befriedigen, in einen neuen Zustand übergehen sollte,
den er „Noosphäre“ (vom griechischen Wort „noos“ - Geist) zu nennen vorschlug.
Dieser Begriff, wie auch der Begriff „Biosphäre“, wurde nicht von Vernadsky
selbst erfunden. In den Jahren 1922-1926 hielt er während seiner langen
Geschäftsreise ins Ausland am College de France Vorlesungen über Biogeochemie
und die Entwicklung der Biosphäre, und 1927 veröffentlichte der französische
Mathematiker Edouard Leroy als Student dieser Vorlesungen einen Artikel
darüber, in dem er zunächst den Begriff „Noosphäre“ verwendete, der später
auch von anderen französischen Wissenschaftlern und Vernadsky verwendet wurde.

Das Werk „Wissenschaftliches Denken als planetares Phänomen“ ist nach den
Tagebüchern Vernadskys und seinen Briefen hauptsächlich in den Jahren
1937-1938 entstanden, den tragischsten Jahren unserer Geschichte. Die
Ereignisse jener Tage waren Vernadsky nicht gleich"|gültig. Seine Freunde und
Schüler wurden unterdrückt. Bei dem Versuch, ihre Unschuld, den Fehler ihrer
Verhaftung zu beweisen, schrieb er Briefe an Stalin, N.I. Yeshow, L.P. Beria.
Schwere Gedanken füllten in diesen Jahren seine Tagebücher. Aber das Buch, das
er für künftige Generationen geschrieben hat, ist von Optimismus und Glauben
an den Triumph der menschlichen Vernunft durchdrungen.

Es ist schwierig, den Inhalt des Buches umfassend zu charakterisieren. Er ist
viel weiter gefasst als der Titel, obwohl die Idee der Weltbedeutung des
wissenschaftlichen Denkens ihn von Anfang bis Ende durchdringt und alle seine
Teile verbindet. Im Wesentlichen ist dieses Buch eine Einführung in die
Doktrin der Noosphäre. Es gibt viel Raum für eine Analyse des Begriffs dieses
Begriffs. Gleichzeitig wird die Rolle der Menschheit bei der Entwicklung der
Biosphäre in groben Zügen von dem großen Künstler dargestellt: das Konzept der
lebenden Materie und ihrer Organisation, die Entwicklung der Biosphäre und die
Unvermeidbarkeit ihrer allmählichen Umwandlung in die Noosphäre, die für einen
solchen Übergang notwendigen Bedingungen, die wichtigsten Entwicklungsstufen
der menschlichen Kultur und ihre weiteren Schicksale, die Biogeochemie als
wissenschaftliche Hauptrichtung der Erforschung der Biosphäre, die
grundlegenden Unterschiede zwischen den lebenden und den kosmischen Substanzen
dieser Hülle der Erde.

Einen besonderen Platz unter den Werken Vernadskys nimmt das Werk
„Wissenschaftliches Denken als planetares Phänomen“ ein. Es zeichnet sich
durch die außerordentliche Breite der in ihr behandelten Themen und die
Besonderheit des darin behandelten Hauptproblems aus. Die Werke Vernadskys
zeichnen sich seit jeher durch die Breite der Sichtweise ihres Autors auf die
Dinge und die Bedeutung des Umfangs der gestellten Fragen aus. In den
veröffentlichten Arbeiten scheinen diese Qualitäten des Wissenschaftlers
jedoch am hellsten und kraftvollsten zum Ausdruck gebracht zu werden. Die
Natur, die menschliche Gesellschaft und das wissenschaftliche Denken werden
von ihm in ihrer untrennbaren Integrität betrachtet, und die uns umgebende
Realität ist in einer wahrhaft universellen Ungeheuerlichkeit gezeichnet.

„Wissenschaftliches Denken als planetares Phänomen“ ist der Höhepunkt von
Vernadskys Werk, das grandiose Ergebnis seiner Reflexionen über das Schicksal
der wissenschaftlichen Erkenntnis, über das Verhältnis von Wissenschaft und
Philosophie, über die Zukunft der Menschheit. Man kann sie als eine
unvollendete, aber beeindruckende Synthese von Ideen charakterisieren, die von
Wissenschaftlern in der letzten Periode seines Lebens entwickelt wurden.

Das Buch enthält tiefe Gedanken über die Entwicklung der Menschheit in
geologischen und sozio-historischen Zeiträumen. Es sollte anerkannt werden,
dass dies die erste Erfahrung in der Weltliteratur ist, in der die Evolution
unseres Planeten als ein einziger kosmischer, geologischer, biogener und
anthropogener Prozess verallgemeinert wird. Das Werk offenbart die führende
transformative Rolle der Wissenschaft und der sozial organisierten Arbeit der
Menschheit in der Gegenwart und Zukunft des Planeten. Wissenschaftliches
Denken, Wissenschaft, wird als die wichtigste Kraft der Transformation und
Evolution des Planeten betrachtet und analysiert.

Es sei darauf hingewiesen, dass das den Lesern angebotene Buch einen tiefen
philosophischen Inhalt hat. Vernadsky interessiert sich nicht nur seit seiner
Jugend für Philosophie, sondern hat sich auch intensiv mit den Werken von
Philosophen aus verschiedenen Schulen und Richtungen auseinandergesetzt. Er
betrachtete die Sammlung und Synthese wissenschaftlicher Fakten als untrennbar
von der philosophischen Reflexion der erzielten Ergebnisse, was besonders
deutlich aus seinen Tagebüchern und seiner Korrespondenz hervorgeht.

Als er 1902 begann, sich mit der Geschichte der menschlichen Kultur zu
befassen, schrieb er an seine Frau Natalia Egorovna: „Ich sehe die Bedeutung
der Philosophie für die Entwicklung des Wissens ganz anders als die meisten
Naturforscher und gebe ihr einen großen, fruchtbaren Wert. Es scheint mir,
dass dies die Seiten ein und desselben Prozesses sind - absolut unvermeidlich
und untrennbar. Sie sind nur in unseren Köpfen getrennt. Wäre einer von ihnen
zum Stillstand gekommen, hätte der andere aufgehört, lebendig zu werden.
Philosophie schließt immer den Embryo ab, nimmt manchmal sogar ganze Bereiche
der zukünftigen Entwicklung der Wissenschaft vorweg, und nur durch die
gleichzeitige Arbeit des menschlichen Geistes auf diesem Gebiet ist die
richtige Kritik an den zwangsläufig schematischen Strukturen der Wissenschaft
angebracht. In der Geschichte der Entwicklung des wissenschaftlichen Denkens
lässt sich klar und genau nachvollziehen, welche Bedeutung der Philosophie als
den Wurzeln und der Lebensatmosphäre des wissenschaftlichen Strebens
zukommt \cite[S. 21]{Mikulinsky1988}. 

Vernadsky blieb den in diesem Brief dargelegten Prinzipien zeitlebens treu.
Ähnliche Aussagen finden sich in mehreren anderen seiner Briefe und Werke,
insbesondere in zahlreichen Publikationen zur Geschichte der
wissenschaftlichen Erkenntnis. Sie alle sind durchdrungen von einer
philosophischen Reflexion über das präsentierte Material.

In den Werken, Briefen und Tagebüchern der dreißiger Jahre treffen wir jedoch,
so scheint es überraschend, auf andere Aussagen Vernadskys, in denen er die
Philosophie von der wissenschaftlichen Erkenntnis trennt und sie sogar
zusammen mit der Religion erwähnt. Um dies zu verstehen, müssen wir
berücksichtigen, dass es sich in diesem Fall um die in jenen Jahren
vorherrschende Philosophie des vulgären dialektischen Materialismus handelt,
der nicht nur den Vertretern der Sozialwissenschaften, sondern auch den
Naturwissenschaftlern vorschreibt, welche Schlussfolgerungen und Schlüsse sie
für ihre vollständige Einhaltung der philosophischen „Gesetze“ ziehen
sollen. Eine solche Philosophie von V.I. Vernadsky konnte nicht akzeptieren,
wofür er von A.M. Deborin kritisiert wurde, der ihm Idealismus vorwarf
\cite[S. 543-569]{Deborin1932}.  V.I. Vernadsky reagierte mit großer Würde auf
diese Kritik, obwohl sie sein Ego verletzte \cite[S. 395-407]{Vernadsky1933}.
Er war immer der Meinung, dass jede Studie auf der unparteiischen Sammlung
möglichst vieler Fakten zum untersuchten Thema, dann auf einer objektiven
Verallgemeinerung dieser Fakten und erst dann auf philosophischer Reflexion
beruhen sollte. Übrigens behandelte Vernadsky Karl Marx als Wissenschaftler
mit tiefem Respekt, gerade weil das „Kapital“ auf einer riesigen Menge
sorgfältig und gewissenhaft gesammelten Faktenmaterials beruhte.

Weitere Einzelheiten zur Entwicklung philosophischer Auffassungen von
V.I. Vernadsky werden in einem redaktionellen Artikel in dem erwähnten Buch
„Philosophische Gedanken eines Naturforschers“ dargelegt. Im gleichen Buch ist
ein ausführlicher Kommentar zum Thema „Wissenschaftliches Denken als
planetares Phänomen“ veröffentlicht sowie, in Form eines Anhangs, Artikel von
B.M. Kedrow, I.W. Kusnezow, S.R. Mikulinski und A.L. Janschin, die zu
verschiedenen Zeiten geschrieben wurden und die aus verschiedenen Perspektiven
die Fragen von Vernadskys Weltbild und seine Lehren über den allmählichen
Übergang der Biosphäre in die Noosphäre behandeln.

In dieser Publikation, die sich an einen möglichst breiten Leserkreis richtet,
haben wir die Anzahl der Kommentare stark reduziert und den notwendigen Teil
davon in die Fußnoten verschoben. Die meisten redaktionellen Anmerkungen
wurden aktualisiert.

In Form eines Anhangs zum Haupttext des Werkes enthält das Buch Vernadskys
Artikel „Über die wissenschaftliche Weltanschauung“ und „Ein paar Worte zur
Noosphäre“ sowie Fragmente des Manuskripts (sechs Absätze) „Wissenschaftliches
Denken als planetares Phänomen“, die, so glaubt man, vom Autor selbst nicht in
den Text aufgenommen wurden. Der Leser mag sich für den Inhalt dieser
Fragmente interessieren, und es ist schwierig, eine Zeitschrift mit kleiner
Auflage zu finden (Questions of History of Science and Technology. 1988, N I).
Deshalb haben wir beschlossen, sie in dieser Ausgabe wiederzugeben, nachdem
wir sie mit dem Original des Autors verglichen und Fehler und Verzerrungen in
der Zeitschriftenversion eliminiert haben.

Der Artikel „Über die wissenschaftliche Weltanschauung“ erschien erstmals 1902
in der Zeitschrift „Fragen der Philosophie und Psychologie“, N 65, und wurde
dann im Laufe seines Lebens mehrmals mit sehr kleinen Korrekturen in
verschiedenen Sammlungen neu veröffent"|licht\footnote{Publiziert nach
  \cite[S. 42-80]{Vernadsky1988}.}. 

Dies ist das erste philosophische Werk von B. Es ist das erste philosophische
Werk Vernadskys, wichtig im Hinblick auf die Formulierung der Ansichten, die
sein zukünftiges Werk prägen. Darin erklärt er die Existenz einer realen, von
unserem Bewusstsein unabhängigen Realität, deren Idee eine wissenschaftliche
Weltanschauung ist, die sich mit der Entdeckung neuer Tatsachen, neuer
Naturphänomene verändert. Es ist interessant, dass als P.I. Novgorodtsev,
nachdem er sich mit dem Manuskript dieses Artikels von Vernadsky vertraut
gemacht hatte, anbot, ihn in der zur Veröffentlichung vorbereiteten Sammlung
„Probleme des Idealismus“ zu veröffentlichen, Wladimir Iwanowitsch dies mit
der Begründung ablehnte, er sei kein Idealist, sondern Realist
\cite{Mochalov1982}.

Der zweite Artikel, „Ein paar Worte zur Noosphäre“ \cite{Vernadsky1944}, kann
als eine direkte Fortsetzung und Weiterentwicklung der in „Wissenschaftliches
Denken als planetares Phänomen“ geäußerten Ansichten betrachtet werden. Dieser
Artikel wurde 1944 veröffentlicht. Er artikuliert klar die Bedingungen, die
den Übergang von der Biosphäre in die Noosphäre gewährleisten, und endet mit
dem festen Glauben des Autors an den Sieg über den Faschismus, denn „die
Ideale unserer Demokratie stehen im Einklang mit dem natürlichen geologischen
Prozess, mit den Naturgesetzen, treffen auf die Noosphäre“.

Diese letzte Veröffentlichung von Vernadsky zu Lebzeiten (der Artikel wurde
bereits 1943 auf dem Höhepunkt des Krieges geschrieben) ist von Optimismus
durchdrungen. Der brillante Wissenschaftler war überzeugt vom Triumph der
menschlichen Vernunft und nicht nur von der Niederlage des Faschismus, sondern
auch von der Beseitigung all dessen, was die Biosphäre noch daran hindert,
eine Noosphäre zu werden.

Nun beginnen sich seine Vorhersagen zu erfüllen.

\subsection*{Hinweis}

In Vernadskys Manuskript „Wissenschaftliches Denken als planetares Phänomen“,
geschrieben 1938 und aufbewahrt im Archiv der Akademie der Wissenschaften der
UdSSR, nach Paragraph 150, dessen Text aus dem Buch „Gedanken eines
Naturalisten“, Ausgabe 1977, entnommen und in dem 1988 erschienenen Buch
„Philosophische Gedanken eines Naturforschers“ vollständig restauriert wurde,
Es folgen sechs weitere Absätze, über die I.I. Mochalov und C.P. Florensky
in ihrem Kommentar zur Ausgabe von 1977 geschrieben haben:

“Bei der Vorbereitung der Arbeit für den Druck hielt es die Redaktion für
notwendig, einige, im Allgemeinen kleinere Textteile wegzulassen ... Die
folgenden fünf Absätze wurden ebenfalls ausgelassen (nach Paragraph 150,
ausgelassen im Buch. - Hrsg.) fünf Absätze (Fehler der Kommentatoren,
eigentlich sind es sechs. - Hrsg.). Ihr Inhalt ist ganz von der ungesunden
Situation inspiriert, die in den Diskussionen der UdSSR über die Biologie in
den Jahren 1936-1938 und in den Auseinandersetzungen über die Methoden der
Radiobiologie im Jahre 1934 herrschte. Sie weichen noch mehr vom allgemeinen
Plan des Gesamtwerkes ab, sind überhaupt nicht mit der vorhergehenden Aussage
verbunden; in ihnen kann man besonders viel Eile, Unvollständigkeit, Einfluss
der rein emotionalen Stimmung des Autors spüren. Es ist bemerkenswert, dass
die Version des Manuskripts, die im Office-Museum von Vernadsky aufbewahrt und
später auf einer Schreibmaschine nachgedruckt wird, offensichtlich noch im
Leben des Autors von seinem Sekretär A.I. Vernadsky erhalten ist. Es gibt
überhaupt keine spezifizierten Absätze; es ist möglich, dass der Text des
Manuskripts in dieser Form nach den Wünschen des Gelehrten selbst nachgedruckt
wurde. Vernadsky spürte zweifellos die Schwäche dieses Teils des Textes.
Bezeichnenderweise finden wir hier seine Aussage, dass er hier Teil eines
Bereichs ist, der weit von seinen Interessen und seinem Wissen entfernt ist.
Der logische Faden der Argumentation des Autors und seine wissenschaftliche
Argumentation werden von solchen Abkürzungen nicht berührt“ (Vernadsky V.B.).
(Vernadsky V.I. Wissenschaftliches Denken als planetares Phänomen. M.: Nauka,
1977. S. 154).

Es kann nicht gedacht werden, dass eine solche Beurteilung der letzten Absätze
des Buches Wissenschaftliches Denken als planetarisches Phänomen nur auf den
Kontext der Zeit der Veröffentlichung dieses Werkes (1977) bezogen war.

In Vorbereitung auf die Veröffentlichung von „Philosophische Gedanken eines
Naturforschers“ behielt die Redaktion, zu deren drei Mitgliedern I.I. Mochalov
gehörte, diesen Kommentar bei und erweiterte ihn leicht. „Wir können
hinzufügen“, so hieß es, „dass viele der Bestimmungen, die in den Absätzen
enthalten sind, auf die in anderen in diesem Buch veröffentlichten Absätzen
Bezug genommen wird. Dies erlaubt uns zu denken, dass wir es hier nicht mit
einer Erweiterung des Manuskripts zu tun haben, sondern mit einer frühen
Version des letzten Teils des Manuskripts“ (Vernadsky V.I. Philosophische
Gedanken eines Naturalisten. M.: Nauka, 1988. S. 437).

Dennoch wurde noch vor der Veröffentlichung des Buches in der Zeitschrift
„Fragen der Geschichte der Naturwissenschaft und Technik“ (1988, N 1), noch
vor dem eigentlichen Jubiläum, ein ausführlicher „Brief an die Redaktion“
veröffentlicht, unterzeichnet von I.I. Mochalow, N.F. Owtschinnikow und
A.P. Ogurtsow. Dieser „Brief“ beschreibt ausführlich, welche Oratorien die
Redaktion der 1977 erschienenen ersten Ausgabe von Vernadskys
“Wissenschaftliches Denken als planetares Phänomen“ erlebte und mit welchen
großen, verzerrenden Banknoten sie veröffentlicht wurde. Anschließend wird
über die Vorbereitung der zweiten Ausgabe dieses Werkes durch Vernadsky
berichtet, und der „Brief“ endet mit dieser Erklärung: „Leider hat im Dezember
1987 die Kommission für die Entwicklung des wissenschaftlichen Erbes von
Vernadsky unter dem Vorsitz von Akad. L.L. Yanshin ihre frühere Entscheidung
revidiert, den vollständigen Text von Vernadskys Buch zu veröffentlichen, und
beschloss, es mit redaktionellen Anmerkungen zu veröffentlichen. Dies
geschieht in einer Zeit, in der die Prinzipien der Öffentlichkeit und der
Wiederherstellung der historischen Wahrheit in unserem Leben etabliert werden.

Der verleumderische Charakter dieser Aussage war offensichtlich, aber da die
Zeitschrift „Fragen der Geschichte der Naturwissenschaft und Technik“ in
vernachlässigbarer Auflage (1857 Exemplare) erscheint und nur von wenigen
Experten gelesen wird, hielt es L.L. Yanshin für möglich, auf diese Anwürfe
nicht zu antworten.

In der Zeitschrift „Nature“ erschien jedoch ein interessanter Artikel von
V.P. Zinchenko „Kultur und Persönlichkeit in der Geschichte der Wissenschaft“,
in dem wiederholt wird, dass „die Kommission für die Entwicklung des
wissenschaftlichen Erbes von Vernadsky die zweite, wiederum verkürzte Ausgabe
dieses Buches vorbereitet“ (Nature, 1989, N 1, S. 122). Dies bezieht sich auf
die Arbeit von B. Vernadskys „Wissenschaftliches Denken als planetares
Phänomen“. Und in der Anmerkung des Editorials heißt es: „Gegenwärtig ist
dieses Buch bereits in der vermeintlich reduzierten Form veröffentlicht
worden“.

Diese wird in einer Zeitschrift gedruckt, die in einer Auflage von 54.000
Exemplaren erscheint und von Millionen von Bürgern unseres Landes gelesen
wird. Sie müssen die Wahrheit kennen.

Wie bereits erwähnt, wurde im März 1988 der 125. Geburtstag von Vernadsky in
unserem Land weithin gefeiert.

Der allgemeine Plan für die Vorbereitung der Aktivitäten im Zusammenhang mit
diesem Datum wurde durch das Dekret des Präsidiums der Akademie der
Wissenschaften der UdSSR N 722 vom 22. Juni 1985 festgelegt, dessen
vollständiger Text veröffentlicht wurde (Bulletin der Kommission für die
Entwicklung des wissenschaftlichen Erbes des Akademikers Vernadsky, 1988, N
2). Zu den Aktivitäten, die für die Veröffentlichung der ausgewählten Werke
Vernadskys zum Jubiläum vorgesehen waren, gehörte auch eine Liste, die von der
Kommission für die Entwicklung des wissenschaftlichen Erbes des Akademikers
Vernadsky festgelegt werden sollte.

Bei diesem Treffen, d.h. mehr als zwei Jahre vor dem Jahrestag, wurde die
Frage eines vollständigen Neudrucks der „Reflexionen des Naturforschers“ von
V.I. Vernadsky, die erstmals 1975 und 1977 mit großen Scheinen in Form von
zwei Büchern veröffentlicht wurden, aufgeworfen. Der Vorschlag wurde
einstimmig angenommen, und im Dezember 1985 war im Arbeitsplan der Kommission
vorgesehen, I.I. Mochalov und N.F. Ovchinnikov über den Stand der Vorbereitung
dieser Werke von Vernadsky zur Neuausgabe zu berichten. Diese Vorbereitung war
jedoch im Dezember noch nicht abgeschlossen. Auf der Sitzung der Kommission am
17. April 1986 erklärten I.I. Mochalov und I.F. Ovchinnikov, dass sie die
Vorbereitungen für die Neuveröffentlichung der „Naturalist's Reflections“
abschließen würden, und demonstrierten die Bücher der Ausgaben von 1975 und
1977 mit zahlreichen maschinengeschriebenen Aufklebern von Notizen, die in der
ersten Ausgabe gemacht worden waren. Bei späteren Sitzungen der Kommission im
Laufe des Jahres 1986 erklärten sie wiederholt, dass der Text der Erstausgabe
der Bücher sorgfältig mit dem Archivmaterial von Vernadsky abgeglichen worden
sei, dass alle Rechnungen restauriert worden seien und dass die Reflexionen
der Naturalisten vollständig für eine Neuausgabe vorbereitet worden seien. Auf
dieser Grundlage nahm die Kommission Vernadskys Buch „The Philosophical
Thoughts of a Naturalist“ in den Redaktionsplan des Wissenschaftsverlags von
1987 auf. Der Titel des Buches entsprach der Idee von Vernadsky, die von den
Teilnehmern der Sitzung der Kommission am 19. Mai 1987 I.I. Mochalov wurde von
einem der verantwortlichen wissenschaftlichen Herausgeber des Buches, und
N.F. Ovchinnikov - von einem seiner Autoren gebilligt.

Bis Ende Juni 1987 gingen weder bei der Kommission für die Erschließung des
wissenschaftlichen Erbes des Akademikers Vernadsky noch beim „Nauka“-Verlag,
wo das Manuskript am 2. April 1987 übergeben wurde, Kommentare der oben
genannten Personen zum Inhalt des Buches ein.

Und später geschahen seltsame Dinge.

Am 28. Juni 1987 schickte Mochalow A.L. Janschin, geschrieben von ihm und
M.S. I.I. Mochalow schickte A.L. Janschin, geschrieben von ihm und
M.S. Bastrykowa, „Blöcke“ für das redaktionelle Vorwort zum Buch
„Philosophische Gedanken eines Naturforschers“, das zu diesem Zeitpunkt noch
nicht am Set eingereicht worden war, und in einem Begleitschreiben sagte, dass
er eine Reihe von nicht wiedergefundenen Rechnungen in der Arbeit „Raum und
Zeit in der belebten und lebendigen Natur“ (nicht in der Arbeit
„Wissenschaftliches Denken als planetarisches Phänomen“), und macht daher
gemeinsam mit M. S. Janschin. Der Vorschlag von S. Bastrykova: die
„Philosophischen Gedanken eines Naturforschers“ im Jubiläumsjahr zu
veröffentlichen und nicht zum Jahrestag (März 1988), da die Arbeit immer noch
bedeutend ist - vor allem was die Restaurierung der Notizen und Kommentare
betrifft.

Die Frage ist: Warum haben I.I. Mochalow, N.F. Owtschinnikow und
M.S. Bastrakowa, die auch ein „Kompilator“ des Buches war, diese Arbeit in den
letzten zweieinhalb Jahren nicht gemacht? Denn die Entscheidung, die
philosophischen Werke Vernadskys genau zu seinem Jubiläum in vollem Umfang neu
zu veröffentlichen, wurde von der Kommission auf ihrer Sitzung am 29. Oktober
1985 getroffen.

23. September 1987 I. Mochalov und N.F. Ovchinnikov schickten L.L. Yanshin den
folgenden Brief, der verschiedene Mängel der ersten Ausgabe von „Naturalist
Reflections“ auflistete und dazu aufrief, „zu den Ursprüngen zurückzukehren,
d.h. den Texten, deren Nachdruck im Verlag „Nauka“ offenbar bereits
abgeschlossen ist, um sie mit den Originalwerken von Vernadsky, die im Archiv
der Akademie der Wissenschaften der UdSSR aufbewahrt werden, zu überprüfen. So
gaben sie erneut zu, dass sie selbst keine solche Arbeit geleistet hatten,
obwohl sie sich dazu verpflichteten und dementsprechend von einem
verantwortlichen Herausgeber, einem anderen Verfasser des Buches, genehmigt
wurden.

Aber die Warnung kam zu spät. Noch bevor wir den ersten Brief von Mochalow
erhielten, der am Vorwort des Buches arbeitete, zweifelten wir an der Qualität
der Druckvorbereitung durch Mochalow und Owtschinnikow. Wir mussten eine für
den Druck vorbereitete Kopie des Werkes aus dem Verlag zurückziehen und
zusammen mit einem korrespondierenden Mitglied der Akademie der Wissenschaften
der UdSSR, S.R. Mikulinsky, noch einmal sorgfältig seinen Text mit den im
Archiv der Akademie der Wissenschaften der UdSSR aufbewahrten Manuskripten von
Vernadsky überprüfen. Mit der großen Hilfe des Archivdirektors B.W. Lewschin
und anderer Mitarbeiter dieser Institution wurde diese Arbeit in relativ
kurzer Zeit erledigt, und in der Tat fehlten viele Notizen von I.I. Mochalow
und N.F. Owtschinnikow zur Vorbereitung von V.I. Vernadskys Arbeit für den
Druck, sowie gefundene Ungenauigkeiten in den Notizen, die sie restaurierten.
All diese Mängel wurden korrigiert, zwei Monate später wurde das Manuskript an
den Verlag zurückgeschickt und neu gesetzt, am 25. November 1987 in den Druck
gegeben, am 15. Januar 1988 für den Druck unterzeichnet und bis Ende Februar
waren bereits zwanzigtausend Exemplare der „Philosophischen Gedanken des
Naturforschers“ an Buchhandlungen versandt worden. An den Tagen des Jubiläums
wurde das Buch aufgekauft.

Der oben erwähnte „Brief an den Redaktionsausschuss“ der Zeitschrift „Fragen
der Wissen"|schafts- und Technikgeschichte“ wird begleitet von
„unveröffentlichten Fragmenten“ von Vernadskys „Wissenschaftliches Denken als
planetares Phänomen“. Von diesen „Fragmenten“ können das Ende von Paragraph
149 und der gesamte Text von Paragraph 150 auf den Seiten 194-195 der
Philosophischen Gedanken des Naturalisten gelesen werden. Fragmente aus diesem
Werk, das von V.S. Napolitanskaja in der Zeitschrift „Century XX and the
World“ (1987, N 9) veröffentlicht wurde, sind ebenfalls vollständig an den
entsprechenden Stellen dieses Buches wiedergegeben, in zwei Fällen mit
Korrekturen des Manuskripttextes durch Vernadsky.

Was ist es also? Warum schreiben die Autoren des „Briefs an den
Redaktionsausschuss“: „Leider hat im Dezember 1987 die Kommission für die
Entwicklung des wissenschaftlichen Erbes von Vernadsky unter dem Vorsitz von
Akad. A.L. Yanshin revidierte ihre früheren Entscheidungen, den vollständigen
Text von Vernadskys Buch zu veröffentlichen, und beschloss, es mit
redaktionellen Anmerkungen zu veröffentlichen“?

Es sind alles reine Lügen. Ein absichtlicher Wunsch, das Buch, das gerade
veröffentlicht wird, in den Augen des Lesers zu diskreditieren. Im Dezember
1987 fanden zwei Sitzungen der Kommission statt: am 4. Dezember und am
29. Dezember zusammen mit dem Organisationskomitee für die
Jubiläumsfeierlichkeiten, das vom Präsidenten der Akademie der Wissenschaften
der UdSSR, Akademiker G.I. Marchuk, geleitet wurde. Auf dem ersten Treffen
wurden Veröffentlichungsfragen nicht diskutiert. Das zweite Treffen wurde
nicht von A.L. Yanshin, sondern vom Vizepräsidenten der Akademie der
Wissenschaften der UdSSR, dem Akademiker P.N. Fedoseev, geleitet. Gemäss Punkt
5 der Tagesordnung berichtete der Direktor des Verlags „Nauka“
S.A. Chibiriyaev kurz über den Stand der Vorbereitungen für die
Veröffentlichung der Werke von Vernadsky. Die Frage der „Revision der früheren
Beschlüsse der Kommission“ wurde bei diesem oder früheren Treffen überhaupt
nicht angesprochen. Und warum wurde es gesetzt, wenn in dem von I.I. Mochalow
und N.F. Owtschinnikow zum Druck übergebenen Exemplar alle Banknoten fehlten.

Bis auf ein weiteres strittiges Fragment, das übrigens auch nicht in der Kopie
der „Philosophischen Gedanken des Naturalisten“ enthalten war, die
I.I. Mochalov und N.F. Ovchinnikov zum Druck übergeben wurde, wurden
I.I. Mochalov und N.F. Ovchinnikov restauriert. Erst in der zweiten Hälfte des
Jahres 1987, als das Manuskript von „Naturalist's Philosophical Thoughts“
bereits im Set war, brachte I.I. Mochalov mehrere Manuskriptseiten zum
Herausgeber N.B. Zolotova und sagte, dass sie in das Manuskript aufgenommen
werden sollten. Bei der Besichtigung stellte sich heraus, dass der Text
Vernadskys auf den von I. Mochalow übergebenen Seiten um fast ein Drittel
gekürzt war, und zwar um nicht näher bezeichnete Rechnungen und den
reproduzierbaren Teil von Vernadskys Text. Vernadskys Text wurde an einigen
Stellen willkürlich verzerrt. Aus wissenschaftlicher Sicht wäre eine solche
Publikation ein Skandal.

Nach eingehender Prüfung des Manuskripts kam die Redaktion zu dem Schluss,
dass die Meinung von I.I. Mochalov und C.P. Florensky, die sie in ihrem
Kommentar zur Ausgabe von 1977 zum Ausdruck brachten, ihre Resonanz hat und
die letzten sechs Absätze insgesamt 12,5 maschinengeschriebene Seiten
umfassen. Der Kommentar zu den „Philosophischen Gedanken des Naturalisten“
spezifiziert all dies und der Leser erfährt dies nicht aus einem Brief an die
Herausgeber von I.I. Mochalov, N.F. Ovchinnikov und A.P. Ogurtsov,
veröffentlicht in „Fragen zur Geschichte der Naturgeschichte und Technik“ und
nicht aus der redaktionellen Notiz in „Nature“, sondern aus dem Buch
„Philosophische Gedanken eines Naturforschers“.

Schließlich sollten wir zu den Worten Vernadskys, die Zweifel an der
Zweckmäßigkeit der Veröffentlichung der letzten Absätze aufkommen lassen,
Folgendes hinzufügen. „Wissenschaftliches Denken als planetares Phänomen“ ist,
zumindest in der ersten Fassung, die aus irgendeinem Grund unveröffentlicht
bleibt, kein vollständiges Manuskript, sondern ein Entwurf, eine
unvollständige und nicht bearbeitete Fassung. Es genügt ein Blick darauf, um
sich selbst davon zu überzeugen. Es gibt viele ungeschriebene Phrasen,
unkoordinierte Wörter, Notizen wie „gehen Sie zurück“ oder „entwickeln Sie
sich weiter“ usw., aber es wurde nie gemacht. Das Manuskript wurde vom Autor
nicht für den Druck vorbereitet. Zwingt uns dieser Umstand allein nicht zur
Vorsicht, zur Nachdenklichkeit beim Publizieren? Der Respekt vor dem Autor
erfordert besondere Aufmerksamkeit bei der Reproduktion, nicht das blinde
mechanische Kopieren des Manuskripts.

Natürlich ist es um der wirtschaftlichen Überlegungen willen, um der Sensation
willen, viel einfacher, den Lärm um Kürzungen bei der Berechnung der einfachen
Dividenden zu machen, als die Vorzüge zu verstehen. Aber so arbeitet ein
Wissenschaftler nicht, der den Gegenstand seiner Forschung respektiert.

Mochalov und seine Freunde machten sich nicht die Mühe, den Text Vernadsky zu
lesen, indem sie das Gerücht über die Kürzungen in Umlauf brachten. Damit
hätten sie dafür gesorgt, dass die ausgelassenen Absätze in einer Reihe von
Fällen die Gedanken wiederholen, die in den vorhergehenden, in dem Buch
„Philosophische Gedanken eines Naturforschers“ veröffentlichten Absätzen
manchmal fast wörtlich ausgedrückt wurden (siehe S. 87, 94, 105. 108-109,
185, 194, 257).

Dies ist der Fall bei den letzten sechs Absätzen von „Wissenschaftliches
Denken als planetarisches Phänomen“, dem einzigen bewussten Akronym in der
Publikation von 1988.

Ich kann nicht umhin, Folgendes zu bemerken. Vernadskys Kommentar zur Ausgabe
von 1977 im Allgemeinen, einschließlich der Stelle, die er hier zitiert hat,
ist viel zu sehr von „Eile“, „Schwäche“ usw. geprägt. Die Frage nach der Natur
der letzten sechs Absätze wurde völlig zu Unrecht benutzt, um andere
Verzerrungen des Textes von V.I. Vernadsky zu verschleiern. Vernadskys Ausgabe
von 1977, die mit den letzten sechs Absätzen nichts zu tun hatte. Für den
Leser ist es offensichtlich, dass der Eifer, all die zahlreichen Abkürzungen
und Verzerrungen in der Ausgabe von 1977 zu rechtfertigen (allein in dem Werk
„Wissenschaftliches Denken als planetares Phänomen“ haben wir etwa 80 davon
gezählt), offensichtlich überzogen ist. Erklärt dies nicht, warum
I.I. Mochalov, jetzt, da die Veröffentlichung des vollständigen Textes des
Wissenschaftlichen Denkens als planetares Phänomen im Buch Philosophische
Gedanken eines Naturforschers es jedem leicht macht, herauszufinden, wie
verzerrt der Text dieses Werkes in der Ausgabe von 1977 war, aus Verlegenheit
so sehr versucht, wie ein Kämpfer für die Wahrheit auszusehen: Sie werden
vergessen, wie er 1977 die Selbstverwaltung von Vernadskys Text gerechtfertigt
hat.

Übrigens, wäre I.I. Mochalov bei der Vorbereitung der Publikation 1977 treu
gewesen. „Wissenschaftliches Denken als planetares Phänomen“, hätte er in dem
von uns zitierten Kommentar nicht spekulieren müssen, dass das Manuskript
„offensichtlich zu Lebzeiten des Autors“ nachgedruckt wurde, denn
offensichtlich ein anderes. Das Manuskript endet mit A. Das Manuskript endet
mit der Notiz von A.D. Schachowskaja vom Januar 1951, also sechs Jahre nach
dem Tod Vernadskys: „Das Original enthält Notizen von Wladimir Iwanowitsch,
Verweise auf die Literatur und seine eigenen Überlegungen. Aber die
Nummerierung ist umständlich, so dass es viel Zeit braucht, um herauszufinden,
auf welche Nummer im Text sich diese oder jene Notiz bezieht, und da das
Manuskript nicht für den Druck bestimmt ist, dachte ich, es sei möglich, sie
im Manuskript zu belassen. A. Schachowskaja“.

Und noch etwas. I.I. Mochalov war einer der drei Herausgeber des Buches
„Philosophische Gedanken eines Naturalisten“ und leitete das Korrekturlesen.
Er gab sie zurück, ohne in irgendeiner Weise auf das Fehlen der letzten
Absätze in „Wissenschaftliches Denken als planetares Phänomen“ zu
reagieren.

Offenbar rettete er die berüchtigten Schlussabsätze „für sich“, für seine
Sonderpublikation, die er vor dem Jahrestag zusammen mit N.F. Owtschinnikow
und A.P. Ogurtsow umsetzte.

\section{Anmerkung von I.I. Mochalov}
\begin{center}
  I.I. Mochalov, V.S. Neapolitanskaya, M.Yu. Sorokina und A.A. Jaroschewsky
\end{center}
Die elektronische Version dieser Notiz wurde für die Veröffentlichung im Buch
\begin{quote}
  B. I. Vernadsky, Über Wissenschaft, Band 1, Wissenschaftliches Schöpfertum.
  Nauchnaya Mysl, Dubna, “Phoenix“, 1997.
\end{quote}
vorbereitet.

Das Buch „Wissenschaftliches Denken als planetares Phänomen“ steht in engem
Zusammenhang mit Vernadskys Plan, das „Hauptbuch des Lebens“ oder „Buch des
Lebens“ zu schreiben, in dem er seine neue Weltanschauung umreißen wollte, die
von der kosmischen Rolle der lebenden Materie ausgeht. Diese neue Sichtweise
hat sich in dem Wissenschaftler, wie er selbst immer wieder betont hat, seit
1916 herausgebildet, als die Grundprinzipien der Biogeochemie festgelegt
wurden. Im Vorwort des Autors zu „Biogeochemische Skizzen“ von 1940 sagt
V.I. Vernadsky, dass er in den 20er und 30er Jahren vier Bücher schrieb:
„Skizzen der Geochemie“, „Biosphäre“, „Geschichte der natürlichen Gewässer“
und „Probleme der Biogeochemie“, „... die sich mit den gleichen Fragen der
Erforschung des Lebens unter dem chemischen Aspekt befassen wie ein
natürlicher Teil der Geschichte und Struktur unseres Planeten
\cite[S. 263]{Vernadsky1992}.

Weiter betonte er hier, dass von den vielen Problemen und Fragen, die in
diesen Büchern aufgeworfen werden, „ein großes Problem, das ich beenden
möchte, bevor ich das Leben verlasse, und das meine ganze Kraft eingefangen
hat, das Problem der biogeochemischen Energie unseres Planeten ist“ (ebd.).

Er erwähnte diese Idee zum ersten Mal in seinen Briefen vom Oktober 1933. Am
28. Oktober schrieb er an S.F. Oldenburg: „Ich denke viel nach und arbeite an
meinem Thema - an der biogeochemischen Energie in der Erdkruste. Aber jetzt
gehe ich unwissentlich beiseite oder tief in die Tiefe, wie Sie es verstehen
wollen - in philosophische Fragen. Persönlich glaube ich nicht, dass sie
tiefer als die wissenschaftliche Interpretation der Welt sind. Ich weiß nicht,
ob ich leben werde, aber das Buch über biogeochemische Energie ist kaum vor
zwei Jahren fertig. Und wenn ich lebe, werde ich mich mit den „Philosophischen
Gedanken eines Naturalisten“ und vor allem mit der genauen Analyse des
Verhältnisses zwischen Wissenschaft und Philosophie, der Zukunft der
Menschheit, der empirischen Verallgemeinerung, der empirischen Idee und der
empirischen Tatsache und ihrer Differenz zu philosophischen Konzepten
befassen, noch einmal Zeit ... Es gibt viele Dinge, die ich gerne sagen würde“
\cite[S. 76]{Rosov1993}. Einige grundlegende Themen der zukünftigen Arbeit
werden hier skizziert.
  
Von Zeit zu Zeit kehrte Vernadsky zu der Idee zurück, ein wissenschaftliches
letztes „Buch des Lebens“ zu schaffen. Zu dieser Zeit war er sehr mit der
Leitung des Radium-Instituts und des Biogeochemischen Labors beschäftigt. Am
20. Dezember 1934 schrieb er in sein Tagebuch: „Es ist doch notwendig, mein
Buch zu schreiben und sich nicht vom Strom der Ideen und neuen Fakten und
empirischen Verallgemeinerungen mitreißen zu lassen ... Die Idee wird immer
heller und heller - in einem Jahr, zwei Jahren, um beide Institutionen bei der
wissenschaftlichen Arbeit an meinem Buch zu verlassen“ (Archiv der RAW,
f. 518, Op. 2, 7). Im April 1936 wurde er, wie der Brief an B.L. Litschkow
beweist, von der Idee ergriffen, „das Buch zu schreiben und fertigzustellen“
über einige Grundbegriffe der Biogeochemie (Korrespondenz von V.I. Vernadsky
mit B.L. Litschkow. Moskau, 1979, S.172. Im Folgenden - Korrespondenz).

V.I. Vernadsky begann in einem Ferienhaus „Kiefernwald“ in Bolschew bei Moskau
ein konzipiertes Buch zu schreiben. Am 13. Mai 1936 sagte er zu
Agrarumweltbehörde Fersman: „Jetzt im Kiefernwald arbeitete gut an dem Buch
und trat über ein Hindernis, das mich in meinem bestehenden Buch „Basic ideas
in biogeochemistry. Ich werde im Ausland daran arbeiten. Muss mit Karlsbad
beginnen“ (Briefe an Vernadsky A.E. Fersman, 1985, S.178).

Wie aus Briefen und Tagebucheinträgen ersichtlich ist, war sich der
Wissenschaftler der Notwendigkeit bewusst, eine neue Sprache, eine neue Logik
für die biogeochemische Beschreibung zu verstehen. Er hielt es für notwendig,
die Logik der Naturerkenntnis in den physikalisch-chemischen, grundlegenden
Wissenschaften von der Logik der Wissenschaften des Biosphärenzyklus zu
trennen.  Am 30. Juli 1936 schrieb V.I. Vernadsky an B.L. Litchkov aus Uzky:
“Hier in Schwarz beendete er seinen Artikel über Goethe, ich möchte ihn in
diesen 10 Tagen, die bis zur Abreise blieben, nicht trennen. Wie es immer
geschieht, ergreift mich ein Gedanke, und darin näherte ich mich - zufällig,
dass ich mich näherte - der Grundposition im Buch, deren Grundlagen ich in
diesem Jahr entwickelt habe: der Notwendigkeit der logischen Erforschung der
Grundlagen der Naturwissenschaft - über die Abwesenheit dieser Grundlage in
der modernen Wissenschaft“. (Korrespondenz, S.179).

So ist die Idee des Buches „Über die Grundbegriffe der Biogeochemie“
gewachsen, einschließlich der Untersuchung der logischen Grundlagen der
Naturwissenschaft. Deshalb mussten wir nicht über die Grundbegriffe der neuen
Wissenschaft schreiben, sondern wissenschaftliche Konzepte als solche und noch
weiter gefasst - über den Unterschied zwischen wissenschaftlicher Sprache und
philosophischer Sprache - analysieren.

Im Sommer und Herbst 1936 arbeitete Vernadsky in der Tschechoslowakei eng an
dem Text. In seinen Briefen an B.L. Lichkow und A.E. Fersman nannte er es
zunächst „Einführung“. „In Karlsbad schrieb er die Einführung zu seinem Buch.
In London werde ich weiter am ersten Kapitel (über die Notwendigkeit, die
Logik der Naturwissenschaft zu klären) und an einem der weiteren Kapitel -
über die Asymmetrie - arbeiten“ (Briefe von Vernadsky A.E. Fersman, M., 1985,
S. 180). V.I. Vernadsky setzte diese Absicht teilweise um, und in London
schrieb er einen Auszug „Über die Logik der Naturgeschichte“ (siehe die
vorliegende Ausgabe). „Wie ich Ihnen schrieb“, teilte er mit B.L. Lichkow mit,
„habe ich mein Buch 'Über die Grundbegriffe der Biogeochemie' stark
weiterentwickelt, die Einführung in Schwarz geschrieben und den ganzen Plan
durchdacht. Jetzt muss ich schreiben, und ich möchte es als meine Hauptaufgabe
einrichten“ (Korrespondenz, S.184).

Die folgende Mitteilung betrifft den 25. Januar 1937: „Ich habe viel über das
Ideal nachgedacht, das wir in der Struktur der Noosphäre haben. Jetzt schreibe
ich - immer noch sporadisch, obwohl ich diese Arbeit als eine Lebensaufgabe
betrachte - „Zu den Hauptproblemen der Biogeochemie“, dem ich einige Rundgänge
beifügen werde, von denen zwei bereits in meinem Plan enthalten sind. 1) Über
die Logik der Naturwissenschaft (die noch nicht vorhanden ist oder, besser
gesagt, die bis zum Ende des Beginns unzusammenhängend und schlecht durchdacht
ist, aber inzwischen ändert ihr korrektes Verständnis im Wesentlichen unsere
Schlussfolgerungen). Die Biosphäre ist „Natur“ für alle geologischen und
biologischen Wissenschaften im weitesten Sinne des Wortes, und es gibt viele
Schlussfolgerungen, die für die gesamte Natur zutreffen, z.B. Entropie, die
Unvermeidbarkeit physikalischer und chemischer Prozesse in reversibler Form
usw., und 2) über Gut und Böse in der Konstruktion von Wissenschaft. Es
scheint mir, dass ich hier nicht in der Lage sein werde, die Grenzen der
Wissenschaft zu überschreiten - mit Ausnahme des kritischen Teils -, der für
mich in meinem historischen Prozess eine direkte Fortsetzung der Entstehung
des Gehirns, des Apparats des Homo sapiens, ist, aber im sozialen Prozess
entwickelt wurde. Es ist die Kraft, die die Biosphäre zur Noosphäre macht“
(Korrespondenz, S.188-189). Beide Richtungen können in dem Buch Scientific
Thought as a Planetary phenomenon gesehen werden.

Das Jahr 1937 war für die Realisierung des Plans erfolglos. Der
Wissenschaftler war sehr krank. Es nahm Zeit in Anspruch, an der Vorbereitung
der nächsten Sitzung des Internationalen Geologenkongresses teilzunehmen, wo
er einen seiner Berichte verfasste. Im August erlitt Vernadsky einen
Schlaganfall. Am 7. September beschwerte er sich bei B.L. Litchkov: „So sieht
das menschliche Leben aus. Natürlich war es ein wenig unverschämt, wie ich
Ihnen schrieb, das Hauptwerk des Lebens in 73 Jahren zu schreiben. Es ist mir
absolut verboten, zwei oder drei Monate lang etwas Ernstes zu studieren und zu
lesen“ (Korrespondenz..., S. 204).

Erst im Winter 1937-1938 wurde die Arbeit fast täglich. Tagebuch vom 4. Januar
1938: „Begann, systematisch an dem Buch zu arbeiten.“
(V.I. Vernadsky. Tagebuch von 1938. Veröffentlichung, einleitende Bemerkung
und Anmerkung von I.I. Mochalov. „Freundschaft der Völker“, 1991, 2, S. 221).
Die Arbeit wurde in der ersten Hälfte des Jahres 1938 sehr aktiv fortgesetzt.
Aus den Tagebucheinträgen kann man seine Etappen nachvollziehen, die sich
thematisch mit der Gliederung des Buches in Kapitel decken (Stichworte im
Tagebuch: „Aristoteles“, „Etappe der Viehzucht“). Am 28. März schrieb der
Wissenschaftler auf: „Er hat gut an dem Buch gearbeitet. Hat viel für die
Verdienste getan. Ich komme zum Ende der Einleitung“ (ebd., S. 246). Offenbar
wuchs im Prozess der „Einführung“ und begann, sich den Umriss eines
unabhängigen Buches anzueignen. Am 22. April notierte V.I. Vernadsky in seinem
Tagebuch: „Ich schreibe das Ende der Einführung zum Buch in der ersten
Ausgabe“ (ebd., S. 247).

Am 3. Mai erscheint die Aufzeichnung: „Ich habe an einem Buch gearbeitet. Nun
gehe ich tiefer auf den dialektischen Materialismus und die in unserem Land
geschaffene philosophische Situation ein“ („Freundschaft der Völker“, 1991, 3,
S. 250). Anscheinend wurde an den letzten Absätzen (151-156) gearbeitet, die
von marxistischer Philosophie sprechen. So wurde Anfang Mai die Hauptarbeit am
Text abgeschlossen und mit der Überarbeitung, Ergänzung und Korrektur
begonnen.

Wahrscheinlich hat Vernadsky damals dem Manuskript, das aus zehn Kapiteln und
156 Absätzen besteht, einen Titel gegeben. Das Archiv enthält zusammen mit dem
Manuskript zwei undatierte Skizzen von Plänen. Einer von ihnen beginnt wie
folgt: „Essay eins. Wissenschaftliches Denken als geologisches Phänomen“. Dann
erscheint ein genauer Name. Am 16. August 1938 schrieb er an B.L. Litchkov:
„Und die Krankheit ergriff mich mitten in meinem zweiten Aufsatz: Über die
Zustände des Weltraums.“ Beide stehen im Zusammenhang mit dem ersten Kapitel
meines Buches „Über die Probleme der Biogeochemie“ („Wissenschaftliches Denken
als planetares Phänomen“). Ich möchte etwas früher drucken, ohne auf die
Bearbeitung des Buches zu warten, die sich verzögert. Beide Artikel möchte ich
als Fortsetzung meiner „Probleme der Biogeochemie“ - zweiter und dritter -
1935 drucken (Korrespondenz, S. 227).

Die beiden fraglichen Artikel wurden in der Tat in das Buch „Probleme der
Biogeochemie“ aufgenommen, das damals nur die erste Ausgabe von 1935
enthielt. In der Notiz „Vom Autor“ zur zweiten Ausgabe, die den Titel trägt
„Über den grundlegenden materiell-energetischen Unterschied zwischen den
lebenden und den kosmischen Naturkörpern der Biosphäre“, schrieb der
Wissenschaftler: „Während der Arbeit vor kurzem an einem Buch mit dem Titel
Basic concepts of biogeochemistry in Verbindung mit dem Verlauf der
wissenschaftlichen Erfassung der Natur, der Autor findet es nützlich, ohne auf
seine Fertigstellung zu warten, die sich zwangsläufig verzögert, zu
identifizieren und separat zu entwickeln einige einzelne Fragen in dem Buch
angesprochen ... Moskau, September 1938“ (V.I. Vernadsky. Probleme der
Biogeochemie. Proceedings of the Biogeochemical Laboratory, Band 16. Moskau,
1980, S.55). Die in diesem Heft veröffentlichte Tabelle der Gegensätze des
Lebendigen und des Schrägen ist in „Wissenschaftliches Denken...“ enthalten
(siehe aktuelle Ausgabe, 142).

Bereits im Herbst 1938 wurde intensiv am Volltext des Buches gearbeitet. Die
letzte Erwäh"|nung davon bezieht sich auf den 7. Dezember 1938 in seinem
Tagebuch: „Gestern habe ich studiert - das Buch neu geschrieben.
(„Freundschaft der Nationen“, 1991, 3, S. 265). Es liegen keine Informationen
über die Fortsetzung der Bearbeitung oder Überarbeitung des Textes vor. Der
Autor hielt es jedoch nicht für vollständig, wenn man den Anweisungen auf
vielen für ihn selbst erstellten Seiten Glauben schenkt: “prüfen“, „Beispiele
nennen“, „Ich werde Ihnen später davon berichten“ und so weiter. Daraus ging
hervor, dass der Wissenschaftler seine Arbeit fortsetzen würde - den
Sachverhalt zu klären, die einzelnen Bestimmungen im Detail zu erweitern, neue
Abschnitte zu verfassen. Im gleichen Plan hieß der zweite Aufsatz „Biosphäre
und Noosphäre“. Der Plan, einen Sonderabschnitt „Über die Moral der
Wissenschaft“ zu schreiben, wurde nicht verwirklicht. Der Aufsatz “Über die
Logik der Naturgeschichte“, der im Rohentwurf erhalten geblieben ist, wurde
nicht weiterentwickelt (siehe vorliegende Ausgabe). Zahlreiche
organisatorische und aktuelle experimentelle Fälle lenkten jedoch vom Thema
ab. Nachdem er 1938 das Radium-Institut verlassen hatte, blieb Vernadsky
Direktor des Biogeochemischen Labors. Bis 1941 gelang es ihm nie, sich seiner
Idee zuzuwenden. Als der Große Vaterländische Krieg begann, wurde er in das
Erholungsdorf Borowoje in Kasachstan evakuiert. Hier wurde er von der Arbeit
an der großen Monographie „Chemische Struktur der Biosphäre der Erde und ihrer
Umwelt“, die durch ihre verallgemeinernde Kraft dem „Buch des Lebens“ am
nächsten kommt, vollauf erfasst.

Und in diesem unvollendeten großen Werk klingen die Motive des
„Wissenschaftlichen Denkens“, aber das gilt besonders für das Kapitel „Zur
Logik der Naturgeschichte“ aus der 3. Ausgabe von „Probleme der Biogeochemie“,
das damals in Borow fertiggestellt wurde.

Die letzten Artikel des Wissenschaftlers, die Ende 1943 geschrieben wurden,
sind den Beziehungen zwischen Wissenschaft und Philosophie, der Rolle des
wissenschaftlichen Denkens im Leben des Planeten und den Schicksalen der
menschlichen Gesellschaft, den Faktoren, die den Übergang von der Biosphäre
zur Noosphäre bestimmen, gewidmet. Eines davon, „Ein paar Worte zur
Noosphäre“, wurde in der Zeitschrift „Erfolge der modernen Biologie“ (1944,
Bd. 18, S. 2, 113-120) veröffentlicht. Es ist auch bekannt, dass er den Text
sowohl an Stalin persönlich als auch an die Zeitung Pravda geschickt hat.
Vernadsky hat in beiden Fällen keine Antwort erhalten. Einen weiteren Artikel
- „Biosphäre und Noosphäre“ - schickte der Wissenschaftler zur
Veröffentlichung in Amerika an seinen Sohn G. Vernadsky, Professor für
Geschichte an der Yale University. Es wurde nach dem Tod des Wissenschaftlers
in der Zeitschrift American Scientist (1945, Vol. 33, 1, S. 1-12) abgedruckt.
(In Rückübersetzung aus dem Englischen wird der Artikel in dem Buch:
V.I. Vernadsky. Biosphäre und Noosphäre. Moskau, 1989, S. 139-150
veröffentlicht).

So beunruhigten die Probleme, die in dem Buch „Scientific Thought“ entwickelt
wurden, Vernadsky bis zum Ende seines Lebens.

Das Original des Buches „Wissenschaftliches Denken als planetares Phänomen“
wird im Archiv der Russischen Akademie der Wissenschaften als Teil des
persönlichen Fonds des Wissenschaftlers aufbewahrt (f. 518). Es ist in drei
Ordnern enthalten, die den Haupttext des Werkes, Anmerkungen des Autors sowie
zugehörige Entwurfsskizzen, einzelne Fragmente und Versionen von Plänen
(Op. 1, 149, 150, 151) enthalten. Der Text ist ein maschinengeschriebener Text
mit einer handschriftlichen Korrektur durch den Autor. Es handelt sich um eine
grobe, nicht bearbeitete Version. Nach seinem Zustand zu urteilen, bewegte und
klebte V.I. Vernadsky wiederholt einige Teile des maschinengeschriebenen
Textes, strich Phrasen durch und trug sie in einer anderen Ausgabe wieder ein.
Als Ergebnis stellte sich heraus, dass in einer Reihe von Fällen die
Konstruktion von Phrasen, die Harmonisierung von Wörtern in Sätzen und die
ganzen Sätze untereinander gebrochen sind. Einige Phrasen wurden hastig
geschrieben, abgeschnitten und manchmal schlecht verstanden. Einige Wörter
sind unleserlich oder gekürzt. Es gibt viele Druckfehler im Text.

Die Notizen des Autors, die in einem speziellen Ordner aufbewahrt wurden,
waren überwie"|gend nicht mit dem Haupttext korreliert. Es handelt sich um
Rohentwürfe von Notizen, auf die der Autor nach den Kommentaren, die er „für
sich selbst“ im Text gefunden hat, zurückkommen wollte. Der Verweisungsapparat
ist nicht vollständig und fehlerhaft, weil V.I. Vernadsky, der mit einer
riesigen Menge an Material arbeitet, oft auswendig gelernt hat; manchmal wird
der Autor nicht erwähnt, manchmal fehlen der Titel des Werkes, seine Ausgabe
oder die Seiten, auf die er sich bezog; oft wurden die Namen der Autoren, die
Titel von Büchern und Zeitschriften unleserlich geschrieben oder abgekürzt. Im
Allgemeinen erforderte die Vorbereitung des Textes des Buches für den Druck
viel mühsame Arbeit, für die Vernadsky selbst keine Zeit hatte.

„Wissenschaftliches Denken als planetares Phänomen“ wurde bisher dreimal
veröffentlicht. Die Geschichte dieser Ausgaben ist an sich schon interessant,
da jede von ihnen eine bestimmte Wendung in der ideologischen und damit
politischen Geschichte unseres Landes widerspiegelt.

Die Gelegenheit, den Leser mit diesem Werk vertraut zu machen, bot sich
erstmals während des „Tauwetters“ von Chruschtschow. Initiiert wurde die
Publikation von W.S. Napolitanskaja, Kuratorin des Vernadsky-Kabinettmuseums
am Institut für Geochemie und Analytische Chemie der Akademie der
Wissenschaften. Sie war fasziniert von ihrer Idee von M.S. Bastrakowa und
N.W. Filippowa aus dem Archiv der Akademie der Wissenschaften, und bald
gesellte sich I.I. Mochalow zu ihnen, der zu dieser Zeit an der Dissertation
arbeitete, die der Arbeit des Wissenschaftlers gewidmet war. Dieses kleine
Team führte die Entschlüsselung des komplexen Autorentextes, seine erste
Bearbeitung und die archäologische Vorbereitung für den Druck durch. Die
Autoren stellten, soweit möglich, unleserliche Wörter wieder her und deckten
Abkürzungen auf; entwirrten und rekonstruierten schlecht verstandene und
grammatikalisch gebrochene Phrasen. Dies geschah nur in wenigen Fällen und
sehr sorgfältig, ohne die Bedeutung der Phrasen, den Stil oder die Sprache des
Autors zu verändern. Sie korrelierten die Notizen des Autors mit dem Text; sie
überprüften, klärten und füllten die Referenzmaschine so weit wie nötig und
möglich aus.

Ende der 1960er Jahre übergaben die Verfasser das von ihnen vorbereitete
Manuskript dem Akademiker B.M. Kedrov zur wissenschaftlichen Bearbeitung und
speziellen Kommentierung. In dieser Zeit verschlechterte sich jedoch das
politische Wetter in der UdSSR, und es dauerte fast zehn Jahre des Kampfes mit
den Eiferern der ideologischen Grundlagen und der intensiven Suche nach
Kompromissen, bis das Manuskript das Licht der Welt erblickte.

B.M. Kedrow, dank dessen Bemühungen, Energie und Einfallsreichtum es gelang,
die Zensurschleuder zu umgehen und das Buch im Druck zu fördern, schlug aus
„taktischen Gründen“ vor, es nicht als separate Publikation zu drucken,
sondern als ob er es unter anderen Werken Vernadskys, die sich theoretischen
Problemen der Wissenschaft widmen, verstecken wollte. 1977 schlug er vor, ihn
nicht als separate Publikation zu drucken, sondern ihn unter anderen Werken
Vernadskys über theoretische Probleme der Wissenschaft zu verstecken.
„Scientific Thought“ wurde 1977 als Teil des zweibändigen Buches „Naturalist
Reflections“ (Buch 2) veröffentlicht. Leider unterzog die Zensur das
Manuskript einer echten „Vivisektion“. Aus konjunkturellen Gründen wurden
Sätze und Wörter geändert, und vor allem wurden zahlreiche Banknoten
hergestellt. Alles, was Naukas Herausgeber als einen Hinweis auf „ideologische
Subversion“ sahen, wurde aus dem Text des Autors entfernt. Sie schlossen nicht
nur einzelne Wörter und Sätze, sondern ganze Absätze, Seiten und sogar Absätze
aus dem Text aus. Die Absätze 73 und 150 bis 156 wurden vollständig
gestrichen; von einigen Absätzen blieben nur einzelne Phrasen übrig, einige
wurden um zwei Drittel oder die Hälfte gekürzt (z.B. Abschnitt 6, 68, 71, 72,
77, 80, 145 usw.). Insgesamt wurden bis zu 3 gedruckte Blätter des Textes des
Autors entfernt. Die Publikation wurde mit Kommentaren versehen, von denen
viele darauf abzielten, V.I. Vernadskys Ideen und Bestimmungen im Hinblick auf
die herrschende Ideologie zu „erklären“ und zu interpretieren.

Es ist ein weiteres Jahrzehnt vergangen. Ende der 80er Jahre erschienen
Zeitschriftenveröffent"|lichungen, deren Zweck es war, den Leser mit den
Texten vertraut zu machen, die 1977 von der Veröffentlichung von „Scientific
Thought“ ausgeschlossen wurden. 67-74 („20. Jahrhundert und die Welt“, 1987,
9. S. 38-43); I.I. Mochalov, N.F. Ovchinnikov und A.P. Ogurtsov
veröffentlichten auf den Seiten „Fragen der Geschichte der Naturgeschichte und
Technik“ die Abschnitte 150-156, besonders „gefährlich“ aus der Sicht der
offiziellen Parteiideologie, denn hier bewertete V.I. Vernadsky den
dialektischen Materialismus und stellte den schädlichen Einfluss des
marxistischen philosophischen Dogmas auf die Entwicklung des
wissenschaftlichen Denkens in der UdSSR fest (VIET, 1988, 1, S. 71-79).

Die zweite Auflage des Buchs „Wissenschaftliche Gedanke als planetares
Phänomen“ wurde 1988 als Teil des Buches „Philosophische Gedanken eines
Naturforschers“ veröffentlicht, das unter der Schirmherrschaft der Kommission
der Akademie der Wissenschaften der UdSSR für das Studium des
wissenschaftlichen Erbes von Vernadsky (Vorsitzender - Akademiker
A.L. Yanshin) herausgegeben wurde. Dieses Buch war ein Nachdruck der
zweibändigen Ausgabe „Naturalist's Reflection“ mit einigen Ergänzungen.
Diesmal enthielt der Text von „Scientific Thought“ fast alle Notizen aus dem
Jahr 1977, mit Ausnahme der „kriminellen“ Absätze 151-156. Die Herausgeber
wagten es nicht, sie auch inmitten von „Perestroika und Glasnost“ zu
veröffentlichen. Sie ließen die Zeilenkommentare, die bereits ihre Bedeutung
verloren hatten, unverändert und „korrigierten“ V.I. Vernadsky; es gab auch
Artikel verschiedener Autoren, die wie die Kommentare vor allem wertenden,
„erklärenden“ Charakter hatten und den für die Mitte der 70er Jahre typischen
Grad des Verständnisses der Ideen des Wissenschaftlers und der ideologischen
Prägungen dieser Zeit widerspiegelten.

1991 veröffentlichte die Akademische Kommission für das Studium von Vernadskys
Erbe das Werk „Scientific Thought as a Planetary Phenomenon“ als separates
Buch (zusammengestellt von F.T. Yanshina, Vorwort und Anmerkungen von
A.L. Yanshin und F.T. Yanshina). Leider ist auch diese neue, dritte Auflage
nicht frei von gravierenden Mängeln. Der Verfasser und Herausgeber druckte
schließlich die leidgeprüften Absätze 151-156, nahm sie aber nicht in den
Haupttext auf und platzierte sie sorgfältig am Ende des Buches unter den
“Anhängen“.

Diese Ausgabe von „Wissenschaftliches Denken als planetares Phänomen“ gibt
den vollstän"|digen Text des Manuskripts wieder, das Ende der 1960er Jahre im
Archiv der Akademie der Wissenschaften für den Druck vorbereitet wurde. Zuvor
wurde sie nochmals mit dem Original verglichen und korrigiert (I.I. Mochalow,
W.S. Napolitanskaja, M.J. Sorokina und A.A. Jaroschewski). Die Verweise werden
auf den aktuellen Stand gebracht. Das Werk ist ausnahmslos gedruckt; die
Absätze 151-156 sind in der von V.I. Vernadsky selbst festgelegten Reihenfolge
in den Text eingefügt. Die Kommentare werden neu abgegeben. Sie befassen sich
hauptsächlich mit den grundlegenden Begriffen von Bedeutung und Originalität,
die von Vernadsky eingeführt oder übernommen wurden, und enthüllen die
Geschichte ihrer Entstehung, Reflexion und Entwicklung in den Werken, die in
den Jahren vor der Arbeit am „Wissenschaftlichen Denken“ geschrieben wurden.

\begin{thebibliography}{xxx}
\bibitem{Deborin1932} A.M. Deborin. \foreignlanguage{russian}{Проблемы времени
  в освещении акад. В. И. Вернадского} (Probleme unserer Zeit und die Sicht
  von Akad. V.I. Vernadsky). In: Izv. AdW der UdSSR. Serie 7.  Abteilung
  Mathematik und Naturwissenschaften, 1932. N 7.
\bibitem{Mikulinsky1988} S.R. Mikulinsky.
  \foreignlanguage{russian}{В.И. Вернадский как историк науки} (V.I. Vernadsky
  als Wissenschaftshistoriker). In: \cite{Vernadsky1988}.
\bibitem{Mochalov1982} I.I. Mochalov. Vladimir Ivanovich Vernadsky. M .:
  Nauka, 1982. 
\bibitem{Pavlov1922} A.P. Pavlov. \foreignlanguage{russian}{Ледниковые и
  межледниковые эпохи Европы в связи с историей ископаемого человека}
  (Glazial- und Interglazial-Epochen Europas und die Geschichte des fossilen
  Menschen). Akademische Rede, 1922. N 2.
\bibitem{Rosov1993} V.A.Rosov. \foreignlanguage{russian}{В.И. Вернадский и
  русские востоковеды} (V.I. Vernadsky und russische Orientalisten).
  St. Petersburg, 1993.
\bibitem{Schuchert1933} C. Schuchert, C. Dunbar. A Text Book of Geology.  NY,
  1933.
\bibitem{Vernadsky1933} V.I. Vernadsky. \foreignlanguage{russian}{По поводу
  критических замечаний акад. А. М. Деборина} (Zu den kritischen Bemerkungen
  von Akad. A. M. Deborin). In: Izv. AdW der UdSSR.  Serie 7. Abteilung
  Mathematik und Naturwissenschaften, 1933. N 3.
\bibitem{Vernadsky1944} V.I. Vernadsky. \foreignlanguage{russian}{Несколько
  слов о ноосфере} (Ein paar Worte zur Noosphäre). In:
  \foreignlanguage{russian}{Успехи современной биологии} (Erfolge in der
  modernen Biologie). 1944. N. 18, Nr. 2, S. 113--120.
\bibitem{Vernadsky1960}  V.I. Vernadsky. \foreignlanguage{russian}{История
  природных вод} (Geschichte der natürlichen Gewässer).  Ausgewählte Werke.
  M.: Verlag der AdW der UdSSR, 1960. Band 4, Buch 2.
\bibitem{Vernadsky1988} V.I. Vernadsky. \foreignlanguage{russian}{Труды по
  всеобщей истории науки} (Arbeiten zur allgemeinen Wissenschaftsgeschichte).
  M.: Nauka, 1988.
\bibitem{Vernadsky1991} V.I. Vernadsky. \foreignlanguage{russian}{Научная
  мысль как планетное явление} (Wissenschaftliches Denken als planetares
  Phänomen).  Verantwortlicher Herausgeber A.L. Yanshin, Moskau, Nauka, 1991.
\bibitem{Vernadsky1992} V.I. Vernadsky. \foreignlanguage{russian}{Труды по
  биогеохимии и геохимии почв} (Arbeiten zur Biogeochemie und Geochemie des
  Bodens). M., 1992.
\bibitem{Vernadsky1997} V.I. Vernadsky. \foreignlanguage{russian}{О науке}
  (Über Wissenschaft), Band 1, \foreignlanguage{russian}{Научное знание.
  Научное творчество. Научная мысль} (Wissenschaftliche Kenntnisse.
  Wissenschaftliche Kreativität. Wissenschaftliches Denken).  Dubna, Phoenix,
  1997.
\end{thebibliography}
\end{document}
