\documentclass[11pt,a4paper]{book}
\usepackage{od}
\usepackage[utf8]{inputenc}
\usepackage[russian,main=ngerman]{babel}

\begin{document}
\begin{titlepage}\centering\mbox{}\vfill
  {\Huge Wissenschaftliches Denken\\[.7em] als planetares Phänomen}
  \vskip3em
  {\LARGE Vladimir Ivanovich Vernadsky}
  \vskip3em
  {\LARGE 1936--38}
  \vfill
  \begin{quote}
  Original: \foreignlanguage{russian}{Научная мысль как планетное явление}.\\
  Quelle: \url{http://vernadsky.lib.ru/e-texts/archive/thought.html}\\
  Übersetzt im April 2020 von Hans-Gert Gräbe, Leipzig.
\end{quote}
\end{titlepage}
\tableofcontents

\chapter[Biosphäre und Denken]{Abteilung 1. Wissenschaftliches Denken und
  Arbeiten als geologische Kraft in der Biosphäre}

\section{Kapitel 1} 

Mensch und Menschheit in der Biosphäre als natürlicher Teil ihrer lebenden
Substanz, als Teil ihrer Organisation. Physikalisch-chemische und geometrische
Heterogenität der Biosphäre: die grundlegende organisierte Unterscheidung --
materiell-energetisch und zeitlich -- ihrer lebenden Materie von ihrer eigenen
Substanz des Kosmischen. Evolution der Arten und Evolution der Biosphäre.
Entdeckung einer neuen geologischen Kraft in der Biosphäre -- der
wissenschaftliche Gedanke der sozialen Menschlichkeit. Seine Manifestation ist
mit der Eiszeit verbunden, in der wir leben, mit einer der wiederkehrenden
geologischen Manifestationen in der Geschichte des Planeten, die durch ihre
Ursache über die Erdkruste hinausgeht.

\paragraph{1.}
Der Mensch ist, wie alle Lebewesen, kein in sich geschlossenes,
umweltunabhängiges Naturobjekt. Doch selbst Naturwissenschaftler unserer Zeit,
die den Menschen und den lebenden Organismus im Allgemeinen der Umwelt ihres
Lebens gegenüberstellen, berücksichtigen dies sehr oft nicht. Aber die
Untrennbarkeit eines lebenden Organismus von der Umwelt kann heute bei einem
modernen Naturforscher keinen Zweifel mehr aufkommen lassen. Biogeochemiker
aus ihr kommt und versucht, diese funktionelle Beziehung genau und vielleicht
tief zu verstehen, auszudrücken und herzustellen. Philosophen und die moderne
Philosophie missachten diese funktionale Abhängigkeit des Menschen als
natürliches Objekt und der Menschheit als Naturphänomen von der Umwelt ihres
Lebens und Denkens mit überwältigender Mehrheit.

Die Philosophie kann dies nicht hinreichend berücksichtigen, weil sie von den
Gesetzen der Vernunft ausgeht, die für sie auf die eine oder andere Weise das
letzte autarke Kriterium ist (selbst in Fällen, wie in den Philosophien der
Religion oder der Mystik, in denen die Grenzen der Vernunft tatsächlich
begrenzt sind).

Ein moderner Wissenschaftler, der von der Erkenntnis der Realität seiner
Umgebung, der zu untersuchenden Welt -- der Natur, des Kosmos oder der
Weltrealität 14 -- ausgeht, kann diese Sichtweise nicht als Ausgangspunkt für
seine wissenschaftliche Arbeit nehmen.

Denn er weiß jetzt mit Sicherheit, dass der Mensch sich nicht auf der
unstrukturierten Oberfläche der Erde befindet, nicht in direktem Kontakt mit
den kosmischen Weiten in der unstrukturierten Natur steht, die ihn natürlich
nicht bindet. Allerdings ist es nicht ungewöhnlich, dass ein moderner
Naturforscher, der tief in den Globus vordringt, dies routinemäßig und unter
dem Einfluss der Philosophie vergisst und damit in seinem Denken nicht
berücksichtigt und nicht geprägt wird.

Der Mensch und die Menschheit sind in erster Linie mit der lebendigen Materie
verbunden, die unseren Planeten bewohnt und von der sie nicht wirklich durch
irgendeinen physikalischen Prozess isoliert werden können. Dies ist nur in
Gedanken möglich.

\paragraph{2.}
Der Begriff des Lebens und Lebenserlebens ist uns im Alltag klar und kann uns
in realen eigenen Manifestationen und in den ihnen entsprechenden Objekten der
Natur -- in natürlichen Körpern -- wissenschaftlich ernsthafte Zweifel
aufkommen lassen. Jahrhundert zum ersten Mal [mit der Entdeckung] von
gefilterten Viren in der Wissenschaft gibt es Fakten, die uns ernsthaft --
nicht philosophisch, sondern wissenschaftlich -- die Frage aufwerfen lassen,
ob wir es mit einem lebenden natürlichen Körper oder mit einem Körper aus
natürlichen nicht lebenden -- schrägen -- Körpern zu tun haben.

Bei Viren wird der Zweifel eher durch wissenschaftliche Beobachtung als durch
philosophische Darstellung verursacht. Das ist der große wissenschaftliche
Wert ihrer Untersuchung. Sie befindet sich jetzt auf dem richtigen und soliden
Weg. Zweifel werden aufgelöst und nichts als ein genaueres Bild des lebenden
Organismus wird geben, mit diesem Ansatz kann nicht scheitern...

Gleichzeitig stoßen wir in der Wissenschaft aber auch auf andere Arten von
Zweifeln, die durch philosophische und religiöse Forschungen verursacht
werden. So erforscht z.B. das Bose-Institut in Kalkutta 15 in seinen Arbeiten
wissenschaftlich die Phänomene, die in der materiell-energetischen Umwelt
Manifestationen, philosophisch gewöhnliches Leben und schiefe natürliche
Körper betreffen. Sie sind nicht charakteristisch, in den kosmischen
Naturkörpern schwach ausgeprägt und in den Lebenden deutlich manifestiert,
sondern beide Gemeinschaften.

Dieses Gebiet, wenn es so existiert, wie Bose es zu etablieren versuchte, die
Phänomene, die den kosmischen und lebenden Naturkörpern gemeinsam sind, bringt
nichts Neues für die scharfe Unterscheidung zwischen ihnen. Sie muss sich auch
in diesem Bereich manifestieren, wenn ihre Existenz bewiesen ist.

Es ist hier nur notwendig, sich den Phänomenen nicht in dem Aspekt zu nähern,
in dem Bose sich ihnen nähert, nicht als Phänomene des Lebens, sondern als
Phänomene lebender Naturkörper, lebender Substanz.

Um Missverständnisse zu vermeiden, werde ich in allen weiteren Ausführungen
den Begriff „Leben“, „Leben“ vermeiden.denn wenn wir sie verlassen würden,
würden wir unweigerlich über die in der Wissenschaft untersuchten Phänomene
des Lebens hinaus in ein der Philosophie fremdes Gebiet oder eine der
Philosophie fremde Wissenschaft oder, wie es am Bose-Institut der Fall ist, in
ein neues Gebiet neuer materiell-energetischer Manifestationen, die allen
natürlichen Körpern der Biosphäre gemeinsam sind und die jenseits der
Grundfrage nach dem lebenden Organismus und der lebenden Materie liegen, die
uns jetzt interessiert.

Ich werde daher die Worte und Begriffe „Leben“ und „Leben“ vermeiden und den
zu untersuchenden Bereich auf die Begriffe „lebender natürlicher Körper“ und
„lebende Materie“ beschränken. Jeder lebende Organismus in der Biosphäre, ein
natürliches Objekt, ist ein lebender natürlicher Körper. Die lebende Materie
der Biosphäre ist die Gesamtheit der in ihr lebenden Organismen.

„Lebende Materie“, so definiert, ist ein Begriff, der recht präzise ist und
die Studienobjekte der Biologie und Biogeochemie vollständig abdeckt. Sie ist
einfach, klar und kann keine Missverständnisse verursachen. Wir studieren in
der Wissenschaft nur den lebenden Organismus und seine Gesamtheit.
Wissenschaftlich gesehen sind sie mit dem Begriff des Lebens identisch.

\paragraph{3.}
Der Mensch ist wie jeder lebende natürliche (oder naturgegebene) Körper
untrennbar mit einer bestimmten geologischen Hülle unseres Planeten verbunden
-- der Biosphäre, die sich von ihren anderen Hüllen stark unterscheidet, deren
Struktur durch ihre eigentümliche Organisation bestimmt wird und die in ihr
als isolierter Teil des natürlich zum Ausdruck kommenden Ganzen Platz nimmt.

Die lebende Materie, wie auch die Biosphäre, hat ihre eigene spezielle
Organisation und kann als eine natürlich ausgedrückte Funktion der Biosphäre
betrachtet werden.

Arrangement ist kein Mechanismus. Der Mechanismus der Organisation
unterscheidet sich schlagartig vom Mechanismus der Organisation dadurch, dass
sie sich ständig in der Entstehung, in der Bewegung all ihrer kleinsten
materiellen und energetischen Teilchen befindet. Im Laufe der Zeit -- in
Verallgemeinerungen der Mechanik und in einem vereinfachten Modell -- können
wir Organisation so ausdrücken, dass niemals einer ihrer Punkte (materiell
oder energetisch) auf natürliche Weise zurückkehrt oder an den gleichen Ort,
an den gleichen Punkt in der Biosphäre gelangt, wie er jemals zuvor war. Sie
kann nur als mathematische Koinzidenz, mit einer sehr geringen
Wahrscheinlichkeit, zu ihr zurückkehren.

Die Hülle der Erde, die Biosphäre, die den ganzen Globus umspannt, hat scharf
isolierte Dimensionen, sie ist weitgehend auf die Existenz einer lebenden
Substanz in ihr zurückzuführen -- sie ist bewohnt. Zwischen seinem kosmischen
leblosen Teil, seinen kosmischen Naturkörpern und der ihn bewohnenden lebenden
Materie findet ein ständiger Stoff- und Energieaustausch statt, der sich
materiell in der Bewegung der Atome durch lebende Materie ausdrückt. Dieser
Austausch im Laufe der Zeit drückt sich in einem sich natürlich verändernden,
ständig nach einem stabilen Gleichgewicht strebenden Gleichgewicht aus. Er
durchdringt die gesamte Biosphäre, und dieser biogene Strom von Atomen erzeugt
ihn weitgehend. Untrennbar und untrennbar ist die Biosphäre während der
gesamten geologischen Zeit so mit der lebenden Substanz, die sie bewohnt,
verbunden.

In diesem biogenen Strom der Atome und in der damit verbundenen Energie
manifestiert sich der planetarische, kosmische Wert der lebenden Materie
scharf. Denn die Biosphäre ist die einzige Erdhülle, in die die kosmische
Energie, die kosmische Strahlung, kontinuierlich eindringt, allen voran die
Sonnenstrahlung, unter Aufrechterhaltung eines dynamischen Gleichgewichts,
Organisation: Biosphäre -- lebende Materie.

Von der geoidalen Ebene erstreckt sich die Biosphäre bis zu den Grenzen der
Stratosphäre und durchdringt diese; sie kann kaum die Ionosphäre erreichen,
das elektromagnetische Vakuum der Erde, das gerade vom wissenschaftlichen
Bewusstsein erfasst wird. Unterhalb des Geoidniveaus dringt lebende Materie in
die Stratosphäre und in die oberen Bereiche der metamorphen und granitischen
Schalen ein. In einem Querschnitt des Planeten erhebt sie sich 20-25 km über
das Geoidniveau und sinkt im Durchschnitt 4-5 km unter dieses Niveau. Diese
Grenzen verändern sich im Laufe der Zeit und erstrecken sich an manchen Orten
auf kleinen, aber dafür weit über sie hinaus. Offenbar muss in den
Meerestiefen lebende Materie an einigen Stellen tiefer als 11 km eindringen,
und ihre Lage ist tiefer als 6 km. 16 In der Stratosphäre erleben wir gerade
das Eindringen des Menschen, der immer untrennbar mit anderen Organismen --
Insekten, Pflanzen, Mikroben -- verbunden ist, und auf diese Weise hat sich
die lebende Materie bereits 40 km von der Ebene des Geoids entfernt und steigt
schnell auf.

Während der geologischen Zeit scheint es einen Prozess der kontinuierlichen
Erweiterung der Grenzen der Biosphäre zu geben: die Besiedlung ihres lebenden
Materials.

\paragraph{4.}
Die Organisation der Biosphäre -- die Organisation der lebenden Materie --
sollte als Gleichgewichte betrachtet werden, die sich bewegen und in
historischer und geologischer Zeit immer um einen genau ausgedrückten
Mittelwert oszillieren. Die Verschiebungen oder Schwankungen dieses
Mittelwertes manifestieren sich kontinuierlich nicht in historischer Zeit,
sondern in geologischer Zeit. Während der geologischen Zeit kehrt in den
zirkulären Prozessen, die für die biogeochemische Organisation
charakteristisch sind, niemals ein Punkt (z.B. ein Atom oder ein chemisches
Element) in die Äonen der Jahrhunderte so zurück wie zuvor.

Dieses charakteristische Merkmal der Biosphäre hat Leibniz [1646-1716] in
einer seiner philosophischen Überlegungen, so scheint es, in der „Theodizee“
sehr anschaulich und bildhaft ausgedrückt. Ende des 17. Jahrhunderts, erinnert
er sich, war er in einer großen säkularen Gesellschaft in einem großen Garten,
und als er über die unendliche Vielfalt der Natur und die unendliche Klarheit
des Geistes sprach, wies Leibniz darauf hin, dass niemals zwei Blätter eines
Baumes oder einer Pflanze ganz identisch sind. Alle Versuche einer großen
Gesellschaft, solche Blätter zu finden, waren natürlich vergeblich. Leibniz
argumentierte hier nicht als Naturbeobachter, der dieses Phänomen als Erster
entdeckte, sondern als Gelehrter, der es aus der Lektüre entnahm. Wir können
verfolgen, dass genau dieses Beispiel des Blattes ein Jahrhundert früher in
der philosophischen Folklore auftauchte. 17

Im Alltag manifestiert sich dies für uns in der Persönlichkeit, in der
Abwesenheit von zwei identischen Individualitäten, die nicht voneinander
unterscheidbar sind. In der Biologie zeigt sich, dass jedes durchschnittliche
Individuum einer lebenden Substanz sowohl in seinen chemischen Verbindungen
chemisch unterscheidbar ist als auch offensichtlich in seinen chemischen
Elementen seine speziellen Verbindungen hat.

\paragraph{5.}
Ihre physikalisch-chemische und geometrische Heterogenität (Kap. 47) ist
äußerst charakteristisch für die Struktur der Biosphäre. Es besteht aus
Lebens- und Sensensubstanz, die während der gesamten geologischen Zeit durch
ihre Genese und Struktur scharf getrennt sind. Lebende Organismen, d.h. alle
lebende Materie, werden aus lebender Materie geboren und bilden während der
Zeit Generationen, die niemals direkt außerhalb desselben lebenden Organismus
aus irgendeiner kosmischen Materie des Planeten erscheinen. Es besteht jedoch
eine kontinuierliche, niemals endende Verbindung zwischen kosmischer und
lebender Materie, die sich als kontinuierlicher biogener Strom von Atomen von
der lebenden Materie zur kosmischen Materie der Biosphäre und zurück
ausdrücken lässt. Dieser biogene Strom von Atomen wird durch lebende Materie
verursacht. Es drückt sich in ihrer nie endenden Atmung, Ernährung, Vermehrung
usw. aus.

In der Biosphäre ist diese Heterogenität ihrer Struktur, die sich über die
gesamte geologische Zeitspanne fortsetzt, der wichtigste dominierende Faktor,
der sie von allen anderen Schalen der Erde scharf unterscheidet.

Sie geht tiefer als die üblicherweise in der Naturwissenschaft untersuchten
Phänomene -- in den Eigenschaften der Raumzeit, zu denen erst in unserer Zeit,
im XX.

Lebende Materie bedeckt die gesamte Biosphäre, erschafft und verändert sie,
aber in Bezug auf Gewicht und Volumen ist sie nur ein kleiner Teil von
ihr. Indirekte, nicht lebende Materie ist vorherrschend; das Volumen wird von
verdünnten Gasen, festem Gestein und in geringerem Maße vom flüssigen
Meerwasser des Weltozeans dominiert. Selbst bei höchsten Konzentrationen, in
Ausnahmefällen und in kleinen Massen macht die lebende Materie zehnprozentige
Anteile an der Materie der Biosphäre aus und beträgt im Durchschnitt kaum ein
bis zwei Hundertstel Gewichtsprozent. Aber geologisch gesehen ist sie die
größte Kraft in der Biosphäre und, wie wir sehen werden, alle Prozesse, die in
ihr ablaufen und enorme freie Energie entwickeln, wodurch eine große
geologisch manifestierte Kraft in der Biosphäre entsteht, deren Kraft noch
nicht quantifiziert werden kann, aber vielleicht alle anderen geologischen
Manifestationen in der Biosphäre übertrifft.

In diesem Zusammenhang ist es angebracht, einige neue Grundkonzepte
einzuführen, mit denen wir uns in der gesamten künftigen Präsentation befassen
werden.

\paragraph{6.}
Dies sind die Begriffe, die mit den Begriffen des natürlichen Körpers
(Naturgegenstand) und des Naturphänomens verbunden sind. Oft wurden sie als
natürliche Körper oder Phänomene bezeichnet.

Lebende Materie ist ein natürlicher Körper oder ein natürliches Phänomen in
der Biosphäre. Die Konzepte des natürlichen Körpers oder Naturphänomens, die
bisher wenig logisch erforscht sind, stellen die Grundbegriffe der
Naturwissenschaft dar. Für unsere Zwecke ist es nicht notwendig, ihre logische
Analyse zu vertiefen. Dies sind Körper oder Phänomene, die durch natürliche
Prozesse entstanden sind -- natürliche Objekte.

Die natürlichen Körper der Biosphäre sind nicht nur lebende Organismen,
lebende Materie, sondern der Großteil der Materie der Biosphäre wird von
Körpern oder unbelebten Phänomenen gebildet, die ich als kosmisch bezeichnen
möchte. Dies sind z.B. Gase, Atmosphäre, Gestein, chemisches Element, Atom,
Quarz, Serpentin, usw.

Neben den lebenden und schrägen Naturkörpern in der Biosphäre, ihren
natürlichen Strukturen, spielen heterogene Naturkörper, wie Boden, Schluff,
Oberflächenwasser, die Biosphäre selbst usw., die aus lebenden und schrägen
Naturkörpern bestehen, die gleichzeitig nebeneinander existieren und komplexe
natürliche Schrägstrukturen bilden, eine große Rolle. Ich werde diese
komplexen Naturkörper als biokosmische Naturkörper bezeichnen. Die Biosphäre
selbst ist ein komplexer planetarischer biokosmischer Naturkörper.

Der Unterschied zwischen lebenden und schrägen Naturkörpern ist, wie wir in
Zukunft sehen werden, so groß, dass der Übergang vom einen zum anderen in
irdischen Prozessen nie beobachtet wird; nirgendwo und niemals begegnen wir
ihm in der wissenschaftlichen Arbeit. Wie wir sehen werden, ist er tiefer, als
wir über physikalische und chemische Phänomene wissen.

Die damit verbundene Heterogenität der Biosphärenstruktur, der starke
Unterschied zwischen ihrer Substanz und ihrer Energie in Form von lebenden und
schrägen Naturkörpern ist ihre Hauptmanifestation.

\paragraph{7.}
Eine der Manifestationen dieser Heterogenität der Biosphäre ist, dass die
Prozesse in lebender Materie scharf anders ablaufen als in kosmischer Materie,
wenn wir sie unter dem Aspekt der Zeit betrachten. In der lebenden Materie
bewegen sie sich auf der Skala der historischen Zeit, in der kosmischen
Materie -- auf der Skala der geologischen Zeit, deren „Sekunde“ viel kleiner
ist als die Dekamyriade, d.h. hunderttausend Jahre historische Zeit 18.
Außerhalb der Biosphäre ist dieser Unterschied noch ausgeprägter, und in der
Lithosphäre beobachten wir wegen der überwältigenden Masse ihrer Substanz die
Organisation, in der sich die meisten Atome, wie die radioaktive Forschung
zeigt, nicht bewegungslos bewegen, für uns spürbar innerhalb von Zehntausenden
von Dekamyriaden -- dem Zeitbereich, der uns jetzt für unsere Messungen zur
Verfügung steht.

Bis vor kurzem herrschte in der Geologie die Vorstellung vor, dass Geologen
die Manifestation geologisch langwieriger Veränderungen, die während der Ära
der menschlichen Existenz auftraten, nicht studieren konnten. In meiner Jugend
wurde gelehrt und gedacht, dass der Klimawandel, die Orografie, die Entstehung
neuer Arten von Organismen in der geologischen Forschung in der Regel nicht
manifest ist, ist für den Geologen kein aktuelles Phänomen. Nun hat sich
dieses naturalistische ideologische Umfeld dramatisch verändert, und wir sehen
immer mehr der uns umgebenden geologischen Kräfte in Aktion. Dies fiel, kaum
zufällig, mit der Einführung des Glaubens an die geologische Bedeutung des
Homo sapiens in das wissenschaftliche Bewusstsein zusammen, mit der Entdeckung
eines neuen Zustands der Biosphäre -- der Noosphäre -- und ist eine seiner
Ausdrucksformen. Es geht natürlich in erster Linie um die Klärung
naturwissenschaftlichen Arbeitens und Denkens innerhalb der Biosphäre, in der
lebende Materie eine große Rolle spielt.

Die abrupt unterschiedliche Manifestation des Lebendigen und des Schrägen in
der Biosphäre unter dem Aspekt der Zeit ist bei aller Bedeutung ein privater
Ausdruck eines viel größeren Phänomens, das sich in der Biosphäre bei jedem
Schritt widerspiegelt.

\paragraph{8.}
Die lebende Materie der Biosphäre unterscheidet sich von ihrer kosmischen
Substanz in zwei Hauptprozessen von großer geologischer Bedeutung, die der
Biosphäre ein völlig anderes Aussehen verleihen, das es für keine andere Hülle
des Planeten gibt. Diese beiden Prozesse manifestieren sich nur vor dem
Hintergrund der geologischen Zeit. Sie hören manchmal auf, gehen aber nie
rückwärts.

Erstens nimmt im Laufe der geologischen Zeit die Nachweiskraft lebender
Substanz in der Biosphäre zu, ihre Bedeutung in der Biosphäre und ihr Einfluss
auf die kosmische Substanz der Biosphäre. Diesem Prozess wurde bisher wenig
Beachtung geschenkt. In Zukunft werde ich mich ständig damit befassen müssen.

Viel mehr Aufmerksamkeit wurde dem anderen Prozess geschenkt, der allen
bekannt ist und der das gesamte wissenschaftliche Denken des XIX. und
XX. Jahrhunderts seit der Mitte des XIX. Es handelt sich um einen Prozess der
Spezies-Evolution während der geologischen Zeit -- eine scharfe Veränderung in
den lebenden natürlichen Körpern selbst.

Nur in der lebenden Materie beobachten wir eine starke Veränderung der
natürlichen Körper selbst im Laufe der geologischen Zeit. Einige Organismen
ziehen in andere ein, sterben aus, wie wir sagen, oder verändern sich radikal.

Lebende Materie ist plastisch, verändert sich, passt sich an Veränderungen in
der Umwelt an, hat aber vielleicht einen eigenen Evolutionsprozess, der sich
unabhängig von Veränderungen in der Umwelt im Laufe der geologischen Zeit in
Veränderungen manifestiert. Dies kann durch das kontinuierliche, mit
Unterbrechungen erfolgende Wachstum des Zentralnervensystems der Tiere während
der geologischen Zeit, in seiner Bedeutung in der Biosphäre und in der Tiefe
der Reflexion lebender Materie auf die Umwelt 19, in ihrem Eindringen in sie
[die Umwelt] angezeigt werden.

Die Plastizität der lebenden Materie ist offensichtlich ein sehr komplexes
Phänomen, da es Organismen gibt, die sich in ihrer morphologischen und
physiologischen Struktur [von] Hunderten von Millionen Jahren bis zu
fünfhundert Millionen und mehr, also unzähligen Generationen, für uns nicht
merklich verändern. Diese so genannte Persistenz 20 ist leider ein in der
Biologie extrem schlecht untersuchtes Phänomen. Als ein für lebende Materie
übliches Phänomen beobachten wir in ihr jedoch einen plastischen
Evolutionsprozess, von dem selbst ein Zeichen in den kosmischen Naturkörpern
nicht vorhanden ist. Für letztere sehen wir heute die gleichen Mineralien, die
gleichen Prozesse ihrer Entstehung, die gleichen Gesteine usw., wie vor zwei
Milliarden Jahren.

Der evolutionäre Prozess der lebendigen Materie, der während der geologischen
Zeit kontinuierlich abläuft, erstreckt sich über die gesamte Biosphäre und auf
verschiedene Weise, weniger dramatisch, aber mit Auswirkungen auf ihre
kosmischen Naturkörper. Allein dadurch können und müssen wir über den
evolutionären Prozess der Biosphäre selbst sprechen, der sich in der trägen
Masse ihrer kosmischen und lebenden natürlichen Körper abspielt und sich im
Laufe der geologischen Zeit deutlich verändert.

Dank der kontinuierlichen und nie endenden Evolution der Arten verändert sich
das Spiegelbild der lebenden Materie in der Umwelt dramatisch. Dadurch wird
der Evolutionsprozess -- die Veränderungen -- in die natürlichen biokosmischen
und biogenen Körper, die eine wichtige Rolle in der Biosphäre spielen,
übertragen -- in die Böden, in Oberflächen- und Untergrundgewässer (Meere,
Seen, Flüsse usw.), in Kohlen, Bitumen, Kalksteine, organogene Erze usw. Böden
und Flüsse aus dem Devon zum Beispiel unterscheiden sich von Böden aus dem
Tertiär und der Gegenwart. Dies ist ein Bereich neuer Phänomene, der vom
wissenschaftlichen Denken kaum berücksichtigt wird. Die Evolution der Arten
verlagert sich in die Evolution der Biosphäre.

\paragraph{9.}
Der Evolutionsprozess ist von besonderer geologischer Bedeutung, weil er eine
neue geologische Kraft geschaffen hat, den wissenschaftlichen Gedanken der
sozialen Menschheit.

Wir erleben gerade seinen hellen Einzug in die geologische Geschichte des
Planeten. In den letzten Jahrtausenden hat der Einfluss einer Spezies lebender
Materie -- der zivilisierten Menschheit -- auf die Veränderung der Biosphäre
intensiv zugenommen. Unter dem Einfluss von wissenschaftlichem Denken und
menschlicher Arbeit geht die Biosphäre in einen neuen Zustand über -- die
Noosphäre.

Die Menschheit durch eine natürliche Bewegung von einer Million -- ein
weiteres Jahr, mit zunehmender Geschwindigkeit in seiner Manifestation,
erstreckt sich über den gesamten Planeten, hebt sich ab, entfernt sich von
anderen lebenden Organismen als eine neue beispiellose geologische Kraft. Mit
einer Rate, die mit der der Multiplikation vergleichbar ist, ausgedrückt durch
die geometrische Progression über die Zeit, entstehen auf diese Weise in der
Biosphäre eine wachsende Zahl neuer kosmischer Naturkörper und neuer großer
Naturphänomene.

Vor unseren Augen verändert sich die Biosphäre dramatisch. Und es kann kaum
ein Zweifel daran bestehen, dass die Verwandlung des auf diese Weise durch die
organisierte menschliche Arbeit manifestierten wissenschaftlichen Denkens kein
zufälliges, vom Willen des Menschen abhängiges Phänomen ist, sondern dass es
sich um einen natürlichen Prozess handelt, dessen Wurzeln tief liegen und der
durch den evolutionären Prozess vorbereitet wurde, dessen Dauer in Hunderten
von Millionen Jahren gezählt wird.

Der Mensch muss begreifen, sobald ihn der wissenschaftliche und nicht der
philosophische oder religiöse Friedensbegriff umarmt, dass er kein zufälliges,
von der Biosphäre oder Noosphäre unabhängiges, frei handelndes Naturphänomen
ist. Es ist die unvermeidliche Manifestation eines großen natürlichen
Prozesses, der natürlich mindestens zwei Milliarden Jahre dauern wird.

Gegenwärtig hört man unter dem Einfluss der umgebenden Schrecken des Lebens,
zusammen mit der beispiellosen Blüte des wissenschaftlichen Denkens, von der
herannahenden Barbarei, dem Zusammenbruch der Zivilisation, der
Selbstzerstörung der Menschheit. Ich glaube, dass diese Gefühle und diese
Urteile das Ergebnis eines unzureichenden Eindringens in die Umwelt sind. Das
wissenschaftliche Denken ist noch nicht zum Leben erwacht, wir leben unter dem
starken Einfluss von noch nicht der Realität des modernen Wissens
entsprechenden, noch nicht überholten philosophischen und religiösen
Fähigkeiten.

Wissenschaftliche Erkenntnisse, die sich als eine geologische Kraft
manifestieren, die die Noosphäre schafft, können nicht zu Ergebnissen führen,
die dem von ihr geschaffenen geologischen Prozess widersprechen. Es ist kein
zufälliges Phänomen -- seine Wurzeln liegen extrem tief.

\paragraph{10.}
Dieser Prozess hängt mit der Entstehung des menschlichen Gehirns zusammen. In
der Geschichte der Wissenschaft wurde sie in Form einer empirischen
Verallgemeinerung von einem tiefgründigen amerikanischen Naturforscher, einem
bedeutenden Geologen, Zoologen, Paläontologen und Mineralogen D.-D. Dana
[1813-1895] in New Haven aufgedeckt. Er veröffentlichte seine Schlussfolgerung
vor fast 80 Jahren. Auf seltsame Weise ist diese Verallgemeinerung noch nicht
ins Leben eingetreten, fast vergessen und noch nicht richtig
entwickelt. Darauf werde ich später zurückkommen. An dieser Stelle möchte ich
auch anmerken, dass Dan seine empirische Verallgemeinerung in der Sprache der
Philosophie und Theologie präsentiert hat, und sie schien mit wissenschaftlich
nicht akzeptablen Ideen verbunden zu sein.

Beim Sprechen der modernen wissenschaftlichen Sprache bemerkte Dana, dass im
Laufe der geologischen Zeit auf unserem Planeten [at] einige seiner Bewohner
immer perfekter erscheinen als das, was vorher auf ihm existierte -- der
zentrale Nervenapparat -- das Gehirn. Dieser Prozess, der als Enzephalose
bezeichnet wird, wird nie rückgängig gemacht, [obwohl] er viele Male, manchmal
für Millionen von Jahren, zum Stillstand kommt. Der Prozess wird also durch
den polaren Zeitvektor ausgedrückt, dessen Richtung sich nicht ändert. Wir
werden sehen, dass der geometrische Zustand des von lebender Materie
eingenommenen Raumes durch polare Vektoren gekennzeichnet ist, es gibt keinen
Raum für Geraden.

Die Entwicklung der Biosphäre steht in Zusammenhang mit der Stärkung des
Evolutionsprozesses der lebenden Materie.

Wir wissen heute, dass die Geschichte der Erdkruste kritische Perioden
offenbart, in denen die geologische Aktivität in ihren verschiedenen
Erscheinungsformen in ihrem eigenen Tempo zunimmt. Dieser Anstieg ist
natürlich in historischer Zeit unsichtbar und kann wissenschaftlich nur auf
der Skala der geologischen Zeit beobachtet werden.

Diese Zeiträume können in der Geschichte des Planeten als kritisch angesehen
werden, und alles deutet darauf hin, dass sie durch Prozesse tief in der
Erdkruste, offenbar jenseits ihrer Grenzen, verursacht werden. Gleichzeitig
gibt es eine Zunahme vulkanischer, orogener, glazialer, meeresbezogener und
anderer geologischer Prozesse, die den größten Teil der Biosphäre gleichzeitig
über ihre gesamte Länge bedecken. 21 Der evolutionäre Prozess fällt in seiner
Verstärkung, in seinen größten Veränderungen mit diesen Perioden
zusammen. Während dieser Zeiträume entstehen große und tiefgreifende
Veränderungen in der Struktur der lebenden Materie, was die Tiefe der
geologischen Bedeutung dieser plastischen Reflexion der lebenden Materie über
die auf dem Planeten stattfindenden Veränderungen anschaulich zum Ausdruck
bringt.

Es gibt keine Theorie, keine exakte wissenschaftliche Erklärung für dieses
Hauptphänomen in der Geschichte des Planeten. Sie wurde empirisch und
unbewusst geschaffen -- ist unmerklich in die Wissenschaft eingedrungen, und
ihre Geschichte wird nicht geschrieben. Amerikanische Geologen, insbesondere
D.-D. Dana, spielten dabei eine große Rolle. Sie hat den wissenschaftlichen
Gedanken unseres Jahrhunderts eingefangen.

Es ist jedoch möglich und notwendig, sich ihr mit Maß und Zahl zu nähern. Es
ist möglich, die geologische Dauer ihrer Dauer zu messen und so die
Veränderung der Geschwindigkeit geologischer Prozesse numerisch zu
charakterisieren. Dies ist eine der engsten Aufgaben in der Radiologie.

\paragraph{11.}
Solange dies nicht geschehen ist, sollten wir zur Kenntnis nehmen und
berücksichtigen, dass der Prozess der Entwicklung der Biosphäre, ihr Übergang
zur Noosphäre, eindeutig eine Beschleunigung des Tempos der geologischen
Prozesse zeigt. Diese Veränderungen, die sich jetzt in der Biosphäre während
[der letzten] paar tausend Jahre aufgrund des Wachstums des wissenschaftlichen
Denkens und der sozialen Aktivität der Menschheit zeigen, hat es in der
Geschichte der Biosphäre noch nie gegeben.

Dies sind zumindest die Ideen, die wir heute aus dem Studium der Evolution der
Organismen im Laufe der geologischen Zeit ableiten können. Für geologische
Zeit ist die Dekamyriade viel weniger als eine Sekunde der historischen
Zeit. Daher wird es auf der Skala von geologischen tausend Jahren mehr als 300
Millionen Jahre geologische Zeit geben. Dies steht nicht im Widerspruch zu den
großen Veränderungen in der Biosphäre, die z.B. im Kambrium stattfanden, als
die Kalkskelettteile der makroskopischen Meeresorganismen entstanden, oder
[im] Paläozän, als die Säugetierfauna wuchs. 22 Wir dürfen die Tatsache nicht
aus den Augen verlieren, dass die Zeit, die wir geologisch erleben, einer so
kritischen Periode entspricht, da die Eiszeit noch nicht vorbei ist -- die
Geschwindigkeit der Veränderungen ist so langsam, dass die Menschen sie nicht
bemerken.

Der Mensch und die Menschheit und sein Reich in der Biosphäre liegen ganz in
diesem Zeitraum und gehen nicht darüber hinaus.

Es ist möglich, die Entwicklung der Biosphäre von Algongka aus, genauer gesagt
von Kambrien aus während 500-800 Millionen Jahren darzustellen. Die Biosphäre
ist mehr als einmal in einen neuen evolutionären Zustand übergegangen. Es gab
neue geologische Manifestationen der Biosphäre, die es zuvor noch nie gegeben
hatte. Das war zum Beispiel im Kambrium, als große Organismen mit kalkhaltigen
Skeletten auftauchten, oder im Tertiär (vielleicht das Ende der Kreidezeit)
vor 15-80 Millionen Jahren, als unsere Wälder und Steppen entstanden und sich
das Leben der großen Säugetiere entwickelte. Das erleben wir auch jetzt, in
den letzten 10-20 Tausend Jahren, als der Mensch, nachdem er einen
wissenschaftlichen Gedanken im sozialen Umfeld entwickelt hat, eine neue
geologische Kraft in der Biosphäre schafft, die nicht in ihr war. Die
Biosphäre ist vergangen oder, genauer gesagt, sie wird durch das
wissenschaftliche Denken der sozialen Menschheit in einen neuen evolutionären
Zustand -- in eine Noosphäre -- verwandelt.

Die Irreversibilität des evolutionären Prozesses ist eine Manifestation des
charakteristischen Unterschieds zwischen der lebenden Materie in der
geologischen Geschichte des Planeten und seinen kosmischen Naturkörpern und
-prozessen. Man kann sehen, dass es mit besonderen Eigenschaften des von einem
Körper aus lebenden Organismen eingenommenen Raumes zusammenhängt, mit seiner
besonderen geometrischen Struktur, wie P. Curie sagte, mit einer besonderen
Bedingung des Raumes. Im Jahr 1862 verstand L. Pasteur zum ersten Mal die
grundlegende Bedeutung dieses Phänomens, das er als erfolglose Asymmetrie
bezeichnete23. Er studierte dieses Phänomen in einem anderen Aspekt, in der
Ungleichheit der linken und rechten Phänomene in den Körper, in der Existenz
für sie richtig und leviznye24. Geometrisch korrekt und leviznös kann sich nur
in dem Raum manifestieren, in dem die Vektoren polar und enantiomorph
sind. Wahrscheinlich hängt das Fehlen gerader Linien und die stark ausgeprägte
Krümmung der Lebensformen mit dieser geometrischen Eigenschaft zusammen. Ich
werde in Zukunft auf diese Frage zurückkommen, aber jetzt halte ich es für
notwendig festzustellen, dass wir es im Inneren von Organismen offenbar mit
einem Raum zu tun haben, der den Raum von Euklid nicht beantwortet und eine
der Raumformen von Roman beantwortet.

Wir haben jetzt das Recht, in dem Raum, in dem wir leben, die Manifestation
von geometrischen Eigenschaften zuzulassen, die allen drei Formen der
Geometrie -- Euklid, Lobachevsky und Roman -- entsprechen. Ob diese logisch
nicht zu leugnende Schlussfolgerung richtig ist, wird die weitere Forschung
zeigen25. Leider ist eine Vielzahl empirischer Beobachtungen, auch
wissenschaftlich fundierte, von Biologen in ihrer Bedeutung nicht aufgenommen
worden und nicht in ihr wissenschaftliches Weltbild eingegangen. Unterdessen
kann, wie P. Curie gezeigt hat, ein solcher besonderer Raumzustand nicht ohne
besondere Umstände im gewöhnlichen Raum entstehen; ein unsymmetrisches
Phänomen sollte in seiner Sprache immer durch denselben unsymmetrischen Grund
verursacht werden. Die grundlegende empirische Verallgemeinerung, dass das
Lebende nur aus dem Lebendigen kommt und dass der Organismus aus dem
Organismus geboren wird, beantwortet dies. Geologisch manifestiert sich dies
in der Tatsache, dass wir in der Biosphäre eine unüberwindbare Grenze zwischen
lebenden und kosmischen Naturkörpern und -prozessen sehen, die in keiner
anderen Erdhülle zu beobachten ist. Es gibt zwei abrupt materiell [und]
energetisch unterschiedliche Umgebungen, die sich gegenseitig durchdringen und
ihre Konstruktionsatome verändern, verbunden mit dem biogenen Strom der
chemischen Elemente. Auf dieses Phänomen werde ich später noch ausführlicher
eingehen.
\end{document}
13. Wir erleben jetzt die außergewöhnliche Manifestation lebender Materie in der Biosphäre, die genetisch mit der Entdeckung des Homo sapiens vor Hunderttausenden von Jahren verbunden ist und auf diese Weise eine neue geologische Kraft geschaffen hat, einen wissenschaftlichen Gedanken, der den Einfluss der lebenden Materie auf die Entwicklung der Biosphäre dramatisch verstärkt. Die Biosphäre, die vollständig von lebender Materie bedeckt ist, scheint ihre geologische Kraft in unendlicher Größe zu vergrößern und bewegt sich, verarbeitet durch das wissenschaftliche Denken des Homo sapiens, in ihren neuen Zustand -- die Noosphäre. 

Wissenschaftliches Denken als Manifestation lebender Materie kann nicht im Wesentlichen ein reversibles Phänomen sein -- es kann in seiner Bewegung stehen bleiben, aber einmal geschaffen und in der Entwicklung der Biosphäre manifestiert, birgt es die Möglichkeit einer unbegrenzten Entwicklung im Laufe der Zeit. In dieser Hinsicht ist der Verlauf des wissenschaftlichen Denkens, zum Beispiel bei der Schaffung von Maschinen, wie seit langem zu beobachten ist, dem Verlauf der Reproduktion von Organismen recht ähnlich. 

In einer kosmischen Umwelt ist die Biosphäre nicht irreversibel. In ihr herrschen reversible zirkuläre physikalisch-chemische und geochemische Prozesse scharf vor. Lebende Materie dringt mit ihren physikalischen und chemischen Manifestationen der Dissonanz in sie ein. 26 

Das Wachstum des wissenschaftlichen Denkens, das eng mit dem Wachstum der menschlichen Besiedlung in der Biosphäre -- der Reproduktion des Menschen und seiner Kultur der lebenden Materie in der Biosphäre -- verbunden ist, sollte auf die der lebenden Materie fremde Umwelt beschränkt und von ihr unter Druck gesetzt werden. Denn dieses Wachstum steht im Zusammenhang mit der Menge an rasch zunehmender lebender Materie, die direkt und indirekt an der wissenschaftlichen Arbeit beteiligt ist. 

Dieses Wachstum und der damit verbundene Druck nehmen dadurch zu, dass sich in diesem Werk die Wirkung der Masse der geschaffenen Maschinen scharf manifestiert, deren Zunahme in der Noosphäre den gleichen Gesetzen unterliegt wie die Reproduktion der lebenden Materie selbst, d.h. sich in geometrischen Progressionen ausdrückt. 

Sowohl die Fortpflanzung der Organismen manifestiert sich im Druck der lebenden Materie in der Biosphäre, als auch der Verlauf der geologischen Manifestation des wissenschaftlichen Denkens drückt die von ihm geschaffenen Instrumente auf das Kosmische, seine abschreckende Umgebung der Biosphäre, und schafft eine Noosphäre, das Reich des Geistes. 

Die Geschichte des wissenschaftlichen Denkens, der wissenschaftlichen Erkenntnis, ihres historischen Verlaufs erscheint von einer neuen Seite, die noch nicht ausreichend verwirklicht ist. Sie kann nicht nur als die Geschichte einer der Geisteswissenschaften angesehen werden. Diese Geschichte ist gleichzeitig die Geschichte der Entstehung einer neuen geologischen Kraft in der Biosphäre, ein wissenschaftlicher Gedanke, der in der Biosphäre bisher nicht vorhanden war. Es ist die Geschichte der Manifestation eines neuen geologischen Faktors, eines neuen Ausdrucks der Organisation der Biosphäre, die sich in den letzten Zehntausenden von Jahren spontan als Naturphänomen entwickelt hat. Es ist nicht zufällig, wie jedes Naturphänomen, es ist natürlich, wie der paläontologische Prozess, der das Gehirn des Homo sapiens und das soziale Umfeld geschaffen hat, in dem in der Folge als ein damit verbundener natürlicher Prozess wissenschaftliches Denken entsteht, eine neue geologische bewusst gelenkte Kraft. 

Aber die Geschichte der wissenschaftlichen Erkenntnis, selbst als die Geschichte einer der Geisteswissenschaften, ist noch nicht verwirklicht und geschrieben worden. Es gibt nicht einen einzigen Versuch, dies zu tun. Erst in den letzten Jahren beginnt sie für uns kaum über die „biblische“ Zeit hinauszugehen, sie beginnt, die Existenz eines einzigen Zentrums ihres Ursprungs irgendwo innerhalb der zukünftigen mediterranen Kultur vor acht- bis zehntausend Jahren zu klären. Erst mit großen Lücken beginnen wir, kulturelle Überreste zu identifizieren, für uns unerwartete, von der Menschheit fest vergessene wissenschaftliche Fakten festzustellen und zu versuchen, sie mit neuen empirischen Verallgemeinerungen zu erfassen27. 



Kapitel 2 



Die Manifestation des historischen Moments, der als geologischer Prozess erlebt wird. Die Entwicklung der Arten lebender Materie und die Entwicklung der Biosphäre zur Noosphäre. Diese Entwicklung lässt sich durch den Lauf der Weltgeschichte nicht aufhalten. Wissenschaftliches Denken und das menschliche Leben als seine Manifestation. 





14. Wir sind uns noch nicht ganz im Klaren darüber, dass wir nicht alle Konsequenzen dieser wunderbaren, beispiellosen Zeit, in der die Menschheit in das XX. 

Wir leben an einem Wendepunkt, in einer äußerst wichtigen, im Wesentlichen neuen Ära des menschlichen Lebens, seiner Geschichte auf unserem Planeten. 

Zum ersten Mal umarmte der Mensch sein Leben, seine Kultur, die gesamte obere Schale des Planeten -- im Allgemeinen die gesamte Biosphäre, das gesamte mit dem Leben verbundene Gebiet des Planeten. 

Wir sind präsent und maßgeblich an der Schaffung eines neuen geologischen Faktors in der Biosphäre beteiligt, der in seiner Macht und Gemeinschaft beispiellos ist. 

Sie wurde in den letzten 20 bis 30 Tausend Jahren wissenschaftlich nachgewiesen, manifestiert sich aber im letzten Jahrtausend eindeutig mit zunehmender Geschwindigkeit. 

Nach vielen Hunderttausenden von Jahren unermüdlichen natürlichen Strebens ist die gesamte Oberfläche der Biosphäre mit einer einzigen sozialen Tierart des Tierreichs bedeckt -- dem Menschen. Es gibt keinen Winkel auf der Erde, der für sie unzugänglich ist. Es gibt keine Grenzen für seine mögliche Vermehrung. Durch wissenschaftliches Denken und staatlich organisierte, von ihr gelenkte Technik schafft das menschliche Leben eine neue biogene Kraft in der Biosphäre, lenkt ihre Fortpflanzung und schafft günstige Bedingungen für ihre Besiedlung von Teilen der Biosphäre, die zuvor nicht in sein Leben eingedrungen waren und sogar jegliches Leben stellen. 





Theoretisch sehen wir keine Grenze für ihre Möglichkeiten, wenn wir die Arbeit von Generationen berücksichtigen; jeder geologische Faktor manifestiert sich in der Biosphäre in seiner ganzen Kraft nur in der Arbeit von Generationen von Lebewesen, in geologischer Zeit. Aber mit der rasch zunehmenden Genauigkeit der wissenschaftlichen Arbeit -- in diesem Fall der Methoden der wissenschaftlichen Beobachtung -- können wir jetzt und in historischer Zeit das Wachstum dieser neuen, im Wesentlichen im Entstehen begriffenen geologischen Kraft klar feststellen und untersuchen. 

Die Menschheit ist eins, und obwohl sie in den unterdrückten Massen bekennt, manifestiert sich diese Einheit doch durch Lebensformen, die sie für einen Menschen unmerklich, spontan, [als Ergebnis] unbewussten Strebens nach ihm, tatsächlich vertiefen und stärken. Das Leben der Menschheit, bei aller Heterogenität, ist unteilbar geworden, vereint. Ein Ereignis, das sich in einer Rückstauecke eines beliebigen Punktes auf einem beliebigen Kontinent oder Ozean ereignet hat, wird reflektiert und hat -- große und kleine -- Folgen an einer Reihe anderer Orte, überall auf der Erdoberfläche. Telegraf, Telefon, Radio, Flugzeuge und Ballons haben den ganzen Globus bedeckt. Die Verbindungen werden immer einfacher und schneller. Von Jahr zu Jahr werden sie immer organisierter und wachsen schnell. 

Wir können deutlich sehen, dass dies der Beginn einer natürlichen Bewegung ist, eines natürlichen Phänomens, das nicht durch die Zufälligkeit der menschlichen Geschichte aufgehalten werden kann. Hier zeigt sich vielleicht zum ersten Mal so deutlich die Verbindung historischer Prozesse mit der paläontologischen Geschichte der Entdeckung des Homo sapiens. Dieser Prozess -- die vollständige Besiedlung der Biosphäre durch den Menschen -- ist bedingt durch den Verlauf der Geschichte des wissenschaftlichen Denkens, ist untrennbar verbunden mit der Geschwindigkeit der Kommunikation, mit dem Erfolg der Bewegungstechnik, mit der Möglichkeit der augenblicklichen Übertragung des Denkens, seiner gleichzeitigen Diskussion überall auf dem Planeten. 

Der Kampf, der mit dieser großen historischen Entwicklung weitergeht, bringt die ideologischen Gegner dazu, ihr tatsächlich zu gehorchen. Staatliche Entitäten, die ideologisch die Gleichheit und Einheit aller Menschen nicht anerkennen, versuchen ohne Zögern mit den Mitteln, ihre spontane Manifestation zu stoppen, aber es besteht kaum Zweifel daran, dass diese utopischen Träume nicht wahr werden können. Früher oder später ist die Entstehung der Noosphäre aus der Biosphäre ein Naturphänomen, das in seinem Kern tiefer und mächtiger ist als die Menschheitsgeschichte. Sie erfordert die Manifestation der Menschheit als Ganzes. Dies ist ihre unvermeidliche Prämisse. 

Dies ist eine neue Etappe in der Geschichte des Planeten, die keinen Vergleich ohne Änderungen seiner historischen Vergangenheit zulässt. Denn dieses Stadium ist wesentlich neu in der Geschichte der Erde, nicht nur in der Geschichte der Menschheit. 

Der Mensch hat zum ersten Mal verstanden, dass er ein Bürger des Planeten ist und in einem neuen Aspekt denken und handeln kann -- muss -, und zwar nicht nur im Aspekt des Individuums, der Familie oder Art, der Staaten oder ihrer Zusammenschlüsse, sondern auch im planetarischen Aspekt. Er kann, wie alle Lebewesen, im planetarischen Aspekt nur im Bereich des Lebens denken und handeln -- in der Biosphäre, in einer bestimmten irdischen Hülle, mit der er untrennbar und rechtmäßig verbunden ist und die er nicht verlassen kann. Seine Existenz ist seine Funktion. Er trägt es überall mit sich herum. Und er ändert sie unweigerlich, rechtmäßig, kontinuierlich. 

15. Gleichzeitig mit der vollständigen Bedeckung der Oberfläche der Biosphäre durch den Menschen -- ihrer vollständigen Besiedlung -, die eng mit dem Erfolg des wissenschaftlichen Denkens, d.h. ihrem zeitlichen Verlauf, verbunden ist, ist in der Geologie eine wissenschaftliche Synthese geschaffen worden, die das Wesen des Moments, in dem die Menschheit ihre Geschichte erlebt hat, auf neue Weise wissenschaftlich offenbart. 

Die geologische Rolle des Menschen hat zu einem neuen Verständnis der Geologen geführt. Das Bewusstsein für die geologische Bedeutung seines gesellschaftlichen Lebens in einer weniger klaren Form kam jedoch schon vor langer Zeit, viel früher, in der Geschichte des wissenschaftlichen Denkens zum Ausdruck. Aber zu Beginn unseres Jahrhunderts haben Ch. Shukhert [1858-1942] in New Haven28 und A.P. Pavlov (1854-1929) in Moskau29 unabhängig voneinander geologisch auf neue Weise den seit langem bekannten Wandel berücksichtigt, den die Entstehung der menschlichen Zivilisation für die umgebende Natur, das Antlitz der Erde, mit sich bringt. Sie fanden es möglich, eine solche Manifestation des Homo sapiens als Grundlage für die Zuweisung einer neuen geologischen Ära zu nehmen, zusammen mit den tektonischen und orographischen Daten, die normalerweise solche Unterteilungen bestimmen. 

Auf dieser Grundlage versuchten sie korrekterweise, das Pleistozän einzuteilen, indem sie sein Ende mit dem Beginn der Entdeckung des Menschen (die letzten hundert oder zweitausend Jahre -- etwa vor einigen Dekamyriaden) bestimmten und in einer besonderen geologischen Epoche -- psychozoisch, nach Schuchert, anthropogen -- nach A.P. Pawlow hervorheben. 

Tatsächlich vertieften und klärten C. Shukhert und AP Pavlov, brachten in den Rahmen der in der Geologie unserer Zeit etablierten Unterteilungen der Erdgeschichte eine Schlussfolgerung ein, die viel früher als sie gezogen wurde und der empirischen wissenschaftlichen Arbeit nicht widersprach. So wurde sie von einem der Schöpfer der modernen Geologie -- L. Agassiz (1807-1873), der aus der paläontologischen Lebensgeschichte stammte, klar erkannt. Er begründete bereits 1851 ein besonderes geologisches Zeitalter des Menschen. 

Aber Agassiz stützte sich nicht auf geologische Fakten, sondern weitgehend auf einen einheimischen religiösen Glauben, der in der vordarwinistischen Ära der Naturgeschichte so stark war; er ging von der Sonderstellung des Menschen im Universum aus30. 

Die Geologie der Mitte des 19. Jahrhunderts und die Geologie des Beginns des 20. Jahrhunderts sind in ihrer Macht und wissenschaftlichen Gültigkeit unvergleichbar, und die Ära von Agassis kann wissenschaftlich nicht mit der Ära von Schuchert-Pawlow verglichen werden. 

Noch früher, als die Geologie gerade erst komponiert wurde und ihre Grundbegriffe noch nicht existierten, brachte J. Buffon (1707-1788) im späten 18. Jahrhundert dieselbe geologische Ära des Menschen klar zum Ausdruck. Er ging von den Ideen der Philosophie der Aufklärung aus -- stellte die Bedeutung der Vernunft im Konzept der Welt dar. 

Der scharfe Unterschied dieser verbal identischen Vorstellungen wird aus der Tatsache deutlich, dass Agassis die geologische Dauer der Welt, die Existenz der Erde während der biblischen Zeit -- sechs- oder siebentausend Jahre, Buffon dachte über die Dauer von mehr als 127 tausend Jahren nach, Shukhert und Paulus -- mehr als eine Milliarde Jahre. 

16. In der Philosophie treffen wir vor langer Zeit auf enge Ideen, die auf einem anderen Weg gewonnen wurden -- nicht durch genaue wissenschaftliche Beobachtung und Erfahrung, die Ch. Shukhert, AP Pavlov, L. Agassis (und D. Dana, der über die Verallgemeinerungen von Agassis Bescheid wusste), sondern durch philosophische Suche und Intuition. 

Die philosophische Weltdarstellung im Allgemeinen und im Besonderen schafft das Umfeld, in dem wissenschaftliches Denken stattfindet und sich entwickelt. Bis zu einem gewissen Grad bestimmt sie ihn, indem sie ihre eigenen Leistungen verändert. 

Die Philosophen gingen von der Freiheit aus, es schien ihnen, in ihrem Ausdruck von Ideen, die Suche nach einem rebellischen menschlichen Gedanken, dem menschlichen Bewusstsein, nicht mit der Realität zu versöhnen. Der Mensch aber baute seine ideale Welt unweigerlich im brutalen Rahmen der ihn umgebenden Natur, der Umwelt seines Lebens, der Biosphäre, der tiefen Verbundenheit mit seiner eigenen, von seinem Willen unabhängigen Welt, mit der er nicht verstand und jetzt nicht versteht. 

In der Geschichte des philosophischen Denkens finden wir viele Jahrhunderte vor Christus Intuition und Konstruktion, die mit wissenschaftlich-empirischen Schlussfolgerungen verbunden werden können, wenn wir diese vorhandenen Gedanken -- Intuition -- in die realen wissenschaftlichen Fakten unserer Zeit übertragen. Ihre Wurzeln sind in der Vergangenheit verloren gegangen. Einige der philosophischen Bestrebungen Indiens vor vielen Jahrhunderten -- die Philosophien der Upanishaden -- könnten so interpretiert werden, wenn sie auf das Gebiet der Wissenschaft des zwanzigsten Jahrhunderts übertragen würden31. 

Teilweise zur gleichen Zeit, aber später, existierten ähnliche Vorstellungen in einem anderen, kleineren, vom indischen Kulturkreis weitgehend abgeschiedenen Kulturraum, im Kreis der hellenischen mediterranen Zivilisation. Wir können ihre Anfänge fast zweieinhalbtausend Jahre zurückverfolgen. Im politischen und sozialen Denken kam die Bedeutung von Wissenschaft und Wissenschaftlern für die Führung der Politik im hellenischen Denken deutlich zum Ausdruck und beeinflusste das Konzept des Staates, [das] von Platon [427-347] gegeben wurde. 

Es ist anscheinend nicht zu leugnen, aber der Zustand der Quellen in den erhaltenen Passagen erlaubt es nicht, es zu behaupten und genau zu behaupten, dass durch Aristoteles [384-322] diese Ideen in der Epoche des hellenistischen Alexander des Großen [356-323] lebendig waren, als für einige Jahrhunderte nach der Zerstörung des persischen Königreichs der enge Austausch von Ideen und Wissen der hellenistischen und indischen Zivilisationen geschaffen wurde. Gleichzeitig wurde eine Kommunikation mit ihnen und mit dem chaldäischen wissenschaftlichen Gedankengut, das einige Jahrtausende lang von hellenistischen und indischen Ideen geprägt war, aufgebaut. Die Geschichte des wissenschaftlichen Arbeitens und Denkens in dieser bedeutenden Epoche ist erst am Anfang der Aufklärung. 

Wir kennen den Einfluss hellenischer politischer und sozialer Ideen besser. Wir können ihren historischen Einfluss auf den historischen Prozess der neuen Wissenschaft und Zivilisation des europäischen Westens, der die theokratische ideologische Struktur des Mittelalters ablöste, genau verfolgen. In Wirklichkeit und deutlich sehen wir ihr Wachstum erst in den XVI-XVII Jahrhunderten in den Ideen und Konstruktionen von F. Bacon (1561-1626), der die Idee der menschlichen Macht über die Natur lebhaft als ein Ziel der neuen Wissenschaft vertrat. 

Jahrhundert, im Jahr 1780, stellte J. Buffon die Manifestation der menschlichen Kontrolle über die Natur in der Geschichte des Planeten nicht als eine Idee, sondern als ein mögliches Naturphänomen dar, das es zu beobachten gilt. Er ging eher von den hypothetischen Konstruktionen der Vergangenheit des Planeten aus, die mit philosophischer Intuition und Theorie verbunden sind, als von genau beobachteten Fakten -- aber er suchte sie. Seine Ideen umfassten philosophisches und politisches Denken und hatten zweifellos einen Einfluss auf das wissenschaftliche Denken. Sie wurden im späten 18. und frühen 19. Jahrhundert häufig von Geologen in ihrer aktuellen wissenschaftlichen Arbeit verwendet. 

17. Die wissenschaftlichen Konstruktionen von Schuchert und Pawlow und all die wissenschaftlichen Arbeiten, denen sie -- weitgehend unbewusst -- vorausgegangen sind, unterscheiden sich zwar wesentlich von diesen philosophischen Konstruktionen, doch zweifellos (es ist möglich, sie historisch zu begründen), blieben sie nicht ohne Einfluss auf den Verlauf des geologischen Denkens, konnten ihm aber keine solide Grundlage geben. 

Aus den Verallgemeinerungen von Schuchert und Pawlow wird deutlich, dass sich der Haupteinfluss des menschlichen Denkens als geologischer Faktor in seiner wissenschaftlichen Manifestation offenbart: Es baut und lenkt hauptsächlich die technische Arbeit der Menschheit, indem es die Biosphäre umgestaltet. 

Diese beiden Geologen konnten ihre Verallgemeinerung vor allem deshalb vornehmen, weil der Mensch zu ihrer Zeit in der Lage war, den gesamten Planeten zu bewohnen. Außer ihm bedeckte kein Organismus, außer mikroskopisch kleinen Arten von Organismen und vielleicht einigen krautigen Pflanzen, solche Gebiete des Planeten bei der Besiedlung. Aber der Mensch hat es auf eine andere Weise getan. Er dachte wissenschaftlich und veränderte die Biosphäre kaum, passte sie sich selbst an und schuf Bedingungen für die Manifestation der ihm eigenen biogeochemischen Reproduktionsenergie. Eine solche Besiedlung des gesamten Planeten wurde zu Beginn des XX. Jahrhunderts deutlich, und es kann davon ausgegangen werden, dass etwa das erste Viertel davon eine Tatsache geworden ist, sie verstärkt sich jedes Jahr mehr und mehr vor unseren Augen. Sie wurde nur möglich durch eine dramatische Veränderung der Lebensbedingungen, die mit der neuen Ideologie, mit einer dramatischen Veränderung der Aufgaben des Staatslebens, mit dem Wachstum der wissenschaftlichen Technologie verbunden war, die zur gleichen Zeit vollzogen wurde. 

Wie I. Ortega-i-Gaset richtig bemerkte32, war das XIX. Jahrhundert in Europa und in der ganzen Welt ab seiner zweiten Hälfte eine historische Periode, in der die Bedeutung der vitalen Interessen der Volksmassen wirklich und ideologisch, in ihrem Bewusstsein und im Bewusstsein ihres Staatsvolkes, zum ersten Mal in der Weltgeschichte an erster Stelle stand. Zum ersten Mal wurde sie im Alltag scharf manifestiert. Zum ersten Mal basiert die neue Ideologie auf dem Bewusstsein der Volksmassen, die als soziale Kraft in der historischen Arena agieren. Sie beginnt, die gesamte Menschheit in einem rasch wachsenden Tempo zu erfassen -- jede Sprache ohne Ausnahme. 

Sie wird sich erst mit der Zeit auf ihre tatsächliche Bedeutung auswirken. 

Der sozial-politisch-ideologische Umbruch wurde im XX. Jahrhundert vor allem durch die wissenschaftliche Arbeit, die wissenschaftliche Definition und Klärung der sozialen Aufgaben der Menschheit und ihrer Organisationsformen hell erleuchtet. 

18. In der Jahrtausende währenden historischen Tragödie, für die Massen von Menschen voller Blut, Leid, Verbrechen, Armut, schwieriger Lebensbedingungen, die wir Weltgeschichte nennen, stellte sich immer wieder die Frage nach einer besseren Lebensweise und den Möglichkeiten, sie zu erreichen. Der Mensch hat die Bedingungen seines Lebens nicht akzeptiert. 

Das Ergebnis der Suche wurde anders gelöst, und in der Geschichte der Menschheit sehen wir zahlreiche (und wie viele von ihnen verschwanden spurlos!) Suchen -- philosophische, religiöse, künstlerische und wissenschaftliche. Jahrtausende an allen Ecken und Enden, wo es eine menschliche Gesellschaft gibt, wurden und werden sie geschaffen. 

Die Weltgeschichte der Menschheit wurde für einen großen Teil der Menschheit erlebt und dargestellt, und an Orten und zu Zeiten war sie für die Mehrheit der Menschen voller Leid, Böses, Mord, Hunger und Armut ein unlösbares Rätsel aus der menschlichen Perspektive der Vernünftigkeit und Güte. Im Allgemeinen haben unzählige philosophische und religiöse Versuche im Laufe der Jahrtausende keine einzige Erklärung geliefert. 

Alle so erhaltenen Lösungen übertrugen und verlagerten die Frage schließlich auf eine andere Ebene -- vom Bereich der grausamen Realität in den Bereich der idealen Konzepte. Es gibt unzählige religiöse und philosophische Lösungen, die in verschiedenen Formen gefunden wurden und die in der Tat mit dem Begriff der Unsterblichkeit der Persönlichkeit verbunden sind, in der einen oder anderen Form im direkten Sinne des Wortes oder in der Zukunft ihrer Auferstehung unter neuen Bedingungen, wo es kein Übel, Leiden und Katastrophen geben wird oder wo sie gerecht verteilt werden. Die tiefgreifendste ist der Begriff der Metampsychose, die entscheidende Frage nicht aus der Sicht des Menschen, sondern aus der Sicht der gesamten lebenden Substanz. Es ist noch heute, vor einigen Jahrtausenden, für viele hundert Millionen Menschen lebendig und anschaulich. Und nichts widerspricht vielleicht modernen wissenschaftlichen Ideen. Der Kurs des wissenschaftlichen Denkens ist in diesem Begriff nirgends zu finden. 

All diese Begriffe -- manchmal weit entfernt von genauen wissenschaftlichen Erkenntnissen -- sind seit Tausenden von Jahren ein mächtiger sozialer Faktor, der die Entwicklung der Biosphäre zur Noosphäre dramatisch beeinflusst, aber weit davon entfernt, entscheidend zu sein oder andere Faktoren bei ihrer Entstehung zu beeinträchtigen. In diesem Aspekt spielten sie über Zehntausende von Jahren manchmal eine große Rolle, gingen manchmal unter anderen verloren, traten in den Hintergrund, konnten vernachlässigt werden. 

19 Denn derselbe historische Prozess der Weltgeschichte spiegelt sich in der Natur um den Menschen herum auf andere Weise wider. Es ist möglich und notwendig, sie rein wissenschaftlich anzugehen und alle Ideen, die sich nicht aus wissenschaftlichen Fakten ergeben, beiseite zu lassen. 

Archäologen, Geologen und Biologen nähern sich nun einem solchen Studium der Weltgeschichte der Menschheit, wobei sie alle tausendjährigen Ideen von Philosophie und Religion beiseite lassen und ein neues wissenschaftliches Verständnis des historischen Prozesses des menschlichen Lebens schaffen. Geologen, die sich tief in die Geschichte unseres Planeten, in der Zeit nach dem Polozän, in der Eiszeit, vertieft haben, haben eine Vielzahl wissenschaftlicher Fakten gesammelt, die das Spiegelbild des Lebens der menschlichen Gesellschaften -- immerhin der zivilisierten Menschheit -- über die geologischen Prozesse unseres Planeten, im Wesentlichen der Biosphäre, offenbaren. Ohne sie unter dem Gesichtspunkt von Gut und Böse zu bewerten, ohne die ethische oder philosophische Seite, die wissenschaftliche Arbeit, zu berühren, stellt das wissenschaftliche Denken eine neue Tatsache in der Geschichte des Planeten von primärer geologischer Bedeutung fest. Diese Tatsache besteht in der Enthüllung einer neuen psychozoischen oder anthropogenen geologischen Ära, die durch den historischen Prozess geschaffen wurde. Im Wesentlichen wird sie paläontologisch durch das Aussehen des Menschen bestimmt. 

In dieser wissenschaftlichen Verallgemeinerung spielen alle unzähligen -- geologischen, philosophischen und religiösen -- Vorstellungen über die Bedeutung des Menschen und die Bedeutung der menschlichen Geschichte keine [wesentliche] Rolle. Sie können sicher beiseite gelassen werden. Die Wissenschaft darf nicht zu ihnen zählen. 

20. Wenn wir uns der Analyse dieser wissenschaftlichen Verallgemeinerung nähern, stellen wir fest, dass ihre Dauer in Millionen von Jahren geschätzt werden kann und dass der historische Prozess der menschlichen Gesellschaften mehrere Dekamyriaden, also Hunderttausende von Jahren, umfasst. 

Zunächst ist es notwendig, einige Voraussetzungen hervorzuheben, die diese Verallgemeinerung definiert. 

Die erste ist die Einheit und prinzipielle Gleichheit aller Menschen, aller Rassen. Biologisch drückt es sich in der Offenbarung im geologischen Prozess aller Menschen als Ganzes in Bezug auf die andere lebende Bevölkerung des Planeten aus. 

Und dies trotz der Tatsache, dass es möglich und sogar wahrscheinlich ist, dass die menschlichen Rassen von verschiedenen Arten der Gattung Homo unterschiedliche Ursprünge haben. Kaum dieser Unterschied geht tiefer in verschiedene Tiere der Vorfahren der Gattung Homo ein. Aber es lässt sich noch nicht leugnen. Eine solche Einheit in Bezug auf alle anderen Tiere wird in der Weltgeschichte im Allgemeinen aufrechterhalten, auch wenn sie in Einzelfällen zu Zeiten und an Orten fehlte oder fast fehlte. Wir begegnen seinen Manifestationen schon jetzt, aber daran ändert sich der allgemeine spontane Prozess nicht. 

In dieser Hinsicht wurde die geologische Bedeutung der Menschheit erstmals in diesem Phänomen manifestiert. Offenbar begann der Mensch bereits vor hunderttausend Jahren, als er das Feuer besaß und damit begann, die ersten Gewehre herzustellen, seinen Vorteil gegenüber den höheren Tieren, deren Kampf einen großen Platz in seiner Geschichte einnahm und schließlich, theoretisch gesehen, vor mehreren Jahrhunderten mit der Entdeckung der Feuerwaffen endete. Schon im XX. Jahrhundert sollte der Mensch besondere Anstrengungen unternehmen, um die Ausrottung aller Tiere zu verhindern -- große Säugetiere und Reptilien, die er aus dem einen oder anderen Grund halten will. Aber viele Jahrtausende früher, kurz vor seinem Erscheinen, war er eine neue Kraft auf unserem Planeten, die zusammen mit anderen früheren einen wichtigen Platz einnahm und zur Ausrottung großer Tierarten führte. Es ist sehr gut möglich, dass es am Anfang nicht [auf] viele andere Raubtiere mit Herdencharakter zu dieser Zeit war. 

21. Viel wichtiger aus geologischer Sicht war eine andere Verschiebung, die sich vor Zehntausenden von Jahren vollzog: die Domestizierung von Herdentieren und die Entwicklung kultureller Pflanzenrassen. Der Mensch begann, die lebendige Welt um sich herum auf diese Weise zu verändern und sich eine neue, nie zuvor auf dem Planeten existierende, lebendige Natur zu schaffen. Die große Bedeutung drückte sich auch auf andere Weise aus -- in der Tatsache, dass er den Hunger auf eine neue Art und Weise los wurde, nur bei einem schwach bekannten Tier -- bewusste, schöpferische Unterstützung durch den Hunger und damit die Möglichkeit der unbegrenzten Fortpflanzung fand. 

Zu dieser Zeit, wahrscheinlich jenseits der Grenzen von vor zehn- oder zweitausend Jahren, wurde erstmals die Möglichkeit der Bildung großer Siedlungen (Städte und Dörfer) und damit die Möglichkeit der Bildung staatlicher Strukturen geschaffen, die sich im Wesen stark von jenen Sonderformen unterscheiden, die durch die Blutkommunikation verursacht werden. Die Idee der Einheit der Menschheit ist real, obwohl sie offensichtlich unbewusst noch mehr Möglichkeiten für ihre Entwicklung erhielt. 

Mit der Eröffnung des Feuers haben die Menschen die Eiszeit erleben können -- jene enormen Veränderungen und Schwankungen der Klima- und Biosphärenbedingungen, die uns nun wissenschaftlich im Wechsel der so genannten Zwischeneiszeiten -- mindestens drei -- auf der Nordhalbkugel eröffnet werden. Es hat sie überlebt, obwohl eine Reihe anderer großer Säugetiere von der Erdoberfläche verschwunden sind. Es ist möglich, dass er zu ihrer Ausrottung beigetragen hat. 

Die Eiszeit ist nicht vorbei, und sie dauert immer noch an. Wir leben in einer Zeit zwischen Gletschern -- die Erwärmung dauert noch an -- aber die Menschen haben sich so gut angepasst, dass sie die Gletscherperiode nicht bemerken. Der skandinavische Gletscher schmolz vor mehreren tausend Jahren an der Stelle von St. Petersburg und Moskau, als der Mensch bereits Haustiere hatte und Landwirtschaft betrieb. 33 

Hunderttausende von Generationen sind in der Geschichte der Menschheit während der Eiszeit vergangen. 

Aber man kann heute kaum noch daran zweifeln, dass der Mensch (wahrscheinlich nicht die Familie Homo) schon viel früher existierte -- zumindest am Ende des Pliozäns, vor einigen Millionen Jahren. Der Piltdown-Mann in Südengland am Ende des Pliozäns, morphologisch anders als der moderne Mensch, hatte bereits Steinwerkzeuge und offensichtlich keine überlebenden Werkzeuge aus Holz und vielleicht Knochen. Sein Hirnapparat war so perfekt wie der des modernen Menschen. 34 Der Sinanthropist aus Nordchina, der anscheinend im frühen Post-Pliozän lebte, wo der Gletscher anscheinend nicht reichte, kannte den Gebrauch von Feuer und besaß Instrumente. 35 

Wahrscheinlich hatte A.P. Pavlov Recht, der vermutete, dass die Eiszeit, die erste Vereisung der nördlichen Hemisphäre, am Ende des Pliozäns begann, und dass zu dieser Zeit ein neuer Organismus mit einem außergewöhnlichen zentralen Nervensystem in der Biosphäre, das schließlich zur Entstehung des Geistes führte, unter Bedingungen, die sich den schweren Eiszeiten näherten, entdeckt wurde und sich nun im Übergang der Biosphäre zur Noosphäre manifestiert. 

Anscheinend haben alle morphologisch unterschiedlichen Menschentypen, verschiedene Geburten und Arten bereits miteinander kommuniziert, waren anfangs anders als der Großteil der lebenden Materie, hatten eine dramatisch andere Art von Kreativität als das umgebende Leben und konnten miteinander blutsverwandt sein. Auf diese Weise wurde die Einheit der Menschheit spontan geschaffen. Anscheinend hatte Osborne Recht36 , dass der Mensch an der Grenze zwischen Pliozän und Post-Pliozän, der noch keine dauerhaften Siedlungen hatte, über große Mobilität verfügte, von Ort zu Ort zog, sich bewusst war und seine scharfe Getrenntheit zeigte -- er suchte Unabhängigkeit von der Umwelt [Umwelt]. 

22. In Wirklichkeit ist dies die Einheit des Menschen, sein Unterschied zu allen lebenden Organismen, eine neue Form der Macht des lebenden Organismus über die Biosphäre, seine größere Unabhängigkeit von ihren Bedingungen als alle anderen Organismen, ist der Hauptfaktor, der schließlich im geologischen Evolutionsprozess der Entstehung der Noosphäre deutlich wurde. Über viele Generationen hinweg manifestierte sich die Einheit der menschlichen Gesellschaften, ihre Kommunikation und ihre Macht -- der Wunsch, Macht über die umgebende Natur auszuüben -- spontan, bevor sie enthüllt und ideologisch bewusst wurden. 

Natürlich war es kein bewusstes Phänomen; es entstand im Kampf der Zusammenstöße; es kam zu gegenseitigen Auslöschungen von Menschen, manchmal zu Kannibalismus und zur Jagd aufeinander, aber in der Regel durchdringen und erschaffen diese drei tatsächlichen Ausdrucksformen zukünftiger Vorstellungen von der menschlichen Einheit, ihrem scharfen Unterschied zu allen Lebewesen und dem Wunsch, die umgebende Natur zu beherrschen, die gesamte Geschichte der Menschheit, zumindest in den letzten Zehntausenden von Jahren. Sie haben ein neues modernes Bestreben vorbereitet, sie ideologisch als Grundlage des menschlichen Lebens zu verstehen. 

Ihre tatsächliche Existenz lässt sich aus ideologischer Sicht nur innerhalb von maximal einem Jahrzehnt wissenschaftlich exakt nachvollziehen. Aber auch so gehen wir bei schriftlichen Denkmälern nicht tiefer als viertausend Jahre, denn die Schriftzeichen gehen nicht viel tiefer, und das Alphabet der Buchstabenzeichen reicht kaum dreitausend Jahre vor unserer Zeit zurück. In den ältesten Denkmälern können wir kaum tausend Jahre vor der Öffnung der idiographischen Schriften echte Anklänge an ideologische Konstruktionen erwarten. Deshalb gehen wir in der überlieferten Legende kaum viel tiefer als sechstausend Jahre vor unserer Zeit, wenn man die ungewöhnliche, heute mündliche Möglichkeit der Übertragung von ideologischen Konstruktionen, die von einer eigentümlichen Zivilisation der damaligen Zeit entwickelt wurden, durch Generationen berücksichtigt. Letzte archäologische Entdeckungen eröffnen vor uns die unerwartete Tatsache, dass das für unsere Lebensbedingungen des kulturellen Stadtlebens, des friedlichen Handelsaustauschs und der Technik des Lebens übliche zivilisierte Stadtleben, früher seine Errungenschaften nicht zuließ, später vergessen und durch Jahrtausende manchmal wieder gefunden wurde; sie lassen denken, dass das schwierige zivilisierte Stadtleben vor langer Zeit -- vielleicht vor Jahrtausenden -- existierte. Im Laufe der Jahrtausende wurden all diese Errungenschaften auf komplexe Weise auf alle Kontinente ausgeweitet, wobei offenbar irgendwann auch die Neue Welt nicht ausgeschlossen wurde. Aus menschlicher Sicht war die Neue Welt nicht neu, und die Kultur, ja sogar die wissenschaftliche Kultur ihrer Staaten am Ende des XV. -- Anfang des XVI. Jahrhunderts -- die Zeit ihrer Öffnung für die westeuropäische Zivilisation war nicht niedriger, sondern in mancher Hinsicht sogar höher als das wissenschaftliche Wissen der Westeuropäer. Sie war nur deshalb zusammengebrochen, weil militärische Ausrüstung und Schusswaffen in Amerika unbekannt waren und einige Jahrzehnte vor der Entdeckung Amerikas im Alltag der Westeuropäer alltäglich geworden waren. 

Es zeigt sich das Bild einer mehrtausendjährigen Geschichte der materiellen Interaktion von Zivilisationen, getrennten historischen Zentren in ganz Eurasien, einem Teil Afrikas, vom Atlantik bis zum Pazifik und Indischen Ozean, die sich zuweilen -- mit mehrjährigen Unterbrechungen -- über die Ozeane ausbreiten. Es ist äußerst charakteristisch, dass sich die Kulturzentren nur an wenigen Orten befanden. Die ältesten sind: Die chaldäische Interfluve wurde von Brested, dem Niltal, Ägypten und dem vorarischen Nordindien gegründet. Sie alle standen viele tausend Jahre lang in Kontakt. Etwas später, mehr als dreitausend Jahre später, wurde das Nordchina-Zentrum eröffnet. Aber hier hat die wissenschaftliche Forschung erst in den letzten drei oder vier Jahren begonnen und wird durch die wilde japanische Invasion gebremst. Hier kann es Überraschungen geben. Offenbar gab es ein temporäres Zentrum an der Pazifikküste -- in Korea oder China -- und an der indischen Küste -- in Annam, dessen Rolle noch recht unklar ist, und es könnte große Entdeckungen geben. 

23. Vor etwa zweieinhalbtausend Jahren gab es „gleichzeitig“ (in der Reihenfolge der Jahrhunderte) eine tiefgreifende Bewegung des Denkens auf dem Gebiet des religiösen, künstlerischen und philosophischen Denkens in verschiedenen kulturellen Zentren: Im Iran, in China, im arischen Indien, im hellenischen Mittelmeerraum (dem heutigen Italien) erschienen große Schöpfer religiöser Systeme -- Zarathustra, Pythagoras, Konfuzius, Buddha, Laotse, Mahavira, die bis heute Millionen von lebenden Menschen erreicht haben. 

Zum ersten Mal ging die Idee der Einheit der Menschheit als Ganzes, der Menschen als Brüder, über die Grenzen der einzelnen Personen hinaus, die sich ihr in ihren Intuitionen oder Inspirationen näherten, und wurde zum Motor des Lebens und des Alltags der Volksmassen oder zur Aufgabe staatlicher Einheiten. Sie hat sich seitdem nicht vom historischen Feld der Menschheit entfernt, ist aber noch weit von ihrer Verwirklichung entfernt. Langsam, mit hunderten von Jahren Stillstand, werden Bedingungen geschaffen, die es ermöglichen, dass sie in der Realität durchgeführt werden kann. 

Es ist wichtig und bezeichnend, dass diese Vorstellungen in den Rahmen jener alltäglichen realen Phänomene eingetreten sind, die im unbewussten Leben, außerhalb des menschlichen Willens, entstanden sind. Sie manifestierten den Einfluss der Persönlichkeit, den Einfluss, durch den sie durch die Organisierung von Massen die umgebende Biosphäre beeinflussen und sich spontan in ihr manifestieren kann. 

Früher manifestierte sie sich in der poetisch inspirierten Kreativität, aus der Religion, Philosophie und Wissenschaft, die allesamt soziale Ausdrucksformen sind, hervorgingen. Religiöse Leitideen scheinen philosophischen Intuitionen und Verallgemeinerungen seit Jahrhunderten, wenn nicht Jahrtausenden vorausgegangen zu sein. 

Die Biosphäre des 20. Jahrhunderts wird zu einer Noosphäre, die vor allem durch das Wachstum der Wissenschaft, des wissenschaftlichen Verständnisses und der darauf aufbauenden sozialen Arbeit der Menschheit geschaffen wird. Ich werde auf die Analyse der Noosphäre weiter unten in einer späteren Erklärung zurückkommen. Es muss nun betont werden, dass seine Entstehung untrennbar mit dem Wachstum des wissenschaftlichen Denkens verbunden ist, das die erste notwendige Voraussetzung für diese Entstehung ist. Nur unter dieser Bedingung kann die Noosphäre geschaffen werden. 

24. Und gerade in unserer Zeit, seit Beginn des 20. Jahrhunderts, gibt es ein außergewöhnliches Phänomen im wissenschaftlichen Denken. Sein Tempo ist ziemlich ungewöhnlich geworden, beispiellos seit vielen Jahrhunderten. Vor elf Jahren habe ich es mit einer Explosion gleichgesetzt -- einer Explosion der wissenschaftlichen Kreativität. 37 Und jetzt kann ich es nur noch schärfer und eindeutiger behaupten. 

Wir erleben im XX. Jahrhundert im Zuge der wissenschaftlichen Erkenntnis, im Zuge der wissenschaftlichen Kreativität in der Geschichte der Menschheit eine Zeit von gleicher Bedeutung, die wir nur in ihrer fernen Vergangenheit finden können. 

Leider erlaubt es uns der Stand der Geschichte des wissenschaftlichen Wissens heute nicht, die wichtigsten logischen Schlussfolgerungen aus dieser empirischen Position genau und definitiv zu ziehen. Wir können es nur als Tatsache behaupten und es in geologischer Hinsicht zum Ausdruck bringen. 

Die Geschichte der wissenschaftlichen Erkenntnisse ist die Geschichte der Entstehung eines neuen wichtigen geologischen Faktors in der Biosphäre -- ihrer Organisation, die in den letzten Jahrtausenden spontan entstanden ist. Es ist kein Zufall, sondern natürlich, da der paläontologische Prozess im Laufe der Zeit natürlich ist. 

Die Geschichte des wissenschaftlichen Wissens ist noch nicht geschrieben worden, und wir beginnen gerade erst -- mit großen Schwierigkeiten und großen Lücken -- darin, vergessene und bewusst unerforschte Tatsachen zu identifizieren, die die Menschheit nicht gelernt hat -- wir beginnen, nach großen empirischen Verallgemeinerungen zu suchen, die sie charakterisieren. 

Wir können diese große, enorme wissenschaftliche und gesellschaftliche Bedeutung wissenschaftlich noch nicht verstehen. Wissenschaftlich zu verstehen bedeutet, das Phänomen in die wissenschaftliche Realität des Weltraums einzuordnen. Wir müssen nun gleichzeitig versuchen, sie wissenschaftlich zu verstehen und ihre Studie zu nutzen, um wichtige Meilensteine in der Geschichte des wissenschaftlichen Wissens -- einer der wichtigsten wissenschaftlichen Disziplinen der Menschheit -- zu setzen. 

Wir erleben einen fundamentalen Bruch im wissenschaftlichen Weltbild, der sich im Laufe des Lebens der lebenden Generationen vollzieht, wir erleben die Schaffung riesiger neuer Wissensgebiete, die den wissenschaftlich abgedeckten Raum vom Ende des letzten Jahrhunderts bis zur Unkenntlichkeit erweitern, und wir erleben in seinem Raum und in seiner Zeit eine Veränderung der wissenschaftlichen Methoden, die wir in den vorhandenen Chroniken und Aufzeichnungen der Weltwissenschaft vergeblich suchen würden. Mit zunehmender Geschwindigkeit entstehen neue Methoden der wissenschaftlichen Arbeit und neue Wissensgebiete, neue Wissenschaften, die vor uns Millionen von wissenschaftlichen Fakten und Millionen von wissenschaftlichen Phänomenen eröffnen, deren Existenz wir gestern noch nicht ahnten. Schwieriger und unvollständiger als je zuvor kann ein einzelner Wissenschaftler dem Lauf der wissenschaftlichen Erkenntnis folgen. 

Die Wissenschaft wird vor unseren Augen wieder aufgebaut. 

Aber mehr noch, es scheint mir, mit erstaunlicher Klarheit, dass der Einfluss der Wissenschaft auf unser Leben, auf die Lebenden und Toten -- die kosmische, umgebende Natur -- zunehmend zunimmt. Die Wissenschaft und das wissenschaftliche Denken, das sie schafft, offenbart in diesem Wachstum der Wissenschaft im XX. Jahrhundert, in diesem sozialen Phänomen der Menschheitsgeschichte, ihren tiefen Sinn, ihren anderen, uns fremden, planetarischen Charakter. Die Wissenschaft offenbart sich uns darin auf eine neue Art und Weise. 

Wir können dieses Phänomen, das wir erleben -- wissenschaftlich untersuchen -- aus zwei verschiedenen Perspektiven untersuchen. Einerseits als eines der Hauptphänomene in der Geschichte des wissenschaftlichen Denkens, andererseits als eine Manifestation der Struktur der Biosphäre, die uns neue große Merkmale ihrer Organisation offenbart. Die enge und untrennbare Verbindung zwischen diesen Phänomenen war der Menschheit noch nie so klar wie heute. 

Wir leben in einer Zeit, in der uns diese Seite des wissenschaftlichen Denkens mit außerordentlicher Klarheit offenbart wird -- der Verlauf der Geschichte des wissenschaftlichen Denkens erscheint uns als ein natürlicher Prozess in der Geschichte der Biosphäre. 

Historischer Prozess -- Manifestation der Weltgeschichte der Menschheit wird vor uns enthüllt -- in einem -- aber vor allem in seiner Konsequenz als ein natürliches, riesiges geologisches Phänomen. 

Dies ist in der Geschichte des wissenschaftlichen Denkens nicht berücksichtigt worden, da es untrennbar mit seinem Hauptmerkmal verbunden ist. 

25. Bisher wird die Geschichte der Menschheit und die Geschichte ihrer spirituellen Manifestationen als ein autarkes Phänomen studiert, das sich frei und unregelmäßig auf der Erdoberfläche, in ihrer Umgebung, als etwas ihr Fremdes manifestiert. Die gesellschaftlichen Kräfte, die sich in ihnen manifestieren, gelten als weitgehend frei von dem Umfeld, in dem sich die Geschichte der Menschheit abspielt. 

Obwohl es viele verschiedene Versuche gibt, die geistigen Manifestationen des Menschen und die Geschichte der Menschheit im allgemeinen mit der Umwelt, in der sie stattfinden, in Verbindung zu bringen, wird immer übersehen, daß erstens diese Umwelt -- die Biosphäre -- eine absolut bestimmte Struktur hat, die alles, was in ihr geschieht, ausnahmslos definiert, die durch die in ihrem Inneren ablaufenden Prozesse nicht radikal gestört werden kann, sondern wie alle Phänomene in der Natur ihre eigenen natürlichen Veränderungen in der Raumzeit hat. 

Die Explosion der wissenschaftlichen Kreativität ist bis zu einem gewissen Grad auch Teil des Übergangs von der Biosphäre in die Noosphäre. Aber darüber hinaus ist der Mensch selbst in seiner individuellen und sozialen Manifestation ganz natürlich, materiell und energetisch mit der Biosphäre verbunden; diese Verbindung wird niemals unterbrochen, solange es den Menschen gibt, und unterscheidet sich nicht wesentlich von anderen Biosphärenphänomenen. 

26. Lassen Sie uns diese wissenschaftlichen empirischen Zusammenfassungen zusammenfassen. 

    1. Der Mensch, wie er in der Natur beobachtet wird, ist wie alle lebenden Organismen, wie jede lebende Substanz, eine bestimmte Funktion der Biosphäre, in ihrer bestimmten Raumzeit. 

    2. Der Mensch in all seinen Erscheinungsformen stellt einen bestimmten natürlichen Teil der Struktur der Biosphäre dar. 

    3. Die „Explosion“ des wissenschaftlichen Denkens im 20. Jahrhundert wurde für die gesamte vergangene Biosphäre vorbereitet und ist tief in ihrer Struktur verwurzelt -- sie kann nicht anhalten und rückwärts gehen. Sie kann sich nur in ihrem eigenen Tempo verlangsamen. Die Noosphäre -- eine Biosphäre, die durch wissenschaftliches Denken neu gestaltet wurde, vorbereitet durch Hunderte von Millionen, vielleicht Milliarden von Jahren, den Prozess, der den Homo sapiens faber hervorgebracht hat -- ist kein momentanes und vorübergehendes geologisches Phänomen. Die Prozesse, die seit Milliarden von Jahren vorbereitet wurden, können nicht vergänglich sein, können nicht aufhören. Daraus folgt, daß sich die Biosphäre unweigerlich auf die eine oder andere Weise -- früher oder später -- in die Noosphäre hinein bewegen wird, d.h. daß es in der Geschichte der Völker, die sie bewohnen, dafür notwendige Ereignisse geben wird und nicht widersprüchliche. 

Die Zivilisation der „kulturellen Humanität“ -- als Organisationsform einer neuen, in der Biosphäre geschaffenen geologischen Kraft -- kann nicht unterbrochen oder zerstört werden, da es sich um ein großes Naturphänomen handelt, das der historisch bzw. geologisch etablierten Organisation der Biosphäre entspricht. Indem sie eine Noosphäre bildet, ist sie mit der Hülle dieser Erde mit all ihren Wurzeln verbunden, was in der Geschichte der Menschheit in vergleichbarer Weise noch nie vorgekommen ist. 

27 Es ist, als ob alle historischen Erfahrungen der Menschheit in der Vergangenheit und die Ereignisse des Augenblicks, die wir gerade durchleben, ihr widersprechen. 

Bevor ich weitermache, kann ich nicht aufhören, nicht einmal kurz. Es scheint mir, dass die im Entstehen begriffene Schaffung der Noosphäre durch menschliches Denken und die mühsame Veränderung des gesamten Umfelds ihrer Geschichte es nicht erlaubt, die Vergangenheit einfach mit der Gegenwart zu vergleichen, wie es früher erlaubt war. 

Jeder weiß um zahlreiche, nicht nur lange Unterbrechungen im Wachstum des wissenschaftlichen Denkens, sondern auch um den Verlust und die Zerstörung von wissenschaftlichen Errungenschaften, die viele Jahrhunderte lang früher gewonnen wurden. Wir sehen zeitweise einen stark ausgeprägten Rückschritt, der große Gebiete eroberte und ganze Zivilisationen physisch zerstörte, die dafür keine unvermeidlichen Gründe in sich trugen. Die Prozesse im Zusammenhang mit der Zerstörung der römisch-griechischen Zivilisation verzögerten die wissenschaftliche Arbeit der Menschheit um viele Jahrhunderte, und viele frühere Errungenschaften gingen für lange Zeit verloren, ein Teil davon für immer. Dasselbe sehen wir für die alten Zivilisationen Indiens und des Fernen Ostens. 

Es scheint verständlich und unvermeidlich, dass die Ängste und Befürchtungen desselben gewaltsamen Untergangs, einer der größten Manifestationen der Barbarei der Menschheit in unserer Zeit, nach dem Weltkrieg 1914-1918, weite Kreise denkender Menschen von hier aus hinweggefegt haben. Die staatlichen Kräfte haben sich nach ihrem Untergang, wie wir jetzt deutlich sehen, nicht der Situation gewachsen gezeigt, und wir erleben die Folgen der instabilen Lage der letzten 20 Jahre, die mit einem tiefen moralischen Wandel verbunden war -- die Folge des weltweiten Massakers, des sinnlosen Todes von mehr als zehn Millionen Menschen in vier Jahren und unzähliger Verluste an Volksarbeit. Zwanzig Jahre nach dem Ende des Krieges stehen wir nun vor der Gefahr eines neuen, noch barbarischeren und noch sinnloseren Krieges. Nun, nicht nur faktisch, sondern auch ideologisch ist die Vernichtung nicht nur von bewaffneten Kämpfern, sondern auch von Zivilisten, einschließlich älterer Menschen, alter Frauen und Kinder, der Weg des Krieges. Was moralisch nicht als Ideal anerkannt war, ist nun grausame Realität geworden. 

28. Als Folge des Krieges von 1914-1918, der zum Zusammenbruch der mächtigsten Staaten mit jahrhundertealter Tradition führte, der Staaten, die in ihren jahrhundertealten Idealen am wenigsten demokratisch und am wenigsten frei waren -- die Pfeiler der alten Traditionen in Europa, kam es zu einer grundlegenden Neubewertung der Werte. Diese Staaten basierten auf der Idee der „Gleichheit“ aller Menschen, die in einem eigentümlichen Rahmen christlicher Religionen zum Ausdruck kommt. Sie war die Grundlage der christlichen Moral. Die Realität entsprach zwar nie diesem Grundprinzip des Christentums (noch weniger als der Islam), aber es wurde überall in den christlichen Ländern lautstark proklamiert und war -- in der Vorstellung -- die Grundlage der Staatsmoral. Die Realität sah ganz anders aus, und jahrhundertelang haben die christlichen Staaten der weißen Rasse praktisch die gesamte Kolonialpolitik betrieben, die Gleichheit in Worten anerkannt, die Völker und Staaten der weißen Rasse rücksichtslos unterdrückt und ausgerottet und ausgebeutet. Der Krieg von 1914-1918 erschütterte die ganze Welt und enthüllte einen scharfen Widerspruch zwischen Worten und Taten, hob die Macht und Bedeutung der nicht-weißen Rasse hervor. 

Dies hatte keinen Einfluss auf die moralische Bedeutung des Islam und des Buddhismus, da es in ihnen -- in der realen Politik der Staaten, die sich zu ihnen bekannten -- keinen Widerspruch gab, wie dies in den christlichen Staaten der Fall war. Diese Religionen verwirklichten im öffentlichen Leben die Gleichheit aller Menschen desselben Glaubens. 

Die moralischen Folgen des Krieges von 1914-1918 waren enorm und hatten unerwartete Folgen für seine Initiatoren und Macher. Die Hauptsache war ein dramatischer Wandel in der Staatsideologie, die sich mehr oder weniger scharf vom Christentum entfernte und zur Spaltung der Menschheit in feindliche, militante, ideologisch unversöhnliche Staatengruppen führte. 

Dies war eine ideologisch unerwartete Folge des Kampfes um Toleranz -- die Zerstörung der Staatskirche oder ihre tatsächliche Machtlosigkeit im Staat. Eine Art Staatsglaube wurde geschaffen. 

Staatsideologien, die offen auf der Idee der menschlichen, tiefen, biologischen Ungleichheit beruhen, wurden erstmals gestärkt und entwickelt. Sie nahm die Form einer Art Staatsreligion oder -philosophie an, die sich nicht hinter dem Ideal einer einzigen Religion für die ganze Menschheit, der Gleichheit aller Menschen, versteckt. Ungleichheit wurde auch innerhalb der weißen Rasse deklariert und mit der Kraft der Staatsmacht vollzogen. Völker, staatliche Parias, erschienen. Die moralischen Werte des Christentums und des „zivilisierten“ Staates verblassten. Infolgedessen sehen wir eine scharfe moralische Spaltung der Menschheit in Staatsgemeinschaften mit unterschiedlichen Moralvorstellungen. 

Der Vernichtungskrieg unter Einsatz aller Mittel wird wie vor dem Christentum, als die Mittel der Vernichtung und Zerstörung im Vergleich zu ihrer heutigen Macht, die uns theoretisch fast grenzenlos erscheint, vernachlässigbar klein waren, als ein Staatsrecht betrachtet. 

In Deutschland, wo die Hegemonie der deutschen Rasse und die Rassengleichheit als Grundlage des Staates anerkannt werden, in Italien, wo die Gleichheit der römischen Bürger während des Römischen Reiches und ihre rechtliche Gleichstellung zur Schau gestellt werden, und in Japan, wo die besondere Stellung Japans in der Menschheit als ein vom Sohn der Sonne geschaffener Staat anerkannt wird. Für diese Staaten ist alles möglich und zulässig: salus reipublicae suprema lex. 38 Diese Staaten sind jedoch der Ansicht, dass ihre Bevölkerung, ihre vollwertigen Bürger, nicht genügend Raum für ihre Entwicklung und ihr Wachstum haben. 

Für sie ist der Krieg der brutalste, was unvermeidlich ist, weil sie in ihrer Aggression auf verständlichen Widerstand stoßen. 

Ihre Staatsideologie ist die Ideologie der Vergangenheit. Überraschenderweise, ohne tief in die Komplexität des Naturprozesses, der uns in unserer Zeit umgibt, einzudringen, die Staatsideologie der Vergangenheit wiederherzustellen, die ihr widerspricht, gleiten sie tatsächlich an der Oberfläche, sehen sie sich offen wissenschaftlichen Verallgemeinerungen gegenüber, leugnen sie, kämpfen mit Windmühlen effektiv gegen staatliche Dekrete. 

Wie schon in den vergangenen Jahrtausenden haben sie durch staatliche Dekrete versucht, die wissenschaftliche Wahrheit zu ermitteln, indem sie staatlich geförderte Tötungen als ein moralisches Gut anerkennen, das die Tugenden der dominanten Rasse fördert. 

Ihr Ideal beruht auf der ideologischen Anerkennung der biologischen Ungleichheit der menschlichen Rassen. Ihre Konstruktionen zählen nicht zu den wissenschaftlichen Errungenschaften; die Philosophie, die ihre staatlichen Aufgaben rechtfertigt, verzerrt gegebenenfalls wissenschaftliche Errungenschaften oder verwirft sie. 

29. Es wird eine instabile Situation geschaffen, die großes Unglück verursachen kann, [aber] weit entfernt vom Zusammenbruch der Weltzivilisation unserer Zeit. Seine Fundamente sind zu tief, um von diesen erstaunlichen Ereignissen der Gegenwart erschüttert zu werden. 

Sogar die Erfahrungen von 1914-1924 haben es deutlich gezeigt. Vierzehn Jahre sind vergangen, und wir können deutlich sehen, dass das Wachstum der Wissenschaft und die Stärke der Menschheit in der natürlichen Umwelt mit ungezügelter Kraft wächst. 

Nirgendwo sehen wir eine Schwächung der wissenschaftlichen Bewegung inmitten von Kriegen, Vernichtung, Tod durch Mord und Krankheit. All diese Verluste werden schnell durch den mächtigen Aufstieg der realen Errungenschaften der Wissenschaft und ihrer organisierten Staatsmacht und Technologie kompensiert. Es hat sogar den Anschein, dass sie in diesem Kreislauf des menschlichen Unglücks noch weiter wächst und genau die Mittel enthält, um die Versuche zur Stärkung der Barbarei zu beenden. 

Wir müssen jetzt Umstände berücksichtigen, die es in der Geschichte der Menschheit noch nie zuvor in einem solchen Ausmaß gegeben hat. Was wir erleben, kann nicht lang und dauerhaft sein und kann nicht verhindern, dass die Biosphäre in die Noosphäre vordringt, aber wir müssen vielleicht einen Versuch barbarischer Kriege überleben, die gegen Kräfte geführt werden, die eindeutig ungleich sind. 

30. Die wichtigste geologische Kraft, die die Noosphäre erzeugt, ist das Wachstum der wissenschaftlichen Erkenntnisse. 

Als Ergebnis der langen Debatte über die Existenz des Fortschritts, die sich in der Geschichte der Menschheit immer wieder manifestiert hat, kann nun argumentiert werden, dass nur in der Geschichte der wissenschaftlichen Erkenntnis der Fortschritt im Laufe der Zeit nachgewiesen werden konnte. In keinem anderen Bereich des menschlichen Lebens, auch nicht in der Staats- und Wirtschaftsordnung, auch nicht bei der Verbesserung des Lebens der Menschheit -- der Verbesserung der Existenzbedingungen aller Menschen, ihres Glücks -- stellen wir lange Fortschritte mit Stillstand, aber ohne Umkehrung fest. Wir nehmen sie auch nicht im moralischen, philosophischen und religiösen Zustand der menschlichen Gesellschaften wahr. Aber im Zuge der wissenschaftlichen Erkenntnis, d.h. der Stärkung der geologischen Kraft des zivilisierten Menschen in der Biosphäre, im Wachstum der Noosphäre, sehen wir es deutlich. 

J. Sarton39 bewies in seinem Buch, dass ab dem 7. Jahrhundert v. Chr. unter Berücksichtigung des fünfzigjährigen Jubiläums und unter Berücksichtigung der gesamten Menschheit, nicht nur der westeuropäischen Zivilisation, das Wachstum der wissenschaftlichen Erkenntnisse kontinuierlich war. Und mit kurzen Stopps wurde das Tempo immer höher und höher. 

Merkwürdigerweise handelt es sich dabei um die gleiche Art der Wachstumskurve, die in der paläontologischen Entwicklung der tierischen Lebewesen beobachtet wird -- das Wachstum ihres zentralen Nervensystems. 

Es scheint mir, dass sich dieser Prozess, wenn wir die Geschichte der Verbesserung der Lebenstechniken berücksichtigen, noch schärfer und heller zeigen würde. Das ist eine Geschichte, die wir noch nicht haben. In den letzten Kapiteln von Sartons Werk aus den XI-XII Jahrhunderten bis nach R.C. manifestiert es sich bereits. 

Offensichtlich weisen 50 Jahre, also etwa zwei Generationen, auf eine durchschnittliche Genauigkeit hin, mit der wir dieses Phänomen heute beurteilen können. Schon vor etwa zweitausend Jahren waren wir um ein Vielfaches genauer. 

Leider wird diese wissenschaftlich-empirische Verallgemeinerung meist nicht berücksichtigt, dennoch ist sie von großer Bedeutung. Natürlich muss sie geklärt werden, aber die Tatsache selbst steht außer Zweifel, und weitere Untersuchungen werden wahrscheinlich zeigen, dass sie noch schärfer ausgedrückt wurde, als wir jetzt denken. 

31. Die folgenden Phänomene werden heute beobachtet und lassen vermuten, dass die Befürchtungen eines möglichen Zusammenbruchs der Zivilisation (in Wachstum und Stabilität der Noosphäre) unbegründet sind. 

Erstens hat es in der Geschichte der Menschheit noch nie ein Universum gegeben, das heute beobachtet werden kann -- einerseits eine vollständige Übernahme der Biosphäre durch den Menschen für das Leben und andererseits eine mangelnde Loslösung der einzelnen Siedlungen aufgrund der Schnelligkeit der Kommunikation und Bewegung. Kommunikation kann für alle sofort und laut geschehen. Bald wird es möglich sein, alle Ereignisse, die Tausende von Kilometern entfernt stattfinden, sichtbar zu machen. Umzüge und Transfers von Dingen können theoretisch in jedem beliebigen Ausmaß beschleunigt werden, und ihr Tempo wächst schneller als je zuvor. 

Zweitens: Niemals in der Geschichte der Menschheit wurden die Interessen und das Wohl aller, nicht die von Einzelpersonen oder Gruppen, als eine wirkliche Staatsaufgabe festgelegt, und die Volksmassen sind zunehmend in der Lage, den Lauf der staatlichen und öffentlichen Angelegenheiten bewusst zu beeinflussen. Erstmals war und ist die Bekämpfung der Armut und ihrer Folgen (Unterernährung) sowie der biologischen Wissenschaften und der staatlich-technischen Aufgabe nicht mehr wegzudenken. 

Drittens wurde zum ersten Mal das Problem der bewussten Regulierung der Fortpflanzung, der Verlängerung des Lebens, der Schwächung von Krankheiten für die gesamte Menschheit zur gleichen Aufgabe gemacht. 

Zum ersten Mal wird die gleiche Aufgabe für das Eindringen der wissenschaftlichen Erkenntnisse in die gesamte Menschheit gestellt. 

Eine solche Ansammlung menschlicher Handlungen und Ideen hat es noch nie gegeben, und es ist klar, dass diese Bewegung nicht aufgehalten werden kann. Insbesondere für die nahe Zukunft stehen die Wissenschaftler vor den beispiellosen Aufgaben einer bewussten Steuerung der Organisation der Noosphäre, von denen sie sich aufgrund des natürlichen Verlaufs des wissenschaftlichen Erkenntniswachstums nicht abbringen können. 

Es gibt einen weiteren Umstand, der noch nicht klar zum Ausdruck gebracht wurde, der sich aber deutlich abzeichnet. Dies ist die Internationalität der Wissenschaft, ihr Wunsch nach Gedankenfreiheit und das Bewusstsein der moralischen Verantwortung von Wissenschaftlern, wissenschaftliche Entdeckungen und wissenschaftliche Arbeit für ein destruktives, der Idee der Noosphäre zuwiderlaufendes Ziel zu nutzen. Diese Strömung hat sich noch nicht entwickelt, aber ich habe den Eindruck, dass sich die wissenschaftliche öffentliche Meinung in der Welt in den letzten Jahren rasch in diese Richtung entwickelt und erweitert hat. In der Geschichte der Philosophie und Wissenschaft, insbesondere in der Epoche der Renaissance und zu Beginn der Neuzeit, als die lateinische Sprache eine wissenschaftliche Sprache außerhalb der Länder und Nationalitäten war, spielte die reale, aber unformulierte Internationale der Wissenschaftler eine große Rolle und hatte tiefe Wurzeln in der mittelalterlichen Einheit der realen, aber unformulierten Jahrhundert-Internationale der Philosophen und Wissenschaftler. 

Die Traditionen der internationalen Wissenschaftler sind also tief verwurzelt, das Bewusstsein für ihre Notwendigkeit dringt zunehmend durch, und dieser Fluss geht im Einklang mit der Schaffung der Noosphäre als Ziel. Aber diesmal muss sich der Charakter einer wissenschaftlichen Internationale zwangsläufig von dem unterscheiden, der sich im muslimischen und katholischen Milieu verbarg, das das Gesicht der Rechtschaffenheit hatte, mehr philosophisch als wissenschaftlich, den Kreis der Gelehrtengenerationen des Mittelalters. Heutzutage sind Wissenschaftler eine echte Kraft; Spezialisten, Ingenieure und Wirtschaftswissenschaftler, Theoretiker, angewandte Chemiker, Zootechniker, Agronomen, Ärzte (die früher die Hauptrolle spielten) bilden die Hauptmasse und repräsentieren die gesamte kreative Kraft der Triebkräfte der Nationen. 

All dies weist darauf hin, dass die reale Situation in unseren turbulenten und blutigen Zeiten es uns nicht erlauben kann, die Kräfte der Barbarei, die jetzt an prominenter Stelle zu stehen scheinen, zu entwickeln und zu besiegen. Alle Ängste und Spekulationen der breiten Öffentlichkeit, der Vertreter humanitärer und philosophischer Disziplinen über die Möglichkeit der Zerstörung der Zivilisation sind mit einer Unterschätzung der Macht und Tiefe geologischer Prozesse verbunden, die jetzt, wie wir sie erleben, mit dem Übergang von der Biosphäre in die Noosphäre einhergehen. 

Ich werde auf die weitere Klärung der Noosphäre und die Unveränderlichkeit ihrer Entstehung und damit die Schaffung neuer Formen menschlichen Lebens zurückkommen. 

Nun noch ein paar Gedanken zum Fortschritt der wissenschaftlichen Erkenntnisse. 

32. Um die gegenwärtige Bewegung der Wissenschaft wissenschaftlich zu verstehen, ist es zunächst notwendig, die Wirklichkeit in den wissenschaftlichen Rahmen zu stellen und den Verlauf der wissenschaftlichen Erkenntnis logisch mit ihr zu verknüpfen. Die Geschichte der Menschheit, wie auch das Leben jedes einzelnen Menschen, kann nicht losgelöst und getrennt von seiner „Umwelt“ betrachtet werden. Diese Aussage erregt in solch allgemeiner Form keinen Zweifel, egal welche Definition von „Umwelt“ wir machen und egal welche Annahmen über die Notwendigkeit, andere, gleichwertige Faktoren aus dem Umfeld unabhängiger, auf philosophischen oder religiösen Überzeugungen basierender Faktoren anzuerkennen, sie würde es nicht erlauben. 

Im wissenschaftlichen Bereich der Natur gehen wir von dieser Grundannahme aus -- dem kausalen Zusammenhang aller Phänomene der Umgebung, reduzieren die Phänomene auf eine einzige. Die Existenz von Faktoren „aus der Umwelt“ unabhängig, in der Wissenschaft wird nicht akzeptiert, auf der Grundlage der Anerkennung der Einheit der Wirklichkeit, der Einheit des Kosmos. 

Ich bin nicht hier, um diese Art des wissenschaftlichen Denkens zu erklären, um zu beweisen, dass es richtig oder notwendig ist. Ich sage nur, was wirklich passiert, dessen Stärke und Korrektheit das moderne wissenschaftliche Denken offenbart, das unser ganzes Leben bei jedem Schritt aufbaut. 

Wenn ich auf dem Boden der wissenschaftlichen Suche bleibe und logisch korrekt argumentiere, brauche ich nicht weiter zu gehen. 

Die Entwicklung der Wissenschaft im XX. Jahrhundert führte -- unerwartet, rein empirisch -- zur Einschränkung dieser jahrhundertealten Regel der wissenschaftlichen Arbeit. Es wurden drei getrennte Realitätsschichten gefunden, innerhalb derer wissenschaftlich fundierte Fakten geschlossen sind. Diese drei Schichten unterscheiden sich anscheinend stark in den Eigenschaften der Raumzeit. Sie durchdringen sich gegenseitig, sind aber definitiv geschlossen, unterscheiden sich stark voneinander in der Aufrechterhaltung und in einer Technik der in ihnen untersuchten Phänomene. Es handelt sich um Schichten: die Phänomene der kosmischen Räume, die Phänomene der Planeten, unsere uns nahestehende „Natur“ und die Phänomene des Mikroskopischen, bei denen die Schwerkraft in den Hintergrund gedrängt wird. 

Die wissenschaftlichen Phänomene des Lebens werden nur in den letzten beiden Schichten der Weltrealität beobachtet. 

In der wissenschaftlichen Erfassung der Wirklichkeit braucht man nicht mit anderen Vorstellungen über sie zu rechnen, indem man in einer wissenschaftlich untersuchten Wirklichkeit die Existenz von Konstruktionen zulässt, die von der wissenschaftlichen Suche in der Aufmerksamkeit nicht akzeptiert und wissenschaftlich in ihr nicht geöffnet worden sind. Gewöhnliche, vorherrschende Vorstellungen von der Welt -- von der Realität -- sind überfüllt mit religiösen, philosophischen, historischen und sozialen Konstruktionen, die oft im Widerspruch zu wissenschaftlich akzeptierten und manchmal von einzelnen Forschern oder Forschergruppen in der wissenschaftlichen Arbeit berücksichtigt werden. 

Der Widerspruch zwischen diesen Begriffen durchzieht das wissenschaftliche Denken; die wissenschaftliche Reichweite der Wirklichkeit ist ständig mit ihnen konfrontiert. Sie bricht fremde Konstruktionen auf, wenn es notwendig ist, und alle anderen von der Menschheit entwickelten Konzeptionen der Wirklichkeit -- religiöse, philosophische, sozialstaatliche -- müssen überarbeitet und ihr im Falle ihres Widerspruchs zur wissenschaftlich gefundenen Wahrheit übergeben werden. Das Primat des wissenschaftlichen Denkens auf seinem Gebiet -- der wissenschaftlichen Arbeit -- existiert immer, ob es anerkannt wird oder nicht, gleichgültig. Seine korrekt getroffenen Bestimmungen sind allgemein verbindlich. Es hängt nicht von unserem Willen ab. Sie ist dem geistigen Leben der Menschheit nur wissenschaftliche Wahrheit inhärent. 

Diese Aussage bedarf im Wesentlichen keines Beweises; sie ergibt sich als empirische Tatsache aus der Beobachtung der Geschichte des wissenschaftlichen Denkens. 

In Momenten wie heute wird das besonders deutlich. 

33. Wissenschaft und wissenschaftliche Arbeit sind, als Ganzes betrachtet, nicht nur das Ergebnis der Arbeit einzelner Wissenschaftler, ihrer bewussten Suche nach wissenschaftlicher Wahrheit. 

Wissenschaft und wissenschaftliche Arbeit, wissenschaftliches Denken, ist in der Regel nicht die Identifizierung des Amtes Wissenschaftler, fern vom Leben, Vertiefung in ihm geschaffen oder unabhängig von der Umgebung sie frei gewählte wissenschaftliche Problem. Der mittelalterliche westeuropäische Mönch, der für kurze Zeit die Wissenschaft seiner Zeit leitete, war jedoch kein Einsiedler der Wissenschaft im Allgemeinen, er war nicht durch tausend Fäden mit dem Leben und dem Priester des alten Ägypten oder Babylon oder einem Wissenschaftler des XVII. Sie und die Mehrheit der Wissenschaftler waren nicht jene Menschen, die nicht von dieser Welt waren, was mehr als einmal künstlerische Kreativität und gemeinsame Gerüchte anzog und anzieht. Das waren nur einzelne Gelehrte, säkulare Menschen -- Amateure, einzelne Mönche oder Einsiedler, aber sie gingen in der allgemeinen Menge der Wissenschaftler völlig verloren, und ihre Rolle, geehrt und manchmal notwendig, ist nur bei genauer und detaillierter Untersuchung des wissenschaftlichen Schaffens sichtbar und berührt. Sie sind nicht die Schöpfer der Wissenschaft. 

Wissenschaft ist die Schaffung von Leben. Aus dem umgebenden Leben holt sich der wissenschaftliche Gedanke das Material, das er in Form der wissenschaftlichen Wahrheit mitbringt. Sie, die Dicke des Lebens, erschafft es zuerst. Sie ist ein natürliches Spiegelbild des menschlichen Lebens in der menschlichen Umwelt -- in der Noosphäre40. Wissenschaft ist die Manifestation des Handelns in der menschlichen Gesellschaft, die Gesamtheit des menschlichen Denkens. 

Die wissenschaftliche Struktur ist in der Regel real, nicht logisch kohärent, in allen ihren Grundlagen bewusst durch das Verstandessystem des Wissens bestimmt. Sie ist voller ständiger Veränderungen, Korrekturen und Widersprüche, sie ist äußerst beweglich wie das Leben, komplex in ihrem Inhalt, und es herrscht ein dynamisches, instabiles Gleichgewicht. 

Logisch stimmig können und manchmal gibt es nur rationalistische oder mystische Konstruktionen philosophischer Systeme oder theologische (und mystische) Offenbarungen von Religion, für die es zunächst für die Wahrheit anerkannte Positionen gibt, streng logisch weiterentwickelt und vertieft werden, unabhängig von den Tatsachen der umgebenden Natur (einschließlich des sozialen Umfelds der Menschheit). 

Das Wissenschaftssystem als Ganzes betrachtet, ist von einem logischen und kritischen Standpunkt aus betrachtet immer unvollkommen. Nur ein Teil davon, auch wenn er wächst, ist unbestreitbar (Logik, Mathematik, wissenschaftlicher Apparat von Fakten). Die wirklich existierenden, in der Geschichte der Menschheit und in der Biosphäre historisch manifestierten Wissenschaften sind immer von unzähligen, für die Zeitgenossen oft untrennbaren, ihnen und ihnen im historischen Prozess fremden, verarbeiteten philosophischen, religiösen, sozialen und technischen Verallgemeinerungen und Errungenschaften bedeckt, deren Verarbeitung im Wesentlichen der Hauptinhalt der Wissenschaftsgeschichte ist. Nur ein Teil, aber, wie wir es wachsen sehen, ein Teil der Wissenschaft, nämlich ihr Hauptinhalt, der von den Wissenschaftlern oft so wenig berücksichtigt wird und anderen Manifestationen des geistigen Lebens der Menschheit oft fremd ist -- die Masse ihrer wissenschaftlichen Fakten und korrekt logisch konstruierten wissenschaftlichen empirischen Verallgemeinerungen ist unbestreitbar und logisch unbedingt für alle Menschen und für alle ihre Darstellungen obligatorisch und unbestreitbar41. Die Wissenschaft als Ganzes hat keine solche Pflicht. 

34. Wissenschaft ist also keineswegs ein logisches Konstrukt, eine Maschine zur Wahrheitssuche. Man kann die wissenschaftliche Wahrheit nicht durch Logik erkennen, man kann sie nur durch das Leben erkennen. Handeln ist ein charakteristisches Merkmal des wissenschaftlichen Denkens. Wissenschaftliches Denken -- wissenschaftliche Kreativität -- wissenschaftliche Erkenntnisse gehen in die Tiefe des Lebens, mit dem sie untrennbar verbunden sind, und ihre bloße Existenz, sie erregen in der Umgebung des Lebens, aktive Manifestationen, die an sich nicht nur die Verbreiter der wissenschaftlichen Erkenntnisse sind, sondern auch ihre unzähligen Formen der Entdeckung zu schaffen, verursachen unzählige große und kleine Quelle des Wachstums der wissenschaftlichen Erkenntnisse. 

Es ist nicht immer so, dass die menschliche Person, auch in unserer Zeit der organisierten Wissenschaft, der Schöpfer einer wissenschaftlichen Idee und wissenschaftlicher Erkenntnis ist; ein Forscher, der ein rein wissenschaftliches Werk lebt, im Großen wie im Kleinen, ist einer der Schöpfer wissenschaftlichen Wissens. Zusammen mit ihm, aus dem Dickicht des Lebens werden von Einzelpersonen, zufällig, dh, Alltag, im Zusammenhang mit wissenschaftlich wichtig und aus den Überlegungen, oft Wissenschaft fremd, Eröffnung wissenschaftlichen Fakten und wissenschaftlichen Verallgemeinerungen, manchmal grundlegende und entscheidende, Hypothesen und Theorien, Wissenschaft ist weit verbreitet. 

Solche wissenschaftliche Kreativität und wissenschaftliche Suche, die von den Handlungen ausgehen, die außerhalb der wissenschaftlichen, bewusst organisierten Arbeit der Menschheit liegen, sind aktiv-wissenschaftliche Darstellung des Lebens der denkenden menschlichen Umwelt der gegebenen Zeit, Darstellung ihrer wissenschaftlichen Umwelt. Durch die Masse an Neuem in dieser Form des wissenschaftlichen Denkens, das in die Wissenschaft eingeführt wurde, und seine Bedeutung für das historische Ergebnis dieses Teils des wissenschaftlich Konstruierten ist, so scheint mir, vergleichbar mit dem, was in den bewusst arbeitenden Wissenschaftler eingeführt wird, was sich durch die bewusste Organisation der wissenschaftlichen Arbeit offenbart. Ohne gleichzeitig existierende wissenschaftliche Organisation und wissenschaftliches Umfeld verschwindet diese immer existierende Form der wissenschaftlichen Arbeit der Menschheit, spontan unbewusst, und gerät weitgehend in Vergessenheit, wie es auf dem Gebiet der mediterranen Zivilisation viele Jahrhunderte lang im christianisierten Römischen Reich, in persischen, arabischen, berberischen, deutschen, slawischen, keltischen Gemeinschaften Westeuropas im Zusammenhang mit dem Staatszerfall der etablierten Staatsformationen im IV-XII Jahrhundert nach R.H., teilweise später, geschah. Die Wissenschaft verliert im Laufe der Zeit ihre Errungenschaften und kommt spontan wieder zu ihnen zurück. 

Die Geschichte der Wissenschaft und die Geschichte der Menschheit offenbaren solche Ereignisse auf Schritt und Tritt. Die Blütezeit der hellenistischen Wissenschaft ließ solche Errungenschaften der alltäglichen chaldäischen Wissenschaft, wie die Algebra von Babylon, beiseite und benutzte sie nicht oder erst spät (nach Jahrtausenden). 

35. Aber die Umwelt des Lebens beeinflusst das wissenschaftliche Denken nicht nur auf diese Weise -- indem sie die wissenschaftlichen Entdeckungen des Lebens überall hinbringt, die äußere wissenschaftliche Suche der einzelnen Personen und ihre Erfassung durch die organisierte Manifestation der wissenschaftlichen Arbeit der Wissenschaftler, den wissenschaftlichen Apparat der Zeit. 



Oftmals waren jedoch die Veränderungen im Wirtschaftsleben, in der Agrarkultur oder in bestimmten Manifestationen des heimischen Erfolges, wie z.B. die Einführung des Kamels (Dromader) in den Wüsten- und Halbwüstenregionen Nordafrikas43 oder die Öffnung des Buchdrucks in den Rheinländern Europas, noch stärker. 44 

Neben diesen Naturphänomenen, deren Folgen für das wissenschaftliche Denken nicht berücksichtigt wurden, wirkt bei ihrer Entstehung durch den Menschen auch das naturwissenschaftliche Denken gleichberechtigt und manchmal vielleicht sogar in größerem Maße in der Biosphäre -- wissenschaftliche Entdeckungen einzelner Denker und Wissenschaftler, die die Weltdarstellung der Menschheit verändern, wie Kopernikus, Newton, Linnaeus, Darwin, Pasteur, P. Curie. In diesen Fällen geschah es bewusst, in anderen -- unerwartet für den Wissenschaftler selbst, wie es vor unseren Augen mit A. Becquerel [1852-1908], der 1896 die Radioaktivität entdeckte,45 oder mit G. Ersted [1777-1851], der den Elektromagnetismus entdeckte,46 oder mit L. Galvani [1737-1798], der den galvanischen Strom entdeckte, geschah. 47 

Maxwell, Lavoisier, Amper, Faraday, Darwin, Dokuchaev, Mendeleev und viele andere machten sich riesige wissenschaftliche Entdeckungen zu eigen, die sie im vollen Bewusstsein ihrer grundlegenden Bedeutung für das Leben, aber unerwartet für ihre Zeitgenossen schufen. 48 

Ihr Denken -- für sie bewusst -- beeinflusste die Seele des Lebens; hier, durch diese Art und Weise hervorgerufen, wandten sie Schöpfungen in einer neuen Form an, die für ihre Zeitgenossen unerwartet und negativ war, oft nach dem Tod ihrer Schöpfer, und spiegelten sich auf neue Weise in der wissenschaftlichen Kreativität wider, schufen im Leben des Menschen eine Revolution seines Lebens, neue unerwartete Quellen wissenschaftlicher Erkenntnis. 

Mit ihnen zusammen, mitten durch das Leben, durch die Umwelt, schaffen Erfinder einen neuen, ähnlichen Kreislauf wissenschaftlicher Probleme, unter ihnen sind oft wissenschaftliche Analphabeten -- aus allen sozialen Schichten und Kreisen, Menschen, die oft keine Beziehung und kein Interesse an der Suche nach wissenschaftlicher Wahrheit hatten. 49 

36. Aus all dem werden wir sehen, dass es möglich ist, Schlussfolgerungen von großer wissenschaftlicher Bedeutung zu ziehen, nämlich 

    1. Der Lauf der wissenschaftlichen Schöpfung ist die Kraft, mit der der Mensch die Biosphäre, in der er lebt, verändert. 

    2. Diese Manifestation des Wandels in der Biosphäre ist ein unvermeidliches, begleitendes Phänomen im Wachstum des wissenschaftlichen Denkens. 

    3. Diese Veränderung der Biosphäre vollzieht sich unabhängig vom menschlichen Willen, spontan, als ein natürlicher, natürlicher Prozess. 

    4. Und da die Umwelt des Lebens eine organisierte Hülle des Planeten ist -- die Biosphäre, der Eintritt in sie, im Laufe ihres geologisch langen Bestehens, ist der neue Faktor ihrer Veränderung -- die wissenschaftliche Arbeit der Menschheit -- ein natürlicher Prozess des Übergangs der Biosphäre in eine neue Phase, in einen neuen Zustand -- in die Noosphäre. 

    5. In dem historischen Moment, den wir erleben, sehen wir ihn klarer, als wir ihn bisher gesehen haben. Hier öffnet sich das „Naturgesetz“ vor uns. Die neuen Wissenschaften -- Geochemie und Biogeochemie -- bieten die erste Gelegenheit, einige wichtige Merkmale des Prozesses mathematisch auszudrücken. 

37. In diesem Aspekt erkennen die Geologen (Abschnitt 15) die Entstehung der Gattung Homo, eines Menschen, als Indikator für eine neue Ära in der Geschichte des Planeten an. Bisher wurden geologische Prozesse als Grundlage für die Einteilung in geologische Systeme und geologische Epochen genommen, die sich auf die gesamte Erdkruste und nicht nur auf die Biosphäre ausdehnen. Dennoch war der dramatische Wandel der Formen der lebenden Bevölkerung des Planeten schon immer ein wesentliches Merkmal geologischer Systeme und Epochen. Wie wir heute wissen, steht sie in engem Zusammenhang mit großen Perioden orogener, tektonischer, vulkanischer -- wir können sagen, kritischer -- Perioden der Erdkrustengeschichte. 

Im menschlichen Zeitalter oder Psychozoikum (Abschnitt 15) haben wir tatsächlich ein schärferes Bild als in den kritischen Perioden der Erdkruste. Wir erleben jetzt eine dramatische Veränderung der gesamten Fauna und Flora, die Vernichtung einer großen Zahl von Arten und die Schaffung neuer kultureller Rassen. Gleichzeitig verändern die Landwirtschaft und die Schaffung eines neuen Antlitzes des Planeten, zweifellos jenseits des Willens und des Verständnisses des Menschen, wilde Arten von Organismen, die sich an die neuen Lebensbedingungen in einer Biosphäre mit veränderter Kultur anpassen. Vor allem aber hat eine Art von Organismen -- der Homo sapiens faber -- den gesamten Planeten umarmt und eine dominierende Stellung unter den Lebewesen eingenommen. Das ist noch nie passiert. 

Wir stehen erst am Anfang des Prozesses und können den Gedanken an die unausweichliche Zukunft noch nicht fassen, aber es ist bereits klar, dass mehr als eine Person davon profitiert. A. Clarke hat eine Reihe von Fakten über die Nutzung aller Vorteile der Zivilisation durch Insekten aufgezeigt und konnte die Aufmerksamkeit auf die Möglichkeit des Ergebnisses lenken, dass mehr Menschen von der Wiederverwertung der Biosphäre durch Insekten profitieren. 50 Auf der anderen Seite sehen wir dasselbe Phänomen bei Krankheiten von Kulturpflanzen, Tieren und Menschen in der Welt der Einzeller, Pilze und Mikroben. 

38. Der Mensch, Homo sapiens, ist zwar ein oberflächliches Phänomen in einer der Schalen der Erdkruste -- in der Biosphäre, aber der neue geologische Faktor, der durch sein Erscheinen in der Geschichte des Planeten eingeführt wurde -- der Geist -- ist in seinen Folgen und ihren Möglichkeiten so groß, dass wir meiner Meinung nach nichts gegen die Einführung dieses Faktors für geologische Einheiten zusammen mit der Stratigraphie und Tektonik einwenden können. Das Ausmaß der Veränderungen ist vergleichbar. 

Darüber hinaus können wir, vielleicht auf diese Weise, die Dauer der geologischen kritischen Periode unseres Planeten mit großer wissenschaftlicher Tiefe verstehen. Bei der Schaffung der Noosphäre erleben wir sie, und offensichtlich erscheint sie uns in einem völlig anderen Licht und wir befinden uns in einer völlig anderen Position als bei der Beurteilung der geologischen Vergangenheit, als wir nicht auf dem Planeten waren. Zum ersten Mal werden die geologischen Auswirkungen des Lebens in ihrer historischen Dauer deutlich, die sich in der Kürze der historischen Zeit manifestiert. 

“Das denkende Schilfrohr 51, der Schöpfer der Wissenschaft in der Biosphäre, kann und muss hier den geologischen Verlauf der Ereignisse anders beurteilen, denn es ist das erste Mal, dass er seine Position in der Organisation des Planeten wissenschaftlich verstanden hat. 

Denn wir können deutlich sehen, dass mit seinem Erscheinen in der Geschichte des Planeten einen neuen mächtigen geologischen Faktor enthüllt, der in Bezug auf mögliche Folgen jene tektonischen Bewegungen übertrifft, die -- rein empirisch, empirische Verallgemeinerung -- die Grundlage der geologischen Unterteilungen der Raumzeit der Erde gelegt wurden. 

Dies wird deutlich, wenn wir berücksichtigen, dass die Dauer geologischer Phänomene eine andere Wirkung hat als die Dauer der aktuellen historischen Ereignisse, in denen wir leben. 52 Hunderttausend Jahre -- die Dekamyriade -- mit einer Dauer von drei Milliarden Jahren, die wir für unser geologisches Beobachtungsgebiet getrost einplanen können, werden einem vernachlässigbaren Bruchteil der geologischen Sekunde entsprechen. 

Nur unsere fernen Nachkommen können die biogene Wirkung des wissenschaftlichen Denkens wirklich erkennen: Es wird sich erst nach Hunderten, kaum Dutzenden von Dekamyriaden hell und deutlich zeigen, wie sich die Dauer jener Verschiebungen, die sich in stratigraphischen Brüchen ausdrücken und die wir unseren geologischen Epochen und Systemen zugrunde legen, verschiebt. 53 Es handelt sich nicht um augenblickliche Revolutionen -- die Dauer ihrer intensiven Manifestation, ausgedrückt z.B. in abweichenden Schichten, betrachtet auf der Skala der historischen Zeit, umfasst eine riesige Zeitspanne -- Hunderte oder Zehntausende von Jahren, kaum weniger. 

Wir arbeiten heute in der Wissenschaft mit einer solchen Präzision, dass wir die Macht der Folgen der geologischen Manifestationen (d.h. der Reflexion in geologischer Zeit) der wiederverwerteten wissenschaftlichen Idee der Biosphäre antizipieren und quantifizieren können. Jetzt beobachten wir nur noch die Manifestationen ihrer geologischen Arbeit in historischer Zeit. Aber selbst hier können wir deutlich sehen, dass sich die Biosphäre radikal verändert hat. 

Die Entstehung des Geistes und seine genaueste Erkennung -- die Organisation der Wissenschaft -- ist eine primäre Tatsache in der Geschichte des Planeten, vielleicht in einer Tiefe des Wandels, die größer ist als alles, was uns bisher in der Biosphäre bekannt war. Sie wurde für eine Milliarde Jahre evolutionärer Prozesse vorbereitet, und wir sehen ihre Wirkung jetzt, die größte nur in geologischen Minuten. 

39. Von entscheidender Bedeutung für das Verständnis des planetarischen Sinns des Lebens aufgrund des Auftretens eines vernünftig denkenden und wissenschaftlich arbeitenden Lebewesens während der geologischen Zeit ist, dass dieses Auftreten mit dem Prozess der Evolution des Lebens verbunden ist, der geologisch immer ohne Verschwendung rückwärts verläuft, aber mit Unterbrechungen in derselben Richtung -- zur Klärung und Verbesserung des Nervengewebes, insbesondere des Gehirns. Es ist auffällig, wenn man die Abfolge der geologischen Schichtung mit den archäologischen und morphologischen Strukturen vergleicht, die ihren Lebensformen entsprechen. 

Dieser evolutionäre Prozess, ausgedrückt durch den polaren Vektor, d.h. das Zeigen von Orientierung, der mehr als zwei Milliarden Jahre dauerte, führte unweigerlich zur Entstehung des menschlichen Gehirns der Gattung Homo, vor etwa einer halben Million Jahren. 

Ohne die Bildung des menschlichen Gehirns gäbe es kein wissenschaftliches Denken in der Biosphäre, und ohne wissenschaftliches Denken gäbe es keinen geologischen Effekt -- Rekonfiguration der Biosphäre durch den Menschen. 

Das charakteristischste Merkmal dieses Prozesses ist die Ausrichtung des evolutionären Prozesses des Lebens in der Biosphäre unter diesem Gesichtspunkt. Diese Ausrichtung ist, wie wir sehen werden, eng mit dem Hauptunterschied verbunden, der die lebende Materie von der kosmischen Materie trennt54 , und entspricht ganz speziellen Offenbarungen der Energiewirkung des zeitlichen Verlaufs des Lebens und der ganz speziellen Geometrie des von Lebewesen besetzten Raumes in der Biosphäre. 

Ich werde auf dieses Problem weiter unten zurückkommen,55 hier werde ich nur anmerken, dass die erste Person, die, ohne die geologischen Konsequenzen zu berücksichtigen, obwohl er ein bedeutender Geologe war, die unveränderliche intermittierende Ausrichtung des Evolutionsprozesses auf die Verbesserung des Gehirns während der geologischen Zeit sah, 1855 J.D. Dana in New Haven war. 56 

Neben der großen empirischen Verallgemeinerung von Ch. Darwin entwickelte sich die empirische Verallgemeinerung von D. Dun während der langen Weltreise auf dem Schiff „Peacock“ (1838-1842) in der Expedition von Wilkes, zeitgleich mit der Expedition „Biggle“ (1831-1836), unter dem Einfluss der Reflexion und wissenschaftlichen Arbeit eines jungen Naturforschers im Laboratorium der Natur. In beiden Fällen arbeiteten sowohl Darwin als auch Dana unter Bedingungen, unter denen ihnen das Leben in der Biosphäre über die wenigen Jahre hinweg kontinuierlich in seinem planetarischen Aspekt offenbart worden war. Diese Form der Arbeit findet in der Wissenschaftsgeschichte nicht oft statt. 

40. Es ist äußerst charakteristisch, dass das geologische Handeln des Menschen bei der Umstrukturierung der Biosphäre erst lange nach seinem Auftreten in der Biosphäre betroffen ist. „Homo, die menschliche Spezies, erschien vor vielen Dekamyriaden (vor etwa einer Million Jahren?); Homo sapiens, wahrscheinlich vor etwa einer halben Million Jahren. 57 

Aber schon vor der Entdeckung der Gattung Homo erreichte das Gehirn seiner Vorfahren oder ihm nahestehender Organismen ein Niveau, das seine geistige Aktivität von der anderer Säugetiere unterschied. Sinanthropus pekinensis, der als der Vorfahre der Gattung Homo angesehen werden kann, hatte bereits eine Kultur, besaß Feuer und offenbar auch Sprache. 58 Die Wurzeln der geologischen Kraft des Geistes lassen sich anscheinend tiefer in die Homo-Ära zurückverfolgen, weit über die Dekamyriaden hinaus bis zur Entdeckung der Gattung Homo. 

Die Auswirkungen des Homo sapiens selbst auf die Erdoberfläche begannen viele tausend Generationen nach seinem Auftreten zu spüren. 

Es ist möglich, dass wir es hier mit Phänomenen zu tun haben, die die anatomische Struktur des Denkapparates -- des Gehirns -- nicht beeinflussen und die das Ergebnis einer langen Einflussnahme des sozialen Umfelds sind. 

Die Methode zur Untersuchung des Gehirns ist anatomisch so empfindlich für den damit verbundenen Verstand, dass einer der größten Anatomen, G.E. Smith [1871-1937],59 angab, er sehe keinen signifikanten Unterschied zwischen dem menschlichen Gehirn und dem Affenhirn. Dies kann kaum anders interpretiert werden, als unsensibel und unvollständig zu sein. Denn es kann kein Zweifel daran bestehen, dass es in der Biosphäre des menschlichen und des Affengehirns einen scharfen Unterschied in der eng verwandten geologischen Wirkung und Struktur des Gehirns gibt. 

Anscheinend sehen wir in der Entwicklung des Geistes eine Manifestation nicht einer grob anatomischen, die sich in der geologischen Dauer durch Veränderung des Schädels offenbart, sondern einer subtileren Veränderung im Gehirn, die mit dem sozialen Leben in seiner historischen Dauer verbunden ist. 

Dann ist es klar, dass lange Generationenwechsel notwendig sind, damit die für den Homo sapiens charakteristischen wissenschaftlichen Erkenntnisse die Arbeit des Menschen beeinflussen können, der die Planetenoberfläche verändert. Zehntausende von Generationen sind seit seinem Auftreten in der Biosphäre vergangen, bevor diese Manifestation sichtbar wurde. 

Diese deutlichere Auswirkung auf die Veränderung der Oberfläche des Planeten kann seit der Eröffnung des Feuers und der Landwirtschaft vor kaum weniger als 80.000 bis 100.000 Jahren betrachtet werden. 60 Aus dieser Zeit, in der sich der Einfluss des Menschen auf die ihn umgebende Natur bereits unweigerlich manifestierte, Wissenschaft und organisierte wissenschaftliche Forschung aber noch weit entfernt waren, vergingen viele neue Jahrtausende bis zum wissenschaftlichen Denken und der damit unweigerlich verbundenen bekannten Organisation, denn wissenschaftliches Denken ist ein soziales Phänomen und nicht nur die Schöpfung herausragender individueller Köpfe. Ihnen müssen Bedingungen des gesellschaftlichen Lebens vorausgehen, unter denen der Einzelne seine Gedanken in einem sozialen Umfeld in die Tat umsetzen kann. Es ist sehr wahrscheinlich, dass diese ersten Organisationsformen der Wissenschaft lange vergänglich waren, und viele Jahrhunderte, oder besser gesagt Jahrtausende, sind vergangen, bevor sie etabliert wurden. 

Leider ist unser Wissen in diesem Bereich trotz bedeutender Fortschritte in Anthropologie, Geschichte und Archäologie immer noch sehr unzuverlässig. 

Ich sehe die folgende Darstellung als eine vorübergehende erste Annäherung, die weiteren größeren Änderungen und Verfeinerungen unterliegt. Die Hauptschlussfolgerung, dass die wissenschaftliche Bewegung des 20. Jahrhunderts eines der größten Phänomene in der Geschichte des wissenschaftlichen Denkens ist, bleibt jedoch unberührt. 

Anscheinend wurden vor 5-6 Tausend Jahren die ersten genauen Aufzeichnungen wissenschaftlicher Fakten im Zusammenhang mit astronomischen Beobachtungen von Himmelslichtern gemacht. Sie wurden in der Region Mesopotamien gegründet, im Gebiet einer der ältesten Kulturen, ihren Zentren. 

Vielleicht gab es sogar schon früher Mathematik -- sowohl Arithmetik, Algebra als auch Geometrie. 

Aus den Bedürfnissen der Landwirtschaft und der damit verbundenen Bewässerung bei der Schaffung kultureller Gesellschaften entwickelten sich gleichzeitig die Anfänge der Geometrie, und aus den Bedürfnissen des komplexen Lebens großer Staaten -- Handel, militärische und fiskalische Bedürfnisse -- entwickelten sich die Grundlagen der Arithmetik. 

Schon damals war klar, dass Begriffe der Ordnung geschaffen wurden, der Bedeutung des Ortes in der Zahlenbezeichnung. Der Begriff der Null war hier bereits auf versteckte Weise festgelegt worden, obwohl er erst in der Blütezeit der wissenschaftlichen Erkenntnis auftauchte -- nicht in der hellenischen Wissenschaft (Abschnitt 42) -- in Westeuropa, bekannt wurde er im Mittelalter, im XI-XII Jahrhundert, Jahrhunderte davor in Indien und Indochina und im Königreich der Inkas -- zumindest im Jahre 609 v. Chr., fast zweitausend Jahre vor seiner Entdeckung in Westeuropa. 61 

Das Bild wird jetzt allmählich genauer. 

Archäologische Funde deuten darauf hin, dass etwa 3000 v. Chr. in der vorarischen Moenjaro-Zivilisation im Indus-Becken, die mit Mesopotamien in Kontakt stand, Null- und Dezimalrechnung bekannt waren. Während der Hammurabi-Ära (2000 v. Chr.) in Babylon erreichte das algebraische Wissen einen Stand, der ohne die Annahme wissenschaftlichen theoretischen Denkens nicht erklärt werden kann. Offensichtlich dauerte es viele Jahrhunderte, wenn nicht Jahrtausende, um dies zu erreichen. 62 

Gleichzeitig deutet alles darauf hin, dass vor 6000-7000 Jahren die Migration -- die Bewegung der Menschen der damaligen sozialen Gebilde (und das damit verbundene Wissen -- die Navigation), ihre Mobilität größer war, als sie in späteren historischen Zeiten zu beobachten war. 63 Zu dieser Zeit konnte die Bevölkerung nicht groß gewesen sein. Kleine Gruppen von Menschen oder Familien könnten sich schnell bewegen. 

Die Domestizierung von Herdentieren und die Entdeckung von Möglichkeiten, sich auf dem Wasser fortzubewegen, könnten es uns ermöglichen, Merkmale dieser fernen Vergangenheit zu verstehen, wie z.B. die Einnahme aller Kontinente und die Überschneidung des Pazifischen und Atlantischen Ozeans durch die gleiche Art des Homo sapiens. Vielleicht eine andere, weniger wahrscheinliche Erklärung dafür, dass es unabhängige Manifestationszentren von Arten der gleichen Gattung Homo für Homo neandertalensis, Homo sapiens und andere gab, die sich im späteren Verlauf der Geschichte vermischten. 

41. Damals hatte die Biosphäre, die den Menschen umgibt, ein völlig anderes Aussehen, das unserer Vorstellung von ihr fremd war. Während dieser heroischen Periode der Entstehung der Noosphäre erlebte der Mensch große geologische Veränderungen. Die Schaffung von Kulturnatur, heimischen Pflanzen und Tieren hatte gerade erst begonnen -- oder war das Schicksal einiger Generationen. Der Mensch hat die Eiszeiten überlebt -- die Entstehung, das Einsetzen und den Rückzug des Eises, das weite Teile Eurasiens bedeckte, insbesondere in seinem westlichen Teil, den arktischen und antarktischen Ländern und Nordamerika. In diesem Zeitraum haben sich das Klima und die gesamte umgebende Natur unter dem Einfluss dieser Prozesse über mindestens eine Million Jahre dramatischer verändert als in unserer Zeit. Der Pegel des Weltozeans -- der Hydrosphäre -- hat erheblich geschwankt, in der Größenordnung des heutigen Tages. Gebiete subtropischer und tropischer Länder in unseren südlichen Breiten und in den nördlichen Breiten der Südhalbkugel haben Pluvialperioden durchlebt (darunter z.B. die Sahara). 64 

Sie wurden vom Menschen genauso erlebt, wie er die Eiszeit erlebt hat. Pluvialperioden, die mit Gletscherperioden synchron verlaufen und sich als Manifestation desselben Phänomens manifestieren, sind unseren Vorstellungen recht fremd, und das menschliche Gedächtnis hat sie längst vergessen. 

Wir kennen heute die Manifestationen der letzten Stadien der letzten Eiszeit in ihren Überresten -- in Grönland und im nördlichen Nordamerika -- in Kanada und Alaska, fast menschenleer, oder in der Antarktis, wo es nur vorübergehende Manifestationen einer Person gibt, die sie und ihre Inseln noch nicht bewohnt hat. 

Wir fangen, wie wir es von der vorhergehenden und den letzten Phasen der letzten Regenzeit hätten erwarten sollen. Wir sehen seine Überreste in tropischen und subtropischen Ländern, in den Feuchtwäldern des tropischen Afrikas, insbesondere der Lumme, und in den Wäldern Südamerikas. Das Amazonas- und Flachlandsystem Zentralafrikas gibt uns eine Vorstellung von dem einmal erwähnten Zustand der Biosphäre. Im Osten Chinas können wir in historischen Legenden und bei Ausgrabungen Anklänge an eine uns damals fremde Biosphäre studieren. 

Der Mensch überlebte den ersten Angriff der Gletscher, den Beginn der Eiszeit (im Pliozän). Vielleicht war es sein anderer Clan, der sozial lebte, nicht der Homo. Er überlebte auch die Entstehung von Feuchtwäldern und Sümpfen, die an die Stelle von Wäldern und Steppen traten, den vorherigen Zustand der Biosphäre -- das „Königreich der Säugetiere“, das Dutzende von Millionen von Jahren dauerte, in einer Umgebung, in der er sich ganz am Ende offenbarte. 

Während dieser kritischen Periode der Biosphäre -- dem beschleunigten Tempo des Wandels in ihrem Erscheinungsbild und dem Übergang zur Noosphäre -- musste sie einen erbitterten Existenzkampf führen. Die Biosphäre war von allen Säugetieren besetzt und umfasste alle Teile der Biosphäre, die für die Besiedlung durch den Menschen günstig waren und ihnen die Fortpflanzung ermöglichten. 

Der Mensch hat eine riesige Anzahl von Arten gefangen, die meisten von ihnen sind heute ausgestorben, große und kleine Säugetiere. Es schien eine wichtige Rolle bei ihrem raschen Aussterben gespielt zu haben, dank der Eröffnung des Feuers und der verbesserten sozialen Struktur. Die Säugetiere gaben ihm Grundnahrungsmittel, mit denen er sich schnell vermehren und große Gebiete erobern konnte. Der Beginn der Noosphäre steht im Zusammenhang mit diesem Kampf des Menschen mit Säugetieren um sein Territorium. 

42. Unser Wissen auf diesem Gebiet ändert sich jetzt rasch, da die alten Kulturen, die ohne Unterbrechung nicht nur in Europa, sondern auch in den indischen und chinesischen Menschheitskonglomeraten, auf dem amerikanischen und afrikanischen Kontinent existierten, erst in ihren materiellen Denkmälern entdeckt werden. 

Man kann sagen, dass uns die alten Kulturdenkmäler Indiens, die dieses große Kulturzentrum mit Chaldäa 4.000 Jahre vor uns verbinden, historisch gesehen erst vor kurzem enthüllt wurden, und fast gleichzeitig beginnen wir, in die Vergangenheit der chinesischen Kulturen einzudringen65 (Abschnitt 43). Sie brachten viele Überraschungen mit sich und wiesen vor allem auf die Verbindung (zumindest in Indien -- in seinem Westen, im Indusbecken) mit Chaldäa (dem Mittelmeerzentrum) und auf das hohe Niveau der lokalen jahrhundertealten (viele tausend Jahre?) einheimischen Kreativität hier hin. 

In einigen Jahren werden sich unsere Wahrnehmungen radikal ändern, denn es ist klar, dass die antiken Zivilisationen Chinas und Indiens, die jetzt entdeckt werden, Tausende von Jahren existiert haben, bis sie das Niveau der durch Entdeckung entdeckten Kultur erreichten. Diese Kulturen sind eindeutig nicht die ältesten. 

Vor dem Hintergrund dieser alten Kulturen, in abgelegenen Zentren -- im Mittelmeerraum, in Mesopotamien, in Nordindien, in Süd- und Mittelchina, in Süd- und Mittelamerika, wahrscheinlich anderswo -- fand die Entstehung des geologischen Werkes wissenschaftlichen Denkens spontan statt, d.h. mit der Kraft und Natur des natürlichen Prozesses der Biosphäre. 

Sie zeigte sich in der Schaffung der Grundbestimmungen -- Verallgemeinerungen der Wissenschaft, theoretisches wissenschaftliches Denken -- in der Arbeit zur Klärung der theoretisch abstrakten Bestimmungen der wissenschaftlichen Erkenntnis als Ziel der Arbeit der Menschheit -- der Suche nach wissenschaftlicher Wahrheit um ihrer selbst willen, zusammen mit dem philosophischen und religiösen Verständnis der Welt um sie herum, seit Jahrtausenden zuvor. 

Mit einer gewissen Fehlerquote, die kaum sehr groß ist, können wir jetzt den Zeitpunkt bestimmen, zu dem es an verschiedenen Orten, scheinbar unabhängig voneinander, zu verschiedenen Zeiten geschah. Dies ist die Entstehungszeit der griechischen Wissenschaft und Philosophie der VII-VI Jahrhunderte v. Chr., der religiösen, philosophischen und wissenschaftlichen Interpretationen in Indien und China der VIII-VII Jahrhunderte. Es ist möglich, dass weitere Entdeckungen unser Verständnis der vorhellenischen Wissenschaft verändern werden, und die Ausgewogenheit der bekannten Wissenschaft davor wird viel größer sein, als wir uns jetzt vorstellen (Abschnitt 45). Die neuen Werke erhöhen alle den Bestand an wissenschaftlichem Wissen, das der Menschheit vor der hellenistischen Wissenschaft bekannt war,66 und bestätigen die Authentizität der hellenistischen Wissenschaftstraditionen über die Bedeutung, die die altägyptische und altchaldäische Wissenschaft für sie hatte. Die ägyptische Wissenschaft wurde von den Griechen in einer Periode der Stagnation erfasst, während die chaldäische Wissenschaft in der lebendigen Kunst gefangen war. Die gemeinsame Arbeit der helleno-chaldäischen Wissenschaftler vor mehr als 2200 Jahren wird in der Wissenschaftsgeschichte immer noch nicht berücksichtigt. Sie war ein Nebenprodukt der gewaltsamen Zerstörung der persischen Monarchie durch die makedonischen Könige, hauptsächlich Alexander, der die hellenistische Kultur übernahm. 

Der Anteil der chaldäischen Wissenschaft wird in der griechischen Wissenschaft wahrscheinlich viel größer sein, als wir denken. 

Nun hat sich uns ganz unerwartet die Tiefe der chaldäischen Wissenschaft der Algebra eröffnet. Diese Werke mögen nur wenige Jahrhunderte, nachdem die unabhängige Arbeit der chaldäischen Wissenschaftler aufhörte oder in das hellenische wissenschaftliche Denken eintrat, durch Hipparchus und Diophantos in unseren wissenschaftlichen -- hellenischen -- Apparat eingeflossen sein (Abschnitt 45). 

Die Chaldäer hatten das Verständnis von Null, als die Griechen kaum ein Alphabet hatten (Abschnitt 40). Aber das Konzept der Null hat das neugierige Denken der Griechen nicht erfasst, und im Westen Europas trat das Leben im Mittelalter durch die Araber und Hindus ein, und die Algebra machte fast ein halbes Jahrtausend später durch Diophantos (über dessen Leben wir nichts wissen) auf sich aufmerksam. 

Es gibt eine Reihe von Annahmen, Vermutungen, wie könnte dies geschehen? Es scheint mir sehr wahrscheinlich, dass es mit der Unvollständigkeit und Zufälligkeit der griechischen mathematischen Literatur zusammenhängt, die uns überlebt hat (III. Jahrhundert v. Chr. -- III. Jahrhundert nach Christus). 

Eine wichtige Tatsache kann nur im Zusammenhang mit dieser höheren Gewalt stehen, und wenn ja, ist sie nicht von Bedeutung. 

Kaum jedoch werden die Änderungen derart sein, dass sie uns zwingen würden, moderne Konzepte inhaltlich zu verändern. 

Es ist möglich, dass sich im Mittelmeerraum, in Indien und China unabhängig voneinander das Bewusstsein für die Notwendigkeit herausgebildet hat, nach einem wissenschaftlichen Verständnis der Umgebung als Sonderfall des Lebens eines denkenden Menschen zu suchen. Die Schicksale dieser Anfänge waren unterschiedlich. 

Die hellenische Wissenschaft hat sich zu einem einzigen modernen wissenschaftlichen Gedanken der Menschheit entwickelt. Sie durchlief Zeiten der Stagnation, entwickelte sich aber am Ende zur Weltwissenschaft des XX. Jahrhunderts -- zum Universum der Wissenschaft. Perioden der Stagnation erreichten die Länge vieler Generationen -- ein großer Verlust dessen, was vorher bekannt war. Die maximalen Unterbrechungen erreichten 500-1000 Jahre, aber dennoch wurde die Tradition nicht völlig unterbrochen (Abschnitt 45). 

43. Für den Bereich der chinesischen Kulturen können wir noch nicht mit Sicherheit den Stand der wissenschaftlichen Erkenntnisse bestätigen, der es uns erlauben würde, über die Entstehung eines wissenschaftlichen Denkens in Ostasien zu sprechen, das sich von philosophischem und religiösem Denken unterscheidet und unabhängig vom griechischen wissenschaftlichen Forschungszentrum ist. Aber die Geschichte der chinesischen kulturellen Manifestation in ihrer Chronologie ist immer noch so schlecht verstanden, dass wir sie heute nicht leugnen können. Wir müssen die weitere Klärung der Ergebnisse der historischen Arbeit, die jetzt in diesem Bereich im Gange ist, abwarten. 

Tatsächlich gaben uns erstmals nur Funde aus staatlichen Ausgrabungen in den Jahren 1934-1935 eine klare Vorstellung von der Geschichte des alten China. Und hier war die historische Legende, die uns überlebt hat, zuverlässiger als wir dachten. 

Diese Kultur ist neuer als die ägyptische und die chaldäische Kultur, von denen ein Teil älter ist als die griechische. Es scheint ein unabhängiges Zentrum für die Geburt wissenschaftlicher Erkenntnisse zu sein. In den kommenden Jahren, wenn China die Schrecken der japanischen Invasion überwunden hat, werden wir in der Lage sein, uns ein klareres Bild zu machen. Wir können es jetzt nicht geben. 

44. Die Elemente für organisiertes wissenschaftliches Denken und eine Reihe von Kenntnissen, die dessen Aufbau ermöglichen würden, existieren seit langem unbewusst, nicht zum Zweck der Erkenntnis anderer, und wurden vor Tausenden von Jahren mit dem Entstehen großer menschlicher Staaten und Gesellschaften geschaffen. Aber lange Zeit gab es in ihnen keinen kühnen und kühnen Gedanken -- die revolutionäre Kühnheit des Einzelnen -- sie hinterließ keine bleibenden Spuren, keine Überzeugung von der Richtigkeit wissenschaftlich belegter Tatsachen und auf dieser Grundlage eine kühne kritische Haltung gegenüber den vorherrschenden religiösen, philosophischen oder alltäglichen Behauptungen. Die wissenschaftliche Erklärung der Natur trat nicht ins Leben, das Motiv des Persönlichkeitsverhaltens. Es gab keine erfolgreichen Versuche, sich dem Einfluss religiöser Ideen zu entziehen und nach Kriterien für das Erlernen der Richtigkeit religiöser und alltäglicher Überzeugungen zu suchen. 

Das Kriterium -- organisiertes wissenschaftliches Denken -- wurde durch abstrakte Arbeit von Individuen geschaffen -- in der Analyse, in der Reflexion über die Richtigkeit logischer Aussagen -- (in der Schaffung von Logik) -- in der Suche nach grundlegenden verallgemeinernden Ideen, in wissenschaftlich beobachteten Fakten, in der Schaffung der Mathematik, in der Schaffung des Apparates wissenschaftlicher Fakten -- der Grundlage ihrer natürlichen Systematik, der empirischen Verallgemeinerung von Fakten. 

Dies konnte nur geschehen, wenn ein Mensch in der Lage war, seinen Willen in der Gesellschaft zu zeigen, sie in einem Umfeld frei zu halten, das von der unvermeidlichen Routine tausender Generationen durchdrungen war. Wissenschaft und wissenschaftliche Organisationen wurden geschaffen, als der Einzelne begann, auf der Grundlage des ihn umgebenden Wissens kritisch zu reflektieren und nach seinen Wahrheitskriterien zu suchen. 

Wir können über die Wissenschaft, das wissenschaftliche Denken, ihr Erscheinen in der Menschheit sprechen -- nur wenn ein einzelner Mensch selbst begann, über die Genauigkeit der Erkenntnisse nachzudenken und begann, die wissenschaftliche Wahrheit nach der Wahrheit zu suchen, als eine Angelegenheit seines Lebens, als die wissenschaftliche Suche ein Selbstzweck war. 

Die Hauptsache war die genaue Feststellung des Sachverhalts und seine Überprüfung, die wahrscheinlich aus der technischen Arbeit heraus und durch die Bedürfnisse des täglichen Lebens entstanden ist. 

Die Erstellung genauer Beobachtungen, die im täglichen Leben benötigt werden, und die astronomische Überprüfung dieser Beobachtungen durch Generationen, die mit den schließlich verstorbenen illusorischen religiösen Überzeugungen in Verbindung gebracht werden, ist eine der ältesten Formen wissenschaftlicher Arbeit. Sie ist wissenschaftlicher Natur, aber der Wissenschaft in ihren Motiven fremd. 

Gleichzeitig ging die Klärung der Tatsachenfeststellung mit einer Reflexion und Verallgemeinerung einher, die zu Logik und Mathematik führte, und hier standen die sozialen Bedürfnisse an erster Stelle. 

Wie jedoch bereits angedeutet (Abschnitt 40), führten sie in der Mathematik vor 4000-2000 Jahren zur Bildung einer Zahl aus dem Dezimalsystem, den ersten grundlegenden Theoremen der Geometrie, den ersten „Symbolen“ (algebraisch). Seit dem XVI. bis XVII. Jahrhundert hat die neue Mathematik -- im Symbol und in der Analyse, in der Geometrie -- das menschliche Denken und Arbeiten umarmt und ihm eine entscheidende Rolle bei der Umarmung der Natur zugewiesen. 

Die Arbeit des logischen Denkens ging noch tiefer. Seine Chronologie -- hauptsächlich im Bereich der indianischen Kulturen -- ist noch nicht festgelegt worden. Dank der ununterbrochenen Arbeit vieler Generationen von Denkern, die den mächtigen Strom von „Jüngern“ verursachten -- viele Tausende von Menschen über viele Generationen hinweg, begann nicht weniger als 3000 Jahre vor Christus in verschiedenen Teilen der Staatsformationen der arischen Bevölkerung Indiens -- die Ausländer auf dem Gebiet der alten vorarischen Kulturen der „dravidischen“ kulturellen Formationen, der mächtige philosophische religiöse Strom, der die Grundlage für große logische Konstruktionen schuf, lebt bis heute. Mit langen Perioden von Unterbrechungen im kreativen Denken -- aufgrund der Tragödien der Geschichte -- schuf das indische logische Denken selbst ein kohärentes System, Jahrhunderte bevor es in der griechischen Zivilisation entdeckt wurde. Sein wirklicher Einfluss auf Aristoteles' Logik, die bis ins 18. bis 19. Jahrhundert die einzige in unserer Wissenschaft dominierende war, ist akzeptabel. 

Das indische logisch-philosophische Denken hatte einen enormen Einfluss auf die Zivilisationen des asiatischen Kontinents, wo es zeitweise innerhalb weniger Generationen eine unabhängige wissenschaftliche Arbeit zur Schaffung neuer wissenschaftlicher Fakten und empirischer Verallgemeinerungen gab. Dieser Einfluss erstreckte sich auf Japan, Korea, tibetische, chinesische und indochinesische Staaten, während im Westen das Gebiet der hellenistischen und muslimischen Kulturzentren -- im Süden und Südosten -- auf die dravidischen Staatsgebilde Ceylon und Malaysia übertragen wurde. In Indien wurde die Tradition des logischen Denkens selbst nicht unterbrochen und im XIX. Jahrhundert unter dem Einfluss der westeuropäischen, vereinigten, modernen, wissenschaftlichen Kultur kraftvoll und tiefgreifend erneuert. Und die wissenschaftliche und philosophische Arbeit hat ein äußerst günstiges Umfeld für kontinuierliche Generationen gefunden, die an geistige Arbeit gewöhnt sind. 

45. Im Mittelmeerraum wuchs aus diesen Jahrhunderten wachsender Suche von Generationen freidenkender Persönlichkeiten hellenisches wissenschaftliches Denken heran, das unter Nutzung der wissenschaftlichen Erfahrung vieler Jahrtausende der Geschichte Kretas, Chaldäas, Ägyptens, kleiner asiatischer Staatsformationen und vielleicht des indischen Kulturzentrums innerhalb von ein oder zwei Generationen in den VI-VII Jahrhunderten vorgebracht wurde. -- die Menschen, die das Fundament der griechischen Wissenschaft gelegt haben. Mit diesem Beginn sind wir in der Konstruktion der Wissenschaft genetisch kontinuierlich verbunden. 

Offenbar gab es sowohl in Chaldäa als auch in Ägypten Perioden des Niedergangs und Stillstands in der Geschichte der Menschheit. In einer dieser Perioden standen die Griechen der Wissenschaft Kleinasiens und Ägyptens gegenüber. 

Wir können diese Perioden der Blütezeit und des Niedergangs des hellenischen wissenschaftlichen Denkens, ihre Geschichte, noch nicht wiederherstellen. Kaum die Blütezeit der vorhellenischen Wissenschaft, deren Wesen uns noch immer nicht hinreichend klar ist, war jemals so mächtig wie die Phänomene, die die Küste Kleinasiens (Milet), Süditaliens und Griechenlands im 6. und 4. -- die Ära der hellenischen Wissenschaft. 

Die hellenistische Wissenschaft behielt ihre Position fast ein Jahrtausend lang, etwa bis in die Jahrhunderte III-IV. v. Chr. Aufhören und abschwächen, schließlich kam es in diesen Jahrhunderten zum Niedergang der wissenschaftlichen Arbeit, der nur zum Teil mit dem Staatszerfall und der politischen Schwächung des Römischen Reiches zusammenhing -- er ist verbunden mit einem tiefgreifenden Wandel in der geistigen Stimmung der Menschheit, seiner Abkehr von der Wissenschaft, der Verringerung der schöpferischen wissenschaftlichen Arbeit und der Umsetzung des schöpferischen Denkens auf dem Gebiet der Philosophie und Religion, in künstlerische Bilder und Formen. 

46. Damals jedoch gab es in den nichtchristlichen Staatsformationen -- Perser, Araber, Inder, Chinesen -- eine unabhängige wissenschaftliche Arbeit, die das wissenschaftliche Niveau nicht sinken ließ, und schließlich wurde in den Ländern des Weströmischen Reiches auf dem Gebiet der internationalen lateinischen Sprache und Kultur unter ihrem Einfluss das wissenschaftliche Denken wiederbelebt, und fast ein Jahrtausend später -- im XIII. Jahrhundert -- kam es zu einem klaren Durchbruch, der im XVI-XVII. Ermöglicht wurde dies durch die Stärkung staatlicher Lebensformen, das Wachstum der Technik in Verbindung mit neuen Lebens- und Staatsbedürfnissen und -- nach blutigen, religiös bedingten Hekatomben über mehrere Generationen hinweg -- nach der Schwächung, die schließlich in bedeutenden und einflussreichen Gruppen und Klassen der Bevölkerung zu einer tiefen Untergrabung der moralischen Wirkkraft des Christentums bzw. des Islam und des Judentums führte. Ein Durchbruch im religiösen Bewusstsein des Westens, der möglicherweise das religiöse Leben der Menschheit in der Realität vertieft und in einer tiefen Krise, aus der die religiöse Kreativität vielleicht bereits hervorgeht, einen realeren Rahmen für ihre Manifestation in den menschlichen Gesellschaften geschaffen hat. Das religiöse Bewusstsein der Menschheit hat die Notwendigkeit einer neuen religiösen Synthese offenbart, die noch immer nach neuen Formen in den neuen Lebensbedingungen sucht. 

Jahrhundert erleben wir eine neue scharfe Wende im wissenschaftlichen Bewusstsein der Menschheit, ich glaube, die größte, die die Menschheit je in ihrem Gedächtnis erlebt hat, ein wenig ähnlich wie in der Ära der hellenischen Wissenschaft, aber mächtiger und breiter in ihrer Ausprägung, universeller. Anstelle von Zehn- und Hunderttausenden von Menschen, die an den Küsten des Schwarzen Meeres und des Mittelmeers verstreut sind und weniger mit ihnen verbunden sind, vor allem hellenische, städtische Kulturzentren -- wissenschaftliches Verständnis und daher wissenschaftliche Suche -- sind jetzt Zehn-, Hundert Millionen Menschen auf dem ganzen Planeten erfasst, wir können sagen, die gesamte menschliche Bevölkerung. 

Wir leben ohnehin in der Ära des größten Bruchs. Das philosophische Denken hat sich als machtlos erwiesen, die geistige Einheit, die die Menschheit zusammenhält, zu kompensieren. Die spirituelle Einheit der Religion erwies sich als Utopie; der religiöse Glaube wollte sie durch physische Gewalt schaffen -- ohne sich vor Morden in Form von blutigen Kriegen und Massenhinrichtungen zurückzuziehen. Religiöses Denken brach in viele Strömungen auseinander. Die Staatsidee, diese vitale Einheit der Menschheit in Form einer einzigen staatlichen Organisation zu schaffen, erwies sich als machtlos. Wir stehen jetzt vor zahlreichen staatlichen Organisationen, die am Vorabend eines neuen Massakers zur gegenseitigen Vernichtung bereit sind. 

Und genau zu dieser Zeit, zu Beginn des 20. Jahrhunderts, erschien die Kraft, die die Einheit der Menschheit schaffen kann -- ein wissenschaftlicher Gedanke, der eine beispiellose Explosion der Kreativität erlebt hat -- in einer klaren realen Form. 

Es ist eine geologische Kraft, die durch Milliarden von Jahren Leben in der Biosphäre vorbereitet wurde. 

Sie offenbarte sich zum ersten Mal in der Geschichte der Menschheit in einer neuen Form, einerseits in der Form der logischen Verpflichtung und der logischen Kontinuität ihrer wichtigsten Errungenschaften und andererseits in der Form eines Universums -- ihrer Abdeckung der gesamten Biosphäre, der gesamten Menschheit -- in der Schaffung einer neuen Stufe ihrer Organisation -- der Noosphäre. Wissenschaftliches Denken wird zum ersten Mal als eine Kraft identifiziert, die die Noosphäre erschafft, mit der Natur eines natürlichen Prozesses. 

       Kapitel 3 



Die Bewegung des wissenschaftlichen Denkens im zwanzigsten Jahrhundert und seine Bedeutung in der geologischen Geschichte der Biosphäre. Ihre Hauptmerkmale sind die Explosion der wissenschaftlichen Kreativität, der Wandel im Verständnis der Grundlagen der Wirklichkeit, des Universums und die effektive, soziale Manifestation der Wissenschaft. 





47. Was sich heute in der Wissenschaftsbewegung abspielt, kann nur aus der Vergangenheit der Wissenschaft mit jener Wissenschaftsbewegung verglichen werden, die mit der Entstehung der griechischen Philosophie und Wissenschaft im VI-V Jahrhundert v. Chr. in Verbindung gebracht wird. 

Leider können wir uns die Summe der wissenschaftlichen Erkenntnisse, die den antiken Hellenen zuteil wurde, noch nicht klar vorstellen, als sich in ihrer Umgebung das wissenschaftliche Denken offenbarte und als es erstmals die wissenschaftliche und philosophische Struktur außerhalb der religiösen, kosmogonischen und poetischen Konstruktionen annahm -- als die erste hellenische städtische Zivilisationspolisa eine wissenschaftliche Methode -- Logik und theoretische Mathematik in der Anwendung auf das Leben schuf, und als es zu einer echten Suche nach wissenschaftlicher Wahrheit wurde, als dem eigentlichen Zweck des individuellen Lebens im sozialen Umfeld. 

Die Umstände dieses, wie die Geschichte zeigt, größten Ereignisses im Leben der Menschheit und in der Entwicklung der Biosphäre, sind in vielerlei Hinsicht rätselhaft und langsam, doch die Geschichte der wissenschaftlichen Erkenntnisse offenbart sie immer tiefer. Die Summe der wissenschaftlichen Kenntnisse über das damalige hellenische Umfeld, die Errungenschaften der hellenischen Wissenschaftsschöpfer, die zu dieser Zeit lebten, und das, was sie von den früheren Generationen der hellenischen Zivilisation erhielten, ist nur in den ersten Umrissen deutlich zu erkennen. Wir beginnen langsam, dies zu verstehen. Auf der einen Seite. 

Auf der anderen Seite hat sich die Vorstellung, die die Hellenen von der Wissenschaft der großen Zivilisationen, die ihnen vorausgegangen sind -- Kleinasien, Kreta, Chaldäer (Mesopotamien), Altägypten, Indien -, erhalten haben, stark verändert. 

Leider hat uns nur ein winziger Teil der hellenischen wissenschaftlichen Literatur erreicht. Die größten Forscher haben keine Spuren der verfügbaren Literatur in uns hinterlassen oder nur Fragmente ihrer wissenschaftlichen Arbeit bei uns erreicht. 

Es stimmt, dass wir die gesamte Mehrheit der Werke Platons und einen bedeutenden Teil der wissenschaftlichen Werke von Aristoteles erreicht haben, aber für letztere gingen viele, die wichtigsten in Bezug auf die wissenschaftliche Suche, verloren. Besonders traurig ist unter diesem Gesichtspunkt der Verlust der Werke bedeutender Wissenschaftler, deren Werke wissenschaftliches Denken und wissenschaftliche Methoden in der Epoche der Blütezeit und Synthese der hellenistischen Wissenschaft waren -- Alkmeon (500 Jahre v. Chr.).Hippokrates von Chios (450-430 Jahre v. Chr.), Philolaus (5. Jahrhundert v. Chr.) und viele andere, aus denen es winzige Auszüge oder Namen gibt. 

Noch trauriger ist der Verlust der ersten Versuche zur Geschichte der wissenschaftlichen Arbeit und des wissenschaftlichen Denkens, die in den Jahrhunderten geschrieben wurden, die den Jahrhunderten ihrer Entdeckung am nächsten liegen. Teilweise verzerrt und unvollständig, kam dieses Werk als namenlose Grundlage zu uns, manchmal gemeistert und im Laufe der Jahrhunderte nach seiner Veröffentlichung verändert. Aber die Originale der Geschichte der Geometrie des Xenokrates (397-314), die Geschichte der Wissenschaft des Ödems von Rhodos (etwa 320), die Geschichtsbücher des Theophrast (372-288) und andere sind im historischen Verlauf der hellenistisch-römischen Zivilisation bis zum Zeitpunkt unserer Zeit verschwunden -- in den ihr am nächsten liegenden Jahrhunderten, vor fast tausend Jahren. 

Tatsächlich hat uns der Grundbestand der hellenistischen Wissenschaft -- das, was ich den wissenschaftlichen Apparat nenne,67 in winzigen Fragmenten erreicht, und zwar nach vielen Jahrhunderten in den Resten der naturhistorischen Werke von Aristoteles und Theophrast und in den Werken griechischer Mathematiker. Und doch hatte er einen großen Einfluss auf die Wiederbelebung -- die Schaffung der westeuropäischen Wissenschaft in den XV-XVII Jahrhunderten. Unsere neue Wissenschaft entstand weitgehend auf der Grundlage ihrer Errungenschaften und entwickelte die darin zum Ausdruck kommenden Ideen und Erkenntnisse. Unterbrochen durch Jahrhunderte, damals im Römischen Reich, wurden die Fäden im XVII Jahrhundert wiederhergestellt. 

48. In jüngster Zeit hat uns der Verlauf der Wissenschaftsgeschichte dazu veranlasst, unser Verständnis des vorhellenischen Erbes, auf dem die hellenische Wissenschaft gewachsen ist, zu ändern, wie ich bereits erwähnt habe (Abschnitt 42). 

Überall wiesen Hellenen auf das große Wissen hin, das sie aus Ägypten, von Chaldäern, aus dem Osten erhalten hatten. Das müssen wir jetzt als richtig akzeptieren. Vor ihnen existierte die Wissenschaft bereits -- die Wissenschaft der „Chaldäer“, die Jahrtausende vor R.H. verließ, erst jetzt vor uns geöffnet wird -- in Fetzen, die mit einer unbestreitbaren Authentizität ihre lange Zeit bis zu unserer Zeitkraft nicht verdächtig bewiesen haben (Sec. 42). 

Es ist jetzt klar, dass wir den zahlreichen Anweisungen der antiken Wissenschaftler und Schriftsteller, die die Schöpfer der hellenischen Wissenschaft und Philosophie berücksichtigt haben und die in ihrer schöpferischen Arbeit von den Errungenschaften der Wissenschaftler und Denker Ägyptens, der Chaldäer, der arischen und nicht-arischen Zivilisationen des Ostens ausgingen, viel mehr reale Bedeutung geben müssen, als wir es in letzter Zeit getan haben. 

Mehrere Jahrhunderte lang arbeiteten babylonische Wissenschaftler mit hellenischen Wissenschaftlern zusammen. Zu dieser Zeit -- in den nächsten Jahrhunderten bis zu unserer Zeitrechnung -- befand sich die babylonische Astronomie in einer neuen Blütezeit. Nach und nach, über mehrere Generationen hinweg, verschmolzen sie mit dem hellenischen Umfeld und waren von dem ungünstigen wissenschaftlichen Umfeld der damaligen Zeit gleichermaßen betroffen (Abschnitt 40). 

Zweifellos wurden die von Wissenschaftlern jener Zeit erhaltenen Erkenntnisse von Hellenes in dieser Mitteilung genutzt. 

Zweifellos hatten sie bis zu diesem Zeitpunkt viel Geld eingesetzt und verbraucht -- vor allem, wenn man die jahrtausendealte Erfahrung und die jahrtausendealte Tradition der Schifffahrt, der Technik, der Landwirtschaft, der Bewässerung, der militärischen Angelegenheiten, des Staatswesens und des Alltags berücksichtigt. 

Jahrhundertelang hatte die griechische Wissenschaft in direktem Kontakt mit der chaldäischen und der ägyptischen Wissenschaft gearbeitet und mit ihnen fusioniert. Obwohl es möglich ist, dass der kreative Gedanke in der ägyptischen Wissenschaft zu dieser Zeit erstarrte -- für die chaldäische Wissenschaft war das nicht der Fall (Abschnitt 42). 

Die hellenische Wissenschaft in ihren Anfängen war direkt eine Fortsetzung des verstärkten kreativen Denkens der vorhellenischen Wissenschaft. Die Tatsache wird zwar festgestellt, aber von der Wissenschaftsgeschichte noch nicht gemeistert. 

Das „Wunder“ der hellenischen Zivilisation -- ein historischer Prozess, dessen Ergebnisse klar sind, dessen Verlauf aber nicht genau nachvollzogen werden kann -- war derselbe historische Prozess wie andere. Sie hatte in der Vergangenheit ein solides Fundament. Nur das Ergebnis seiner Auswirkungen -- das Tempo, mit dem es erreicht wurde -- war zeitlich einmalig und in seinen Folgen in der Noosphäre außergewöhnlich. 

49. Der Verlauf des wissenschaftlichen Denkens unserer Zeit, des zwanzigsten Jahrhunderts, mag -- nach dem wahrscheinlichen Ergebnis -- zu noch grandioseren Folgen führen, aber in seinem Verlauf unterscheidet er sich deutlich und dramatisch von dem, was in einem kleinen Gebiet des Mittelmeers geschah -- der Küste Kleinasiens, den Inseln und Halbinseln Griechenlands, Sizilien, Süditalien und einzelne Städte des Mittelmeers, der Ägäis, des Schwarzen und des Asowschen Meeres, wo die hellenische Kultur eindrang, und zu dieser Zeit konzentrierte sich das wissenschaftliche Schaffen vor allem in Kleinasien, Mesopotamien und Süditalien, dann griechisch in Kultur und Sprache.

Jahrhunderts und der Bewegung, die die hellenistische Wissenschaft, ihre wissenschaftliche Organisation, geschaffen hat, besteht erstens in ihrem Tempo, zweitens in dem Gebiet, das sie erobert hat -- sie erstreckte sich über den gesamten Planeten -, zweitens in der Tiefe der Veränderungen, die sie bewirkte, in den Vorstellungen von der wissenschaftlich verfügbaren Wirklichkeit und schließlich in der Kraft der Veränderung der Wissenschaft des Planeten und der dadurch eröffneten Wege der Zukunft. 

Diese Unterschiede sind so groß, dass man eine wissenschaftliche Bewegung voraussehen kann, deren Ausmaß in der Biosphäre noch nicht existiert. 

Diese Bewegung rechtfertigt die geologische Grenze, die Ch. Shukhert und A. Pavlov kürzlich in der Erdgeschichte mit dem Aufkommen der menschlichen Vernunft festgestellt haben. Die Noosphäre wird in der nächsten, historischen Zeit noch schärfer erscheinen. 

50. Wir können hier -- ein seltener Fall in der Geschichte des Wissens -- den Beginn der modernen Wissenschaftsbewegung so genau und scharf feiern, wie es nicht möglich war, uns in die Vergangenheit zurückzuführen. 

Offenbar konnten die alten Hellenen selbst dies zu einem Zeitpunkt getan haben, als in den V-IV-Jahrhunderten v. Chr. die Wissensgeschichte, die in den ersten Jahrhunderten n. Chr. zum Teil in den Händen von Forschern lag, in den Originalen geschrieben wurde und im Allgemeinen verloren ging. 

Wir können daher unsere Epoche, für die uns alle Dokumente vorliegen, nicht genau mit dieser kritischen Epoche in der Geschichte des wissenschaftlichen Denkens vergleichen. Unsere Epoche lässt sich bis zum Ende des 19. Jahrhunderts datieren, auf die Jahre 1895-1897, als die Phänomene im Zusammenhang mit dem Atom mit seiner Zerbrechlichkeit entdeckt wurden (Abschnitt 55). 

Sie manifestiert sich in einer enormen Anhäufung neuer wissenschaftlicher Fakten, die man mit einer Explosion in ihrem Tempo gleichsetzen kann. Rasch entstehen neue wissenschaftliche Wissensgebiete, es entstehen auch zahlreiche neue Wissenschaften, das wissenschaftliche empirische Material wächst, und eine wachsende Zahl von Fakten, gezählt in Millionen, wenn nicht Milliarden, werden systematisiert und im wissenschaftlichen Apparat berücksichtigt. Sie werden immer systematisierter und einfacher zu verstehen; dies ist die so genannte Spezialisierung der Wissenschaft -- eine außerordentliche Vereinfachung der Fähigkeit, Milliarden von Fakten im wissenschaftlichen Apparat zu verstehen. Ich nenne den wissenschaftlichen Apparat einen Komplex von quantifizierten oder qualitativ genauen Naturkörpern oder Naturphänomenen. Sie wurde im 18. und hauptsächlich im 19. und 20. Jahrhundert geschaffen und bildet die Grundlage all unserer wissenschaftlichen Erkenntnisse. Es wurde nach dem definitiv festgelegten, jahrhundertealten, allesamt wissenschaftlich vertieften Werk systematisiert -- es wird in jeder Generation kritisch überprüft und spezifiziert. Der wissenschaftliche Apparat von einer Milliarde Milliarden wachsender Fakten, die allmählich und kontinuierlich durch empirische Verallgemeinerungen, wissenschaftliche Theorien und Hypothesen abgedeckt werden, ist die Grundlage und die Hauptkraft, das Hauptinstrument für das Wachstum des modernen wissenschaftlichen Denkens. Dies ist eine beispiellose Schöpfung neuer Wissenschaft. 

Wir haben oft eine negative Einstellung zur Spezialisierung, aber in Wirklichkeit erweitert die Spezialisierung in Bezug auf ein Individuum die Möglichkeiten seines Wissens extrem, indem sie das ihm zur Verfügung stehende wissenschaftliche Feld erweitert. 

Jahrhunderts die Grenzen zwischen den einzelnen Wissenschaften rasch verwischen. Wir spezialisieren uns zunehmend nicht auf Wissenschaften, sondern auf Probleme. Dadurch ist es einerseits möglich, sehr tief in das untersuchte Phänomen einzudringen und andererseits seine Reichweite unter allen Gesichtspunkten zu erweitern. 

51. Aber eine noch dramatischere Veränderung vollzieht sich jetzt bei den

grundlegenden Methoden der Wissenschaft. Hier haben die Folgen der neu

entdeckten Bereiche der wissenschaftlichen Fakten gleichzeitig eine

Veränderung der Grundlagen unserer wissenschaftlichen Kenntnisse, des

Verständnisses der Umwelt, eines Teils der Jahrtausende, die unberührt

blieben, und eines Teils selbst der allerersten, die erst in unserer Zeit ganz

unerwartet enthüllt wurden, bewirkt.







Eine solche völlig unerwartete und neue Hauptfolge neuer Bereiche wissenschaftlicher Tatsachen ist die Heterogenität des Kosmos, die sich vor uns offenbart hat, die Realität und die Heterogenität unseres Wissens darüber, die ihm entspricht. Die Heterogenität der Realität entspricht der Heterogenität der wissenschaftlichen Methodik, Einheiten und Standards, mit denen sich die Wissenschaft befasst.



Wir müssen nun zwischen drei Realitäten unterscheiden: 1) Realität auf dem Gebiet des menschlichen Lebens, die natürlichen Phänomene der Noosphäre und unseres Planeten als Ganzes; 2) die mikroskopische Realität atomarer Phänomene , die sowohl das mikroskopische Leben als auch das Leben von Organismen erfasst, selbst durch Instrumente, die für das bewaffnete Auge eines Menschen nicht sichtbar sind, und 3) die Realität kosmischer offener Räume, in denen das Sonnensystem und sogar die Galaxie verloren gehen und im Bereich des noosphärischen Teils der Welt nicht wahrnehmbar sind. Dies ist ein Bereich, der teilweise von der Relativitätstheorie abgedeckt wird, die uns als Ergebnis ihrer Entstehung offenbart wurde. Die wissenschaftliche Bedeutung der Relativitätstheorie basiert für uns nicht auf ihr selbst, sondern auf dem neuen experimentellen und beobachtenden Material, das mit neuen Entdeckungen der Sternastronomie verbunden ist. 68



Die Relativitätstheorie ist geprägt von Extrapolationen und Vereinfachungen der Realität, Annahmen, deren Überprüfung durch wissenschaftliche Erfahrung und wissenschaftliche Beobachtung auf der Grundlage der Noosphäre zumindest jetzt nicht mehr möglich ist. Aus diesem Grund ist es in der aktuellen wissenschaftlichen Arbeit, die einen unbedeutenden Platz einnimmt, viel mehr an dem Philosophen interessiert als an dem Naturforscher, der dies nur in den Fällen berücksichtigt, in denen er sich der kosmischen Realität nähert. In der Biosphäre kann er nicht berücksichtigt werden, ihre Manifestationen werden nicht wissenschaftlich beobachtet.



Es wird jetzt klar, dass wir hier wie auf dem Gebiet der Atomwissenschaften wissenschaftlichen Phänomenen ausgesetzt sind, die zuerst vom menschlichen Denken abgedeckt werden und im Wesentlichen zu anderen Bereichen der Realität gehören als dem, in dem das menschliche Leben stattfindet und ein wissenschaftlicher Apparat geschaffen wird.



Denn der Bereich der menschlichen Kultur und die Manifestation des menschlichen Denkens -- die gesamte Noosphäre -- liegt außerhalb der kosmischen Weiten, wo sie als infinitesimal verloren geht, und außerhalb der Region, in der die Kräfte von Atomen und Atomkernen mit der Welt ihrer Teilchen regieren, wo sie als unendlich groß fehlt.



Diese beiden neuen Wissensbereiche -- Raumzeit ist extrem klein und Raumzeit sind unbegrenzt groß -- sind so neu und im Wesentlichen das Grundprinzip, das durch das wissenschaftliche Denken des 20. Jahrhunderts eingeführt wurde. in die Geschichte und in den Gedanken der Menschheit.



Zu dem zuvor bekannten Bereich des menschlichen Lebens (der Noosphäre), in dem sich die Wissenschaft noch entwickelte, kamen zwei neue hinzu, die sich stark davon unterscheiden -- die Welt der Weite des Kosmos und die Welt der Atome und ihrer Kerne, in Bezug auf die offenbar Weg, um die grundlegenden Parameter des wissenschaftlichen Denkens zu ändern -- Konstanten der physischen Realität, mit denen wir den gesamten Inhalt der Wissenschaft quantitativ vergleichen.



Wir können noch nicht alle Schlussfolgerungen in der Arbeitsmethodik vorhersehen, die sich daraus ergeben werden. Im Allgemeinen wird diese Komplexität nur wissenschaftlich empirisch festgestellt. Es war weder durch die Wissenschaft noch durch philosophisches oder religiöses Denken vorgesehen. Nur in einem Teil davon -- nicht in dem Hauptteil -- sehen wir die Fäden seines Ursprungs, die in die ferne Vergangenheit führen, die erst zu Beginn des 17. Jahrhunderts, als Levenguk die unsichtbare Welt der Organismen enthüllte, und am Ende des 18. Jahrhunderts, als V. Herschel sie entdeckte, deutlich wurde die Welt jenseits unseres Sonnensystems. Aber erst jetzt wird klar, wenn die wissenschaftliche Theorie wissenschaftlich fundierte Tatsachen berücksichtigt hat, dass es nicht nur darum ging, zwischen den Größen zu unterscheiden, sondern um eine völlig andere Herangehensweise unseres Denkapparats an die Realität in ihren atomaren und kosmischen Aspekten.



52. Die nahe Zukunft wird uns wahrscheinlich viel erklären, aber es kann bereits argumentiert werden, dass die Grundidee, auf der jede Philosophie basiert, die absolute Unveränderlichkeit der Vernunft und ihre tatsächliche Unveränderlichkeit nicht der Realität entspricht. In der wissenschaftlichen Arbeit sind wir tatsächlich auf die Unvollkommenheit und Komplexität des wissenschaftlichen Apparats des Homo sapiens gestoßen. Wir hätten dies aus einer empirischen Verallgemeinerung, aus einem Evolutionsprozess vorhersehen können. Homo sapiens ist nicht die Vollendung der Schöpfung, er ist nicht der Besitzer eines perfekten Denkapparats. Es dient als Zwischenglied in einer langen Kette von Wesen, die eine Vergangenheit haben und zweifellos eine Zukunft haben werden, die einen weniger perfekten Denkapparat als er haben und einen perfekteren als er haben werden.



In den Schwierigkeiten, die Realität zu verstehen, die wir erleben, haben wir es nicht mit einer Krise der Wissenschaft zu tun, wie manche Leute denken, sondern mit der langsamen und schwierigen Verbesserung unserer wissenschaftlichen Grundmethodik. Es gibt eine große Arbeit in dieser Richtung, die bisher beispiellos war.



Vivid Ausdruck davon ist die dramatische und schnelle Veränderung in unserem Verständnis der Zeit . Zeit ist für uns nicht nur untrennbar mit dem Raum verbunden, [a] als ob sein anderer Ausdruck. Die Zeit ist mit Ereignissen gefüllt, die so real sind wie der Raum mit Materie und Energie. Dies sind zwei Seiten desselben Phänomens. Wir studieren nicht Raum und Zeit, sondern Raum-Zeit. Zum ersten Mal tun wir dies bewusst in der Wissenschaft.



Die Wissenschaft hat auch einen neuen und tiefgreifenden Ansatz für die wissenschaftliche Erforschung des Weltraums.



Zum ersten Mal im frühen neunzehnten Jahrhundert. N.I. Lobachevsky (1793-1856) stellte die Frage in einer wissenschaftlich lösbaren Form, ob für unsere Galaxie (Universum) der reale (physische) Raum ein euklidischer Raum ist oder ein neuer Raum, den er und unabhängig J. Boley (1802-1860) als geometrisch mächtig etabliert haben existieren zusammen mit dem Raum der euklidischen Geometrie.



Wir werden später sehen, welche Bedeutung der von Lobachevsky angegebene Forschungsweg für die Struktur der Biosphäre hat, wenn wir eine logische Korrektur in seine Argumentation einführen, die mir unvermeidlich erscheint.



Es gibt keine Daten, um die Schlussfolgerungen der Geometrie und der gesamten Mathematik im Allgemeinen mit ihren Zahlen und Symbolen von anderen naturwissenschaftlichen Daten zu trennen. Wir wissen, dass die Mathematik historisch aus einer empirischen wissenschaftlichen Beobachtung der Realität, insbesondere ihrer Biosphäre, entstanden ist.



Natürlich waren theoretische Konstruktionen immer abstrakter als natürliche Objekte und haben daher möglicherweise keinen Platz in natürlichen Körpern und natürlichen Phänomenen der Biosphäre, selbst wenn sie logisch korrekt aus empirischem Wissen abgeleitet sind. Wir sehen dies bei jedem Schritt, da alles, was empirisch in der Wissenschaft etabliert ist, im Wesentlichen auch in seinen theoretisch akzeptablen Erscheinungsformen unendlich ist, genau wie die unendliche Biosphäre, in der sich das wissenschaftliche Denken manifestiert.



Wir wissen, dass die Geometrie von Euklid und Lobatschewski zwei der unzähligen möglichen sind. Sie fallen in drei Typen (Euklidisch, Lobatschewski und Riemann) und entwickeln derzeit eine gemeinsame Geometrie, die alle abdecken. Während Lobachevsky war dies nicht bekannt, und deshalb konnte er die Frage nach der einheitlichen Geometrie des Kosmos aufwerfen. Mit dem gleichen Recht können wir über die geometrische Heterogenität der Realität sprechen, über die gleichzeitige Manifestation von materieller Energie, hauptsächlich materiellen, physikalischen Zuständen des Raums, die verschiedene Geometrien unterscheiden, in der Realität. Wir werden später sehen, dass dieses Problem jetzt in der Heterogenität der Biosphäre, in ihren inerten und lebenden natürlichen Körpern offenbart wird. Ich werde später darauf zurückkommen. 69 Es müssen bisher unbekannte Prozesse des Übergangs eines solchen physikalischen Raumzustands mit einer geometrischen Struktur in einen Raum mit einem anderen beobachtet werden.



53. Zur gleichen Zeit erschien eine neue und die Analyse vertiefte sich in den alten Wissensgebieten, die wie die Mathematik eine hohe Perfektion in der Logik erreichten . Sie befindet sich derzeit in einer Umstrukturierung. Von geringerem Interesse für uns ist sein eher philosophischer Teil -- die Erkenntnistheorie.



Die Logik des Aristoteles ist die Logik der Konzepte . Inzwischen beschäftigen wir uns wie in der Wissenschaft mit natürlichen Körpern und Naturphänomenen, deren Konzept verbal bewegungslos ist, aber im historischen Verlauf des wissenschaftlichen Wissens sich grundlegend in seinem Verständnis ändert, was den Wissensstand dieser Generation äußerst tief und scharf widerspiegelt. Die Logik von Aristoteles ist selbst in seinen jüngsten Änderungen und Ergänzungen des 17. Jahrhunderts, die wichtige Änderungen einführten, ein zu grobes Werkzeug und erfordert eine eingehendere Analyse. In einem separaten Ausflug werde ich weiter unten darauf zurückkommen.



54. Mathematik und Logik sind nur die Hauptmethoden der Bauwissenschaft. Seit dem 17. Jahrhundert, dem Jahrhundert der Schaffung neuer westeuropäischer Wissenschaft und Philosophie, ist ein neues Feld der wissenschaftlichen Synthese und Analyse gewachsen -- die Methodik der wissenschaftlichen Arbeit . Sie erstellt, überprüft und bewertet den Hauptinhalt der Wissenschaft -- empirisch ihren wissenschaftlichen Apparat. Ich habe bereits (§ 50) von seiner großen Bedeutung in der Geschichte der Wissenschaft gesprochen, die alle wächst und grundlegend ist.



Auf seltsame Weise wird die Methodik der wissenschaftlichen Arbeit, die eine große Literatur und Handbücher von größter Vielfalt enthält, von der philosophischen Analyse nicht vollständig abgedeckt. Mittlerweile gibt es getrennte wissenschaftliche Disziplinen wie die Fehlertheorie, einige Bereiche der Wahrscheinlichkeitstheorie, der mathematischen Physik, der analytischen Chemie, der historischen Kritik, der Diplomatie usw., nur dank derer der wissenschaftliche Apparat die Fähigkeit erhält, in das Unbekannte einzudringen, das das 20. Jahrhundert charakterisiert. und eröffnet vor der Wissenschaft unserer Zeit unbegrenzte Möglichkeiten für die weitere Umarmung der Natur.



Die Methodik der wissenschaftlichen Arbeit ist, wie aus dem Vorstehenden hervorgeht, nicht Teil der Logik, und noch mehr [der Erkenntnistheorie].



In letzter Zeit wurde in diesem Bereich eine wesentliche Änderung vorgenommen, die wahrscheinlich von größter Bedeutung ist. Es entsteht eine neue eigenartige Methode, in das Unbekannte einzudringen, die durch Erfolg gerechtfertigt ist, die wir uns aber (bildlich) mit einem Modell nicht vorstellen können. Es ist, als ob es in Form eines durch Intuition erzeugten „Symbols“ ausgedrückt wird, d.h. Es ist für den Forscher unbewusst, unzählige Fakten zu erfassen, ein neues Konzept, das der Realität entspricht. Wir können diese Symbole noch nicht logisch verstehen, aber wir können sie mathematisch analysieren und auf diese Weise neue Phänomene entdecken oder theoretische Verallgemeinerungen für sie erstellen, die in allen logischen Schlussfolgerungen durch Fakten verifiziert sind und deren Maß und Anzahl genau berücksichtigt werden.



Diese Methode des Suchens und Entdeckens hat übrigens breite Anwendung in der Physik des Atoms 70 gefunden -- dem Gebiet wissenschaftlicher Erkenntnisse, das vollständig im mikroskopischen Teil der Welt liegt. Das Konzept der Größe von h , Photon, Quantum ist ein eindrucksvolles Beispiel für diese neue, wahrscheinlich enorme Kraft der Kraft der wissenschaftlichen Durchdringung und der Erweiterung wissenschaftlicher Methoden. Neue wissenschaftliche Disziplinen werden geschaffen, wie neue Mechaniken, und neue Zweige der Mathematik wachsen aus ihnen heraus.



Unser mathematischer und logischer Apparat verändert sich grundlegend im Vergleich zu dem, was der Wissenschaftler vor 40-50 Jahren zur Verfügung hatte.



Es ist jedoch klar, dass dies nur der Anfang ist. Mit Mühe, aber unwiderruflich, werden neue Methoden zum Eindringen in das Unbekannte geschaffen, die mit der Suche und Schaffung neuer Bereiche der theoretischen Physik verbunden sind, in denen das visuelle Bild von Phänomenen entweder verdeckt ist oder überhaupt nicht konstruiert werden kann.



Diese neue Technik ist jedoch nicht nur auf neue Wissensgebiete wie die Atomphysik anwendbar. Natürlich ist bei seiner Verwendung große Vorsicht geboten, und in der wissenschaftlichen Literatur gibt es viele fruchtlose und fehlerhafte Anwendungen, aber dies ist unter den Bedingungen all unserer wissenschaftlichen Arbeiten unvermeidlich, bei denen wir viel unnötige und unnötige Arbeit leisten. Wir arbeiten hier, wie die Natur funktioniert, wie die Organisation der Biosphäre offenbart wird (Abschnitt 3). Es ist äußerst wichtig, dass gleichzeitig mit der neuen Methodik noch größere Phänomene beobachtet werden, die möglicherweise zur Schaffung neuer Wissensbereiche führen -- neuer Wissenschaften .



Das Tempo ihrer Entstehung und der Bereich ihrer Gefangennahme in den letzten vierzig Jahren sind kontinuierlich gewachsen.



55. Vor vierzehn Jahren habe ich dieses Merkmal wissenschaftlicher Erkenntnisse mit einer Explosion verglichen , und dieser Vergleich scheint mir die Realität richtig auszudrücken.



Wir können den Beginn dieser Explosion mit außergewöhnlicher Genauigkeit verfolgen . E. Rutherford, 71 , wies zutreffend darauf hin, dass die moderne Entwicklung der Physik, die unser Weltbild verändert hat, auf 9/10 aufgrund der Radioaktivität bei den von der modernen Physik vorgebrachten Problemen zurückzuführen ist.



jedoch untrennbar mit der Radioaktivität, für drei Jahre an verschiedenen Orten Natürlich kann man über die Richtigkeit dieser Einschätzung argumentieren, wie das Experiment wunderbar ging fast gleichzeitig die Eröffnung von drei neuen Phänomenen in der Tat -- die X -- Strahlen in Würzburg V. Roentgen 1895 72 Radioaktivität von Uran von A. Becquerel in Paris im Jahr 1896, 73 Elektronen in Cambridge von D. D. Thomson im Jahr 1897. 74 Ihr Zufall bestimmte eine Explosion der wissenschaftlichen Kreativität. Aber ohne das Öffnen der Hauptphänomene Radioaktivität -- Gebrechlichkeit -- Atomen, -- erklärt und X -- Strahlen, und Elektronen, und ihr Aussehen, die moderne Physik würde nicht. 75



Die Entdeckung der Radioaktivität sowie X -- Strahlen und Elektronen, aus der wissenschaftlichen Genauigkeit verfolgt werden , mit denen es nicht immer möglich , zu tun ist. 1. März 1896 Becquerel in der Sitzung der Pariser Akademie einen Bericht über die Uran -- Strahlung Strahlen, in der Dunkelheit, ähnlich wie das Fotografieren X vor -Strahlen, Roentgen open [einige] Monate. Dies war die Entdeckung der Radioaktivität. Die ersten Bilder von W. Roentgen geschickt, an der Pariser Akademie vom 20. Januar gezeigt wurden, 1896 und Becquerel sofort, zugleich auf der Grundlage angeblicher Verbindungen X -- Strahlen mit einem Glas von Kathodenfluoreszenzlampen, begann er seine Experimente. Er ging den experimentellen richtigen Weg und ging von im Wesentlichen falschen Prämissen aus. Die Entdeckung von Röntgenstrahlen ergab die Existenz von „dunklen“ Strahlen, die in Materie eindringen und auf eine fotografische Platte einwirken. Becquerel wandte sofort an, ausgehend von der Fluoreszenz, mit der er sie verband. Diese neuen experimentellen Ideen mit Uransalzen, die neue Strahlung entdeckt hatten, bewiesen, dass sie mit dem Uranatom verbunden sind, Röntgenstrahlen und Strahlung dafür erhalten hatten . In den kommenden Monaten schufen die Kräfte einer riesigen Armee von Physikern auf der ganzen Welt die Doktrin der Radioaktivität, und die rasche Entwicklung eines neuen Weltverständnisses begann. Der Keim der Explosion war die Entdeckung der Radioaktivität.



Wir wissen jetzt, dass es in den Annalen der Wissenschaft zahlreiche Hinweise auf einzelne Tatsachen, Beobachtungen und Überlegungen gibt, die hier zusammenhängen.



A. Becquerel selbst glaubte, Radioaktivität nur entdeckt zu haben, weil er mit seinem ganzen Leben und dem Leben seiner Vorfahren darauf vorbereitet war. Er sagte: „Die Entdeckung der Radioaktivität sollte im Labor des Museums (Museum d'Histoire Naturelle in Paris, alter Jardins des Plantes) gemacht werden, und wenn mein Vater 1896 am Leben wäre, wäre er sein Autor gewesen. 76



In der Tat ist das physikalische Labor des Naturhistorischen Museums in Paris ein völlig außergewöhnliches Phänomen in der Geschichte der Wissenschaft. Kontinuierlich seit 1815, d.h. Seit bereits 123 Jahren gehören die Direktoren zur Familie Beckerel: Urgroßvater, Großvater, Vater und Sohn -- A.S. Becquerel (1788–1878), A.E. Becquerel (1820-1891), A.A. Becquerel (1852-1908), J. Becquerel (1878-1953). Es entstehen Werke, die seit ihrer Kindheit von Generationen nacheinander ausgeführt werden und sich auf die Themen beziehen, mit denen sie sich befassen, sowie auf ihre Entdeckung und im Wesentlichen auf das Phänomen der Radioaktivität.



Becquerel hatte recht, zwangsläufig in der Tat -- es ist ganz neu, niemand sollte Phänomene -- radioaktive Zerfall, Zerbrechlichkeit, bestimmte Zeiten der Existenz des Atoms hatten in der Familie Bq unmittelbar nach der Entdeckung geöffnet werden X -- Strahlen. Denn nur in dieser Familie richtete sich die wissenschaftliche Aufmerksamkeit mehrerer Generationen von Physikern auf die Phänomene Glühen, Elektrizität und Lichtwirkung (Fotografien). Bereits A.S. Becquerel, ein Physiker mit großem Interesse, der hauptsächlich mit Elektrizität experimentierte, untersuchte das Phänomen der Phosphoreszenz zusammen mit Bio und seinem Sohn A.E. Becquerel, 1839. Teilweise im Zusammenhang mit diesen Arbeiten entdeckte Stokes 1852 die von ihm als Fluoreszenz bezeichnete Phosphoreszenz von Uran, die die Grundlage zahlreicher späterer Arbeiten von A.E. Becquerel (1859 und folgende), zuerst mit seinem Vater, dann mit seinem Sohn, der später Radiumstrahlung in Uran entdeckte. Schon damals wurden die Merkmale dieser Phosphoreszenz aufgedeckt, die nicht geklärt wurden, scheint mir bis zum Ende bisher. 77 Becquerels beschäftigten sich 1896 mit Uran -- kontinuierlich seit über 40 Jahren.



56. Es ist daher nicht verwunderlich, dass Uransalze 1896 das erste Untersuchungsobjekt waren und sofort zur Entdeckung der Radioaktivität führten. Die große Erfahrung und Vertrautheit mit diesen Phänomenen haben Becquerel für Familien zur Verfügung angehäuft drei Generationen, wenn X -- Strahlen Roentgen neu entdeckt g -Strahlung im Zusammenhang mit Lumineszenzerscheinungen Becquerel untersucht.



Ich habe mich ausführlicher mit dieser Geschichte befasst, weil wir sie kaum ruhig und ohne Zweifel auf einen einfachen Fall und auf Zufall reduzieren können. A. Becquerel, der es klar gemacht hat, war sich dessen bewusst, wie ich betonte.



Unwillkürlich hört das Denken vor solchen Zufällen auf und sucht nach einer wissenschaftlichen Erklärung für sie.



Die Geschichte des menschlichen wissenschaftlichen Denkens ist eine wissenschaftliche Disziplin, d.h. es sollte sich bemühen, wissenschaftlich genaue Fakten wissenschaftlich zu verknüpfen, nach Verallgemeinerungen zu suchen und diese in das System und die Ordnung zu verteilen. Die Entdeckung der Radioaktivität durch A. Becquerel und ihre Herstellung durch Untersuchung der Lichteigenschaften von Uran, die in der Familie der Physiker Becquerel drei Generationen andauerte, ist eine wissenschaftliche Tatsache, mit der wir rechnen müssen.



Wir können nicht vor ihm aufhören. Wenn Laplace nur ein wenig richtig wäre und die mathematische Formel (“Laplace-Formel“) das Tempo der Weltbewegung, des „Lebens“ der Welt erfassen könnte, müssten wir auf solche Manifestationen in wissenschaftlichen Entdeckungen des Ausmaßes der Entdeckung der Phänomene der Radioaktivität warten, die wir erlebt haben.



Allein aus diesem Grund können wir dieses echte frühere Zusammentreffen von Arbeiten, die seit mehreren Generationen über Uran durchgeführt werden, nicht ignorieren, da die Entdeckung der Radioaktivität im richtigen Moment erfolgt. Es gibt keinen Fall in der Wissenschaft und solche Zufälle in ihrer Geschichte sind nicht so selten. 78 Die Erfolge der Analyse nach Laplace lassen meines Erachtens vermuten, dass Laplace in seinem Bild nicht in gewissem Maße falsch sein konnte. Aber in welcher?



57. Die Folgen der Entdeckung von Becquerel wurden vom ganzen Leben der Menschheit, all seinem philosophischen Denken, all seiner wissenschaftlichen Weltanschauung erfasst.



Die Konsequenzen der Relativitätstheorie von A. Einstein 10 Jahre nach A. Becquerel, der sich bereits in der wissenschaftlichen Atmosphäre befand, alte Ideen mit Radioaktivität zu brechen, in der Atmosphäre des Sieges der atomistischen Weltanschauung, ihres siegreichen Marsches, zeigen dasselbe Bild. Die Relativitätstheorie ist aus dem wissenschaftlich-theoretischen und mathematischen Denken hervorgegangen. Seine Geschichte ist viel besser untersucht als die Geschichte der Radioaktivität.



Aber auch hier sind ein bescheidener Anfang 79 und ein kontinuierliches, immer intensiver werdendes wissenschaftliches empirisches Material wissenschaftlicher Fakten charakteristisch , wobei die Relativitätstheorie genetisch und logisch verwandt ist. Für einen Naturforscher sollte nur diese Seite genauer Fakten, nicht mathematische und philosophische Konzepte, von vorrangiger Bedeutung sein.



58. Ein weiteres charakteristisches Merkmal wissenschaftlicher Erkenntnisse sollte berücksichtigt werden, da es im laufenden Prozess eine wichtige Rolle spielt.



Wie wir sehen werden (§ 46), unterscheidet sich die Wissenschaft im sozialen Leben stark von Philosophie und Religion darin, dass sie für alle Zeiten, soziale Medien und Staatsformationen im Wesentlichen gleich und gleich ist .



Es ist wahr, dass die Menschheit mit der schwierigen Erfahrung der Geschichte dazu kommt, denn sowohl die Religion als auch die staatlichen sozialen Formationen versuchen seit Tausenden von Jahren, Einheit zu schaffen und alle mit Gewalt in ein einziges Verständnis von Sinn und Zweck des Lebens einzubeziehen. In der jahrtausendealten Geschichte der Menschheit gab es noch nie ein solches Verständnis. Die ganze Zeit gab es gleichzeitig widersprüchliche oder koexistierende unterschiedliche Verständnisse von ihnen. Ein solches Streben, das nun nach einem erfolglosen Kampf und verlorener Kraft zu einer klaren Illusion für alle zu werden scheint, beginnt in die Vergangenheit zurückzugehen. Es gab solche Versuche in der Geschichte der Philosophie, die auch zum völligen Zusammenbruch führten.



Es ist möglich, sozialstaatliche Vereinigungen außer Acht zu lassen, da sie aus noosphärischer Sicht niemals wesentliche Teile davon umfassten. Die sogenannten Weltreiche haben immer im Wesentlichen getrennte Teile des Landes besetzt und waren immer gleichzeitig existierend, kamen -- durch Gewalt oder Leben -- ins Gleichgewicht miteinander. Die Idee einer einheitlichen staatlichen Vereinigung der gesamten Menschheit wird erst in unserer Zeit Wirklichkeit, und das wird offensichtlich bisher nur ein reales Ideal, an dessen Möglichkeit nicht gezweifelt werden kann. Es ist klar, dass die Schaffung einer solchen Einheit eine notwendige Voraussetzung für die Organisation der Noosphäre ist und die Menschheit unweigerlich dazu kommen wird.



In der Geschichte der Religionen war in jeder Form -- theistisch, pantheistisch oder atheistisch -- ein echtes Verlangen nach Einheit unvermeidlich, da sie alle auf Glauben beruhen und rationalistische Zweifel an ihrer Richtigkeit überwinden. Das Leben hat diesen Wunsch unweigerlich gebrochen, aber die Gläubigen glauben trotz der bitteren Erfahrung von Generationen an die Verwirklichung dieses Ideals. Mit dem Wachstum der Wissenschaft nimmt die wahre Bedeutung dieses Glaubens an die Weltgeschichte rapide ab. Für die westliche christliche Kirche, für den Katholizismus endete die Möglichkeit einer solchen Vereinigung tatsächlich mit der Schaffung protestantischer Kirchen, die von der Staatsmacht unterstützt wurden, und mit der gleichen Rechtfertigung muslimischer religiöser Sekten. Die tiefe Krise der Religion, die jetzt herrscht, bringt sie in dieser Hinsicht aus dem wirklichen Boden der Geschichte. Es ist unwahrscheinlich, dass atheistische Ideen, die im Wesentlichen dasselbe Glaubensgegenstand sind und auf philosophischen Schlussfolgerungen beruhen, so stark werden, dass die Menschheit eine einheitliche Sichtweise erhält. Im Wesentlichen sind dies auch religiöse Konzepte, die auf Glauben beruhen.



59. Noch weniger kann Einheit schaffen -- die Universalität des Verstehens -- philosophisches Denken. Es basiert immer auf Zweifel und Rationalisierung des Bestehenden. Es gab nie eine Zeit, in der eine Philosophie als wahr anerkannt wurde. Philosophie basiert immer auf Vernunft und ist eng mit der Persönlichkeit verbunden. Persönlichkeitstypen entsprechen immer unterschiedlichen Philosophietypen. Die Persönlichkeit ist untrennbar mit der philosophischen Reflexion verbunden, und der Geist kann kein Maß dafür angeben, sondern die gesamte Persönlichkeit vollständig umfassen. Die Philosophie löst niemals die Geheimnisse der Welt. Sie sucht sie. Sie versucht, das Leben mit Vernunft zu umarmen, aber sie kann dies niemals erreichen. Die philosophische Wahrheit kann immer von einer freien, suchenden Persönlichkeit in Frage gestellt werden. Durch den jahrtausendealten Prozess ihrer Existenz hat die Philosophie einen mächtigen menschlichen Geist geschaffen, sie hat eine tiefe Analyse durch den Geist der menschlichen Sprache durchlaufen, Zehntausende von Jahren inmitten des sozialen Lebens erarbeitet, abstrakte Konzepte entwickelt, Wissenszweige wie Logik und Mathematik geschaffen, die Grundlagen unseres wissenschaftlichen Wissens. Die von ihr geschaffene Psychologie, in der innere Erfahrung und Reflexion über sich selbst eine große Rolle spielen, entwickelt sich ebenfalls zu einem von ihr unabhängigen wissenschaftlichen Feld. Dieser Bereich von Phänomenen ist so groß und unendlich, tief wie die uns umgebende Realität.



Die Wissenschaft ist vor einem Jahrtausend aus der Philosophie hervorgegangen. Es ist äußerst charakteristisch und historisch wichtig, dass wir drei oder vier unabhängige Zentren für die Schaffung der Philosophie haben, die nur für wenige, zwei oder drei Generationen miteinander in Verbindung stehen und Jahrhunderte und Jahrtausende einander unbekannt geblieben sind. Die Arbeit des Denkens -- sozial, religiös, philosophisch und wissenschaftlich -- ging in ihnen unabhängig über viele Jahrhunderte, wenn nicht Jahrtausende weiter. Dies waren die Zentren des Mittelmeers, der Indianer und der Chinesen. Vielleicht müssen Sie hier das Zentrum des Pazifik-Amerikaners anbringen, das weit hinter den ersten drei liegt und von dem wir wenig wissen. Er verschwand und starb im 16. Jahrhundert in einer historischen Katastrophe. Anscheinend befinden sich die philosophischen und religiösen Zentren der Alten Welt seit Generationen, die Pythagoras, Konfuzius und Shakya-Muni nahe stehen, seit geraumer Zeit im kulturellen Austausch.



Ein neuer Austausch, vergleichbar mit diesem ersten, begann in Jahrhunderten in unserer Nähe. Über Jahrhunderte hinweg wurde in diesen Zentren unabhängig voneinander philosophisch gedacht, am stärksten in Indien und im hellenisch-semitischen. Es ist merkwürdig, dass wir im Laufe der Geschichte der Philosophie eine extreme Analogie des historischen Prozesses in der Entwicklung sowohl philosophischer Systeme als auch logischer Strukturen sehen. Offenbar ging die indische Logik tiefer aristotelischen und der Verlauf der philosophischen indischen Denkens seit fast tausend Jahren (mit einer Genauigkeit von ein paar hundert Jahren, plus oder minus -- die Chronologie der indischen Philosophie ist immer noch sehr unvollkommen ist) erreicht westliche Philosophie Höhe des Endes des XVIII Jahrhundert, die .. ist E. unsere Philosophie nur im 18. Jahrhundert. holte das indische philosophische Denken ein. Viele Jahrhunderte lang wurde die Tradition des philosophischen Denkens und seine lebendige Erfahrung nicht unterbrochen, aber im politischen Niedergang der indischen Kultur erstarrte das kreative philosophische Denken Indiens und wahrscheinlich in den XI-XII Jahrhunderten. Der bekannte Philosoph Ramanuya (1050-1137) war jahrhundertelang der letzte große Vertreter von ihr. Die philosophische Kultur und die philosophischen Interessen wurden jedoch nicht unterbrochen, und von Zeit zu Zeit entstand bis zum 17. Jahrhundert und später ein eigenständiger Gedanke. Im 19. Jahrhundert. Unter dem Einfluss der europäischen Wissenschaft begann nach mehr als dreitausendjähriger lebendiger philosophischer Tradition in Indien eine Wiederbelebung des unabhängigen Denkens auf der Grundlage der Universalität wissenschaftlicher Erkenntnisse. 



Seit mehr als einem Jahrtausend hat das indische philosophische Denken einen tiefgreifenden Einfluss auf tibetische, chinesische, koreanische und japanische Staaten.



Dieser Einfluss manifestierte sich über viele Jahrhunderte mit großen Unterbrechungen und traf insbesondere in den chinesischen Staaten in diesem unabhängigen Zentrum der menschlichen Kultur auf philosophische Fragen, die unabhängig entstanden und eine tiefe und lange Geschichte hatten, die sich gerade erst vor uns zu öffnen begann. In der Zeit des Niedergangs des kreativen philosophischen Denkens in Indien hörten die Beziehungen zu diesen verwandten Manifestationen philosophischer Quests auf und wurden erst in unserer Zeit wieder aufgenommen. Gerade zu der Zeit, als die Berichterstattung über diese alten Zivilisationen durch die mächtige Kraft unserer Wissenschaft erfolgte.



60. Das neunzehnte Jahrhundert und insbesondere das zwanzigste nach dem barbarischen Krieg von 1914-1918 veränderten die religiöse und philosophische Struktur der gesamten Menschheit radikal und schufen eine solide Grundlage für eine einzige universelle Wissenschaft, die die gesamte Menschheit umfasste und ihr wissenschaftliche Einheit verlieh.



Die Bewegung begann Mitte des 18. Jahrhunderts. in Nordamerika, wo die Briten und Franzosen den Grundstein für die nordamerikanische wissenschaftliche Arbeit legten. Noch früher begann es im 16. Jahrhundert in Südamerika in seinem spanischen und portugiesischen Kulturumfeld, aber hier erstarrte es schnell und schuf erst im 19. Jahrhundert ein solides wissenschaftliches Umfeld.



Ganz anders war es in Nordamerika, wo durch allmähliches und kontinuierliches Wachstum ein mächtiges wissenschaftliches Zentrum angelsächsischer wissenschaftlicher Arbeit entstand, das heute die mächtigste wissenschaftliche Organisation der Menschheit ist. In Kanada ist das anglo-französische Arbeitszentrum, das mit dem angelsächsischen zusammengelegt wurde, erhalten geblieben.



Zu Beginn des achtzehnten Jahrhunderts. Die Grundlagen der wissenschaftlichen Forschung wurden nach Moskau, Russland, verlegt und mit staatlicher Unterstützung schnell auf dem asiatischen Kontinent verbreitet und nach Nordamerika verlagert. Hier wurde dank der Expansion des großen russischen Volkes wissenschaftliches Denken und Arbeiten in ein dem Westen fremdes Land eingeführt, eine andere Lebenstradition.



Die kraftvolle Entwicklung der Kolonialmacht Großbritanniens und die Besonderheit ihrer Politik, die Ende des 19. bis 20. Jahrhunderts führte. Die Schaffung des britischen Empire, von dem gesagt werden kann, dass es den gesamten Planeten zu einem einzigen kulturellen Ganzen zusammengeschlossen hat, hatte einen starken Einfluss auf die Abdeckung seiner riesigen Gebiete durch eine einzige Wissenschaft. In Nordamerika, Australien, Neuseeland und Südafrika wurden im 19. Jahrhundert mächtige wissenschaftliche Zentren unabhängiger wissenschaftlicher Arbeit geschaffen schuf ein niederländisch-afrikanisches Wissenschaftszentrum. Nicht weniger wichtig war, dass unter dem Einfluss des englischen wissenschaftlichen Denkens die alte Zivilisation Indiens und Birmas von wissenschaftlichem Denken und wissenschaftlicher Arbeit einbezogen und angenommen wurde. Hier wurden Zentren wissenschaftlicher Arbeit geschaffen und die wissenschaftliche Wiederbelebung Indiens begann, basierend auf einer einheitlichen Wissenschaft und ihrer Philosophie und Religion. Durch indisches Denken strömen immer mehr Menschen einer anderen philosophischen Kultur als Christen in das wissenschaftliche Umfeld.



Kreatives modernes wissenschaftliches Denken drang langsam in den muslimischen Osten, Nordafrika, Kleinasien und Persien in diesem Kulturbereich ein, der vom 8. bis 12. Jahrhundert an der Spitze des wissenschaftlichen Denkens der Menschheit stand, in dem jedoch religiöse und politische Ereignisse langsam verblassten wissenschaftliche Arbeit, erst in unserem Jahrhundert gestoppt.



In der Mitte des 19. Jahrhunderts, nach vielen Jahren der Unterbrechung, wurde Japan mit der westeuropäischen Kultur verbunden und wie Russland eineinhalb Jahre zuvor durch staatliche Maßnahmen mächtige Zentren der wissenschaftlichen Kultur gegründet und fest mit der Weltwissenschaft verbunden.



Schließlich trat China nach dem Zusammenbruch der Mandschu-Dynastie schnell in die wissenschaftliche Arbeit der Menschheit ein. Es ist merkwürdig, dass in der Ära von Peter dem Großen China den Europäern und Russen, einschließlich eines fortgeschrittenen Landes in Bezug auf seine wissenschaftliche Bedeutung, erschien, und man könnte dann für das Moskauer Königreich überlegen, in welche Richtung es sich wenden sollte -- nach Westen oder Osten, um sich der Welt anzuschließen Wissenschaft. Denn nur zu Peters Zeiten dank des Erfolgs der genauen Kenntnis des späten XVII -- frühen XVIII Jahrhunderts. Die potentielle Kraft der neuen Wissenschaft hat die Augen der Zeitgenossen völlig beeinflusst. China wurde im 17. Jahrhundert durch die Jesuiten und andere katholische Missionen von der neuen Wissenschaft in ihrer staatlichen Anwendung und erst zu Beginn des 18. Jahrhunderts abgedeckt. Dieses mehr als ein Jahrhundert alte Werk wurde zerstört, und erst nach der Schwächung der Mandschu-Dynastien errichtete China starke Zentren wissenschaftlicher Arbeit. Als der chinesische Bogdykhan Kangxi 1693 eine breite Toleranz gewährte und die erste Anwendung von genauem Wissen in Form astronomischer Beobachtungen in Bezug auf ihren angewandten und wissenschaftlichen Wert in das staatliche System Chinas eingeführt wurde, blieb China in seiner Technologie und in seinen wissenschaftlichen Grundlagen nicht hinter dem Stand der Dinge zurück im modernen Westeuropa, und er war wissenschaftlich und technisch mächtiger als das damalige Moskauer Königreich. Als Kangxi 1723 einige Jahre vor seinem Tod aus religiösen Gründen starb, nachdem er den Kontakt zum wissenschaftlichen Denken des Westens verloren hatte, war China sofort rückständig, da der Sieg der Newtonschen Weltanschauung und neuer Methoden der Mathematik die reale Staatsmacht wissenschaftlicher Erkenntnisse bis zur Mitte des Jahrhunderts ungewöhnlich erhöhte . China bezahlte im 19. Jahrhundert grausam Kangxis Fehler. erwies sich vor der Gefangennahme von Amerikanern und Europäern als hilflos. Es begann in der Mitte des 18. Jahrhunderts. Eine langsame Wiederbelebung führte die Chinesen zu einem starken Bewusstsein für die Notwendigkeit, die Macht einer einzigen Wissenschaft zu beherrschen. Sie sind jetzt fest auf diesem Weg.



61. Also im zwanzigsten Jahrhundert. Ein einziger wissenschaftlicher Gedanke bedeckte die gesamte Oberfläche des Planeten, alle Zustände auf ihm. Überall wurden zahlreiche Zentren des wissenschaftlichen Denkens und der wissenschaftlichen Forschung geschaffen.



Dies ist die erste Grundvoraussetzung für den Übergang der Biosphäre zur Noosphäre. Vor diesem allgemeinen und so vielfältigen Hintergrund entfaltet sich eine Explosion der wissenschaftlichen Kreativität des 20. Jahrhunderts, die die Grenzen und Abgrenzungen der Staaten nicht berücksichtigt. Jede wissenschaftliche Tatsache, jede wissenschaftliche Beobachtung, wo und von wem auch immer sie gemacht wird, tritt in einen einzigen wissenschaftlichen Apparat ein, sie wird klassifiziert und auf eine einzige Form reduziert, sie wird sofort zur gemeinsamen Eigenschaft für Kritik, Reflexion und wissenschaftliche Arbeit.



Die wissenschaftliche Arbeit wird jedoch nicht allein von einer solchen Organisation bestimmt. Es erfordert ein günstiges Entwicklungsumfeld, und dies wird durch die breiteste Popularisierung wissenschaftlicher Erkenntnisse, ihre Vorherrschaft in der Schulbildung, die völlige Freiheit der wissenschaftlichen Suche, ihre Befreiung von jeglichen routinemäßigen, religiösen, philosophischen oder sozialen Wegen erreicht.



Das 20. Jahrhundert ist das Jahrhundert der Bedeutung der Massen. Gleichzeitig sehen wir darin eine energetische, breite Entwicklung der verschiedensten Formen der öffentlichen Bildung. Und wenn man weit davon entfernt ist, dass die angegebenen Fesseln entfernt wurden, werden sie sich im Laufe der Zeit unweigerlich zerstreuen. Die große Bedeutung der demokratischen und sozialen Organisationen der Werktätigen, ihrer internationalen Vereinigungen und der Wunsch nach maximalem wissenschaftlichen Wissen können nicht aufhören. Bisher entsprach diese Seite der Organisation der Arbeitnehmer und ihrer Länderspiele in ihrem Tempo und ihrer Tiefe nicht dem Zeitgeist und erregte nicht genügend Aufmerksamkeit. Diese Arbeit findet auf der ganzen Welt außerhalb des Rahmens von Staaten und Nationalitäten statt. Dies ist ebenso eine Voraussetzung für die Noosphäre wie kreative wissenschaftliche Arbeit.



62. Dieses starke Wachstum wissenschaftlicher Erkenntnisse, von immer größerer Intensität und zunehmender Reichweite fällt mit einer tiefen kreativen Stagnation in verwandten Bereichen zusammen, die eng mit der Wissenschaft verbunden sind -- in der Philosophie und im religiösen Denken.



In der Philosophie des Westens gibt es in unserem Jahrhundert trotz der großen, sogar wachsenden Literatur eine Schwäche in der neuen kreativen Arbeit, unzureichende Tiefe. Philosophische Arbeit nach dem großen Wohlstand in der Zeit des 17. Jahrhunderts. vor dem Beginn des 19. Jahrhunderts. Seit einem ganzen Jahrhundert hat es nichts geschaffen, was der wissenschaftlichen Kreativität des 19. und 20. Jahrhunderts entspricht. Es zerfällt in Einzelheiten, erfasst nicht die allgemeinen Fragen des Lebens, wiederholt das Alte, verliert an Wert für einen wissenschaftlich arbeitenden Denker. Alte, längst tote Konzepte versuchen zu existieren, ohne sich wesentlich in der neuen Umgebung zu verändern, die von der Wissenschaft geschaffen wurde und die sie nicht verstehen. Erst in den letzten Jahren waren diese alten Trends minderwertig, eine neue Bewegung beginnt, aber sie steht bereits unter dem direkten Einfluss des neuen wissenschaftlichen Denkens und der von ihm geschaffenen neuen wissenschaftlichen Weltanschauung. Beobachtet und wichtig für einen Wissenschaftler, der in Bereichen arbeitet, die mit dem Studium des Lebens zusammenhängen, insbesondere für die Biogeochemie, ist die beginnende Bewegung auch mit dem Einfluss neuen wissenschaftlichen Denkens auf ihn verbunden. Die Wissenschaft, die das Neue enthüllt, bricht die alten philosophischen Ideen und zeigt den Weg.



Tatsache ist, dass es in der Geschichte der Philosophie ein Phänomen gibt, das für das wissenschaftliche Denken in unserer Zeit unmöglich ist: Es gibt nur eine Wissenschaft für die ganze Menschheit, es gibt tatsächlich mehrere Philosophien , deren Entwicklung über ein Jahrtausend, lange Jahrhunderte und lange Generationen unabhängig weiterging.



Neben der europäisch-amerikanischen Philosophie gibt es Philosophien aus Indien und China. Und wenn die chinesische Philosophie in einem jahrhundertealten Schlaf liegt und ihre Naturphilosophie der Wissenschaft unserer Zeit stark widerspricht, erwacht die Philosophie Indiens jetzt nach einem jahrhundertealten kreativ latenten Zustand klar und abrupt.



Es scheint mir, dass für neue Bereiche der Wissenschaft -- und insbesondere für die Naturwissenschaften -- die philosophischen Konzepte Indiens jetzt von großem Interesse sind. Nach Jahrhunderten der Stagnation beginnen sie erst unter dem Einfluss der Blütezeit der wissenschaftlichen Weltforschung und der Umarmung des spirituellen Lebens dieses Teils der Menschheit wiederzubeleben, der es geschafft hat, die Errungenschaften der philosophischen Arbeit ihrer Vorfahren über Tausende von Jahren zu bewahren. Aber die Bedeutung dieser umfassenderen und vielleicht tieferen philosophischen Konzepte Indiens für die Wissenschaft scheint mir in Zukunft zum Ausdruck zu kommen. Ab und zu kommt ein neuer wissenschaftlicher Gedanke in den Vordergrund.



63. Das religiöse Bewusstsein der gesamten Menschheit befindet sich derzeit in einer tiefen Krise, teilweise, aber kaum hauptsächlich aufgrund des Wachstums wissenschaftlicher Erkenntnisse und seiner Inkonsistenz mit wissenschaftlichen Errungenschaften, um damit umzugehen.



Zum ersten Mal drückt sich die Verleugnung der Religion als eine der Normen der Kultur der Menschheit deutlich in staatlichen Ideen aus. Tatsächlich gab es in einer Reihe von Staaten und großen Kulturen, zum Beispiel China, Epochen, in denen die Ideologie des politischen Systems Ausdruck eines religiösen Verständnisses der Umwelt war. Unweigerlich und bis zu einem gewissen Grad wird unbewusst dieselbe soziale Struktur wie eine Form der religiösen Manifestation des Lebens, eine obligatorische sozio-staatliche Struktur, die nicht bezweifelt werden kann, jetzt in den Strömungen der Religion offenbart, die die Religion leugnen. In der Tat ist dies, wie es in China der Fall war, eine sozialstaatliche Religion.



Die Menschheit lebt in einer tiefen Krise des religiösen Bewusstseins und steht wahrscheinlich am Rande einer neuen religiösen Kreativität. Alte religiöse Konzepte sollten vor allem unter dem Einfluss des Wachstums des wissenschaftlichen Denkens vertieft und wieder aufgebaut werden.



Solch ein passiver Zustand im Sinne jahrhundertealter führender großer Ideen des philosophischen Denkens und des religiösen Realitätsbewusstseins, insbesondere des Verständnisses des Lebens, mit der Explosion der wissenschaftlichen Kreativität, deren Stärke zunimmt, schafft einen beispiellosen Wert in der Geschichte der Menschheit in der Wissenschaft und neue wissenschaftliche Probleme, die sich in diesem Aspekt eröffnen neue Bedeutung und Beleuchtung.



64. Ein weiteres neues Phänomen verändert gerade in unserem 20. Jahrhundert alle Bedingungen für das Wachstum der wissenschaftlichen Kreativität dramatisch. und gibt ihnen einen besonderen Charakter und eine besondere, bisher beispiellose Bedeutung.



Unsere Zeit ist in dieser Hinsicht wesentlich anders und beispiellos, denn anscheinend befinden wir uns zum ersten Mal in der Geschichte der Menschheit unter den Bedingungen eines einzigen historischen Prozesses, der die gesamte Biosphäre des Planeten abgedeckt hat . Die komplexen historischen Prozesse, die teilweise über mehrere Generationen hinweg unabhängig und abgeschlossen waren, sind gerade zu Ende gegangen und haben am Ende unseres 20. Jahrhunderts ein einziges, untrennbar miteinander verbundenes Ganzes geschaffen . Ein Ereignis, das in den Tiefen Indiens oder Australiens stattgefunden hat, kann sich in Europa oder Amerika scharf und tief widerspiegeln und dort Konsequenzen von unzähliger Bedeutung für die Geschichte der Menschheit hervorrufen. Und vielleicht ist die Hauptsache -- die materielle, wirklich kontinuierliche Verbundenheit der Menschheit, ihre Kultur -- eine stetige und schnelle Vertiefung und Stärkung. Die Kommunikation wird intensiver und vielfältiger und konstanter.



Die Geschichte der vergangenen mentalen Kultur der Menschheit ist uns jetzt so wenig bekannt, dass wir uns jene Stadien der Vergangenheit nicht klar vorstellen können, die zur modernen Universalität des menschlichen Lebens geführt haben , die damit bedeckt ist -- seine Einheit -- egal wo in der Biosphäre sie leben. Jetzt können sie sich nirgendwo davor verstecken -- weder im Bereich des spirituellen Lebens noch im Bereich des Alltags. Und das Tempo der Konsolidierung der Universalität ist so groß, dass das Bewusstsein für lebende Generationen durchaus real ist. Es besteht kein Grund, darüber zu streiten.



Die Zunahme der Universalität und des Zusammenhalts aller menschlichen Gesellschaften wächst stetig und macht sich fast jedes Jahr in wenigen Jahren bemerkbar.



Das wissenschaftliche Denken ist für alle gleich, und die gleiche wissenschaftliche Technik, die allen gemeinsam ist, hat jetzt die gesamte Menschheit erfasst, sich in der gesamten Biosphäre verbreitet und sie in die Noosphäre verwandelt.



Dies ist ein neues Phänomen, das dem derzeit beobachteten Wachstum der Wissenschaft, der Explosion der wissenschaftlichen Kreativität, besondere Bedeutung beimisst.



65. Gleichzeitig ist anzumerken, dass sich Ende des 19. Jahrhunderts eine neue Situation in ihrem Wesen für die Wissenschaft verschärfte, die sich im 17.-19. Jahrhundert langsam zu entwickeln begann. Im zwanzigsten Jahrhundert. es hat unter dem Einfluss des intensiven Wachstums des wissenschaftlichen Denkens in erster Linie den angewandten Wert der Wissenschaft sowohl im Hostel als auch bei jedem Schritt zum Ausdruck gebracht: im privaten, im persönlichen und im kollektiven Leben.



Das Staatsleben in all seinen Erscheinungsformen wird in beispiellosem Maße vom wissenschaftlichen Denken erfasst. Die Wissenschaft fängt sie immer mehr ein.



Die Bedeutung der Wissenschaft im Leben, die, wie wir sehen werden, eng mit einer Veränderung der Biosphäre und ihrer Struktur mit dem Übergang zur Noosphäre verbunden ist, nimmt im gleichen, wenn nicht sogar schnellen Tempo zu wie das Wachstum neuer Bereiche wissenschaftlicher Erkenntnisse.



Und zusammen mit dieser Zunahme der Anwendung wissenschaftlicher Erkenntnisse auf das Leben, die Technologie, die Medizin und die staatliche Arbeit werden noch mehr geschaffen als in neuen Bereichen der Wissenschaft, neuen angewandten Wissenschaften, einer neuen Technik und neuen Anwendungen, die [schnell] bis zum Äußersten geschaffen und vorgebracht werden Bei neuen Problemen und Aufgaben der Technologie im weitesten Sinne werden öffentliche Mittel in bisher beispiellosen Beträgen für angewandte, aber im Wesentlichen wissenschaftliche Arbeiten ausgegeben.



Die Bedeutung der Wissenschaft und ihrer Probleme wächst in diesem Aspekt im Leben noch schneller als in neuen Wissensgebieten. Darüber hinaus sind es genau diese neuen Bereiche wissenschaftlicher Erkenntnisse, die den angewandten Wert der Wissenschaft , ihre Bedeutung in der Noosphäre , extrem erweitern und vertiefen .



 



Abteilung 2

über wissenschaftliche Wahrheiten



 



Kapitel 4



 



Die Position der Wissenschaft im modernen Staatssystem.



 





66. Eine solch lebenswichtige Bedeutung der Wissenschaft, die Teil des Bewusstseins der modernen Menschheit ist, entspricht historisch nicht, das heißt, Ausgehend von der Vergangenheit , der vorherrschenden realen Situation von ihr und ihrer Einschätzung im Leben.



Die Wissenschaft, an ihrem modernen sozialen und staatlichen Platz im Leben der Menschheit, reagiert nicht auf die Bedeutung, die sie bereits jetzt in sich hat. Dies betrifft sowohl die Position der Menschen in der Wissenschaft in der Gesellschaft, in der sie leben, als auch ihren Einfluss auf die staatlichen Ereignisse der Menschheit, ihre Teilnahme an der Staatsmacht und vor allem die Einschätzung der herrschenden Gruppen und bewussten Bürger -- der „öffentlichen Meinung“ des Landes -- der Realität die Kraft der Wissenschaft und von besonderer Bedeutung im Leben ihrer Affirmationen und Errungenschaften.



Der Mensch hat aus den neuen Grundlagen des modernen Staatslebens noch keine logischen Schlussfolgerungen gezogen. Die Zeit, die jetzt durchlebt wird -- die Zeit der radikalen und tiefen Demokratisierung des staatlichen Systems -, die Wahrheit, die noch nicht festgestellt wurde, aber die Formen dieses Systems bereits stark beeinflusst, muss die Position von Wissenschaft und Wissenschaftlern im staatlichen System zwangsläufig radikal verändern, aber noch nicht festgelegt haben. Die Bedeutung der Massen und ihrer Interessen, nicht nur für die politische, sondern auch für ihre soziale Reflexion, verändert die Interessen des Staates dramatisch. Das alte „Raison d'état“ und die Ziele der Existenz des Staates, die auf den historischen Interessen der Dynastien und verwandter Klassen und Gruppen beruhen, werden schnell durch ein neues Verständnis des Staates ersetzt. Der Wert der Dynastien vor unseren Augen verlagert sich schnell auf das Gebiet der Traditionen. Eine neue Idee kommt auf, es ist unvermeidlich, ob es zu früh oder zu spät ist, aber in staatlicher Echtzeit triumphiert sie -- die Idee einer staatlichen Vereinigung der Bemühungen der Menschheit . 80 Sie kann nur mit dem breiten Einsatz natürlicher Ressourcen zum Wohle des Staates stattfinden -- im Wesentlichen der Massen. Dies ist nur mit einer radikalen Veränderung der Position von Wissenschaft und Wissenschaftlern im staatlichen System möglich. Im Wesentlichen ist dies eine Zustandsmanifestation des Übergangs der Biosphäre zur Noosphäre. Wie bereits mehrfach erwähnt, ist dieser natürliche Prozess, der sich vor unseren Augen entwickelt hat, unvermeidlich und unvermeidlich. Und besteht ein Zweifel daran, dass der aktuelle Stand der Wissenschaft und der Wissenschaftler im Staat ein vorübergehendes Phänomen ist? Wir müssen mit seiner raschen Veränderung rechnen.



67. Aber jetzt ist das nicht. Dies wirkt sich besonders deutlich auf die Höhe der staatlichen Mittel aus, die für rein wissenschaftliche Zwecke ausgegeben werden, die keine militärischen -- aggressiven oder defensiven -- Werte aufweisen, die nicht mit Industrie, Landwirtschaft, Handel, Kommunikationslinien und den Interessen der Gesundheit und Bildung der Bevölkerung zusammenhängen. Bisher hat kein einziger Staat -- systematisch und systematisch -- erhebliche staatliche Mittel für die Lösung wichtiger wissenschaftlicher theoretischer Probleme, für Aufgaben, die weit vom modernen Leben entfernt sind, für seine Zukunft und für das Ausmaß staatlicher Bedürfnisse ausgegeben, die oft mit ihnen verwechselt werden. 81



Es ist noch nicht in das allgemeine Bewusstsein eingegangen, dass die Menschheit ihre Stärke und ihren Einfluss in der Biosphäre extrem ausbauen kann -- um durch bewusste staatliche wissenschaftliche Arbeit unermesslich bessere Lebensbedingungen für die nächsten Generationen zu schaffen. Eine solche neue Richtung staatlicher Aktivität, die Aufgabe des Staates als eine Form kraftvoller neuer wissenschaftlicher Forschung, scheint mir bereits in naher Zukunft eine unvermeidliche Konsequenz aus dem historischen Moment, den wir erleben, der Umwandlung der Biosphäre in die Noosphäre. Dies ist ein unvermeidlicher geologischer Prozess. Ich werde darauf zurückkommen.



Schon jetzt sehen wir seinen Ansatz. Tatsächlich wird die Wissenschaft durch die Phänomene des Lebens immer spontaner in Regierungsereignisse und zum Wohle der Sache eingeführt, aber ohne einen klar und bewusst durchdachten Plan nimmt sie eine zunehmend führende Position ein.



Dieser Zustand ist offensichtlich vorübergehend -- instabil, aus Sicht des politischen Systems und vor allem der Organisation der Noosphäre.



Aus eigener Initiative setzen Wissenschaftler aufgrund dieser Situation immer mehr öffentliche Mittel für das Wachstum wissenschaftlicher Erkenntnisse ein, die von Staatsmännern nicht bewusst dafür vorgesehen sind. Auf diese Weise erhalten sie die ständig wachsende Möglichkeit der Entwicklung der Wissenschaft dank der immer stärkeren Anerkennung ihrer angewandten Bedeutung für die Entwicklung von Technologien, die sonst nicht erreicht werden können. In dieser Hinsicht hat das 20. Jahrhundert eine große Verschiebung nach vorne gemacht, deren Bedeutung und Stärke noch nicht verstanden und offenbart wurde.



Die Anforderungen der Wissenschaft werden jedoch nicht formuliert, insbesondere werden ihre Unvermeidlichkeit und ihr Nutzen für die Menschheit nicht erkannt. Sie erhielten keinen Ausdruck in der sozialen und staatlichen Struktur. Es gibt keine ausgearbeiteten Staatsformulare, die eine schnelle und bequeme Lösung zwischenstaatlicher Probleme ermöglichen. Dies ist unweigerlich der größte Teil der Probleme bei der Schaffung der Noosphäre in Bezug auf Budget oder Finanzen.



In den Haushalten einzelner Staaten dieser Art können Fragen der Unterentwicklung aufgeworfen werden, die in staatlichen Zuweisungen für die Bedürfnisse von Akademien, in denen solche Zuweisungen bestehen, und in staatlichen Mitteln zur Unterstützung der wissenschaftlichen Arbeit, in denen solche Mittel vorhanden sind, aufgeworfen werden. Im Allgemeinen sind sie im Vergleich zu den bevorstehenden Herausforderungen vernachlässigbar. Dies gilt gleichermaßen für die kapitalistischen Länder und unseren sozialistischen Staat, wenn wir die Ausgaben in einer einzigen goldenen Währung ausdrücken.



68. Es scheint mir jedoch, dass wir uns jetzt an einem Wendepunkt befinden. Die staatliche Bedeutung der Wissenschaft als schöpferische Kraft als Hauptelement, die für die Schaffung des nationalen Reichtums als echte Möglichkeit ihrer raschen und massiven Schaffung unabdingbar ist, hat bereits das allgemeine Bewusstsein durchdrungen; Offensichtlich wird die Menschheit diesen Weg nicht verlassen können, da echte Wissenschaft die maximale Kraft ist, um die Noosphäre zu erschaffen.



Spontan wird als Manifestation des natürlichen Prozesses die Schaffung der Noosphäre in ihrer vollen Manifestation durchgeführt; ob früher oder später, es wird das Ziel der Staatspolitik und des Sozialsystems. Dies ist ein Prozess, der in den Tiefen der geologischen Zeit verwurzelt ist, wie aus dem evolutionären Prozess der Schaffung des Gehirns des Homo sapiens hervorgeht (Abschnitt 10). Ein kraftvoller Prozess, der während der geologischen Zeit in der Biosphäre stattfindet und eng mit den Energieerscheinungen der Evolution von Organismen zusammenhängt, kann in seinem Verlauf nicht durch Kräfte verschoben werden, die sich in historischen Zeitrahmen manifestieren.



Die alten Träume und Gefühle von Denkern, die in den meisten Fällen versuchten, sie in Form einer künstlerischen Nachbildung der Zukunft auszudrücken, Utopien -- um ihre manchmal exakten wissenschaftlichen Gedanken in die Form des wissenschaftlichen Sozialismus und Anarchismus zu gießen -, die immer von der Wissenschaft erfasst wurden, scheinen dem Realen, dem Bekannten nahe zu sein geringste Umsetzung.



Eine große eigentümliche Verschiebung findet in der sozialen Ideologie unserer Zeit statt, die unzureichende Aufmerksamkeit auf sich zieht und nicht ausreichend berücksichtigt wird, da die zuvor angegebene geologische Genese des wissenschaftlichen Denkens und seine durch den Evolutionsprozess geschaffene Grundlage unklar sind. Sie gibt nicht zu, dass das wissenschaftliche Denken enorm ist, ... 82



Seit dem Ende des 18. Jahrhunderts, als die Stärke der Kirchen in der europäisch-amerikanischen Zivilisation geschwächt wurde, wurde in der Ära der Philosophie der Aufklärung und später der Weg für ein freieres philosophisches Denken geöffnet. Der philosophische Strom setzte sich im wissenschaftlichen Denken durch. Einerseits war er nicht trennbar oder untrennbar mit der modernen Wissenschaft (Bildungsphilosophie, Formen des Leibianismus, Materialismus, Sensualismus, Kantianismus usw.) und andererseits mit verschiedenen Erscheinungsformen christlicher Philosophien und idealistischer Philosophie verbunden Systeme -- Berklianismus, deutscher Idealismus der Zeit nach Kant, mystische Quests, die zeitweise in einen scharfen Konflikt mit den Errungenschaften der Wissenschaft gerieten und sich selbst auf dem Gebiet der wissenschaftlichen Erkenntnisse nicht als mit ihnen verwandt betrachteten.



Illusion und Glaube an den Vorrang der Philosophie vor Religion und Wissenschaft sind klar und dominant geworden. Sie könnten tiefe Wurzeln in Bezug auf die Wissenschaft haben, da es oft schwierig ist, den obligatorischen Kern wissenschaftlicher Konstruktionen von dem Teil der Wissenschaft zu unterscheiden, der im Wesentlichen bedingt, vorübergehend und logisch äquivalent zu philosophischen oder religiösen Erklärungen auf dem Gebiet der wissenschaftlichen Erkenntnisse ist.



Dies könnte und kann jetzt vor allem deshalb geschehen, weil sich die Logik wissenschaftlicher Erkenntnisse, insbesondere der Wissenschaft, noch in einem vernachlässigten und kritisch schlecht konzipierten, nicht untersuchten Zustand befindet. 83



69. Unsere Aufmerksamkeit sollte jetzt natürlich nicht den künstlerischen, utopischen Bildern des zukünftigen Sozialsystems gewidmet werden, sondern nur der wissenschaftlichen Verarbeitung der sozialen Zukunft, zumindest in künstlerischer Form.



Hier können wir die anarchistischen Konstruktionen der Zukunft außer Acht lassen, die noch keine lebenswichtigen Manifestationen oder großen Köpfe gefunden haben, die ausreichend tief und auf neue Weise die wissenschaftlich zulässige und mögliche soziale Struktur offenbart haben, die mit dieser Form des sozialen Lebens verbunden ist und sich vom Sozialismus unterscheidet.



Beide Strömungen des sozialen Denkens schätzten die mächtige und unvermeidliche Kraft der Wissenschaft für das richtige soziale System richtig und gaben ihnen maximales Glück und vollständige Befriedigung der materiellen Grundbedürfnisse der Menschheit. In der wissenschaftlichen Arbeit der gesamten Menschheit wurde hier und da ein Mittel erkannt, das der Existenz des Menschen Sinn und Zweck verleihen und ihn vor unnötigem Leiden -- elementarem Leiden -- Hunger, Armut, Mord im Krieg, Krankheit -- hier auf der Erde bewahren kann. In diesem Sinne entspricht sowohl der Gedankenfluss, ob er aus wissenschaftlichen oder philosophischen Konstruktionen stammt, vollständig den Vorstellungen der Noosphäre als einer Phase der Geschichte unseres Planeten, die hier empirisch anhand wissenschaftlicher Daten bestätigt wird.



Der Glaube an die Kraft der Wissenschaft umfasste stetig den Gedanken der Menschen der Renaissance, wurde aber in den allerersten Aposteln des Sozialismus und Anarchismus gefunden -- von Saint-Simon [1760-1825] und Godwin [1756-1836] -- großen und tiefen kreativen Ausdrucksformen.



Diese Recherchen wurden Mitte des 19. Jahrhunderts in den Werken prominenter Wissenschaftler und Politiker -- Karl Marx [1818-1883] und Engels [1820-1895] -- und in den Folgen, die sie für den sozialstaatlichen Sieg des Sozialismus verursachten -- in Form des Bolschewismus in Russland und Teile Chinas und der Mongolei.



K. Marx, ein bedeutender wissenschaftlicher Forscher und unabhängig denkender Hegelianer, erkannte die enorme Bedeutung der Wissenschaft in einer Zukunft mit einem sozialistischen System. Gleichzeitig trennte er die Wissenschaft nicht von der Philosophie und glaubte, dass ihr korrekter Ausdruck sich nicht widersprechen könne. Es war damals -- vor fast 100 Jahren -- durchaus verständlich.



K. Marx und [F.] Engels lebten in der Philosophie, sie bestimmte ihr gesamtes bewusstes Leben, unter ihrem Einfluss wurde ihre geistige Erscheinung aufgebaut. Fast niemand in ihrer Zeit hätte vorhersehen können, dass sie, Zeitgenossen der sichtbaren beispiellosen Blüte und des Einflusses der idealistischen deutschen Philosophie, Zeitgenossen von Hegel, Schelling, Fichte, tatsächlich in der Zeit ihres tiefen Niedergangs und der Entstehung einer neuen Weltströmung lebten, viel tiefer und verwurzelt und in seiner Macht -- die Blütezeit der exakten Wissenschaften und Naturwissenschaften des 19. Jahrhunderts. In dieser Hinsicht rechtfertigte die Realität nicht die Ideen von [Marx] und Engels -- der Vorrang der Wissenschaft vor philosophischen Konstruktionen im 20. Jahrhundert kann jetzt keine Zweifel aufkommen lassen. In Wirklichkeit ist die wissenschaftliche Grundlage der Arbeit von Marx und Engels jedoch unabhängig von der Form -- einem Relikt der 1840er Jahre, in das sie -- die Menschen ihres Jahrhunderts -- sie gekleidet haben. Das Leben fordert seinen Tribut und es ist sinnlos, mit ihm zu streiten.



Tatsächlich wurde die Bedeutung der Wissenschaft als Grundlage der sozialen Reorganisation im sozialen System der Zukunft von Marx nicht aus philosophischen Ideen abgeleitet , sondern als Ergebnis einer wissenschaftlichen Analyse wirtschaftlicher Phänomene. Marx und Engels haben insofern Recht, als sie wirklich den Grundstein für den „wissenschaftlichen“ (nicht philosophischen) Sozialismus gelegt haben, weil sie durch eine gründliche wissenschaftliche Untersuchung wirtschaftlicher Phänomene, hauptsächlich K. Marx, die tiefste soziale Bedeutung des wissenschaftlichen Denkens offenbarten, die sich philosophisch intuitiv ergab frühere Suche nach „utopischem Sozialismus“.



In dieser Hinsicht steht das Konzept der Noosphäre, das sich aus biogeochemischen Konzepten ergibt, in völliger Harmonie mit der Grundlage der Idee und durchdringt den „wissenschaftlichen Sozialismus“. Ich werde in Zukunft darauf zurückkommen.



Die weit verbreitete Verbreitung sozialistischer Ideen und ihre Berichterstattung über die Machthaber, ihr Einfluss auf eine Reihe großer kapitalistischer Demokratien haben bequeme Formen geschaffen, um die Bedeutung wissenschaftlicher Arbeit als Schaffung nationalen Wohlstands anzuerkennen.



Neue Formen des Staatslebens werden realistisch geschaffen. Sie zeichnen sich durch ein zunehmendes Auftreten tiefer Elemente sozialistischer Staatsstrukturen aus. Die staatliche Planung wissenschaftlicher Arbeit für angewandte Regierungszwecke ist eine dieser Erscheinungsformen.



Aber mit dem Anstieg der Bedeutung der Wissenschaft im öffentlichen Leben ist am Ende unvermeidlich eine weitere Änderung des Staatsaufbaus unvermeidlich -- die Stärkung seiner demokratischen Grundlage. Denn die Wissenschaft ist in der Tat zutiefst demokratisch . Darin gibt es weder Griechen noch Juden. 84



Man kann kaum glauben, dass mit einem solchen Primat der Wissenschaft die Massen -- lange Zeit und überall -- die Bedeutung verlieren könnten, die sie in modernen Demokratien erlangen. Der Prozess der Demokratisierung der Staatsmacht -- mit der Universalität der Wissenschaft -- in der Noosphäre ist ein spontaner Prozess.



Natürlich kann der Prozess Generationen dauern. Eine, zwei Generationen in der Geschichte der Menschheit, die die Noosphäre als Ergebnis der geologischen Geschichte erschaffen -- des geologischen Moments.



70. Das Bewusstsein für die grundlegende Bedeutung der Wissenschaft für das „Gute der Menschheit“, ihre enorme Kraft für das Böse und das Gute, verändert langsam und stetig das wissenschaftliche Umfeld.



Bereits in den Utopien -- auch in den alten hellenistischen Utopien -- in Platon war die Staatsmacht in den Händen von Wissenschaftlern vertreten -- eine Idee, die sich mehr oder weniger deutlich in der überwältigenden Zahl von Utopien manifestierte. 85



Die tatsächlich beobachtete Zunahme der staatlichen Bedeutung von Wissenschaftlern spiegelt sich jedoch extrem stark in ihrer wissenschaftlichen Organisation wider und verändert die öffentliche Meinung des wissenschaftlichen Umfelds.



Die alte, charakteristische des XVI-XVII, teilweise XVIII. Jahrhunderts -- die Ära der kleinen Staaten Westeuropas und die Herrschaft einer einzigen wissenschaftlichen Sprache -- die nichtstaatliche Vereinigung von Wissenschaftlern und Schriftstellern, die in diesem Jahrhundert eine wichtige Rolle spielte, verlor ihre Bedeutung in den XIX-XX Jahrhunderten, als das Wachstum der Staaten und das Wachstum Die Wissenschaft verursachte ein Erwachen und einen Druck des nationalen und staatlichen Patriotismus. Wissenschaftler aller Länder haben eine große, oft führende Rolle in dieser Bewegung übernommen, da die wirklichen Interessen der Wissenschaft -- universell, verblasst oder vor dem Diktat des sozialen oder staatlichen Patriotismus auf den zweiten Platz zurückgegangen sind.



Gleichzeitig begannen sie im Zusammenhang mit staatlichen Bedürfnissen, die mit den Aufgaben wissenschaftlicher Erkenntnisse und einiger zwischenstaatlicher Vereinigungen (die nach dem Krieg von 1914 bis 1918 zum Völkerbund führten) einhergingen, im 19. Jahrhundert. Zahlreiche verschiedene internationale wissenschaftliche Vereinigungen auf globaler Ebene, die nach dem Krieg von 1914 bis 1918 schwer beschädigt wurden und weit davon entfernt sind, das neue Vorkriegsniveau zu erreichen.



71. Der Krieg von 1914-1918 und seine Konsequenzen -- das Wachstum faschistischer und sozialistischer Gefühle und Enthüllungen -- verursachten unter Wissenschaftlern tiefste Gefühle. Ein noch größerer Einfluss könnte durch das Ende dieses seit langem vorbereiteten Krieges und die Umarmung der gesamten Menschheit als Ganzes verursacht werden, die sich dank des Erfolgs der Wissenschaft auf dem Gebiet der menschlichen Kommunikation in einem beispiellosen Ausmaß und Tempo in einem kulturellen Austausch manifestiert. Der Krieg hatte tiefgreifende Konsequenzen, die unweigerlich die Position der Wissenschaft beeinflussten. Eine davon ist die tiefe moralische Erfahrung der akademischen Gemeinschaft der Welt, die mit den Schrecken und Grausamkeiten des größten Verbrechens verbunden ist, an dem Wissenschaftler aktiv teilgenommen haben -- es wurde von so vielen Wissenschaftlern, die daran teilnahmen, als Verbrechen anerkannt. Der moralische Druck des nationalen und staatlichen Patriotismus, der viele Gelehrte zu ihm führte, schwächte sich ab, und die moralische Seite der wissenschaftlichen Arbeit, die moralische Seite der Arbeit der Wissenschaftlerin, seine moralische Verantwortung für sie als freie Person in einem öffentlichen Umfeld, sah ihn zum ersten Mal als Alltag an Phänomen . 86



Die Frage nach der moralischen Seite der Wissenschaft, unabhängig von der religiösen, staatlichen oder philosophischen Manifestation der Moral, ist für den Wissenschaftler an der Reihe.



Er wird zu einer wirksamen Kraft, mit der er immer mehr rechnen muss. Es wird von einer langen, noch nicht geschriebenen, noch nicht einmal realisierten Geschichte vorbereitet. 87 Es steht völlig außerhalb der sogenannten wissenschaftlichen Moral, die zum Beispiel von der moralischen Laique des französischen Staates versucht wird, die eine soziale und philosophische Struktur darstellt, eine komplexe und entfernte Beziehung zur Wissenschaft hat, wenn wir ihren Inhalt analysieren, und sich völlig von der Manifestation des moralischen Elements in der Wissenschaft unterscheidet Arbeit, auf die ich an anderer Stelle in diesem Buch zurückkommen werde. 88 Der Name hier entspricht nicht der Realität. Dies ist eine Moral, die nicht mit der Wissenschaft verbunden ist, sondern mit der Philosophie und den wirklichen Erfordernissen der Staatspolitik, einem Versuch, die religiöse christliche Moral zu ersetzen. Es entstand als Ergebnis eines langen Kampfes für religiöse Toleranz, als Kompromiss der Ideen der französischen Revolution mit der wirklichen Druckkraft katholisch gesinnter Bürger. Dies ist ein Versuch der staatlichen Moral der Demokratie, der auf der Idee der Solidarität beruht, ein Versuch, der eindeutig keine Zukunft hat. Die Staatsmoral -- was auch immer sie sein mag -- ist in diesem Fall politisch-demokratisch, ebenso wenig kann eine so tiefe Bewegung befriedigen, die seit 1914 immer mehr in die Kreise der Wissenschaftler eingedrungen ist, so wie sie sie nicht beruhigen kann [und ] alte religiöse Ethik. Die vorübergehende Form eines demokratischen politischen Systems ist ein zu leichtes oberflächliches Phänomen, um die persönliche Moral eines modernen Gelehrten aufzubauen, der über die Zukunft nachdenkt. Der historische Prozess hat bereits das Konzept der Demokratie grundlegend verändert, indem er die Bedeutung der wirtschaftlichen Basis des staatlichen Systems deutlich gemacht und auch die Idee der staatlichen Vereinigung der gesamten Menschheit realistisch umgesetzt hat, um die Noosphäre zu schaffen und umzusetzen -- den Einsatz aller staatlichen Mittel und aller Macht der Wissenschaft zum Nutzen der gesamten Menschheit. Ein solches demokratisches Ideal des Wissenschaftlers ist extrem weit von der bürgerlichen Moral der französischen Radikalen entfernt.



72. Die Staatsmoral eines einzelnen Staates, auch wenn der Sozialist in seiner modernen Form den kritischen freien Gedanken eines modernen Wissenschaftlers und sein moralisches Bewusstsein nicht befriedigen kann, weil er nicht die notwendigen Formen dafür liefert.



Sobald sie in der wissenschaftlichen Gemeinschaft entstanden sind und ein unbefriedigtes Gefühl der moralischen Verantwortung für das Geschehen und die Überzeugung der Wissenschaftler von ihren tatsächlichen Handlungsmöglichkeiten haben, können sie nicht in der historischen Arena verschwinden, ohne zu versuchen, sie umzusetzen.



Diese moralische Unzufriedenheit des Wissenschaftlers nimmt ständig zu, seit 1914 hat alles zugenommen und ernährt sich kontinuierlich von den Ereignissen der Weltumwelt. Es ist mit den tiefsten Manifestationen der Persönlichkeit der Wissenschaftlerin verbunden, mit ihren Hauptmotivationen für die wissenschaftliche Arbeit.



Diese Impulse eines freien Menschen, der sich der umgebenden Persönlichkeit wissenschaftlich bewusst ist, sind tiefer als jede Form von Staatssystem, die durch wissenschaftliches Denken bei der Beobachtung des Verlaufs historischer Phänomene kritisch geprüft werden.



73. In der Vergangenheit in der Geschichte der Menschheit gab es in China einen Versuch, staatliche Moral zu schaffen -- aber er wurde isoliert von anderen geschaffen, wenn auch in einem großen kulturellen Zentrum -, als sich die geologische Kraft des wissenschaftlichen Denkens kaum manifestierte und es kein Bewusstsein dafür gab.



Vor mehr als 2000-2200 Jahren wurde die Idee, prominente Personen im Staat auszuwählen, beim Aufbau der chinesischen Staaten durch weit verbreitete Wettbewerbe populärer Schulkinder verwirklicht, um Staatsgelehrte zu schaffen, in deren Händen die Staatsmacht übertragen werden sollte. Eine solche Wahl der Staatsleute in der Idee dauerte viele Jahrhunderte, ist mit dem Namen Konfuzius verbunden und hat wirklich ihren Ausdruck im Leben bekommen.



Aber die Wissenschaft, die zur gleichen Zeit verstanden wurde, war sehr weit von der wirklichen Wissenschaft dieser Zeit entfernt. Es war höchstwahrscheinlich ein Stipendium, eine großartige Kultur auf einer tiefen moralischen Basis, es gab den Wissenschaftlern, die an der Spitze der Regierung standen, keine neue wirkliche Macht in die Hände. Als China im 16. und 17. Jahrhundert mit der rasch aufkommenden neuen westeuropäischen Wissenschaft konfrontiert war, versuchte er einige Zeit, sie in den Rahmen seiner traditionellen Wissenschaft einzuführen. Dies endete jedoch, wie ich bereits angedeutet habe (§ 60), zu Beginn des 18. Jahrhunderts. ein völliger Zusammenbruch, und natürlich ist dieses eigentümliche historische Phänomen weit entfernt von dem, was das weltweite Wissenschaftlerteam jetzt sieht.



Im zwanzigsten Jahrhundert, mit dem Zusammenbruch des alten China, brachen die Überreste des alten Konfuzianismus zusammen. Ein einziger wissenschaftlicher Gedanke, ein einziges Wissenschaftlerteam und eine einzige wissenschaftliche Methodik traten in das Leben der chinesischen Völker ein und übten schnell ihren Einfluss auf ihre wissenschaftliche Arbeit aus. Es besteht kaum ein Zweifel daran, dass die überlebenden Jahrtausende, die am Leben geblieben sind und mit einer einzigen Weltwissenschaft verschmolzen sind, die Weisheit und Moral des Konfuzianismus tiefgreifende Auswirkungen auf das weltwissenschaftliche Denken haben werden, da sie auf diese Weise in einen Kreis neuer Gesichter einer tieferen wissenschaftlichen Tradition als die westeuropäische Zivilisation eintreten . Dies sollte sich vor allem in einem Verständnis der grundlegenden wissenschaftlichen Konzepte manifestieren, die an philosophische Konzepte grenzen.



74. Der Krieg von 1914-1918 [gg.] In den XIX-XX Jahrhunderten stark geschwächt. internationale Organisationen von Wissenschaftlern. In einigen Fällen haben sie ihren vollständig internationalen (in Form eines zwischenstaatlichen) Charakters noch nicht wiederhergestellt. Die tiefe Zwietracht zwischen Faschismus und Demokratie -- der Sozialismus im gegenwärtigen historischen Moment und eine scharfe Verschärfung der staatlichen Interessen, die in mehreren Ländern auf Stärke und letztendlich auf einen neuen Krieg beruht, um bessere Lebensbedingungen für die Bevölkerung (einschließlich Länder wie z Deutschland, Italien, Japan -- mächtige Zentren wissenschaftlicher Arbeit, reich an organisierten wissenschaftlichen Apparaten) lassen hier keine rasche ernsthafte Verbesserung zu erwarten.



Es sollte angemerkt werden, dass neue Formen der wissenschaftlichen Brüderlichkeit -- nichtstaatlich organisierte Formen des wissenschaftlichen Weltumfelds -- allmählich gesucht und sich abzeichnen .



Diese Formulare sind flexibler, individueller und befinden sich erst im Stadium eines Trends -- formlose und noch nicht etablierte Suchanfragen.



In den letzten Jahren, den 1930er Jahren, erhielten sie jedoch die ersten Grundlagen der Organisation und zeigten sich deutlich für alle, zum Beispiel im „Brain Trust“ von Roosevelts Beratern, das große Aufmerksamkeit erregt hatte und die staatliche Politik der Vereinigten Staaten beeinflusst hatte und beeinflusst hat. Ich musste wirklich mit ihm rechnen.



Dies ist offensichtlich eine Form der wissenschaftlichen Organisation -- inländisch, die eine große Zukunft hat. Noch früher -- theoretisch, aber nicht in der Ausführung -- und in einer bürokratischeren Form in der Struktur derselben Ordnung -- wurde die staatliche Planungskommission in unserem Land eingerichtet.



Die Idee eines „wissenschaftlichen Think Tanks“ der Menschheit wird vom Leben vertreten -- der Slogan findet Echos.



Es wurde auch in öffentlichen Versammlungen anlässlich des 300-jährigen Jubiläums der Harvard University in Boston und Cambridge im Jahr 1936 erwähnt. Seine Hauptbedeutung lag jedoch in dieser persönlichen Kommunikation auf dieser Grundlage, die hier zwischen großen wissenschaftlichen Forschern aller Länder stattfand. dort versammelt. Der Gedanke war geboren.



Darüber hinaus scheint es mir möglich, dass diese Idee eine große Zukunft hat.



Es ist schwer zu sagen, wie es in naher Zukunft aussehen wird. Aber sie verlässt die historische Arena, die sie betreten hat, kaum vorübergehend. Ihre Wurzeln sind eng mit dem Verlauf des wissenschaftlichen Denkens verbunden und werden kontinuierlich damit gespeist.



Kapitel 5



 



Die Unveränderlichkeit und Universalität korrekt abgeleiteter wissenschaftlicher Wahrheiten für jeden Menschen, für jede Philosophie und für jede Religion. Der obligatorische Charakter der Errungenschaften der Wissenschaft auf ihrem Gebiet ist ihr Hauptunterschied zu Philosophie und Religion, deren Schlussfolgerungen möglicherweise nicht so verbindlich sind.



 





75. Es gibt ein grundlegendes Phänomen, das das wissenschaftliche Denken definiert und wissenschaftliche Ergebnisse und wissenschaftliche Schlussfolgerungen klar und einfach von Aussagen der Philosophie und Religion unterscheidet -- dies ist die allgemeine Verbindlichkeit und Unbestreitbarkeit korrekt gezogener wissenschaftlicher Schlussfolgerungen, wissenschaftlicher Aussagen, Konzepte, Schlussfolgerungen . Wissenschaftliche, logisch korrekt durchgeführte Handlungen haben eine solche Kraft nur, weil die Wissenschaft ihre eigene bestimmte Struktur hat und eine Region von Fakten und Verallgemeinerungen , wissenschaftlichen, empirisch ermittelten Fakten und empirisch gewonnenen Verallgemeinerungen darin vorhanden ist, die im Wesentlichen nicht wirklich bestritten werden können. Solche Tatsachen und Verallgemeinerungen können, wenn sie zuweilen durch Philosophie, Religion, Lebenserfahrung oder sozialen gesunden Menschenverstand und Tradition geschaffen werden, nicht als solche bewiesen werden. Weder Philosophie noch Religion noch gesunder Menschenverstand können sie mit dem Grad an Zuverlässigkeit etablieren, den die Wissenschaft bietet. Ihre Fakten, ihre Schlussfolgerungen und Schlussfolgerungen sollten alle auf der Grundlage wissenschaftlicher Erkenntnisse überprüft werden.



Diese allgemeine Verbindlichkeit eines Teils der Errungenschaften der Wissenschaft unterscheidet sich stark von der, die für Axiome, selbstverständliche Darstellungen, die den grundlegenden geometrischen, logischen und physikalischen Darstellungen zugrunde liegen, zugelassen werden muss. Vielleicht ist der Unterschied nicht im Wesentlichen, sondern auf die Tatsache zurückzuführen, dass die Axiome über viele Generationen, über Jahrtausende hinweg so offensichtlich geworden sind, dass ein logischer Prozess eine Person von ihrer Richtigkeit überzeugt. Es ist jedoch möglich, dass dies auf die Struktur unseres Geistes zurückzuführen ist, d.h. am Ende das Gehirn. Es ist möglich, dass sich auf diese Weise die Noosphäre im Denkprozess manifestiert. 89



Für die Aufgaben, die ich in diesem Buch gestellt habe, muss ich mich nicht mit diesem Thema befassen, das wissenschaftlich und philosophisch unzureichend ist und dem es an Lösungen mangelt, auf denen die wissenschaftliche Arbeit fest basieren könnte. Im Gegensatz zu Axiomen sind allgemein verbindliche wissenschaftliche Wahrheiten nicht selbstverständlich und müssen in jedem Fall kontinuierlich durch Vergleich mit der Realität überprüft werden. Dieser echte Test ist die tägliche Hauptarbeit des Wissenschaftlers.



Eine solche Universalität und Unbestreitbarkeit ihrer Aussagen und Schlussfolgerungen fehlt nicht nur in allen anderen spirituellen Konstruktionen der Menschheit -- in der Philosophie, in der Religion, in der Kunst, im sozialen Alltagsumfeld des gesunden Menschenverstandes und in der jahrhundertealten Tradition. Darüber hinaus können wir nicht entscheiden, wie wahr und richtig die Aussagen selbst der grundlegendsten religiösen und philosophischen Vorstellungen über einen Menschen und seine reale Welt sind. Ganz zu schweigen von poetischen und sozialen Verständnissen, bei denen die Willkür und Individualität von Aussagen in all ihren jahrhundertealten Offenbarungen keinen Zweifel aufkommen lässt. Gleichzeitig wissen wir, dass das bekannte -- manchmal große Teil der Wahrheit -- wissenschaftlich korrekte Verständnis der Realität in ihnen steckt. Es kann sich in einer Person tief und vollständig manifestieren, im Geist nicht tief umschlossener künstlerischer bunter Bilder, musikalischer Harmonie, auf der moralischen Ebene des Verhaltens einer Person.



Dies sind alles Bereiche tiefer Manifestation der Persönlichkeit -- Bereiche des Glaubens, der Intuition, des Charakters, des Temperaments.



Es gibt viele Religionen sowie Philosophien, poetische und künstlerische Ausdrücke, gesunden Menschenverstand, Traditionen, ethische Standards, vielleicht ebenso im Grenzbereich, unter Berücksichtigung der Schattierungen wie einzelne Individuen und unter Berücksichtigung des Allgemeinen -- wie viele ihrer Typen. Aber die Wissenschaft ist eine und eine, denn obwohl die Zahl der Wissenschaften ständig wächst, werden neue geschaffen -- sie sind alle in einer einzigen wissenschaftlichen Struktur verbunden und können sich nicht logisch widersprechen.



Diese Einheit der Wissenschaft und die Vielfalt der Vorstellungen über die Realität der Philosophien und Religionen einerseits und andererseits ist unbestreitbar und bindend, im Wesentlichen logisch, unbestreitbar, ein wesentlicher Teil des Inhalts wissenschaftlicher Erkenntnisse und letztendlich -- jeder wissenschaftliche Fortschritt -- unterscheidet die Wissenschaft scharf von der angrenzenden Eindringen in das Denken von Wissenschaftlern, philosophische und religiöse Affirmationen.



Wenn das unbestreitbar wissenschaftliche Material wächst, nimmt die Kraft der Wissenschaft und ihre geologische Wirkung in der umgebenden Biosphäre zu -- auch die Position der Wissenschaft im Leben der Menschheit vertieft sich und ihr vitaler Einfluss wächst schnell.



76. Es ist leicht sicherzustellen, dass die unbestreitbare Kraft der Wissenschaft nur mit einem relativ kleinen Teil der wissenschaftlichen Arbeit verbunden ist, der als Hauptstruktur der wissenschaftlichen Erkenntnisse betrachtet werden sollte . Wie wir sehen werden, hatte es eine komplexe Geschichte, die gleichzeitig entwickelt wurde. Dieser Teil des wissenschaftlichen Wissens umfasst Logik, Mathematik und die Erfassung von Fakten, die als wissenschaftlicher Apparat bezeichnet werden können . Wissenschaft ist ein dynamisches Phänomen, das sich ständig verändert und vertieft, und seine unbestreitbare Kraft manifestiert sich mit voller Klarheit nur in jenen Epochen, in denen diese drei Hauptmanifestationen wissenschaftlicher Erkenntnisse gleichzeitig wachsen und sich vertiefen.



Mathematik und Logik haben bei richtiger Anwendung immer ihre Bedeutung und Unbestreitbarkeit erkannt, aber der wissenschaftliche Apparat hat Denkern und sogar Wissenschaftlern selbst noch nicht genügend Aufmerksamkeit geschenkt, die ihn nicht als eines der Hauptergebnisse ihrer Arbeit betrachteten, sondern als Hypothesen und Theorien sind Erklärungen, die mehr oder weniger logisch damit zusammenhängen.



Im Alltag, in dem die Interessen des Alltags, der Gesellschaft, der Philosophie oder der Religion vorherrschen, ist das Bewusstsein für die außergewöhnliche Bedeutung wissenschaftlich fundierter Tatsachen noch unterentwickelt. Der wissenschaftliche Apparat wird vollständig von der Verbesserung und Vertiefung der Systematisierungs- und Forschungsmethodik durchdrungen und aufrechterhalten. Auf diese Weise erfasst und erfasst die Wissenschaft jedes Jahr Millionen neuer Fakten für die Zukunft mit immer größerem Tempo und erstellt auf dieser Grundlage viele große und kleine empirische Verallgemeinerungen. Weder wissenschaftliche Theorien noch wissenschaftliche Hypothesen werden trotz ihrer Bedeutung für die aktuelle wissenschaftliche Arbeit in diesen grundlegenden und entscheidenden Teil der wissenschaftlichen Erkenntnisse einbezogen.



Es muss jedoch daran erinnert werden, dass ohne wissenschaftliche Hypothesen empirische Verallgemeinerungen und Kritik an Tatsachen nicht präzise gestellt werden können und dass ein wesentlicher Teil der Tatsachen selbst, der wissenschaftliche Apparat selbst, dank wissenschaftlicher Theorien und wissenschaftlicher Hypothesen geschaffen wird. Der wissenschaftliche Apparat sollte immer kritisch berücksichtigt werden, und jeder Wissenschaftler, der die Fakten bewertet und daraus empirische Verallgemeinerungen macht, sollte die Möglichkeit von Fehlern berücksichtigen, da die Manifestation [Einfluss] bei der Feststellung der Fakten wissenschaftlicher Theorien und wissenschaftlicher Hypothesen sie [Fakten] verzerren kann.



Die Hauptbedeutung von Hypothesen und Theorien ist offensichtlich. Trotz des enormen Einflusses, den sie auf das wissenschaftliche Denken und die wissenschaftliche Arbeit des Augenblicks ausüben, sind sie immer vergänglicher als der unbestreitbare Teil der Wissenschaft, der wissenschaftliche Wahrheit ist und Jahrhunderte und Jahrtausende durchlaufen hat. Vielleicht gibt es sogar eine Schöpfung eines wissenschaftlichen Geistes, der darüber hinausgeht historische Zeit -- unerschütterlich in der geologischen Zeit -- „ewig“.



Das unbestreitbare ewige Grundgerüst der Wissenschaft, das weit davon entfernt ist, ihren gesamten Inhalt abzudecken, sondern die schnell wachsende Datenmenge, die Menge an Wissen, abdeckt, besteht daher aus 1) Logik, 2) Mathematik und 3) dem wissenschaftlichen Apparat von Tatsachen und Verallgemeinerungen, der infolgedessen kontinuierlich wächst wissenschaftliche Arbeit im geometrischen Verlauf, wissenschaftliche Fakten, deren Anzahl jetzt viel höher ist als unsere numerischen Darstellungen -- etwa 10 10 , wenn nicht 10 20 . Es gibt so viele wie „wie viele Sandkörner im Meer“. Diese Tatsachen werden jedoch so zusammengefasst, dass Wissenschaftler zusammen -- die Wissenschaft dieser Zeit -- sie leicht und bequem nutzen können. Unzählige empirische Verallgemeinerungen bauen logisch und manchmal mathematisch auf diesem wissenschaftlichen Apparat auf.



Dieser Hauptteil der Wissenschaft, der sowohl in der Philosophie als auch in der religiösen Konstruktion der Welt fehlt, ist umgeben von wissenschaftlichen Hypothesen, Theorien, Leitideen, manchmal Konzepten, deren unbestreitbare Zuverlässigkeit bestritten werden kann.



Diese Position der Wissenschaft in der sozialen Struktur der Menschheit versetzt Wissenschaft, wissenschaftliches Denken und Arbeiten in eine ganz besondere Position und bestimmt ihre besondere Bedeutung im Medium der Manifestation des Geistes -- in der Noosphäre.



77. Diese Vorstellung von der besonderen Situation wissenschaftlicher Wahrheiten, ihrer Bindung wird immer noch nicht allgemein akzeptiert. Darüber hinaus muss man mit der entgegengesetzten Darstellung rechnen. Der Begriff der Universalität wissenschaftlicher Wahrheiten ist eine neue Errungenschaft in der Geschichte der Kultur, die gerade ihren Weg im Bewusstsein der Menschheit ebnet.



Religiöse Überzeugungen, die auf dem Glauben an die Besonderheit religiöser Wahrheiten beruhen -- insbesondere Überzeugungen über sie als Offenbarungen des Göttlichen, die nicht bestritten werden können und als bedingungslose Wahrheit für alle -- Gläubige und Ungläubige -- wahrgenommen werden sollten -- sind bindend und können keinen Zweifel aufkommen lassen , -- sind weit davon entfernt, überlebt zu werden, und erst nach großem und langem Leid und einem jahrhundertelangen Kampf gelang es einem bedeutenden Teil der westeuropäischen und amerikanischen Staaten, einen Kompromiss zu erzielen. Es wurde die Gelegenheit geschaffen, die ideologisch nicht eingefrorenen und formal vorherrschenden religiösen Aussagen des Glaubens an christliche, jüdische, muslimische und andere Kirchen, die echte Macht haben, tatsächlich zu ignorieren. Die bekannte -- unzureichende -- Freiheit des wissenschaftlichen Denkens ist jedoch gewährleistet.



Seit dem Ende des 18. Jahrhunderts hat die Idee der universell verbindlichen wissenschaftlichen Wahrheiten, die unter sozialen Bedingungen außergewöhnlich sind, mit Schwankungen in die eine oder andere Richtung immer größere reale Macht erlangt, kann aber nicht als gesichert angesehen werden, selbst wenn es um einfache Toleranz geht -- die Anerkennung ihrer Stärke zusammen mit Religion und Philosophie. Der Kampf ist noch nicht vorbei. Für die überwältigende Masse der Menschheit ist die religiöse Wahrheit höher und überzeugender als die wissenschaftliche, und letztere muss nachgeben, wenn zwischen ihnen ein Widerspruch besteht. Aber von Natur aus kann es nicht zugeben.



Der Kampf insgesamt spricht eindeutig für wissenschaftliche Erkenntnisse. Im zwanzigsten Jahrhundert. Ein Siegeszug des wissenschaftlichen Denkens -- in Schwächung und Freiheit von religiösen Beschränkungen -- deckt die gesamte Menschheit ab. Der Osten Europas, ganz Asien und Afrika, Südamerika und die Ozeaninseln sind davon bedeckt. Mit der Einbeziehung des großen Zentrums der jahrtausendealten Kultur -- Indien -- in die moderne wissenschaftliche Arbeit seit der Wiederbelebung nach vielen Jahrhunderten der Stagnation im 20. Jahrhundert. Durch ihr freies wissenschaftliches und philosophisches Denken gewann die wissenschaftliche Organisation neue Kraft -- Wissenschaftler -- für die das religiöse Bewusstsein über Generationen hinweg völlige Freiheit der wissenschaftlichen Suche gelassen hat . Es scheint mir, dass es für die Zukunft notwendig ist, diese neue Stärkung der wissenschaftlichen Arbeit der Menschheit zu berücksichtigen.



78. In letzter Zeit haben wir in diesem Bereich eine Verschlechterung erlebt, weil anstelle des zunehmend schwächeren religiösen Pathos des Glaubens an die Unveränderlichkeit und in der Zukunft der universellen Einheit des religiösen Verständnisses von Mensch und Wirklichkeit vorübergehende soziale und staatliche Ideen handeln und sich brutal vor denen schützen, die es sein könnten Zweifel an ihrer Unveränderlichkeit. Es erscheint eine im Wesentlichen neue soziale Lebensform, die sich auch ideologisch stark ungünstig auf die Freiheit der wissenschaftlichen Suche auswirkt.



Dies ist im Wesentlichen auf die Nichtanerkennung der Gedanken- und Forschungsfreiheit zurückzuführen, die in den demokratischen Staaten Europas und Nordamerikas des 20. Jahrhunderts herrscht. Es wurde größtenteils im Zusammenhang mit und während des Kampfes für die Religionsfreiheit erhalten, nachdem eine einzige katholische Kirche andere Gläubige nicht zerstören konnte. In dem schwierigen politischen und sozialen Umfeld hat der kirchliche Druck seit Jahrhunderten nachgelassen, aber die Staatsmacht hat die gleichen Druckmittel eingesetzt, um die Freiheit des wissenschaftlichen Denkens zu bekämpfen und ihre sozialen und politischen Gegner zu bekämpfen. Im Wesentlichen sollte wissenschaftliches Denken mit dem richtigen Verlauf der Staatsarbeit nicht mit der Staatsmacht kollidieren, denn es ist die Hauptquelle des nationalen Reichtums, die Grundlage der Staatsmacht . Der Kampf dagegen ist ein schmerzhaftes, vorübergehendes Phänomen im staatlichen System.



Die Staatsmacht kämpfte auch gegen religiöse Überzeugungen, in Wirklichkeit nicht mit ihrer Ideologie, sondern mit ihrer aus ihrer Sicht schädlichen Identifikation in diesem gesellschaftspolitischen Umfeld, das der Hauptuntergrund der Staatsmacht war. Klassen-, Partei- und persönliche Interessen und die Aufrechterhaltung einer ungleichmäßigen Verteilung des nationalen Reichtums, die nicht das erfolgreiche Leben aller gewährleistet, bestimmten die öffentliche Ordnung. Sie bestimmten auch die staatliche Politik in Bezug auf die Glaubensfreiheit und bis zu einem gewissen Grad die damit verbundene Freiheit der wissenschaftlichen Kreativität.



79. Nur in wenigen Ländern stellte sich heraus, dass die Möglichkeit einer freien wissenschaftlichen Forschung ziemlich vollständig, aber immer noch unvollständig war. Es wird am vollständigsten in den Ländern der skandinavischen, großen angelsächsischen Demokratien (aber zum Beispiel nicht im britischen Empire in Indien) und in Frankreich, vielleicht in China, erreicht.



In unserem Land war es nie, nicht jetzt.



In einer Reihe von Staaten nimmt diese staatliche Einschränkung des freien wissenschaftlichen Denkens explizit oder implizit den Charakter einer Staatsreligion an.



Es ist die Staatsreligion Japans in der Lehre des Kaisers als Nachkomme der Sonne. Der Staat kämpft wie mit einem Verbrechen und erkennt die Richtigkeit dieses Dogmas nicht an, mit der obligatorischen Erziehung aller Kinder in allen Schulen.



Dies zeigt sich in den faschistischen Ländern weniger ideologisch -- in Deutschland und Italien sowie in unserem sozialistischen Staat. Im zaristischen Russland gab es kontinuierliche Versuche, eine Staatsreligion nach ihren Grundsätzen zu schaffen -- eine politische Religion, wie S. S. Uvarov vor hundert Jahren sagte. 90 Mit der vollständigen Unterwerfung des Klerus unter den Staat war die Religion lebhafter politischer Natur und stand in direktem Widerspruch zur öffentlichen Meinung, die sich nicht frei ausdrücken konnte. 



Jetzt befinden wir uns in einer Übergangsphase, in der ein großer Teil der Menschheit nicht in der Lage ist, richtig zu beurteilen, was geschieht, und das Leben gegen die Grundbedingungen für die Schaffung der Noosphäre verstößt.



Offensichtlich ist dies ein vorübergehendes Phänomen.



80. Die Staatsmacht geht im Wesentlichen in diesem Kampf gegen ihre Interessen auf dem Weg, nicht die Stärke des Staates aufrechtzuerhalten, sondern ein bestimmtes soziales System aufrechtzuerhalten, und dieser Kampf ist Ausdruck tieferer Merkmale als diejenigen, die in der Wirtschaftsstruktur der Gesellschaft zu finden sind. Sie sind sowohl für kapitalistische als auch für sozialistische (und anarchistische?) Staatsformationen charakteristisch.



Es war real, dass das, was wirklich im Zentrum des jahrhundertealten Kampfes mit der Staatsmacht um Gedankenfreiheit stand, als im Wesentlichen ein Kampf um den Schutz der bestehenden sozialen und wirtschaftlichen Verteilung des nationalen Reichtums, um ein staatlich anerkanntes religiöses Verständnis des Lebens und um die Interessen der Machthaber im Gange war.



Unter solchen Bedingungen sind die Wurzeln des anhaltenden staatlich-sozialen Drucks auf die Freiheit der wissenschaftlichen Forschung weniger tief, nachdem sie in den Hintergrund ihrer ideologischen Rechtfertigung -- der religiösen Grundlagen der Staatspolitik -- getreten sind. Sie sind realer und deutlich vorübergehender.



Der gesellschaftspolitische Druck auf die Freiheit der wissenschaftlichen Forschung kann das wissenschaftliche Denken und die wissenschaftliche Kreativität nicht lange aufhalten, da das moderne gesellschaftspolitische Staatsleben in seinen Grundlagen immer tiefer von den Errungenschaften der Wissenschaft erfasst wird und in seiner Stärke immer mehr von ihr abhängt.



Eine solche Zustandsbildung in der Noosphäre ist unweigerlich fragil: Die Wissenschaft darin wird letztendlich tatsächlich ein entscheidender Faktor sein.



Dies muss sich zwangsläufig in der Staatsstruktur manifestieren. Die Interessen wissenschaftlicher Erkenntnisse sollten in der aktuellen öffentlichen Ordnung zum Ausdruck kommen. Die Freiheit der wissenschaftlichen Forschung ist die Hauptvoraussetzung für den maximalen Arbeitserfolg. Sie toleriert keine Einschränkungen. Ein Zustand, der ihm maximalen Spielraum bietet, minimale Barrieren setzt, maximale Stärke in der Noosphäre erreicht, ist darin am stabilsten. Grenzen werden durch eine neue Ethik gesetzt, wie wir später sehen werden, mit wissenschaftlichem Fortschritt.



Dies ist unvermeidlich, da es mit dem spontanen natürlichen Prozess der vollständigen Umwandlung der Biosphäre in die Noosphäre verbunden ist, der unvermeidlich kommt. Am Ende dieser Transformation kann ihre Struktur in der Noosphäre im Wesentlichen kein Hindernis für die Freiheit der wissenschaftlichen Forschung sein.



81. Die Korrelation der Wissenschaft mit den philosophischen Lehren, die dem politischen System zugrunde liegen, das die Freiheit der wissenschaftlichen Forschung nicht anerkennt, ist komplizierter. Die eine oder andere Philosophie ersetzt die ausgehende religiöse Ideologie.



Die Position der Philosophie in der Struktur der menschlichen Kultur ist sehr eigenartig. Es ist untrennbar und in vielerlei Hinsicht mit dem religiösen, gesellschaftspolitischen, persönlichen und wissenschaftlichen Leben verbunden. Es nimmt eine sich verändernde Position in Bezug auf die Religion ein, und es gibt eine enorme Bandbreite an Verständnis und Ideen, die ständig wächst. Eine große Anzahl von Problemen, die sie betreffen oder die mit ihr zusammenhängen können, die ständig zunehmen, ein kontinuierlicher Übergang von ihr zu allen Fragen des Alltags und des Staatslebens, des gesunden Menschenverstandes und der Moral ermöglichen es jedem, der darüber nachdenkt und nachdenkt, was passiert. Die Vorbereitung darauf sowie auf die Religion ist jede sich selbst reflektierende Persönlichkeit -- ihre Lebensweise und ihr soziales Leben.



Sie können ein Philosoph sein, und ein guter Philosoph, ohne wissenschaftliche Vorbereitung, müssen Sie nur tief und unabhängig über alles um Sie herum nachdenken und bewusst in Ihrem eigenen Rahmen leben. In der Geschichte der Philosophie sehen wir im übertragenen Sinne ständig Menschen „vom Pflug“, die sich ohne weitere Vorbereitung als Philosophen herausstellen. In sich selbst, in der Reflexion über sich selbst , in der Vertiefung in sich selbst -- auch außerhalb der Ereignisse der äußeren Persönlichkeit der Welt -- kann ein Mensch die tiefste philosophische Arbeit verrichten und sich großen philosophischen Errungenschaften nähern.



Gleichzeitig wird Philosophie gelehrt, und tatsächlich kann und sollte Philosophie studiert werden. Die Werke großer Philosophen sind die größten Denkmäler für das Verständnis des Lebens und des Verständnisses der Welt, indem sie Menschen in verschiedenen Epochen der Menschheitsgeschichte tief denken. Dies sind lebendige menschliche Dokumente von größter Wichtigkeit und Lehre, aber sie können in ihren Schlussfolgerungen und Schlussfolgerungen nicht universell bindend sein, da sie Folgendes widerspiegeln: 1) Erstens die menschliche Persönlichkeit in ihrer tiefsten Reflexion über die Welt, und es kann unendlich viele Persönlichkeiten geben -- es gibt keine zwei Identitäten; und reflektieren in 2 entwickelt ihr Verständnis der Realität; solche Verständnisse mögen im Wesentlichen nicht so viele sein; Sie können zu einer kleinen Anzahl von Grundtypen zusammengebaut werden. Aber es kann keinen unter ihnen geben, wahrer als alle anderen. Es gibt kein klares und eindeutiges Kriterium dafür und kann es auch nicht geben.



Diese Sicht der Philosophie, ihre Position im kulturellen Leben, ist nicht dominant. Die scharfe Trennung von Philosophie und Wissenschaft, die hier durchgeführt wird, wird nicht allgemein akzeptiert und kann auf Einwände stoßen. Der Hauptpunkt ist jedoch, dass viele verschiedene Philosophien gleichzeitig existieren und dass die Wahl zwischen ihnen auf der Grundlage der Wahrheit eines von ihnen nicht logisch getroffen werden kann -- es gibt eine Tatsache, gegen die es keinen Grund gibt, zu argumentieren. Sie können nur glauben, dass es nicht immer so sein wird, obwohl es immer so war.



Für meine Aufgaben reicht es aus, sich auf eine solche Tatsache zu stützen -- in ihrem Ausdruck -- und in unserer Ära der Allgegenwart des menschlichen Lebens und der unbestreitbaren Universalität wissenschaftlicher Tatsachen und wissenschaftlicher Wahrheiten, wissenschaftlich korrekt festgelegt, notwendig und korrekt, wird sie die Philosophie scharf von einer einzigen Wissenschaft trennen, ohne immer weiter zu gehen ob es so sein wird oder nicht.



82. Daraus ergibt sich, dass Philosophie studiert werden muss, aber nur mit Hilfe des Lernens ist es unmöglich, Philosoph zu werden. Das Hauptmerkmal der Philosophie ist das innere aufrichtige Werk der Reflexion, das auf die uns umgebende Realität als Ganzes oder ihre einzelnen Teile abzielt.



Die Grundlage der Philosophie ist das Primat des menschlichen Geistes . Philosophie ist immer rational. Denken und tiefes Eindringen in den Denkapparat -- den Geist -- gehen unweigerlich in die philosophische Arbeit ein. Für die Philosophie ist die Vernunft der oberste Richter; Die Gesetze des Geistes bestimmen seine Urteile. Dies ist der höchste Beginn des Wissens. Für einen Naturforscher ist die Vernunft eine vorübergehende Manifestation der höheren Lebensformen des Homo sapiens in der Biosphäre, die ihn in eine Noosphäre verwandelt: Sie ist und kann nicht die ultimative, maximale Form der Manifestation des Lebens sein. Das menschliche Gehirn kann ihnen nicht erscheinen. Der Mensch ist nicht die „Krone der Schöpfung“. Eine philosophische Analyse des Geistes kann kaum eine entfernte Vorstellung von der möglichen Kraft des Wissens auf unserem Planeten in seiner geologischen Zukunft geben. Das Wachstum des Geistes im Laufe der Zeit, soweit es untersucht wurde, gibt uns über die gesamten Jahrtausende der Existenz der Wissenschaft keine Daten dafür. Diese Möglichkeit kann jedoch nicht als real bestritten werden. In der Größenordnung von zehn Jahrtausenden kann eine Veränderung des menschlichen Geistesapparats wahrscheinlich oder sogar unvermeidlich sein.



Auf der Grundlage der tiefsten Analyse des Geistes und der psychischen Manifestation des lebendigen Selbst in seinen maximalen Manifestationen in der realen menschlichen Ära kann diese grundlegende Grundlage der Philosophie jedoch nicht als Maß für wissenschaftliche Erkenntnisse dienen, da moderne wissenschaftliche Erkenntnisse in ihrem wissenschaftlichen Apparat unvermeidlich sind Die aufregende Zukunft der Noosphäre hat eine wissenschaftliche empirische Basis, die viel mächtiger und solider ist als die angegebene Basis der Philosophie.



Der Reflexionsprozess, d.h. die Anwendung der Vernunft auf das Verständnis der Realität, sowohl für die Wissenschaft als auch für die Philosophie. Es sollte jedoch im Zusammenhang mit dem angegebenen unterschiedlichen Charakter in diesen Manifestationen des spirituellen Lebens des Individuums stehen.



Die philosophische Reflexion hängt mit der Frage zusammen, die seit Jahrhunderten vor ihm steht und noch nicht geklärt ist, da sie von vielen Philosophen immer noch geleugnet wird und nicht logisch widerlegt werden kann (aber von anderen nicht bewiesen werden kann): Gibt es einen speziellen Bereich der Philosophie? Erkenntnis, eine besondere Manifestation des Geistes -- „innere Erfahrung“ -- die es der Philosophie ermöglicht, neue Manifestationen der Realität zu enthüllen.



Obwohl dies immer noch umstritten ist, haben Philosophen in Wirklichkeit immer, wenn sie über die Realität nachdenken, ihren eigenen kognitiven Apparat -- den Geist -- korrekt in ihn eingeführt und ihn demselben Denkprozess unterzogen, den sie anderen Seiten der „äußeren Realität“ zuwandten.



Solche Arbeiten finden in der Wissenschaft nicht statt, vor allem, weil sie extrem lange Zeit und spezielle Kenntnisse erfordern und ihre Einführung in die aktuelle Arbeit eines Wissenschaftlers ihm keinen Raum für seine wichtigsten wissenschaftlichen Überlegungen lassen würde.



Ich werde nicht auf diese Seite der philosophischen Arbeit eingehen, da sie über jene Errungenschaften der Philosophie hinausgeht, die einen Naturforscher interessieren könnten, der in neuen Wissensgebieten wie der Biogeochemie arbeitet. Denn für diese Wissensbereiche wurde die philosophische Arbeit der Analyse der neuen Leitkonzepte, auf denen diese Wissenschaften, Ideen, die dem philosophischen Denken oft fremd und neu sind, überhaupt nicht durchgeführt. Diese philosophische Analyse, die für das Wachstum der Wissenschaft so notwendig ist, ist für den Wissenschaftler, wie ich bereits betont habe, einfach wegen der unvermeidlichen Rettung seiner Gedanken unzugänglich.



Bis solche Arbeiten von Philosophen ausgeführt werden und das philosophische Neue, das durch eine wissenschaftliche Suche in unserer Ära einer Explosion wissenschaftlicher Kreativität eingeführt wird, geklärt ist, muss ein Wissenschaftler, der in diesen neuen Wissensgebieten arbeitet, warten und sollte in den meisten Fällen die Meinungen von Philosophen außer Acht lassen, die dies nicht behandelt haben Philosophische Analyse der Vielzahl von im Wesentlichen neuen Fakten, Phänomenen und empirischen Verallgemeinerungen, wissenschaftlichen Theorien und wissenschaftlichen Hypothesen, die kontinuierlich durch wissenschaftliche Kreativität erstellt wurden. Für einen Wissenschaftler ist es völlig klar, dass der Philosoph ohne die angegebenen Arbeiten an neuem Material zu verzerrten Schlussfolgerungen kommen muss.



Im Folgenden werde ich noch einmal auf diese Frage zurückkommen. Da die Arbeit des Philosophen auf die Reflexion der Realität im Allgemeinen, der natürlichen Körper und der Phänomene der Realität im Besonderen gerichtet ist, kann der Wissenschaftler die Arbeit des Philosophen nicht ignorieren, muss seine Leistungen nutzen, kann ihr aber nicht die gleiche Bedeutung geben, die er dem Hauptteil seiner Philosophie beimisst Wissen.



Wenn wir uns der wirklichen Manifestation der Philosophie in der Kultur der Menschheit zuwenden, müssen wir mit der Existenz vieler mehr oder weniger unabhängiger, vielfältiger, ähnlicher und unähnlicher, widersprüchlicher philosophischer Systeme und Konzepte rechnen, von denen ein großer Teil keine Anhänger hat, aber dank ihrer Anwesenheit immer noch das Leben beeinflussen kann gedruckt auf alle zugänglichen Ausdrücke.



Sie finden unter ihnen scharf widersprüchliche, sich gegenseitig ausschließende Ideen und Systeme, positiv und negativ, optimistisch und pessimistisch, mystisch, rationalistisch und „wissenschaftlich“.



Von ihrer Koordination und der Suche nach einem einzigen, allgemeinen und umfassenden Konzept kann keine Rede sein. 91 Im Gegenteil. Versuche, eine einheitliche Philosophie zu schaffen, die für alle obligatorisch ist, sind längst in den Bereich der Vergangenheit zurückgegangen. Die Versuche, es wiederzubeleben, die in unserem sozialistischen Staat unternommen werden, indem eine offizielle, obligatorische dialektische Philosophie des Materialismus für alle geschaffen wird, sind angesichts des schnellen und tiefen wissenschaftlichen Wissens zum Scheitern verurteilt. Nach 20 Jahren besteht kaum ein Zweifel daran, dass das Leben selbst ohne Kampf ihre vergängliche Bedeutung deutlich zeigt.



Die Kraft der Philosophie liegt in ihrer Heterogenität und in einem weiten Bereich dieser Heterogenität.



Im Laufe der Zeit wächst die Vielfalt der philosophischen Ideen in unserer Zeit aufgrund der Komplexität und Vertiefung des Lebens, dank des Wachstums wissenschaftlicher Erkenntnisse, der Entstehung neuer Wissenschaften und der enormen Bedeutung neuer wissenschaftlicher Probleme und Entdeckungen in einem Ausmaß, wie es noch nie zuvor war. Trotzdem bleibt der Philosoph immer mehr hinter der philosophischen Verarbeitung wissenschaftlicher Erkenntnisse zurück.



83. Die Position der modernen Philosophie des Westens wird durch die Tatsache weiter erschwert, dass es in der Menschheit, im Osten, hauptsächlich in Indien, einen weiteren Komplex großer philosophischer Konstruktionen gibt, die sich unabhängig voneinander ohne ernsthaften Kontakt und Einfluss der westlichen Philosophie entwickelten und viele Jahrhunderte lang lebten sein unabhängiges Leben. Dieser Komplex philosophischer Konstruktionen entwickelte sich außerhalb des Einflusses des Monotheismus in einer religiösen Atmosphäre, die uns völlig fremd ist, in den Hochgebirgsregionen des Südens, in tropischer Natur, völlig fremd für Westeuropäer -- Christen oder Juden, in einem künstlerischen oder sozialen Umfeld.



Die größte Tatsache in der Geschichte der Kultur, die gerade die Tiefe ihrer Bedeutung offenbart hat, war, dass die wissenschaftlichen Erkenntnisse des Westens Ende des 19. Jahrhunderts tief und untrennbar mit Menschen verbunden waren, die unter dem Einfluss großer östlicher philosophischer Konstruktionen standen, die westlichen Wissenschaftlern fremd waren, aber philosophisches Denken Der Westen hat diesen Eintritt der lebenden, ihm fremden Philosophie des Ostens in das westliche wissenschaftliche Denken bisher nur schwach reflektiert; Dieser Prozess fordert gerade seinen Tribut.



Wissenschaftler, die unserer philosophischen und religiösen Kultur fremd sind und einen zahlenmäßig großen Teil der Menschheit umfassen, sind gleichberechtigt in die wissenschaftliche Arbeit eingetreten und nehmen darin schnell eine gleichberechtigte Stellung ein. Es ist klar, dass die Frage einer kurzen Zeit ist, wann sich dies mit unbestreitbarer Überzeugungskraft manifestieren und Konsequenzen haben wird, die von der westlichen Philosophie nicht berücksichtigt werden.



Die immer stärker werdende wissenschaftliche Arbeit steht unter dem immer stärkeren Einfluss von Menschen einer anderen religiösen und philosophischen Kultur als unsere europäisch-amerikanische.



Wir werden später sehen, dass die neuen Bereiche der Naturwissenschaften, zu denen die Biogeochemie gehört, auf dem Gebiet der Philosophie des Ostens für sich selbst wichtigere und interessantere Leitlinien treffen als in der Philosophie des Westens.



Unter dem Einfluss der modernen Wissenschaft, in erster Linie neuer Wissensbereiche, begann nach einer jahrhundertelangen Pause die Wiederbelebung der philosophischen Arbeit in Indien auf der Grundlage ihrer alten Philosophie und der modernen Weltwissenschaft , möglicherweise aufgrund ihrer unerwarteten Nähe zu neuen wissenschaftlichen Konzepten . 92 Es wird lebendig und wiedergeboren -- ist auf dem Vormarsch, wenn die Philosophie des Westens kreativ noch im Verlust ist.



Es scheint, dass bei einem solch chaotischen Zustand des philosophischen Denkens des 20. Jahrhunderts in Ermangelung einer lebhaften, großen Kreativität im Westen die Unmöglichkeit, ein Kriterium für die Wahrheit seiner Aussagen zu finden und gleichzeitig gleichwertige und entgegengesetzte lebendige philosophische Ideen im Osten zu existieren, die Bedeutung dieser Philosophie für Die kreative Blüte des wissenschaftlichen Denkens müsste zweitrangig sein. In Wirklichkeit ist dies nicht der Fall, insbesondere in einer Zeit, in der neue Wissenschaften entstehen, Wissensfelder, die der Wissenschaft bisher fremd waren und deren Probleme immer noch das Erbe jahrhundertealter westeuropäischer, hauptsächlich philosophischer und religiöser Kreativität sind.



Tatsache ist, dass eine philosophische Analyse abstrakter Konzepte, von denen viele in der neuen Wissenschaft, in ihren neuen Problemen und in wissenschaftlichen Disziplinen auftauchen, für die wissenschaftliche Abdeckung neuer Bereiche notwendig ist . Ein Wissenschaftler kann in der Regel nicht hierher gehen, dank der Technik der philosophischen Analyse, die viele Jahre der Vorbereitung erfordert -- so tief wie ein Philosoph. Darüber hinaus sind nicht alle wissenschaftlichen Aussagen im Allgemeinen verbindlich, sie werden in der Philosophie überhaupt nicht bewertet, und es können lange Zeit Zweifel am logischen Wert der wichtigsten wissenschaftlichen Schlussfolgerungen bestehen. Dies sollte in den neuen Wissenschaften und bei wesentlich neuen Problemen besonders ausgeprägt sein. Zwar ist hier nur die uralte philosophische Vorbereitung des Denkens oft noch schwächer.



In Bereichen, die gerade von der Wissenschaft erfasst wurden, wie es jetzt der Fall ist, begegnen wir vorgefertigten Ideen, die von Philosophen entwickelt oder zum Ausdruck gebracht wurden, bevor wir sie mit der Wissenschaft aufnehmen, mit der wir rechnen müssen. Die Wissenschaft muss sie überwinden. Zum Teil entsprechen sie nicht der Realität, zum Teil sind sie jedoch etwas zur Erklärung der Realität geeignet, die zum ersten Mal neue wissenschaftliche Erkenntnisse in diesen Bereichen liefert; Nur Klärung und ein neues Verständnis der Realität sind erforderlich.



Die Explosion der wissenschaftlichen Kreativität, die jetzt erlebt wird, ist jedoch nicht nur mit der Schaffung neuer Bereiche wissenschaftlicher Erkenntnisse, neuer Wissenschaften verbunden (§ 94): Sie verläuft entlang der gesamten Front der wissenschaftlichen Kreativität, verändert dramatisch und tiefgreifend alles, selbst die ältesten wissenschaftlichen Konzepte, wie zum Beispiel grundlegende. Wie Zeit und Materie spiegelt es sich im gesamten Inhalt der Wissenschaft und in ihren ältesten, seit langem unbeweglichen Errungenschaften wider.



Darüber hinaus stehen Wissenschaft und Philosophie in ständigem engen Kontakt, da sie sich zum Teil auf denselben Studiengegenstand beziehen.



Der Philosoph, der sich selbst vertieft und damit seine systematische Reflexion verbindet, ein Bild der Realität, in das er einfängt, und viele tiefe Manifestationen der Persönlichkeit, die von der Wissenschaft kaum berührt oder völlig unberührt bleiben, bringt, wie ich erwähnte, mit seiner Methodik Generationen heraus, eine logische Tiefe, was für einen Wissenschaftler im Allgemeinen nicht verfügbar ist. Denn es erfordert eine vorbereitende Vorbereitung und Vertiefung, die Spezialisierung, Zeit und Mühe erfordert, die ein Wissenschaftler ihnen nicht geben kann, deren Zeit vollständig von seiner speziellen Arbeit erfasst wird. Da die Analyse grundlegender wissenschaftlicher Konzepte durch philosophische Arbeit erfolgt, kann und sollte ein Naturforscher sie (natürlich kritisch) für seine Schlussfolgerungen verwenden. Er hat keine Zeit, es selbst zu bekommen.



Die Grenze zwischen Philosophie und Wissenschaft verschwindet -- entsprechend den Objekten ihrer Forschung -, wenn es um allgemeine naturwissenschaftliche Fragen geht. Manchmal werden diese verallgemeinernden wissenschaftlichen Ideen sogar als Wissenschaftsphilosophie bezeichnet. Ich halte dieses Verständnis jahrhundertealter Objekte des Wissenschaftsstudiums für falsch, aber es bleibt die Tatsache, dass sowohl der Philosoph als auch der Wissenschaftler gleichzeitig die allgemeinen Fragen der Naturwissenschaften behandeln und sich der Philosoph auf wissenschaftliche Fakten und Verallgemeinerungen stützt, aber nicht nur auf diese.



Ein Wissenschaftler sollte nicht über die Grenzen wissenschaftlicher Tatsachen hinausgehen, da es möglich ist, innerhalb dieser Grenzen zu bleiben, selbst wenn er sich wissenschaftlichen Verallgemeinerungen nähert.



84. Dies ist ihm jedoch nicht immer möglich und wird nicht immer von ihm getan.



Die enge Verbindung zwischen Philosophie und Wissenschaft bei der Erörterung allgemeiner naturwissenschaftlicher Fragen (die „Wissenschaftsphilosophie“) ist eine Tatsache, die als solche zu betrachten ist und die damit zusammenhängt, dass der Naturforscher in seiner wissenschaftlichen Arbeit oft ohne Angabe oder gar Erkenntnis davon ausgeht, z die Grenzen genauer, wissenschaftlich fundierter Fakten und empirischer Verallgemeinerungen. Offensichtlich kann in einer so konstruierten Wissenschaft nur ein Teil ihrer Aussagen als universell bindend und unveränderlich angesehen werden.



Aber dieser Teil umfasst und durchdringt ein weites Feld wissenschaftlicher Erkenntnisse, da wissenschaftliche Fakten -- Millionen von Millionen Fakten -- dazu gehören . Ihre Zahl wächst stetig, sie werden in Systemen und Klassifikationen angegeben. Diese wissenschaftlichen Fakten bilden den Hauptinhalt wissenschaftlicher Erkenntnisse und wissenschaftlicher Arbeit.



Sie sind bei ordnungsgemäßer Installation unbestreitbar und allgemein verbindlich. Zusammen mit ihnen kann es Systeme bestimmter wissenschaftlicher Tatsachen geben, deren Hauptform empirische Verallgemeinerungen sind .



Dies ist der Hauptgrund der Wissenschaft, der wissenschaftlichen Fakten, ihrer Klassifikationen und empirischen Verallgemeinerungen, die in ihrer Zuverlässigkeit nicht zu bezweifeln sind und die Wissenschaft scharf von Philosophie und Religion unterscheiden . Weder Philosophie noch Religion schaffen solche Tatsachen und Verallgemeinerungen.



85. Daneben haben wir in der Wissenschaft zahlreiche logische Konstruktionen, die wissenschaftliche Fakten miteinander verbinden und einen historisch vorübergehenden, sich ändernden Inhalt der Wissenschaft darstellen -- wissenschaftliche Theorien, wissenschaftliche Hypothesen, funktionierende wissenschaftliche Hypothesen, Marktbedingungen, Extrapolationen usw., deren Zuverlässigkeit normalerweise ist klein, schwankt stark; Aber die Dauer ihrer Existenz in der Wissenschaft kann manchmal sehr lang sein und Jahrhunderte dauern. Sie ändern sich für immer und unterscheiden sich wesentlich von religiösen und philosophischen Ideen nur dadurch, dass ihr individueller Charakter, ihre Persönlichkeitsmanifestation , die für philosophische, religiöse und künstlerische Konstruktionen so charakteristisch und lebendig ist, scharf in den Hintergrund tritt, möglicherweise aufgrund der Tatsache, dass sie dennoch basieren sie, sind verbunden und auf objektive wissenschaftliche Tatsachen reduziert, die in ihrer Herkunft durch dieses Attribut begrenzt und definiert sind.



Es gab und gibt Perioden in der Geschichte der Wissenschaft, in denen sie die Grundlage abdeckten, d. H. Wissenschaftliche Fakten, empirische Verallgemeinerungen, Systeme und Klassifikationen.



Aufgrund der Komplexität der Struktur der Wissenschaft ist es nicht so einfach, die grundlegende Natur ihrer Struktur und ihren scharfen und fundamentalen Unterschied zur Philosophie zu verstehen.



Im Laufe der Zeit konnte und sollte sein Skelett, das für alle als universell obligatorisch und unveränderlich angesehen werden kann, langsam keine Zweifel aufkommen lassen .



Wissenschaft schafft und getrennt von seinen historischen Wurzeln -- künstlerische Inspiration, 93 religiöses Denken (Magie, Theologie, etc ...), Philosophie -- zu verschiedenen Zeiten an verschiedenen Orten, unterschiedlich für die wichtigsten Merkmale seiner Struktur. Die Geschichte dieser Auswahl kann jetzt nur noch allgemein dargestellt werden.



86. Die Hauptmerkmale der Struktur der Wissenschaft -- Mathematik, Logik, wissenschaftlicher Apparat -- entwickelten sich im Allgemeinen unabhängig voneinander, und der historische Verlauf ihrer Identifizierung war unterschiedlich.



Zuallererst stachen die mathematischen Wissenschaften heraus , deren Unveränderlichkeit und Universalität zweifelsfrei ist.



Zeitgenossen seiner Entstehung erkannten die Bedeutung der Mathematik nicht und sie wurde nach Jahrtausenden verstanden. Aber diese Unveränderlichkeit existierte wirklich und machte sie im kulturellen Umfeld der Menschheit, wo sie ans Licht kam, zu einem unbewusst entsprechenden Einfluss. Wie jetzt offenbart und wie bereits erwähnt (§ 42), müssen wir der alten chaldäischen Mathematik (im vierten Jahrtausend vor uns) jetzt viel mehr Bedeutung beimessen als zuvor. Algebra und Analyse erreichten hier eine solche Tiefe, die sich selbst in der alten hellenischen Mathematik nicht bis zum Ende widerspiegelte. In der hellenistischen Ära war es jedoch für Wissenschaftler durchaus zugänglich, da die chaldäische wissenschaftliche Arbeit in der Zeit des IV. Jahrhunderts v. Chr. Durchgeführt wurde und VI von R.Kh. in Kontakt mit der hellenischen Wissenschaft. Anscheinend umfasste der geometrische Gedanke der Griechen, der in Kraft und Tiefe nicht mit dem vorher vergleichbar war, immer noch nicht das gesamte Gebiet des damals existierenden mathematischen Wissens. 94



Die hellenische Mathematik entwickelte sich fast ein Jahrtausend lang, wurde jedoch im Mittelalter für fast ein Jahrtausend unterbrochen und vom XVI. Etwa ein Jahrhundert wiederbelebt. Sie entwickelte sich kontinuierlich zu unserer Zeit und entwickelte sich in Form einer neuen Mathematik, die seit dem 17. Jahrhundert rasant und kontinuierlich wächst.



In den letzten drei Jahrhunderten wurde eine grandiose Struktur der mathematischen Wissenschaften geschaffen, deren Wahrheit keine Zweifel aufkommen lässt und die eine der höchsten Manifestationen des menschlichen Genies ist.



Heutzutage ist die Wissenschaft an die Grenzen ihrer Universalität und Unbestreitbarkeit gestoßen. Sie stand vor den Grenzen ihrer modernen Methodik. Philosophische und wissenschaftliche Fragen verschmolzen wie zu Zeiten der hellenischen Wissenschaft.



Einerseits haben Logistik und Axiomatik kognitive Probleme angegangen, die ungelöst sind und die wir nicht wissenschaftlich angehen können. Andererseits nähern wir uns einer ebenso unzugänglichen, rein wissenschaftlichen Lösung mit Hilfe einer höheren Geometrie und der Analyse von Problemen der realen Raumzeit.



Abgesehen von diesen philosophischen Wurzeln wissenschaftlicher Erkenntnisse, die sich nur auf ein großes Gebiet neuer Mathematik und empirischer Verallgemeinerungen stützen, entwickelt sich jedoch eine Explosion wissenschaftlicher Erkenntnisse, die wir jetzt erleben und auf deren Grundlage eine Person die Biosphäre verändern wird. Dies ist die Hauptbedingung für die Schaffung der Noosphäre.



87. Kaum viel später, ebenso wie die Schaffung des hellenischen (noch früheren) und hinduistischen (§ 42) Genies, wird ein weiterer Teil des exakten Wissens geschaffen, der so universell bindend ist wie die mathematischen Wissenschaften -- die Schaffung logischer Wissenschaften und Denkmethoden.



In hellenistischer Zeit in der Logik des Aristoteles, wir sind stark, aber unvollständig für diese Zeit 95 der Konstruktion -- die „Gesetze“ , dass wir für unveränderlich nehmen müssen.



In seinem Hauptteil war die Logik des Aristoteles eine Manifestation der analytischen Kraft seiner Persönlichkeit, aber einige der logischen Entdeckungen, die in dieser Logik offenbart wurden, sind mit Platon verbunden und wurden, wie bereit, von Aristoteles aus dem gegenwärtigen Leben der Platonakademie in Athen aufgenommen. (Aristoteles trat 306 v. Chr. Bei).



Nach dem Konzept von V. Jaeger, das ich als Arbeitshypothese für möglich halte, war Aristoteles der erste Grieche, „in dem wir auf eine echte Abstraktion stoßen. Er besaß all sein Denken.“ 96 Vor der aristotelischen Philosophie gab es nur ontologische Logik; Aristoteles teilte es in Elemente ein -- ein Wort oder ein Konzept und eine Sache. Es scheint mir jedoch, dass diese Idee im letzten Teil während der weiteren Arbeit geändert werden sollte, da in der Logik des Demokrit das Konzept einer Sache offenbar tiefer ausgedrückt wurde als in der Logik des Aristoteles und in dieser Hinsicht näher an der modernen wissenschaftlichen Logik des Naturforschers.



Die tiefe Entwicklung erreichte die logische Vertiefung der Hindus -- ungefähr in denselben Jahrhunderten, als das hellenische logische Denken die Realität wissenschaftlich umfasste. Die Unabhängigkeit der Schaffung tiefer hinduistischer logischer Systeme scheint uns wahrscheinlicher zu sein, wenn sie genauer untersucht werden. Zur gleichen Zeit, drei Jahrhunderte vor Christus und die ersten Jahrhunderte nach dem Beginn unserer Ära war der Austausch von Ost und West tief und kontinuierlich; gleich, was wir erst in den letzten 50 Jahren in unvergleichlich größerem Maßstab beobachtet haben.



Erst in der zweiten Hälfte des 19. Jahrhunderts trat die Logik in einen neuen Entwicklungspfad ein, der in unserer Zeit beschleunigt wurde. Zusammen mit der aristotelischen Logik, basierend auf Argumentation, nach den Gesetzen des gesunden Menschenverstandes, wurden neue Zweige der Logik geschaffen, und in dieser Logik (Exakte Logik der Angelsachsen) verschmilzt die Logik mit der Mathematik (Logistik). Diese neuen Trends in der Logik lassen sich bis ins 17. Jahrhundert zurückverfolgen, aber die Blütezeit der neuen Logik und die Hindernisse für das Verständnis ihrer Errungenschaften, die jetzt spannende Gedanken sind, gehören zum 20. Jahrhundert.



Wie wir sehen werden, erfordert die Entwicklung der Biogeochemie eine weitere Verfeinerung der logischen Probleme. Es scheint mir, dass sie zur Schaffung einer Logik der Phänomene der Noosphäre führen werden. Ich werde später darauf zurückkommen. 97



Die Logik ist eng mit der Philosophie verbunden und wurde lange Zeit wie die Psychologie damit identifiziert. Es entwickelte sich hauptsächlich auf philosophischer und nicht auf wissenschaftlicher Basis -- dies ist einer der Gründe, warum es jetzt hinter den Anforderungen der Naturwissenschaften, hauptsächlich der beschreibenden Naturwissenschaften und der Geowissenschaften, zurückgeblieben ist.



Ein Teil der Konstruktionen, logische Darstellungen, verlässt den Kreislauf der Wissenschaft und sollte sich auf die Philosophie beziehen. 98



88. Viel später wurde die dritte Grundlage der Wissenschaft geschaffen -- der wissenschaftliche Apparat der Tatsachen -- ein System und eine Klassifikation wissenschaftlicher Tatsachen, deren Genauigkeit an die Grenze stößt, wenn wissenschaftliche Tatsachen in den Elementen der Raumzeit ausgedrückt werden können -- quantitativ und morphologisch.



Millionen Millionen wissenschaftlicher Fakten auf dieser Grundlage werden kontinuierlich erstellt, systematisiert und in eine für die wissenschaftliche Arbeit geeignete Form gebracht.



Ein beispielloser wissenschaftlicher Apparat der Menschheit , der bequem zu betrachten ist, wird geschaffen und alles wächst , alles wächst und verbessert sich. Dies ist die Grundlage der neuen Wissenschaft unserer Zeit. Dies ist im Wesentlichen die Schöpfung des 17.-20. Jahrhunderts, obwohl einzelne Versuche und ziemlich erfolgreiche Konstruktionen davon Jahrhunderte zurückreichen. Dies gibt jedoch keinen Hinweis auf die wahre Geschichte der Schaffung des wissenschaftlichen Apparats -- so wie es jetzt ist.



Mit Ausnahme der Astronomie stehen uns im Wesentlichen nur die Errungenschaften der letzten Jahrhunderte zur Verfügung.



Dies gibt jedoch keinen Hinweis auf die wahre Geschichte der Schaffung des wissenschaftlichen Apparats. Diese Geschichte erregte überhaupt nicht genug Aufmerksamkeit, da Wissenschaftshistoriker auf seltsame Weise hauptsächlich auf allgemeine Fragen philosophischer und verallgemeinernder Natur achteten, aber nicht einmal ein Bild von der Schaffung eines wissenschaftlichen Apparats für eine neue Zeit gaben. Der moderne wissenschaftliche Apparat wurde in den letzten drei Jahrhunderten fast vollständig geschaffen, aber Fragmente aus den wissenschaftlichen Apparaten der Vergangenheit fielen hinein. Wir kennen diese Vergangenheit kaum.



Tatsächlich gab es in der Geschichte des wissenschaftlichen Denkens mehrere Versuche, es zu schaffen, die mehrere Generationen hintereinander umfassten. Ein wirklich großer wissenschaftlicher Wissensapparat begann sich mehrmals bewusst zu bilden, aber dann verschwand er oder entwickelte sich nicht mehr in den turbulenten Ereignissen des politischen oder öffentlichen Lebens. Die Gründe waren komplex, aber tief. Erstens waren dies Perioden von Kriegen, dem Fall von Kultur, Bürgerkrieg und Eroberung, in denen die wissenschaftliche Arbeit nicht genügend Raum für Entwicklung fand. Aber dies waren moralische Gründe, als eine Person in den Lasten des Lebens keine Unterstützung in der Wissenschaft suchte, sondern in der Philosophie oder Religion. Diese Erfahrungen waren so tiefgreifend, dass es keine Arbeitszentren oder Menschen gab, die den Apparat herstellten.



Darüber hinaus waren die Gründe sozusagen spezifischer. Es gab keinen Druck oder eine andere mächtige Art, Bücher zu verteilen, und das wissenschaftliche Gedächtnis der Menschheit, das in diesem wissenschaftlichen Apparat konzentriert war, konnte nicht ausreichend bewahrt werden und auf bessere Zeiten warten.



Wir kennen die Bewegung, die im IV. Jahrhundert vor Christus begann, genauer. Aristoteles begann 335-334 mit der Schaffung eines wissenschaftlichen Apparats. BC, als er nach Athen zurückkehrte und ein neues Hochschulzentrum gründete, unabhängig von der Akademie seines Lehrers Platon, verstarb er. Likey war nicht nur das Zentrum philosophischer, sondern auch wissenschaftlicher Arbeit. Letzteres setzte sich durch. Darin organisierte er eine Zusammenfassung und Erforschung des Faktenmaterials der Wissenschaften, einschließlich historischer und staatlicher, -- er organisierte tatsächlich einen wissenschaftlichen Apparat, der dem Ende des 4. Jahrhunderts entsprach. BC Dies war ein wissenschaftliches Phänomen von größter Bedeutung, aber es hatte nicht die Wirkung, die es wirklich verursachen sollte.



Nach dem Tod von Theophrastos (288 v. Chr.) Waren die Manuskripte und die Bibliothek von Aristoteles unter den turbulenten Bedingungen des damaligen Lebens nur wenigen zugänglich, aber am Ende wurden sie im unterirdischen Raum und nur etwa ein Hundertstes Jahr vor Christus aufbewahrt, d.h. e. Nach 180 Jahren wurden sie in einem verletzten Zustand von Apollikon von Theos gekauft (ungefähr hundert Jahre v. Chr.), er wurde in Ordnung gebracht und neue Kopien wurden angefertigt. Sulla, der Athen einnahm (86. Jahr v. Chr.), Übertrug sie nach dem Tod Apollikons nach Rom, und hier brachte Tyrannion aus Amizos sie in Ordnung, und Andronicos aus Rhodos führte sie erneut in die Literatur ein (um 70 v. Chr.) R.H.). Dies ist die zuverlässigste Vorstellung vom Schicksal von Aristoteles 'Manuskripten. 99 Dies zeigt jedenfalls, dass der von Aristoteles über zweihundert Jahre organisierte wissenschaftliche Apparat nicht verfügbar war und das wissenschaftliche Denken nicht beeinflussen konnte. Tatsächlich führte ihn diese Pause in eine neue, fremde Umgebung, die ihn nicht vollständig schätzen konnte.



89. Zwei Phänomene sollten beachtet werden. Erstens wurde die Tatsache, dass er sich nicht zufällig zu sammeln begann, sondern Ausdruck des wissenschaftlichen Werkes eines der größten wissenschaftlichen Genies war, nicht von einem Kollektiv oder vielmehr von einem Kollektiv geschaffen, um eine ihm von einer außergewöhnlichen Person und unter ihrer Führung übertragene Aufgabe zu erfüllen. 100 Und zweitens, dass dies in einer Zeit geschah, in der es Bedingungen gab, unter denen neben dem philosophischen Wissen und dem Verständnis der Umwelt eine sich rasch entwickelnde Technik vor dem Hintergrund der außerordentlichen Expansion der Kulturwelt und des einzigen Moments, das in diesen Tagen, als die alten Zivilisationen wieder mächtiger wurden, wieder aufgenommen wurde Indien und China, Ägypten, die Chaldäer und die Hellenen gingen nach Jahrhunderten der Isolation einen lebhaften Austausch von Ideologie und Weltlichkeit ein.



Aristoteles, der eng mit der griechischen Zivilisation Mazedoniens verbunden ist, deren Sprache sich von der griechischen unterscheidet, wurde in Thrakien geboren, griechisch von Vater und Kultur. Er ist eine völlig außergewöhnliche Person in der Weltgeschichte. Wir sahen seine außergewöhnliche Bedeutung in der Befreiung der Wissenschaft aus den Tiefen der Philosophie, in der sie vor ihr verloren ging. Aristoteles in der Wissenschaft war als Wissenschaftler und Philosoph gleichermaßen großartig, in den letzten Jahren seines Lebens eher ein Wissenschaftler als ein Philosoph. Er war nicht nur der Schöpfer in einer hellen Form seiner Logik, sondern auch seines wissenschaftlichen Apparats. Die Figur des Aristoteles vor einem historischen Hintergrund wird uns im Zusammenhang mit der Vertiefung unseres Wissens über die Geschichte der Philosophie klar, als sie versuchten, sich von seinem Buchverständnis, von der engen Gelehrsamkeit zu entfernen und lebende Menschen aus Aristoteles, Platon und anderen lebenden Menschen neu zu erschaffen.



Es scheint mir, dass W. Jaeger (1912) die historische Bedeutung und historische Arbeit dieser Errungenschaften von Aristoteles sehr klar und richtig umrissen hat. Er sagte: „Für die Menschen unserer Zeit war das wissenschaftliche Studium von„ Kleinigkeiten “lange Zeit nicht ungewöhnlich. Wir betrachten es als ein Phänomen voller Errungenschaften der Tiefe der Erfahrung, von dem nur dieser Weg zu echtem Wissen über die Realität führt. Es erfordert ein lebendiges historisches Gefühl, das nicht oft angetroffen wird um in unserer Zeit klar zu erkennen, wie seltsam und abstoßend diese Forschungsmethode für den durchschnittlich gebildeten Griechen des 4. Jahrhunderts v. Chr. zu sein schien und welche revolutionäre Innovation zu dieser Zeit von Aristoteles eingeführt wurde. Schritt für Schritt die Methoden, die heute der zuverlässigste Besitz und das gebräuchlichste Werkzeug sind. Die methodisch durchgeführte Technik zur Organisation der Beobachtung von Einzelheiten wurde der exakten neuen Medizin vom Ende des 5. Jahrhunderts v. Chr. und im 4. Jahrhundert v. Chr. entnommen Die Astronomie des Ostens mit ihren Katalogen und Aufzeichnungen, die im Laufe der Jahrhunderte aufbewahrt wurden. Ehemalige Forscher der Naturphilosophie gingen nicht über den Erklärungsbereich hinaus, indem sie einzelne, auffällige Phänomene erraten. Die Akademie gab, wie gesagt, keine Sammlung und Beschreibung von Einzelheiten, sondern eine logische Klassifizierung von abstrakten (universellen) Arten und Gattungen. „“ Aristoteles war der erste, der die Sinneswelt als Träger des überall (universellen) in der Materie enthaltenen (immaterielle Form) erforschte. Formen. „ Diese Aufgabe war im Vergleich zum Empirismus der älteren Medizin und Astronomie neu. Er brauchte unermessliche Arbeit und Geduld, um seine Zuhörer auf diesen neuen Weg zu bringen. „ 101



Aristoteles selbst übertrug diese kombinierten Fähigkeiten des genauen Faktenwissens auf alle Bereiche der damaligen Wissenschaft und sammelte mit Hilfe seiner Schüler eine große Menge an Material in Lyceum. Einige Beispiele werden es deutlich machen: Er veröffentlichte kritisch 158 Verfassungen, organisierte kollektive Arbeit, enzyklopädische Größe, einheitliche Form gemäß der Geschichte aller Wissenschaften des hellenischen Zivilisationszentrums. Dies war im Wesentlichen eine Geschichte über die schrittweise Entwicklung des menschlichen Wissens, die von einem seiner größten Schöpfer in der kritischen Ära seiner ersten Blütezeit in der hellenischen Welt herausgegeben und organisiert wurde. Der fast vollständige Verlust dieser Werke ist unersetzlich. Wie Sie wissen, wurde für seine Forschungen auf dem Gebiet der Naturgeschichte -- Mineralogie, Botanik und Zoologie -- dieselbe Arbeit geleistet, die in dürftigen Überresten in verzerrter Form zu uns kam. In der Geschichte des menschlichen Denkens repräsentiert Aristoteles ein einmaliges Phänomen. „Egal wie hoch Aristoteles 'Ideal (Leben) an sich war, seine Verwirklichung im Geist einer Person ist noch erstaunlicher. Dies ist und bleibt ein psychologisches Wunder, in das man nicht tiefer eindringen kann.“ 102



Die Entwicklung des wissenschaftlichen Denkens im Gegensatz zum philosophischen im Lyceum-Zentrum in Athen stoppte bereits nach dem Tod von Aristoteles 'zweitem Nachfolger -- Straton von Lampsak, d.h. am Ende des zweiten Jahrhunderts v Fragen der Philosophie, der Religion und der Moral erregten den Verstand denkender Menschen und nahmen Likey in Besitz.



Zu dieser Zeit blieb das Zentrum der wissenschaftlichen Arbeit in Alexandria jedoch erhalten, was eine spirituelle Fortsetzung von Aristoteles 'Ideen der letzten Periode seines Lebens war. In Alexandria zeigten das Museum und die Bibliothek einen starken Unterschied zwischen wissenschaftlicher und philosophischer Arbeit, das wissenschaftliche Denken wurde frei und die kraftvolle, ausgefeilte Technik der ptolemäischen Ära bildete die Grundlage für experimentelle Arbeiten. Hier hat sich eine beispiellose wissenschaftliche Arbeit in den Bereichen Medizin und Naturwissenschaften, exakte Philologie, Mathematik und Logik vor einem scharfen Hintergrund entwickelt, die vom Druck der Philosophie befreit ist. Diese Entwicklung erreichte Ende des ersten Jahrhunderts v. Chr. Ihr Maximum und hielt vielleicht ihren Anfang fest. Aber zweifellos vergingen anscheinend mehrere Jahrhunderte, die sich nur wenig kreativ manifestierten. Es ist möglich, dass die wissenschaftliche Arbeit in diesem Zentrum mehrere Jahrhunderte nach R.Kh. und nach dem Fall des Museums und der Bibliothek in Alexandria.



90. Wissenschaftlicher Apparat, dh Die kontinuierliche Systematisierung und methodische Verarbeitung, und demnach ist die Beschreibung möglich, genau und vollständig für alle Phänomene und natürlichen Körper der Realität, ist in der Tat der Hauptteil der wissenschaftlichen Erkenntnisse. Es sollte im Laufe der Zeit kontinuierlich wachsen und sich verändern, feiern und bewahren, da das wissenschaftliche Gedächtnis der Menschheit alles, was um uns herum geschieht, tiefer und tiefer in die Vergangenheit des Planeten vordringen sollte, in seinem Leben zunächst das sich verändernde Bild des Kosmos wissenschaftlich markieren sollte -- für uns den Sternenhimmel. Wissenschaft existiert nur, solange dieses Aufzeichnungsgerät richtig funktioniert; Die Kraft wissenschaftlicher Erkenntnisse hängt in erster Linie von der Tiefe, Vollständigkeit und Reflexionsrate der darin enthaltenen Realität ab. Ohne einen wissenschaftlichen Apparat gibt es keine Wissenschaft, selbst wenn Mathematik und Logik existieren würden. Das Wachstum von Mathematik und Logik kann jedoch nur in Gegenwart eines wachsenden und ständig beeinflussenden wissenschaftlichen Apparats erfolgen. Denn sowohl Logik als auch Mathematik sind nichts Unbewegliches und müssen in sich die Bewegung des wissenschaftlichen Denkens widerspiegeln, die sich vor allem im Wachstum des wissenschaftlichen Apparats manifestiert.



Auf seltsame Weise wird dieser Wert des wissenschaftlichen Apparats in der Struktur und in der Geschichte des wissenschaftlichen Denkens immer noch nicht berücksichtigt, und es gibt keine Geschichte seiner Entstehung. Inzwischen ist dies der fragilste Teil der Struktur wissenschaftlicher Erkenntnisse. Eine Unterbrechung seiner Entstehung während einer oder zwei Generationen reicht aus, damit die wissenschaftliche Arbeit der Menschheit aufhört oder sich vielmehr so ​​schwach manifestiert, dass ihre geologische Rolle im Gesamtmaßstab des menschlichen Lebens geglättet wird. Es muss Jahrhunderte dauern, bis der Apparat wieder hergestellt ist. In der Geschichte des Homo sapiens, die auf Millionen von Jahren geschätzt wird, bedeuten Jahrhunderte das natürlich nicht so wie in unserem heutigen Leben. Aber der wissenschaftliche Apparat ist eine Manifestation unseres gegenwärtigen Lebens und seiner Geschichte, die von der Menschheit in ihren zum Ausdruck gebrachten Denkmälern, Aufzeichnungen, Traditionen, Mythen, religiösen und philosophischen Kreativität verwirklicht wird und nicht über zehntausend Jahre hinausgeht. In dieser Größenordnung sind hundert Jahre eine lange Dauer. Die Überreste der materiellen Kultur gehen viel tiefer und beweisen die Existenz eines denkenden Menschen und sein soziales Leben vor Hunderttausenden von Jahren (§ 21). Aber wie wir gesehen haben, geht die Wissenschaft in Form von Logik, Mathematik und dem wissenschaftlichen Apparat für uns nicht tiefer als dreitausend oder viertausend Jahre. Die Geschichte dieser drei- bis viertausend Jahre kennen wir genauer; mit zunehmender Vollständigkeit in der Reihenfolge der Annäherung an unsere Zeit. Es ist möglich, dass vor Aristoteles versucht wurde, einen wissenschaftlichen Apparat zu schaffen. Wir können dies nicht leugnen, wir müssen versuchen, es zu lösen, aber bisher scheint es uns, dass Aristoteles die erste Person war, die diese Initiative ergriffen hat. Für uns ist es jetzt viel wichtiger, dass der Apparat auf seine Initiative hin endlich eingefroren ist (§ 68) und wir nun genau verfolgen können, wie er wieder in einer viel mächtigeren Form geschaffen wurde.



91. Die Geschichte des Niedergangs der Zivilisation des Mittelmeers kann nun mit ausreichender Genauigkeit in der Geschichte Westeuropas und Westasiens nachvollzogen werden. Der Tod des wissenschaftlichen Apparats in seiner Größenordnung schien den Zeitgenossen eine Kleinigkeit zu sein, da sie seine wirkliche Zukunft nicht berücksichtigen konnten, die ein Mensch erst im 19. und 20. Jahrhundert fühlen konnte.



Wir können die unverständlichen zeitgenössischen Zeitgenossen des wissenschaftlichen Zentrums verfolgen, das in Athen existierte und Mitte des 3. Jahrhunderts v. Chr. Von Aristoteles nach Straton geschaffen wurde. Zeitgenossen konnten dies nicht sehen. Dieses Zentrum schien ihnen vor Justinian (527 bis 565 von R.H.) zu existieren, d.h. viele weitere Jahrhunderte. Justinian schloss 529 die Higher School of Athens und hörte auf, Philosophie darin zu unterrichten, aber lange Zeit hatte er nicht die wissenschaftliche Arbeit, die unter Aristoteles stand.



In den unruhigen blutigen Ereignissen hörte die wissenschaftliche Arbeit von Alexandria auf. Wir wissen jedoch bisher nicht genau, weder wie noch wann. Erst kürzlich wurde klar, dass dieses wissenschaftliche Zentrum, anscheinend auch mit einer reduzierten wissenschaftlichen Arbeit, in den arabischen Staaten außerhalb von Alexandria, die sukzessive damit verbunden waren, mehrere Jahrhunderte dauerte. Es ist sehr wahrscheinlich, dass seine wissenschaftliche Bedeutung größer war als wir denken, und dass es sich in der Blütezeit der wissenschaftlichen Arbeit in den arabischen Staaten des Mittelalters widerspiegelte.



Es besteht jedoch kaum ein Zweifel daran, dass der wissenschaftliche Apparat zu dieser Zeit weniger mächtig war als in der Blütezeit der alexandrinischen Schule.



Aber die Staaten der arabischen Kultur konnten keine solide wissenschaftliche Arbeit bewahren und entwickeln. In dem blutigen und zerstörerischen religiösen Kampf mit dem Christentum einerseits und den militärischen Eroberern Zentralasiens andererseits, die dem Islam und dem Christentum fremd waren, erstarrte die lebendige kreative Arbeit in ihnen.



Dank der schwierigen Bedingungen des politischen und sozialen Lebens fand sie ihren Platz im lateinischen Westen, wo im 13. Jahrhundert eine wissenschaftliche Wiederbelebung begann, die schließlich zur modernen Wissenschaft führte.



92. Der wissenschaftliche Apparat erhielt dank der Eröffnung der Typografie Ende des 15. Jahrhunderts eine starke Gelegenheit, in einem bisher nicht möglichen Umfang für die Zukunft erhalten zu bleiben.



Alle folgenden Jahrhunderte erhöhten die Möglichkeiten seiner Erhaltung und Schaffung, und in den XVI, XVII Jahrhunderten. Die mächtige neue westeuropäische Wissenschaft ist gewachsen. Zu dieser Zeit wurde der wissenschaftliche Apparat auf dem Gebiet der philologischen Geschichtswissenschaften sowie der physikalischen und chemischen Wissenschaften speziell entwickelt und vertieft. In geringerem Maße wurde der wissenschaftliche Apparat der Naturwissenschaften selbst und der Biowissenschaften im weiteren Sinne des Wortes identifiziert und zusammengestellt.



Der Apparat der physikalischen und chemischen Wissenschaften erreichte die größte Entwicklung, als er von der wissenschaftlichen Theorie erfasst wurde und in Form von geometrischen und numerischen Ausdrücken ausgedrückt werden konnte. Von großer Bedeutung waren die Verallgemeinerungen von Newton, die zur Schaffung eines so ausgeprägten Bildes des Universums führten. Dieses Bild umfasste weder die Wissenschaften über das Leben noch die Wissenschaften über den Menschen, d.h. deckte nicht die überwiegende Mehrheit des modernen wissenschaftlichen Apparats ab. Es erlaubte jedoch etwas, was bisher in der Wissenschaft nicht in nennenswertem Maße möglich war, Ereignisse vorherzusagen und mit großer Genauigkeit vorauszusehen. Dies machte einen großen Eindruck und führte zu Missverständnissen über die Natur des wissenschaftlichen Apparats und die Aufgaben der wissenschaftlichen Forschung.



In den Wissenschaften der deskriptiven Naturwissenschaften wurden Mitte des 17. Jahrhunderts moderne Grundlagen gelegt, aber die letzte Verschiebung wurde von C. Linnaeus (1707-1778) vorgenommen. Die Systematik der Naturwissenschaften wurde verfügbar, und die Aufgabe der genauen und einfachen Berechnung aller natürlichen Naturkörper wurde gestellt. Linneus 'erster Kalkül von Tieren und Pflanzen führte zu mehreren tausend Arten. Derzeit ist dieser Betrag angemessen -- oder übersteigt eine Million.



Aber die Hauptsache ist, dass Linnaeus eine Massenbewegung verursachte, viele Tausende, wahrscheinlich Hunderttausende von Menschen in seiner Zeit, die sich dem Studium der Wildtiere zuwandten, einer genauen und systematischen Definition von Tier- und Pflanzenarten.



XIX Jahrhundert war der Hauptgrund für die Schaffung eines wissenschaftlichen Apparats. Es umfasste Lebens- und Sonderorganisationen -- teilweise international -- für die Sammlung, Klassifizierung und Systematisierung wissenschaftlicher Fakten sowie den zunehmenden Wunsch, sie zu erweitern und zu rationalisieren. Gleichzeitig wird das gesamte Material durch kollektive Arbeit, durch Generationen an maximales Wachstum angepasst: Dafür wurden spezielle Organisationsformen geschaffen.



Es gibt unzählige von ihnen -- Institute, Labors, Observatorien, wissenschaftliche Expeditionen, Stationen, Aktenschränke, Herbarien, internationale und nationale wissenschaftliche Kongresse und Verbände, Marineexpeditionen und Geräte für wissenschaftliche Arbeiten: Schiffe, Flugzeuge, Stratosphären, Fabriklabors und -stationen, Organisationen innerhalb von Trusts, Bibliotheken, abstrakte Zeitschriften, konstante Tabellen, geodätische und physikalische Vermessungen, geologische, topografische, boden- und astronomische Vermessungen, Ausgrabungen und Bohrungen usw. usw.



Wenn immer möglich, werden Fakten in Zahlen und Maßen ausgedrückt, wenn möglich, ihre Genauigkeit wird numerisch bewertet und, falls erforderlich, ihre Wahrscheinlichkeit -- dies ist für physikalische, chemische und astronomische Daten unvermeidlich geworden.



Tatsachen biologischer und geologischer Natur, die nicht vollständig mathematisch und numerisch ausgedrückt werden können, und historische geisteswissenschaftliche Tatsachen, einschließlich der Geschichte der Philosophie, die nur durch Worte und Konzepte ausgedrückt werden, unterscheiden sich jedoch, wie wir später sehen werden, wesentlich von Worte und Konzepte philosophischer und religiöser Konstruktionen.



Dieser Unterschied deckt alle Konzepte und Darstellungen des wissenschaftlichen Apparats ab. Es ist mit der besonderen logischen Natur der Konzepte und Ideen verbunden, aus denen der wissenschaftliche Apparat besteht. Im Gegensatz zu der Vielzahl von Konzepten in wissenschaftlichen Theorien und wissenschaftlichen Hypothesen, in der Religion und in der Philosophie sind die Wörter und Konzepte des wissenschaftlichen Apparats unweigerlich mit natürlichen Körpern und natürlichen Phänomenen verbunden, und die ihnen entsprechenden Wörter müssen in jeder Generation für ihr korrektes Verständnis verglichen werden. Erfahrung und Beobachtung mit der ihnen entsprechenden Realität. Die Logik, die sie beantwortet, muss sich unweigerlich von der Logik abstrakter Konzepte unterscheiden. Ich werde unten darauf zurückkommen.



Es ist jedoch notwendig, auf sehr verbreitete Vorstellungen über die unterschiedliche Natur des Materials des wissenschaftlichen Apparats einzugehen, die durch mathematische und numerische Daten ausgedrückt werden, und ein solcher Ausdruck ist nicht zugänglich. Ende des 18. und Anfang des 19. Jahrhunderts. Unter Wissenschaftlern war die Meinung weit verbreitet, dass die Wissenschaft erst dann ihren vollen Ausdruck erhält, wenn sie von einer Zahl mathematischer Symbole in der einen oder anderen Form abgedeckt wird. Dieser Wunsch trug zweifellos in einer Reihe von Bereichen zum enormen Fortschritt der Wissenschaft des 19. und 20. Jahrhunderts bei. In dieser Form entspricht es jedoch eindeutig nicht der Realität, da mathematische Symbole weit davon entfernt sind, die gesamte Realität zu erfassen, und der Wunsch danach in einer Reihe bestimmter Wissenszweige nicht zu einer Vertiefung, sondern zu einer Begrenzung der Stärke wissenschaftlicher Errungenschaften führt.



Der Unterschied zwischen dem Inhalt der Wissenschaft und dem nichtwissenschaftlichen Wissen, auch dem philosophischen, liegt nicht im Bereich der Wissenschaft in der Mathematik, sondern in der besonderen, genau angegebenen logischen Natur der Konzepte der Wissenschaft.



Wir haben es in der Wissenschaft nicht mit absoluten Wahrheiten zu tun, sondern mit unbestreitbar genauen logischen Schlussfolgerungen und relativen Aussagen, die innerhalb bestimmter Grenzen schwanken und in denen sie logisch den logisch unbestreitbaren Schlussfolgerungen des Geistes entsprechen.



93. Wir sehen also, dass es einen Teil der Wissenschaft gibt, der obligatorisch und wissenschaftlich wahr ist. Darin unterscheidet es sich stark von jedem anderen Wissen und jeder spirituellen Manifestation der Menschheit -- es hängt nicht von der Epoche oder vom sozialen und staatlichen System oder von der Nationalität und Sprache oder von individuellen Unterschieden ab.



Das:



    Mathematische Wissenschaften in ihrer Gesamtheit.

    Logische Wissenschaften sind fast ausschließlich.

    Wissenschaftliche Fakten in ihrem System, Klassifikationen und empirische Verallgemeinerungen sind der wissenschaftliche Apparat als Ganzes.



Alle diese Aspekte des wissenschaftlichen Wissens -- eine einzige Wissenschaft -- entwickeln sich rasant und der von ihnen abgedeckte Bereich wächst.



Neue Wissenschaften sind völlig von ihnen durchdrungen und werden in ihrer gesamten Waffenkammer geschaffen. Ihre Schaffung ist das Hauptmerkmal und die Kraft unserer Zeit.



Der lebendige, dynamische Prozess eines solchen wissenschaftlichen Wesens, der die Vergangenheit mit der Gegenwart verbindet, spiegelt sich spontan in der Umwelt des menschlichen Lebens wider und ist eine ständig wachsende geologische Kraft, die die Biosphäre in die Noosphäre verwandelt. Dies ist ein natürlicher Prozess, unabhängig von historischen Unfällen.



 



Abteilung 3

Neue wissenschaftliche Erkenntnisse und der Übergang der Biosphäre zur Noosphäre



 



Kapitel 6



 



Neue Probleme des 20. Jahrhunderts -- neue Wissenschaften. Die Biogeochemie ist ihre untrennbare Verbindung mit der Biosphäre.



 





94. Heutzutage kann der Rahmen einer separaten Wissenschaft , in die wissenschaftliche Erkenntnisse auseinanderfallen, den Bereich des wissenschaftlichen Denkens eines Forschers nicht genau bestimmen und seine wissenschaftliche Arbeit nicht genau charakterisieren. Die Probleme, die ihn zunehmend beschäftigen, passen nicht in den Rahmen einer separaten, bestimmten, etablierten Wissenschaft. Wir sind nicht auf Wissenschaften spezialisiert, sondern auf Probleme.



Das wissenschaftliche Denken des Wissenschaftlers unserer Zeit, mit bisher beispiellosem Erfolg und Stärke, vertieft sich in neue Bereiche von großer Bedeutung, die vorher nicht existierten oder ausschließlich das Schicksal der Philosophie oder Religion waren. Der Horizont wissenschaftlicher Erkenntnisse wächst im Vergleich zum 19. Jahrhundert -- in beispiellosem und unvorhergesehenem Ausmaß.



Probleme, die über die Grenzen einer Wissenschaft hinausgehen, schaffen unweigerlich neue Wissensbereiche, neue Wissenschaften, deren Anzahl und Geschwindigkeit zunehmen und die das wissenschaftliche Denken des 20. Jahrhunderts charakterisieren.



Manchmal, ziemlich oft, ist es möglich, im Namen einer neuen Disziplin die Komplexität ihres Inhalts, die Zugehörigkeit sowohl der wissenschaftlichen Fakten der neuen Disziplin als auch ihrer Methodik, ihrer empirischen Verallgemeinerungen, ihrer führenden Grundideen, wissenschaftlichen Hypothesen und Theorien zu verschiedenen alten wissenschaftlichen Bereichen auszudrücken. So entwickelte sich im 19. Jahrhundert am Ende die physikalische Chemie , deren Probleme sich sowohl von der Physik als auch von der Chemie unterscheiden und eine besondere Synthese dieser beiden wissenschaftlichen Disziplinen erfordern, wobei eine davon vorherrschend ist. Das Überwiegen chemischer Darstellungen und Phänomene spiegelt sich häufig in seinem Namen wider -- Chemie, aber nicht Physik. Im zwanzigsten Jahrhundert. im Zusammenhang damit wurde eine andere Wissenschaft gebildet -- eine verwandte, aber klar unterschiedliche chemische Chemie . Ihre körperliche Neigung ist klar. In beiden Fällen -- sowohl in der physikalischen Chemie als auch in der chemischen Physik -- wird ihr Platz im System wissenschaftlicher Erkenntnisse -- auf dem Gebiet der chemischen Wissenschaften -- zum einen physikalisch -- zum anderen durch ihren Namen klar und präzise bestimmt.



Dies ist nicht in der noch komplexeren und jüngeren wissenschaftlichen Disziplin der Fall, die sich im 20. Jahrhundert zu Beginn der Biogeochemie entwickelt hat (§ 96).



95. Und da dies sich deutlich in seinem Namen widerspiegelt, spielen chemische Darstellungen und chemische Phänomene eine führende Rolle im Vergleich zu geologischen und biologischen Problemen und Phänomenen, deren Inhalt den Namen ausmacht und beeinflusst.



Aufgrund der Natur der chemischen Objekte seiner Untersuchung tritt es jedoch nicht nur vollständig in die Chemie ein, sondern auch in ein völlig anderes, neues, immer noch riesiges Wissensgebiet -- die Atomphysik . Der Name bestimmt nicht genau seine Position im Wissenssystem.



In dieser Hinsicht ist es analog zu der physikochemischen Disziplin, die die Aufgabe hat, Atome in ihrer chemischen Manifestation zu untersuchen, und die entweder der Atomphysik, dann der physikalischen Chemie oder der Kristallochemie zugeschrieben wird, die eindeutig von der physikalischen Chemie isoliert sein sollte und nicht weniger eng ist zur Physik der Atome. Es wird nicht von der physikalischen Chemie abgedeckt, da die Eigenschaften des Atomkerns darin zum Vorschein kommen. Die Forschungsmethodik ist wesentlich anders.



Es erfasst auch das Gebiet der Radiologie -- den Zerfall von Atomen und den Nachweis von Isotopen. Im Gegensatz zur Chemie müssen Isotope als Basis verwendet werden, keine chemischen Elemente.



96. Die Biogeochemie ist eng mit einer bestimmten Region des Planeten verbunden -- vollständig mit einer bestimmten Erdhülle -- der Biosphäre 103 und ihren biologischen Prozessen bei ihrer chemisch-atomaren Detektion.



Der Bereich seiner Gerichtsbarkeit wird einerseits durch die geologischen Manifestationen des Lebens bestimmt, die in diesem Aspekt stattfinden, und andererseits durch biochemische Prozesse innerhalb von Organismen der lebenden Bevölkerung des Planeten. In beiden Fällen wirken nicht nur chemische Elemente, da die Biogeochemie Teil der Geochemie ist, d.h. gewöhnliche Gemische von Isotopen, aber auch verschiedene Isotope desselben chemischen Elements als Untersuchungsobjekte.



Die Biosphäre wurde 1875 von E. Suess [1831-1914] als Lebensgebiet auf der Erde bezeichnet. Sie wurde jedoch als ein besonderes reales Phänomen auf unserem Planeten bezeichnet -- ein natürlicher Körper, viel früher im späten 18. bis frühen 19. Jahrhundert.



Die Biosphäre in der Biogeochemie ist jedoch nur formal mit den Ideen von Suess verbunden. Dies ist in der Tat ein Lebensbereich auf unserem Planeten, aber nicht nur das ist charakteristisch dafür. Die Suess-Biosphäre ist das Gesicht unseres Planeten, wie es im übertragenen Sinne ausgedrückt wurde, in der Reflexion des Planeten im außerirdischen Weltraum. Es unterscheidet sich stark von der Biosphäre, wie aus dem Studium der Biogeochemie hervorgeht.



Die Biogeochemie untersucht die Biosphäre in ihrer Atomstruktur und verlässt das Gesicht des Planeten [das Antlitz], dh sein oberflächliches geografisches Bild und die Gründe für seine Manifestation, die von E. Suess neben oder an zweiter Stelle untersucht wurden.



Die Biosphäre in der Biogeochemie zeigt sich als eine spezielle, extrem isolierte Erdhülle auf unserem Planeten, die aus einer Reihe konzentrischer, die gesamte Erde überspannender, kontaktierender Formationen besteht, die als Geosphären bezeichnet werden. Es hat eine Struktur, die seit Milliarden von Jahren existiert. Diese Struktur ist mit der aktiven Teilnahme des Lebens an ihr verbunden, sie ist weitgehend in ihrer Existenz bestimmt und zeichnet sich vor allem durch dynamisch bewegliche, stabile, geologisch lange Gleichgewichte aus, die im Gegensatz zur mechanischen Struktur innerhalb bestimmter räumlicher Grenzen quantitativ beweglich sind und in Bezug auf die Zeit. 104



Wir können die Biogeochemie als die Geochemie der Biosphäre , einer bestimmten Erdschale betrachten -- der äußeren, die an der Grenze des Weltraums liegt. Eine solche formal korrekte Definition seines Fachgebiets würde jedoch nicht im Wesentlichen den gesamten Inhalt umfassen.



Die Einführung des Lebens als charakteristisches Unterscheidungsmerkmal der in der Biosphäre untersuchten Phänomene verleiht der Biogeochemie einen ganz besonderen Charakter und erweitert eine neue Art von Fakten, die eine spezielle wissenschaftliche Methodik für ihre Forschung erfordern, die es bequem macht, die Biogeochemie als separate wissenschaftliche Disziplin zu unterscheiden. Aber nicht nur die Frage nach der Bequemlichkeit wissenschaftlicher Arbeit wirft die Notwendigkeit einer solchen Trennung von Biogeochemie und Geochemie auf.



Das Wesen der Materie erfordert dies auch -- den tiefgreifenden Unterschied zwischen den Phänomenen des Lebens und den Phänomenen der inerten Materie. 105



Das Feld der Phänomene, die in lebloser inerter Materie auftreten, dominiert in der Geochemie, und nur in der Biosphäre spiegelt sich das Leben deutlich wider. Aber auch hier überschreitet es [lebende Substanz] nicht ein Zehntelprozentgewicht . Sie ist überhaupt nicht außerhalb der Biosphäre.



In energetischer Hinsicht umfasst das Leben die gesamte Biosphäre -- steht trotz seiner unbedeutenden, relativ großen Masse an erster Stelle. Die Biosphäre selbst nimmt einen besonderen Platz auf dem Planeten ein und ist stark von ihren anderen Gebieten getrennt, da es sich sowohl um physikalische, chemische, geologische als auch biologische Gebiete handelt. Es sollte als spezielle Hülle des Planeten betrachtet werden, obwohl die Biosphäre in der Gesamtmasse des Planeten ein unbedeutendes Glied ist. Das Gesicht der Erde -- die Biosphäre -- ist der einzige Ort auf dem Planeten, an dem kosmische Materie und Energie eindringen.



Angesichts all dessen ist es zweckmäßig, die Biogeochemie als eigenständige Wissenschaft herauszustellen, einen besonderen Teil der Geochemie.



Aber in seiner anderen Hauptaufgabe geht es über die Geochemie hinaus. Weil es sich nur den grundlegenden Eigenschaften des Lebens nähert, untersucht es im atomaren Aspekt nicht nur die Reflexion des Lebens in der Biosphäre, sondern auch die Reflexion von Atomen und ihre Eigenschaften in lebenden Organismen der Biosphäre -- im Aspekt dieser Erdhülle sind sie untrennbar mit ihr verbunden.



Eine ganze Reihe neuer Probleme -- biologische, die die Verwendung von Experimenten ermöglichen und nicht auf wissenschaftliche Beobachtungen in der Natur (d. H. In der Biosphäre) beschränkt sind -- werden nur im biogeochemischen Bereich der wissenschaftlichen Forschung aufgedeckt, der vollständig über Geochemie und Biogeochemie hinausgeht, wenn wir letztere als Geochemie der Biosphäre betrachten .



Dies macht es noch dringlicher, die Biogeochemie von der Geochemie als separate Wissenschaft zu isolieren.



97. Aber mehr als das. Wie wir sehen werden, erleben wir jetzt geologisch die Trennung des Reiches der Vernunft in der Biosphäre , die sein Aussehen und seine Struktur -- die Noosphäre -- grundlegend verändert. 106



Verknüpfung der Lebensphänomene im Aspekt ihrer Atome und Berücksichtigung, dass sie in die Biosphäre gehen, d.h. In einer Umgebung mit einer bestimmten Struktur, die sich während der geologischen Zeit nur relativ verändert und genetisch untrennbar damit verbunden ist, wird unweigerlich klar, dass die Biogeochemie auf tiefste Weise mit den Wissenschaften in Kontakt kommen sollte, nicht nur über das Leben, sondern auch über den Menschen, mit den Geisteswissenschaften .



Das wissenschaftliche Denken der Menschheit wirkt nur in der Biosphäre und verwandelt es im Verlauf seiner Manifestation letztendlich in die Noosphäre, umarmt es geologisch mit Vernunft.



Allein aufgrund dieser Tatsache ist die Biogeochemie nicht nur mit dem Gebiet der Biowissenschaften, sondern auch mit den Geisteswissenschaften verbunden.



Das wissenschaftliche Denken ist Teil der Struktur -- Organisation -- der Biosphäre und ihrer Manifestationen darin. Ihre Entstehung im Evolutionsprozess des Lebens ist das wichtigste Ereignis in der Geschichte der Biosphäre, in der Geschichte des Planeten (Abschnitt 13). Bei der Klassifizierung der Wissenschaften sollte die Biosphäre als Hauptfaktor berücksichtigt werden, was meines Wissens bewusst nicht getan wurde. Die Wissenschaften über die Phänomene und natürlichen Körper der Biosphäre sind besonderer Natur.



98. Je näher die wissenschaftliche Erfassung der Realität an einer Person liegt, desto größer wird zwangsläufig das Volumen, die Vielfalt und die Tiefe der wissenschaftlichen Erkenntnisse. Die Zahl der Geisteswissenschaften wächst stetig , deren Zahl theoretisch unendlich ist, denn die Wissenschaft ist die Schöpfung des Menschen, seine wissenschaftliche Kreativität und seine wissenschaftliche Arbeit; Es gibt keine Grenzen für die Suche nach wissenschaftlichem Denken, ebenso wie es keine Grenzen für endlose Formen gibt -- Manifestationen einer lebenden Person, insbesondere des Menschen, die alle Gegenstand wissenschaftlicher Suche sein können, verursachen viele spezielle spezifische Wissenschaften.



Der Mensch lebt in der Biosphäre, ist untrennbar damit verbunden. Er kann es nur direkt mit allen Sinnen studieren -- er kann es fühlen -- sie und ihre Objekte.



Er kann nur durch die Konstruktionen des Geistes über die Biosphäre hinaus eindringen und geht von einigen relativ Kategorien unzähliger Tatsachen aus, die er in der Biosphäre durch eine visuelle Untersuchung des Himmelsgewölbes und durch Untersuchung der Reflexionen der kosmischen Strahlung oder der in die Biosphäre fallenden kosmischen außerirdischen Materie in der Biosphäre erhalten kann.



Offensichtlich kann das wissenschaftliche Wissen über den Kosmos, das nur auf diese Weise in Bezug auf Vielfalt und Abdeckungstiefe gewonnen werden kann, nicht einmal mit den wissenschaftlichen Problemen und den von ihnen abgedeckten wissenschaftlichen Disziplinen verglichen werden, die den Objekten der Biosphäre und ihren wissenschaftlichen Erkenntnissen entsprechen.



Ein Mensch kann die Objekte der Biosphäre mit allen Sinnen direkt umarmen, und gleichzeitig baut der menschliche Geist, der materiell und energetisch untrennbar mit der Biosphäre, ihrem Objekt, verbunden ist, die Wissenschaft auf. Er führt in die wissenschaftlichen Konstruktionen die Erfahrungen des Menschen ein, die mächtiger und stärker sind als die, die in ihm erregt sind, von ihm das einzige verfügbare visuelle Bild des Sternenhimmels und der Planeten. Um die Himmelskörper und den aus ihnen gebauten Kosmos zu studieren, kann eine Person nur ihre Strahlung, ihre physiologische Wirkung (Vision), ihre physikalisch-chemische Analyse und ihre Abdeckung des mathematischen Denkens verwenden. Nur vergleichsweise unbedeutend sind die Energie- und Materialmanifestationen kosmischer Körper wie kosmischer Staub oder kosmische Gase, Meteoriten, die beim Eintritt in die Biosphäre zu Erdobjekten werden. Sie werden dadurch für das menschliche Denken so zugänglich wie möglich, spielen aber im Bild der menschlichen Realität und in den Erfahrungen der menschlichen Person eine relativ unbedeutende Rolle.



Die Phänomene, die mit dem Weltraum jenseits der Grenzen unseres Planeten verbunden sind, sind im wissenschaftlichen Apparat wahrscheinlich für mehr als Hunderte Millionen schnell wachsender genauer Daten verantwortlich.



Dennoch ist die Anzahl solcher wissenschaftlich fundierter Tatsachen im Vergleich zu den Objekten der wissenschaftlichen Berichterstattung über die Biosphäre und mit ihrem Einfluss und Eindringen in die menschliche Persönlichkeit, die äußerst vielseitig sind, unbedeutend.



Unser Wissen über den Raum unterscheidet sich sehr von dem Wissen über die Wissenschaften, die auf Biosphärenobjekten beruhen. Es gibt uns nur die allgemeinen Grundkonturen seiner Struktur.



99. Aber auch auf der anderen Seite der Biosphäre, nicht von oben nach oben -- in die kosmischen Weiten, sondern in die Eingeweide der Erde ..., tief in den Planeten hinein, begegnen wir ähnlichen Bedingungen -- mit den natürlichen Grenzen genauen Wissens, weil ein Mensch dies nicht tut kann diese Umgebung direkt studieren und ihren Charakter und ihre Struktur gemäß den Gesetzen seines Geistes und auf der Grundlage der Echos der darin auftretenden Phänomene schließen, die er mit seinen Werkzeugen erfassen und auf seine Sinne reduzieren kann.



Hier wird einem Menschen jedoch die Hauptsache vorenthalten, die ihm die Möglichkeit gibt, kosmische Weiten tief zu erfassen -- die Vision , die so eng und untrennbar mit dem Gehirn verbunden ist und es ermöglicht, aus der sichtbaren umgebenden Person -- die Realität -- wiederherzustellen, die ausschließlich durch wissenschaftliche Erkenntnisse, die Wissenschaften der Biosphäre 107 ( § 32).



Andererseits ist seine Abdeckung dieser Region des Planeten vielfältiger, da es: 1) das für seine Sinne direkt zugängliche Gebiet im Laufe der Zeit allmählich vertiefen kann und die Grenze dieser Vertiefung weit über die Biosphäre hinausgehen wird . Mit jedem Jahrzehnt bewegt es sich tiefer und schneller in die Tiefe und 2) es kann die Tiefen des Planeten verbinden -- die Erdkruste unter der Biosphäre und möglicherweise die engsten tief sitzenden Krustenbereiche, die untrennbar materiell mit der Biosphäre verbunden sind, mit diesem vielfältigen und tief wissenschaftlich abgedeckten Faktenmaterial. das aus Wissenschaften stammt, die die Biosphäre studieren. Aus diesem Grund haben wir in diesem Bereich der Realität in einigen Jahrhunderten (wissenschaftlich seit dem 17. Jahrhundert) 108 Wissen erlangt, das mit dem Wissen über den Raum vergleichbar ist, und die Prognose für die Zukunft ist günstiger als für die wissenschaftliche Konstruktion des Weltraums.



Dies liegt an der Tatsache, dass wir hier nicht über die Grenzen eines natürlichen Körpers hinausgehen -- der Planet, auf dem wir existieren, und daher unter Berufung auf das Studium der Biosphäre nicht nur allgemeine Linien des Phänomens erhalten kann, sondern in gewissem Maße auch ein farbenfrohes Bild der Realität. 109



Kapitel 7



 



Die Struktur wissenschaftlicher Erkenntnisse als Manifestation der Noosphäre, verursacht durch einen geologisch neuen Zustand der Biosphäre. Der historische Verlauf der planetarischen Manifestation des Homo sapiens durch die Schaffung einer neuen Form kultureller biogeochemischer Energie und der damit verbundenen Noosphäre.



 





100. Wissenschaften über die Biosphäre und ihre Objekte, dh Alle humanitären Wissenschaften, ausnahmslos Naturwissenschaften im eigentlichen Sinne des Wortes (Botanik, Zoologie, Geologie, Mineralogie usw.), alle technischen Wissenschaften -- angewandte Wissenschaften im weiteren Sinne -- sind die Wissensgebiete, die dem wissenschaftlichen Denken des Menschen am zugänglichsten sind. Hier konzentrieren sich Millionen von Millionen kontinuierlich wissenschaftlich fundierter und systematisierter Fakten, die das Ergebnis organisierter wissenschaftlicher Arbeit sind und mit jeder Generation ab dem XV-XVII. Jahrhundert schnell und bewusst unkontrolliert wachsen.



Insbesondere wissenschaftliche Disziplinen zur Struktur des Instruments wissenschaftlicher Erkenntnisse sind untrennbar mit der Biosphäre verbunden und können wissenschaftlich als geologischer Faktor, als Manifestation seiner Organisation betrachtet werden. Dies ist die Wissenschaft der „spirituellen“ Kreativität des Menschen in seinem sozialen Umfeld, die Wissenschaft des Gehirns und der Sinnesorgane, Probleme der Psychologie oder Logik. Sie bestimmen die Suche nach den Grundgesetzen menschlicher wissenschaftlicher Erkenntnisse, der Kraft, die sich in unserer geologischen Ära gewandelt hat, der menschlichen Biosphäre, die in einen natürlichen Körper eingeschlossen ist, der in seinen geologischen und biologischen Prozessen neu ist, in seinen neuen Zustand, in die Noosphäre 110, auf die ich weiter unten zurückkommen werde. 111



Seine Entstehung in der Geschichte des Planeten, die vor mehreren Zehntausenden von Jahren intensiv (auf der Skala der historischen Zeit) begann, ist ein Ereignis von großer Bedeutung in der Geschichte unseres Planeten, das hauptsächlich mit dem Wachstum der Biosphärenwissenschaften verbunden ist und offensichtlich kein Zufall ist. 112



Man kann daher sagen, dass die Biosphäre das Hauptgebiet wissenschaftlicher Erkenntnisse ist, obwohl wir uns erst jetzt ihrer wissenschaftlichen Trennung von der uns umgebenden Realität nähern.



101. Aus dem Vorhergehenden geht hervor, dass die Biosphäre auf die Tatsache reagiert, dass sie im Denken der Naturforscher und in den meisten Argumenten der Philosophie in Fällen, in denen sie den Kosmos nicht als Ganzes berührten, sondern auf der Erde blieben, der Natur in ihrem üblichen Verständnis entspricht, insbesondere der Natur der Naturforscher.



Aber nur diese Natur ist nicht amorph und nicht formlos, wie es seit Jahrhunderten angenommen wurde, sondern hat eine bestimmte, sehr genau begrenzte Struktur 113, die als solche in allen mit der Natur verbundenen Schlussfolgerungen und Schlussfolgerungen reflektiert und berücksichtigt werden sollte.



In der wissenschaftlichen Forschung ist es besonders wichtig, dies nicht zu vergessen und dies zu berücksichtigen, da der Wissenschaftler und Denker unbewusst im Gegensatz zur menschlichen Persönlichkeit zur Natur durch die Größe der Natur gegenüber der menschlichen Person unterdrückt werden.



Aber das Leben in all seinen Erscheinungsformen und in den Erscheinungsformen der menschlichen Persönlichkeit, einschließlich, verändert die Biosphäre dramatisch in einem solchen Ausmaß, dass nicht nur die Gesamtheit des unteilbaren Lebens und in einigen Problemen eine einzelne menschliche Persönlichkeit in der Noosphäre in der Biosphäre nicht ignoriert werden kann.



102. Wildtiere sind das Hauptmerkmal der Manifestation der Biosphäre, sie unterscheiden sie scharf von anderen irdischen Muscheln. Die Struktur der Biosphäre ist in erster Linie vom Leben geprägt.



Wir werden später sehen (Abschnitt 135), dass zwischen den physikalisch-geometrischen Eigenschaften lebender Organismen -- in der Biosphäre erscheinen sie in Form ihrer Aggregate -- lebende Materie und zwischen den gleichen Eigenschaften der inerten Materie im Gewicht und in der Anzahl der Atome, die die überwiegende Mehrheit der Biosphäre ausmachen, in mancher Hinsicht liegt eine unpassierbare Kluft. Lebende Materie ist der Träger und Schöpfer freier Energie, die in keiner irdischen Hülle in einem solchen Ausmaß existiert. Diese freie Energie -- biogeochemische Energie 114 -- deckt die gesamte Biosphäre ab und bestimmt im Wesentlichen ihre gesamte Geschichte. Es verursacht und verändert die Intensität der Migration chemischer Elemente, die die Biosphäre bilden, dramatisch und bestimmt ihre geologische Bedeutung.



Innerhalb der Grenzen der lebenden Materie wird im letzten zehn Jahrtausend eine neue Form dieser Energie erzeugt, die in ihrer Bedeutung rasch zunimmt, in ihrer Intensität und Komplexität sogar noch größer. Diese neue Energieform, die mit dem Leben menschlicher Gesellschaften, der Gattung Homo und anderer (Hominiden) in ihrer Nähe verbunden ist und gleichzeitig die Manifestation gewöhnlicher biochemischer Energie bewahrt, bewirkt gleichzeitig eine neue Art der Migration chemischer Elemente, die weit zurückbleiben Die übliche biochemische Energie lebender Materie auf dem Planeten.



Diese neue Form der biogeochemischen Energie, die als Energie der menschlichen Kultur oder kulturelle biogeochemische Energie bezeichnet werden kann, ist die Form der biogeochemischen Energie, die derzeit die Noosphäre erzeugt. Später werde ich auf eine detailliertere Darstellung unseres Wissens über die Noosphäre und ihre Analyse zurückkommen. Aber jetzt muss ich kurz sein Aussehen auf dem Planeten identifizieren.



Diese Form der biogeochemischen Energie ist nicht nur dem Homo sapiens, sondern allen lebenden Organismen eigen. 115 Aber in ihnen ist es jedoch im Vergleich zu gewöhnlicher biogeochemischer Energie unbedeutend und beeinflusst das Gleichgewicht der Natur kaum merklich und nur in geologischer Zeit. Es ist mit der mentalen Aktivität von Organismen verbunden, mit der Entwicklung des Gehirns in den höheren Manifestationen des Lebens und beeinflusst die Form, die den Übergang der Biosphäre zur Noosphäre nur mit der Erscheinung des Geistes bewirkt.



Anscheinend hat sich seine Manifestation in menschlichen Vorfahren über Hunderte von Millionen von Jahren entwickelt, aber es konnte sich nur in unserer Zeit in Form von geologischer Kraft ausdrücken, als Homo sapiens die gesamte Biosphäre mit seinem Leben und seiner kulturellen Arbeit umfasste.



103. Die biogeochemische Energie lebender Materie wird in erster Linie durch die Reproduktion von Organismen bestimmt, ihre Beständigkeit, bestimmt durch die Energie des Planeten, den Wunsch, ein Minimum an freier Energie zu erreichen -- wird durch die Grundgesetze der Thermodynamik bestimmt, die der Existenz und Stabilität des Planeten entsprechen.



Es drückt sich in der Atmung und in der Ernährung von Organismen aus -- „Naturgesetzen“, die noch nicht in ihrem mathematischen Ausdruck zu finden sind, deren Suchaufgabe jedoch 1782 von K. Wolf an der damaligen Petersburger Akademie der Wissenschaften klar festgelegt wurde. 116



Offensichtlich ist diese biogeochemische Energie, ihre Form auch dem Homo sapiens inhärent. Es ist, wie alle anderen Organismen, ein Artenmerkmal, 117 und es scheint uns im Laufe der historischen Zeit unverändert zu sein. In anderen Organismen ist eine andere Form der „kulturellen“ biogeochemischen Energie unverändert oder verändert sich kaum. Diese andere Form drückt sich in den häuslichen oder technischen Lebensbedingungen von Organismen aus -- in ihren Bewegungen, im Alltag und beim Bau von Wohnungen, in ihrer Bewegung der umgebenden Substanz usw. Es macht, wie ich bereits angedeutet habe, einen unbedeutenden Teil ihrer biogeochemischen Energie aus.



Beim Menschen wächst und nimmt diese Form der biogeochemischen Energie, die mit dem Geist verbunden ist, im Laufe der Zeit zu und bewegt sich schnell an den ersten Ort. Dieses Wachstum ist wahrscheinlich mit dem Wachstum des Geistes selbst verbunden -- ein Prozess, der sehr langsam zu sein scheint (wenn es wirklich passiert) -, aber hauptsächlich mit der Verfeinerung und Vertiefung seiner Verwendung, verbunden mit einer bewussten Veränderung des sozialen Umfelds und insbesondere mit Wachstum wissenschaftliche Erkenntnisse.



Ich gehe davon aus, dass die Skelette des Homo sapiens, einschließlich des Schädels, seit Hunderten von Jahrtausenden keinen Grund mehr gegeben haben, sie als einer anderen Art von Person zugehörig zu betrachten. Dies ist nur unter der Bedingung zulässig, dass sich das Gehirn einer paläolithischen Person in seiner Struktur nicht wesentlich vom Gehirn einer modernen Person unterscheidet. Gleichzeitig besteht kein Zweifel daran, dass der menschliche Geist des Paläolithikums für diese Art von Homo keinen Vergleich mit dem Geist des modernen Menschen ertragen kann. Daraus folgt, dass der Geist eine komplexe soziale Struktur ist, die sowohl für eine Person unserer Zeit als auch für eine paläolithische Person auf demselben Nervensubstrat aufgebaut ist, jedoch unter verschiedenen sozialen Bedingungen, die sich in der Zeit (im Wesentlichen Raum-Zeit) zusammensetzen.



Ihre Veränderung ist das Hauptelement, das letztendlich zur expliziten Umwandlung der Biosphäre in die Noosphäre führte, vor allem zur Schaffung und zum Wachstum eines wissenschaftlichen Verständnisses der Umwelt.



104. Die Erzeugung kultureller biogeochemischer Energie auf unserem Planeten ist ein wichtiger Faktor in seiner geologischen Geschichte. Es wurde während der gesamten geologischen Zeit vorbereitet. Der entscheidende Hauptprozess ist hier die maximale Manifestation des menschlichen Geistes. Im Wesentlichen ist dies jedoch untrennbar mit der gesamten biogeochemischen Energie lebender Materie verbunden.



Das Leben durch atomare Migrationen im Lebensprozess verbindet alle Migrationen von Atomen der inerten Materie der Biosphäre miteinander.



Organismen leben nur so lange, bis der Material- und Energieaustausch zwischen ihnen und der umgebenden Biosphäre aufhört. 118 In der Biosphäre werden grandiose bestimmte chemische Kreislaufprozesse der Atommigration offenbart, in die lebende Organismen als regelmäßig untrennbar eintreten, häufig der Hauptteil des Prozesses. Diese Prozesse bleiben während der geologischen Zeit unverändert, und beispielsweise dauert die Migration von Magnesiumatomen, die in das Chlorophyll gelangen, mindestens zwei Milliarden Jahre lang kontinuierlich durch unzählige genetisch verwandte Generationen grüner Organismen. Lebende Organismen allein mit solchen atomaren Migrationen sind untrennbar und untrennbar mit der Biosphäre verbunden und bilden einen regelmäßigen Teil ihrer Struktur.



Dies sollte bei der wissenschaftlichen Erforschung des Lebens und bei der wissenschaftlichen Beurteilung aller seiner Manifestationen in der Natur niemals vergessen werden. Wir können nur damit rechnen, dass eine kontinuierliche Verbindung -- das Material und die Energie eines lebenden Organismus mit der Biosphäre, eine Verbindung einer sehr spezifischen Natur, „geologisch ewig“, die wissenschaftlich genau ausgedrückt werden kann -- bei all unseren wissenschaftlichen Ansätzen für Lebewesen immer vorhanden ist und sich darin widerspiegeln sollte alle unsere logischen Schlussfolgerungen über ihn.



Um die Geochemie der Biosphäre zu untersuchen, müssen wir zunächst die logische Bedeutung dieser Verbindung, die unweigerlich in alle unsere mit dem Leben verbundenen Konstruktionen einfließt, genau bewerten. Es hängt nicht von unserem Willen ab und kann nicht von unseren Experimenten und Beobachtungen ausgeschlossen werden, es muss von uns immer als etwas Grundlegendes betrachtet werden, das Lebewesen innewohnt.



Auf diese Weise sollte sich die Biosphäre ausnahmslos in allen unseren wissenschaftlichen Urteilen widerspiegeln. Es sollte sich in allen wissenschaftlichen Erfahrungen und in wissenschaftlichen Beobachtungen manifestieren -- und in allen Gedanken der menschlichen Persönlichkeit, in allen Spekulationen, denen die menschliche Persönlichkeit -- auch durch Gedanken -- nicht entkommen kann.



Die Vernunft kann auf diese Weise nur mit der maximalen Entwicklung der Hauptform der menschlichen biogeochemischen Energie, d.h. bei maximaler Reproduktion.



105. Die potenzielle Möglichkeit, die Oberfläche eines gesamten Planeten durch Ausbreitung durch einen Organismus oder durch eine Art zu erfassen, ist allen Organismen inhärent, da für alle das Gesetz der Fortpflanzung in derselben Form in Form eines geometrischen Fortschritts ausgedrückt wird. Ich habe lange auf die Hauptbedeutung dieses Phänomens für die Biogeochemie hingewiesen, 119 und an meiner Stelle werde ich in diesem Buch darauf zurückkommen.



Offensichtlich ist das Phänomen der Erfassung der gesamten Oberfläche des Planeten durch eine Art weit verbreitet für Wasserlebewesen in der Nähe des mikroskopischen Planktons von Seen und Flüssen und für einige Formen -- im Wesentlichen auch aquatische -- von Mikroben, Oberflächenbedeckungen des Planeten, die sich in der Troposphäre ausbreiten. Bei größeren Organismen beobachten wir dies bei einigen Pflanzen fast vollständig.



Für einen Menschen beginnt dies in unserer Zeit ans Licht zu kommen. Im zwanzigsten Jahrhundert umfasst es den gesamten Globus und alle Meere. Dank des Erfolgs der Kommunikation kann ein Mensch untrennbar mit der ganzen Welt verbunden sein, nirgendwo kann er einsam sein und sich hilflos in der Größe der irdischen Natur verlieren.



Jetzt hat die Zahl der Menschen auf der Erde einen beispiellosen Wert erreicht und nähert sich zwei Milliarden Menschen, obwohl das Töten in Form von Kriegen, Hunger und Unterernährung, die kontinuierlich Hunderte Millionen Menschen betreffen, den Prozess extrem schwächt. Aus geologischer Sicht wird es eine unbedeutende Zeit dauern, kaum mehr als ein paar hundert Jahre, bis diese Überreste der Barbarei gestoppt sind. Dies kann jetzt frei gemacht werden; Die Möglichkeit, dass dies nicht geschehen würde, liegt bereits in den Händen des Menschen, und der rationale Wille wird diesem Weg unweigerlich folgen, da er der natürlichen Tendenz des geologischen Prozesses entspricht. Darüber hinaus sollte dies so sein, da die Möglichkeiten, schnell und fast spontan dafür zu handeln, zunehmen. Die wahre Bedeutung der Massen von Menschen, die am meisten darunter leiden, wächst unkontrolliert.



Die Zahl der Menschen auf unserem Planeten begann vor ungefähr 15 bis 20.000 Jahren zuzunehmen, als eine Person im Zusammenhang mit der Öffnung der Landwirtschaft weniger abhängig wurde von einem Mangel an Nahrungsmitteln. Anscheinend gab es also vor etwa 10-8.000 Jahren die erste Explosion der menschlichen Fortpflanzung. 120 G.F. Nikolai (1918-1919) 121 versuchte, die reale Reproduktion der Menschheit und die Entwicklung der Landwirtschaft, die reale Bevölkerung des Planeten durch den Menschen, numerisch zu bewerten. Nach seinen Schätzungen leben auf der gesamten Erdfläche 11,4 Menschen pro Quadratkilometer, was 2,10 -- 4% der möglichen Bevölkerung entspricht. Angesichts der von der Sonne empfangenen Energie ermöglicht die Landwirtschaft, 150 Menschen pro 1 km 2 zu tränken , d. H. die ganze Erde (Land) wird 22.5109 unteilbar haben, d.h. 22-24 mal mehr als sie jetzt leben. 122 Aber der Mensch produziert Energie für Nahrung und zum Leben, nicht nur mit landwirtschaftlicher Arbeit. Angesichts dieser Möglichkeit schätzte Nikolai grob, dass die Erde in der historischen Ära, die in unserer Zeit mit der Nutzung neuer Energiequellen begann, von drei menschlichen Hexalionen (31016) bewohnt werden könnte, d. H. mehr als zehn Millionen Mal höher als die Zahl der modernen Menschheit. Diese Zahlen sollten gegenwärtig, wenn mehr als 20 Jahre seit Nikolais Kalkül vergangen sind, stark erhöht werden, da in Wirklichkeit eine Person derzeit Energiequellen nutzen kann, über die in den Jahren 1917-1919. Nikolai dachte nicht -- die Energie, die mit dem Atomkern verbunden ist. Wir müssen jetzt einfacher sagen, dass die Energiequelle, die der Geist im Energiezeitalter des menschlichen Lebens, in das wir eintreten, erfasst, praktisch unbegrenzt ist. Daraus ergibt sich, dass kulturelle biogeochemische Energie (§ 17) die gleiche Eigenschaft hat. Nach Nikolais Kalkül haben Maschinen in seiner Zeit die menschliche Energie mehr als verzehnfacht. Wir können jetzt keine genauere Berechnung geben. Jüngste Berechnungen des American Geological Committee zeigen jedoch, dass die heute weltweit eingesetzte Wasserkraft bis Ende 1936 60 Millionen PS erreichte: In 16 Jahren ist sie um 160 Prozent gestiegen, hauptsächlich in Nordamerika. 123 Schon jetzt ist es notwendig, den Kalkül von Nikolai um mehr als das Eineinhalbfache zu erhöhen.



Tatsächlich spielen all diese Kalküle der Zukunft, ausgedrückt in numerischer Form, keine Rolle, da unser Wissen über die Energie, die der Menschheit zur Verfügung steht, als rudimentär bezeichnet werden kann. Natürlich ist die Energie, die der Menschheit zur Verfügung steht, keine unbegrenzte Menge, weil es wird durch die Größe der Biosphäre bestimmt. Dies bestimmt die Grenze der kulturellen biogeochemischen Energie.



Wir werden sehen (§ 138), dass es auch eine Grenze für die biogeochemische Grundenergie der Menschheit gibt -- die Übertragungsrate des Lebens, die Grenze der menschlichen Fortpflanzung.



Siedlungsrate 124 -- Der von Nikolai im Wesentlichen berücksichtigte Wert von V basiert auf der Bevölkerung des Planeten, die tatsächlich für eine Person unter eindeutig ungünstigen Bedingungen für ihr Leben beobachtet wird. Darüber hinaus werden wir in Zukunft sehen, dass es in der Biosphäre bisher unbekannte Phänomene gibt, die zu einer stationären maximalen Anzahl von Unteilbaren führen, die in einer bestimmten geologischen Ära unter den gegebenen Bedingungen von Biozönosen pro Hektar existieren könnten.



106. Die Menge der menschlichen Bevölkerung auf dem Planeten können wir erst zu Beginn des neunzehnten Jahrhunderts mit einiger Genauigkeit berücksichtigen. Sie wird in diesem Fall mit einem hohen Prozentsatz möglicher Fehler berechnet. In den letzten 137 Jahren hat unser Wissen stark zugenommen, es kann jedoch nicht davon ausgegangen werden, dass es die Genauigkeit erreicht hat, die die Wissenschaft derzeit möglicherweise benötigt. Für ältere Zeiten sind Zahlen nur bedingt. Sie helfen uns jedoch, den Prozess zu verstehen.



Die folgenden Daten können in dieser Hinsicht für uns relevant sein.



Die Zahl der Menschen im Paläolithikum erreichte wahrscheinlich einige Millionen. Es ist zulässig, dass es aus einer Familie stammt. Es ist aber auch die gegenteilige Ansicht möglich. 125



Im Neolithikum geht es wahrscheinlich um zig Millionen auf der gesamten Erdoberfläche. Es ist davon auszugehen, dass es auch in historischer Zeit nicht hundert Millionen erreicht oder leicht überschritten hat. 126



G.F. Nicholas für 1919 schlug vor, dass die menschliche Bevölkerung des Planeten jedes Jahr um 12 Millionen Menschen zunimmt, d. H. pro Tag steigt um etwa 30.000 Menschen. Nach einer kritischen Zusammenfassung von Kulisher (1932) 127 betrug die Weltbevölkerung im Jahr 1800 850 Millionen Menschen (A. Fisher geht von 775 Millionen aus). Für die weiße Rasse kann man ihre Zahl in 1000 gleich 30 Millionen und in 1800 -- 210 Millionen, in 1915 -- 645 Millionen nehmen. Für die gesamte Menschheit für 1900 laut Kulisher ungefähr 1700 Millionen. und nach A. Goetner (1929) 128 -- 1564 Millionen und nach ihm 1925 -- 1856 Millionen.



Offensichtlich hat diese Zahl inzwischen mehr oder weniger etwa zwei Milliarden erreicht. Die Bevölkerung unseres Landes (ungefähr 160 Millionen) macht ungefähr 8% der Weltbevölkerung aus. Die Bevölkerung der ganzen Welt wächst schnell und anscheinend steigt der Prozentsatz unserer Bevölkerung relativ an, da ihr Wachstum größer ist als das durchschnittliche Wachstum. Im Allgemeinen müssen wir bis zum Ende des Jahrhunderts auf einen signifikanten Überschuss von 2 Milliarden warten.



107. Die Vermehrung von Organismen, dh Die Manifestation biogeochemischer Energie der ersten Art, ohne die es kein Leben gibt, ist untrennbar mit dem Menschen verbunden. Aber ein Mann aus seiner Selektion aus der Masse des Lebens auf dem Planeten besaß bereits Werkzeuge, sogar sehr grobe, die es ihm ermöglichten, seine Muskelkraft zu steigern, und waren die erste Manifestation moderner Maschinen, die ihn von anderen lebenden Organismen unterschieden. Die Energie, die sie fütterte, wurde jedoch durch die Ernährung und Atmung des menschlichen Körpers selbst erzeugt. Es ist wahrscheinlich Hunderttausende von Jahren her, als ein Mensch, die Gattung Homo, und seine Vorfahren Geräte aus Holz, Knochen und Stein besaßen. Langsam, über viele Generationen hinweg, wurde die Fähigkeit zur Herstellung und Verwendung dieser Werkzeuge entwickelt, die Fähigkeit wurde perfektioniert -- der Geist in seiner ersten Manifestation.



Diese Werkzeuge wurden bereits im ältesten Paläolithikum vor 250.000 -- 500.000 Jahren beobachtet.



Während dieser Zeit erlebte die Biosphäre größtenteils kritische Zeiten. Anscheinend begann am Ende des Pliozäns eine scharfe Veränderung -- in seinem Wasser- und Wärmezustand begann und entwickelte sich die Eiszeit ständig. Wir leben offenbar immer noch, während der Verfall seiner letzten Manifestation, vorübergehend oder endgültig, unbekannt ist. In diesen einer halben Million Jahren sehen wir starke Klimaschwankungen; Relativ warme Perioden -- die Zehntausende und Hunderttausende von Jahren dauerten -- wurden in der nördlichen und südlichen Hemisphäre durch Perioden ersetzt, in denen sich Eismassen langsam bewegten -- auf einer historischen Zeitskala, die beispielsweise in der Nähe von Moskau eine Leistung von bis zu einem Kilometer erreichte. Sie sind vor siebentausend Jahren in der Region Leningrad verschwunden und besetzen immer noch Grönland und die Antarktis. Anscheinend bildeten sich Homo sapiens oder seine unmittelbaren Vorfahren kurz vor Beginn der Eiszeit oder in einer seiner warmen Lücken. Der Mensch erlebte die Schwere der Kälte dieser Zeit. Dies war möglich, weil zu dieser Zeit im Paläolithikum eine große Entdeckung gemacht wurde -- die Beherrschung des Feuers.



Diese Entdeckung wurde an einem oder zwei, vielleicht wenigen weiteren Orten gemacht und verbreitete sich langsam unter der Bevölkerung der Erde. Anscheinend haben wir hier einen allgemeinen Prozess großer Entdeckungen, bei dem nicht die Massenaktivität der Menschheit, die Glättung und Verbesserung von Einzelheiten, sondern die Manifestation einer separaten menschlichen Individualität eine Rolle spielt. Für eine nähere Zeit und in so vielen Fällen können wir dies, wie wir später sehen werden (Abschnitt 134), genau verfolgen.



Die Öffnung des Feuers war das erste Mal, dass ein lebender Organismus eine der Naturkräfte besaß und beherrschte. 129



Zweifellos liegt diese Entdeckung, wie wir jetzt sehen, dem späteren zukünftigen Wachstum der Menschheit und unserer wahren Stärke zugrunde.



Dieses Wachstum verlief jedoch äußerst langsam, und es ist für uns schwer vorstellbar, unter welchen Bedingungen es auftreten könnte. Das Feuer war den Vorfahren oder Vorgängern der Hominiden, die die Noosphäre bilden, bereits bekannt. Die jüngste Entdeckung in China enthüllt uns die kulturellen Überreste von Sinanthropus, die auf seine weit verbreitete Verwendung von Feuer hinweisen, anscheinend lange vor der letzten Vereisung in Europa, Hunderttausende von Jahren vor unserer Zeit. Als er diese Entdeckung machte, haben wir jetzt keine Daten, die irgendwie plausibel sind. Sinanthropus besaß bereits Intelligenz, hatte unhöfliche Werkzeuge, benutzte Sprache und führte den Bestattungskult durch. Er war bereits ein Mann, aber uns durch zahlreiche morphologische Merkmale fremd. Es ist möglich, dass er einer der Vorfahren der modernen menschlichen Bevölkerung Chinas ist. 130



108. Die Öffnung des Feuers ist umso überraschender, als die Manifestation von Feuer und Licht vor dem Menschen relativ selten war und sich hauptsächlich dann manifestierte, wenn sie einen großen Raum in Form von kaltem Licht einnahm, wie das Leuchten des Himmels, Auroren, leise elektrische Entladungen, Sterne und Planeten, leuchtende Wolken. Die Sonne allein, die Quelle des Lebens, war eine lebendige Manifestation von Licht und Wärme, die den Planeten beleuchtete und erwärmte.



Lebende Organismen haben seit langem eine Manifestation von kaltem Licht entwickelt. Es betraf so große Phänomene wie das Leuchten des Meeres, das normalerweise Hunderttausende von Quadratkilometern einnimmt, oder das Leuchten der Tiefen des Meeres, dessen Bedeutung uns erst jetzt klar wird. Das Feuer, begleitet von hohen Temperaturen, manifestierte sich in lokalen Phänomenen und nahm selten große Räume ein, wie z. B. Vulkanausbrüche.



Aber diese grandiosen Ereignisse auf menschlicher Ebene konnten offensichtlich in ihrer zerstörerischen Kraft in keiner Weise zur Öffnung des Feuers beitragen. Der Mensch musste sie näher bei sich suchen und weniger schreckliche und gefährliche Manifestationen der Natur als Vulkanausbrüche, und jetzt in ihrer Manifestation die Kräfte des modernen Menschen übertreffen. Wir beginnen erst, uns ihrer Verwendung realistisch zu nähern, unter Bedingungen, die für einen Paläolithiker unzugänglich und undenkbar waren. 131



Er musste nach Phänomenen suchen, die ihm in den umgebenden Alltagsphänomenen des Lebens Wärme und Feuer geben; in seinen Lebensräumen -- in Wäldern, Steppen, unter Wildtieren, in enger (für uns längst vergessener) Kommunikation, mit der er lebte. Hier konnte er in einer für ihn sicheren Form auf Feuer und Erwärmung in einer für ihn sicheren Form treffen. Dies waren einerseits Feuer, das Verbrennen von lebender und verstorbener lebender Materie. Dies waren genau die Feuerquellen, die der Paläolithiker benutzte.



Er verbrannte Bäume, Pflanzen, Knochen, dasselbe, was außerhalb seines Willens Feuer um ihn herum gab. Dieses Feuer vor dem Menschen wurde aus zwei sehr unterschiedlichen Gründen verursacht. Einerseits verursachten Blitzentladungen Waldbrände oder brannten trockenes Gras. Der Mensch leidet immer noch unter den Bränden, die durch diesen Weg verursacht werden. Natürliche Bedingungen in der Eiszeit, insbesondere in der Zwischeneiszeit, könnten noch günstigere Bedingungen für Gewitter bieten. Aber es gab noch einen anderen Grund, der ein vom Menschen unabhängiges Feuer verursachte.



Dies war die lebenswichtige Aktivität niederer Organismen, die zu Bränden in den trockenen Steppen, 132 zum Verbrennen von Kohleflözen und zum Verbrennen von Torfmooren führte, was mehrere menschliche Generationen dauerte und eine günstige Gelegenheit bot, Feuer zu empfangen. Wir haben direkte Hinweise auf solche Kohlebrände im Altai, im Kusnezker Becken, wo sie im Pliozän und im Postpliozän auftraten, aber wo sie in der historischen Zeit auftraten und wo sie jetzt zu rechnen sind. Die Ursachen dieser Brände sind noch nicht vollständig geklärt, aber alles deutet darauf hin, dass wir hier kaum die Phänomene eines rein chemischen Prozesses der Selbstentzündung haben, d. H. intensive Sauerstoffoxidation von Luft, fragmentierter Kohle oder deren spontane Verbrennung aufgrund der Wärme, die während der Oxidation von Eisensulfidverbindungen in Kohle entsteht. 133



Am wahrscheinlichsten sind biochemische Phänomene, die mit der Aktivität thermophiler Bakterien verbunden sind. Kürzlich haben wir für Torfmoore direkte Beobachtungen von B.L. Isachenko und N. I. Malchevsky. 134



Dieses Phänomen erfordert nun eine sorgfältige Untersuchung.



109. Solche warmen Regionen im Winter und Sommer sowie Orte mit heißen Quellenausgängen waren wertvolle Geschenke der Natur an den Paläolithiker, der sie auch als Stämme und Nationalitäten nutzen sollte, die wir noch in der Lebensphase des Paläolithikums fanden oder kürzlich benutzten.



Mit der großen Beobachtung des Menschen dieser Zeit und seiner Nähe zur Natur erregten solche Orte zweifellos seine Aufmerksamkeit und hätten von ihm genutzt werden müssen, insbesondere in der Zeit der Eiszeit.



Es ist merkwürdig, dass wir unter den tierischen Instinkten die Verwendung derselben biochemischen Prozesse beobachten. Dies wird in der Familie der Hühner, den sogenannten Unkrauthühnern oder den Großfüßen (Megapodidae) in Ozeanien und Australien beobachtet, die die Fermentationswärme nutzen, d.h. Ein bakterieller Prozess zum Entfernen von Küken aus Eiern, zum Erzeugen großer Sandhaufen oder vom Boden, gemischt mit organischer Fäule, die verrotten kann. 135 Diese Haufen können eine Höhe von 4 Metern erreichen und die Temperatur in ihnen steigt nicht unter 44. Anscheinend sind dies die einzigen Vögel mit einem solchen Instinkt.



Es ist möglich, dass Ameisen und Termiten die Temperatur ihrer Wohnungen zweckmäßigerweise erhöhen.



Aber diese schwachen Versuche sind unvergleichlich mit der planetaren Revolution, die der Mensch gemacht hat.



Der Mensch als Energiequelle, Feuer -- die Produkte des Lebens -- trockene Pflanzen. Zahlreiche Mythen über seine Entstehung sind erhalten und geschaffen worden. 136 Das Charakteristischste war jedoch, dass der Mensch hierfür Methoden verwendete, die bei den Methoden, die er zur Erzeugung von Feuer in der Biosphäre vor seinen Entdeckungen beobachtete, kaum Feuer gaben. Anscheinend bestand die älteste Methode darin, die Muskelkraft einer Person in Wärme umzuwandeln (starke Reibung trockener Gegenstände), einen Funken zu schnitzen und ihn aus Stein zu fangen. Das komplexe Brandschutzsystem wurde schließlich vor Hunderten und mehr von Tausenden von Jahren im Alltag entwickelt.



Die Oberfläche des Planeten hat sich nach dieser Entdeckung dramatisch verändert. Überall blitzten, gingen aus und es erschienen Feuer, wo nur ein Mann lebte. Dank dessen konnte der Mensch die Kälte der Eiszeit überleben.



Der Mensch schuf Feuer in einer lebenden Umgebung und setzte es dem Brennen aus. Auf diese Weise gewann er durch die Steppen- und Waldbrände an Stärke im Vergleich zur umgebenden Tier- und Pflanzenwelt, die ihn aus einer Reihe anderer Organismen herausholte und ein Prototyp seiner Zukunft war. Nur in unserer Zeit, in den XIX-XX Jahrhunderten, besaß der Mensch eine andere Licht- und Wärmequelle -- elektrische Energie. Der Planet begann noch mehr zu leuchten, und wir stehen am Anfang der Zeit, deren Bedeutung und Zukunft bisher nicht in unserer Aufmerksamkeit liegt.



110. Viele Dutzend, wenn nicht Hunderttausende von Jahren sind vergangen, bevor eine Person andere Energiequellen besaß, von denen einige, wie zum Beispiel Dampfenergie, eine direkte Folge der Öffnung des Feuers waren.



Im Laufe der langen Jahrtausende hat der Mensch seine Position in einem Lebensumfeld dramatisch verändert und die lebendige Natur des Planeten radikal verändert. Es begann in der Eiszeit, als der Mensch begann, Tiere zu zähmen, aber über viele Jahrtausende hinweg spiegelte sich dies nicht hell in der Biosphäre wider. In der Altsteinzeit stellte sich heraus, dass nur der Hund mit einer Person in Verbindung gebracht wurde.



Eine radikale Veränderung begann auf der Nordhalbkugel nach dem Abgang des letzten Gletschers außerhalb der Vereisung.



Dies war die Entdeckung der Landwirtschaft, die Nahrung unabhängig von Wildtieren schuf, und die Entdeckung der Viehzucht beschleunigte zusätzlich zu ihrer Bedeutung für die Nahrung die Bewegung des Menschen.



Es ist schwer vorstellbar, unter welchen Bedingungen die Landwirtschaft hätte entstehen können. Natur , die Person zu der Zeit rund um zwanzigtausend, wenn nicht mehr Jahre, 137 von dem scharf unterschieden sie heute in den gleichen Orten gesehen. Dies ist nicht nur eine Konsequenz, wie kürzlich angenommen, durch die Veränderung der kulturellen Arbeit der Menschheit, sondern auch durch eine spontane Veränderung der Umgebung der Eiszeit, in der wir jetzt leben. Wir sehen deutlich, dass eine Person selbst in einem kürzeren historischen Zeitraum, den letzten 5 bis 6 Tausend Jahren, geologische Veränderungen in der Biosphäre erlebte. Die Regionen China, Mesopotamien, Kleinasien und Ägypten können Orte in Westeuropa sein, außerhalb der damaligen Taiga, in Bezug auf Klima, Wasserregime, Geomorphologie, sie unterschieden sich stark von den modernen, und dies kann nicht durch die kulturelle Arbeit der Menschheit und ihre Folgen erklärt werden. unvermeidlich, aber vom Menschen unvorhergesehen. Zusammen mit der Kulturarbeit der Menschheit, die spontan abnimmt oder an Intensität zunimmt, ist der einhundert oder zweitausend Jahre dauernde Gefrierprozess des Gletschermaximums ein Prozess der anthropogenen Ära.



111. Landwirtschaft mit der modernen Kraft der Kultur kann nicht die gesamte Landoberfläche bedecken. Nach modernen Schätzungen (1929) überschreitet die landwirtschaftlich genutzte Fläche 13 Millionen Quadratmeter nicht. km, d.h. 2,5% der Planetenoberfläche. 138 Wenn man nur ein Land nimmt, werden es 8,6 Prozent sein. Wahrscheinlich sollte diese Zahl als übertrieben angesehen werden, aber im Allgemeinen vermittelt sie einen Eindruck von der enormen kulturellen biogeochemischen Energie, mit der sich die Menschheit im Laufe von 20.000 Jahren, wenn nicht sogar mehr, der Oberfläche des Planeten verändert hat. Es sollte bedacht werden, dass die Arktis und Antarktis, die Halbwüsten und Wüsten Nord- und Südafrikas, Zentralasiens, Arabiens, der Prärie Nordamerikas, eines bedeutenden Teils Australiens, des Hochplateaus und der Hochberge Tibets und Nordamerikas für die Landwirtschaft schwierig oder überhaupt nicht zugänglich sind. Sie machen zusammen mindestens ein Fünftel des Landes aus. Ich muss sagen, dass Taiga und Regenwälder für einen Menschen zu Beginn seiner kulturellen Arbeit selbst mit der Eröffnung des Feuers eine fast unüberwindbare Barriere für die Landwirtschaft darstellten. Gleichzeitig musste er lange mit der Resistenz kämpfen, die Insekten und wilde Säugetiere, Pflanzenparasiten und Unkraut ihm entgegenbrachten, und eine riesige und oft die überwiegende Mehrheit der Produkte seiner Arbeit einfangen. Selbst jetzt, in unserer Landwirtschaft, fangen Unkräuter 15 bis 14 Pflanzen ein -- zunächst war diese Zahl sicherlich minimal. 139 Gegenwärtig haben wir dank des sozialistischen Aufbaus unseres Landes etwas genauere Zahlen, um die Intensität und Möglichkeit dieser Form menschlicher biogeochemischer Energie zu berücksichtigen. Wir machen eine außergewöhnliche Erweiterung des Aussaatgebiets durch. Wie N. I. Vavilov und seine Mitarbeiter angeben: Nur in den letzten zwei Jahren (1930-1931) hat die Anbaufläche um 18 Millionen Hektar zugenommen, was nach alten Maßstäben ein Jahrzehnt erforderlich gemacht hätte. 140 Bei den geplanten Berechnungen großer Fachkräfte wurde das allgemeine Bild unseres Landes geklärt. Die von ihm belegte Fläche beträgt 21,4 Millionen km², d.h. 16,6% Sushi. Davon sind 5,68 Millionen Quadratkilometer für die Landwirtschaft außerhalb der Nordgrenze unpraktisch. Und all das unbequeme Land für die Landwirtschaft ist ungefähr 11,85 Millionen Quadratkilometer groß. Das bequeme Land ist 9,53 Millionen Quadratkilometer groß. Daher liegt der größte Teil unseres Landes außerhalb der Grenzen der modernen Landwirtschaft oder wird als für die Landwirtschaft ungeeignet angesehen. 141 Dieser Bereich kann jedoch erheblich verbessert und reduziert werden. Der Plan der staatlichen Rückgewinnungsarbeiten nach der Berechnung von L.I. Prasolova, 142 wird es um etwa 40% erhöhen. Offensichtlich ist dies nicht das Ende der Möglichkeiten, und es kann kaum Zweifel geben, dass die Menschheit, wenn sie es für notwendig oder wünschenswert hält, Energie entwickeln könnte, die die gesamte Landfläche unter Landwirtschaft erfasst, und vielleicht noch mehr. 143



112. Wir haben in China immer noch eine intensive Landwirtschaft, die sich über Generationen entwickelt hat 144, die mehr als 4000 Jahre lang in stationärer Form auf einer Fläche von etwa 11 Millionen Quadratkilometern existierte. Zweifellos veränderte sich zu dieser Zeit das Gebiet des Staates, aber das entwickelte System und die Fähigkeiten der Landwirtschaft blieben erhalten und veränderten das Leben und die Natur in der Umgebung. Erst in der jüngsten Zeit, in unserem Jahrhundert, war diese Bevölkerungsmasse in einer instabilen Bewegung und viele jahrtausendealte Fähigkeiten werden zerstört. Für China können wir über die Pflanzenzivilisation (Goodnear) sprechen. 145 In unzähligen Generationen, seit mehr als viertausend Jahren, hat die Bevölkerung das Land verändert und sich in seinem Leben mit der umgebenden Natur verschmolzen. Die meisten landwirtschaftlichen Produkte werden wahrscheinlich hier abgebaut, und dennoch ist die Bevölkerung der ewigen Bedrohung durch Unterernährung ausgesetzt. 146 Mehr als drei Viertel der Bevölkerung sind Landwirte. „Der größte Teil Chinas ist ein altes Land etablierter Landwirtschaft, dessen Boden so nahe an der wirtschaftlichen Grenze kultiviert wird, dass große Erträge schwer zu erzielen sind. Die Chinesen sind tief in einem riesigen Gebiet von etwa 11 Millionen Kilometern verwurzelt. Das charakteristischste Element der chinesischen Landschaft ist nicht der Boden, nicht die Vegetation. Nicht Klima, sondern Bevölkerung. Menschen sind überall. In diesem alten Land gibt es kaum einen Ort, der vom Menschen und seinen Aktivitäten nicht verändert wurde. Wie tief das Leben unter dem Einfluss von verändert wurde m der Umwelt ist so wahr, dass der Mensch die Natur verändert und verändert hat und ihr einen menschlichen Fußabdruck gegeben hat. Die chinesische Landschaft ist ein biophysikalisches Aggregat, dessen Teile so eng miteinander verbunden sind wie der Baum und der Boden, auf dem sie wächst. So tief in der Erde verwurzelt, dass es schafft ein einziges, alles aufregendes Aggregat -- nicht Mensch und Natur als getrennte Phänomene, sondern ein einziges organisches Ganzes. „ 147 Und trotz dieser unermüdlichen Arbeit von vielen tausend Jahren sind etwas mehr als 20 Prozent der Fläche Chinas von der Landwirtschaft besetzt 148. Der Rest der Fläche kann für ein so großes und natürlich reiches Land mit staatlichen Maßnahmen verbessert werden, die nur auf wissenschaftlicher Ebene unserer Zeit möglich wurden. Mit Tausenden von Jahren Arbeit lebt die Bevölkerung auf einer Fläche von 3.789.330 km 2 durchschnittlich 126,3 Menschen pro Quadratkilometer. Dies ist fast die Grenze für die Maximierung der Nutzung landwirtschaftlicher Flächen. Dies wird, wie Cressy richtig hervorhebt, aus Sicht der ökologischen Botanik eine Art Höhepunktbildung sein. „Hier haben wir eine alte stabilisierte Zivilisation, die die Ressourcen der Natur bis an ihre Grenzen nutzt. Bis neue äußere Kräfte Veränderungen bewirken, finden hier kleine und innere Verschiebungen statt.“



“Die chinesische Landschaft ist ebenso zeitaufwändig wie riesig im Weltraum, und die Gegenwart ist ein Produkt vieler Jahrhunderte. Wahrscheinlich lebten mehr Menschen auf den Ebenen Chinas als irgendwo anders auf einem ähnlichen Raum auf der Erde. Buchstäblich Billionen von 149 Männern und Frauen haben zu den Umrissen der Hügel und Täler und zum Bau der Felder beigetragen. Der Staub selbst wird durch ihr Erbe belebt. „ Diese viertausend Jahre alte Kultur musste, bevor sie ihre stabilisierte Form annahm, die Phasen einer gewaltigeren und tragischeren Vergangenheit durchlaufen, denn die Vergangenheit der chinesischen Natur verlief in einer völlig anderen Umgebung, in einer völlig anderen Natur, in feuchten Wäldern und Sümpfen, um sie zu unterwerfen und in die Kultur zu bringen Die Arten, die -- um Wälder zu zerstören und ihre Tierpopulation zu besiegen -- zehntausend Jahre brauchten. Jüngste Entdeckungen zeigen, dass zur gleichen Zeit wie in Europa eine Person die Bewegungen von Eismassen erlebte, in China eine Kultur unter den Bedingungen der Pluralperiode geschaffen wurde. 150 Offensichtlich sind die Wurzeln des Bewässerungssystems, dank dessen es in China Landwirtschaft gibt, weit in der Geschichte verwurzelt, 20.000 Jahre oder mehr. Bis zum Ende des 20. Jahrhunderts könnte eine solche Biozönose in einem bestimmten Gleichgewicht bestehen. Aber es konnte nur aufgrund der Tatsache existieren, dass China bis zu einem gewissen Grad isoliert war, dass die Bevölkerung von Zeit zu Zeit durch Morde, Sterben an Hunger und Hunger und an Überschwemmungen zerschnitten wurde; Die Bewässerungsarbeiten waren schwach, um die Kraft von Flüssen wie dem Gelben Fluss zu bewältigen. Jetzt gehört das alles schnell der Vergangenheit an.



In China sehen wir das jüngste Beispiel einer abgelegenen Zivilisation, die seit Jahrtausenden lebt. Wir sehen, dass die chinesische Wissenschaft zu Beginn des 18. Jahrhunderts, als sie hoch stand, an einer historischen Wende stand und die Gelegenheit verpasste, zur richtigen Zeit in die Welt der Wissenschaft einzutreten. Er trat erst in der zweiten Hälfte des 19. Jahrhunderts bei.



113. Die Landwirtschaft könnte sich nur dann als geologische Kraft manifestieren und die umgebende Natur verändern, wenn gleichzeitig die Viehzucht stattfindet, d. H. Als gleichzeitig mit der Auswahl und Zucht der Pflanzen, die Emu zum Leben brauchte, eine Person die Tiere auswählte und zu züchten begann, die sie brauchte. Eine Person führte unbewusst geologische Arbeiten durch, die eine bessere Reproduktion bestimmter Arten von Pflanzen- und Tierorganismen verursachten, immer verfügbare konzentrierte Nahrung erzeugten und Nahrung mit bestimmten Arten von Tieren versorgten, die sie benötigte. In der Viehzucht erhielt er nicht nur Futter, sondern erhöhte auch seine Muskelkraft, was es ihm ermöglichte, das früher von der Landwirtschaft besetzte Gebiet zu erweitern.



Bei der Arbeit mit Rindern erhielt er eine neue Energieform, die es ihm ermöglichte, einen größeren Teil der Bevölkerung zu ernähren, große Siedlungen und städtische Kultur zu schaffen und sich von der Bedrohung durch Hunger als unvermeidliches Phänomen zu befreien.



Er ging nicht über wild lebende Tiere hinaus.



In den letzten Jahrhunderten, in unserem Jahrhundert von Dampf und Elektrizität, sind die Arbeitskräfte von Nutztieren, die Muskelenergie von Tieren und Menschen in den Hintergrund des Wachstums der Landwirtschaft gerückt. Der Mensch geht jedoch auch jetzt noch nicht über die Grenzen der lebenden Natur hinaus, da die primäre Energiequelle für Elektrizität und Dampf dieselbe lebende Natur in Form lebender Vegetation oder noch mehr der früheren lebenden Organismen ist, die jetzt durch geologische Prozesse verändert werden. es wird aus Kohle und Öl gewonnen. Auf diese Weise nutzt ein Mensch letztendlich immer die Energie eines Sonnenstrahls, der durch lebende Materie gegangen ist, die modern oder in fossilem Licht erhalten ist und die Erde Hunderte von Millionen von Jahren beleuchtet, bevor eine Person darauf erscheint.



In der Landwirtschaft und Viehzucht manifestiert sich dies hauptsächlich in der gedankengesteuerten kulturellen biogeochemischen Energie, die neue Bedingungen für das Leben eines Menschen in der Biosphäre schafft. Auf diese Weise wurden hauptsächlich wild lebende Tiere drastisch verändert. Viele zehntausende Jahre lang wurde die inerte Substanz der Biosphäre vom Menschen nur in einem Ausmaß beeinflusst, das mit einer starken Veränderung seines Lebensumfelds nicht zu vergleichen ist.



Als Ergebnis dieser Arbeit wurde ein neues Gesicht der Erde geschaffen, in dem wir jetzt leben und das erst in den letzten Jahrtausenden spürbar geworden ist. Jetzt werden die Veränderungen mit jedem Jahrzehnt dramatischer.



Aber die Landwirtschaft allein, auch ohne Viehzucht, verändert die umliegende Natur dramatisch. Denn in der ihn umgebenden lebendigen Natur sind alle freien Bereiche mit lebender Materie gefüllt, und um ein neues Leben zu führen, muss eine Person ihren Platz reinigen, den Bereich von einem anderen Leben reinigen. Darüber hinaus muss er das Leben, das er schafft, ständig vor dem Druck des Lebens schützen -- vor Tieren und Pflanzen, die in den leeren Raum stürzen, den er öffnet. Er muss die Früchte seiner Arbeit vor Tieren und Pflanzen schützen, ohne dass sie sie fressen -- vor Säugetieren, Vögeln, Insekten, Pilzen usw. Wir sehen, dass er damit auch jetzt noch nicht ganz fertig wird.



Die Landwirtschaft leistet zusammen mit der Viehzucht, die ständig von menschlichen Gedanken und Arbeitskräften bewacht wird, am Ende eine großartige geologische Arbeit. Das alte Leben wird zerstört, ein neues wird geschaffen -- neue Tier- und Pflanzenarten, die durch das Denken und die Arbeit des Menschen geschaffen wurden und von den alten ausgehen, die in einer anderen Umgebung geschaffen wurden. Aber die Welt der wilden Tiere und Pflanzen, die nicht direkt vom Menschen berührt wird, verändert sich unweigerlich in einem neuen Lebensumfeld, das durch die biogeochemische Energie des Menschen geschaffen wird.



114. Die Viehzucht selbst ohne Landwirtschaft führt zu enormen Veränderungen in der Umgebung. Denn es nimmt Nahrung auf und verurteilt das langsame oder schnelle Aussterben großer Säugetiere, von denen der Mensch nur wenige Arten ausgewählt hat. Der Mensch erschien am Ende des Tertiärs, in der Ära des Königreichs in der Biosphäre -- wie Osborne 151 richtig hervorhob -- großer Säugetiere.



Gegenwärtig kann gesagt werden, dass diese Säugetiere praktisch entweder ausgestorben sind oder schnell verschwinden und nur in Reservaten und Parks erhalten bleiben, in denen sich ihre Anzahl in einem stationären Zustand befindet. Die Beobachtung in diesen großen Reserven zeigt, dass fast immer zusätzlich zum menschlichen Willen ein stationäres dynamisches Gleichgewicht hergestellt wird, bei dem die Fortpflanzung durch eine begrenzte Menge an Nahrung für Pflanzenfresser und die Anzahl der Raubtiere, die sie als Nahrung dienen, reguliert wird. 152 Bei einem Mangel an Nahrung -- einer Schwächung ihres Körpers -- wird dies außerdem durch die von lebenden Organismen verursachten Krankheiten bestimmt. Die gesamte verbleibende Anzahl wilder pflanzenfressender Säugetiere ist jedoch nicht mit der Anzahl der Haustiere (Pferde, Schafe, Rinder, Schweine, Ziegen usw.) vergleichbar. Man könnte meinen, dass ihre Anzahl im Tertiär kaum höher war als die Anzahl der modernen Haustiere. Diese Anzahl haben wir nicht wir wissen es ziemlich genau, aber wir haben dennoch eine bestimmte Vorstellung davon. Jetzt ist es hunderte Male größer als die Anzahl der Menschen auf der Erde. Nach M. Smith (1910) 153 entsprach es 1381011 zu Beginn des Jahrhunderts. H.Rew), 154 ist die Zahl 1929 für Pferde, Rinder, über Ec, Ziegen und Schweine erreichten 1571010. Die nicht berücksichtigten Arten von Haustieren werden die Reihenfolge der Zahlen nicht ändern. Wir können also sagen, dass sie in Milliarden zwischen 16 und 138 Milliarden schwankt und die Zahl der Menschen deutlich übersteigt. Diese Zahl schwankt stark. da es unter menschlicher Kontrolle steht. So verringerte sich nach I. Dufrenois 155 von 1900 bis 1930 die Zahl der Rinder um ein Viertel, ersetzt durch Maschinen. Wenn wir neue Energiequellen beherrschen, nimmt diese Menge vor unseren Augen schnell ab, da beispielsweise die Anzahl der Pferde, Esel und Maultiere aufgrund der Zunahme von Traktoren und Autos abnimmt.



115. Die Manifestation von Viehzucht und Landwirtschaft wurde vor 20 bis 7000 Jahren an verschiedenen Orten zur gleichen Zeit geschaffen und nahm in unserer Zeit allmählich an Intensität zu. Der Übergang vom nomadischen (nomadischen) Jagd- oder Nahrungssammelleben zum modernen Siedlungsleben zu einem Leben, das hauptsächlich auf der Landwirtschaft beruhte, fand zu verschiedenen Zeiten am Rande der Wüstenzone in den mittleren Breiten vom modernen Marokko bis zur Mongolei statt. Es ist möglich , dass dies eine Folge des Klimawandels war, nach dem Weggang der letzten der Eisdecke und der Schwächung der pluvial 156 Periode.



Vor sieben bis achttausend Jahren haben wir die ersten mächtigen Agrarstaaten und die ersten großen Städte. Der Mensch hatte die Möglichkeit, sich mit weniger Unterbrechungen frei zu reproduzieren. Eine städtische Zivilisation der keltischen, berberischen Staaten und ihrer Vorgänger wurde geschaffen -- Ägypten, Kreta, Kleinasien, Mesopotamien, Mesopotamien, Nordindien, China. Wir treten in Jahrhunderte ein (von denen wir überlebt haben, Legenden sind zu uns gekommen und unzählige materielle Denkmäler, die durch archäologische Ausgrabungen freigelegt wurden), deren Wert und Kraft in den letzten drei Jahrhunderten kontinuierlich und schnell zugenommen haben.



Man kann sagen, dass innerhalb von 5-7000 Jahren mit zunehmendem Tempo eine kontinuierliche Schaffung der Noosphäre und eine solide -- hauptsächlich ohne Rückwärtsbewegung, aber mit abnehmender Dauer -- die kulturelle biogeochemische Energie der Menschheit zunimmt. Das Bewusstsein wächst, dass diesem Wachstum keine unüberwindlichen Grenzen gesetzt sind, dass dies ein spontanes geologisches Phänomen ist.



116. Es ist zweckmäßig, einige Tatsachen zu zitieren. Ungefähr 4236 Jahre vor R.H. Es ist möglich, den ägyptischen Kalender (basierend auf den Langzeitbeobachtungen von Sirius) zu beginnen, der der Chronologie der gesamten Alten Welt bis zum gegenwärtigen Moment zugrunde liegt, als sich herausstellte, dass er über die gesamte Noosphäre verteilt war. 157 Noch früher als zu dieser Zeit, innerhalb von 5–4 Tausend Jahren vor Christus. In Indien, Mesopotamien und Kleinasien gab es eine städtische Kultur mit einer solchen Lebenstechnik, die wir vor einigen Jahren noch nicht gekannt hatten und die eine Bevölkerung von vielleicht Millionen Menschen umfasste. Gegen Ende dieser Zeit, dreitausend Jahre vor unserer Ära, begann die Bewegung auf Tieren, und innerhalb von anderthalb tausend Jahren war sie weit verbreitet und umfasste Stiere, Kamele und Pferde. 33 Jahrhunderte vor Christus wurde in den Tempeln Mesopotamiens ein Brief verwendet. Die Notizen wurden in schwieriger piktografischer Schrift und ungefähr 16-15 Jahrhunderte vor Christus gemacht. In Neuasien ist unter den Semiten das alphabetische Alphabet offen. Es kann gesagt werden, dass 2,5 Tausend Jahre vor R.Kh. Wir hatten eine klare Manifestation des wissenschaftlichen Denkens und über zweitausend Jahre in Mesopotamien -- die Öffnung des Dezimalsystems. Zu dieser Zeit wurden die alten -- einige Jahrhunderte vor den Aufnahmen -- kopiert und Bibliotheken aufbewahrt. Zwischen dem 15. und 14. Jahrhundert BC Wir sehen einen breiten Austausch in der damaligen Kulturwelt von Wissenschaftlern, Philosophen und Ärzten. Seit mehr als zweitausend Jahren wird Bronze entdeckt, anscheinend zur gleichen Zeit an verschiedenen Orten und etwa 1400 Jahre vor R.Kh. -- Eisen, das seit mehreren Jahrhunderten in Gebrauch ist.



Mit diesen enormen Errungenschaften kamen wir in die ersten Jahrhunderte vor Christus, als die wissenschaftliche, philosophische, künstlerische und religiöse Kreativität eine enorme Entwicklung erreichte und den Grundstein für unsere Zivilisation legte.



Während des letzten halben Jahrtausends ab dem fünfzehnten Jahrhundert. Bis zum zwanzigsten Jahrhundert setzte sich die Entwicklung des starken Einflusses des Menschen auf die umgebende Natur und sein Verständnis durch ihn fort und wurde immer stärker. Zu dieser Zeit umfasste eine einzige Kultur die gesamte Oberfläche des Planeten (§ 64): die Öffnung der Typografie, die Kenntnis aller bisher unzugänglichen Gebiete der Erde, die Beherrschung neuer Energieformen -- Dampf, Elektrizität, Radioaktivität, die Beherrschung aller chemischen Elemente und ihre Verwendung für die menschlichen Bedürfnisse, die Schaffung eines Telegraphen und Radio, Eindringen durch kilometerlange Bohrungen in die Erde und Aufstieg einer Person durch Luftmaschinen über 20 km von der Oberfläche des Geoids und durch Apparate -- über 40 km. Die tiefgreifenden sozialen Veränderungen, die die Massen unterstützt haben, haben ihre Interessen in erster Linie spezifisch zum Ausdruck gebracht, und die Frage der Beendigung von Hunger und Hunger ist real geworden und kann nicht außer Sichtweite geraten.



Die Frage nach einer geplanten, einheitlichen Aktivität zur Beherrschung der Natur und der korrekten Verteilung des Reichtums, die mit der Schaffung der Einheit und Gleichheit aller Menschen, der Einheit der Noosphäre, verbunden ist, ist an der Tagesordnung. Die Bewegung kann nicht gedreht werden, hat aber den Charakter eines heftigen Kampfes, der jedoch auf den tiefen Wurzeln des spontanen geologischen Prozesses beruht, der zwei oder drei Generationen dauern kann, vielleicht mehr (was nach der Entwicklungsrate im letzten Jahrtausend kaum wahrscheinlich ist) ) In diesem Übergangszustand, inmitten des intensiven Kampfes, in dem wir leben, scheint es unwahrscheinlich, dass die langen Stopps des laufenden Prozesses des Übergangs der Biosphäre in die Noosphäre ebenfalls unwahrscheinlich erscheinen.



Die wissenschaftliche Berichterstattung über die Biosphäre, die wir beobachtet haben, ist eine Manifestation dieses Übergangs.



Diese Nicht-Zufälligkeit und Verbindung mit der Struktur des Planeten -- seiner oberen Hülle 158 -- müssen wir in Zukunft -- wenn wir von den Konzepten der Biogeochemie sprechen -- einer tiefen, aufmerksamen logischen Analyse unterziehen.



All dies ist das Ergebnis einer genauen Beobachtung, und als solche sollte es, da es korrekt durchgeführt wird, als wissenschaftliche Verallgemeinerung berücksichtigt werden.



Dies ist eine wissenschaftliche Beschreibung eines Naturphänomens, die über den Rahmen seiner Hypothese, Theorie oder Extrapolation hinausgeht.



117. Wenn wir die etablierten wissenschaftlichen Disziplinen auf diese Weise betrachten, sehen wir deutlich die Existenz von Wissenschaften verschiedener Art, zum einen jene, deren Objekte -- und damit Gesetze -- die gesamte Realität umfassen -- sowohl unseren Planeten und seine Biosphäre als auch kosmische Weiten -- Dies sind Wissenschaften, deren Objekte den grundlegenden, allgemeinen Phänomenen der Realität entsprechen. Ein anderer Typ ist mit Phänomenen verbunden, die unserer Erde inhärent und charakteristisch sind.



In diesem letzteren Fall können theoretisch zwei Fälle von wissenschaftlich untersuchten wissenschaftlichen Objekten zugelassen werden: Planetenphänomene und individuelle, rein terrestrische Phänomene.



Jetzt kann man diese beiden Fälle jedoch nicht immer mit Sicherheit und mit ausreichender Sicherheit unterscheiden. Dies ist eine Frage der Zukunft.



Dies umfasst alle Biosphären-, Geistes- und Geowissenschaften -- Botanik, Zoologie, Geologie, Mineralogie -- in ihrer Gesamtheit.



Angesichts dieses Wissensstandes können wir in der Noosphäre die Manifestation des Einflusses zweier Bereiche des menschlichen Geistes auf seine Struktur unterscheiden: Wissenschaften, die allen Realitäten gemeinsam sind (Physik, Astronomie, Chemie, Mathematik) und Geowissenschaften (biologische, geologische und humanitäre Wissenschaften).



118. Eine Sonderstellung nimmt die Logik ein, die untrennbar mit dem menschlichen Denken verbunden ist und alle Wissenschaften gleichermaßen umfasst -- sowohl die Geisteswissenschaften einerseits als auch die mathematischen Wissenschaften andererseits.



Im Wesentlichen sollte es in den Bereich der planetaren Realität eintreten, aber nur dadurch kann eine Person die gesamte Realität verstehen und wissenschaftlich annehmen -- einen wissenschaftlich konstruierten Kosmos.



Wissenschaftliches Denken ist sowohl ein individuelles als auch ein soziales Phänomen. Es ist untrennbar mit dem Menschen verbunden. Ein Mensch kann bei der tiefsten Abstraktion das Feld seiner Existenz nicht verlassen. Wissenschaft ist ein echtes Phänomen und als Mensch selbst eng und untrennbar mit der Noosphäre verbunden. Die Persönlichkeit wird zerstört -- „aufgelöst“ -- wenn sie die logische Reichweite ihres Geistes verlässt.



Aber der Apparat des Geistes, der eng mit dem Wort und dem Konzept verbunden ist -- dessen logische Struktur, wie wir sehen werden, komplex ist (siehe den Exkurs zur Logik am Ende des Buches) -, deckt nicht das gesamte Wissen des Menschen über die Realität ab.



Wir sehen und wissen -- aber wir wissen auf alltägliche und nicht wissenschaftliche Weise, dass wissenschaftliches kreatives Denken über die Logik hinausgeht (auch in Logik und Dialektik in ihren verschiedenen Verständnissen). In seinen wissenschaftlichen Errungenschaften stützt sich ein Mensch auf Phänomene, die nicht durch Logik abgedeckt sind (egal wie erweitert wir sie verstehen).



Intuition, Inspiration -- die Grundlage der größten wissenschaftlichen Entdeckungen, die weiter auf streng logischen Grundlagen beruhen und auf dieser beruhen -- werden weder durch wissenschaftliches noch durch logisches Denken verursacht, sind nicht mit dem Wort und dem Konzept in seiner Entstehung verbunden.



In diesem Grundphänomen in der Geschichte des wissenschaftlichen Denkens betreten wir ein Feld von Phänomenen, das noch nicht von der Wissenschaft erfasst wurde, aber wir können nicht nur damit rechnen, wir müssen unsere wissenschaftliche Aufmerksamkeit darauf stärken.



Dies ist ein Bereich philosophischer Konstruktionen, der etwas herausgefunden hat, aber im Allgemeinen befindet sich der Bereich dieser Phänomene in einem chaotischen Zustand.



Am tiefsten und interessantesten ist es, dass es von der Philosophie der Hindus, sowohl ihrer alten als auch der modernen Suche, angenommen wird. Es gibt Versuche, sich in diesem Bereich zu vertiefen, der kaum von der Wissenschaft beeinflusst wird. 159 Wie tief es das menschliche Denken leiten kann, wissen wir wissenschaftlich nicht.



Wir sehen nur, dass ein riesiger Bereich von Phänomenen, die wissenschaftlich logisch sind, eng mit dem sozialen System und letztendlich mit der Struktur der Biosphäre -- und noch mehr der Noosphäre -- verbunden sind, eine Welt künstlerischer Konstruktionen ist, die in einigen Teilen ihrer eigenen, beispielsweise in der Musik oder Architektur, nicht reduzierbar sind , jede signifikante verbale Darstellung -- hat einen großen Einfluss auf die wissenschaftliche Analyse der Realität. Das Management dieses Erkenntnisapparats, das sich in der Logik für ein wissenschaftliches Verständnis der Realität wenig widerspiegelt, ist eine Frage der Zukunft.



119. Die Biogeochemie, deren Objekte Atome und ihre chemischen Eigenschaften sind, sollte größtenteils der Kategorie der allgemeinen Wissenschaften zugeordnet werden. Als Teil der Geochemie ist sie jedoch ebenso wie die Biochemie-Geochemie eine Wissenschaft des zweiten Typs, die mit einem kleinen definierten natürlichen Körper des Universums verbunden ist -- mit der Erde oder im allgemeinsten Fall -- mit dem Planeten.



Bei der Untersuchung der Manifestationen von Atomen und ihrer chemischen Reaktionen auf unserem Planeten geht die Biogeochemie mit ihren Wurzeln über den Planeten hinaus, stützt sich auf Atome wie Chemie und Geologie und verbindet auf diese Weise stärkere Probleme als die für die Erde charakteristischen mit der Wissenschaft der Atome. Atomphysik -- mit den Grundlagen unseres Verständnisses der Realität in ihrem kosmischen Kontext.



Dies ist weniger klar in Bezug auf die Phänomene des Lebens, die es unter dem Aspekt der Atome untersucht.



Gehen biogeochemische Probleme über den Planeten hinaus? Und wie tief ist ihr Ausgang?



 



Abteilung 4

Biowissenschaften im System des wissenschaftlichen Wissens



 



Kapitel 8



 



Ist das Leben eine ewige Manifestation der Realität oder vorübergehend? Die natürlichen Körper der Biosphäre leben und sind träge. Die komplexen natürlichen Körper der Biosphäre sind biokosal. Die Grenze zwischen Lebenden und Inerten ist in ihnen nicht unterbrochen.



 





120. Die Position des Lebens im wissenschaftlichen Universum ist uns völlig unklar. In der wissenschaftlichen Literatur wurde eine Tradition etabliert, um dieses Problem zu umgehen und es mit vollständig philosophischen und religiösen Konstruktionen zu versehen, die jetzt lose mit wissenschaftlichen verbunden sind und von realen, wissenschaftlich verlässlichen Konstruktionen der Wissenschaft unserer Zeit getrennt sind oder ihnen sogar widersprechen.



Weder das philosophische noch das religiöse Denken der modernen Menschheit kann mit der schnellen Wachstumsrate der Naturwissenschaften im 20. Jahrhundert Schritt halten. Infolgedessen verlieren philosophische oder religiöse Lösungen für Probleme in der Wissenschaft zunehmend an Bedeutung. Die Wissenschaft muss sich diesem Problem selbst nähern. Das ist nicht jetzt.



Wir wissen nicht nur nicht, wo wir die Lebenslinie in die wissenschaftliche Realität umsetzen sollen, sondern wir umgehen das Problem selbst in der Wissenschaft.



Wenn nun die Biogeochemie die Verbindung des Lebens nicht nur mit der zuvor bekannten Physik der Teilkräfte und mit chemischen Kräften, sondern auch mit der Struktur der Atome, mit Isotopen wissenschaftlich auf den neuesten Stand bringt, kann das wissenschaftliche Denken nicht in einer so trägen Position bleiben.



Es ist nicht bekannt, ob das Leben nur ein irdisches, planetarisches Phänomen ist oder ob es als kosmischer Ausdruck der Realität erkannt werden sollte, die Raum-Zeit, Materie und Energie sind. Man kann sich jetzt in der wissenschaftlichen Arbeit an jede dieser Ansichten halten, ohne genau festgelegten wissenschaftlichen Daten zu widersprechen. Die erste Idee, dass das Leben nur ein irdisches und kein planetarisches Phänomen ist, muss sich jedoch offenbar bald nicht mehr verteidigen.



Wissenschaftlich gesehen ist das Leben seit langem als ein auf der Erde einzigartiges Phänomen anerkannt. Wir können es nicht zweifellos als ein allgegenwärtiges planetarisches Phänomen betrachten, da für große Planeten, die weit von der Sonne entfernt sind, wie Jupiter, Saturn, Uranus (Pluto?), Niedrige Temperaturen das Leben irdisch ähnlich machen, unglaublich, wenn wir es betrachten dass es keine anderen Lebensformen gibt als die, die durch das thermodynamische und chemische Feld unserer Biosphäre definiert sind. Solche Ideen wurden zum Beispiel wiederholt von Preyer geäußert, der das Leben bei hohen Temperaturen zuließ. Bisher handelt es sich um wissenschaftliche Annahmen, die nicht auf Fakten beruhen, sondern von der hypothetisch zugelassenen Möglichkeit ausgehen. In Gebieten mit sehr niedrigen Temperaturen -- jenseits der in der Biosphäre möglichen Grenzen -- bleibt zweifellos ein latentes Leben bestehen, anscheinend auf unbestimmte Zeit.



Für unsere Erde wissen wir nicht mit einer signifikanten Wahrscheinlichkeit, dass sich geologische Ablagerungen in der Zeit ihrer Geschichte gebildet haben, als es kein Leben auf ihr gab. 160 Aber ihre tatsächliche Abwesenheit ist noch nicht vollständig bewiesen, und es ist möglich, zwei entgegengesetzte Vorstellungen zuzugeben: 1) Das Leben auf der Erde erschien innerhalb der geologischen Zeit, 2) es existierte bereits aus (der Zeit) der ältesten archaischen Gesteine, wie wir wissen. Geologen, die an dieser letzten Arbeitshypothese festhalten, äußern ihre Meinung, indem sie ihren Namen ändern -- Archäose statt Archä. Offensichtlich wird bei den ältesten archäischen Gesteinen eine Zunahme magmatischer Gesteine ​​beobachtet, und eine der Hauptaufgaben der Geologie ist nun die genaue wissenschaftliche Klärung dieser Idee. Haben wir die geologisch ältesten metamorphen Gesteine ​​lebloser Ablagerungen erreicht? Es gibt gute Gründe, dies zu bezweifeln, aber Zweifel sind keine Beweise. Die Lösung dieser Frage, die durchaus möglich ist, ist die Aufgabe des Tages.



Auf der anderen Seite deutet vieles darauf hin, dass sich das Leben jetzt nicht nur auf der Erde befindet, sondern auch auf anderen Planeten. Dies kann als mehr als wahrscheinlich angesehen werden.



Ziemlich plausible Hinweise auf die Möglichkeit der Existenz von Leben auf Mars und Venus, im Grunde ähnlich wie bei uns. Und hier ist die Frage in einem solchen Stadium, dass wir auf ihre schnelle, unbestreitbare wissenschaftliche Lösung in die eine oder andere Richtung warten können. Dies ist noch nicht geschehen, aber eine positive Lösung scheint mir am wahrscheinlichsten.



Unter den gegebenen Umständen scheint es mir möglich zu sein, zu berücksichtigen, dass in naher Zukunft die Existenz eines planetarischen und nicht nur irdischen Lebens in der Realität hergestellt wird.



121. Auf dieser Grundlage ist es bereits wissenschaftlich möglich, in der Wissenschaft eine allgemeine Frage zu stellen, ob das Leben nur ein irdisches Phänomen oder nur ein Merkmal von Planeten ist oder ob es in gewissem Umfang und in irgendeiner Form großräumige Phänomene, Phänomene, widerspiegelt Kosmische Weiten, so tief und ewig wie Atome, Energie und Materie sind für uns und enthüllen geometrisch die Raumzeit.



Angesichts der schlechten Entwicklung unseres Wissens auf diesem Gebiet kann man sogar zugeben, dass irdisches und sogar planetarisches Leben ein Sonderfall der Manifestation des Lebens ist, da ein Sonderfall der Manifestation elektrischer Phänomene das Nordlicht oder Gewitter der Erdatmosphäre sein wird. Wir befinden uns hier in einem fast fremden Wissenschaftsfeld wissenschaftlicher Hypothesen und sogar wissenschaftlicher Fantasie, das nur als die Idee des Lebens in für die Erde ungewöhnlichen Temperatur- und Gravitationsgebieten betrachtet werden kann.



Wir können nicht einmal eine solche Annahme wissenschaftlich verwerfen. Wir sind also weit von einem wissenschaftlichen Verständnis des Lebens entfernt.



In der Philosophie -- in ihren entgegengesetztesten Systemen -- wurde die Frage nach der Ewigkeit des Lebens viele Male gestellt und gestellt. In einer Reihe von philosophischen Systemen wird das Leben als eine der wichtigsten immerwährenden Manifestationen der Realität angesehen. 161



Die Frage nach dem Leben im Weltraum muss nun in der Wissenschaft gestellt werden. Dies wird durch eine Reihe empirischer Daten angeführt, auf denen die Biogeochemie aufbaut, eine Reihe von Fakten, die darauf hinweisen, dass das Leben zu denselben allgemeinen Erscheinungsformen der Realität gehört wie Materie, Energie, Raum, Zeit; In diesem Fall fallen die biologischen Wissenschaften zusammen mit den physikalischen und chemischen Wissenschaften in die Gruppe der Wissenschaften über allgemeine Phänomene der Realität.



122. In der Biogeochemie ist es zweckmäßig, in dieser Hinsicht das logische Konzept der konkreten Naturwissenschaften zu verwenden, das sich in der Biosphäre besonders vielfältig und anschaulich manifestiert, jedoch dem philosophischen und logischen Denken wenig Beachtung schenkt. Obwohl sie es unweigerlich benutzen, scheint es mir, dass sie seine Bedeutung nicht ganz verstehen.



Ich kenne seine eingehende philosophische und logische Analyse nicht.



Dieses Konzept ist das Konzept eines natürlichen Körpers. Als natürlicher Körper in der Biosphäre bezeichnen wir jedes Objekt als logisch von der Umwelt abgegrenzt, das durch natürliche Prozesse in der Biosphäre oder in der Erdkruste im Allgemeinen entstanden ist.



Jedes Gestein (und die Formen seines Standorts -- Batholith, Bestand, Schicht usw.) wird ein so natürlicher Körper sein, es wird jedes Mineral (und seine Formen), jeden Organismus als Individuum und als komplexe Kolonie, Biozönose (einfach und komplex) geben ), jeder Boden, Schlick usw., eine Zelle, ihr Kern, Gen, Atom, Atomkern, Elektron usw., Kapitalismus, Klasse, Parlament, Familie, Gemeinschaft usw., Planet, Stern und usw. -- Millionen von Millionen möglicher „natürlicher Körper“. Wie aus den obigen Beispielen ersichtlich ist, sehen wir hier zwei Kategorien von Konzepten. Einige von ihnen entsprechen Konzepten, deren Thema tatsächlich in der Natur existiert und nicht nur die Schaffung eines logischen Prozesses ist. Zum Beispiel ein bestimmter Planet, ein bestimmter Boden, ein bestimmter Organismus usw. Andererseits sind Konzepte, die ganz oder teilweise die Schaffung eines komplexen logischen Prozesses darstellen, eine Verallgemeinerung unzähliger Fakten oder logischer Konzepte. Zum Beispiel Boden, Fels, Stern, Staat usw.



Wissenschaft wird tatsächlich durch die Isolierung natürlicher Körper aufgebaut, und in der wissenschaftlichen Arbeit ist es gleichzeitig wichtig, nicht nur die ihnen entsprechenden Konzepte, sondern auch tatsächlich existierende wissenschaftlich definierte natürliche Körper genau zu berücksichtigen.



Für einen natürlichen Körper stimmen ein Wort und ein Konzept unweigerlich nicht überein.



Das ihm entsprechende Konzept ist nicht dauerhaft und unveränderlich, es ändert sich manchmal sehr stark und im Wesentlichen im Verlauf der wissenschaftlichen Arbeit, im Verlauf des menschlichen Lebens.



Das Wort, das dem Konzept des natürlichen Körpers entspricht, kann Jahrhunderte und Jahrtausende lang existieren.



Die Philosophie geht unweigerlich nicht über die Grenzen von Begriffswörtern hinaus. Sie hat keine Möglichkeit, sich Konzeptobjekten zu nähern. Dies ist der Hauptunterschied zwischen der logischen Arbeit eines Wissenschaftlers und eines Philosophen.



Es gab zum Beispiel eine Zeit in der Ära des Demokrit aus Abdera, in der es anders war. Aber jetzt ist diese Zeit unwiderruflich vergangen.



Im Gegensatz zur Philosophie beschränkt sich die Wissenschaft in der logischen und methodischen Analyse niemals auf Wörter, die natürlichen Körpern entsprechen. Es wird direkt -- ständig überprüft durch wissenschaftliche Erfahrung und Beobachtung -- mit den entsprechenden natürlichen Körpern selbst betrachtet.



Dieser Unterschied zeigt sich besonders deutlich auf dem Gebiet der exakten Naturwissenschaften im Vergleich zu einem großen Bereich der Probleme der Geisteswissenschaften. Obwohl in den Geisteswissenschaften eine direkte Anziehungskraft auf „natürliche Körper“ unvermeidlich ist und alles mit der Verfeinerung der wissenschaftlichen Arbeit zunimmt. In dieser Hinsicht XIX und XX Jahrhunderte. hier glätten sie den wesentlichen Unterschied zu den Naturwissenschaften. Die Genauigkeit und Zuverlässigkeit der Wissenschaften des Menschen, die selbst ein „natürlicher Körper“ für das wissenschaftliche Denken sind, ist bereits gewachsen. Wir sind erst zu Beginn des Wandels anwesend.



Ich werde später auf Fragen eingehen, die mit der logischen Bedeutung des „natürlichen Körpers“ zusammenhängen. 162



Ich spreche hier nur insoweit an, als es zum Verständnis des Folgenden notwendig ist.



Ich stelle fest, dass dieses Thema in der modernen Logik nicht genügend Beachtung gefunden und nicht wissenschaftlich weiterentwickelt wurde. Und doch hatte der große Naturforscher und Philosoph Demokrit (und wahrscheinlich sogar ein früherer Denker Leucippus) vor mehr als 2500 Jahren, noch vor Aristoteles, eine klare Vorstellung von diesem Problem -- aber es erstarrte, als Aristoteles 'Logik wissenschaftliches und philosophisches Denken umfasste. Wir können nur auf die wahrscheinliche Entwicklung der Ideen und Werke des Demokrit schließen, auf die Existenz der Literatur in den Jahrhunderten vor Beginn unserer Ära, die sie widerspiegelte.



All diese Literatur ist vor mehr als tausend Jahren verschwunden, und nur archäologische Ausgrabungen können sie uns vielleicht öffnen.



Aber Tatsache war. Es existierte und beeinflusste jahrhundertelang das kreative Denken des Menschen in der Biosphäre, aber der Verlauf seiner Identifizierung und seines Verblassens ist uns unbekannt. 163 Anscheinend begegnen wir unabhängig und in der Geschichte der indischen Logik in engen Jahrhunderten demselben Phänomen.



Wahrscheinlich. Ein und dieselben Gründe haben es verursacht: das Fehlen gesellschaftspolitischer Lebensbedingungen für die Entwicklung der Technologie und für die Identifizierung der druckfreien Religion und Philosophie der wissenschaftlichen Arbeit des Einzelnen.



123. In der Biogeochemie werden natürliche Körper, die für die Biosphäre charakteristisch sind, vorgeschlagen -- lebende natürliche Körper und komplexe natürliche Körper aus inerten und lebenden -- Biokosekörpern -, die außerhalb der Biosphäre nicht existieren.



Einige dieser natürlichen Körper wurden schon vor vielen Zehntausenden von Jahren vor der Entdeckung der Wissenschaft identifiziert und identifiziert -- hervorgehoben durch das Alltagsleben. Dies sind Menschen, Tiere, Pflanzen, Wälder, Felder usw. Eine große Anzahl von ihnen wird zugewiesen und wird ständig von einer Wissenschaft zugewiesen. Wie zum Beispiel Plankton, Benthos usw. Die Bewegung des wissenschaftlichen Denkens wird in erster Linie durch die Genauigkeit und Anzahl solcher Einrichtungen natürlicher Körper bestimmt, deren Anzahl im Laufe der wissenschaftlichen Zeit kontinuierlich wächst, zusammen mit der Einrichtung neuer natürlicher Körper, die alten werden verfeinert, und manchmal wird bei der Analyse alter Konzepte eine neue Wissenschaft geschaffen.



Als lebendiges Beispiel für diese Art von Prozess (an dem ich in meiner Jugend teilnehmen musste und an dem meine Gedanken wuchsen) reicht es aus, sich an die Schöpfung in Russland am Ende des 19. Jahrhunderts zu erinnern und darüber nachzudenken. eine mächtige Bewegung auf dem Gebiet der Etablierung eines neuen Bodenkonzepts, die zu einem neuen Verständnis der Bodenkunde führte. In der damaligen Literatur vor allem unter dem Einfluss der Gedanken eines großen Naturforschers V.V. Dokuchaev, wir werden zahlreiche Echos der Klärung in einem neuen Licht des alten Konzepts des Bodens als natürlichen Körper finden, über das lange vor Dokuchaev gesprochen wurde, das aber nicht verstanden wurde. 164 Die Idee des Bodens als natürlichen Körper, der sich von Gesteinen und Mineralien unterscheidet, ist von zentraler Bedeutung, und wie immer war das Verständnis von V.V. Dokuchaev nicht das einzige und letzte. 165



Ein neues Konzept eines natürlichen Körper ist die Idee der lebenden Substanzen, als eine Reihe von lebende Organismen, 166 zugrunde liegenden Geochemie daher und Biogeochemie.



124. Es ist äußerst charakteristisch, dass in der Biosphäre natürliche Körper scharf ausgeprägter Natur beobachtet werden. Natürliche Körper sind inert -- zum Beispiel Mineralien, Gestein, Kristalle, im Labor erzeugte chemische Verbindungen, menschliche Arbeitsprodukte, Nester, Hydrometeore, Vulkanprodukte usw. Lebende Organismen unterscheiden sich stark von ihnen -- lebende natürliche Körper -- alle Millionen ihrer Spezies und alle Millionen ihrer Individuen. Gruppe lebender Organismen -- lebende Substanzen sind auch natürliche Körper -- lebende, wie Gruppen von Unteilbaren derselben Art -- homogene lebende Körper oder verschiedene Arten -- morphologisch unterschiedliche, heterogene lebende Körper Es gibt eine Reihe anderer komplexer lebender natürlicher Körper, zum Beispiel Biozönosen usw.



In der Biosphäre kann man viele natürliche Körper unterscheiden, die gleichzeitig aus lebender und inerter Materie bestehen. Solche zum Beispiel Böden, Schlick usw. Das Studium solcher natürlichen Körper spielt in der Wissenschaft eine große Rolle, da man in ihnen den Prozess des Einflusses des Lebens auf die inerte Natur untersuchen kann -- dynamisches, stabiles Gleichgewicht, Organisation der Biosphäre. Es ist möglich, unzählige solcher komplexen natürlichen Systeme, die dem System entsprechen, logisch zu konstruieren: lebende natürliche Körper -- inerte natürliche Körper , beginnend mit jenen, in denen lebende natürliche Körper nach Masse fast die gesamte Substanz des Systems, fast die gesamte Masse eines komplexen natürlichen Körpers, bis zu jenen, in denen In Bezug auf das Gewicht überwiegen inerte natürliche Körper auf die gleiche Weise oder sogar noch intensiver.



Es ist zweckmäßig, noch inerte natürliche Körper zu trennen, die durch den Lebensprozess entstanden sind, beispielsweise Kohlen, Kieselgur, Kalksteine, Öle, Asphalte usw., in deren Struktur und Eigenschaften wir den früheren Einfluss des Lebens wissenschaftlich nachweisen können.



125. Obwohl ich später ausführlicher auf die Bedeutung des Begriffs der Naturkörper in der Logik der Naturwissenschaften zurückkommen werde, finde ich es in dieser Einleitung nützlich, einige Merkmale, die die Arbeit eines Wissenschaftlers von der Arbeit eines Philosophen unterscheiden, auf dieses grundlegende Objekt der Wissenschaft (und nicht nur der Naturwissenschaft) hervorzuheben.



Der Philosoph nimmt das Wort, das den natürlichen Körper nur als Konzept definiert, und zieht daraus alle Schlussfolgerungen, die sich logisch aus einer solchen Analyse ergeben.



In harmonischen Systemen kann er aus einer solchen Analyse so tiefe, wenn auch unvollständige Schlussfolgerungen ziehen, dass der Wissenschaftler Neues in ihm entdeckt und das er berücksichtigen muss. Denn neben der natürlichen Begabung des Einzelnen erfordert die philosophische Analyse das Lernen, das sich über Jahrtausende entwickelt hat. Es erfordert Gelehrsamkeit und schwieriges Denken, es erfordert ein Leben lang. Insbesondere in breiten und umfassenden natürlichen Körpern, zum Beispiel in den Konzepten von Realität, Raum, Zeit, Raum, menschlichem Geist usw., kann ein Wissenschaftler im Allgemeinen nicht so tief und doch so deutlich gehen wie ein Philosoph. Im Allgemeinen hat er nicht genug Zeit und Energie dafür.



Ein Wissenschaftler sollte verwenden -- sich einer kreativen und suchenden philosophischen Arbeit bewusst sein -, aber seine unvermeidliche Unvollständigkeit und unzureichende Genauigkeit bei der Bestimmung natürlicher Körper auf dem durchzuführenden Gebiet nicht vergessen. Er sollte immer Korrekturen an den Schlussfolgerungen des Philosophen vornehmen und dabei den Unterschied zwischen den realen natürlichen Körpern, die er studierte, und den Konzepten über sie (die Wörter sind in beiden Fällen dieselben) berücksichtigen, mit denen der Philosoph arbeitet. Diese Änderungen in einigen Perioden der wissenschaftlichen Entwicklung können, wie es in unserer Zeit der Fall ist, die Schlussfolgerungen des Philosophen radikal verändern und ihre Bedeutung für den Naturforscher vollständig schwächen.



Der Wissenschaftler, der das Konzept eines bestimmten natürlichen Körpers logisch analysiert, kehrt ständig zu seiner wissenschaftlichen Forschung zurück -- nach Anzahl und Maß wie ein natürlicher Körper.



Während der wissenschaftlichen Arbeit kehren Wissenschaftler häufig direkt zur Revision der Eigenschaften eines natürlichen Körpers durch Maß und Gewicht, Erfahrung, Beschreibung und Verfeinerung der Beobachtung zurück, tausende Male über Jahrzehnte, Jahrhunderte. Infolgedessen kann sich die gesamte Idee eines natürlichen Körpers radikal ändern. So änderten sich die Vorstellungen des Naturforschers über Quarz, natürliches Wasser oder Nagetiere als natürliche Körper im 18., 19. und 20. Jahrhundert radikal, und die logisch korrekten Schlussfolgerungen in diesen Jahrhunderten erwiesen sich als ungenau. Vieles, „selbstverständlich“ im 19. Jahrhundert und früher -- wird sich in unserer Zeit als falsch herausstellen -- und „selbstverständlich“ in unserer Zeit wird sich im 21. Jahrhundert als falsch herausstellen.



Wir haben dies dank neuer wissenschaftlicher Entdeckungen in natürlichen Körpern wie beispielsweise Raum-Zeit oder Wasser deutlich erlebt.



Der Philosoph ist nun gezwungen, mit der Existenz von Raum-Zeit zu rechnen und nicht mit zwei „natürlichen Körpern“ -- Raum und Zeit, unabhängig voneinander. In diesem Fall konnte er es philosophisch ableiten, aber der Philosoph konnte die Richtigkeit seiner Schlussfolgerung nicht beweisen. Separate Philosophen -- letztendlich durch Intuition -- kamen offenbar zu dieser Idee und beeinflussten das wissenschaftliche Denken, aber nur wissenschaftliches Denken und wissenschaftliche Arbeit bewiesen die Unvermeidlichkeit, die Realität der Raumzeit als einen einzigen umfassenden natürlichen Körper zu erkennen, von dessen Grenzen vorerst oder vielleicht Vielleicht kann im Wesentlichen ein wissenschaftlicher Gedanke, der die Realität studiert, nicht herauskommen. 167



Aus der ganzen Summe unseres genauen Wissens wird nun klar, dass die Untrennbarkeit der Raumzeit eine empirisch-wissenschaftliche Position ist, die fest in das 20. Jahrhundert eingetreten ist. in die wissenschaftliche Arbeit.



Anstelle von zwei natürlichen Körpern -- Raum und Zeit -- stellte sich eines heraus. Ende des 17. Jahrhunderts. Ihre getrennte Existenz wurde von Newton mathematisch gerechtfertigt und führte in der Gravitationstheorie zu großen wissenschaftlichen Errungenschaften. Im Denken von Newton, der dazu kam, ist der Einfluss philosophischer und theologischer Ideen deutlich sichtbar. Newton selbst, der der Theologie eine entscheidende Bedeutung beimaß, hielt sie nicht für untrennbar miteinander verbunden. Nur in unserer Zeit haben wir eine neue tiefe Wende durchlaufen, und im Kosmos-System erschien die Raumzeit als untrennbar vereint und bedeckte sie anscheinend vollständig, aber vielleicht nicht identisch mit ihr.



In diesem Beispiel sehen wir deutlich, dass die natürlichen Körper der Realität in ihrer Komplexität heterogen sind. In der Raumzeit ist es möglich, dass alle natürlichen Körper, die wissenschaftlich erfasst sind, enthalten sind. 168



126. In einem anderen speziellen Beispiel -- Wasser -- haben wir eine spezifischere und eindeutigere Idee.



Das Konzept des Wassers bis zum Ende des 18. Jahrhunderts war äußerst vage. Bei Beobachtungen der Natur gab es jedoch nur in wenigen Fällen Zweifel an ihrer tatsächlichen Existenz, wo sie jetzt für uns eine elementare wissenschaftliche Wahrheit ist. So war es mit absolut trockenen Körpern oder mit unsichtbarem Wasserdampf. Erst in unserer Zeit ist das Hauptphänomen des Eindringens der gesamten Biosphäre und anscheinend der gesamten Erdkruste durch einen einzigen natürlichen Körper klar geworden -- den Wasserhaushalt der Erdkruste. 169 Zahlreiche, teils fantastische, teils wissenschaftsähnliche Darstellungen von Naturphilosophen und Theosophen verschwinden, setzen sich bis in unsere Zeit fort und haben wahrscheinlich Unterstützung in der Psychologie der Massen für ihre ständige Identifizierung.



Es ist möglich, dass diese wissenschaftliche Verallgemeinerung einen Rest aufweist, der noch nicht von der Wissenschaft abgedeckt ist, der solchen Suchen nicht entspricht, sie aber erregt.



Ende des 18. Jahrhunderts. Die chemische quantitative Zusammensetzung von Wasser wurde bestimmt, und seitdem hat sich das Konzept des Wassers so dramatisch verändert, dass die philosophische Analyse des Wassers, seine naturphilosophische Untersuchung, zu einem Anachronismus geworden ist. Eine grundlegende Änderung ist eingetreten. Dies geschah nicht sofort -- durch Trägheit wurde die fruchtlose Arbeit der Naturphilosophen, die jetzt völlig vergessen ist, im 19. Jahrhundert fortgesetzt. noch ein paar Generationen.



Das Interesse an diesen Themen verschwand in der westlichen Philosophie erst in den 1830er Jahren, als die fantastische schöpferische Arbeit der Naturphilosophen begann, den Erfolgen wissenschaftlicher Erkenntnisse zu stark zu widersprechen. Etwa zur gleichen Zeit und ein oder zwei Jahrzehnte später wurde das wissenschaftliche Konzept des Wassers schließlich vom indischen philosophischen Denken akzeptiert und berücksichtigt, das zu dieser Zeit zumindest auf dem Niveau der westlichen Philosophie stand, wenn nicht sogar höher.



Im zwanzigsten Jahrhundert. Wir erleben eine neue, nicht weniger drastische Veränderung in unserem Verständnis dieses natürlichen Körpers, die uns dazu zwingt, alle unsere Vorstellungen von Wasser in der Natur und insbesondere in der Biosphäre radikal zu überdenken. Die Komplexität der Struktur allen Wassers wurde offenbart, zuerst assoziativ, dann unvermeidlich elektrolytisch zersetzt Schließlich liegt der physikochemische Unterschied zwischen seinen Molekülen selbst aufgrund der Existenz mehrerer Wasserstoff- und Sauerstoffstoffe 170 an der Grenze von 18 verschiedenen Kombinationen -- und wenn wir die möglichen Assoziationen von Molekülen und deren elektrolytischem Di berücksichtigen Verein, dann Hunderte von chemisch reinen Gewässern mit unterschiedlichen Strukturen.



Alle Versuche, die „philosophische“ Untersuchung von Gewässern fortzusetzen -- wenn wir mystische Ideen ignorieren, die im wissenschaftlichen Bereich nicht speziell berücksichtigt werden -, sind für den Wissenschaftler ein klarer Anachronismus, und dieser Bereich ist über den Rahmen der philosophischen Kreativität hinausgegangen.



Wir begegnen jedoch auch Versuchen theosophischer Suchen, die weit entfernt von Philosophie und Wissenschaft sind und näher an den ersteren liegen -- den Früchten der Unwissenheit und der Suche nach anderen unzähligen Wegen der Logik der Natur als dem harten und dornigen Weg der Wissenschaft.



127. Aus dem Vorhergehenden geht die enorme logische Bedeutung des Konzepts des natürlichen Körpers für die wissenschaftliche Arbeit hervor.



Es ist so großartig, dass ein Naturforscher normalerweise nicht darüber nachdenkt.



In Wirklichkeit ist für den wissenschaftlichen Denker jede Realität, jeder wissenschaftlich konstruierte Kosmos ein natürlicher Körper, der sich in der Raumzeit befindet. Andernfalls kann ein Wissenschaftler nicht arbeiten, kann nicht wissenschaftlich denken.



Für einen Wissenschaftler besteht natürlich kein Zweifel an der Realität des Themas der wissenschaftlichen Forschung und kann es nicht sein, da er wie ein Wissenschaftler arbeitet und denkt.



Ein einziger, miteinander verbundener, wissenschaftlich bestimmter Kosmos ist für ihn -- weil Erfahrung, Beobachtung sowie logische und mathematische Analyse den anderen -- den natürlichen Hauptkörper -- nicht zeigen. Ob die Raumzeit damit übereinstimmt, wird die wissenschaftliche Forschung zeigen. Während das Gebiet der wissenschaftlichen Forschung keine Raum-Zeit verlässt. Aber der Wissenschaftler muss die Möglichkeit zulassen -- d.h. muss wissenschaftlich studieren -- alle möglichen Kombinationen der Identität des Kosmos, wissenschaftlich ausgedrückt mit Raum-Zeit und seiner Nichtübereinstimmung. Dies ist ein ungelöstes wissenschaftliches Forschungsproblem.



Ebenso kann das Problem eines einzelnen Kosmos, der wissenschaftlich ausgedrückt wird, nicht als wissenschaftlich gelöst angesehen werden. Unsere Erde tritt als Teil des Sonnensystems ein. Das Sonnensystem tritt -- zusammen mit Millionen solcher Systeme -- als untrennbarer Teil in eine bestimmte kosmische Insel ein -- eine bestimmte Galaxie. Sind andere existierende Galaxien, die wir beobachten können, miteinander verwandt? Logische Einschränkungen zur Behebung dieses Problems sind jetzt nicht mehr sichtbar.



Der Mensch, die Biosphäre, die Erdkruste, die Erde, das Sonnensystem, ihre Galaxie (die Weltinsel der Sonne) sind natürliche Körper, die untrennbar miteinander verbunden sind. Es gibt ein und dieselbe Raumzeit für alle, aber es wurde noch nicht entschieden, ob die Raumzeit alle in diesen Räumen wissenschaftlich zugänglichen Phänomene abdeckt oder nicht.



Es ist auch nicht wissenschaftlich belegt, ob beispielsweise Nebel und andere Weltinseln -- Galaxien -- ein untrennbarer Bestandteil eines einzigen -- unseres -- Kosmos sind. Dies ist nur wissenschaftlich wahrscheinlich und die Notwendigkeit einer anderen Sichtweise in der wissenschaftlichen Arbeit nicht.



Kapitel 9



 



Biogeochemische Manifestation einer unpassierbaren Linie zwischen lebenden und inerten natürlichen Körpern der Biosphäre.



 





128. Die Biogeochemie führt in die wissenschaftliche Untersuchung von Lebensphänomenen eine völlig andere Interpretation natürlicher lebender Körper ein -- lebende Organismen, Biozönosen, lebende Substanzen, heterogene und homogene usw. und komplexer inerter lebender biokositischer natürlicher Körper -- Böden, Schlick usw. . als das, was der Biologe in seiner tausendjährigen Arbeit gewohnt ist.



Es führt ein neues Verständnis der Tierwelt ein, das im Wesentlichen nicht dem alten widerspricht, sondern es ergänzt und vertieft.



Betrachtet man einen lebenden Organismus im Hinblick auf die Biosphäre, so wendet er sich seinen Atomen zu, die untrennbar mit den Atomen verbunden sind, die die Biosphäre bilden. Das Leben manifestiert sich in der kontinuierlichen Bewegung von Atomen von der Biosphäre zur lebenden Materie auf planetarischer Ebene einerseits und in ihren umgekehrten Migrationen von lebender Materie zur Biosphäre andererseits.



Lebende Materie ist die Gesamtheit der in der Biosphäre lebenden Organismen -- lebende natürliche Körper -- und wird auf planetarischer Ebene untersucht, während das unteilbare Individuum, auf das die Aufmerksamkeit des Biologen gerichtet ist, auf der Skala der in der Biogeochemie untersuchten Phänomene auf den zweiten Platz fällt. Die Migration chemischer Elemente, die der lebenden Materie der Biosphäre entsprechen, ist ein riesiger planetarischer Prozess, der hauptsächlich durch die kosmische Energie der Sonne verursacht wird und die Geochemie der Biosphäre und die Regelmäßigkeit aller darauf auftretenden physikochemischen und geologischen Phänomene aufbaut und bestimmt, die die Organisation der Erdhülle bestimmen.



Im nächsten Aufsatz -- über Biosphäre und Noosphäre -- werde ich dieses Phänomen, soweit wir es jetzt wissen, betrachten. 171



129. Ein lebender Organismus, der unter atomaren Gesichtspunkten und in seiner Gesamtheit betrachtet wird, zeigt sich in der Biogeochemie in einem völlig anderen Ausdruck als einem völlig anderen natürlichen Körper als in der Biologie, auch wenn der Biologe ihn auch in seiner Gesamtheit untersucht hat -- Biozönosen, Pflanzengemeinschaften, Herden, Wälder, Wiesen usw.



Die Biogeochemie erreicht die Atome chemischer Elemente, Isotope und durchdringt die Phänomene des Lebens in einem anderen Aspekt als der Biologe -- in mancher Hinsicht tiefer, in anderen jedoch verliert sie wichtige Merkmale der in der Biologie vorgebrachten Lebensphänomene aus ihrem Horizont.



Das morphologisch und physiologisch exakte Erscheinungsbild der lebenden Natur und insbesondere lebender Individuen ist eine Hilfsrepräsentation in den Phänomenen des Lebens in der Biogeochemie. Der Biologe kommt der gewöhnlichen und farbenfrohen Welt der Phänomene näher, in denen wir wild lebende Tiere umarmen, deren untrennbarer Teil wir darstellen. Die von den Biowissenschaften untersuchte Tierwelt ist unseren sensorischen Wahrnehmungen näher als ihr abstrakterer, anderer Ausdruck, der durch die Biogeochemie gegeben wird.



Andererseits drückt es deutlich solche Manifestationen des Lebens aus, die in der biologischen Herangehensweise an die Phänomene des Lebens in den Hintergrund treten.



Dies zeigt sich am besten in der Interpretation von Körpern und in anderen Ansätzen zu den Phänomenen des Lebens natürlicher Körper, insbesondere taxonomischer Einheiten -- Arten, Unterarten, Rassen, Gattungen usw.



Offensichtlich sind alle wichtigen Schlussfolgerungen der Biologie -- da sie auf genauen wissenschaftlichen Beobachtungen und Experimenten beruhen und auf Fakten und empirischen Verallgemeinerungen basieren, die logisch korrekt auf ihnen basieren -- wissenschaftliche Errungenschaften, die nicht im Widerspruch zu biogeochemischen Fakten und empirischen Verallgemeinerungen stehen können, die ebenso wissenschaftlich belegt sind .



Auf dieser Grundlage wird deutlich, dass alle natürlichen lebenden Körper, die den taxonomischen Einheiten des Biologen entsprechen, einen neuen Ausdruck erhalten, der sich grundlegend vom vorherigen taxonomischen Ausdruck des Biologen unterscheidet, aber im Wesentlichen identisch ist.



130. Es ist am bequemsten, dies an einem bestimmten Beispiel auszudrücken, an einer taxonomischen Unterteilung -- Gattung, klare Linie, Unterart, Art usw.



Ich werde mich auf die Ansicht konzentrieren.



Eine Art ist für einen Biologen eine Kombination von morphologisch homogenen Unteilbaren. Er ist in der Biogeochemie ziemlich verantwortlich für die lebende Materie eines Biogeochemikers.



Für einen Biologen wird dies durch die Form des Körpers, die histologische und anatomische Struktur, die physiologischen Funktionen, die Art des Integuments, die Phänomene der Ernährung, der Fortpflanzung usw. bestimmt.



Die wichtigste ist die Dauer der Manifestation derselben morphologischen und physiologischen Struktur des Körpers durch Reproduktion während der geologischen Zeit. Der Biologe sieht darin eine Manifestation der Phänomene der Vererbung. Die morphologische und physiologisch genaue Beschreibung durch den Biologen liegt seiner taxonomischen Aussage zugrunde. Die chemische Zusammensetzung beginnt den Biologen erst ernsthaft zu interessieren.



Numerische Daten -- Gewicht, Volumen, Reproduktion, Größe -- werden bei weitem nicht immer angegeben, sondern in qualitativer Manifestation -- gelegentlich zur quantitativen Veranschaulichung: Ihre maximale Genauigkeit -- numerischer Durchschnittsausdruck und die numerisch ausgedrückten Schwankungsgrenzen -- fehlen normalerweise.



131. Für einen Biogeochemiker wird eine biologische Spezies hauptsächlich durch die genauen Zahlenwerte des unteilbaren Durchschnitts bestimmt, dessen Gesamtheit eine lebende Spezies ist, die dem Typ des Biologen entspricht.



Alle Artenmerkmale in biogeochemischen Begriffen sollten quantitativ genau und in mathematischen Größen ausgedrückt werden -- numerisch und geometrisch. Für den geometrischen Ausdruck ist es bei der Verfeinerung der Arbeit unweigerlich notwendig -- und anscheinend ist es immer möglich -, nach ihrer quantitativen Identifizierung zu streben.



Daher sollte ein biogeochemischer lebender Organismus in seiner Gesamtheit in Zahlen ausgedrückt werden.



Diese Zahlen sollten sich auf den unteilbaren Durchschnitt beziehen.



Die biogeochemischen Zahlen, die die Art bestimmen, sind zweifach. Einige von ihnen sind die gleichen, die ein Biologe hätte geben können und sollen. Sie charakterisieren das morphologisch unterscheidbare Individuum der Art und manifestieren sich scharf auf einem separaten Unteilbaren.



Wenn ein Biologe systematisch versucht hätte, die von ihm untersuchten Phänomene zu quantifizieren, hätte die Biologie im Wesentlichen genügend quantitative Daten für biogeochemische Schlussfolgerungen sammeln müssen.



Dies war nicht wirklich der Fall. Tatsächlich sehen wir in der Geschichte des biologischen Wissens, dass sogar genaue Bestrebungen nach den quantitativen Merkmalen einer Art erstarrten, die die Aufmerksamkeit eines Biologen auf sich zogen. So schwächte sich die numerische Definition des Durchschnittsgewichts von Unteilbaren, die für Naturforscher der zweiten Hälfte des 18. Jahrhunderts, insbesondere für Wirbeltiere, eher üblich ist, im nächsten Jahrhundert ab. Das gleiche Bedürfnis, vielleicht in geringerem Maße, muss für die Anzahl der in jeder neuen Generation geschaffenen Unteilbaren angegeben werden -- Mengen, die auf einem Unteilbaren oder einem Paar Unteilbarer berechnet werden -- Samen, Eier, lebende Junge.



Jetzt gibt es hier in der Biologie keine ausreichende Anzahl von Daten, und die Methodik für ihre Herstellung wurde nicht entwickelt, und verstreute Zahlen werden nicht gesammelt und im Ozean verstreut, eine ständig wachsende qualitative Identifizierung.



Es ist unmöglich zu glauben, dass eine solche Abweichung von der Zahl und dem geometrischen Bild, die im Wesentlichen damit verbunden sind, die Arbeit des Biologen weniger genau und tiefgreifend machen würde. Noch wahrscheinlicher ist, dass es tiefer gehen kann als die Arbeit eines Biogeochemikers. Eine genaue Beschreibung des Naturforschers umfasst Bereiche von Phänomenen, die mit wesentlich abstrakteren Ausdrucksformen der Realität noch nicht erreicht werden können. Der Biologe nimmt in seiner genauen Beschreibung das Individuum als Quelle, unabhängig von der Form, in der er seine Manifestation in anderen Individuen ausdrückt. Er geht auf andere Individuen über und gibt unweigerlich die Grenzen an, innerhalb derer sich ein bestimmtes morphologisches Merkmal ändert.



Der Biogeochemiker befasst sich mit Aggregaten und mit durchschnittlichen statistischen Ausdrücken von Phänomenen. Gleichzeitig lenkt er sein Hauptaugenmerk auf den mathematischen Ausdruck von Phänomenen: Ausdruck durch Zahlen oder geometrische Bilder. Unweigerlich wird das Phänomen geglättet und eine Reihe von vom Biologen beobachteten Manifestationen werden vom Biogeochemiker nicht abgedeckt.



Der Biologe in seinem Wunsch, die Phänomene des Lebens auszudrücken, ging vom Lebenden unteilbar aus, ging, spezifizierte qualitativ das Heterogene, ging landeinwärts und erreichte die Grenze des Auges des Sichtbaren. Die Grenze ist die Wellenlänge der Strahlungsschwingungen -- ultraviolett -- des unsichtbaren Teils des Spektrums.



Wenn der Biologe auf eine separate Unteilbarkeit achtet, die von ihm untersuchte Korrektheit feststellt und von wiederholten Beobachtungen ausgeht und den Durchschnitt biometrisch erreicht, kann er im Wesentlichen tiefer eindringen und die Seiten von Lebensphänomenen abdecken, die außerhalb des biogeochemischen Ansatzes zur Untersuchung von Lebensphänomenen bleiben. Mit diesem Ansatz werden viele wichtige Manifestationen des Unteilbaren geglättet, wenn man sich auf die „mittelgroße“ unteilbare (§ 129) Biogeochemie stützt.



Die Biogeochemie kann sich diesen fehlenden Phänomenen jedoch in einem anderen Aspekt nähern, die Möglichkeit erhalten, sie einzufangen und sie im Laufe der geologischen Zeit zu untersuchen. So erscheinen sie beispielsweise im Übergangsprozess der Biosphäre zur Noosphäre und in den vormenschlichen Stadien, die der modernen Biosphäre vorausgingen.



132. Es kann keine Widersprüche zwischen der biologischen und biogeochemischen Beschreibung lebender natürlicher Körper geben -- wenn sie richtig gemacht werden.



Wie aus dem vorhergehenden hervorgeht, ergänzt die Biogeochemie die Arbeit eines Biologen und führt in die Untersuchung von Lebensphänomenen solche Manifestationen ein, von denen Biologen wenig oder gar keine Bedenken haben. Ihre Daten sind viel abstrakter als konkrete und facettenreiche Beschreibungen des Biologen.



Dies ist eine allgemeine Folge jedes Auftretens in der Beschreibung der lebenden Natur, ihrer mathematischen Abdeckung. Denn bei einer solchen Berichterstattung werden zwangsläufig nur einige grundlegende Merkmale der Phänomene berücksichtigt, aber die Mehrzahl der Merkmale, die als qualitativ als komplizierend bezeichnet werden, sekundäre Einzelheiten werden verworfen.



Die Biogeochemie stammt aus Atomen und untersucht den Einfluss von Atomen, die einen lebenden Organismus aufbauen, auf die Biochemie der Biosphäre und ihre Atomstruktur. Von den vielen Zeichen eines lebenden Organismus wählt sie einige aus, aber diese werden nur die bedeutendsten in ihrer Reflexion in der Biosphäre sein.



Durch die genaue Bestimmung aller Phänomene eines lebenden Organismus und seiner selbst -- chemisch, geometrisch und physikalisch -- wird der Organismus auf das Maß und die genau bestimmte Anzahl reduziert, wodurch er auf numerische Konstanten reduziert werden kann. Die Anzahl dieser Konstanten für jede Art ist unbedeutend.



Die Biogeochemie definiert lebende Materie -- insbesondere Arten -- mit den folgenden numerischen Konstanten:



1) Die durchschnittliche Anzahl von Atomen, im Durchschnitt eine unteilbare Spezies, für alle chemischen Elemente, die in einer bestimmten lebenden Substanz enthalten sind. Diese Zahlen werden durch genaue chemische quantitative Analyse erhalten. Sie können sie als Prozentsatz der Anzahl der Atome und als Prozentsatz ihres Gewichts ausdrücken. Quantitativ sollten sich Atome (oder ihr Gewicht) auf den durchschnittlichen Organismus beziehen.



2) Das Durchschnittsgewicht des durchschnittlichen Unteilbaren -- wird durch Wiegen einer ausreichenden Anzahl von Unteilbaren erhalten.



3) Die durchschnittliche Bevölkerungsrate der Biosphäre durch diesen Organismus aufgrund seiner Reproduktion. Diese Bevölkerungskonstante des Planeten kann entweder in der Anzahl der Unteilbaren oder im Gewicht der pro Zeiteinheit geschaffenen neugeborenen Nachkommen ausgedrückt werden. Dies ist die wichtigste Konstante, die der biogeochemischen Energie entspricht. Seine Bedeutung beruht auf der Tatsache, dass es die Migration von Elementen jeglicher Art von Organismen unter den natürlichen Bedingungen seines Lebens numerisch in Beziehung setzt, wobei die Geschwindigkeit der Entstehung neuer Generationen dieser Arten und die begrenzende Oberflächenebene, auf der eine solche Entstehung stattfinden kann, berücksichtigt werden -- mit dem Planeten, mit der Biosphäre.



Auf diese Weise wird eine Zahl eingeführt, die die taxonomische Einheit charakterisiert, einen Wert, der mit den Eigenschaften des Planeten und mit den Eigenschaften des gegebenen Organismus verbunden ist.



Diese drei Arten von Größen, die durch Beobachtung erhalten werden, können leicht als numerische charakteristische Konstanten ausgedrückt werden.



Für die ersten beiden ist dies völlig klar, und es ist leicht, sich auf die Form dieser Konstanten und ihre numerischen Ausdrücke zu einigen.



Es sollte bedacht werden, dass ein Biogeochemiker die Gesamtheit der Organismen in der äußeren Umgebung untersucht. Die Umgebung für ihn ist die Biosphäre, deren Größe genau definiert ist und die in der geologischen Zeit nahezu unverändert oder unverändert ist. Wenn sie sich in der geologischen Zeit ändern, können sie für lebende Organismen, deren Leben innerhalb historischer Zeitgrenzen liegt, ohne erkennbaren Fehler in Beobachtungen aufgenommen werden -- sie verschwinden in der durchschnittlichen Anzahl von Aggregaten (lebenden Substanzen) -- unverändert. Tatsächlich ist die Biosphäre ein einziges Ganzes, ein großer biokasischer natürlicher Körper, in dessen Umgebung alle biogeochemischen Phänomene auftreten. Die durchschnittliche Anzahl von Atomen und das Gewicht einer lebenden homogenen Substanz hängen vollständig von der Struktur der Biosphäre ab, aber für diese Konstanten können die Größen der Biosphäre gemäß der Methode zu ihrer Bestimmung möglicherweise nicht berücksichtigt werden.



Andernfalls wird die Zahl für die durchschnittliche Bevölkerungsrate der Biosphäre durch diese homogene lebende Substanz genommen. Die Dimensionen der Biosphäre müssen in sie eingeführt werden.



133. Diese drei Arten von Konstanten decken jedoch nicht alle biologischen Probleme ab, mit denen ein Biogeochemiker rechnen muss und die er versucht, vollständig durch Zahlen auszudrücken.



Es gibt noch ein Hauptphänomen, das von wissenschaftlicher Arbeit und wissenschaftlichem Denken wenig abgedeckt wird und für das es derzeit keinen einfachen und bequemen numerischen Ausdruck gibt. Der numerische Ausdruck ist jedoch möglich und die Biogeochemie kann nicht darauf verzichten.



Der gewundene, komplizierte Verlauf der Geschichte der wissenschaftlichen Erkenntnisse, der Biogeochemiker, näherte sich hier einem neuen wissenschaftlich unbehandelten Feld von Phänomenen, weit über die Grenzen eines genau definierten Feldes der Biogeochemie hinaus.



Wie so oft sollte er in diesem Fall versuchen, einen numerischen Ausdruck für diese neuen Phänomene zu erstellen, mit denen er sich so spezifisch befasst hat -- in präzisen Beobachtungs- und experimentellen Arbeiten. Er kann nicht weiter gehen, ohne zuerst seinen Weg freizumachen.



Dies ist ein Phänomen des Rechts-Links-Verhaltens, das außerhalb der Verarbeitung des wissenschaftlichen und philosophischen Denkens blieb. Auch geometrisch ist dieses Phänomen kaum betroffen. Und doch ist dies zweifellos eine der wichtigen geometrischen Eigenschaften des realen Raums, die im Raum beobachtet werden und auf deren Eigenschaften die Geometrie aufgebaut ist. Rechts- und Linkshändigkeit werden in der Geometrie jedoch nicht immer beobachtet. Sie sind nur einigen Formen der Geometrie eigen und erscheinen beispielsweise nicht in der Geometrie gleichmäßiger Dimensionen. Eine genaue Untersuchung der Geometrie von Rechts- und Linkshändigkeit ist für die Vertiefung der biogeochemischen Arbeit von großer Bedeutung.



Pasteur 172 der erste, basierend auf Erfahrung und Beobachtung, fing in den 1860-1880er Jahren seine Hauptbedeutung in biochemischen Prozessen und seine Wurzeln außerhalb des Kreislaufs des Lebens im kosmischen Aspekt auf. 173 Er brachte eine der Manifestationen von links-rechts-rechts vor, die sogenannte Asymmetrie. 174



Leider verwirrte dieser Name, der aufgrund der kristallographischen Darstellungen der ersten Hälfte des 19. Jahrhunderts sehr unglücklich war, das wissenschaftliche Denken, da er nicht das gesamte Phänomen als Ganzes abdeckte, wie Pasteur es richtig verstand und wie es sich nicht aus der Asymmetrie in seiner kristallographischen Definition ergab .



Tatsächlich beschäftigen wir uns hier mit den besonderen geometrischen und physikalischen Eigenschaften des Raums, den lebende Organismen und ihre Aggregate einnehmen, und in der Biosphäre nur für sie charakteristisch. 175



Ich werde den von P. Curie eingeführten Begriff -- den Zustand des Raums -- weiterhin verwenden, um ihn zu klären, jedoch um ihn zu klären. Wir können jetzt sagen, dass Pasteur für lebende Organismen die Existenz eines besonderen Zustands entdeckt hat, der sich von dem üblichen physikalisch-geometrisch charakterisierten Raumzustand unterscheidet -- einem Zustand des Linken und der Richtigkeit. Dieser Raumzustand existiert in der Biosphäre nur für die Phänomene des Lebens, dh im Leben und in natürlichen Biokos-Körpern.



In diesem Sinne ist es zweckmäßig, da es sich um reale Phänomene handelt, das Konzept des Lebens nach Möglichkeit zu vermeiden und es in der Biogeochemie durch einen besonderen Raumzustand zu ersetzen -- den Zustand der Rechts-Links-Natur lebender natürlicher Körper -- lebender Substanzen -- und den Teil biosynthetischer natürlicher Körper, der besteht aus.



134. Dies ermöglicht es uns, das riesige historische Erbe wissenschaftlicher Definitionen und Recherchen im Zusammenhang mit philosophischen und religiösen Konstruktionen loszuwerden. Sie dringen tief in das wissenschaftliche biologische Denken ein, mehr als in jedem anderen Bereich der Naturwissenschaften. Dies ist verständlich, da es sich um ein Feld von Phänomenen handelt, in denen neben Wissenschaft, Philosophie und Religion in jüngster Zeit eine beherrschende Stellung eingenommen wurde und das nun zu jedem Thema behandelt wird. Dies gab der wissenschaftlichen Arbeit eine gewisse soziale Stärke und ein gewisses Interesse, schwächte und verzerrte jedoch die wissenschaftliche Suche weiter. Je geringer der Einfluss von Philosophie und Religion ist, desto freier und produktiver kann sich das wissenschaftliche Denken in diesem Bereich wissenschaftlicher Erkenntnisse bewegen.



Der Hauptgrund für diesen Einfluss, insbesondere die Philosophie, ist die Suche und Erklärung der Eigenschaften des „Lebens“. Das Leben als Ganzes wird nicht als eine Kombination von lebenden Organismen, lebenden natürlichen Körpern betrachtet, sondern als eine besondere Manifestation von etwas, das in der Natur vor allem in lebenden Organismen klar offenbart ist, sondern nicht nur in ihnen stattfinden kann.



Es scheint mir, dass die Annahme des Lebens als besondere Eigenschaft, die sich außerhalb eines bestimmten Zusammenhangs mit den Funktionen eines lebenden Organismus manifestieren kann, in der Biologie einen weiten Spielraum für das Eindringen philosophischer, ganz zu schweigen von religiösen, mystischen Ideen eröffnet. Die gesamte Biologie wird immer noch von außen von Annahmen durchdrungen, die in sie eingedrungen sind -- ob die Seele, das spirituelle Prinzip, die Lebensenergie, die Entelechie und die Lebenskraft gleichgültig sind, macht keinen Unterschied. Indem der Biologe diese besonderen lebenswichtigen Eigenschaften anstelle der spezifischen Daten von Erfahrung und Beobachtung anstelle von lebenden natürlichen Körpern -- Lebewesen oder Lebewesen (dh Sammlungen von Lebewesen) -- einsetzt, führt er unmerklich ein weites Feld von Ideen in die Wissenschaft ein, die außerhalb des exakten Wissens in einem weiten Bereich der humanitären Hilfe entstanden sind Wissenschaften und Philosophie.



Natürlich geht ein exakter Naturforscher in Wirklichkeit niemals über die Grenzen eines lebenden Organismus hinaus und untersucht das Leben nur insoweit, als es sich in der Struktur und den Eigenschaften lebender Organismen manifestiert. Mit einer solchen Erweiterung des Lebensbegriffs sind aber auch andere Vorstellungen vom Ort seiner Manifestation akzeptabel, mit denen man rechnen muss. Solche Darstellungen fanden in naturphilosophischen Suchen und in wissenschaftlichen Studien über spirituelle, psychologische und parapsychische Phänomene statt. Da sie an einem separaten Lebewesen untersucht werden können, kann ihre Abwesenheit von vornherein nicht als erwiesen angesehen werden, und ein Wissenschaftler, der unter diesen Bedingungen arbeitet und sich dessen klar bewusst ist, muss prüfen, ob dieses Phänomen vorliegt. Die Frage kann nicht durch logisches Denken und nicht durch historische Forschung gelöst werden, sondern nur durch genau festgelegte wissenschaftliche Erfahrung und Beobachtung. Bisher hat die Erfahrung unter dem Gesichtspunkt spiritueller Erklärungen zu einem negativen Ergebnis geführt, aber es öffnen sich Phänomene, die auf die Existenz der Eigenschaften lebender Organismen hinweisen, die nicht mit genauem Wissen registriert sind.



Dies ermöglicht es, diese Errungenschaften fälschlicherweise als Hinweis auf die Existenz besonderer Eigenschaften des Lebens zu übertragen. In Wirklichkeit weist dies nur auf die Existenz neuer Eigenschaften eines lebenden natürlichen Körpers hin. Das Gebiet der wissenschaftlichen Erkenntnisse ist ein Bereich in seiner Struktur, der äußerst komplex ist und in dem es nicht immer leicht zu trennen ist, was auf genauen Tatsachen und logischen Schlussfolgerungen daraus beruht und was eine Hypothese, Intuition ist oder historisch von einer der Wissenschaft fremden Philosophie oder in sie hinein gewachsen ist Religionen, in denen die Wurzeln dieser Ideen liegen.



Die Lebensvorstellungen, die nicht mit dem lebenden Organismus oder seinen Aggregaten verbunden oder indirekt mit ihnen verbunden sind, haben umso mehr das Existenzrecht, dass das Spektrum der Lebensmanifestationen von Lebewesen extrem groß ist und dass unser gesamtes Wissen untrennbar mit der tiefsten und mächtigsten nervösen Organisation des Vertreters verbunden ist das Leben des Homo sapiens. In diesem Fall muss man zwischen der Manifestation eines lebenden Organismus in zwei Aspekten unterscheiden -- bei der Manifestation von Aggregaten lebender Organismen, wie dies in der Biogeochemie der Fall ist, und zweitens bei der Manifestation einzelner Individuen -- für eine Person, eine Person, die in einigen Fällen stark vom Durchschnittsniveau abweicht. Ausgehend von den dem Menschen innewohnenden Manifestationen und dem Erkennen oder Akzeptieren der grundlegenden Identität der Manifestationen des Lebens für alle lebenden Organismen wurde in der Wissenschaft weitgehend ein riesiges Feld der Geisteswissenschaften geschaffen, in dem solche Manifestationen lebender Organismen an erster Stelle stehen, dass die überwiegende Mehrheit nicht existiert und oft nur dem Menschen eigen.



Die von der Biogeochemie untersuchten Phänomene befassen sich nur mit Aggregaten von Organismen, und wenn sie untersucht werden, besteht keine Notwendigkeit, über den Rahmen der mit Aggregaten verbundenen Phänomene hinauszugehen. Hier können wir uns ganz sicher als allgemeine Eigenschaft des Lebens identifizieren und damit die Gesamtheit der lebenden Organismen, den besonderen Raumzustand, den sie einnehmen, verstehen.



Und jetzt stehen wir in der Biogeochemie vor der Notwendigkeit, mit solchen Manifestationen lebender Substanzen in der Biosphäre umzugehen, bei denen eine individuelle Persönlichkeit der menschlichen Bevölkerung einen großen Einfluss auf die Prozesse in der Biosphäre haben kann. Genau dies geschieht im gegenwärtigen historischen Moment, wenn wir den Übergang der Biosphäre zur Noosphäre untersuchen. Wir untersuchen hier den Einfluss des wissenschaftlichen Denkens auf den geologischen Prozess, und in diesem Fall können sich das Denken und der Wille des Individuums oft dramatisch ändern und im natürlichen Prozess manifestieren.



135. Die Idee der lebenden Materie in der Biogeochemie, dh in der Gesamtheit der lebenden natürlichen Körper, sollte auf die gleiche Weise ausgedrückt werden wie vor langer Zeit für inerte natürliche Körper, und sollte vollständig auf genauen Zahlen beruhen. Für inerte Körper (zum Beispiel für astronomische Beobachtungen) begann es vor Tausenden von Jahren, aber für chemische und physikalische Eigenschaften, für die Beschreibung von Mineralien, geografischen Phänomenen usw. wurde dies erst in den letzten drei Jahrhunderten getan. Seit der zweiten Hälfte des 19. Jahrhunderts. Eine solche Abdeckung der inerten natürlichen Körper der Biosphäre ist obligatorisch geworden -- Tiere und Pflanzen wurden teilweise gefangen -- und die Anzahl der erhaltenen Zahlen wächst unkontrolliert und beträgt Millionen.



In der Biogeochemie sind dies die Zahlen des Gewichts der lebenden Materie, die Anzahl der Atom- und Gewichtszusammensetzungen, die Anzahl der Reproduktion, die biogeochemische Energie (Population des Planeten), die rechts und links quantitativ ausgedrückt werden.



Wenn der auf diese Weise erhaltene Begriff der lebenden Materie mit den numerisch ausgedrückten inerten (oder Biokas-) Naturkörpern der Biosphäre verglichen wurde, wurde sofort erstens die Möglichkeit eines solchen Vergleichs klar, der logischerweise vorher keine Zweifel aufkommen ließ, und zweitens die Existenz einer scharfen materiellen Energie Unterschiede zwischen lebenden und inerten natürlichen Körpern. Es gibt keine Prozesse in der Biosphäre, bei denen dieser Unterschied verschwindet. Bei einem kontinuierlichen biogenen Austausch von Atomen und Energie zwischen lebenden und inerten natürlichen Körpern der Biosphäre besteht eine ganze Kluft in ihrer Struktur und ihren Eigenschaften.



Dieser Unterschied ist eine wissenschaftliche Tatsache oder vielmehr eine wissenschaftliche Verallgemeinerung. Die Folge davon ist die Ablehnung der Möglichkeit der Existenz einer spontanen Keimbildung lebender Organismen aus inerten natürlichen Körpern unter modernen und bestehenden Bedingungen während der geologischen Zeit, dh über 2 Milliarden Jahre.



Dies wird -- unter dem Einfluss philosophischer, aber nicht wissenschaftlicher Überlegungen -- von vielen Wissenschaftlern immer noch nicht anerkannt und ist in der philosophischen und populärwissenschaftlichen Literatur weit verbreitet. Seit Hunderten von Jahren -- und jetzt -- gibt es Versuche, Experimente zur Abiogenese durchzuführen.



In der Biogeochemie ist das Fehlen eines Übergangs eine empirische wissenschaftliche Verallgemeinerung und keine Hypothese oder ein theoretisches Konstrukt.



Diese empirische Verallgemeinerung lautet wie folgt:



Es gibt keine Übergänge zwischen lebenden und inerten Naturkörpern der Biosphäre -- die Grenze zwischen ihnen in der gesamten geologischen Geschichte ist scharf und klar. Materiell-energetisch unterscheidet sich ein lebender natürlicher Körper in seiner Geometrie, ein lebender natürlicher Körper, von einem natürlichen inerten Körper. Die Biosphärensubstanz besteht aus zwei Zuständen, die sich energetisch energetisch unterscheiden -- lebend und inert.



Lebende Materie ist zwar in der Biosphäre materiell unbedeutend, aber energetisch gesehen erscheint sie in erster Linie darin.



Dies bestimmt eine neue äußerst wichtige Eigenschaft der Biosphäre -- ihre geometrische Heterogenität. Wie wir sehen werden (§ 138), kann davon ausgegangen werden, dass lebende Materie eine andere Geometrie aufweist als Euklid.



136. Bevor wir fortfahren, müssen wir versuchen, die grundlegenden Daten über unser Verständnis des Lebens zu analysieren und einige neue Konzepte einzuführen.



Ich habe bereits die Existenz von natürlichen Biokos-Körpern angesprochen (§ 123). Hier ist es notwendig, in wenigen Worten auf sie einzugehen. Ich habe nur darauf hingewiesen, dass wir die Biosphäre selbst als einen Biokos-Körper betrachten können.



Im Wesentlichen ist jeder Organismus ein Biokosakörper. Es sind nicht alle Lebewesen. Während seiner Ernährung und Atmung dringen ständig träge Körper in ihn ein, die völlig untrennbar mit ihm verbunden sind. Zum Teil fallen sie mechanisch als Fremdkörper hinein, als Körper, die für ihn im Wesentlichen unnötig sind oder deren Werte wir nicht verstehen. Bei der Berechnung des Gewichts und der chemischen Zusammensetzung eines lebenden Organismus in der Biosphäre kann man diese Fremdsubstanz, die immer Teil des Körpers ist, nur berücksichtigen. Ohne sie gibt es keinen lebenden Organismus in der Biosphäre. Diese Substanz sollte (in durchschnittlicher Anzahl) in den Aggregaten von Organismen berücksichtigt werden, da sie die besondere biogene Migration von Atomen widerspiegelt -- das Hauptphänomen, das von der Biogeochemie untersucht wird. Ich werde nicht weiter darauf eingehen und es beweisen, aber ich werde ein oder zwei Beispiele geben. Regenwürmer oder Holothurianer enthalten ständig Erde oder Schlamm in ihrem Körper, dessen Prozentsatz ein wahrnehmbarer Teil ihres Körpers ist und der sofort zahlreiche biochemische Reaktionen in ihrem Körper eingeht. Diese Organismen in der Biosphäre schienen ohne eine solche Substanz von Drittanbietern keine Sekunde zu existieren, das heißt, sie konnten nicht leben. In der Biogeochemie müssen wir sie so berücksichtigen, wie sie sind und leben, und nicht gereinigt und von diesen Substanzen befreit werden, die immer in ihnen vorhanden sind.



Dies sind schärfere Beispiele, aber für jeden lebenden Organismus haben wir Teile seines Körpers, die in einem lebenden Prozess, in lebenserhaltenden Atommigrationen (ewig veränderte Lebensbalance, in den Phänomenen Stoffwechsel, Atmung und Ernährung) streng genommen nicht berücksichtigt werden können individuell lebendig. Ein lebender Organismus ist immer bis zu einem gewissen Grad ein biokosnöser natürlicher Körper, aber in ihm herrscht zum Zeitpunkt des Lebens die Substanz des Lebens vor, die scharf in Masse, aber nicht immer in Volumen enthalten ist. Insgesamt zeigt ein solcher Biokosalkörper scharf seine lebenden Eigenschaften, auch wenn sie volumenmäßig nicht vorherrschend sind. Beispielsweise sind in einer Reihe von Organismen große Teile des Raums, den sie einnehmen, Gashohlräume und Blasen. Diese Gashohlräume leben natürlich nicht, aber wir werden unten sehen, dass sie sich geometrisch von inerten natürlichen Körpern unterscheiden.



Ein lebender Organismus als Ganzes, obwohl er daher in seiner Zusammensetzung bis zu einem gewissen Grad ein biokosnöser natürlicher Körper ist, unterscheidet sich jedoch stark von echten Biokosalkörpern, hauptsächlich in den Eigenschaften des von ihm eingenommenen Raums. Geometrisch und physikalisch unterscheidet sich dieser Raum vom Raum inerter natürlicher Körper der Biosphäre. Darüber hinaus stellt es ein autarkes System in der Biosphäre dar, das einheitlich, autark, in der Lage ist, sich selbst zu verteidigen und aktiv auf die äußere und innere Umgebung und andere lebende Organismen zu reagieren. Der tierische Organismus manifestiert sich in der Biosphäre als ein kleines, ihm fremdes Ganzes, als seine eigene separate Welt, Monade, die auf natürliche Weise mit der äußeren Umgebung verbunden ist. Der Biokosalkörper ist ein komplexeres System lebender Organismen -- Monaden und inerte natürliche Körper -, die in Wechselwirkung stehen, sich aber nicht miteinander vermischen. Die überwiegende Mehrheit der natürlichen Gewässer, Böden, Schlämme usw. sind unzählige Beispiele für biokasische Naturkörper.



137. Es scheint mir, dass es lange an der Zeit ist, diesen scharfen Energie-Material-Unterschied zwischen lebender und inerter Materie der Biosphäre, der durch die Biogeochemie festgestellt wurde, für die erste wissenschaftliche Arbeit zu nutzen und die wissenschaftlichen Schlussfolgerungen, die sich aus einem solchen Vergleich ergeben, wissenschaftlich zu berücksichtigen.



Ich werde im Folgenden kurz auf diese Unterschiede eingehen, die, wie wir sehen werden, weit von denen entfernt sind, die Biologen und Philosophen des Westens in ihrer vitalistisch-materialistischen, dauerhaften Jahrhundertkontroverse verwenden.



Sie sind nicht sichtbar und nicht klar, wenn sie einen einzelnen Organismus untersuchen, sondern erscheinen als reales Phänomen, als Tatsache, wenn sie zusammengenommen werden. Sie sind für einen Naturforscher, der ein Individuum studiert, kaum wahrnehmbar, aber sie zeigen sich deutlich in der lebenden Materie der Biosphäre.



Und sie sind solche, es scheint mir, dass sie mit dem Konzept des Lebens als privates planetarisches Phänomen unvereinbar sind.



138. Die wichtigsten dieser Unterschiede sind folgende:



I. Das Leben auf der Erde -- nur in der Biosphäre -- manifestiert sich zum einen in Form lebender Organismen -- lebender natürlicher Körper mit ihrem eigenen autarken Volumen, dem Lebensfeld -- sowohl im Medium der universellen Gravitation als auch im mikroskopischen Teil der Welt, 176 wo Gravitationskräfte auftreten nicht dominieren, sind von untergeordneter Bedeutung.



Wie Sie wissen, sind die 177 Größen natürlicher Körper keineswegs gleichgültig, im Gegenteil, sie sind vielleicht das charakteristischste Zeichen im Realitätssystem. Für lebende Organismen ist die Bandbreite dieser Phänomene sehr groß. Ab einer Ordnung mit großen Molekülen chemischer Verbindungen, einer Ordnung in Parametern von 10 -- 6 cm, reicht es für große Individuen von Pflanzen und Tieren bis zu Parametern 10 4 cm. Der Bereich beträgt 10 10 .



Der Raumzustand (Volumen), der dem Körper eines lebenden Organismus entspricht, egal wie klein oder groß, ist asymmetrisch. Dies manifestiert sich in Gerechtigkeit und Linken 178 -- in der Ungleichheit der Phänomene des Salzens und Salzens. In der Biosphäre ist diese Eigenschaft des Raums nur lebenden Organismen eigen. Organogene Mineralien (Öl, Kohle, Humus usw.) behalten geologisch lange Verbindungen bei, die biochemisch erhalten wurden, wobei der Unterschied zwischen rechts und links deutlich zum Ausdruck kommt, diese Eigenschaft jedoch während der geochemischen Zerstörung nicht wiederhergestellt wird. Ein solcher Raumzustand in einem lebenden Organismus wird zweckmäßigerweise Pasteurs Asymmetrie genannt. 179



II. Die Haupteigenschaft der Asymmetrie, d. H. Der spezielle Zustand der Raumzeit, der dem Leben und dem von ihm eingenommenen Volumen entspricht, besteht darin, dass Ursache und Wirkung der darin beobachteten Phänomene der gleichen Asymmetrie entsprechen müssen. 180 In den kristallinen Körpern, die von den für ihr Leben notwendigen Organismen gebildet werden, drückt sich die Asymmetrie in der Dominanz der linken oder rechten Isomere aus. Es ist möglich , dass die Rechte von Pasteur, die glaubten , dass die Hauptkörper notwendig für das Leben -- für Proteine und deren Abbauprodukte, 181 -- immer von den linken Isomeren dominiert. Dieser Bereich der Phänomene ist leider wenig erforscht und man kann hier in naher Zukunft unerwartete Entdeckungen von Bedeutung erwarten. P. Curie hat die Möglichkeit verschiedener Formen der Asymmetrie ganz richtig berücksichtigt und eine geometrische Struktur ausgedrückt, wobei der Zusammenhang in der Position offenbart wird, dass das unsymmetrische Phänomen durch dieselbe unsymmetrische Ursache verursacht wird. Basierend auf diesem Prinzip (wir können es das Curie-Prinzip nennen) folgt daraus, dass der spezielle Zustand des Lebensraums eine spezielle Geometrie hat, die nicht die übliche euklidische Geometrie ist. 182



Ich werde dies als Arbeitshypothese akzeptieren, bis es theoretisch verifiziert ist. Dieser Bereich der Phänomene in den Hauptmerkmalen wurde in den Werken von Pasteur 183 in den Jahren 1860 bis 1880 geklärt . P. Curie beschäftigte sich in den 1890er Jahren mit diesen Phänomenen, doch der plötzliche Tod unterbrach sein Leben 1906, bevor er seine Erfolge darlegen konnte . 184



Das Konzept des „Zustands des Raumes“ (espace dtat) ​​wurde in seiner 1925 von 185 seiner Frau und seiner Tochter veröffentlichten Biographie in die Wissenschaft eingeführt . So definierte er Pasteurs Asymmetrie in seiner Familie im Zeitalter seiner kreativen Arbeit an diesem Problem, das nicht veröffentlicht und geschrieben werden sollte.



III. Die wirkliche, logisch korrekte Schlussfolgerung aus dem Pasteur-Curie-Prinzip ist das Redi-Prinzip 186 , das die Bildung von Organismen in der Biosphäre regelt. Omne vivum e vivo ist eine Manifestation der Pasteur-Asymmetrie, da sonst in der Biosphäre kein Rechts-Links-Effekt erzeugt werden kann, der der Pasteur-Asymmetrie entspricht. Im Wesentlichen ist diese Aufrechterhaltung der Lebensspanne während der gesamten geologischen Zeit durch Teilung, Knospung oder Geburt die Hauptmanifestation der besonderen Raumzeit lebender natürlicher Körper, ihrer besonderen Geometrie.



IV. Eine reale, logisch korrekte Schlussfolgerung aus dem Pasteur-Curie-Prinzip wird sein, dass die dem Leben entsprechenden Phänomene zeitlich irreversibel sind, da der Raum eines lebenden Organismus mit Pasteurs Asymmetrie nur polare Vektoren besitzen kann, die der Zeitvektor dafür sein werden. 187



V. In der Biosphäre manifestiert sich das Redi-Prinzip in der Ausbreitung von Organismen aufgrund der Reproduktion, ein Phänomen, das für seine Struktur von größter Bedeutung ist. Die Neuansiedlung bewirkt eine biogene Migration von Atomen in der Biosphäre und geht mit einer enormen Freisetzung von freier Energie, biogeochemischer Energie, einher. 188 Diese biogeochemische Energie manifestiert sich im Aspekt der historischen Zeit.



Biogene Migrationen der Biosphäre unterscheiden sich stark von Migrationen chemischer Elemente, die nicht mit lebender Materie zusammenhängen. Dieses letzte Phänomen wird sichtbar -- manifestiert sich in der Masse der terrestrischen Materie -- nur unter dem Aspekt der geologischen Zeit.



VI. Extrem charakteristisch ist das maximale Maximum -- aufgrund der Größe der Atomkomplexe -- die Loshmidt-Zahl an erster Stelle und die Endgeschwindigkeit wellenartiger Bewegungen -- „Schall“ (einschließlich Atmung in einer Gas- oder Wasseratmosphäre) -- der Wert der biogeochemischen Reproduktionsenergie.



Eine Folge davon ist die außergewöhnliche Bedeutung mikroskopisch dispergierter lebender Materie und ihre große Rolle bei der Streuung chemischer Elemente in der Biosphäre. Dies liegt an den Gesetzen der Thermodynamik -- bei maximaler Nutzung der freien Energie.



VII. Die Nährstoffmigration von Elementen ist hauptsächlich mit der Atmung von Organismen verbunden und beruht auf der Größe und den Eigenschaften der inerten Substanz des Planeten. Aus diesem Grund hat es eine Grenze, die einerseits mit der Loshmidt-Zahl verbunden ist, die die Anzahl der Gasmoleküle pro 1 cm3 Volumen und damit die Anzahl der Unteilbaren bestimmt, die mit ihnen im Atemaustausch stehen, und andererseits mit der Reproduktion durch Dies spiegelt die Größe der Erdoberfläche wider, die Oberfläche der Biosphäre.



Viii. Die Fläche, die der Population von Organismen zur Verfügung steht, ist begrenzt -- woher die Existenz einer begrenzten Menge (Masse des Lebens) lebender Materie, die auf unserem Planeten existieren kann. Dieser Wert ist -- in bestimmten kleinen Schwankungsbereichen -- während der geologischen Zeit konstant.



IX. Die Fortpflanzung ist im mikroskopischen Teil der Welt am schnellsten, wodurch (die Loshmidt-Zahl -- Abschnitt VI) die Größe des Organismus begrenzt ist, da die Fortpflanzung umgekehrt proportional zum Volumen des Organismus ist (Regel von E. Snejadetsky). Unterhalb einer bekannten Größe können Organismen existieren, die sich zeitweise vermehren (die Umwelt ihres Lebens zerstören -- einen lebenden Organismus) und sich schnell in einen latenten Zustand verwandeln.



X. Lebende Organismen, die über einen Stoffwechsel verfügen, bilden selbst ihre eigene chemische Elementarzusammensetzung, die für ihre Eigenschaften charakteristisch (und speziesspezifisch) ist und in bestimmten Grenzen unverändert bleibt. Wir haben hier eine Analogie mit bestimmten chemischen Verbindungen ohne stöchiometrische Beziehungen.



Xi. Aufgrund der großen Menge an biogeochemischer Energie haben wir hier Millionen natürlicher biogener Körper -- Arten von Organismen und sogar Millionen weiterer Millionen biochemischer chemischer Verbindungen, die in ihnen erzeugt werden, im Gegensatz zu inerter Materie mit ihren 2-3 Tausend Mineralien und den ihnen entsprechenden chemischen Verbindungen.



XII. Infolge des radioaktiven Zerfalls von Elementen und der biogeochemischen Energie sammelt die Biosphäre im Laufe der Zeit freie Energie an und mit der Bildung der Noosphäre wird dieser Prozess extrem verstärkt (Ektropie). 189



Xiii. Lebende Organismen haben die Fähigkeit, Isotopenmischungen chemischer Elemente, dh die Atomgewichte chemischer Elemente innerhalb des kleinsten Volumens eines lebenden Körpers, zu verändern. Ähnliche Prozesse scheinen in den inerten natürlichen Körpern der Biosphäre sehr unterschiedlich zu verlaufen. Diese Phänomene sind alle sehr wenig untersucht, aber es ist anzunehmen, dass sie nur außerhalb der Biosphäre in ihnen auftreten und mit Gasphänomenen verbunden sind, die in Hochdruckregionen auftreten. Eine genaue Bestimmung des Atomgewichts von Elementen in auf diese Weise gebildeten Mineralien ist erforderlich.



139. Zusammenfassend sehen wir, dass zwischen dem lebenden natürlichen Körper der Biosphäre und seinen Komplexen (lebende Materie) und Assoziationen (Biozönosen und Biokosalkörper) und seinen inerten natürlichen Körpern -- Mineralien, Kristalle, Gesteine ​​usw. -- in ihrer unzähligen Vielfalt -- Es gibt eine scharfe unpassierbare Linie.



Dies ist keine philosophische oder wissenschaftliche Hypothese oder Theorie -- es ist eine empirische Verallgemeinerung unzähliger genau logisch und empirisch festgestellter Tatsachen. Sie können nur aufgrund von Kritik an diesen Tatsachen oder Widerspruch gegen sie gegen andere empirische Verallgemeinerungen bestritten werden, die dem einen oder anderen der im vorhergehenden Absatz 138 genannten Absätze widersprechen.



Weder logisch noch philosophisch können sie widerlegt werden. Sie alle beziehen sich auf einen bestimmten natürlichen Körper -- einen lebenden Organismus.



Alle hier angegebenen Verallgemeinerungen gehen nicht über die im Leben von Organismen und ihren Aggregaten beobachteten Phänomene hinaus. Sie berühren sich nicht und geben keine Erklärung für das Leben; Sie bringen nur Fakten zusammen und ziehen logische Schlussfolgerungen aus einer wissenschaftlichen Beschreibung der Realität.



Sie entsprechen den logisch beherrschten Konzepten der Biogeochemie. Aber auf dem Gebiet des biologischen Denkens stehen sie in ihrem literarischen modernen Ausdruck oft im Konflikt mit lebendigen Ideen, die über die Phänomene des Lebens herrschen.



In der Kollision philosophischer Ideen mit diesen empirischen Verallgemeinerungen können wir sie beiseite lassen und eine logische Bewertung als philosophische Fiktionen zulassen. Denn philosophische Ideen basieren auf der Analyse allgemeiner Konzepte, die nicht immer die vollständig zugrunde liegenden wissenschaftlichen Fakten und wissenschaftlichen empirischen Verallgemeinerungen abdecken. In dieser Hinsicht sind alle Probleme, mit denen sich beispielsweise Vitalisten und Materialisten, Wissenschaftler oder Philosophen befassen, egal, sie fallen aus unserem Horizont und wir begegnen ihnen im Bereich unserer Studie nicht wirklich.



Das Leben in den von der Biogeochemie untersuchten Phänomenen wird fast ausschließlich von natürlichen lebenden Körpern erfasst, und nur im Problem der Noosphäre müssen mit Faktoren gerechnet werden, die streng genommen nicht von gewöhnlichen Vorstellungen über lebende natürliche Körper abgedeckt werden, aber in der Biogeochemie können wir sie nur in lebenden natürlichen Körpern untersuchen.



140. Die Biologie umfasst das Leben im weiteren Sinne und es wird logisch richtig sein zu fragen, ob sie sich in biologischen Prozessen manifestiert, die gegen Schlussfolgerungen aus lebenden natürlichen Körpern verstoßen können.



Die enge Verbindung der Biogeochemie mit der Biologie, die in Zukunft nur noch zunehmen sollte, wirft diese Frage in der Biogeochemie auf. Eine weitere Analyse der Noosphäre, die gerade erst begonnen hat, wirft diese Frage noch tiefer und anschaulicher auf.



In der Biologie ist von großer Bedeutung, man kann sagen, das Phänomen, das die Eigenschaften höherer Formen des menschlichen Lebens erfüllt. In einem breiten Verständnis natürlicher Phänomene umfasst dies sowohl soziale als auch spirituelle Manifestationen des Menschen, die untrennbar mit den biologischen Grundlagen des menschlichen Körpers verbunden sind. Hier müssen wir mit dem extremen Einfluss des riesigen kulturellen Erbes rechnen, das mit der Vergangenheit verbunden ist. Der Biologe ist untrennbar mit diesem philosophischen, religiösen und sozialen Erbe verbunden, das er nicht vollständig loswerden kann, egal wie er danach strebt.



In dieser Hinsicht ist die Situation des Biogeochemikers sehr unterschiedlich, was in seinen Problemen einerseits durch Prozesse begrenzt ist, die sich in natürlichen lebenden Körpern widerspiegeln, und andererseits durch Prozesse, die von den Eigenschaften chemischer Elemente, ihrer Gemische und Isotope, dh Atome, abhängen. Für den Biogeochemiker wird jedoch im Bild der vor ihm eröffneten Noosphäre erstmals die Manifestation der höchsten Eigenschaften eines lebenden Organismus im biogeochemischen Aspekt, die in Biologie und Philosophie eine so wichtige Rolle spielen, in sein Mandat aufgenommen.



Und für ihn stellt sich die Frage: Handelt es sich hier um neue Manifestationen der Phänomene des Lebens, die nicht durch die Kategorien von Phänomenen abgedeckt sind, die er untersucht und die durch die Konstanten der lebenden Materie ausgedrückt werden? Oder haben wir es hier im Wesentlichen mit denselben Phänomenen zu tun, die in allen von ihm untersuchten Lebewesen schwächer biogeochemisch exprimiert werden? In der Noosphäre manifestiert sich der wirkliche Einfluss des menschlichen Geistes auf die Geschichte des Planeten stark biogeochemisch.



Der menschliche Geist ist das Hauptthema des philosophischen Denkens und wird von der wissenschaftlichen Forschung viel weniger erfasst als alle anderen biologischen Manifestationen auf unserem Planeten. Aber der Biogeochemiker geht in dieser Studie nirgendwo in der Noosphäre über die Grenzen lebender und biokosnöser natürlicher Körper hinaus und kann daher alle philosophischen und wissenschaftlichen Hypothesen und Theorien ignorieren, die mit dem Verständnis der spirituellen Aspekte des menschlichen Denkens zusammenhängen. Aus dieser oder jener Entscheidung über diese Manifestationen des spirituellen Lebens eines Menschen werden seine Schlussfolgerungen überhaupt nicht verletzt.



Die Hauptfrage, die hier auftaucht, wird die Frage sein, ob der menschliche Geist unter diesem Wort alle spirituellen Manifestationen der Persönlichkeit eines Menschen darstellt, etwas Neues und sogar Besonderes nur für höhere Wirbeltiere oder sogar für Menschen, oder ob dies eine Eigenschaft aller lebenden natürlichen Körper ist . Die eine oder andere Antwort auf diese Frage kann in der Biogeochemie nicht von Bedeutung sein, da in der Noosphäre der entscheidende und bestimmende Faktor das geistige Leben des Menschen in seiner besonderen Identifizierung ist.



141. In einer völlig anderen Position befindet sich ein Biologe, der gezwungen ist, auf dem Gebiet der spirituellen Umgebung zu arbeiten, die durch Jahrhunderte philosophischen, religiösen und sozialen Denkens geschaffen wurde und bei jedem Schritt auf vorgefertigte, widersprüchliche Konzepte stößt, die oft durch poetische und künstlerische Intuition auf der Grundlage der tiefsten Manifestationen des Menschen geschaffen wurden Persönlichkeit.



Er ist jetzt nicht in der Lage, diese Probleme zu verstehen und zu lösen. Es scheint mir jedoch, dass er mit einer strengen und vorsichtigen Haltung gegenüber dem Druck seiner realen spirituellen Situation und mit einer strengeren Haltung gegenüber dem Konzept des Lebens den schädlichen Einfluss seiner spirituellen Umgebung minimieren kann.



Denn in Wirklichkeit studiert der Biologe wie der Biochemiker nicht „Leben“, sondern lebende Materie (im angegebenen Sinne) und stellt einen separaten lebenden natürlichen Körper vor -- einen lebenden Organismus. Wenn ein lebender Organismus (und seine Gesamtheit -- lebende Substanz) in der wissenschaftlichen Arbeit eines Biologen mit dem Konzept des Lebens identisch ist, ist es für die Befreiung von philosophischen und theologischen Konzepten, die der wissenschaftlichen Forschung fremd sind, bequemer, in der Biologie vom Konzept eines lebenden natürlichen Körpers -- einem lebenden Organismus -- und nicht vom Konzept auszugehen des Lebens.



Ob es neben einem lebenden Organismus Manifestationen des Lebens gibt oder nicht, mag für einen modernen Biologen nicht von Interesse sein, da alle seine Arbeiten auf dem Gebiet der Untersuchung eines lebenden und eines toten Organismus liegen. Dies nennt er in Wirklichkeit Leben. „Leben“ für den Philosophen und Theologen ist vielleicht nicht identisch mit einem lebenden Organismus und seinen Aggregaten.



Der Biologe und Biogeochemiker kann jedoch die Existenz eines anderen, besseren Verständnisses des Lebens nicht ignorieren als das, von dem sie ausgehen, das seit Jahrhunderten mit dem von ihm untersuchten Gebiet in Kontakt steht. Sie treffen sich bei jedem Schritt mit ihm und sollten vor der Reichweite seines Einflusses jederzeit in Alarmbereitschaft sein. Sie sollten sich dieser anderen Ideen bewusst sein und ihren möglichen und akzeptablen Wert in ihrer Arbeit bewerten.



142. Bevor ich darauf eingehe, halte ich es für nützlich, die Bestimmungen 130 in Form des Unterschieds zwischen lebenden und inerten natürlichen Körpern in ihren Erscheinungsformen in der Biosphäre zu reduzieren und in einer neuen Form darzustellen.



Hier ist die Zusammenfassung:



    Inerte natürliche Körper Lebende natürliche Körper      



 



I. Körper ähnlich lebenden lebenden natürlichen Körpern



natürliche zerstreute Körper, erscheinen nur in



im inerten Teil der Biosphäre dort. Biosphäre und nur in Form



Dispergierte inerte Substanz dispergierter Körper in Form



konzentriert sich in der Biosphäre; in lebenden Organismen und ihren



tiefere Teile des Planeten aggregieren -- in



es wird durch Druck übertönt. makroskopisch (Feld



Es entsteht entweder, wenn die Schwerkraft stirbt) oder in



lebende Materie oder der Einfluss mikroskopischer Schnitte



auf der Biosphäre der bewegten Gasrealität.



oder flüssige Phasen, immer



Bio-Skew-Körper sein.



 



II. In inerten natürlichen Körpern



Es gibt keine Manifestationen der Wahrheit und charakterisiert den Staat



Der Linke unterliegt nicht den Gesetzen des besetzten Raums



Symmetrien eines Festkörpers. Körper eines lebenden Organismus und



Aus diesem Grund, wenn seine Manifestationen in



Gerechtigkeit und Linker manifestieren sich im lebenden Organismus



in einem homogenen anisotropen Medium. In fest



Raum kristalliner lebender Organismen



Festkörper, Asymmetrie manifestiert sich.



geometrisch speziell, aber die gleiche Asymmetrie



ausgedrückt in manifestiert in dispergiert



Euklidische Geometrie, es sind keine kolloidalen Partikel



verstößt gegen die Symmetriegesetze und die Medien, aus denen sich zusammensetzt



keine Manifestation lebender Materie. Gesetze



Asymmetrie wird nicht bemerkt. feste Symmetrie



                                   kristalline Strukturen



                                   kaputt. Dissymmetrie



                                   vielleicht in der Biosphäre



                                   gebildet nur aus



                                   unsymmetrisches Medium -



                                   „Geburt“ (Prinzip



                                   Curie).



 



III. Neues inertes natürliches. Neues lebendes natürliches.



der Körper der Körper ist geschaffen -- der lebende Organismus -



physikalisch-chemisch und nur von einem anderen geboren



geologische Prozesse eines lebenden Organismus.



Unabhängig von der bisherigen früheren Abiogenese in der Biosphäre ist dies nicht der Fall.



natürliche Körper, lebend oder nicht, und ein Zeichen seiner früheren



inert. Die Prozesse seiner Manifestation in der geologischen



Formationen können in der Zeit gehen. Lebender Organismus



lebende Körper, die sich in ihren eigenen verändern, werden von Generationen von geboren



Manifestationen und Biokosalleben gleich geben (in



natürliche Körper, eingebettet in die Essenz der Nähe)



lebender natürlicher Körper. Organismus (Redi-Prinzip).



                                   Während der geologischen



                                   keine Zeit



                                   jetzt herausgefunden



                                   die Gesetze der Mutationsprozesse und



                                   Geburt morphologisch und



                                   physiologisch anders



                                   neue Generation



                                   andere Organismen als



                                   alt (Artenentwicklung).



 



IV. Prozesse, die träge Prozesse erzeugen, die Leben schaffen



natürlicher Körper, reversibel zum natürlichen Körper,



Zeit. Der Raum ist in der Zeit irreversibel.



was sie gehen, ist nicht zu unterscheiden von vielleicht stellt sich heraus



isotrope oder anisotrope Folge einer bestimmten



Euklidische Räume. Zustand



                                   Raumzeit



                                   ein Substrat haben



                                   entsprechend nichteuklidisch



                                   Geometrie.



 



V. Keine Reproduktion. Erzeugt einen lebendigen natürlichen Körper



träger natürlicher Körper entsteht durch Fortpflanzung -



physikalisch-chemisch und die Schaffung eines neuen Lebens



geologische Prozesse eines natürlichen Körpers aus



synthetisch reproduzierbares bisheriges Leben



Experimente. natürlicher Körper aus



                                   Generation zu Generation. Es ist



                                   Komplex erstellt



                                   biochemischer Prozess



                                   ohne deine zu verlassen



                                   Zustand des Raumes.



 



VI. Die Anzahl der inerten natürlichen Die Anzahl der lebenden natürlichen



Körper ist unabhängig von der Größe der Körper quantitativ verwandt



Planeten, und wird mit den Dimensionen eines bestimmten bestimmt



Eigenschaften der planetaren Erdschale -



Materie-Energie. Biosphäre der Biosphäre. Gültig -- und



empfängt und gibt kontinuierlich erfordert Überprüfung -- arbeiten



Materie-Energie in die kosmische wissenschaftliche Hypothese von



Raum. Es gibt mit ihm einen kosmischen Austausch des Lebens



kontinuierliche natürliche Körper.



materielle Energie



austauschen.



 



VII. Bereich und Ausmaß der Manifestation Masse der lebenden Substanzen



träge natürliche Körper nicht (Aggregate des Lebens



innerhalb des Planeten und der natürlichen Körper begrenzt) ist nahe



ihre Masse variiert in der Grenze und anscheinend



geologische Zeit. bleibt



                                   beweglich unverändert in



                                   geologischer Verlauf



                                   Zeit. Sie ist entschlossen



                                   am Ende die Menge



                                   und Schwingungen der Strahlung



                                   Solarenergie



                                   Umarmung der Biosphäre.



 



Viii. Mindestgröße Mindestgröße live



inerter natürlicher Körper natürlicher Körper



bestimmt durch Dispersion bestimmt durch Atmung,



Materie-Energie -- ein Atom, hauptsächlich Gas



Elektron, Korpuskel, Nährstoffmigration von Atomen



Neutron usw. Maximum (nach dem Prinzip von E. Snyadetsky und



Die Größe wird durch die Größe der Loshmidt-Zahl bestimmt.



Planet, der selbst maximale Größe haben kann, durch



als bewacht angesehen werden



bio-natürlicher Körper. In geologischer Zeit nicht



Aspekt unserer Präsentation, es überschreitet die Dimensionen für



bestimmt durch die Größe der Tiere und Pflanzen,



Biosphäre, die speziell gleich Hunderten von Metern ist.



bio-natürlicher Körper. Es kommt wahrscheinlich darauf an



Der Größenbereich ist riesig -- tiefe Gründe



10 22 . Gelegenheit bestimmen



                                   Existenz im Biokos



                                   natürlicher Körper der Biosphäre



                                   Zustände des Raumes



                                   auf die Lebenden reagieren



                                   natürlicher Körper.



                                   Der Schwingungsbereich beträgt



                                   10 10 .



 



IX. Die chemische Zusammensetzung des Inerten Die chemische Zusammensetzung des Lebenden



natürliche Körper ganz natürliche Körper geschaffen



ist eine Funktion der Komposition für sich aus der Umwelt



die Umgebung, in der sie sind, die Umgebung, aus der sie stammen



werden erstellt. Es kann durch Ernährung und Atmung ausgedrückt werden.



so dass er entschlossen ist zu wählen, was sie brauchen



“Spiel“ von physikalisch und chemisch und Leben und Fortpflanzung -- für



geologische Prozesse bei der Schaffung neuen Lebens



Fluss der geologischen Zeit. natürliche Körper -



                                   chemische Elemente. Sie sind



                                   während anscheinend



                                   können ihre Zusammensetzung ändern



                                   Isotope, ändern Sie sie



                                   atomare Substanzen.



                                   Die überwiegende Mehrheit



                                   seine chemische Zusammensetzung



                                   sie schaffen wie



                                   unabhängig in bestimmten



                                   die Größe der Körper in der Biosphäre,



                                   in biocos natürlich



                                   der Körper des Planeten.



 



X. Menge der verschiedenen Chemikalien Menge der Chemikalie



Verbindungen -- Moleküle und lebende Verbindungen



Kristalle -- in inerten natürlichen Körpern und



natürliche Körper der Erdkruste, die Menge charakterisiert



- daher die Biosphäre, sie leben natürliche Körper



begrenzt. Es gibt nur wenige unbegrenzt. Wir wissen es bereits



Tausende von natürlichen „terrestrischen“ und Millionen von Arten von Organismen



wahrscheinlich sowohl „Raum“ als auch Millionen von Millionen



chemische Verbindungen -- Moleküle, die ihren Molekülen entsprechen und



und Kristallgitter.



räumliche Gitter. Auf diese Weise



begrenzt



die Anzahl der inerten Arten



natürliche Körper der Biosphäre und ihrer



bio-natürliche Körper.



 



XI. Alle natürlichen Prozesse in den natürlichen Prozessen der Lebenden



Bereiche natürlicher inerter Materiekörper in ihrer Reflexion in



- mit Ausnahme der Phänomene der Biosphäre zunehmen



Radioaktivität -- freie Energie reduzieren



freie Energie der Biosphäre.



(reversible Prozesse), in diesem



Fall freie Energie in



Biosphäre.



 



XII. Isotopenmischungen ändern sich offenbar



(terrestrische chemische Elemente) von nicht-isotopen Gemischen



variieren in inert natürlich ist charakteristisch für



Körper der Biosphäre (mit Ausnahme des Eigentums an lebender Materie.



radioaktiver Zerfall). Bewährt für Wasserstoff



Anscheinend gibt es auch Kalium. Phänomen



natürliche Prozesse außerhalb erfordert dringend



Biosphäre -- die Bewegung von Gasen unter genauer Untersuchung. Als



hoher Druck, dass es teuer ist



stören dann stetige Energie in der Migration



Isotopenmischung, aber mit den chemischen Elementen des Lebens



Zum anderen die theoretische Untersuchung von Substanzen



chemische Elemente von Meteoriten sollten real sein



- galaktische Materie -- scharf



zeigt an, dass Isotopen die chemische Freisetzung verzögern



Gemische in ihnen sind die gleichen wie in Elementen aus biogenen



irdische Elemente. Konstanz der Migration. Zum ersten Mal dies



Das Phänomen des Atomgleichgewichts wurde beobachtet. K.



nur in erster Näherung und von Baer für Stickstoff.



Vielleicht ist das echt



bestehende Abweichungen



mit mehr ans Licht kommen



empfindliche Technik.



 



Kapitel 10



 



Die Biowissenschaften sollten den physikalischen und chemischen Wissenschaften auf Augenhöhe sein und die Noosphäre abdecken.



 





143. Aus dem vorhergehenden Aufsatz geht absolut klar und wissenschaftlich sicher hervor, dass es in der Biosphäre zwischen dem lebenden natürlichen Körper und dem inerten oder biokositären natürlichen Körper ein undurchdringliches Gesicht gibt, das sich in präzisen, unwiderlegbar etablierten Phänomenen von enormer Größe und Bedeutung ausdrückt. Diese Phänomene gehen weit über das Leben hinaus und sind eng miteinander verbunden, charakteristisch für die Struktur der regulären Erdhülle -- der Biosphäre.



Die materiellen und energetischen Unterschiede zwischen den Gruppen natürlicher Körper im Vergleich zum vorherigen Absatz sind eine einfache Darstellung der Tatsachen und empirischen Verallgemeinerungen, die streng daraus abgeleitet sind. In diesem Vergleich gibt es keine Hypothesen und Theorien, auch keine wissenschaftlichen. Daraus folgt logischerweise, dass Biologen mit dieser Schlussfolgerung rechnen müssen und sie nicht unbeaufsichtigt lassen dürfen.



Dies ist nicht wirklich der Fall. Man kann mir sogar argumentieren, dass jede massenbiologische wissenschaftliche Arbeit normalerweise ideologisch in scharfem Widerspruch zu diesem großen realen Naturphänomen steht. Es wird von einem Biologen nicht berücksichtigt und nicht berücksichtigt. Die Biogeochemie als Zweig der Biowissenschaften zeigt zum ersten Mal genau und eindeutig ihre Bedeutung.



Die Biologie in dieser Hauptfrage für sie -- der Unterschied zwischen Lebenden und Toten -- hat eine jahrtausendealte Vergangenheit und hat starke Traditionen und Arbeitsfähigkeiten geschaffen, die die Biowissenschaften scharf von anderen Zweigen der exakten Wissenschaft unterscheiden. Es scheint mir, dass in einer etwas verzerrten Form der gleiche Unterschied zwischen lebenden natürlichen Körpern und inerten Körpern besteht, der in Sec. 142.



Die Biowissenschaften sind alle bis jetzt von Ideen und Fähigkeiten des Denkens umgeben und durchdrungen, die der exakten Naturwissenschaft im Wesentlichen fremd sind, da es sich um aktuelle wissenschaftliche Arbeiten und Gedanken handelt. Historisch gesehen stützte sie sich zunächst auf religiöse Ideen, dann auf religiöse und philosophische, schließlich auf philosophische, und stützte sich in einem solchen Ausmaß und in einem solchen Aspekt auf sie, dass dieser Staat im 20. Jahrhundert für alle konkreten Wissenschaften der trägen Natur längst in den Bereich der Tradition zurückgegangen war .



Die Biologie wird immer noch von ihnen verschlungen und durchdrungen. Dies hängt zum Teil von der Art des Forschungsgebiets ab. In ihrem Fachgebiet erfasst die Biologie alle Probleme und Wissenschaften, die den Menschen betreffen, und daher befinden sich ihre Forscher unweigerlich in einer anderen Position als Forscher träger Natur. Darin ist eine Person gleichzeitig Gegenstand und Gegenstand der Forschung. Im Denken des Biologen steht der Mensch unweigerlich an erster Stelle und dient daher als Vergleichsmaßstab für die Phänomene des Lebens. Aus diesem Grund spielen in der Biologie Phänomene, die im Wesentlichen zur Natur gehören (und bevor die Biosphäre in die Noosphäre und in die gesamte Natur übergeht), einen sekundären Platz ein -- Phänomene, die mit der geistigen Aktivität des Menschen verbunden sind. Gleichzeitig durchdringen und beherrschen religiöse und philosophische Denkfähigkeiten und ihre vorgefertigten Ideen sowie ein wissenschaftliches Verständnis der Natur häufig alle Bereiche der Geisteswissenschaften (sie müssen auch als Psychologie eingestuft werden). Basierend auf diesen Wissensgebieten erwies sich die wissenschaftliche Arbeit des Biologen, die nicht direkt mit dem Menschen verbunden war, in größerem Maße als die Wissenschaft der trägen Natur von der Philosophie erfasst, da das geistige Leben des Menschen als der höchste Ausdruck aller von ihm untrennbaren Lebewesen dargestellt wird. Lebewesen, von Bakterien über höhere Pflanzen und höhere Tiere bis hin zu Menschen, schienen ein einziges unauflösliches Ganzes zu sein, das vom Leben durch Materie bedeckt war. Anstatt natürliche Körper der Biogeochemie zu leben, stand das Leben in der Biologie an erster Stelle.



Zusammen mit dem Leben muss der Biologe, um es zu erklären und seine spezifische Manifestation in der lebenden Natur zu verstehen, die ausschließlich aus lebenden natürlichen Körpern besteht, Unterstützung bei dieser Herangehensweise an das Leben in religiösen und philosophischen Aufgaben suchen, die sich seit Jahrhunderten vollständig mit dem Leben beschäftigen. Gleichzeitig kam er zu einer völlig anderen Vorstellung vom Unterschied zwischen Lebenden und Inerten als der in Kap. 142.



Um den bestehenden Widerspruch zu verstehen, muss kurz auf den philosophischen Hintergrund der Biologie eingegangen werden.



144. Ich werde mich nur mit philosophischen Suchen befassen, die sich als solche bewusst in der wissenschaftlichen Arbeit von Biologen widerspiegeln. Ich werde alle philosophischen Ideen, die keine lebenden Vertreter haben, außer Acht lassen, jeden spürbaren Einfluss auf das moderne biologische Denken in seiner Massenmanifestation. In diesem Zusammenhang wurden zwei wichtige philosophische Bewegungen vorgebracht, die eine jahrtausendealte Geschichte haben -- die Suche nach idealistischen oder materialistischen Formen des philosophischen Denkens.



Der Einfluss des Materialismus -- in seinen verschiedenen Erscheinungsformen -- auf die wissenschaftliche naturhistorische Arbeit ist durchaus verständlich und sogar unvermeidlich, da materialistische Philosophien den Verlauf des Realismus darstellen, dh die gemeinsame Grundlage von Wissenschaft und Philosophie bei der Untersuchung der Probleme der Außenwelt. Der Naturforscher geht in seiner Arbeit von der Realität der Außenwelt aus und studiert sie nur innerhalb der Grenzen ihrer Realität.



Neben der wissenschaftlichen Arbeit in der ersten Hälfte des 19. Jahrhunderts wurde die Arbeit als gleichwertige und naturphilosophische Arbeit auf dem Gebiet der deskriptiven Naturwissenschaften, insbesondere der Biowissenschaften, fortgesetzt.



Dies erklärt den enormen Einfluss idealistischer philosophischer Forschungen auf das biologische Denken im Laufe der Geschichte. Dies ist auf die große philosophische Bewegung zurückzuführen, die die westeuropäische, am meisten deutsche Philosophie des späten 18. und frühen 19. Jahrhunderts geprägt hat. Die weltweite Bedeutung in der Geschichte des menschlichen Denkens, deren Einfluss -- in seinen Epigonen -- bisher deutlich zum Ausdruck kommt.



Die unzureichend tiefen philosophischen, materialistischen Ideen kamen erst Mitte des 19. Jahrhunderts deutlich zum Ausdruck. Zu dieser Zeit traten sie in Deutschland im Zusammenhang mit der wissenschaftlichen und philosophischen Arbeit von Karl Marx und Friedrich Engels in den Einflussbereich des Hegelianismus ein. In dieser neuen, radikal veränderten Form erhielten sie nach der Oktoberrevolution staatliche Unterstützung als offizielle Philosophie in unserem Land. Und hier haben wir mangels Freiheit der philosophischen Suche einen großen Einfluss auf die biologisch-wissenschaftliche Arbeit. Aber dieser Einfluss ist rein oberflächlich, man könnte sogar sagen, offiziell formal. Bisher ist in dieser philosophischen Bewegung kein einziger ursprünglicher Denker aufgetaucht, und es gibt keine Manifestation ihres Einflusses auf das kreative biologisch-wissenschaftliche Denken, der aus wissenschaftlichen Ergebnissen ersichtlich ist.



Um die wahre Bedeutung dieser komplexen Form der materialistischen Konzeption, die der Hegelianismus in der biologisch-wissenschaftlichen Arbeit der Welt durchdringt, richtig einzuschätzen, genügt es, sich ihrer Manifestation zuzuwenden, in der die Freiheit des philosophischen Denkens besteht. Es geht dort in seiner Bedeutung unter den unzähligen neuen philosophischen Suchen in ihrer Reflexion in den biologischen Wissenschaften verloren. Diese Strömung in unserer ideologischen Umgebung, die sich in biologischen Problemen manifestiert, ist eine Gewächshauspflanze, deren Wurzeln kaum an der Oberfläche haften.



145. Der Einfluss des philosophischen Denkens als Ganzes spiegelt sich in unserer Zeit viel stärker in biologischen Problemen wider, nicht in seinen materialistischen Erscheinungsformen.



Hier begegnen wir teilweise einer philosophischen Überarbeitung der zeitgenössischen Bedeutung der Philosophie in der wissenschaftlichen Arbeit -- einerseits mit philosophischer Skepsis und andererseits mit Versuchen einer neuen philosophischen Kreativität, die die Philosophie unter dem Einfluss der mächtigen wissenschaftlichen Bewegung des 20. Jahrhunderts wieder aufbaut. Neue Formen realistischer Philosophie werden geschaffen. Es scheint mir, dass einige dieser neuen Formen des philosophischen Denkens die ernsthafte Aufmerksamkeit eines Naturforschers verdienen.



Skeptische Formen des philosophischen Denkens kommen aus dem Primat der Wissenschaft auf ihrem Gebiet über Philosophien und Religion. Sie erkennen an, dass in den Bereichen, die von der wissenschaftlichen Arbeit abgedeckt werden, die Rolle der Philosophie hauptsächlich mit der Analyse wissenschaftlicher Konzepte zusammenhängt, wobei die jahrhundertealte Arbeit des philosophischen Denkens in ihrer historischen Manifestation verwendet wird. Es gibt jedoch Wissensbereiche, in denen die Wissenschaft noch keinen festen Boden hat oder die mit ihren Methoden möglicherweise überhaupt nicht angegangen werden können. Solche Bereiche sind philosophisch akzeptabel, aber philosophische Schlussfolgerungen aus ihrer Studie sind für Wissenschaft und Wissenschaft nicht erforderlich, wenn sie genau wissenschaftlich fundierten Tatsachen widersprechen und logisch korrekt daraus empirisch-wissenschaftliche Verallgemeinerungen gemacht werden.



Die Wissenschaft ist untrennbar mit der Philosophie verbunden und kann sich in ihrer Abwesenheit nicht entwickeln. Es kann im Widerspruch zu den Grundlagen der Philosophie (ganz zu schweigen von skeptischen Philosophien) oder in ihren realistischen Systemen oder in ihren Systemen stehen, die genau wissenschaftlich fundierte Wahrheiten als eine wirklich unbestreitbare Tatsache anerkennen und der Ansicht sind, dass es für sie keinen solchen Widerspruch zu ihnen geben kann wie zum Beispiel eine Reihe neuer indischer Philosophien. Gleichzeitig kann die Wissenschaft nicht so tief in die Analyse von Konzepten eintauchen. Die Philosophie schafft sie und stützt sich nicht nur auf wissenschaftliche Arbeit, sondern auch auf die Analyse des Geistes.



Unter den verschiedenen philosophischen Systemen unserer Zeit, die zunehmend unter dem Einfluss wissenschaftlicher Erkenntnisse entstehen, gibt es eine Reihe von Philosophien, die Vorläufer ihres zukünftigen Wohlstands, mit denen ein moderner Wissenschaftler nur rechnen kann. Unter ihnen Aufmerksamkeit schenken sollen Biologen jetzt Philosophie der Ganzheitlichkeit, 190 im Wesentlichen auf einer Analyse des natürlichen Körpers gebaut, die die Grundlage der biogeochemischen Arbeit. Es scheint mir, dass es oder eine andere dazu analoge Philosophie den fruchtlosen Streit zwischen den Mechanikern und Vitalisten -- größtenteils schulisch -, der von Philosophen in die Biologie eingeführt wurde und sich nicht aus beobachteten Tatsachen ergibt, irgendwann auflösen wird. Diese Philosophie des Holismus ist auch deshalb interessant, weil sie auf neue Weise versucht, die Erkenntnistheorie wieder aufzubauen, die tief im letzten Jahrhundert im wissenschaftlichen Denken von Physikern und Mathematikern verwurzelt war und es ermöglichte, einige grundlegende wissenschaftliche Konzepte zu klären, bevor sie im 20. Jahrhundert zum Talmudismus und zur Scholastik überging. Aufgrund ihrer Loslösung von bestimmten realen Tatsachen und der eingehenden Analyse der allgemeinen Konzepte der Erkenntnis, die zu den wichtigsten kontroversen und unklaren philosophischen, logischen und psychologischen Konstruktionen führte, fand die Erkenntnistheorie in der Naturwissenschaft nur unter Mathematikern und theoretischen Physikern einen geeigneten Boden. In anderen Bereichen der Naturwissenschaften wird es -- ohne nennenswerte wissenschaftliche Ergebnisse -- hauptsächlich von Philosophen und Wissenschaftlern verwendet, die eine philosophische Ausrichtung der sogenannten wissenschaftlichen Philosophie haben, die sich wesentlich von lebendiger wissenschaftlicher Arbeit unterscheidet.



Es scheint mir, dass die Philosophie des Holismus mit seinem neuen Verständnis eines lebenden Organismus als eines einzigen Ganzen in der Biosphäre, d. H. Eines natürlichen, sich selbst identifizierenden lebenden Körpers, zum ersten Mal versucht, der Erkenntnistheorie einen neuen Blick zu geben. Bis jetzt wurde sie von einem Naturforscher, Beobachter der realen Biosphäre, ignoriert, der ständig mit realen natürlichen Körpern konfrontiert ist, mit den Zehntausenden von individuellen Fakten, die er in seiner Arbeit abdecken und berücksichtigen musste. Wir stehen jetzt vor einer merkwürdigen philosophischen Bewegung, die für die Lösung des besonderen Problems der undurchdringlichen Grenze zwischen lebenden und inerten natürlichen Körpern der Biosphäre, d. H. Lebenden und Toten, in ihrer wissenschaftlichen realen Identifizierung von großer Bedeutung sein könnte.



Dieser philosophische Trend ist keiner. Whiteheads Philosophie eröffnet vielleicht merkwürdige Ansätze. 191



Einige Echos des neuen indischen philosophischen Denkens können als bemerkenswert angesehen werden.



Die nahe Zukunft wird vielleicht neue, wissenschaftlich akzeptable Wege für eine philosophische Analyse grundlegender biologischer Konzepte eröffnen.



146. Angesichts des gegenwärtigen Standes der Biologie und ihrer untrennbaren Verbindung mit der Philosophie werde ich hier versuchen, in den Thesen die Beziehung zwischen Lebenden und Toten (dh nur die Beziehung zwischen lebenden und toten natürlichen Körpern der Biosphäre) zu verringern, die jetzt die wissenschaftliche Arbeit der Biologen dominiert. Diese Thesen geben nur ein allgemeines Bild der wissenschaftlichen Massenarbeit -- einzelne Wissenschaftler, die sich außerhalb des Mainstreams der biologischen Arbeit befinden, bleiben am Rande.



Sie können berücksichtigen:



1. Es gibt keine wissenschaftlich genauen Daten, die belegen, dass im Leben besondere Lebenskräfte existieren, die nur den Lebenden eigen sind. Selbst als wissenschaftliche Hypothese (und nur in Bezug auf Individuen, die lebende Materie bilden) sind diese Ideen, die einst die Wissenschaft beherrschten, in unserer Zeit fast ein Anachronismus.



2. Darstellungen, die das Wesen des Lebens und den Unterschied zwischen lebenden Organismen und inerten Naturkörpern in Form von besonderer Lebensenergie, Entelechie, Monaden, Lebensimpulsen (lan vitale) usw. erklären, die sich von Zeit zu Zeit ergeben, sind im Wesentlichen bildliche Ausdrucksformen des Lebens Kräfte, vergängliche Schöpfungen des Geistes, die in der Vergangenheit nie zu einer wissenschaftlich wichtigen Entdeckung oder Verallgemeinerung geführt haben.



3. Mitte des 19. Jahrhunderts. Die „Vitalität“ in der wissenschaftlich-biologischen Arbeit des Arztes und Naturforschers ist vollständig verschwunden. Zu diesem Zweck konnten sie nicht durch ihre in Absatz 2 angegebenen ideologischen Epigonen ersetzt werden. Nachdem sie all diese naturphilosophischen Erklärungen verworfen hatten, gingen Naturbiologen mit überwältigender Mehrheit den Weg, Wildtiere zu erforschen, und ignorierten ihre lebendige Natur als Natur, materiell und energetisch nicht von inert zu unterscheiden. Zum Teil gingen sie von materialistischen philosophischen Vorstellungen aus, dass es im Wesentlichen keinen Unterschied zwischen lebender und toter Natur gibt und dass am Ende alle Phänomene des Lebens durch physikalische und chemische Manifestationen bis zum Ende erklärt werden, so wie alle Phänomene der inerten Materie erklärt werden. Aber Naturbiologen, die diese philosophische Prämisse im Wesen des Glaubens nicht teilten, gingen denselben Weg, glaubten jedoch, dass sie, wenn sie diesen Weg eingeschlagen hätten, entweder auf neue Phänomene stoßen würden, die diese Hypothese ablehnen müssten, oder es würde sich als wahr herausstellen .



4. Man kann jetzt sehen, dass der Biologe nach fast 100 Jahren Weltarbeit am Ende keine einzige Anweisung erhalten hat, mit der er 1938 behaupten könnte, er sei näher an der Klärung des Problems als 1838. Tatsächlich stellte er die philosophische Frage nach der Vitalität und ihren Analoga, wandte jedoch nur wissenschaftliche Experimente und Beobachtungen an, die ihm zur Lösung zur Verfügung standen. Da er jedoch aufgrund der Unrichtigkeit dieser Hypothese nicht von einer wissenschaftlichen, sondern von einer philosophischen Hypothese ausging, stellte er seine wissenschaftlichen Experimente und Beobachtungen unter die Bedingungen, die für eine Lösung am ungünstigsten waren. Bei aller Aufmerksamkeit richtete sich die Aufmerksamkeit in diesem Fall nicht auf die Suche nach dem Unterschied zwischen Lebenden und Inerten, sondern auf die Suche nach Ähnlichkeit gemäß der anfänglichen philosophischen Prämisse. In dem riesigen unerforschten Feld der Phänomene werden immer unendlich viele wissenschaftliche Fakten enthüllt, die oft äußerst interessant sind und wissenschaftliche Forschung erfordern. Die Verfügbarkeit wissenschaftlicher Forschungskräfte ist zwangsläufig begrenzt. Da der Forscher die Bedeutung der neu entdeckten Tatsachen und ihr wissenschaftliches Interesse nicht sofort einschätzen kann, lenkt er seine Arbeit unweigerlich in Richtung Ähnlichkeit, in Wirklichkeit wählt er sie nur aus. Mit dieser Art wissenschaftlicher Arbeit kann die Unterscheidung zwischen lebend und träge übersprungen werden; Wie wir gesehen haben (§ 142), wurde es tatsächlich von Biologen vermisst. Diese Phänomene erwiesen sich als biologisch nahezu unstudiert.



5. Basierend auf dem gleichen Konzept der Identität der Lebenden und der Inerten, das während der endgültigen Vertiefung der Studie, der Lebenden und der Inerten enthüllt wurde, stellte der Biologe ein weiteres Problem auf, das enorme Arbeit verursachte und den Gedanken auf den falschen Weg lenkte. Diese Arbeit war bisher erfolglos.



Dies ist das Problem der spontanen Keimbildung lebender Organismen aus inerter Materie. Die überwiegende Mehrheit der Biologen, die auf den philosophischen Konzepten des Materialismus oder auf der wissenschaftlichen Hypothese der Möglichkeit der Identität des Lebenden und des Inerten beruhen, ist von der Unvermeidlichkeit seiner Existenz überzeugt. Darüber hinaus wird allgemein angenommen, dass die Abiogenese bei jedem Schritt in der uns umgebenden Biosphäre stattfindet. 192 Andere glauben, dass dies in einer der Epochen der geologischen Geschichte des Planeten geschehen ist. In diesem letzteren Fall ist es, wie in Kap. 142, kann nicht geleugnet werden, aber es erfordert Umweltbedingungen, die uns möglich, aber im Wesentlichen unklar erscheinen. Dieser Zustand schafft auf der Erde diesen besonderen Raumzustand, der den Raum des Körpers eines lebenden Organismus von inerten natürlichen Körpern unterscheidet. 193 Außerhalb lebender Organismen ist ein solcher Raum in der Biosphäre unbekannt.



6. In den letzten Jahren wurde in der Biosphäre ein neues Phänomen der Existenz lebender Organismen oder ihrer Stadien entdeckt, das für unsere Augen unsichtbar ist und selbst mit den stärksten Mikroskopen im ultravioletten Licht bewaffnet ist. Dies sind Organismen gleicher Größe mit Molekülen, dh etwa 10 -- 6 cm. Dies ist das Phänomen der Viren, die anscheinend eine große Rolle in den Lebensprozessen der Biosphäre spielen. Viren haben Fortpflanzung. Ihre Cluster sind mikroskopisch sichtbar. Sie produzieren eine Vielzahl von Krankheiten pflanzlicher und tierischer Organismen. In latenter Form in der Biosphäre wurden sie in biokosaler Materie gefunden -- in Böden, in der Troposphäre, in natürlichen Gewässern; Es besteht kaum ein Zweifel daran, dass sie sich in der Hydrosphäre befinden -- im Meerwasser und in Meereskörpern. Stanley identifizierte sie 1936 in Form eines homogenen chemischen Körpers -- eines Proteins mit einer bestimmten chemischen Formel und Größe eines Moleküls. 194 Diese Beobachtungen von Stanley wurden überprüft, bestätigt und fanden andere Proteinkörper, die ebenfalls in den „Kristallen“ erhalten wurden und auch eine bestimmte chemische Formel besaßen.



Wenn diese Phänomene in der Form bestätigt würden, wie sie von Biologen und Biochemikern beschrieben wurden, hätten wir hier „lebende Proteine“, deren Existenz von einer Reihe von Biologen 195 zugelassen und auf dieser Grundlage als möglich angesehen wurde. Natürlich könnte jeder Chemiker mit solchen Eigenschaften den gleichen Standpunkt vertreten. Wir müssen jedoch die Schlussfolgerung klarstellen: Derzeit können wir nur behaupten, dass diese Viren -- Proteinmoleküle -- bisher nur in lebenden Organismen vorkommen -- das heißt, sie werden in dem ihnen entsprechenden speziellen Raumzustand gebildet.



Die Sache ist jedoch nicht so einfach. Stanley und nach ihm andere erhielten Proteine ​​- Viren durch Kristallisation mit Ammoniumsulfat, aber sie bewiesen zum einen nicht, dass es sich tatsächlich um Kristalle handelt -- das heißt zu dreidimensionalen anisotropen homogenen Körpern und zum anderen, dass diese Kristalle frei von Viren sind.



Es ist bekannt, dass Kristalle von Proteinkörpern besondere Eigenschaften haben, insbesondere, dass sie in Flüssigkeiten quellen. Ihre Wachstumsbedingungen wurden nicht untersucht; kann nicht als Beweis für die Einheitlichkeit des Proteins durch wiederholte Rekristallisation in (NH4) 2SO4 angesehen werden. Wenn Proteinkristalle mit ihrer Intususzeption anschwellen und wachsen, können die kleinsten Viren nicht einmal durch zehnfache Kristallisation getrennt werden, wie dies bei Stanley der Fall war. Darüber hinaus wurde die Schlussfolgerung über die Kristallstruktur dieser Proteine ​​jedoch nur auf der Grundlage einer einfachen mikroskopischen Beobachtung ihres Aussehens gezogen. Dies ist kein Beweis.



Bis zum letzten Jahr gab es keine einzige Beobachtung, die die Gleichmäßigkeit von Proteinkristallen und ihre dreidimensionale Anisotropie bewies. Kristallographische Messungen für Proteine ​​wurden nicht durchgeführt. Unter diesen Bedingungen war es durchaus akzeptabel, dass es sich bei Kristallen von Proteinen, die Viren enthalten, um flüssige oder mesomorphe Körper handelt. Und wenn dem so ist, dann sind es immer Proteine ​​mit unsichtbaren Viren, das heißt, es gibt kein lebendes Protein.



Im vergangenen Jahr wurden einige wichtige Werke veröffentlicht, die es uns ermöglichen, dies genauer zu formulieren. Unabhängig voneinander haben Bernal und insbesondere Bowden und Mitarbeiter 196 bewiesen, dass die kristallinen Proteine ​​von Stanley und anderen keine Kristalle sind, wenn sie im Röntgenlicht untersucht werden, sondern entweder Flüssigkeiten oder feste mesomorphe Strukturen. Sie besitzen nicht die Eigenschaften homogener dreidimensionaler anisotroper Strukturen. Gleichzeitig bewies die Arbeit von Bernal und seinen Mitarbeitern 197 eine homogene anisotrope Struktur, die den Kristallen für Hämoglobin und eine Reihe von Proteinen vollständig entspricht. Die neue exakte Technik ermöglichte es erstmals Proteinkristallen, ihre Elemente im räumlichen Gitter numerisch auszudrücken. Dies erwies sich für Proteine ​​mit den Eigenschaften von Viren als unmöglich. Anscheinend sollte in dieser Form die Frage nach der Existenz lebender Proteine ​​mit einer gründlicheren Kontrolle verschwinden. Ohne den Tatsachen zu widersprechen, können sie jedoch als Proteine ​​betrachtet werden, die lebende (möglicherweise in einem latenten Zustand) Viren enthalten. Weder der Flüssigkristall kein isomorpher fester Körper kann nicht von den kleinsten Viren von 10 getrennt werden -- 6 cm Größe „Rekristallisation“, auch wenn mehrere als ausreichend erachtet , um die Proteine herzustellen , in filtrierbare Viren eintreten. In diesen mesomorphen oder flüssigen „Kristallen“ gibt es während ihrer Bildung keine Kristallisationsströme, die die Kristallisation auch von Körpern mit Abmessungen von 10 -- 7 cm beeinflussen und dadurch die bei der Kristallisation erhaltenen Substanzen klären können.



147. Es kann nun nützlich sein, aus dem Archiv der Wissenschaft die Arbeit des halb vergessenen Forschers A. Besham (1816-1908) abzurufen. 198 Das Schicksal dieses Forschers ist äußerst eigenartig. Wir werden später sehen, dass er ein direkter Vorgänger und Unterstützer von Pasteur ist, wenn es darum geht, Asymmetrie herzustellen, eine der Hauptmanifestationen lebender Organismen. Aber alle Versuche von Besham, auf die Bedeutung seiner Arbeit und seines Kritikers Pasteur aufmerksam zu machen, fanden kein Echo. Nachdem er fast 100 Jahre alt geworden war, überlebte er Pasteur (der sechs Jahre älter war) um dreizehn Jahre und veröffentlichte vor seinem Tod (1905) eine nicht ganz unparteiische, aber ernsthafte Aufmerksamkeit für die Werke von Pasteur. 199 Ihre Bedeutung für dieses und eine Reihe anderer Probleme beginnt sich jetzt zu klären. 200



Besham ist der Vorläufer der Wissenschaftler, die das Konzept der Viren etabliert haben -- eine unsichtbare Lebensform von der Größe von Molekülen. Er glaubte, dass diese kleinsten lebenden Körper alle Organismen durchdringen und eine große Rolle in ihnen spielen. Genau wie die Zelle, in der sie sich befinden, existieren sie auf unbestimmte Zeit und werden nur durch äußere Ursachen zerstört. Er nannte sie Mikroenzyme und gab ihre chemische Analyse. Das Interesse seiner Arbeit liegt in der Tatsache, dass er auf die Biosphäre aufmerksam machte und zu beweisen versuchte, dass sie im Boden, in sedimentären und organogenen Gesteinen, im Meerwasser weit verbreitet sind. 201



Beshams Arbeit in dieser Richtung verdient Aufmerksamkeit, Wiederholung und Überprüfung mit einer neuen Technik, die in ihrer Genauigkeit mit Beshams Technik und in der neuen Umgebung, die durch die Entdeckung gefilterter Viren geschaffen wurde, nicht zu vergleichen ist. 202



148. Das Versagen [der Reproduktion] der Abiogenese mit den ständig andauernden Versuchen, auf diese Weise einen lebenden Organismus zu erhalten, und die Kritik an diesen Versuchen, die im Wesentlichen auf einem gesunden Empirismus beruhen, veranlassten viele Biologen, die sich der Einheit des Lebens und des Ausmaßes des Prozesses, der ihm in der Biosphäre entspricht, bewusst sind, nach seinem anderen Ursprung zu suchen auf unserem Planeten -- Leben aus dem Weltraum bringen. Abiogenese ist, wie Pasteur betonte, nur in einer unsymmetrischen Umgebung denkbar. Es ist nicht außerhalb der lebenden Organismen auf unserem Planeten. Die organogene Substanz der Biosphäre, die einige Eigenschaften des dem Leben entsprechenden Raumzustands beibehält, ist kein solcher Raumzustand. Es enthält nur inerte Substanzen, bei denen die Gleichheit von rechts und links durch ein früheres Leben verletzt wird. Mit dem Tod des Körpers und seinem Übergang in eine inerte Substanz verschwand die Ursache dieser Verletzung, die eine Manifestation des Lebens war. Die bisherigen Experimente zur Abiogenese in einer solchen Biokosalumgebung führten zu negativen Ergebnissen. 203



Wie aus Kap. 142 kann die Möglichkeit der Existenz einer solchen Umgebung in anderen geologischen Epochen nicht geleugnet werden. Und die Annahme eines solchen Phänomens widerspricht nicht den biologischen Vorstellungen. Geologisch haben wir jedoch keine Hinweise auf die Realität dieses Phänomens. Wenn wir uns der Abwanderung des Lebens aus dem Weltraum zuwenden, müssen wir seine Möglichkeit testen. Sehr gründliche Experimente, die kürzlich von A. Becquerel über die Beständigkeit von Mikroorganismen gegen niedrige Temperaturen im Weltraum und deren Eindringen durch kontinuierliche ultraviolette Strahlung durchgeführt wurden, führten ihn zu dem Schluss, dass niedrige Temperaturen kein Grund sind, der die Möglichkeit des Eindringens latenter Lebensformen auf die Erde ausschließt, sondern ultraviolette Strahlen wirken tödlich. Becquerel kam von hier zu dem Schluss, dass dieser Prozess unmöglich ist. Es scheint mir jedoch, dass angesichts der unendlichen Vielfalt lebender Organismen und ihrer extremen Anpassungsfähigkeit eine solche Schlussfolgerung verfrüht ist. Neue Erfahrungen sind erforderlich.



Tatsächlich entspricht die Frage in dieser Form -- in Form des Eindringens einzelner Unteilbarer auf die Erde -- nicht dem tatsächlich in der Biosphäre beobachteten Phänomen. Die Frage ist nach der Existenz einer komplexen Symbiose -- der Schaffung der Biosphäre.



149. Aus alledem kann geschlossen werden, dass es in der Biologie auf der Grundlage der darin verfügbaren wissenschaftlichen Fakten und empirischen Verallgemeinerungen und der Art ihrer Probleme, wie sie jetzt gestellt werden, keine feste Unterstützung gibt, um zu entscheiden, ob es einen unpassierbaren Unterschied zwischen Leben und Leben gibt inerte natürliche Körper der Biosphäre. Obwohl die Biologie in ihrer Arbeit von der Annahme ausgeht, dass es keinen solchen Unterschied gibt, um das Leben zu erklären, wird diese Abwesenheit von ihr als fertig akzeptiert und folgt nicht aus den von ihr genau festgelegten Tatsachen und Verallgemeinerungen. Die Analyse zeigt, dass die Frage tatsächlich von ihr offen gelassen wurde.



Der Biologe hat die entgegengesetzte wissenschaftliche Verallgemeinerung, die durch die Biogeochemie in das wissenschaftliche Denken eingeführt wurde, über den scharfen Unterschied zwischen lebenden Organismen aus allen inerten Körpern der Biosphäre, der durch keinen natürlichen Prozess verletzt wird, noch nicht kritisiert oder berücksichtigt. Da wir durch Fakten motiviert bleiben, bleibt dies bedingungslos wahr.



Zwei gegensätzliche wissenschaftliche Schlussfolgerungen bleiben nebeneinander und berühren sich nicht.



Das kann natürlich nicht lange dauern.



Es scheint mir, dass der Grund dafür sehr komplex ist. Hundert Jahre sind seit dem Zusammenbruch vitalistischer Ideen vergangen, die einst die wissenschaftliche Arbeit der Biologen beherrschten, aber nichts Positives wurde an ihre Stelle gesetzt.



Einer der Hauptgründe dafür ist, dass das Phänomen des Lebens in der Biologie nicht in seiner vollen Erscheinungsform spielt. Das Phänomen des Lebens in seiner Größenordnung kann nicht wissenschaftlich gelöst werden, da es nur von einem lebenden Organismus ausgeht, von einem natürlichen Körper, an dem ein Biologe tatsächlich beteiligt ist, ohne zuvor eine genaue logische und nicht philosophische Analyse der Konzepte des Lebens und eines lebenden Organismus zu haben, ohne ihn von seiner Umwelt zu lösen, ohne einen solchen die Analyse seiner Position in der Biosphäre. Ein Biologe spricht normalerweise über das Leben, studiert aber einen lebenden Organismus. Seine verallgemeinernde Idee zielt auf das Konzept des Lebens ab, nicht auf einen lebenden Organismus.



In seiner logischen Hauptkategorie für wissenschaftliche Arbeit nimmt er einen lebenden Organismus oder vielmehr die Gesamtheit lebender Organismen, und für seine verallgemeinerten Ideen nimmt er Leben, das nicht streng durch den Körper begrenzt ist. Der Biologe stammt von einzelnen lebenden Organismen, die von der Biosphäre abstrahiert und isoliert sind. Das Leben ist ein planetarisches natürliches geologisches Phänomen, das die Biosphäre und die Noosphäre aufbaut und sich in den Massen der Materie manifestiert, die im Vergleich zur Masse der Biosphäre vielleicht unbedeutend sind, aber in der Masse der Biosphäre und in ihrem Energieeffekt genau quantifiziert werden und eine führende Rolle in der Biosphäre spielen.



Der Biogeochemiker, der sich in dieser Hinsicht mit Leben befasste und sich hauptsächlich mit den biologischen Manifestationen des Lebens und der Gesamtheit lebender Organismen befasste, stieß sofort auf einen scharfen, undurchdringlichen physikalisch-chemischen Unterschied zwischen lebender Materie und inerter Substanz.



Es gibt kein „Leben“ außerhalb des lebenden Organismus in der Biosphäre. Auf planetarischer Ebene ist das Leben die Gesamtheit der lebenden Organismen in der Biosphäre mit all ihren Veränderungen während der geologischen Zeit.



Diese Position, die tatsächlich von einem Biologen anerkannt wurde, fehlt in seinen theoretischen Prämissen oder ist eher verdeckt.



Dies ist jedoch nur einer der Hauptgründe für den Unterschied in den Schlussfolgerungen der beiden Strömungen des biologischen Denkens, des alten Jahrhunderts und der neuen, biogeochemischen Untersuchung des Lebens auf planetarischer Ebene, in Bezug auf Atome.



Der zweite, anscheinend der Hauptgrund, jedenfalls der Hauptgrund, ist, dass alle Positionen der Biologen -- sowohl vitalistisch als auch materialistisch -- nicht aus wissenschaftlichen Fakten hervorgegangen sind, sondern durch philosophische und religiöse Ideen geschaffen wurden. Sie als solche sind ein fremder Körper in der Masse der Fakten, mit denen sich ein Biologe in seiner täglichen wissenschaftlichen Arbeit befasst.



150. Es ist kaum möglich, sich mit Kritik und Diskussion von Versuchen materialistischer oder vitalistischer Lebensvorstellungen zu befassen. Es wäre richtiger, sie beiseite zu lassen. Das Argument in ihrem philosophischen Umfang wird uns keinen einzigen Schritt bewegen. Alles, was gesagt werden könnte, ist im Grunde gesagt. Ein Bild von der wirklichen Geschichte ihres Eindringens in die Wissenschaft zu geben, würde einen tieferen Einblick in die Geschichte der philosophischen Forschung erfordern, deren Konsequenz sie sind, die mich weit vom Hauptziel meines Buches ablenken und gleichzeitig nichts Neues hervorbringen würde, das die aufgewendete Arbeit rechtfertigt.



Zuallererst müsste ich eine große harte Arbeit überwinden -- nach den primären Quellen. Denn die unvermeidliche Vorarbeit für eine solche Studie wird kaum berührt und nicht im richtigen Umfang durchgeführt. Wir können nicht einmal einen allgemeinen korrekten Überblick über den äußeren Verlauf ihres Eindringens in das wissenschaftliche Denken geben. Befürworter unterschiedlicher Ströme geben unterschiedliche Schemata an, um zu verstehen, welche Richtigkeit ohne eine große neue Arbeit an den Primärquellen unmöglich ist.



Wir können uns auf die folgende kurze Schlussfolgerung beschränken, die für unseren Zweck ausreicht. Denn es ist klar und kaum zweifelhaft, dass sowohl materialistische als auch vitalistische Vorstellungen vom Leben fertig in die Biologie eingetreten sind und in einem anderen, ihm fremden Ideenfeld aufgewachsen sind.



Die einzelnen biologischen Positionen, die mit diesen Darstellungen verbunden sind, sind eher Illustrationen für sie als ihr Beweis oder eine Folge davon. Soweit ich das beurteilen kann, beziehen sie sich außerdem hauptsächlich auf die Struktur eines separaten Organismus und gehen damit über die Grenzen der Biogeochemie hinaus, die sich mit der Manifestation des gesamten Lebens -- einer Reihe von Organismen -- in der Biosphäre und in der Noosphäre sowie deren Reflexion befasst Strukturen -- ein Teil des Lebens auf der Gesamtheit der Organismen.



Letztendlich geben uns die jahrhundertealten philosophischen Suchen von Philosophen und Biologen nach dem Unterschied zwischen Leben und Trägheit keine wissenschaftlich wichtigen Hinweise, um die Existenz von Ähnlichkeiten oder Unterschieden zu erkennen.



Ihre Wurzeln liegen tief in der Vergangenheit, in der jahrhundertealten Kultur des Westens -- sowohl im theologischen als auch im philosophischen Denken -- sowie in ihrer alltäglichen Reflexion in der Wissenschaft der letzten Jahrhunderte -- hauptsächlich in den Geisteswissenschaften -- dringen sie in Historiker, Ärzte und Soziologen ein.



Diese historische Vergangenheit -- philosophisch und religiös -- muss vom Naturforscher berücksichtigt und verstanden werden, wenn er sich diesen Ideen nähert.



Ein Naturforscher sollte dies in seiner wissenschaftlichen Arbeit berücksichtigen. Er kann dieser Vergangenheit nicht gleichgültig gegenüberstehen, wie er es jetzt oft tut. Denn er kann fertige philosophische Ideen nicht akzeptieren, ohne seiner Arbeit Schaden zuzufügen, nur wenn sie sein kreatives Denken nicht einschränken oder wenn sie ihm aus der beobachtbaren wissenschaftlichen Realität herausfließen.



Wenn er mit ihnen rechnet, bringt er unweigerlich Konsequenzen in seine wissenschaftliche Arbeit ein, die er nicht erkennt und die er ohne eingehende Kritik, die über seine Stärken hinausgeht, nicht vorhersehen kann.



Für den Naturforscher wird der richtige Weg sein, diese philosophischen Ideen über seine Arbeit beiseite zu lassen und nicht mit ihnen zu rechnen. Daraus wird seine wissenschaftliche Arbeit nur an Klarheit und Klarheit gewinnen.



151. Aber die gegenwärtige Situation der Biologie und ihre Exkursionen in die Philosophie sind auch schädlich für die Philosophie.



Die erwartungsvolle Haltung des Naturforschers gegenüber den Affirmationen der Philosophie erweckt bei Philosophen den Eindruck, dass Wissenschaftler anhand ihrer Daten die Grundprinzipien der philosophischen Strömungen des Materialismus über das Fehlen eines fundamentalen Unterschieds zwischen Leben und Trägheit erkennen. Im allgemeinen Verlauf des biologischen Denkens sind vitalistische Ideen so weit in die Vergangenheit zurückgegangen, dass ihre wahre Bedeutung in der Massenarbeit wenig Wirkung hat. Die überwiegende Mehrheit der Naturforscher ist weit von ihnen entfernt.



Naturphilosophen, deren Bedeutung im modernen philosophischen Denken in seinem globalen Umfang gering ist, scheinen einen soliden Boden zu bekommen und sich in ihren Zweifeln zu beruhigen. Dies spiegelt sich in ihrer Arbeit wider, die langsam einfriert und zu einer trockenen formalen Scholastik oder einem verbalen Talmudismus ausartet, insbesondere in solchen Fällen wie in unserem Land, in dem der dialektische Materialismus eine Staatsphilosophie ist und die starke Unterstützung staatlicher Macht genießt, ideologische und sachliche Unmöglichkeit einer freien Kritik daran und die freie Entwicklung aller anderen philosophischen Ideen.



Der offizielle dialektische Materialismus selbst, eine von vielen Formen dieser Tendenz des philosophischen Denkens, besitzt jedoch keine solche Freiheit. In der Zwischenzeit wurde er nie systematisch und philosophisch bis zum Ende ausgearbeitet, voller Zweideutigkeiten und Gedankenlosigkeit. In den letzten zwanzig Jahren haben sich seine offiziellen Aussagen mehr als einmal geändert, die ersteren wurden als ketzerisch anerkannt, neue wurden geschaffen. Unsere Philosophen sollten mit der harten Disziplin, in der sie arbeiten, dieser neuen unter der Androhung von Verfolgung und materiellen Widrigkeiten ohne Zweifel gehorchen und die von ihnen angegebenen Lehren öffentlich aufgeben und ihre Fehler eingestehen. Es ist leicht vorstellbar, wie das Ergebnis aussehen wird und wie fruchtbar es war, in einer so schwierigen realen Situation ideologisch zu arbeiten. Infolgedessen wurde eine Situation geschaffen, die der Position der orthodoxen Kirche während der Autokratie und dem allmählichen Niedergang der lebendigen Arbeit, der Arbeit in diesem Bereich der Philosophie, des Rückzugs in sichere Wissensbereiche, der Veröffentlichung von Klassikern und Vorgängern sehr ähnlich ist. Eine neue Korruption des Denkens wurde geschaffen.



152. Es scheint mir, dass in diesen 20 Jahren neben dem Nachdruck alter Werke, die in der vorrevolutionären Zeit veröffentlicht wurden, kein einziges rein philosophisches Werk veröffentlicht wurde und es nicht einmal eine Geschichte der Schaffung des dialektischen Materialismus gibt, die auf den Primärquellen basiert. 204 Ein solcher Rückgang des philosophischen Denkens im Bereich des dialektischen Materialismus in unserem Land und offenbar Chancen für sein Aussehen ist das Ergebnis einer Art der Philosophie zu verstehen, Ziele und Abnahme Tiefe der philosophischen Arbeit, dank der Existenz des Glaubens unter unseren Philosophen, die die philosophische Wahrheit , dass keine weiteren erreicht kann sich ändern und in Frage gestellt werden.



Diese Idee ist sowohl K. Marx als auch F. Engels im Wesentlichen fremd, ganz zu schweigen von Feuerbach.



Es wurde auf russischem Boden mitten in der Auswanderung geschaffen und entwickelte sich historisch völlig unbewusst zu einem staatlichen ideologischen Phänomen, dessen Folgen für eine Reihe größerer frei denkender Kommunisten unerwartet waren.



Der Kampf der Kreise ging schließlich leise und unerwartet in die Staatsphilosophie einer siegreichen Interpretation des dialektischen Materialismus über.



In den letzten 10 Jahren hat sich dies aufgrund der Stärkung eines bestimmten Kurses immer deutlicher manifestiert.



Infolgedessen sehen wir oder haben stattdessen eine riesige Literatur vorübergehender Natur, die nach bewussten oder unbewussten Fehlern und Häresien sucht, nach Abweichungen von der offiziell anerkannten staatlichen Philosophie. Gleichzeitig veränderte sich die Staatsphilosophie selbst in sehr wichtigen Schattierungen in Anerkennung der korrekten Interpretation des dialektischen Materialismus. Ein solch trauriger Arbeitszustand in unserem Land auf dem Gebiet des dialektischen Materialismus mit enormen materiellen Möglichkeiten, der für keine der Philosophien (außer für theologisch-katholische und muslimische Philosophien im Mittelalter) beispiellos war, musste aufgrund einer Reihe von Merkmalen unweigerlich auf andere Weise geschehen In der Struktur der Staatsphilosophie in unserem Land ist einerseits der Einfluss der Auswanderung von Kreisen, deren Bedeutung bereits angedeutet wurde, und andererseits die Komplexität, die vom Leben unseres Landes unabhängig ist Umgebung geschaffen, in der dialektischen Materialismus.



153. Der dialektische Materialismus, in der Form, in der er sich realistisch in der Geschichte des Denkens manifestiert, wurde von seinen Schöpfern -- Marx, Engels und Uljanow-Lenin -- nie in kohärenter Form dargestellt. Dies waren große Denker und nicht weniger große politische Persönlichkeiten. Sie zeichnen sich durch den für politische Persönlichkeiten ungewöhnlichen Umfang ihrer wissenschaftlichen Erkenntnisse und wissenschaftlichen Interessen aus. Für ihre Zeit standen sie auf ihrem Niveau und waren gleichzeitig willensstarke Individuen, Organisatoren der Massen. Sie standen aktiv feindlich gesinnt und reagierten scharf negativ auf religiöse Aktivitäten, wobei sie sie letztendlich historisch als eine Kraft bewerteten, die den Interessen der Massen und der Freiheit der wissenschaftlichen Kreativität feindlich gegenüberstand. Gleichzeitig legten sie großen Wert auf das philosophische Denken, dessen Vorrang vor dem Wissenschaftlichen bei ihnen keinen Zweifel aufkommen ließ.



Ihre philosophische Ideologie war eng mit ihren politischen Aktivitäten verbunden und hinterließ Spuren in ihrer wissenschaftlichen Suche und ihrem wissenschaftlichen Verständnis. Dies waren in erster Linie Philosophen, Vertreter von Bestrebungen und Organisatoren der Aktionen der Massen, deren soziales Wohl -- auf realer planetarischer Basis -- der Zweck und Sinn ihres Lebens war. Wir sehen am Beispiel dieser Menschen den realen, enormen Einfluss der Persönlichkeit nicht nur auf den Verlauf der Menschheitsgeschichte, sondern auch auf die Noosphäre.



Die Grundlage der sowjetischen Staatsphilosophie wurde in die polemischen Schriften gelegt, die von ihren Autoren -- Marx, Engels, Lenin, Stalin -- niemals für einen solchen Zweck vorgesehen waren; ihre Reden zu praktischen und politischen Fragen des Lebens, in denen die Philosophie manchmal einen untergeordneten Platz einnahm. Dies waren zweitens Entwurfshefte, die aus den nach ihrem Tod hinterlassenen Manuskripten extrahiert wurden, häufig Aufsätze und Zusammenfassungen zum Lesen von Philosophen, die historisch, wissenschaftlich und kritisch nie veröffentlicht wurden. Sie wurden mit dem wissenschaftlichen Apparat und mit der Ehrfurcht gläubiger Studenten veröffentlicht und sind, wie immer unter diesen Bedingungen, voller Widersprüche, und in anderen Fällen, zum Beispiel wie in Engels 'Dialektik der Natur, kann die Zugehörigkeit aller Aussagen zu Engels nicht als bewiesen angesehen werden. Nur wenige Werke von Marx und teilweise Engels haben einen anderen Charakter, aber sie reichen völlig nicht aus, um eine solide Konstruktion einer neuen Philosophie auf ihnen zu schaffen. Die lebenswichtige Arbeit von Marx und Engels ging auf einer anderen Ebene. Marx war der größte Wissenschaftler, der in Capital seine Ergebnisse auf exakte wissenschaftliche Weise erzielte, sie jedoch in der von ihm und Engels unabhängig verarbeiteten Sprache der Hegelschen Philosophie präsentierte, die zu Lebzeiten nicht der (hauptsächlich) wissenschaftlichen Methode und wissenschaftlichen Forschung entsprach. Ein großer Verstand könnte sich diese besondere Form der Präsentation leisten.



Sogar während des Lebens von Marx -- als die letzten Bände seines „Kapitals“ veröffentlicht wurden, war diese Darstellung ein klarer Anachronismus und wird heutzutage noch größer. Im Wesentlichen ist natürlich nicht die Form der Präsentation der wissenschaftlichen Arbeit wichtig, sondern die reale Methode, mit der das Obige erhalten wird. Marx 'Präsentationsform führt den Leser in die Irre, als hätte er sie philosophisch erhalten. Tatsächlich wird es nur so ausgedrückt, aber in Wirklichkeit wurde es durch die exakte wissenschaftliche Methode des Historikers und Ökonomen-Denkers erhalten, wie es Marx in seiner wissenschaftlichen Arbeit war.



Es wurde ein perfekter Anachronismus, da es vom Bereich der politischen Ökonomie und Geschichte auf den Bereich der Naturwissenschaften und der exakten Wissenschaften übertragen wurde. Dieser Transfer, der bereits in den Werken von Marx und Engels beobachtet wird, erlangte in den Epigonen einen ganz besonderen Charakter und wurde zur Staatsphilosophie eines großen und starken Staates, der eng mit der Internationale verbunden ist.



Drittens wurde die Situation durch die Tatsache kompliziert, dass die Autoren dieser philosophischen Recherchen Menschen waren, die tatsächlich eine diktatorische Macht in beispielloser Tiefe und in beispiellosem Ausmaß besaßen, und darüber hinaus die philosophische Ideologie des dialektischen Materialismus als die anfängliche Grundlage ihrer politischen und praktischen Tätigkeit oder einer Person wie Marx und Engels betrachteten , freie Kritik in unserem Land aus dem gleichen Grund, nicht vorbehaltlich. Tatsächlich werden ihre Schlussfolgerungen als unfehlbare Dogmen anerkannt, die vom gesamten Staatsmachtapparat verteidigt werden.



Die Stagnation des philosophischen Denkens in unserem Land und sein Übergang zur kargen Scholastik und zum Talmudismus, die vor diesem Hintergrund großartig gedeihen, sind eine direkte Folge dieses Zustands.



Dieses im Wesentlichen große historische Phänomen wurde in unserem Land durch die ursprüngliche Unterordnung -- unverändert bei allen Änderungen der Staatsformen -- der Religion unter den Staat vorbereitet. Die offizielle Orthodoxie im fürstlichen und zaristischen Russland bereitete die Bühne für eine offizielle Philosophie, die sie ersetzte und ein lebendiges Erscheinungsbild der offiziellen Religion mit all ihren Konsequenzen erhielt.



154. Aber dies ist historisch und im Wesentlichen nur die inländische Seite. Die ihm zugrunde liegende Ideologie und der damit verbundene Glaube sind viel wichtiger.



Der dialektische Materialismus ist im scharfen Gegensatz zu modernen Formen der Philosophie weit entfernt von philosophischer Skepsis. Er ist überzeugt, dass er eine universelle Methode hat -- ein unfehlbares Kriterium der philosophischen und wissenschaftlichen Wahrheit. Dies spiegelte sich im Temperament der Gründer Marx und Engels wider, die dank der Einbeziehung der damals lebendigen Hegelschen Philosophie ihren wissenschaftlichen Errungenschaften eine tragfähige Glaubensform und nicht nur eine philosophische Lehre verleihen konnten, um eine politische Kraft zu schaffen, die die Massen bewegen konnte und sich im „Kommunistischen Manifest“ 48 deutlich manifestierte des Jahres -- in einem brillanten und tiefgreifenden Werk, das die Ära der Mitte des letzten Jahrhunderts widerspiegelt, als der Vorrang der Philosophie vor der Wissenschaft die europäisch-amerikanische Zivilisation ideologisch dominierte.



Im Gegensatz zu anderen Formen des Materialismus, mit denen er grundsätzlich nicht einverstanden ist, ist der dialektische Materialismus in seiner Entstehung und aufgrund seiner Urteile eng mit dem Idealismus in seiner Hegelschen Form verbunden.



Es ist alles andere als klar, ob es möglich ist, sie als frei von dem Einfluss einer solchen Geschichte zu betrachten und sie vollständig den philosophischen Strömungen des Materialismus zuzuschreiben.



Soweit ich weiß, wurde dieses Thema historisch nicht geklärt, und in seiner Identifizierung, die er in unserem Land akzeptierte, werden seine idealistischen Grundlagen stark betont, und die materialistischen sind äußerlich.



Aber dies ist ein kontroverser Bereich, weit entfernt von meinen Interessen und meinem Wissen, und ich hätte dies nicht berührt, wenn wir nicht den scharfen Unterschied zwischen den philosophischen Strömungen des Materialismus und dem dialektischen Materialismus nur in ihrem Aspekt herausgefunden hätten, der den Naturforscher am stärksten und scharfesten betrifft beeinflusst die wissenschaftliche Arbeit in unserem Land.



Die materialistische Philosophie unterschied sich -- und das war ihre Stärke -- von anderen philosophischen Strömungen der Neuzeit darin, dass sie nicht mit der Wissenschaft in Konflikt geriet und, wenn möglich, vollständig auf ihren Errungenschaften beruhte. Sie verallgemeinerte und entwickelte sie. Im Wesentlichen setzte sie die große Bewegung fort, die sich im 17.-18. Jahrhundert auf der Grundlage einer neuen Wissenschaft, einer neuen Philosophie und eines neuen Lebens und einer neuen Technologie entwickelte, die zu dieser Zeit geschaffen wurden.



Der Materialismus versuchte im Wesentlichen, eine wissenschaftliche Philosophie oder eine Wissenschaftsphilosophie zu werden. Dies gelang nicht wirklich, weil er in seinen logischen Schlussfolgerungen als Teil der Aufklärungsphilosophie des späten 18. Jahrhunderts, als er zum ersten Mal in der historischen Arena auftauchte, schnell hinter der Wissenschaft dieser Zeit zurückblieb.



Im Hinblick auf dieses Buch ist jedoch nicht der Erfolg oder Misserfolg des Materialismus in seiner historischen Offenbarung in seiner Blütezeit am Ende des 18. Jahrhunderts und in den 1860er Jahren von Bedeutung, sondern die Grundlage seiner Ideologie, die immer den Vorrang der Wissenschaft vor der Philosophie anerkannt hat. Er akzeptierte alles, was die Wissenschaft bewies, als obligatorisch für sich.



Der von Marx und Engels geschaffene dialektische Materialismus akzeptierte dies nicht, und dies unterscheidet sich stark von allen Formen des philosophischen Materialismus und unterscheidet sich von diesem Standpunkt aus nicht vom idealistischen Hegelianismus.



Damit unterscheidet er sich stark von der philosophischen Skepsis, die eine realistische Weltanschauung, wie sie wissenschaftlich offenbart ist, als einzige Möglichkeit akzeptiert und im Vergleich dazu weder religiöse noch philosophische Darstellungen als gleichwertig anerkennt. Im Gegensatz zum philosophischen Materialismus betrachtet die philosophische Skepsis die wissenschaftliche Idee der Realität angesichts des wachsenden wissenschaftlichen Wissens und der Unvollkommenheit des menschlichen Geistes nicht als ihre vollständige Repräsentation. Aber für ihn haben wissenschaftliche Errungenschaften in diesem historischen Moment und in dieser Form des menschlichen Gehirns den Charakter der genauesten Errungenschaften der Realität. Der dialektische Materialismus geht nicht von den Daten der Wissenschaft aus, ist nicht durch ihre Grenzen begrenzt, basiert nicht auf ihnen, sondern versucht, sie zu verändern und zu entwickeln, indem er sie an ihre Ideen anpasst, deren ursprüngliche Prinzipien die Gesetze der Hegelschen Dialektik sind. Es scheint mir, dass diese Dialektik so eng mit der gesamten Hegelschen Philosophie verbunden ist, dass durch sie ihm fremde geistige Konstruktionen in die geistige Umgebung des Materialismus eintreten, vom Standpunkt des Materialismus aus gesehen mystisch sind und ihn verzerren, was beispielsweise in diesem Fall die Manifestation der Dialektik in der Natur ist. in wissenschaftlicher Hinsicht in der Biosphäre.



Die Einführung der Dialektik der Natur in die philosophischen Horizonte unseres Landes, in seine offizielle Philosophie, in unserer Zeit enormen Wachstums und der Bedeutung der Wissenschaft, ist ein erstaunliches historisches Phänomen.



Es war eine Form des posthumen Einflusses der Werke von Marx und Engels, basierend auf dem Glauben -- offiziell und nicht philosophisch oder wissenschaftlich usw. ausgedrückt.



155. In unserer philosophischen Literatur wird die Wirksamkeit, dh die gleiche Bedeutung des methodischen Denkens und der Richtungen dialektischer Philosophen für die aktuelle wissenschaftliche Arbeit, scharf betont und durch die Regierung in die wissenschaftliche Arbeit eingeführt.



Dialektische Philosophen sind überzeugt, dass sie mit ihrer dialektischen Methode aktuelle wissenschaftliche Arbeiten unterstützen können.



Sie glauben an seine Bedeutung für die Wissenschaft, aber die wahre Manifestation dieses Glaubens entspricht ihm nicht.



Das scheint mir ein Missverständnis zu sein. Noch nie hat eine Philosophie in der Geschichte des Denkens eine solche Rolle gespielt und spielt keine Rolle. In der Methodik der wissenschaftlichen Arbeit kann kein Philosoph einem Wissenschaftler den Weg weisen, besonders in unserer Zeit. Er ist nicht in der Lage, die komplexen Probleme, mit denen der Naturforscher jetzt in seiner aktuellen Arbeit konfrontiert ist, genau zu erfassen. Die Methoden der wissenschaftlichen Arbeit auf dem Gebiet der experimentellen Wissenschaften und der deskriptiven Naturwissenschaften sowie die Methoden der philosophischen Arbeit, zumindest auf dem Gebiet des dialektischen Denkens, unterscheiden sich stark. Es scheint mir, dass sie in verschiedenen Denkebenen liegen, da es sich um spezifische Naturphänomene handelt, dh um empirisch ermittelte Tatsachen und empirische Verallgemeinerungen, die auf wissenschaftlichen Tatsachen beruhen. Es scheint mir, dass die Angelegenheit so klar ist, dass es keinen Grund gibt, darüber zu streiten. Unsere dialektischen Philosophen sollten nicht zu ihrem eigenen Vorteil in diesen Bereich wissenschaftlicher Erkenntnisse eingreifen. Denn hier ist ihr Versuch im Voraus zum Scheitern verurteilt. Hier kämpfen sie mit der Wissenschaft auf ihrem ursprünglichen Boden.



Die Wissenschaft erlebte in der Renaissance im 17. und 19. Jahrhundert eine ähnliche Störung des religiösen Denkens und der religiösen Konstruktionen, die grundlegend falsch war. Obwohl der Kampf hier noch nicht vorbei ist, ist es unwahrscheinlich, dass irgendjemand leugnen wird, dass der Sieg auf der Seite der Wissenschaft geblieben ist, dass die meisten religiösen Konstruktionen dieser Art in die Vergangenheit zurückgegangen sind oder im Wesentlichen wieder aufgebaut, neu interpretiert und von der Realität weg in den Bereich des persönlichen Glaubens und der Interpretation verschoben werden. Die historische Erfahrung wurde von den offiziellen Philosophen unseres Landes nicht berücksichtigt, und trotz ihrer Unkompliziertheit und unzureichenden wissenschaftlichen Kompetenz gerieten sie in einen scharfen Konflikt mit wissenschaftlichem Denken und Arbeiten, die in unserem Land -- zusammen mit dem dialektischen Materialismus -- als Grundlage des politischen Systems ideologisch hoch eingestuft sind.



Die Prekarität, den „dialektischen Materialismus“ auf eine solche Höhe zu bringen, wirkt sich unweigerlich auf seine wahre Kraft beim Staatsaufbau aus, entspricht nicht der Realität und erweist sich unweigerlich als vorübergehend.



Beginnen Kollision mit den realen Anforderungen des Lebens , die unweigerlich die gleiche Untersuchung haben muss, was stattfand .. .. höchste ... 205 in den alten christlichen Ländern.



156. In meiner wissenschaftlichen Arbeit musste ich mich oft mit dieser Situation auseinandersetzen und mich in öffentlichen Reden sogar an den Kampf meiner Vorgänger wissenschaftlicher Erkenntnisse der vergangenen Jahrhunderte erinnern.



1934 versuchten schlecht ausgebildete Philosophen, die die Planung der wissenschaftlichen Arbeit des ehemaligen Geologischen Komitees leiteten, fälschlicherweise durch dialektischen Materialismus zu beweisen, dass die Bestimmung des geologischen Alters mit radioaktiven Mitteln auf fehlerhaften Positionen beruhte -- dialektisch nicht bewiesen. Sie glaubten, dass Fakten und empirische Verallgemeinerungen, auf die sich Radiologen stützten, dialektisch unmöglich waren. Zu ihnen gesellten sich einige Geologen, die sich mit Philosophie beschäftigten und die wissenschaftliche Leitung des Komitees leiteten. Sie haben meine Arbeit um zwei Jahre verzögert, weil das Radium-Institut, an dessen Spitze ich stand, nicht mit der Arbeit der Geologen des Komitees in Kontakt treten und die Forschung auf eine solide Grundlage stellen konnte. Nach einer nachlässigen Rede auf einer öffentlichen Sitzung des Ausschusses des stellvertretenden Wissenschaftsdirektors, Professor M. M. Tetyaev, einem großen Geologen, der öffentlich auf die Unvereinbarkeit des dialektischen Materialismus mit den Schlussfolgerungen der Radiologen hinwies, war es bereits möglich, eine öffentliche Diskussion zu diesem Thema zu führen. Dies könnte geschehen, weil die gesamte radiologische Arbeit des Ausschusses durch seine Rede angegriffen wurde. Ich könnte in Qualität eingreifen und. über. Vorsitzender des von der All-Union Radiological Conference ausgewählten Ausschusses für geologische Zeit, um eine öffentliche Diskussion über dieses Thema zu erreichen. Es fand unter meinem Vorsitz in den Räumlichkeiten des Geologischen Komitees statt, und ich machte es zur Bedingung, dass wir uns als unzureichend kompetent in der dialektischen Philosophie nur mit der wissenschaftlichen Seite von Phänomenen befassen werden. Bei diesem Treffen, an dem mehrere hundert Geologen und Philosophen teilnahmen, wurde allen unwiderstehlich klar, dass alle Philosophen und viele Geologen eine auffallende Unkenntnis der grundlegenden Fakten und Errungenschaften auf dem Gebiet der Funkgeologie zeigten. Wir konnten unsere Arbeit frei entwickeln, vor allem aufgrund der Tatsache, dass die philosophischen Führer des Geologischen Komitees bald Ketzer in der offiziellen Interpretation des dialektischen Materialismus waren und aus dem Komitee entfernt wurden, aber dennoch unsere wissenschaftliche Arbeit für mehrere Jahre schädigten -- schwächten.



Das Phänomen, das hier ans Licht gekommen ist -- Fehler bei der Interpretation des dialektischen Materialismus durch offizielle Vertreter der Philosophie -- ist ein gewöhnliches und weit verbreitetes Phänomen unseres Lebens. Es gibt nur wenige Philosophen, die die von ihnen vorgebrachten philosophischen Sätze nicht aufgeben mussten und dies als unbewussten Fehler oder bewusste, verborgene Abkehr von der offiziellen Philosophie oder sogar als bewusste Zerstörung des Staates erklärten. Die Tatsache, dass dieses Phänomen weit verbreitet ist und Hunderten unserer dialektischen Philosophen gemeinsam ist, zeigt, dass es für jeden Wissenschaftler eindeutig schwierig ist, die dialektische Methode in einem modernen wissenschaftlichen Umfeld anzuwenden. Denn wie aus Kap. Nach dem historischen Verlauf der Entwicklung des dialektischen Materialismus gibt es unter seinen Gründern keinen einzigen großen Denker, der eine bis zum Ende durchdachte Philosophie vollständig interpretieren würde. Es wurde von ihnen in der Hitze des Kampfes und der Kontroversen von Fall zu Fall geschaffen.



Keiner von ihnen gab einen vollständigen Bericht ab, und solche Versuche weniger prominenter Denker erwiesen sich ausnahmslos als kurzlebig. In ihnen wurden Fehler gefunden, sie wurden aus dem Verkehr gezogen, sie konnten nicht referenziert werden. Dies ging Dutzende Male weiter, und es blieb keine einzige Ausstellung übrig, die als stabil angesehen werden konnte. Die derzeitige offizielle Darstellung sowohl des dialektischen Materialismus als auch der Geschichte der Kommunistischen Partei, deren Ideologie es ist, reicht von 1936 bis 1937 zurück, und es besteht keine Gewissheit, dass sie in ein oder zwei Jahren keine neue Überarbeitung erfordern werden.



Ich musste mich mit einer anderen Manifestation dieses wissenschaftlichen Umfelds treffen. Unerklärlicherweise waren die Kant-Laplace-Hypothese und das Erkennen der Möglichkeit der Abiogenese mit dialektischem Materialismus verbunden, und ihre Ablehnung wurde aus dialektischer Sicht als inakzeptabel angesehen. Die Präsentation stieß auf Zensurschwierigkeiten. Bereits 1936 stieß ich in meinem Bericht „Über die Probleme der Biogeochemie“ bei einem Treffen der Akademie auf Einwände dieser Art. Und im folgenden Jahr konnte ich in einer offiziellen Rede auf dem Internationalen Geologischen Kongress die moderne Unwissenschaftlichkeit der Kant-Laplace-Hypothese und ihre Unvereinbarkeit mit den Daten der Funkgeologie mit der stillschweigenden Zustimmung unserer Geologen, einschließlich derer, die sich als Dialektiker betrachten, feststellen.



In diesem Fall geht es nicht um eine solche Intervention des dialektischen Materialismus in die wissenschaftliche Arbeit eines Naturforschers, wie bereits erwähnt.



Grundsätzlich kann ein Naturforscher in einigen Fällen das Recht und die Nützlichkeit der Intervention von Philosophen in seine wissenschaftliche Arbeit nicht leugnen, wenn es um wissenschaftliche Theorien, Hypothesen, Verallgemeinerungen nicht empirischer Natur und kosmogonische Konstruktionen geht. Hier tritt der Naturforscher unweigerlich in philosophischen Boden.



In unserem Land und hier ist das wissenschaftliche Denken in einer Position, die seine ordnungsgemäße wissenschaftliche Arbeit beeinträchtigt. In diesem Fall stößt unser wissenschaftliches Denken auf ein obligatorisches philosophisches Dogma mit einer bestimmten Philosophie, die, wie wir gesehen haben, keine stabile Darstellung hat. In Ermangelung einer freien wissenschaftlichen und philosophischen Forschung in unserem Land mit einer außergewöhnlichen Zentralisierung der vorläufigen Zensur durch die staatlichen Behörden und allen Methoden zur Verbreitung wissenschaftlicher Erkenntnisse -- ob in gedruckter Form oder in Wortform -- wird dieses Dogma als für alle verbindlich anerkannt und von der gesamten Macht der Staatsmacht durchgesetzt.



Fußnoten:



14Hier und in Zukunft werde ich über Realität statt über Natur und Raum sprechen. Der Naturbegriff ist, wenn wir ihn historisch betrachten, ein komplexer Begriff. Es deckt oft nur die Biosphäre ab, und es ist bequemer, sie in diesem Sinne oder gar nicht zu verwenden (Abschnitt 6). Historisch gesehen wird dies der überwiegenden Mehrheit der Anwendungen dieses Konzepts in der Naturwissenschaft und in der Literatur entsprechen. Das Konzept des „Kosmos“ kann bequemer sein, um es nur auf den Teil der Realität anzuwenden, der von der Wissenschaft abgedeckt wird, und in diesem Fall ist eine philosophisch pluralistische Vorstellung von der Realität möglich, bei der es kein einziges Kriterium für den Kosmos gibt.



15Das Bose-Institut in Kalkutta wurde 1917 vom indischen Wissenschaftler Bose Jegdish Chandra (1858-1937) gegründet. Das Institut befasste sich mit Problemen der Physik, Biophysik, anorganischen und organischen Chemie, Biochemie, Pflanzenphysiologie, Selektion, Mikrobiologie usw. -- Ed.



16Benthische lebende Organismen kommen in der Tat in allen Tiefen der Ozeane vor, auch in solchen, die 11 km überschreiten. (Siehe: G. M. Belyaev , Fauna der Ultra-Abgründe des Weltozeans, M .: Nauka, 1966; auch bekannt als Tiefsee -Ozeangräben und ihre Fauna, M .: Nauka, 1989). -- Ed.



17Siehe zum Beispiel Lucretius Car [Über die Natur der Dinge, Prince. 2, M., 1913, S. 54].



18Für Dekameriaden siehe: V.I. Wernadskij , Zu einigen der nächsten Probleme der Funkgeologie, -- Proceedings of the Academy of Sciences, 7. Reihe von OMEN, 1935, 1, S. 1-18. [Siehe auch: V. I. Vernadsky , Selected Works, M., 1954, Band 1, S. 659].



19Auf die Entwicklung des Nervengewebes in der gesamten geologischen Geschichte der Biosphäre wurde wiederholt hingewiesen, aber meines Wissens wurde sie nicht bis zum Ende wissenschaftlich und philosophisch analysiert. Da es sich hier nicht um eine Hypothese und nicht um eine Theorie handelt, kann die Tatsache ihrer Entwicklung nicht geleugnet werden -- man kann nur einer Erklärung widersprechen. Die Anerkennung des Redi-Prinzips begrenzt die Anzahl der Erklärungen.



20Persistenten ... Siehe: V. I. Vernadsky , Chemische Struktur der Biosphäre der Erde und ihrer Umwelt, M., 1965, S. 269. -- Ed.



21Genauere stratigraphische Studien, die während des vierzigsten Jahrestages der Nachkriegszeit in verschiedenen Teilen unseres Planeten durchgeführt wurden, lassen uns die Idee der „kritischen Epochen“ in der Geschichte der Erde leicht ändern. Orogene Phänomene sowie Überschreitungen erwiesen sich auf verschiedenen Kontinenten und sogar in getrennten Teilen großer Kontinente als sehr unterschiedlich. [Siehe: Tectonics of Eurasia (M .: Nauka, 1966), A.L. Yanshin , Über die sogenannten Weltüberschreitungen und -regressionen (Bull. MOIP, 1973, 2)]. In der Geschichte der Erde kam es jedoch zweifellos zu Ausbrüchen vulkanischer Aktivitäten auf dem Territorium moderner Kontinente. Gemessen an den Massenberechnungen der von AB hergestellten Vulkanprodukte Ronov, in den letzten 600 Millionen Jahren, kamen sie im mittleren Devon am Ende des Karbon vor -- am Anfang des Perm, am Ende der Trias und weniger bedeutsam in der Mitte der Kreidezeit und im Neogen. Jeder solche Ausbruch des Vulkanismus führte zu planetarischen Veränderungen in der Zusammensetzung der Atmosphäre -- zu einer Zunahme ihres CO 2 -Gehalts und einer Abnahme des Sauerstoffgehalts, was einerseits zu einer Abnahme der Temperatur führte, die das Auftreten polarer Eiskappen erreichte, und andererseits zu einer raschen Entwicklung der Vegetation und die Rückführung von Sauerstoff in die Atmosphäre infolge von Photosyntheseprozessen. [Siehe: M.I. Budyko , Klima und Leben (M., 1974)]. Anscheinend wurden in diesen Epochen „die wichtigsten und größten Veränderungen in der Struktur lebender Materie“ geschaffen, d.h. Sie waren „kritisch“ in dem Sinne, dass V.I. Wernadskij. -- Ed.



22Zahlreiche Funde kleiner Säugetiere sind heute aus Ablagerungen verschiedener Horizonte der Ober- und Oberkreide bekannt, und die ältesten Überreste primitiver Säugetiere sind sogar in Trias-Ablagerungen zu finden. Die rasche evolutionäre Entwicklung dieser Wirbeltierklasse begann jedoch nach dem Aussterben der Dinosaurier im Paläozän, das weitgehend die Grenze der Kreide- und Paläogenperioden der Erdgeschichte bestimmt. -- Ed.



23Das Prinzip wurde von P. Curie (1859-1906) formuliert, war aber ganz klar intuitiv bewusst und wurde von L. Pasteur (1822-1895) ausgedrückt. Ich habe es hier als besonderes Prinzip herausgestellt ( L. Pasteur , Oeuvres, Vers 1, Paris, 1922; P. Curie , Oeuvres, Paris, 1908).



24Überraschenderweise blieb das Phänomen der „Rechtshändigkeit“ und der „Linkshändigkeit“ außerhalb des philosophischen und mathematischen Denkens, obwohl einige große Philosophen und Mathematiker wie Kant und Gauß sich ihm näherten. Pasteur war ein perfekter Innovator des Denkens, und es ist äußerst wichtig, dass er zu diesem Phänomen und dem Bewusstsein seiner Bedeutung kam, basierend auf Erfahrung und Beobachtung. Curie ging von den Ideen von Pasteur aus, entwickelte sie aber aus physikalischer Sicht. Zur Bedeutung dieser Ideen für das Leben siehe: V.I. Vernadsky , Biogeochemical Essays (1922-1932), M.-L., 1940; [Das meiste davon wurde in dem Buch veröffentlicht: V.I. Vernadsky , Arbeiten zur Biogeochemie und Geochemie von Böden, M., 1992, p. 22-271]; er , Probleme der Biogeochemie, vol. 1, M.-L., 1935; [ V.I. Wernadskij , Probleme der Biogeochemie. -- Proceedings of the Biogeochemical labor, t. 16, M., 1980, S. 10-54].



25Das mathematische Denken hat seit langem die gleiche Zulässigkeit der Suche nach Manifestationen nichteuklidischer Geometrien in der uns umgebenden Realität erkannt. Wahrscheinlich war der Gedanke daran Euklid selbst klar, als er das Postulat paralleler Linien von Axiomen trennte. Lobachevsky (1793-1856) versuchte auf Raumfläche, die Existenz von Dreiecken zu beweisen, die er aufgrund der Ablehnung dieses Postulats herleitete. Es scheint mir, dass A. Poincare (La science et l'hypothese, Paris, 1902, S. 3, 66) die Möglichkeit der Suche nach Manifestationen nichteuklidischer Geometrie in unserer physischen Umgebung am deutlichsten betont hat. Diese Frage ließ während der von A. Einstein verursachten Fermentation des Denkens keine Zweifel aufkommen (vgl. A. Einstein , Geometrie und Erfahrung; Erweiterte Fassung des Festvortrages, Berlin, 1921). Man kann einwenden, dass es in diesen Fällen tacito consensu (stillschweigend akzeptiert) sein sollte, dass die eine oder andere Geometrie in der gesamten Realität dieselbe ist, während es in diesem Fall um die geometrische Heterogenität des Raums in unserer Realität geht. Der Raum des Lebens ist anders als der Raum der inerten Materie. Ich sehe keinen Grund, eine solche Annahme als im Widerspruch zu den Grundlagen unseres genauen Wissens stehend anzusehen.



26Die Erde als Ganzes hat auch eine irreversible Entwicklung, wie Arbeiten mit radioaktiver Bestimmung des Alters von Gesteinen des frühen Präkambriums zeigen. Die biologische Evolution ist durch ein stark unterschiedliches Entwicklungstempo gekennzeichnet (siehe: A. L. Yanshin . Evolution geologischer Prozesse in der Erdgeschichte. L .: Nauka, 1988). -- Ed.



27Die rasche Veränderung unseres Wissens durch archäologische Ausgrabungen lässt uns auf sehr große Veränderungen in naher Zukunft hoffen.



28 C. Schuchert und CO Dunbar , Ein Lehrbuch der Geologie (New York, 1933), S. 80.



29 A.P. Pawlow , Die geologische Geschichte der europäischen Länder und des Meeres im Zusammenhang mit der Geschichte des fossilen Menschen (M.-L., 1936) S.105 und f.



30Agassiz drückte diese Idee in einer polemischen Arbeit gegen den Darwinismus aus ( L. Agassiz , Ein Essay der Klassifikation, London, 1859). Vielleicht liegt dies daran, dass sie trotz vieler wichtiger Überlegungen nicht den Einfluss erlangt hat, den sie haben könnte.



31Die Philosophie des Ostens, hauptsächlich Indiens, ist in Verbindung mit der neuen kreativen Arbeit, die in ihr unter dem Einfluss des Eintritts der westlichen Wissenschaft in die indische Kulturarbeit stattfindet, für die Biowissenschaften von viel größerem Interesse als die westliche Philosophie, die -- selbst in ihren materialistischen Teilen -- von tiefen Echos tief durchdrungen ist Jüdisch-christliche religiöse Suche.



32 J. Ortega-y-Gasset , Der Aufstand der Massen. London, 1932, p. 19, pass.



33Die Zeit des Maximums der letzten Vereisung wird nun nach den Methoden der Radiokohlenstoff-Geochronologie in 18 bis 20.000 Jahren bestimmt. Es erreichte nicht Moskau, sondern nur das Valdai-Hochland; In der Nähe von Leningrad schmolz die Eisdecke vor etwa 10-12.000 Jahren. -- Ed .



34Der Schädel aus der Piltdown-Höhle, der 1912 von Charles Dawson aus fragmentarischen Überresten zusammengesetzt wurde, wurde entweder von ihm selbst oder von einem anderen leichtfertigen Anthropologen geschmiedet. Dies ist der Schädel eines sehr modernen Mannes mit den Kiefern eines affenähnlichen Affen ( FC Howell . Früher Mann. New York. 1965, S. 24-25) -- Ed .



35Sinanthropus lebte vor 350-400.000 Jahren, d.h. im mittleren Pleistozän, etwas später als V.I. Wernadskij. Seine Annahme, dass die Gattung Homo bereits „vor mehreren Millionen Jahren“ existierte, erwies sich jedoch als richtig. Die berühmten Ausgrabungen von Dr. L. Lika in der Oldovay-Schlucht an der Grenze zwischen Kenia und Tansania, die in wissenschaftlichen und populärwissenschaftlichen Fachzeitschriften ausführlich behandelt wurden, zeigten, dass in Ostafrika ein primitiver Mann, der als besondere Art des Homo habilis (Fachmann) eingestuft wurde, zweifellos zwischen 1800 und 1900 lebte vor tausend Jahren. Die späteren Funde von R. Lika am Ostufer des Rudolph-Sees führten zu der weit verbreiteten Annahme, dass die Menschen in Ostafrika vor 3 Millionen Jahren lebten. Die letzte Zahl ist jedoch nicht zuverlässig, da fragmentarische Überreste des Schädels im Talus gefunden wurden und nicht genau bekannt sind Welche Schicht treten sie auf? Das moderne Erscheinungsbild des Homo sapiens (Homo sapiens) erschien vor 40-45.000 Jahren nicht in Afrika, sondern in ziemlich nördlichen Breiten Europas und Asiens, wahrscheinlich nicht ohne Einfluss und Anpassung an extreme Bedingungen der Eiszeit [Siehe: I.K. Ivanova . Das geologische Zeitalter eines fossilen Menschen. (M., 1965); das gleiche auf Deutsch (Stutgart, 1972)] -- Ed .



36HF Osborn . Das Zeitalter der Säugetiere in Enrope, Asien und Nordamerika. New York, 1910.



37 V.I. Wernadskij . Gedanken zur modernen Bedeutung der Wissensgeschichte. Bericht gelesen auf der ersten Sitzung der Kommission für Wissensgeschichte 14.X.1926 -- Verfahren der Kommission für Wissensgeschichte. T. 1, L., 1927, S. 6.



38Das Wohl des Staates ist das höchste Gesetz. -- Ed.



39G. Sarton . Einführung in die Wissenschaftsgeschichte. V.1, Cambridge, 1927; V.2, 1931.



40 Dies muss unweigerlich zu neuen Formen des Staatslebens führen, da nun staatliche Hindernisse für das freie wissenschaftliche Denken geschaffen wurden (§ 28), während die außerordentliche Bedeutung der Wissenschaft im Staat zunimmt.



41In meinem Einführungsvortrag an der Moskauer Universität vor 33 Jahren -- im akademischen Jahr 1902/1903, mehrfach nachgedruckt ([Vorlesung „Über die wissenschaftliche Weltanschauung“], „Fragen der Philosophie und Psychologie“, Buch 65 [V]. M., 1902, S. 1410-1465; Zusammenstellung über die Philosophie der Naturwissenschaften. M., 1906, S. 104-157; Essays and Speeches, Bd. II, S. 1922, S. 5-40), versuchte ich, die Struktur der Wissenschaft herauszufinden. Vieles müsste jetzt daran geändert werden, aber die Grundlage scheint mir richtig zu sein. Dieses Buch ist teilweise das letzte Ergebnis meiner Gedanken und Forschungen, deren erster Ausdruck meine Rede im Jahr 1902 war.



42Unbewusst in dem Sinne, dass ein wissenschaftliches Ergebnis oder ein Phänomen des Lebens, das eine wissenschaftlich wichtige oder notwendige Tatsache (oder Verallgemeinerung) schafft, dieses Ziel nicht hatte, als es geschaffen oder manifestiert wurde.



43 Ch.a. Julien . Histoire de l'Afrique du Nord. Tunesien, Maroc, Algrie. Paris, 1931. S. 178. Zur Bedeutung dieses Phänomens siehe: S. Gsell . „Memoires de l'Acad. De Inter“, 1926, 43; EFGautier . Les Sieges Obscurs du Maghzeb. Paris, 1927. S. 181.



44Wir dürfen nicht vergessen, dass die Typografie in Korea einige Jahrhunderte vor Coster und Gutenberg entdeckt wurde und im chinesischen Staat weit verbreitet war. Es gab jedoch nicht den Faktor, der ihm Vitalität verlieh: In Korea und China gab es zu dieser Zeit keine lebendige wissenschaftliche Arbeit.



45 Henri Becquerel selbst glaubte, dass er Uran nur nahm, weil dieses Element von seinem Großvater und Vater untersucht wurde (§ 55).



46Oersted entdeckte 1820 den Elektromagnetismus ( HC Oersted . Die Entdeckung des Elektromagnetismus im Jahr 1820. Kopenhagen, 1920).



47Das von Galvani entdeckte Phänomen wurde von Volta richtig erklärt. Galvanis Erklärung war falsch, aber der „Galvanismus“ mit unzähligen Konsequenzen vor der Elektrizitätslehre wurde von ihm entdeckt (mehr dazu unter: JL Alibert . Eloge Historique de Louis Galvani. Paris).



48Interessanterweise wurde die Bedeutung dieser Entdeckungen für das Leben Jahrzehnte nach dem Tod von Maxwell, Lavoisier, Faraday, Mendeleev und Ampère erkannt.



49R. Arkwright ... [Arkwright, Richard (1732-1792) -- Englischer Mechaniker, Erfinder der Seidenwickelmaschine. -- Ed. ]; Gram Zenob Theophile ... [Gramme (1826-1901) -- belgischer Elektrotechniker, einer der Erfinder des Dynamos -- Ed. ]. 



50 A. Clark . Die neue Evolution. Zoogenese. B., 1930.



51“Das denkende Schilf“ -- aus einem Gedicht von F.I. Tyutcheva -- Ed.



52Die Geschichte der geologischen Teilung in Verbindung mit ihrem Charakter entwickelte sich durch Tappen. Zum Beispiel über die Dauer der Prozesse von Vulkanausbrüchen, Verfestigung von Laccolithen usw. Schatten, dass die Menschheit eine geologische Rolle spielen könnte.



53Die durchschnittliche Dauer jeder der meisten geologischen Perioden beträgt 45-65 Millionen Jahre, d.h. 450-650 Decameriad -- Ed .



54V.I. Wernadskij . Probleme der Biogeochemie, vol. 2. Über den grundlegenden materiellen und energetischen Unterschied zwischen lebenden und inerten natürlichen Körpern der Biosphäre. M.-L., 1939, S. 34. -- Ed.



55Siehe den Artikel „Rechts und Links“ ( V. I. Vernadsky . Philosophische Gedanken eines Naturforschers. M: Nauka, 1988). -- Ed.



56Siehe: JD Dana . Crystacea. Mit Atlas der sechsundneunzig Platten, Vers 2. Philadelphia, 1855, S. 1295; Ämerican Journal of Science and Arts. „NH, 1856, S. 14.



57Gegenwärtig ist die Geschichte der Entwicklung der Hominiden viel umfassender verstanden. Das älteste (weibliche) Skelett der Gattung Homo wurde in Äthiopien in Schichten mit einem Alter von etwa 2,4 Ma gefunden. Die letzte Art dieser Gattung -- Homo sapiens, zu der wir gehören, erschien vor 45-50.000 Jahren im Spätpaläolithikum. -- Ed.



58Mandibeln von Peking Man. -- Nature, 1937, Vers 139, N 3507, S. 120-121; vgl. F. Weidenreich . Die Mandibeln von Sinanthropus Pekinensis: eine vergleichende Studie (Paleontologia Sinica, Serie D, 7. Fasc., 3, Nanking and Peping: National Geological Survey).



59 GE Smith , Menschheitsgeschichte. NY, 1929.



60Berichte N.I. Wawilow ist gezwungen, die Zeit der Schaffung der Landwirtschaft zu vertiefen. [Siehe: N.I. Vavilov . Ursprungszentren von Kulturpflanzen. L., 1926. Es wurde nun festgestellt, dass der Vorfahr des modernen Menschen, der Synanthropus, der vor etwa 400.000 Jahren lebte, die Verwendung von Feuer bereits kannte. Die Anfänge der Landwirtschaft erschienen auch vor mehr als 100.000 Jahren. -- Ed. ].



61Die Unabhängigkeit des alten indischen mathematischen Denkens vom alten Hellenischen ist sehr zweifelhaft. Man sollte jedoch nicht aus den Augen verlieren, dass die Verwendung der Null, die der hellenischen Mathematik fremd ist, in der alten hinduistischen Kulturwelt bereits im 7. Jahrhundert bekannt war. BC, vielleicht früher. Unter diesem Gesichtspunkt ist die Kenntnis der Null in Peru im 7. Jahrhundert bemerkenswert. BC Siehe: FN Ludendorff .



62O. Neugebauer , Vorlesungen über Geschichte der antiken mathematischen Wissenschaften, Erster Band, „Vorgriechische Mathematik“, Berlin, 1934. [Siehe auch: O. Neugebauer . Genaue Wissenschaft in der Antike. M., 1968.]



63Die Migrationstheorie wurde kürzlich von G.E. Smith in einer Reihe von Werken seit 1915 ( GE Smith . Die Migrationen der frühen Kultur, NY, 1915; vgl.: GE Smith , Human History, NY, 1929; siehe auch die Arbeit seines Schülers W. Perry . Kinder der Sonne. Eine Studie in der frühen Geschichte der Zivilisation (mit sechzehn Karten, London, 1923).



64Pluvial, d.h. nasse Epochen entsprachen wärmeren interglazialen Epochen. Der letzte von ihnen war während des sogenannten holozänen Klimaoptimums vor 8-10.000 Jahren. Zu dieser Zeit verbreitete sich anstelle der Sahara eine Savanne mit einer reichen Tierwelt [Siehe: A. Lot . Auf der Suche nach Tassili-Fresken. M.: Verlag Wostoch. Literatur (1962) und spätere Werke desselben Autors (1973, 1984); L.S. Berg . Ausgewählte Werke, Band II. Physische Geographie. M. (1958)] -- Ed .



65Die Natur der Bewegung im Zusammenhang mit der Bewegung des wissenschaftlichen Denkens ist gut bekannt, um die Grundlagen von R. Rolland zu verstehen (La vie de Ramakrishna. Paris, 1929; alias La vie de Vivekananda et l'Evangile universel, t. I-II. Paris, 1930; S. Radhakrishnan . Indische Philosophie, t. I-II. London, 1929-1931). Diese Bewegung ist mit tiefer religiöser Kreativität verbunden.



66Siehe die Arbeit von O. Neugebauer .



67Siehe: V.I. Wernadskij . Probleme der Biogeochemie, vol. II. M., 1939, p. 9-10. [Oder: V.I. Wernadskij . Probleme der Biogeochemie. -- Proceedings of the Biogeochemical Laboratory, T. 16, M., 1980, p. 55-84 -- Ed .].



68 V.I. Wernadskij . Das Problem der Zeit in der modernen Wissenschaft. -- Verfahren der Akademie der Wissenschaften. 7 series OMEN, 1932, 4, p. 511-541; auf Französisch Sprache: Le Probleme du temps dans la science contemporaine. Suite -- Revue generale des science pures et appliquees. Paris, v. 46, 7, p. 208-213, 10, S. 308-312. [Oder: V.I. Wernadskij . Die philosophischen Gedanken eines Naturforschers, M., 1988, p. 228-273 -- Ed .].



69Siehe Abteilung 4, Kap. 128. -- Ed.



70Dieser Name, der von Leroy und anderen verwendet wird, scheint nicht sehr erfolgreich zu sein, da sich ähnlich wie in diesem Bereich der wissenschaftlich erkennbaren nicht nur die Physik, sondern auch die Biologie oder Chemie verändert. Es ist richtig, den Namen „atomistisch“ zu halten und die Phänomene des Atomkerns zu berücksichtigen.



71 E. Rutherford . Zusammenfassende Vortäge zum Haupthema: „Radioaktivität“; Lord Rutherford von Nelson -- Cambridge; Erinnerungen an die Frühzeit der Radioaktivität. -- Zeitschrift für Elektrochemie und Angewandte Physikalische Chemie. 1932, Bd. 38, 8a, S. 476.



72Zur Geschichte der Entdeckung des Röntgenstrahls, die ohne die Entdeckung von Becquerel und seine Folgen im Wesentlichen nicht verstanden werden könnte, siehe: MV Laue . Ansprache bei der Eröffnung der Physikertagung in Würzburg. -- Physikalische Zeitschrift, Bd. 34. Leipzig, 1933, S. 889-890; O. Glasser . Wilhelm Conrad Röntgen und die Geschichte der Röntgenstrahlen. -- Berlin, 1931, S. 162. Vgl. Neue Literatur zur Politik gegen das freidenkende Röntgen: J. Stark . Zur Geschichte der Entdeckung der Röntgenstrahlen. -- Physikalishe Zeitschrift, 1935, Bd. 36; A. F. Ioffe . Wilhelm Conrad Röntgen. -- Advances in Physical Sciences, 1924, Bd. IV, Nr. 1, S. 1-18; M. Wein . Zur Geschichte der Entdeckung der Röntgenstrahlen. -- Physikalische Zeitschrift, 1935, Bd. 36, S. 536; G. Garig . Röntgenjubiläum im „dritten Reich“. -- Archiv der Wissenschafts- und Technikgeschichte. M.-L., 1936, No. VIII, S. 301-308. Prof. Prof. Goodspeed hatte Röntgenbilder früher als Röntgenstrahlen, warf jedoch nicht das Problem der Priorität auf, da er, wie viele andere vor Röntgenstrahlen, an der Öffnung vorbeikam.



73 H. Becquerel . „Comptes rendus hebdomadaires des seances de l'Acadmie des Sciences. Paris, t. 122, 1896, S. 501-503, 559-564, 688-694, 762-767, 1086-1088.



74 D. D. Thomson . Cambridge Arbeiten zur Entdeckung des Elektrons [The Corpuscular Theory of Matter. London 1907. -- Ed .]. (Siehe: Ein brillanter historischer Überblick über die Entdeckung eines Elektrons: Compton. Das Elektron, seine intellektuelle und soziale Bedeutung, -- Nature, 1937, Vers 39, Nr. 3510, S. 231). Crookes passierte das Elektron, das er beobachtete, O. Richardson war ihm nahe, aber Thomson arbeitete in einer Atmosphäre von [Ideen] der Radioaktivität.



75Es scheint mir, dass die bloße Annahme des Zufalls dieses Zufalls jetzt wissenschaftlich falsch ist. Wir haben diese Zeit bereits verlassen, als es möglich war. Es ist mit Vorstellungen über die Zufälligkeit wissenschaftlicher Entdeckungen verbunden. Aber die Wissenschaft, einschließlich der Physik, ist eine Manifestation der Organisation der Noosphäre, der Verlauf ihrer Entwicklung ist ein wissenschaftlich ausgedrückter natürlicher Prozess. Es kann keine „Chance“ geben, bis wir über den Rahmen des wissenschaftlichen Denkens hinausgehen.



76Die Geschichte der Familie Becquerel ist sehr merkwürdig. Generationen beschäftigten sich mit Phosphoreszenz, Phänomenen der Lumineszenz und Elektrifizierung. Becquerel selbst glaubte, dass die Entdeckung der Radioaktivität möglicherweise viel später erfolgen würde, wenn er die Untersuchung von Uransalzen in der erblichen Familie nicht durchführen würde. Aber praktisch angegangen. ( V. I. Vernadsky . Die Aufgabe des Tages auf dem Gebiet des Radiums. -- Bulletin der Akademie der Wissenschaften, Reihe 6, St. Petersburg, 1911, 1, S. 61-72). [Oder: Fav. Op. in 5 vols., T. 1, -- M.-L., 1954, S. 620-628; Der Beginn und die Ewigkeit des Lebens. M., 1989, S. 196-220. -- Ed .].



77 H. Becquerel . Op. cit.



78Schon in der Einführung in den Verlauf der Geschichte der Naturwissenschaften, die 1902 an der Moskauer Universität gelesen wurde, habe ich versucht, die Hauptbedeutung dieses Merkmals wissenschaftlicher Erkenntnisse hervorzuheben, das in anderen Erscheinungsformen des spirituellen Lebens der Menschheit fehlt. Im Allgemeinen bleibe ich zu diesem Thema unter dem gleichen Gesichtspunkt, den ich damals geäußert habe. [Dies bezieht sich auf die Vorlesung „Über die wissenschaftliche Weltanschauung“. -- Ed. ].



79Die Rolle von Poincare. Einsteins erste Arbeit. Siehe Über Einstein: D. Reichinstein . Albert Einstein, sein Lebensbild und seine Weltanschauung. Praga, 1935.



80Siehe das Buch „Unsere gemeinsame Zukunft. Bericht der Internationalen Kommission für Umwelt und Entwicklung“ (Moskau: Fortschritt, 1989). 1987 wurde dieses Buch in sechs Sprachen in Kopenhagen veröffentlicht. -- Ed .



81Dies wurde 1938 geschrieben. In den letzten Jahrzehnten haben Industriestaaten, einschließlich der UdSSR, große Geldbeträge für die Entwicklung der theoretischen Grundlagenforschung bereitgestellt. -- Ed.



82Der Satz ist nicht fertig. -- Ed .



83Seit dem Schreiben dieses Werkes ist mehr als ein halbes Jahrhundert vergangen. In den letzten zwanzig Jahren wurde die Logik wissenschaftlicher Erkenntnisse in den Werken des Club of Rome, in den Werken vieler Wissenschaftler, einschließlich der sowjetischen, erheblich weiterentwickelt. -- Ed .



84Worte aus dem Brief des Paulus an die Galater (Kelten), die die Gleichheit der Anhänger des frühen Christentums symbolisieren. -- Ed .



85Um Beispiele zu nennen ... [Sie wurden nicht in den Materialien für das Manuskript gefunden -- Ed .].



86Nach dem Zweiten Weltkrieg, der mit dem Einsatz von Atomwaffen endete, ist dieses Bewusstsein für die Verantwortung für die Ergebnisse ihrer wissenschaftlichen Forschung in der Welt der Wissenschaftler noch stärker geworden (die Organisation des Weltverbandes der wissenschaftlichen Arbeitnehmer und seiner Charta, die „Charta der wissenschaftlichen Arbeiter“, die Pagoš-Bewegung usw.). . Wissenschaftler auf der ganzen Welt haben gegen Kriege gekämpft, um nukleare Abrüstung und konventionelle Waffen auf ein Minimum zu reduzieren. -- Ed .



87Auf seltsame Weise hört man immer noch sehr oft, dass die Wissenschaft weder gut noch böse kennt -- sie weiß nicht, wie ihre Natur es nicht weiß. Wie noch erwähnt werden wird (§ 101), fällt die Natur in Bezug auf die Lebenden mit der Biosphäre zusammen. „Gut“ und „Böse“ sind wie alles andere auch die Schöpfung der Noosphäre. Die wissenschaftliche Moral, die in der Noosphäre stattfindet, ist möglich, deren schwacher Ausdruck die utilitaristische Moral von Brentham und seinen Anhängern ist. Entwickeln Sie am Ende des Buches.



88Konzipiert von V.I. Wernadskij hat das Kapitel „Über die Moral der Wissenschaft“ nicht geschrieben. -- Ed .



89Für Axiome siehe: A. Eisler . Wörterbuch der philosophischen Begriffe Historisch-quellenmässig bearb. -- Aufl. Hrsg. unter Mitwirkung der Kunstgesellschaft, Bd. 1. Berlin. 1927. S. 161.



90Uvarov sprach 1832 definitiv mit dem Rektor der Moskauer Universität Dvigubsky darüber. Er sprach „über die politische Religion“ mit zwei unumstrittenen Dogmen wie dem Christentum: Autokratie und Leibeigenschaft (siehe: N. Barsukov . Leben und Werk von M. N. Pogodin , Buch 4, St. Petersburg, 1891, S. 98; A. Ivanovsky . Ivan Mikhailovich Snegirev. Biografische Skizze. St. Petersburg, 1871, S. 113-115).



91 Ich unterscheide hier nicht zwischen metaphysischen und philosophischen Ideen, die sich gleichermaßen in wissenschaftlichen Konzepten widerspiegeln und mit denen gleichermaßen gerechnet werden muss.



92Vergleiche: S. Radhakrishnan . Indische Philosophie, v. II. London, 1931, p. 778.



93 Dies wurde sehr deutlich erkannt und wiederholt zum Ausdruck gebracht und oft arbeitete Goethe so wissenschaftlich [1749-1832].



94Archäologische Ausgrabungen und die Erfolge der Geschichte des Alten Ostens und Ägyptens verändern unsere Ideen. Die historische Kritik der antiken griechischen Autoren und die Vertiefung des gesamten verfügbaren Materials zwingt dazu, Skepsis abzulehnen, was von den notwendigen und nützlichen häufig zu Fehlern und Unfruchtbarkeit des Wissens in diesem Bereich führt. Die Geschichte der Technologie zeigt uns eine große Menge an wissenschaftlichen Erkenntnissen, über die wir vor 10 bis 20 Jahren nicht einmal zu sprechen wagten. Zivilisation 5-4 Tausend Jahre vor Christus es scheint uns jetzt unvergleichlich bedeutender als wir kürzlich dachten. Aber die Hauptsache ist natürlich die Entdeckung alter wissenschaftlicher Aufzeichnungen. Die Entschlüsselung der numerischen Platten der Chaldäer, die eindeutig auf ein hohes wissenschaftliches Niveau hinweist, hat eine Reihe völlig unerwarteter wissenschaftlicher Erkenntnisse in diesem Umfeld eröffnet, die uns nicht bekannt waren. In Bezug auf die Chaldäer ist es wichtig, dass im Laufe der Jahrhunderte gemeinsame Arbeiten durchgeführt wurden (zu diesem Thema siehe: R. Archibald . Babylonische Mathematik. -- Ïsis „, 1936, v. 26, S. 63-81; O. Neugebauer . Uber Vorgriehische Mathematik ( Hamburger Mathematische Einzelschriften) Hf. 8. Leipzig. 1929; alias Vorlesungenuber Geschichte der Antiken Mathematischen Wissenschaften. Erster Band. Vorgriechische Mathematik. Berlin. 1934. Zur Bedeutung der Werke von O. Neugebauer siehe: R. Archibald , Op. Cit. S. 65-66. [Siehe auch: O. Neugebauer . Genaue Wissenschaften in der Antike. M., 1968].



95Es ist möglich, dass wir in der Logik der Atomisten (Demokrit?), Die wenig Beachtung fand, den Beginn dieses neuen Verständnisses der Logik finden, das sich durch die Entwicklung der neuen Wissenschaft des 20. Jahrhunderts offenbart. Siehe für epikureische Logik ... [Also der Autor -- Ed. ].



96 W. Jaeger . Aristoteles. Grundlagen der Geschichte seiner Entwicklung. Übersetzen. mit Auther's Corr. und addit. von R. Robinson. Oxford, 1934, p. 369-370.



97Dies bezieht sich auf den Artikel „Über die Logik der Naturwissenschaften“. -- Ed .



98Dies sind die „Logiken“ von Philosophen wie Hegel, die psychologische Logik. Unnötig zu erwähnen, dass es sich um unrealistische Logiken wie die „Logik der Engel“ handelt, wenn sie es wären, Karinsky. Siehe M.I. Karinsky . Zeitschrift des Bildungsministeriums. [Link nicht gefunden -- Ed .



99Es gab eine andere Legende, die darauf hinwies, dass sich die gesamten Werke von Aristoteles in der Bibliothek in Alexandria unter Ptolemaios Philadelphus (309-246 v. Chr.) Befanden. Zum Stand der Ausgabe siehe: Über Grundriss der Geschichte der Philosophie usw. (Tl. 1. Die Philosophie des Altertums. Herausgegeben von Dr. K. Praechter). Berlin, 1926, S. 365-366. (Vgl. Kl. Usuner . Schriften, II, S. 307 ff .; III, S. 151 ff.)



100Ich stütze mich auf die Schlussfolgerungen von W. Jaeger und berücksichtige andere lebendige Ideen über diese bemerkenswerte Ära in der Geschichte des menschlichen Denkens. Vergleiche: W. Jaeger . Aristoteles. Op. cit., p. 326, 330, 334, 336, 339.



101 W. Jaeger . Op. cip., p. 369-370.



102W. Jaeger . Ebenda, p. 405.



103Siehe: V.I. Wernadskij . Biosphäre. L. 1926. [In dem Buch: V.I. Wernadskij. Lebende Materie und Biosphäre. M., Science, 1994, p. 315-401]; er ist . Probleme der Biogeochemie, Band 1. M.-L., 1935. [V.I. Wernadskij. Probleme der Biogeochemie -- Transaktionen des biogeochemischen Labors, t.16. M., 1980, S. 10-54]; er ist . Biogeochemische Aufsätze (1922-1932). M.-L., 1940. [V.I. Wernadskij. Transaktionen zur Biogeochemie und Geochemie von Böden. M., 1992, S. 22-271]; Vgl.: E. Le Roy. L'exigence idealiste et le fait de l'evolution. Paris, 1927. p. 102, 111, 155, 175.



104Siehe V.I. Wernadskij . Essays zur Geochemie. M., 1934, S. 51-64. [V.I. Wernadskij. Verfahren in der Geochemie. -- M., 1994, S. 203-236.]



105Ich werde auf diese Frage weiter unten zurückkommen. [siehe Sec. 142. -- Ed .]



106Das Wort „Noosphäre“ und das entsprechende Konzept wurde von E. Leroy erstellt. Siehe: E. Le Roy . Les origines humaines et l'evolution de l'intelligence. Paris, 1928, p. 46.



107Auf dem Gebiet der geologischen (und biologischen) Wissenschaften kann man in der wissenschaftlichen Arbeit die Ideen über die Realität außer Acht lassen, die durch die Erkenntnistheorie erzeugt werden und die jetzt beispielsweise in der Physik so berücksichtigt werden. In diesen Wissenschaften gibt es keine solchen deduktiv abgeleiteten Darstellungen aus der wissenschaftlichen Theorie, wie wir sie auf dem Gebiet vieler physikalischer Phänomene haben, die es uns ermöglichen, sie -- mit einem gewissen Nutzen -- mit philosophischen Methoden zu betrachten. Für die Physik ist dieser philosophische Ansatz jedoch im Wesentlichen von untergeordneter Bedeutung.



108Erst vor unseren Augen -- im 20. Jahrhundert -- wurden Bohrungen durchgeführt und Material aus Tiefen gewonnen, die über das Niveau des Geoids hinausgingen, das [aufgrund] der natürlichen Abweichungen dieses Niveaus nicht tatsächlich erreicht wurde. Bedeutende Vertiefungen -- in den Minen -- begannen im 17. Jahrhundert. Die Idee von Parsons (1935) -- maximales Bohren -- ist jetzt real.



109Wie die Biosphäre, die eine der Schalen der Erdkruste ist, weisen die Tiefen der Kruste auf regelmäßige konzentrische Regionen hin -- natürliche Körper. Siehe: V.I. Wernadskij . Essays zur Geochemie. M., 1934, p. 51-64; [V.I. Wernadskij. Verfahren in der Geochemie. M., 1994, p. 203-236.]



110 E. Le Roy . Les origines humaines et l'evolution de l'intelligence. 111. La noosphre et l'hominisation. Paris, 1928. S. 37-57.



111Siehe V. I. Vernadsky. Die chemische Struktur der Biosphäre der Erde und ihrer Umwelt. M., 1965; M., 1987, Kap. XXI.



112Ich werde später auf diesen Prozess zurückkommen. Hier stelle ich den Gedanken an Leroy (1928) fest: Deux grands faits, abweichende l'esquels tous les autres samblent presque svanouir, dominante dans l'histoire passe de la Terre: la vitalization de la matire, puis l'hominisation de la vie. -- Op. cit., S. 47. In der Geschichte der Erde herrschen zwei große Tatsachen vor, vor denen alle anderen fast geglättet zu sein scheinen: die Wiederbelebung der Materie und die Humanisierung des Lebens. Das erste ist hypothetisch, aber wir sehen deutlich den Beginn des zweiten.



113Diese „Struktur“ ist sehr eigenartig. Dies ist kein Mechanismus und nichts Bewegliches. Dies ist ein dynamisches, sich ständig änderndes, mobiles Gerät, das sich in jedem Moment ändert und niemals zur vorherigen Bildbalance zurückkehrt. Am nächsten ist ein lebender Organismus, der sich jedoch im physikalisch-geometrischen Zustand seines Raumes von ihm unterscheidet. Der Raum der Biosphäre ist physikalisch geometrisch heterogen. Ich halte es für zweckmäßig, diese Struktur mit einem speziellen Organisationskonzept zu definieren. Siehe 4. Vergleiche: V. I. Vernadsky. Probleme der Biogeochemie, vol. 1. Die Bedeutung der Biogeochemie für die Untersuchung der Biosphäre. L., 1934. V. I. Vernadsky. Probleme der Biogeochemie. -- Proceedings of the Biogeochemical Laboratory, T. 16, M., 1980, S. 10-54.



114Das Konzept der biogeochemischen Energie wurde von mir 1925 in einem noch nicht gedruckten Bericht an die R. Rosenthal-Stiftung in Paris eingeführt (der Fonds existiert nicht mehr). Dieser Fonds gab mir die Möglichkeit, mich zwei Jahre lang ruhig der Arbeit zu ergeben. In der Presse wurde es von mir in einer Reihe von Artikeln und Büchern gegeben: Biosphäre. L., 1926, S. 30-48; tudes biogochimiques. 1. Die Übertragung der Übertragung auf die Biosphre. -- Proceedings of the Academy of Sciences, 6 Reihen, t.20, 9, S.727-744; tudes biogochimiques. 2. Das maximale Maximum der Übertragung des Biosphre. -- Proceedings of the Academy of Sciences, 6 Series, 1927, t.21, 3-4, S.241-254; Über die Reproduktion von Organismen und ihre Bedeutung für den Mechanismus der Biosphäre. Art.1-2. -- Proceedings of the Academy of Sciences, 6. Reihe, 1926, V.20, 9, S.697-726, 12, S.1053-1060; Sur la multiplication des organismes et son rle dans la mecanisme de la biosphre, S. 1-2. -- „Revue gnrale des Sciences pure et appliques. Paris, 1926, S. 37, 23, S. 661-698; S. 700-708; Bakteriophagen und die Übertragungsrate des Lebens in der Biosphäre. -- Nature, 1927, 6, S. 433 -446.



Bericht an die R. Rosenthal-Stiftung „Lebende Materie in der Biosphäre“ siehe: V. I. Vernadsky. Lebende Materie und Biosphäre. M., 1994, S. 555-602.



115V. I. Vernadsky. Biosphere, S. 30-48. Siehe: V.I. Wernadskij. Lebende Materie und die Biosphäre, M., 1994, S. 330-341; Über die Reproduktion von Organismen im Mechanismus der Biosphäre. -- Op. cit., 9, S. 697-726; 12, S. 1053–1060. Publ. unter dem Namen „Über die Reproduktion von Organismen und ihre Bedeutung für die Struktur der Biosphäre“ in dem Buch: V. I. Vernadsky. Transaktionen zur Biogeochemie und Geochemie von Böden. (M., 1992, S. 75-101).



116V. I. Vernadsky. Chemische Elemente, ihre Klassifizierung. -- Fav. Op. t. 1. M., 1954, S. 50.



117Für Arten siehe: V.I. Wernadskij. Überlegungen zur Komposition der Komposition chimique de la matiere vivante. -- Proceedings of the Biogeochemical Laboratory, t.1, 1930, S. 5-32.



118Das völlige Fehlen eines Austauschs gegen latente Lebensformen kann noch nicht als bewiesen angesehen werden. Es ist extrem verlangsamt -- aber vielleicht gibt es hier tatsächlich in einigen Fällen keine Atommigration -- es macht sich erst in der geologischen Zeit bemerkbar.



119Siehe: V. I. Vernadsky. Biosphere, S. 37-38. In dem Buch: V.I.Vernadsky. Lebende Materie und Biosphäre. (M., 1994, S.335; aka. Etudes biogochimiques. 1. Die Übertragung der Biosphre. -- Op. Cit., 9, c. 727-744. In dem Buch Lebendige Substanz und die Biosphäre, S. 414-424; auch bekannt als Biogeochemical Essays. (1922-1932) M.-L., 1940, S. 59-83. In dem Buch V. I. Vernadsky. Arbeiten zur Bodenbiogeochemie und Geochemie. M., 1992, S. 75-101.



120VG Childe. Der Mensch macht sich. London, 1937, S. 78-79.



121GFNikolai. Die Biologie des Krieges. 1. Betrachtung eines Naturforschers den Deutschen zur Besinnung. Band 1. Zrich. 1919, S. 54.



122GF Nikolai ,, Op. cit., S. 60.



123 Wasserkraft der Welt (Nachrichten und Ansichten) -- „Nature“, 1938, v. 141, 3557, S. 31.



124Informationen zur Übertragungsgeschwindigkeit des Lebens finden Sie unten. siehe: V. I. Vernadsky. tudes biogochimiques. 1. Die Übertragung der Übertragung auf die Biosphre. -- Op. cit., S. 727-744. In dem Buch. „Lebende Materie und die Biosphäre“, S. 413-424; er ist. Biogeochemische Aufsätze, S.118-125. In dem Buch. V. I. Vernadsky. Lebende Materie und die Biosphäre, M., 1994, S. 437-444; er ist. Die chemische Struktur der Biosphäre der Erde und ihrer Umwelt, M., 1965; M., 1987, Kap. XX.



125Siehe E. Le Roy. Copyright-Hinweis nicht gefunden.



126B.P. Weinberg. Bis zum zwanzigsten Jahrtausend, dem Beginn der Arbeit zur Zerstörung der Ozeane. Essay über die Geschichte der Menschheit von ihrem Urzustand bis 2230 (Scientific Fantasy). -- Sibirische Natur. Omsk, 1922, 2, S. 21 (zu Beginn unserer Ära erlaubt es eine Bevölkerung von 80 Millionen).



127A. und E. Kulischer. Kriegs -- und Wanderzge. Wetlgeschichte als Vlkerbewegung. Berlin-Leipzig. 1932, S.135.



128A. Hettner. Die Bande der Kultur über die Erde. 2 umgearbeite und erw. Aufl., Leipzig-Berlin, 1929, S.196.



129VGChilde. Der Mensch macht sich. London, 1937, S. 56. Vgl.: IGFraser. Mythen über den Ursprung des Feuers, London, 1930.



130Siehe über die Technik von Sinanthropus und sein Feuer: B. L. Bogaevsky. Technologie primitive kommunistische Gesellschaft. -- „Geschichte der Technologie“, Bd., Teil 1. M.-L., 1936, S. 26-27. Das Feuer wurde auch von dem Pithecanthropus kontrolliert, der zu Beginn des Pleistozäns vor kaum mehr als 550.000 Jahren lebte. Vergleiche: B. L. Bogaevsky. Dekret cit., S.11, 67. Die Verwendung von Feuer für Pithecanthropus kann noch nicht als bewiesen angesehen werden, ist aber sehr wahrscheinlich.



131Nur im zwanzigsten Jahrhundert. Durch Bohren in Larderelllo auf Initiative von Le Comte erhielt eine Person überhitzten Dampf mit einer Temperatur von 140 als Energiequelle. Noch später, in Soffioni, in New Mexico, in Sonoma, wurde diese Technik weiterentwickelt. Vor seinem Tod arbeitete Parsons an einem realisierbaren Projekt mit Hilfe von Tiefbohrungen, um aus Sicht des Menschen eine unerschöpfliche Energiequelle aus der inneren Wärme der Erdkruste zu erhalten. Der Versuch, Energie aus den kalten Tiefen des Ozeans zu gewinnen, den der französische Akademiker Claude nur dank des kriminellen Rowdytums von 1936 nicht unternahm, kann als analog angesehen werden. Zweifellos haben wir praktisch unerschöpfliche Macht in den Händen des Menschen.



132Die spontane Verbrennung trockener Kräuter in der Steppe, in der Pampa, in Wäldern wird manchmal verweigert. Gegenwärtig ist die Quelle des Feuers fast immer eine Person, aber es gibt Fälle, die meines Erachtens mit Sicherheit auf die Möglichkeit eines Prozesses der Selbstentzündung in den Steppen aufgrund der direkten Einwirkung der Sonne hinweisen. Die Ursache des Phänomens wurde nicht aufgeklärt. Für solche Fälle siehe: E. Popping. Reise in Chile, Peru und auf dem Amasonenstrom whrend der Jare 1827-1832 Bd.I., Leipzig, 1835, S.398. GDHale Carpenter. Ein Naturforscher am Viktoriasee. Mit einem Bericht über Schlafkrankheit und Tse-tse-Fliege. London: 1920, S. 76-77.



133Siehe: M. A. Usov. Zusammensetzung und Tektonik von Lagerstätten in der südlichen Region des Kusnezker Kohlebeckens. Novonikolaevsk, 1924, S. 58; er ist. Untergrundbrände im Stadtteil Prokopyevsky. -- Bulletin des West Siberian Geological Exploration Trust, 1933, 4, S. 34 ff .; V. A. Obruchev. Unterirdische Brände im Kusnezker Becken -- ein geologischer Prozess. -- Nature, 1934, 3, S. 83-85. bereits I.F. Der Deutsche, der 1796 das Kohlebecken von Kusnezk entdeckte, weist auf diese Phänomene hin. Siehe: JFHermann. Beachten Sie die Charbons de Terre dans Les Environs de Kousnetzk auf Sibria. -- „Nova acta Academiae Scientiarum Imperialis Petropolitanae“. SPb., 1793, S. 376-381. Vgl. V. Jaworsky und LK Radugina. Die Erdbrnde im Kusnetzk Becken und die mit ihren Verbrundenen Erscheinungen. -- Geologische Rundschau, 1933, Hf. 5. V. I. Yavorsky und L. K. Radugina. Kohlebrände im Kusnezker Becken und verwandte Phänomene. -- Mining Journal, 1932, 10, S. 55.



134Siehe: B. L. Isachenko und N. I. Malchevskaya. Biogene Selbsterhitzung von Torfchips. -- Berichte der Akademie der Wissenschaften, 1936, Bd. IV, 8. S.364.



135Siehe: A. Brem. Tierleben. Die 4., vollständig überarbeitete und deutlich erweiterte Ausgabe von Professor Otto Zur-Strassen. Autorisierte Übersetzung herausgegeben von Professor N. M. Knipovich, Professor des Psychoneurologischen Instituts und des S.-Petersburg Women's Medical Institute, v. 7. Die Vögel. SPb., 1912, S. 15.



136Siehe: IGFraser. Op cit.



137Es scheint mir, dass die Beobachtungen von N.I. Wawilow über die Zentren für die Erzeugung von Kulturtieren und -pflanzen wird gezwungen sein, eine wesentlich längere Dauer als vor 20.000 Jahren vor Beginn der Landwirtschaft zuzugeben (siehe zum Beispiel: N. I. Wawilow. Das Problem der Herkunft von Kulturpflanzen. M.-L., 1926).



138H.Rew. Agrarstatistiken -- Encyclopaedia Britannica, 1, London, 1929, S. 388 (14,5 Millionen km 2 oder 10% des Landes ohne Antarktis)



139A. I. Maltsev. Die neuesten Erfolge bei der Untersuchung von Unkräutern in der UdSSR. -- Erfolge und Perspektiven auf dem Gebiet der Genetik und Selektion. L., 1929, S. 381.



140N.I. Vavilov, N. V. Kovalev und N. S. Pereverzev. Pflanzenproduktion im Zusammenhang mit den Problemen der Landwirtschaft der UdSSR. -- Pflanzenbau, t.1, Teil 1. L.-M., 1933, S. VI.



141L. I. Prasolov. Landfonds für die Pflanzenproduktion in der UdSSR. -- Pflanzenbau, t.1, Teil 1. L.-M., 1933, S. 31.



142 Ebenda, S. 37.



143Die Möglichkeit, die Ozeane in der einen oder anderen Form zu erobern, wurde in wissenschaftlichen Utopien mehr als einmal zu einem Zeitpunkt identifiziert, als klar war, dass die physische Bedeutungslosigkeit des Menschen vor ihrer Macht liegt. In der merkwürdigen Utopie von B.P. Weinberg (“Zum zwanzigsten Jahrestag des Beginns der Arbeiten zur Zerstörung der Ozeane. Ein Aufsatz über die Geschichte der Menschheit von ihrem primitiven Zustand bis 2230 (Wissenschaftliche Fantasie). -- Sibirische Natur. Omsk, 1922, 2, S. 30). Das Stadium der Menschheit, das kommen wird, wenn die menschliche Fortpflanzung das gesamte Land übernimmt -- das Stadium der Zerstörung des Ozeans. B. P. Weinberg räumt ein, dass dieses Thema im 21. Jahrhundert ernsthaft diskutiert wird. Bis zu einem gewissen Grad sind diese Themen zweifellos real für den menschlichen Geist. von der Vergangenheit bis Übrigens ist im Verhältnis zur Miniatur die Idee von Faust Goethe, auch Miniatur für das 18. und den Beginn des 19. Jahrhunderts, bereits ein echter Prototyp der Zukunft. Heutzutage ist die Frage nach permanenten Nichtland-Flächen unter den Meeren und Ozeanen auch nur der erste Anfang der Zukunft.



144Anscheinend ist der Beginn der Bildung der Landwirtschaft -- landwirtschaftliche Gemeinschaften -- viel älter als die Chronologie, die dem Neolithikum zugeschrieben wurde. Es geht kaum über 100.000 Jahre hinaus -- eine Dekameriade.



145F. Goodnow. China Eine Analyse. Baltimore, 1926.



146GBKressy. Chinas geografische Grundlagen; eine Übersicht über Land und Leute. New York -- London, 1934, p. 101.



147 Ebenda, S. 1-2.



148Ich habe die von G. Cressi bereitgestellten Daten zur Anbaufläche nach Provinz und Fläche der kleinbäuerlichen Landwirtschaft verwendet und mit der Fläche Chinas verglichen. Die Zahlen liegen zwischen 16,7 und 18,5 Prozent. Diese Daten beziehen sich auf 1928 und 1932. In dem statistischen Bericht von Kressi (S. 395) für das landwirtschaftliche China (mit Ausnahme der Khingan-Berge, der zentralasiatischen Steppen und Wüsten und der an Tibet angrenzenden Region) werden mehr als 477 Millionen Quadratkilometer für eine Bevölkerung von 477 Millionen -- 22 Prozent angegeben Bereich. Somit ist klar, dass sich die Bevölkerung auf ein kleines Gebiet konzentriert, das bis zum Ende genutzt wird.



149Natürlich geht es nicht um Billionen, sondern um eine viel größere Anzahl von Menschen, die auf dem Boden Chinas leben, da die Existenz von Menschen der Gattung Homo und ihres Vorgängers Sinantrop darauf als hunderttausende von Jahren etabliert angesehen werden kann. Die Entstehung einer neuen Art oder Gattung, die moderne Menschen hervorbringen könnte, könnte in einer Familie oder in einer Herde auftreten, könnte aber auch auf einem ziemlich großen Gebiet auftreten. Aber selbst im ersten Fall sollte die Anzahl der aus einem Paar geborenen Organismen in der Dauer von Hunderttausenden von Jahren (selbst wenn sie eingeführt werden) viel größer als 1010 sein, Änderungen an den gemeinsamen Vorfahren eines einzelnen unteilbaren. Siehe dazu: H.Rew. Op. cit.



150Für das alte China siehe: M. Granet. La Zivilisation Chinoise. Paris, 1926, p. 82 und Wörter



151H. Osborn. Das Zeitalter der Säugetiere in Europa, Asien und Nordamerika. NY, 1910.



152James Stevenson Hamilton. Südafrikanisches Eden; von Sabi Grame, Reserve zum Kruger National Park. London, 1937.



153M.Smith. Landwirtschaftliche Grafiken, Ernte und Viehbestand der Vereinigten Staaten und der Welt -- „Bulletin des Landwirtschaftsministeriums der Vereinigten Staaten“. Washington, 1910, 10, S. 67.



154H.Rew. Cyclncyclopedia Britannica „, t.1, 1929, S. 388.



155[46] G. Dufrenoy. Revue gnrale des science pures et appliques. Paris, 1935, 46, S. 72.



156Eine kurze Zusammenfassung wurde kürzlich von N. Nelson auf globaler Ebene gegeben: Siehe: N. Nelson. Prähistorische Archäologie; Vergangenheit, Gegenwart und Zukunft. -- „Science“, 1937, v. 85, 2195, S. 87.



157Vielleicht liegt die Wahl nur zwischen dieser Zahl -- 4236 Jahre und 2776 Jahre vor Christus. Alles, was wir jetzt angesichts des Fortschritts der Forschung in Geschichte und Archäologie wissen, zeigt, dass die erste Zahl korrekt ist. Siehe: N. Nelson. Geschichte des Kalenders. L., 1925.



158In Wirklichkeit ist dies möglicherweise die zweite obere Hülle der Erdkruste -- die vom Leben eingefangene Stratosphäre -- hauptsächlich vom Menschen (der Noosphäre), und sie sollte mit der Biosphäre gerechnet werden (siehe: V. I. Vernadsky. An den Grenzen der Biosphäre. -- Izvestiya AN, geologische Reihe, 1938, (1, S. 3-24) In dem Buch: V. I. Vernadsky. Lebende Materie und die Biosphäre. (Moskau, 1994, S. 501-517. Es muss angenommen werden, dass die darüber liegenden Kugeln 60-1000 sind km, gehören nicht zur Erdkruste, sollten aber als Teilungen des Planeten betrachtet werden, die der Erdkruste ähnlich sind, dh sie werden eine konzentrische Region des Planeten sein. Thue und die Biosphäre -- ihre obere Hülle. Dies wird offensichtlich bald enthüllt.



159Um Missverständnisse zu vermeiden, muss ich einen Vorbehalt machen, dass ich mich nicht auf theosophische Suchen beziehe, die grundsätzlich weit von der modernen Wissenschaft und der modernen Philosophie entfernt sind. Sowohl im neuen als auch im alten hinduistischen Denken gibt es philosophische Bewegungen, die in unserer modernen Wissenschaft nichts widersprechen (sie sind weniger widersprüchlich als viele philosophische Systeme des Westens), wie zum Beispiel einige Systeme, die mit Advaita Vedanta verbunden sind, oder sogar religiöse und philosophische Suchen Wie viel kenne ich sie zum Beispiel, den modernen großen religiösen Denker -- Aurobindo Ghosh (1872- [1950])



160Für die Erde kennen wir jetzt geologisch nicht genau die Ablagerungen, die sich ohne Leben bilden würden. Die ältesten Archaeen -- in ihren Sedimentgesteinen -- zeigen deutlich die Existenz des Lebens. Die Verwitterungsprozesse, denen seine Gesteine ​​ausgesetzt waren, sind die gleichen biogeochemischen Prozesse wie moderne. Azoische Schichten wurden nirgendwo genau bestimmt -- sedimentäre Metamorphosen befinden sich in den ältesten Teilen der Erdkruste. Wir müssen jedoch berücksichtigen, dass dieses Ergebnis nicht endgültig ist, da die alten archäischen Schichten noch nicht ausreichend untersucht sind. Die Schlussfolgerung ist noch nicht endgültig.



161Siehe R.Eisler Literatur für philosophische Ansichten der mediterranen und westeuropäischen Kultur. Wrterbuch der philosophischen Begriffe. Historisch -- quellenmssing Bär. Aufl. Hrsg. unter Mitwirkung der Kunstgesellschaft. Bd. I-III. Berlin 1927-1929. Diese Ideen sind in den philosophischen Systemen des indischen Denkens noch heller und lebendiger. Siehe: S. Radhakrishnan. Indische Philosophie. London, 1929-1931.



162 Siehe die Exkursion „Über die Logik der Naturwissenschaften“ (1936).



163Archäologische Ausgrabungen und die Erfolge der Geschichte des Alten Ostens und Ägyptens verändern unsere Ideen. Die historische Kritik der antiken griechischen Autoren und die Vertiefung des gesamten verfügbaren Materials zwingt dazu, Skepsis abzulehnen, die als wissenschaftlich und nützlich häufig zu Fehlern und zur Unfruchtbarkeit des Wissens auf diesem Gebiet führte. Die Geschichte der Technologie zeigt uns eine große Menge an wissenschaftlichen Erkenntnissen, über die wir vor 10 bis 20 Jahren nicht einmal zu sprechen wagten. Zivilisation 5-4 Tausend Jahre vor Christus es scheint uns jetzt unvergleichlich bedeutender als wir kürzlich dachten.



164Vergleiche: V. I. Vernadsky. Eine Seite aus der Geschichte der Bodenkunde. (In Erinnerung an V.V.Dokuchaev) -- Scientific Word, 1904, No. 6, S. 5-26. [V. I. Vernadsky. Verfahren zur Wissenschaftsgeschichte in Russland. M., 1988, S. 268-285].



165 Siehe wichtige Werke des Bodenwissenschaftlers A. I. Nabokih (zum Beispiel: Zum Thema Bodenklassifikationen. Warschau, 1900; Klassifikationsproblem in der Bodenkunde, Teil I. St. Petersburg, 1902).



166Wir müssen erkennen, dass sich die lebende Substanz der Geochemie logischerweise stark von der lebenden Substanz von Naturforschern und vielen philosophierenden Naturforschern unterscheidet. Die lebende Substanz der Geochemie ist der natürliche Körper der Biosphäre und repräsentiert die Gesamtheit ihrer natürlichen Körper einer anderen Ordnung -- lebende Organismen. Es basiert ausschließlich auf wissenschaftlichen Beobachtungen, die ziemlich genau und konkret sind.



167Ausgehend vom Artikel „Das Studium der Phänomene des Lebens und der neuen Physik“ betrachtete V. I. Vernadsky die Raumzeit nicht nur als mathematisches oder physikalisches Konzept, sondern auch als biologischen, als „einen einzigen umfassenden natürlichen Körper“. Im System der Konzepte von V. I. Vernadsky ist Raum-Zeit jedoch ein Zeichen für die Existenz lebender Körper und daher des gleichen natürlichen Körpers, der gleichen Manifestation der Natur wie jeder andere und daher kein universeller, nicht philosophischer Begriff.



168Siehe V. I. Vernadsky. Das Problem der Zeit in der modernen Wissenschaft. -- Proceedings of the Academy of Sciences, 7. Reihe, OMEN, 1932, S. 11-541; er: Le Probleme du temps dans la science contemporaine, Suite -- Revue generale des science pures et appliques. Paris, 1935, v. 48, Nr. 7, S. 208-213; 10, S. 308-313. Siehe V. I. Vernadsky. Die philosophischen Gedanken eines Naturforschers, M., 1988, S. 228-254.



169V. I. Vernadsky. Wasserhaushalt der Erdkruste und der chemischen Elemente. -- Nature, 1933, No. 8-9, S. 22-27. Ausgewählte Werke. In 5 Bänden. M., Verlag der Akademie der Wissenschaften der UdSSR, 1960, Bd. 4, Buch 2, S. 630-636.



170[Das heißt. Isotope von Wasserstoff und Sauerstoff]



171 Der Aufsatz wurde nicht geschrieben.



172L. Pasteur. Oeuvres, v. 1, Paris, 1922.



173Siehe V. I. Vernadsky. Biogeochemische Aufsätze (1922-1932). M. -L., 1940, p. 188-195. In dem Buch. : V. I. Vernadsky. Arbeiten zur Biogeochemie und Geochemie von Böden, -- M., 1992, p. 186-193. Obwohl Pasteur seiner Zeit in der Kristallographie weit voraus war, kannte er die Arbeit von Bravais gut, der viel später das wissenschaftliche Denken außerhalb Frankreichs beeinflusste.



174Auf seltsame Weise war dieses Wort hauptsächlich in der deutschen Literatur, geschrieben durch das Wort Asymmetrie. Die Asymmetrie entspricht jedoch dem Mangel an Symmetrie (in homogenen Strukturen entspricht sie der Hemieder des triklinen Systems). Diese Nomenklatur, die nach den Deutschen in unserer wissenschaftlichen Literatur verwendet wird, sollte offensichtlich abgeschafft werden, da sie Verwirrung stiftet.



175Pasteur verbindet sich nicht mit der „Richtigkeit“ und dem „Linken“ einer Person. Siehe: V. I. Vernadsky. Probleme der Biogeochemie, vol. 4. Über Gerechtigkeit und Linken. -- V. I. Vernadsky. Probleme der Biogeochemie. -- Proceedings of the Biogeochemical Laboratory, T. 16. -- M., 1980, p. 165-178.



176Siehe zum Beispiel V. I. Vernadsky. Biogeochemische Aufsätze, p. 91, 136 usw. Im Buch. : V. I. Vernadsky. Arbeiten zur Biogeochemie und Geochemie von Böden (M., 1992, S. 108, 135.



177W. Jaeger. Lectures en Symmetry und seine französischen Artikel.



178Anscheinend müssen beide Manifestationen -- „rechts“ und „links“ -- entgegen dem, was Pasteur dachte, existieren. Dies ist jedoch nicht bewiesen -- es ist zu überprüfen. W. Ludwig. Das Rechts-Links-Problem in Tierreich. Leipzig, 1932; V. I. Vernadsky. Biogeochemische Aufsätze, p. 186-193. In dem Buch. V. I. Vernadsky. Arbeiten zur Biogeochemie und Geochemie von Böden, M., 1980, p. 184-192.



179Siehe: V. I. Vernadsky. Probleme der Biogeochemie, vol. 4, p. 16. Im Buch. „Proceedings of the Biogeochemical Laboratory“, T. 16, M., 1980, S. 177].



180 Angesichts der Tatsache, dass die Raumzustände (Curie), die die Asymmetrie (d. H. Das Brechen der Symmetrie) offenbaren, unterschiedlich sein können, beispielsweise die Dissymmetrie des Magnetfelds.



181Siehe: V. I. Vernadsky. Biogeochemische Aufsätze. -- Das Studium von Lebensphänomenen und neuer Physik, p. 175-197. In dem Buch. : V. I. Vernadsky. Transaktionen zur Biogeochemie und Geochemie von Böden, M., 1992, p. 173-195.



182P. Curie. Oeuvres. Paris, 1908. Curie und in der Kristallographie vertieften sich die Gehleistungen. Einige seiner wichtigen Korrekturen am damals weit verbreiteten Verständnis der Kristallographie (1880) wurden von ihm wiederentdeckt und zum Leben erweckt, obwohl später andere Autoren gefunden wurden, deren Werke vergessen wurden.



183L. Pasteur. Recherches sur la dissimmtrie molculaire des produits organiques naturels (Leons bekennt sich zum 20. Januar 1860) -- Leons de chimie bekennt sich 1860 von MM Pasteur, Cahour, Wurts, Berthelot, Saint-Claire Deville, Barral et Dumas, Paris, 1861, p. 1-48; er ist. Oeuvres, v. 1, Paris, 1922, p. 243. V. I. Vernadsky. Biogeochemische Aufsätze, p. 188-195. In dem Buch. : V. I. Vernadsky. Transaktionen zur Biogeochemie und Geochemie von Böden. M., 1992, p. 186-193.



184 Er wurde am 19. April 1906 von einem Schrottfahrer niedergeschlagen, als er eine der Straßen von Paris überquerte.



185Diese Biographie wurde laut M. Curie hauptsächlich von seiner Tochter I. Joliot-Curie verfasst. Es spricht von Asymmetrie als Raumzustand -- eine Definition, die auch in einem Auszug aus P. Curies Tagebuch vorkommt.



186Siehe: V. I. Vernadsky. Essays zur Geochemie. M., 1934, p. 209-210. Ich habe es erstmals 1924 als Redi-Prinzip eingeführt: W. Vernadsky. La Geochimie. Paris, 1924. In dem Buch. : V. I. Vernadsky. Arbeiten zur Geochemie, M., 1994, p. 342-344.



187Siehe: V. I. Vernadsky. Das Problem der Zeit in der modernen Wissenschaft. -- Verfahren der Akademie der Wissenschaften, 7 Reihe OMEN, Nr. 4, p. 511-541 (auf Französisch: Problme du temps dans le science contemporain. Suite. -- Revue gnrale des science pures et appliques, 1935, v. 46, 7, S. 208-217, S. 308-312. .: V. I. Vernadsky, Philosophische Gedanken eines Naturforschers, M., 1988, S. 228-254.



188Das Konzept, das ich 1926-1927 eingeführt habe. Siehe: V. I. Vernadsky. Über die Reproduktion von Organismen und ihre Bedeutung für den Mechanismus der Biosphäre. -- Proceedings of the Academy of Sciences, 6 Reihe, 1926, v. 20, 9, p. 697-726; 12, p. 1053-1060; er ist. Biosphäre. L., 1926. In dem Buch. : V. I. Vernadsky. Lebende Materie und Biosphäre. S. 315-401.



189Ektropie ist ein Begriff, den der deutsche Biologe F. Auerbach geprägt hat, um sich auf ein Konzept zu beziehen, das der Entropie entgegengesetzt ist, dh Konzentration und Akkumulation freier Energie. Dem Wissenschaftler zufolge besitzen nur lebende Organismen diese Eigenschaft. „Das Leben ist die Organisation, die die Welt geschaffen hat, um gegen die Abwertung von Energie zu kämpfen.“ (F. Auerbach. Ektropismus oder die physikalische Theorie des Lebens. St. Petersburg, 1911, S. 48).



190J. Smith. Holismus und Evolution. 2 ed., L., 1927.



191EIN Whitehead. Prozess und Realität. L., 1929.



192Ich erinnere mich an die Gespräche mit dem bekannten Naturwissenschaftler I.P. Borodin, die ich nach meinen Berichten bei der Gesellschaft der Naturforscher in Leningrad geführt habe, deren Vorsitzender er war. I. P. glaubte, dass die Abiogenese wahrscheinlich dennoch stattfand, vielleicht kontinuierlich in der Welt der niederen Organismen, die für das Auge unsichtbar waren. Er konnte ein solches Verständnis der Welt nicht ablehnen. Borodin -- ein bedeutender Naturforscher, philosophisch und religiös war keineswegs ein Materialist. Für philosophische Materialisten ist die Abiogenese einer der Grundsätze ihres Glaubens.



193L. Pasteur. Oeuvres. Paris, 1922. Die Tatsache, dass Pasteur die Abiogenese zuließ und experimentell daran arbeitete, wird normalerweise übersehen.



194Stanley WM Stanley und H. Loring. Eigenschaften gereinigter Viren. Relazioni de IV Congresso Internazionale di patologia compareata. Roma, 15-20 Maggio, Roma, 1939.



195Siehe Grasses merkwürdige historische Zusammenfassung, ein Befürworter der Abiogenese. H. Grasset. Etüde historique et Kritik über les gnrations spontanes et lhtrognei. Paris, 1913.



196FC Bauden NW Pirie, ID Bernall und F. Frankuchen. Flüssigkristalline Substanzen aus virusinfizierten Pflanzen. -- „Nature“, 1936, v. 138, Nr. 3503, p. 1051-1052.



197JD Bernal und F. Frankuchen. Strukturspitzen von Protein „Cristal“ aus virusinfizierten Pflanzen. -- „Nature“, 1937, v. 197, Nr. 3526, p. 923-924.



198Über Besham (A. Bchamp) siehe: V. I. Vernadsky. Essays zur Geochemie. M., 1934, p. 329. Bereits im Jahr von Bechams Tod wurde versucht, ihn unter dem Einfluss des amerikanischen Arztes Leverson zu rehabilitieren. Siehe: ED Hume. Lebensarchitekten. (Ein Essay über die bakteriologische Arbeit von Antoine Bchamp). Nachdruck aus „The Forum“, London, 1915. Er. Bchamp oder Pasteur? Ein verlorenes Kapitel in der Geschichte der Biologie. Gegründet auf MS von MB Leverson. London, 1932,



199A. Bchamp. Les grands problms mdicaux. Paris, 1905.



200Allgemeine Bewertung von E. Hume. Op. cit.



201Vielleicht enthalten sie Metallelemente, so Beshams Migrationsanalyse. Siehe: A. Bchamp. Annales de chimie et de physiologie. [Link nicht gefunden]. Diese Analysen sollten geklärt werden.



202A. Bchamp Annales de Chimie et Physiologie. Diese Ansichten erhalten nun bekannte Leitlinien zu der ungelösten, aber aktuellen Frage der Möglichkeit, ein latentes Leben auf unbestimmte und geologisch lange Zeit aufrechtzuerhalten.



203Pasteur wiederholte die Erfahrung, die Keorhopon in den 1820er Jahren in Elton (Deutschland) beobachtete, und machte auf sich aufmerksam.



204 Im Original wird dieser Teil des Satzes vom Autor durchgestrichen.



205 Das Skript ist unhörbar.

\end{document}

