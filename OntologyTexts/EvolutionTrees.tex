\documentclass[11pt,a4paper]{article}
\usepackage{od}
\usepackage[main=english,german,russian]{babel}
\usepackage[utf8]{inputenc}
\usepackage{hyperref}
\usepackage{wrapfig}

\newenvironment{code}{\tt \begin{tabbing}
\hskip12pt\=\hskip12pt\=\hskip12pt\=\hskip12pt\=\hskip5cm\=\hskip5cm\=\kill}
{\end{tabbing}}
\def\dq{{\char34}}

\title{Modelling TRIZ System Evolution Tree Concepts}

\author{Tom Strempel, with addons by Hans-Gert Gr\"abe}

\date{Version of April 6, 2021}

\begin{document}
\maketitle

\section{Aim of the work}

The aim of this paper is to elaborate a proposal for an ontological modelling
of the areas of \emph{TRIZ System Evolution Concepts} based on the approaches
in \cite{TESE2018} and \cite{Shpakovsky2016} and further own investigations.
The work fits into the activities of the \emph{WUMM Ontology Project}
\cite{WUMM} to model core TRIZ concepts using modern semantic web means.  The
work consists of two parts -- a \emph{turtle file}, in which the semantic
modelling is performed based on the SKOS framework \cite{SKOS}, and \emph{this
  text}, in which the backgrounds and motivations of the modelling decisions
are explained in more detail. The usefulness of the modelling is demonstrated
with some examples.

\section{Starting point} 

The central concern of practical TRIZ applications is the analysis, evaluation
and transformation of systems in order to improve their operational behaviour.
As in the lecture, the term \emph{transformation} is understood in a broad
sense and also includes the planning and design of new systems as the
transformation of a system that is only available as vague conceptual
requirements into a system that operates in the real world. TRIZ provides a
whole methodological toolbox that can be used together with domain-specific
concepts for the systematic planning and implementation of such a
transformation task. In the seminar we observed that this TRIZ toolkit is
embedded in broader reasoning contexts in which engineering experience and
scientific knowledge are systematised and generalised.

One of the aspects examined in this context is the evolution of classes of
engineering systems in a historical or even technological context in order 
\begin{enumerate}
\item to extract repeating patterns of engineering procedures as «laws»
  \cite{Altshuller1979}, «laws, evolutionary lines and trends» \cite{KS} or
  just «engineering trends» \cite{TESE2018} or
\item to identify evolutionary connections in the unfolding of the history of
  technology \cite{Shpakovsky2016}.
\end{enumerate}

Exploring this aspect, the focus on the exact form of the transformation of a
single system described above was left and, in the style of \emph{distant
  reading}, a variety of information about historical transformations in
different classes of systems has to be analysed in order to extract
transformation patterns from it.  If, for example, the «development of
display» \cite[p. 22]{TESE2018}, \cite[ch. 5]{Shpakovsky2016} is analysed,
this is based on a much stronger abstraction of the system concept compared to
the system concept of classical TRIZ modelling, even if this more
comprehensive abstraction is only rarely explicated in the relevant works --
for example as a \emph{class of systems} in the narrower sense. In the rest of
this paper, the concept of system is used in the same vague generality of an
intuitive understanding as an externally given (metaphysical) concept as in
the referenced works, without attempting to go into more details.

The central to TRIZ understanding, that engineering achievements can be
conceptualised as system transformations, leads in the analysis of historical
technology development to the structure of a directed graph with the
prototypical link
\begin{center}\tt
  OldSystem \textrm{---}\fbox{isTransformedInto}$\to$ NewSystem
\end{center}
In the first approach, this graph is considered as a set of such links to be
classified. The graph structure plays a subordinate role, because even in the
concept of \emph{development line} a rather linear progression is postulated
(e.g. \cite[Figure 4.104]{KS}, but see \cite[4.8.4 and Figure 4.72]{KS}). In
the second approach \cite{Shpakovsky2016}, the graph structure is considered
more consistently, but also with the aim to classify the links in more detail.

The aim of these conceptualisations is on the one hand to develop the
methodology of \emph{evolutionary potential analysis} \cite[4.8.7]{KS} and on
the other hand to consolidate and improve the central TRIZ tools such as the
40 application standards («principles») or the 76 inventive standards applied
in SuField modelling.

\section{The Conceptualisations}

The conceptualisations to be developed follow the basic assumptions and
settings that are elaborated in more detail in \cite{Graebe2021}. In
particular, the following namespace prefixes are used:
\begin{itemize}[noitemsep]
\item \texttt{ex:} -- the namespace of a special system to be modelled. 
\item \texttt{tc:} -- the namespace of the TRIZ concepts (RDF subjects).
\item \texttt{od:} -- the namespace of WUMM's own concepts (RDF predicates,
  general concepts). 
\end{itemize}
Furthermore the \texttt{SKOS} ontology is used to model basic properties of
concepts.

Our central task is to model the links in concrete evolution trees. The
full evolution tree as an edge-marked graph then can be conceptualised as a
set of such links in the usual way.  A link in such a concrete evolution graph
has the typical shape
\begin{center}\tt
  ex:TVWithLargePixels ex:decreasePixelSize ex:TVWithMediumPixels .
\end{center}
where the transformation predicate \texttt{ex:decreasePixelSize} is assigned
to certain evolution patterns (even several).
\begin{code}\tt
ex:decreasePixelSize \\
\> a rdf:Property, skos:Concept ; \\
\> od:usesPattern tc:SegmentationPattern ; \\
\> skos:prefLabel {\dq}Decrease pixel size{\dq}@en ; \\
\> skos:definition {\dq}{\dq}{\dq}Decrease pixel size by segmentation \\
\>\> of one big pixel in several smaller ones{\dq}{\dq}{\dq}@en .
\end{code}

\section{Concepts}

In this section the concepts developed by N. Shpakovsky in
\cite{Shpakovsky2016} are introduced and discussed.

\subsection{Structured information field}

There are three types of problems in modern engineering: the solution of
urgent technical problems, the forecast of the evolution of technical system
and patent protection or circumvention. While the importance of the first two
types are very easy to understand for non-engineers, the third type requires
some explanation. If a technical system is patented one must either pay the
owner of the patent or develop a competing system which is not covered by the
patent, both options are money and labor intensive. If such measures are not
taken a company risks to be sued out of contracts.

An example for that would be the ongoing case between Heckler \& Koch (HK) and
C. G. Haenel over the production of the new standard issue weapon of the
German army. The case is centered around the over-the-beach capability of the
weapon, which means that the weapon can still be fired after being submerged
in water for a short time. If the water is not removed the weapon can jam or
missfire. HK patented a solution for this problem in
\href{https://patents.google.com/patent/EP2018508B1/en}{EP2018508B1} by adding
a fluid passage to the recoil spring mechanism. HK argues that the system used
in the Haenel Mk556 is a violation of it's patent, while Haenel says it is
distinct from this patent. HK managed to kick Haenel out of the procurement
process with this argumentation. Haenel thus lost a big order because of a
(perceived) patent violation. This example illustrates very well why patent
protection and circumvention is an essential part of modern engineering.

Conventional solution methods such as trial and error or brainstorming are not
suited for the requirements of these three problems. A higher level system,
the \textit{science of invention}, must be created to adequately tackle these
problems.

In order to develop a systematic approach to the inventive content of
technical developments, it is necessary to classify concrete technical
artefacts according to various aspects. Corresponding classification goals
determine both the \emph{purposes} of the respective systematisation and the
required \emph{degree of abstraction} and thus contextualise it.  Within such
a \emph{context} a systematisation must obey the following five requirements:
\begin{enumerate}[noitemsep]
\item Objective classification criteria (objectiveness)
\item The presence of all significantly different versions (fullness)
\item Suitable degree of generalisation and specificity
\item Visualization (to find gaps for patent circumvention)
\item Sufficient description or prediction of not yet existing versions
  (informativity)
\end{enumerate}

In the following sections the concept of the evolution pattern and tree are
described which will fulfill these requirements.

\subsection{Evolution patterns and trees}
% scope is object, not the technical system

N. Shpakovsky \cite{Shpakovsky2016} distinguishes 10 essentially different
basic evolution patterns of technical systems:
\begin{enumerate}[noitemsep]
\item Mono-Bi-Poly
\item Trimming
\item Expanding-trimming
\item Segmentation
\item Geometrical evolution
\item Object structure evolution
\item Evolution of surface properties
\item Dynamization
\item Increasing the controllability
\item Increasing the coordination of the elements
\end{enumerate}

From these ten basic evolution patterns, more specific evolution patterns can
be derived. The evolution patterns 1--4 are patterns that provide resources
for other evolution patterns. For example, there is no possibility for
dynamization on an unsegmented monolith. The structure of the object is given
by patterns 5--7. Patterns for dynamization, controllability, and coordination
are inserted at points that seem reasonable. This hierarchical structure of
transformations is shown in Fig. \ref{fig:basic_evo}. It is not required to
follow an evolution pattern to its end before applying a different one. The
direction of evolution is also not strictly given.

\begin{figure*}[htb]
  \centering
  \includegraphics[width=\linewidth]{figures/basictree.png}
  \caption{\small Basic evolution tree \cite{Shpakovsky2016}}
  \label{fig:basic_evo}
\end{figure*}

A basic principle in TRIZ is the interaction of a tool and an (work) object:

\begin{center}\tt
  Tool \textrm{---}\fbox{interactsWith}$\to$ Object
\end{center}

Analogous to this a transformation is the application of an evolution
pattern to a system, which subsequently becomes a transformed version of the
system. Equivalents of these Patterns can be found in the TRIZ principles
e.g. the Segmentation Pattern corresponds to the TRIZ \textit{Principle of
  decomposition or segmentation}.

An evolution tree is build out of multiple patterns which are combined at
certain locations (see Fig. \ref{fig:basic_evo}). The positions of these
locations are specific for each specialized evolution tree and are only shown
in approximation here.

The Evolution Tree is a self-similar concept, e.g. an object is approximately
similar to itself. An Evolution tree can thus contain another evolution tree,
e.g. the evolution tree of the screen contains the evolution tree of a plasma
screen, which could be analysed further.

\subsection{Not laws but recommendations}

Shpakovsky never calls his concepts of the evolution pattern and tree
\emph{laws} but uses the terms \emph{requirements}, \emph{rules} and, in
context with construction instructions, \emph{recommendations}. Thereby he
weakens himself the objectivity of his concepts. A really explicit explanation
of this change of the notion from law to recommendation is not given, but the
circumstance can be understood on the basis of the created evolution tree of
the screen.

The trunk of an evolution tree, for example, should consist of only one
evolution pattern (cf. \cite[p. 122f]{Shpakovsky2016}), but it becomes clear
that in the case of the screen two evolution patterns serve as the trunk,
namely trimming and segmentation. Here it is appropriate, due to the nature of
the object, not to follow the recommendation. This would not be possible with
a law or it should not occur at all due to the nature of a law.

\subsection{External influence}

The transition between some steps of evolution patterns require external
development, e.g. the transition from a changeable image (flip-book cinema) to
the cinematographer was a joint product of many inventors. External
involvement is required for adding and evolving objects e.g. in the
mono-bi-poly, segmentation and expanding-trimming pattern.

In a discussion with N. Shpakovsky it was clarified that this external
influence can be seen as taking the same and new components from a super
system, which is outside the scope of the special evolution model under
consideration. Components are selected on the basis of their benefit in
increasing productivity and other quality parameters. External influence is
not modelled in the RDF part.

\subsection{Construction of evolution trees}
% 2 Sachen: Welche Sachen nehme ich rein, und in welchem detaillierungsgrad

There are two questions to answer before building an evolution tree: Which
systems should it include as nodes and at which level of detail are they to be
included?  Context and a subsequent demarcation between the inside and outside
of the system is required. But the determination of the context is implicit or
very weak by using the elementary function. Objectivity, rules and laws are
only applicable where the context can be adequately defined, which is not
entirely the case for the concept of evolution trees.

After an email correspondence with N. Shpakovsky the following conclusion for
creating a evolution tree was finally formulated:

After defining the elementary function, that should at best be dividable to the
\textit{subject-action-object} level, the simplest transformation of the
object is used as the starting point. Because the evolution tree is a self
similar concept the number of possible transitions and transformation versions
gets too big to keep track off. The inevitable limiting of this explosion of
versions follows a purely voluntary approach, as such objectivity can't be
guaranteed. The creation of the evolution tree in the given example does not
follow neither historical development nor a timeline. The sequence of nodes is
subject only to technological trends. That is, there is no goal of building a
tree as such, or rather, the goal is to construct an \emph{information field}
in which the past, present and possible variants of the system under analysis
are located.

\subsection{Determination of unknown versions}

\begin{figure}[htb]
  \centering
  \includegraphics[width=.9\linewidth]{figures/removabledisplay.png}
  \caption{\small Section of the specific evolution tree of the screen
    \cite{Shpakovsky2016}, see \url{http://www.target-invention.com/} for the
    complete tree}
	\label{fig:spec_evo}
\end{figure}

For the analysis of an object, both the basic tree of evolution pattern (see
Fig. \ref{fig:basic_evo}) and the specific evolution tree (see
Fig. \ref{fig:spec_evo}) must be considered.  By comparing the two trees, gaps
as well as unfinished evolution patterns can be discovered. The highest level
of the pattern of dynamization consists of a complete decoupling of the
individual components. For a laptop, this would mean separating the screen and
the peripherals. At around 2002, the time when this evolution tree of the
screen was created, such a version did not yet exist. Hence, recognizing this
gap, a useful new version could be predicted. Nowadays, complete dynamization
is achieved by integrating the computing core into the screen and connecting
the peripherals via Bluetooth. Thus it was demonstgrated that evolution trees
are able to predicte future developments.

\subsection{Patent circumvention}

There is often the problem that a patent already exists for a technology to be
applied solving problem.  In this situation it is either required to pay high
license fees to the patent owner or to circumvent the patent.

The legal method of patent circumvention, which consists of using loopholes
and erroneous patent descriptions to invalidate a patent, is not always
applicable.

It is alternatively possible to modify the object under investigation to
develop a better product. This inventive method has the disadvantage of having
large development costs and having to change the basic design. On the other
hand, it is not possible to obtain an alternative patent without modification.

From this conflict, typical for TRIZ, a synthesis emerges in the form of the
legal-inventive method. This new method aims at finding transformation
versions not yet covered by patents through evolution trees.

The search for existing patents can be additionally facilitated by using the
object and transformation names as keywords.  

But as the example of the legal dispute between HK and Haenel shows, this is
not necessarily a perfect protection. Even a legal dispute that is still open
can preclude the conclusion of the desired contract, because the contractor
does not want to take the risk or is not allowed to do so due to legal
regulations.

\section{Modelling}

Ontologies are based on the «modelling of models». This is done over several
levels, where the lower levels are used to model real-world examples and thus
have a high level of specificity. Higher levels are used to model concepts and
even more general concepts, see \cite{Graebe2021}. In this work three levels
are used for modelling which results in three RDF namespaces:

\begin{itemize}[noitemsep]
\item \texttt{ex:} -- Level 1 (Real world examples and patterns)
\item \texttt{tc:} -- Level 2 (Subjects and concepts)
\item \texttt{od:} -- Level 3 (Predicates) 
\end{itemize}

\subsection{Modelling the evolution tree concepts}

The file \textit{EvolutionTree.ttl} contains the description of the concepts
used in the RDF modelling of the special evolution trees presented in
\cite{Shpakovsky2016}. It also contains all modelled basic evolution patterns
and thus the basic evolution tree.

All RDF subjects or nodes in the corresponding graph of evolution tree
concepts have URI's in the namespace \texttt{tc:}. There was no need to
introduce new predicates in the (WUMM internal) namespace \texttt{od:} because
the already existing ones covered every need. Every chapter and subsection in
the table of contests is modelled with at least one subject or triple,
e.g. the segmentation pattern has its own RDF subject
\texttt{tc:SegmentationPattern} and is described in it.

\begin{code}\tt
tc:SegmentationPattern \\
\> od:subConceptOf tc:BasicEvolutionPattern ; \\
\> od:hasSubConcept tc:Monolith, tc:TwoParts, tc:ManyParts, tc:Granules, \\
\> tc:Powder, tc:Paste, tc:Liquid, tc:Foam, tc:Fog, tc:Gas, tc:Plasma, \\
\> tc:Field, tc:Vacuum, tc:IdealObject ; \\
\> a skos:Concept, od:AdditionalConcept ; \\
\> skos:prefLabel {\dq}Segmenting objects and substances{\dq}@en ; \\
\> skos:example {\dq}Segmentation of an aircraft propulsion unit{\dq}@en ; \\
\> skos:broader tc:Segmentation, tc:TrendofTransitiontoMicrolevel .
\end{code}
\begin{code}\tt
tc:Liquid \\
\> od:subConceptOf tc:SegmentationPattern ; \\
\> a skos:Concept, od:AdditionalConcept ; \\
\> skos:prefLabel {\dq}Liquid{\dq}@en . \\
\end{code}

Relations or edges between subjects are modelled by the predicates
\texttt{od:subConceptOf} and \texttt{od:hasSubConcept}, e.g.
\texttt{tc:SegmentationPattern} is a \texttt{tc:BasicEvolutionPattern} and
thus they are linked together using the preedicate \texttt{od:subConceptOf}.
Different transformation versions like \texttt{tc:Monolith} are also
referenced that way.

Some refinements of a generic evolution pattern as \texttt{tc:FlatSurface} to
\texttt{tc:CylindricalSurface} from \texttt{tc:GeometricalEvolutionPattern}
are not related as subconcepts since the direction of evolution can go in both
directions in specific examples. Some modern monitors use curved displays
instead of flat ones. CRT displays have a cylindrical surface due to
constraints in manufacturing. By using better glass it is possible to get a
CRT display with a flat surface. With that a conflict would arise if the
evolution is only possible in one direction. Shpakovsky also introduces the
MATChEM operator from the wider TRIZ as an extra pattern that is not linked.

As TRIZ trends are used as the basic evolution patterns they must be
referenced from the modelled patterns, e.g. \texttt{tc:Segmentation} and
\texttt{tc:TrendofTransitiontoMicrolevel} are referenced by
\texttt{tc:SegmentationPattern} via \texttt{skos:broader}. Evolution pattern
are more specific than their corresponding TRIZ principles because they are
seen in the context of the evolution tree and thereby \texttt{skos:broader} is
used instead of \texttt{skos:narrower}.  

\begin{code}\tt
tc:Segmentation \\
\> od:hasRecommendation tc:Segmentation\_1, tc:Segmentation\_2,\\\>\>
tc:Segmentation\_3 ; \\ 
\> od:hasAltshuller73Id {\dq}01{\dq} ; \\
\> od:hasAltshuller84Id {\dq}01{\dq} ; \\
\> a od:Principle ; \\
\> rdfs:label {\dq}Principle of decomposition or segmentation{\dq}@en .
\end{code}

Triples usually consist of the RDF subject (TRIZ concept), the referenced
subjects via predicates, the predicate \texttt{a} (short for
\texttt{rdf:type}), and further SKOS labels, examples and definitions. All
information about one subject is modelled inside a single triple. The file
also contains inline comments marked with \texttt{\#} where the currently
modelled part is marked or further described to keep track.

Concepts for the application of the evolution trees are also modelled via
subconcepts and the SKOS ontology: 
\begin{code}\tt
tc:FrontalSearch \\
\> od:subConceptOf tc:InformationFieldSearch ; \\
\> a skos:Concept, od:AdditionalConcept ; \\
\> skos:prefLabel {\dq}Frontal search of an Information Field{\dq}@en ; \\
\> skos:definition {\dq}{\dq}{\dq}Search starts from random points. \\
\>\> Search of the whole information field for relevant information.{\dq}{\dq}{\dq}@en .
\end{code}

\subsection{Modelling the evolution tree of the screen}

N. Shpakovsky modelled the evolution tree of the screen based on the
elementary function \textit{To visualize information}. We see a display in
this context as an \textit{artificially created object specially designed for
  the role of a tool in the realization of the elementary function}
\cite{Shpakovsky2016}. This differentiates it from a sheet of paper with
information written on it. The terms display and screen are used
interchangeably in this work. The main axis of development runs along the
trimming transition from the cinematographer, the trimmed cinematographer, CRT
TV set to the flat display. Further transitions are special applications of
the segmentation pattern. The evolution tree trunk is marked by using
\texttt{od:usesPattern tc:EvolutionTreeTrunk} in the transitions. \texttt{ex:}
is used as the namespace of subjects and predicates for the model because it
describes an real world example. As the granularity of this specific evolution
tree is very fine some predicates or transitions can be reused several times
for transforming subjects.

First the screen will be defined as an specific evolution tree:
\begin{code}\tt
ex:Screen \\
\> a tc:SpecificEvolutionTree ; \\
\>\> skos:prefLabel {\dq}Specific evolution tree of the screen{\dq}@en ;
\end{code}

\begin{figure}[htb]
  \centering
  \includegraphics[width=.9\linewidth]{figures/audio.png}
  \caption{\small Pattern of adding audio \cite{Shpakovsky2016}}
  \label{fig:audio}
\end{figure}

We use the marked transformations for adding sound to the screen (see
Fig. \ref{fig:audio}) as an example for the structure of the modelling done in
\textit{ScreenExample.ttl}.

\begin{code}\tt
ex:Cinematograph \\
\> a ex:Screen ; \\
\> ex:transitionsTo ex:ImageOnly, ex:FlatScreen, ex:SmoothScreen,\\\>\>
ex:ImmovableScreen ;\\ 
\> ex:trimCinemaBuilding ex:MechanicalTVSet ;\\
\> skos:prefLabel {\dq}Cinematograph{\dq}@en .
\end{code}

We choose the cinematographer from the evolution tree trunk as the starting
point of our example. Fig. \ref{fig:audio} shows that different objects that
are part of the cinematographer can be branched out and be described. No
transformation needs to take place because we only look at an already existing
object. For this purpose the \texttt{ex:transitionsTo} predicate is used as it
describes a transition without changing the object. This leads to
\texttt{ex:ImageOnly} to which we add sound via the \texttt{ex:addSound}
transition. The \texttt{tc:MonoBiPolyPattern} and \texttt{tc:BiSystem} are
used as the types of the transitions because components are added to build a
higher level system. Where applicable the more special evolution pattern
(e.g. \texttt{tc:BiSystem}) is used, if not only the basis evolution pattern
(e.g. \texttt{tc:MonoBiPolyPattern}) is used.

\begin{code}\tt
\# Image expansion\\[4pt]
ex:ImageOnly\\
\> a ex:Screen ;\\
\> ex:addSound ex:ImageSound ;\\
\> skos:prefLabel {\dq}Image only{\dq}@en .\\[4pt]
ex:ImageSound \\
\> a ex:Screen ; \\
\> ex:addSmell ex:ImageSoundSmell ;\\
\> ex:transitionsTo ex:OneLoudspeaker ;\\
\> skos:prefLabel {\dq}Image and sound{\dq}@en .\\[4pt]
\# ...\\[4pt]
ex:addSound\\
\> a rdf:Property, skos:Concept ;\\
\> od:usesPattern tc:MonoBiPolyPattern, tc:BiSystem ;\\
\> skos:prefLabel {\dq}Add sound{\dq}@en .
\end{code}

Now we are in the mono-bi-poly pattern of accompanying sound in which the
predicate \texttt{ex:addLoudspeaker} is used to repeatedly add new
loudspeakers to the system to evolve it into a poly-system. This new
poly-system is of higher complexity due to higher coordination between the
loudspeakers for example by the Dolby Surround 7.1 specification. Using the
same transition repeatedly is not always possible. 

\begin{code}
\# Mono-Bi-Poly of accompanying sound\\[4pt]
ex:OneLoudspeaker \\
\> a ex:Screen ;\\
\> ex:addLoudspeaker ex:StereophonicSystem ;\\
\> skos:prefLabel {\dq}One loudspeaker{\dq}@en .\\[4pt]
ex:StereophonicSystem \\
\> a ex:Screen ; \\
\> ex:addLoudspeaker ex:QuadrophonicSystem ;\\
\> skos:prefLabel {\dq}Stereophonic system {\dq}@en .\\[4pt]
ex:QuadrophonicSystem \\
\> a ex:Screen ; \\
\> ex:addLoudspeaker ex:SpatialSoundSystem ;\\
\> skos:prefLabel {\dq}Quadrophonic system{\dq}@en .\\[4pt]
ex:SpatialSoundSystem\\
\> a ex:Screen ; \\
\> skos:prefLabel {\dq}Spatial sound system{\dq}@en .\\[4pt]
ex:addLoudspeaker\\
\> a rdf:Property, skos:Concept ;\\
\> od:usesPattern tc:MonoBiPolyPattern, tc:BiSystem, tc:PolySystem ;\\
\> skos:prefLabel {\dq}Add loudspeaker{\dq}@en ;\\
\> skos:definition {\dq}Add one or more loudspeakers{\dq}@en .
\end{code}

\begin{wrapfigure}[26]{l}{0.4\textwidth}
  \begin{center}\vspace*{-1em}
    \includegraphics[width=0.3\textwidth]{figures/segmentation.png}
  \end{center}
  \caption{\small Segmentation of the screen \cite{Shpakovsky2016}}
  \label{fig:segmentation}
\end{wrapfigure}

With that we successfully modelled a branch of the evolution tree of the
screen, the same approach to modelling is used to completely model the tree.
One particularity must be explained further namely the further segmentation of
the screen shown in Fig.~\ref{fig:segmentation}. Shpakovsky uses the generic
evolution transformations here even if it is a specific evolution tree. This
was modelled by using extra subjects (\texttt{ex:ManyParts}, \texttt{ex:Sand}
etc.) and linking them with the \texttt{ex:segmentation} predicate. Branches
are transitioned to with \texttt{tc:transitionsTo} as was explained before.

% a tc:SpecificEvolutionTree

\subsection{Modelling the ship propulsion evolution tree}

Souchkov in \cite{KS} describes the evolution tree using the example of the
boat. The terms boat and ship are used interchangeably here even if a ship is
assumed to have some other characteristics as a boat, e.g. being ocean-going
and having a higher displacement. 

This is done via the standard TRIZ methodology and can be implemented in
Shpakovsky's more specific concept of an evolution tree (see
\textit{BoatExample.ttl}). A boat as a technical system has undergone a very
wide array of transformations. Thus it is vital to specify the elementary
function by looking at the main axis of the evolution, the tree trunk.

Souchkov splits the transformations into three categories: New transformations
for delivering the main function, existing transformations that could be
developed further and completed or discontinued transformations. We are
interested in the new transformations for delivering the main function as this
is used as the main axis of development. Developments follows through the
transformation line to tree trunk, rowboat, sailboat, steamboat, diesel-boat,
water-jetboat and atom-boat. Hence the corresponding elementary function is
\textit{provide the boat with a power source} as they all, with one exception,
describe what the engine or power source is. A tree trunk has no power source,
a rowboat uses muscle power, a sailboat the wind, a steamboat a steam machine
and so on. As the water-jet is a means of propulsion and does not describe the
power source, but how the power is used for propulsion (e.g. propeller, paddle
wheel), it is misplaced on the evolution tree trunk.

Modelling this discrepancy is done via using a different evolution pattern for
the transition between the diesel-boat and water-jet-boat, namely the
expanding-trimming pattern. Consequently the transition is not part of the
modelled evolution tree, while all other transitions between the main
transformations are. 

The granularity of this tree is very coarse. The transitions between the main
versions e.g. from sailboat to steamboat are given in very big steps in the
overall technical development. First ships with sails were developed in the
second millennium b.c. in the South China Sea, while the first practical
steamers were build in the early 19th century. This massive time frame shows
that the tree is very coarsely build up (at least concerning the time scale).
Because of that transitions our predicates are very specific as they must
depict large developments. Finer grained evolution trees have the advantage of
reusing predicates, this is not applicable here. Again we are running into the
self similarity of the evolution tree as one tree node could be easily
expanded, e.g. the torpedo boat node could be expanded to include all
development on torpedo boats. Furthermore hybrid vessels, e.g. boats with
stream and sail propulsion, are also omitted. The terms boat and ship are used
interchangeably in the evolution tree and this approach was further adopted.
Predicates can possibly only be applied for this example and thus not be
easily used for other objects. The branching transformations can also be seen
as evolution trees themselves as they describe extended elementary functions
like \textit{transporting tourists via boat}.

There are two possibilities of modelling the evolution tree trunk, on one hand
the segmentation pattern can be used, on the other the MATChEM operator, both
must describe the power source of the ship. If the segmentation pattern is
used the manual labour of rowing a boat with one or multiple rowers could be
described with the monolithic, two-part and many-part transformations. Because
a tree trunk has no power source of propulsion on board it would be omitted
from the evolution tree trunk. Sails use wind and thus gas for propulsion, so
no problems would arise here. Diesel engines burn a diesel air mix and
directly drive the propulsion device via a transmission, it would be thus
modelled as a liquid or aerosol. This is not the case for a steam engine or
nuclear reactor, they use boilers or reactors to boil water into steam and
then drive a steam engine or turbine. This would mean that three types of
power sources would be modelled with gas, which is not ideal. If we would
model the fuel instead other problems would arise because a boiler can burn
coal (e.g. a many-parts transformation), oil (e.g. the liquid transformation)
or an coal-oil mixture. Furthermore would a sailboat be on a higher place in
the segmentation pattern as the diesel engine. While it could be argued that
the ideality of using wind is higher, because no fuel needs to be burned, the
cost of executing it's function is still high as a lot of complex rigging,
sails and high manpower is required.

\begin{figure*}[htb]
  \centering
  \includegraphics[width=\linewidth]{figures/boat.png}
  \caption{\small Souchkov's evolution tree of the boat translated from German
    \cite{KS}}
	\label{fig:boat}
\begin{center}
  \begin{tabular}{r@{: }l}
    \textbf{Bold font} & new transformations for delivering the main function\\
    Normal font & transformations that can be developed further\\
    \textit{Italic font} & discontinued transformations
  \end{tabular}
\end{center}
\end{figure*}
\begin{code}\tt
ex:Boat \\
\> a tc:SpecificEvolutionTree ; \\
\> skos:prefLabel {\dq}Boat{\dq}@en, {\dq}Boot{\dq}@de ; \\
\> skos:altLabel {\dq}Power source of the boat{\dq}, \\
\>\> {\dq}Energiequelle bzw. Motor des Boots{\dq} ; \\
\> skos:definition {\dq}Specific evolution tree of the boat power source{\dq}@en . \\
\end{code}

Due to these problems a modified MATChEM operator will be used. This operator
is used for remodelling the technical system around different fields so that
their working principle is entirely different but their function remains the
same. For our fields we are using mechanical, air, thermal, chemical,
electrical and atomic fields. These model the evolution steps better and in
the right progression. The electric motor is omitted in the model but was used
with U-boats together with a diesel engine.

\begin{code}
ex:BoatMATChEMOperator\\
\> a tc:SpecificEvolutionPattern, tc:MATChEMOperator ;\\
\> skos:prefLabel {\dq}Use a boat specific MATChEM operator{\dq}@en ;\\
\> skos:definition {\dq}{\dq}{\dq}Mechanical - Muscle power\\
\> Air - Wind power via sail\\
\> Thermal - Steam engine\\
\> Chemical - Diesel engine\\
\> Electrical - Electric motor\\
\> Atomic - Nuclear fission reactor{\dq}{\dq}{\dq}@en .\\[4pt]
ex:Rowboat \\
\> a ex:Boat ; \\ 
\> ex:addPaddles ex:BoatWithManyRowers ; \\
\> ex:addRecreationalInstallations ex:RecreationalRowboat ; \\
\> ex:replacePaddleWithSail ex:Sailboat ; \\
\> skos:prefLabel {\dq}Rowboat{\dq}@en ; \\
\> skos:definition {\dq}Manual labour as the power source{\dq}@en . \\[4pt]
ex:replacePaddleWithSail \\
\> a rdf:Property, skos:Concept ; \\
\> od:usesPattern tc:SegmentationPattern, tc:Gas ; \\
\> skos:prefLabel {\dq}Replace Paddle with sail{\dq}@en ; \\
\> skos:definition {\dq}{\dq}{\dq}Rudder and thus muscle power is replaced by sails \\
\>\> and thus wind power{\dq}{\dq}{\dq}@en .
\end{code}

The transformations that can be developed further can be described as
incomplete evolution patterns. The expanding-trimming pattern is used to
describe these branching transformations because if one for example converts a
ship for military use adding special components (weapons, armour etc.)  while
other components (excess weight, cargo bays etc.) are removed or trimmed.

\subsection{Evolution tree of the aircraft propulsion device}

\begin{figure*}[htb]
  \centering
  \includegraphics[width=0.75\linewidth]{figures/aircraft.png}
  \caption{\small Shpakovsky's evolution tree of the aircraft propulsion
    device \cite{Shpakovsky2003}}
	\label{fig:aircraft}
\end{figure*}

As a further example N. Shpakovsky discusses the evolution of the aircraft
propulsion device. Airplanes don't include sailplanes as it is stated that an
airplane must use an engine. This engine drives the propulsion device, in its
earliest form a propeller, which in turn moves the aircraft forward. The
elementary function of the object \textit{aircraft propulsion device} is
defined as \textit{A propulsive device is what an aircraft uses to push off
  the surrounding space while flying.} \cite{Shpakovsky2003}.

The starting point of the evolution is the single-bladed propeller, which
corresponds to the monolith in the segmentation pattern. Subsequently the two
parts transformation led to a double bladed propellers. The Evolution Tree
Trunk consists here of the entire tree as the object develops along the
segmentation pattern.

For granules, liquid and vacuum exists no current transformation. These gaps
can be used to theorise about potential inventions, cost and feasibility must
also be considered to get a good solution. While for example water can be
sprayed into the propeller area to increase the thrust, this can only be done
for short periods as water is expensive to carry all the time. Throwing away
rocket stages for further acceleration of a space craft are also proposed for
the granules transformation, but fuel tanks bear little resemblance to
granules. The granules, liquid and gas transformations are thus left empty on
the tree and are thus not modelled. 

The same approach as described before is used to model the evolution tree. An
example for adding an extra propeller row is provided below, the full version
is contained in the file \textit{AircraftPropulsionDeviceExample.ttl}.

\begin{code}\tt
ex:AircraftPropulsionDevice \\
\> a tc:SpecificEvolutionTree ; \\
\> skos:prefLabel {\dq}Propulsion device for aircraft{\dq}@en ; \\
\> skos:definition {\dq}{\dq}{\dq}A propulsive device is what an aircraft \\
\>\> uses to push off the surrounding space while flying{\dq}{\dq}{\dq}@en
.\\[4pt] 
ex:MultiBladePropeller \\
\> a ex:AircraftPropulsionDevice ; \\
\> ex:addPropellerRow ex:DoubleRowPropeller ; \\
\> skos:prefLabel {\dq}Multiblade propeller{\dq}@en .\\[4pt]
ex:addPropellerRow \\
\> a rdf:Property, skos:Concept ; \\
\> od:usesPattern tc:SegmentationPattern, tc:ManyParts ; \\
\> skos:prefLabel {\dq}Add propeller row{\dq}@en .
\end{code}

\begin{thebibliography}{xxx}
\raggedright
\bibitem{Altshuller1979} Genrich Altshuller (1979).  Creativity as an exact
  science (in Russian). English version: Gordon and Breach, New York 1988.
\bibitem{Graebe2021} Hans-Gert Gr\"abe (2021). About the WUMM modelling
  concepts of a TRIZ ontology.
  \url{https://wumm-project.github.io/Texts/WOP-Basics.pdf}.
\bibitem{KS} Karl Koltze, Valeri Souchkov (2017).  Systematische
  Innovationsmethoden (in German).  Hanser, Munich. ISBN 978-3-446-45127-8.
\bibitem{TESE2018} Alex Lyubomirsky, Simon Litvin, Sergei Ikovenko et al.
  (2018). Trends of Engineering System Evolution (TESE).  TRIZ Consulting
  Group. ISBN 9783000598463.
\bibitem{Shpakovsky2016} Nikolay Shpakovsky (2016). Tree of Technology
  Evolution. English translation of the Russian original (Forum, Moscow
  2010).\\ \url{https://wumm-project.github.io/TTS.html}
\bibitem{Shpakovsky2003} One of the evolution trends of an aircraft propulsive device. \url{http://www.gnrtr.com/Generator.html?pi=211&cp=3}
\bibitem{SKOS} SKOS -- The Simple Knowledge Organization System.
  \url{https://www.w3.org/TR/skos-reference/}.  
\bibitem{WUMM} The WUMM Project. \url{https://wumm-project.github.io/} 
\end{thebibliography}

\end{document}
