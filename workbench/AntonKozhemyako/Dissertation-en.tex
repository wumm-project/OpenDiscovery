\documentclass[11pt,a4paper]{article}
\usepackage{od}
\usepackage[utf8]{inputenc}
\usepackage[english,russian]{babel}

\title{Features of TRIZ applications for solving organizational and management
  problems: schematization of an inventive situation and working with models
  of contradictions}

\author{Anton Kozhemyako, Chelyabinsk}
\date{tba.} 

\begin{document}
\maketitle


\section{Referee Report by Ilya Ilyin}

\subsection*{Overview}
The dissertation consists of two sections. The first section is devoted to
schematization which is applied at the stage of analyzing an inventive
situation in the process of solving organizational and management problems.
The second section is devoted to the choice of the operational zone and
resource allocation of the operational zone for resolving organizational
contradictions. The dissertation contains case studies explaining application
of these tools and provides recommendations when these tools can be applied.

\subsection*{Significance}
There is a significant interest to apply TRIZ tools, originally developed to
solve problems of improving technical systems, to solving organizational and
managerial problems.  Indeed, there is a notion that recommendations for
reformulating and solving technical problems in the engineering systems are
defined in such a way that their application to organizational problems can be
useful. For example, coordination, evolution, completeness of system, and any
elements of system analysis. This is a natural assumption considering that one
can argue that some of the tools were originated in the business analysis and
were adopted by TRIZ community for engineering analysis.  Examples would
include S-curve analysis, which usually describe company’s lifecycles in terms
of revenue or cash flow, structural analysis of the industry, SWAT analysis,
etc.  Therefore, the task of combining system analysis and TRIZ solving tools
for solving business problems is significant.

Analytical instruments developed to transition weakly defined business problem
in to the set of specific technical requirements and contradictions became
standard elements of the technical consulting practices. The similar
structural approach would be beneficial in the analysis of business
organization.

In addition, when trying to apply the most powerful TRIZ concept of
contradiction to business situation there is a need for guiding principles to
define contradictory requirements, operational zone and time, as well as
available resources. It is more difficult in the relatively abstract and
qualitative organizational situation than in the course of analysis of the
engineering system. The set of such guiding principles would be very
beneficial for practical application of TRIZ tools in the solving of
organizational contradictions.

\subsection*{Scientific Novelty}
In general, the work contains elements of novelty, since it considers
principles of practical application of tools developed by the author to the
new set of organizational problems which loosely defined in the beginning.

However, the enthusiasm is diminished since there is a plurality of tools and
methodologies developed for analysis of business situation in the management
consulting. Some of them were borrowed by the TRIZ methodologists and adapted
for solving technical problems. The question would be why not to take a step
back and to apply those original tools to the business situation in a way that
would lead to better definition of possible organizational contradictions.
Such combination of known business analysis tools and suggested TRIZ
analytical instruments could derive a truly universal project roadmap.

The particular novelty lays in the author’s definition of layers and places
for the elements of the organizational system and their use in the
schematization methodology.  Schematization was enhanced by connection with
MPVs for various stakeholders. Also novel is a broader definition of the
operational zone including a tool, a product, and their environment as a part
of the operational zone. This way all components of the operational zone can
be later explored to extract specific resources for further problem solving.

\subsection*{Approach}
The author presented the thorough analysis of existing methods and
developments in the modern TRIZ practice. The description of the developed
tools contains necessary details for their practical application. The
recommendations are supported by real-life case studies. These case studies
significantly contribute to the credibility of the work and bring it to the
standards of research in the area of business analysis.

The approach would benefit from review of methods of the modern organizational
strategy particularly in the view of some of the presented examples. The
potential sources might include Michael Porter’s Competitive Strategy or
corresponding publications from Harvard Business Essentials. The reviewer’s
specific request would be to compile and present the typical roadmap of the
analytical project, where developed tools would be allocated to the
corresponding stages of the analysis.

\subsection*{Conclusions}
In conclusion, the presented work of Anton Kozhemyako demonstrates the
sufficient level of candidate’s qualification, is based on the significant
amount of theoretical research and practical implementation, satisfies main
requirements for MATRIZ TRIZ Master and is recommended for certification.

\section{Referee Report by Mikhail Semenovich Rubin}

\subsection*{General characteristics, goals and objectives of the study}
The work of A.P. Kozhemyako is devoted to the topic of using TRIZ in solving
problems in organizational and management tasks. The objectives of the study
were to propose a way of formalizing business tasks using schematization,
develop areas of transition from schematization to TRIZ tools and create a
method for determining the operational zone in organizational and managerial
tasks.  To achieve these goals, the SMD approaches, terms and tools were used
and methodologies were developed following the school of G.P. Shchedrovitsky.

\subsection*{Relevance of the work}
The topic of using TRIZ tools and methods in non-technical areas is
sustainably a trend in the development of TRIZ since the 1980s and has become
very relevant at present in connection with the increasing practice of using
TRIZ in non-technical areas: TRIZ in business, TRIZ in IT and others.  The
author identifies organized social systems and organizational and management
tasks, considering this a wider class of systems than, for example, business
systems.

Quite rightly, the key features of the use of TRIZ in non-technical and, in
particular, in social systems to resolve conflict situations are the
importance of identifying and resolving contradictions, the formulation and
analysis of the operational zone of conflicts. The work pays considerable
attention to methods for identifying the operational zone for conflicts in
management tasks.

\subsection*{Scientific novelty of the work}
The author identifies two main aspects as novelty of his work:
\begin{itemize}
\item “a method has been developed for applying schematization to the
  preparation of organizational and managerial tasks for the further use of
  TRIZ mechanisms as an indispensable condition for the analysis of inventive
  situation in the field of organizational and managerial tasks”.
\item a method has been developed for determining the operational zone in
  organizational and managerial tasks.
\end{itemize}
In particular, the author notes, that “schematization supports a unique
mechanism for determining management layers, as well as concepts of place and
material, which provides new opportunities for statement of private tasks in
solving organizational and managerial problems, with the possibility of
scaling the resulting solutions.“

Other aspects of the novelty of the work are related to the concept of the
operational zone in managerial tasks and methods for its determination and
analysis. In the work, in particular, it is noted that novelty “lies in the
fact that using the method developed by the author, the solver can not only
define the operational zone as the physical contour of space {\ldots} but also
define the operational zone directly from the wording of a technical
contradiction, and subsequently allocate resources of the operational zone and
use them to solve the task, using the IFR operator”, although this description
of the operational zone for non-technical systems was known even before the
work the author.

The main feature of the method to describe the operational zone for management
systems, which is presented by the author, is the use of a new parametric
approach in the description of the operational zone. This approach is well
suited to the description of the operational zone in complex social systems
with objects distributed in time, space and other characteristics.

\subsection*{The practical value of the results}
The main value of the presented work can be seen precisely in the author’s
great experience of practical application of well-known TRIZ tools for solving
inventive problems in areas of business systems and business intelligence.
Convincing application examples are provided.  TRIZ tools for analyzing
business systems are supplied, highlighting technical and physical
contradictions (contradictions of requirements and contradictions of
properties), conflicting pairs, operational zone, IFR, analysis and
application of resources.

\subsection*{Disadvantages of the work}

\paragraph{1.}
In the presented work, the question of the object of study remained blurred
and poorly described.  Most often, the author uses the concept of
“organizational management tasks” (to avoid tautology, the opponent considers
it more correct to use the term management tasks). Management tasks are
planning tasks, the formation of the structure, motivation, control and other
management activities that are in the area of management, not TRIZ. It would
be more correct to talk only about inventive management tasks. But then the
features of such inventive tasks are exclusively for management systems, and
not, for example, business systems, social, or socio-technical systems in
general.  Moreover, attempts to artificially separate management systems from
the object of management (for example, from business in business systems, from
production technologies in enterprises, from plants in the ministry and etc.)
dramatically reduces the ability to analyze contradictions and search for
resources to resolve them. The author gives examples exclusively from the
field of business systems and not infrequently moves from management systems
to business systems.

\paragraph{2.}
The use of the SMD methodology, which is proposed by the author in the work,
has a number of disadvantages. It, according to G.P. Shchedrovitsky and other
authors, is poorly formalized.  “In the SMD methodology, all 'ordinary' words
are occupied by their unusual meanings, therefore people who first encountered
methodologists have an absolutely false understanding of what is being said.
So, the words 'thinking', 'activity', 'science', 'methodology', ... have in
the speeches of the SMD-methodologists a very, very different meaning than
what the speakers mean” (\url{http://praxos.ru}). The same applies to the
terms “layer”, “place”, “material”, which the author offers to use in TRIZ. In
fact, it means that a layer is not a part of substances, place is not a part
of space, and material is not a substance. Even in the text itself ambiguity
arises, when the author uses these terms in his usual sense, for example,
“workplace”, “bottlenecks”, “learned material”, etc.

\paragraph{3.}
Schematization has no fundamental advantage over already known methods of
analysis in TRIZ. The author claims that “despite a certain similarity with
the functional a model is a slightly different tool.” You have to agree with
that, but there are no benefits for it in analysis of a system or a problem
situation. The author claims, for example, that “the main purpose of
schematization is the separation of the system of tasks from the primary
inventive situations.“ But the methods of analysis adopted in TRIZ cope with
this not worse, but are formalized better: tools for structural-component and
functional analysis, su-field analysis, causal and RCA+ and many other systems
analysis methods that allow to develop models of systems and to identify
contradictions in them.

\paragraph{4.}
An example is the analysis of the problem in Appendix 5 (Fig. 12). Instead of
schematization that is applied by the author in this problem, we can
distinguish the system components: Big contract, Customers and Customer
Groups, Old sales system, New sales system, Business processes, CRM system,
sales managers, ROP(?) (manager). Based on this list of components, you can
build a table of relationships between the components, a system of functions
and determine NE at the junction of these relationships. There will be no less
tasks than in the diagram in fig. 12. We can add wild bonds to this scheme,
and then its completeness will be higher. Schematization and the concept of
“layer” is not required to highlight the contradictions, since for
contradictions, the conflict of requirements for components (objects) is
important, regardless of the hierarchy of these objects.

\paragraph{5.}
The author rightly asserts that the concepts of “layer” in terms of the SMD
methodology and the system operator are different concepts, but at the same
time the author draws for some reason a parallel between these concepts,
compares them with each other. This is similar, for example, to a comparison
of the system operator with a functional model or with a morphological box. At
the same time it admits erroneous statements. For example, the author claims
that clients are a supersystem for sellers. If we consider that the system is
a subset of the supersystem, then you can see that there is no subset of
“sellers” among customers. This error leads to a number of erroneous
statements, in particular, about the benefits of analyzing “layers” over a
system operator.

\paragraph{6.}
The author suggests to use as a basis for the analysis of management systems
schematization with the allocation of hierarchical “layers”, explaining this
with the basic properties of organizational and management systems based on
the hierarchy of management. Between the well-known self-organizing systems
and systems with network management, in which there is no clearly defined
management hierarchy. Hierarchy is only one of the possible and not the most
effective ways to manage your organization.

\paragraph{7.}
The use of the terms technical and physical contradiction is not correct for
business systems. It is more correct to use the terms of the contradiction of
requirements and properties. Using the author of the display scheme TC
(technical contradiction) in the form of a rhombus has significant
disadvantages associated with leveling the difference between TC and FC
(physical contradiction).

\paragraph{8.}
The technique described by the author for the formulation of the operational
zone on the basis of a conflicting pair is not new for TRIZ and is described
in TRIZ works, for example, in ARIZ-71-B. This approach and methodology for
non-technical and intangible systems is used in ARIZ-U-2010. Quote from the
famous 2004 publications: “The nature of the interaction field in business
systems predetermines the nature of the space or zone of conflict. This is not
a physical space, as it usual happens in technical systems, but rather a
multidimensional space-set consisting of conflicting elements and relations
between them.“ This is consistent with the methodology of OZ descriptions for
organizational and management systems in the author's work.

\paragraph{9.}
A method for describing the operational zone for management systems using a
new parametric approach is effective and relevant for the development of TRIZ
as a theory. It can productively be used in the formulation and analysis of
the operational zone of problems in complex social systems with objects with
distributed parameters. However, I would like to see more examples applying
this approach.

\subsection*{Conclusions}
The work of Kozhemyako Anton Petrovich presented for defense is the result of
a large practical work of the author in the field of TRIZ application in
analysis of business systems, identifying inventive tasks, analysis and
solving contradictions in them. The topic of work is relevant, the author’s
activity has practical results. A parametric approach to highlighting and the
analysis of the operational zone in management systems is a new and useful
contribution to the development of TRIZ.  In view of the correction of the
above remarks, I recommend the Dissertation Council MA TRIZ to consider the
proposed work for the assignment of the title TRIZ Master.

\end{document}
