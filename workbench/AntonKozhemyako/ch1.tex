\section{Introduction}
This work belongs to the field of Theory of Inventive Problem Solving.

The work consists of 2 sections and Applications.

The first section is devoted to schematization used in the analysis phase of
inventive situations\footnote{An \emph{inventive situation} is a situation
  characterized by the presence of necessity to satisfy the demand of a
  specific supersystem without a clearly defined set of problems for the
  further solution or direction of solving the problem. [42]} solving
organizational and management problems. It is shown that most problems in
organized social systems are posed by the task manager in a form that is not
enough informative for its processing using TRIZ tools.  This is typical for
tasks also in other areas, for example: “increase productivity lines by 5\%."
Such tasks always have high uncertainty, since resources to achieve the goal
in these tasks are drawn from soft systems such as people and their
interaction.

The most convenient way of initial presentation of organizational management
tasks for further processing, according to the author, is the schematization
approach used by the followers of G.P. Shchedrovitsky, which was initially
developed by the Moscow methodological circle specifically for analysis of
this class of tasks. The section describes in detail how TRIZ tools “are
joined” with a preliminary analysis of the problem using schematization.

The second section is devoted to the choice of the operational
zone\footnote{The \emph{operational zone} is a part of the physical space
  where a conflict or an undesirable effect is located that generates an
  inventive situation [42]. It is worth noting that the area in which the
  conflict in solving organizational and managerial tasks is located, is not
  necessarily defined as a physical point in space -- remark by the author.}
for organizational management tasks and the allocation of resources of the
operational zone. This is also said by M.S.  Rubin in his works: “The nature
of the interaction field in business systems determines the other nature of
the space or zone of conflict. This is not physical space, as is usually the
case in technical systems, but rather multidimensional space-set, consisting
of conflicting elements and relationships between them” [44].

The paper indicates the scope of this tool and explains why this approach is
preferable precisely for this class of problems.

In the appendix practical application of the described methods are shown.

\subsection{Relevance of the research topic}
Since the 90s, the issue of application of the methodology has been actively
discussed in the TRIZ environment for solving problems in social systems ([1],
[2], [3], [4], [5], [36], [37] and many others sources). A number of TRIZ
specialists successfully use its tools in projects in relation to business
systems and other organized social systems, for example, to the evolution of
creative teams [37], to other organized social systems whose purpose is not to
make a profit (state structures management, military units, police, judicial
system, healthcare, etc.). For that TRIZ specialists have accumulated many
successful cases in solving problems in social systems, which indicates the
intensive development of TRIZ in this direction (in a number of works,
business systems are referred not to social [5], but to information systems
[one]). Some examples of applying TRIZ to business tasks are also available on
the website.  author [6].

Since any artificially created (organized) system is social the system tends
to entropy [23], [38], in such systems an obligatory function is management.
When implementing the control function in a dynamic environment changing
supersystems, modern managers face many challenges, which the author calls
organizational and managerial [11]. The term “organizational management task”
will be described in more detail below.

Used on most of these problems, as a rule, does not cause difficulties in
managers and is solved by analogy, since most situations with which faced
manager in everyday practice, are typical [10]. However in conditions of
modern rapidly changing economic and social reality the manager faces many
inventive situations [39], which are difficult resolve in the usual way [11].
It should be noted that the use of TRIZ for solutions to organizational and
managerial tasks are still not structured, up to initial provisions. Take at
least an emphasis in the definition of business systems: a number of authors
(NN Khomenko [37], V.A. Korolev [5], Shmakov B.V. [8], etc.) call such systems
social, and, for example, E.A. Sosnin and B.N. Poyzner [1] -- informational).
In a number works like systems belong to some indefinite set -- so called
“non-technical systems“ (for example, [7]) or business systems that are
already somewhat more accurately [2]. Although, of course, business systems
are important, but still subsystems of a larger system -- organized social
systems, to which refer, as mentioned above, not only systems aimed at
extracting arrived. Organizational and managerial tasks can arise in any of
the listed types of organized social systems, as in any of these systems there
is a management function, during the implementation of which inventive
situation.

Many attempts have been made to transfer TRIZ tools designed for solving
problems of transforming technical systems into organizational management
tasks, some of which have taken root well, and some are a product direct, but
ineffective transfer from one sphere to another, therefore the use of such
tools is doubtful, for example, the “direct translation“ of the matrix
G.S. Altshuller into the language of business systems [8] (it should be noted
here that along with attempts to directly but ineffectively transfer the
techniques of technical resolution contradictions there are also deeply
developed versions for solving problems in the field of business, for example,
the matrix of D. Mann, author of Hands-on Systematic Innovation for Business
and Management“ [9]), however, techniques can only be applied after
formulating a contradiction, which is hardly possible at the beginning of work
on organizational and management task.

In general, the problem of studying the inventive situation before applying
TRIZ tools is a separate large-scale task. Largely application TRIZ for
solving such problems is “lame” precisely because of a lack of reliable
inventive situation analysis tool.

In addition, there is an assumption that the term “business system“ [2]
includes partly (but does not absorb) both social systems and information
ones.  However, this classification does not outline the sharp edges of such
systems and it is difficult to understand where the social system ends and the
information begins. Since the systems described by the above concepts are very
intertwined and the boundaries between them blurred, the author does not
consider it appropriate to highlight separately in the business system social,
separately informational subsystems. It’s enough to understand that business
systems are part of a larger system: organized social systems.  The author
believes that it is easier to deal with the \textbf{concept of organizational
  and managerial tasks}, this term indicates that \textbf{the task is set in
  any organized social system by an entity with the goal of producing a
  certain improvement in the interaction of elements of the business system
  located in informational and social relationships} [40].

Why does the author call this type of task organizational and managerial?  It
is known that organizational tasks are associated with optimizing the
allocation of resources with the point of view of getting the most out of
them. In what follows, it will be shown that organizational tasks can be set
at the level of both content and level generalized objects in a business
system (definitions of a generalized object and content are given below).
\textbf{Organizational tasks} are related to the organization of relations
between generalized objects in a business system and filling in generalized
objects of business systems taking into account its properties.
\textbf{Management tasks} are tasks related to \textbf{increasing the
  efficiency} of business system elements located in certain relationships
with each other.  Since the subject of the task usually it is required to
increase the efficiency of a business system or some of its subsystem, more
often in all, he resorts to both organizational changes and managerial
impacts. In the literature, these types of effects do not often differ (for
example, [23]), however, these concepts are sharply divorced in the works of
G.P. Schedrovitsky [10], therefore, the author suggests focusing precisely on
this classification and talking about organizational-management tasks in case
it is required to increase the effectiveness of any an organized social system
(in particular, a business system) or some of its subsystems. \textbf{In the
  following, the author uses the term “organizational managerial tasks”
  precisely in the context of tasks related to promotion the effectiveness of
  an organized social system or its subsystems.}

It is worth noting that the author did not meet generally accepted
terminology, it is clear describing similar systems, with the exception of the
generally accepted position that design an organized social system is designed
for the goals of the customer (short-term, medium or long term). Moreover,
terms describing the structure organized social systems, in addition to the
generally accepted classification describing hierarchy of internal structure:
organization, TOP-management, departments, departments and the design teams
[26], [28], [32] are not known to the author.

Thus, the urgent question is the elaboration of an inventive situation when
statement of organizational and managerial tasks with a view to the
possibility of further analysis provided that most solvers get such tasks in
insufficient formalized form, therefore, a preliminary analysis method is
required similar tasks. It should be noted that with the problem of
formalization of organizational management tasks faced not only specialists in
the field of TRIZ. This members of the Moscow Methodological Circle (MMK) were
actively engaged in the problem under the leadership of G.P. Schedrovitsky
[10], as a result of their activities, technology appeared schematization of
such tasks [41], which copes with this problem, but generates another: this
tool perfectly helps in the initial “entry into the task“, that is, in the
initial analysis of the inventive situation, but is practically useless for
its further solution, however, with the problem of “solving in depth”
management tasks through the identification of contradictions and their
subsequent resolution TRIZ mechanisms do a great job. This thesis is confirmed
by work experience.  the author together with representatives of the
methodological school G.P. Shchedrovitsky in a number projects.

In addition, when you try to use the RBI operator to resolve inconsistencies
in organizational and managerial tasks [11], the solver inevitably faces
difficulties in determining the operational area (OZ) and determining the
resources that can be mobilize for the search for the most effective solution,
since the boundaries of health outlined by abstract concepts, rather than the
physical frame of conflict, as in technical tasks. In this paper, an attempt
is made to formalize the selection OZ when resolving contradictions in
managerial tasks. The author shows how in this way it is possible to outline
the operational area not at a point in space, as is done in technical tasks,
and in the plane of abstract concepts that are often used in description of
conflict in organized social systems (motives, incentives, reaction, values,
desire, competencies, key performance indicators, etc.), which is the core of
a number of organizational and managerial tasks [11]. Application a similar
approach, including the use of a shortened version of ARIZ and its elements to
solve such problems is given in a number of cases on the author’s site [6].

The practical need for preparing organizational and managerial tasks for
further analysis using TRIZ tools is long overdue. Also obvious the need to
offer a simple and convenient mechanism for determining operational zones,
since the lack of a methodologically developed mechanism for determining OZ
restrains the use of ARIZ for this class of problems [11]. The author believes
that short ARIZ versions (in 6-7 steps) -- an excellent tool for solving
organizational management tasks, which managed to prove itself in practice as
reliable a tool that provides a sustainable result.

\subsection{Goals and objectives of the study}
The objective of this work:
\begin{itemize}
\item To propose a way to formalize business tasks using schematization and
  draw up a roadmap for applying schematization when deciding organizational
  and management tasks in order to further successful application of TRIZ
  tools.
\item Develop areas of transition from schematization to TRIZ tools, including
  number, at the level of the terminological apparatus. This question is
  relevant also because between TRIZ and SMD specialists methodologists
  (followers of the school of GP Shchedrovitsky) last quite an active exchange
  of information has been going on for several years, however the issue of
  transition between tools today no one worked out;
\item Develop a method for determining the operational area in the
  organizational management tasks, taking into account the peculiarities of
  the wording conflicting elements in similar tasks.
\end{itemize}
\subsection{Scientific novelty of the research}
The scientific novelty of this work is as follows:
\begin{itemize}
\item The author has developed a method for applying schematization to
  training organizational and managerial tasks for the further use of
  mechanisms TRIZ as an indispensable condition for the analysis of the
  inventive situation in the field organizational and management tasks.

  The author conducted a detailed analysis of the work of G.P. Shchedrovitsky
  and based studied material developed a sequence of schematization for
  analysis inventive situation simplifying the further use of tools TRIZ in
  order to solve such problems. The author considers analogues of such
  approaches used in TRIZ (system operator and functional modeling during the
  FSA), and concludes: \textbf{schematization has a unique mechanism for
    determining managerial layers, and also the concepts of a generalized
    object and content, which gives new opportunities for setting private
    tasks in solving organizational management tasks, with the ability to
    scale received decisions. These features in the existing perimeter of TRIZ
    tools are absent, which significantly inhibits the use of TRIZ mechanisms
    for solving organizational and management problems.}
\item The author proposed a system for setting goals based on the results of schematization
inventive situation by sequential analysis [11]:
\begin{itemize}
\item models of a functioning system (MFS) at the junction of a system -- a
  supersystem;
\item the degree of controllability by layers in the diagram;
\item relationships (communications, functions, processes);
\item generalized objects and their filling.
\end{itemize}
\item Compiled a roadmap:
\begin{itemize}
\item Identify the problem situation.
\item Define the conflict area and identify conflicting pairs (objects and
  subjects of organizational and managerial tasks)
\item Apply system elements around the conflicting pair and identify secondary
  problem situations related to the task.
\item Define the relationship between the elements of the system at the level
  of generalized objects (“generalized object” -- a term disclosed in detail
  in the text of the dissertation. The term does not replace, but complements
  the concept of “element system“, is a subsystem of a system element). If
  necessary identify processes.
\item Identify the immediate elements of the system, including “regulators”.
\item Identify the conflicting areas of the generalized object and content.
\item Set up a system of tasks by conducting analysis:
  \begin{itemize}
  \item models of a functioning system (ISF) at the junction of the system -
    supersystems;
  \item degree of controllability by layers in the diagram;
  \item interconnections (communications, functions, processes);
  \item generalized objects and their filling.
  \end{itemize}
\end{itemize}
\item A method for determining the operational zone in the organizational
  management tasks characterized by a high abstraction of descriptive
  characteristics, as a result of which it is impossible to outline a piece of
  space, in which develops the conflict (unlike most technical tasks).  This
  method makes it possible to use ARIZ mechanisms to resolve contradictions in
  organizational and managerial tasks with high degree of abstraction.
  \textbf{The novelty is that using the method, developed by the author, the
    solver can not only determine the operational zone as a physical contour
    of space (it is worth noting that this the possibility reinforces the use
    of schematization, where such sections it’s very convenient to select),
    but also to determine the operational zone directly from wording of
    technical contradiction, and subsequently highlight operational zone
    resources in the form of factors determining the state systems and
    properties indicated in technical contradiction.} Skill allocating
    resources from abstract concepts -- the most important skill in solving
    organizational and management tasks, since the physical contour of space
    in the order of the solver may simply not be [11].
\item Compiled a roadmap:
  \begin{itemize}
  \item Formulate a pair of TP;
  \item Choose a working TP;
  \item The conflicting pair of the selected TP forms the operational zone,
    including: tool and product;
  \item Allocate resources in the form of a group of factors affecting the
    tool and product;
  \item Next, we work in the ARIZ logic: we assign the RBI rule, substitute it
    into rule RBI resources of the product and tool, etc.
  \end{itemize}
\end{itemize}
It is worth noting that the author came across the opinion of a number of
experts that in relation to organizational and managerial tasks, it is
incorrect to use the term “Technical contradiction.“ A number of TRIZ
specialists consider it worth highlighting market, organizational,
interpersonal and psychological (intrapersonal) contradictions [40]. The
author does not agree with this principle of division, since TP is a form
representations of the conflict, and the indicated contradictions do not apply
to the form presentation of information, and to the level of solving the
problem (in TRIZ was originally designated macro- and microlevels, and such a
classification refers specifically to the level of solution tasks. If the
concept of “technical contradiction” introduces some confusion, you can apply
the already established concept of “dialectical contradiction of the first
kind”) [11].  Of course, an understanding of the typical levels of formation
of contradictions in solving organizational and managerial tasks -- important
information for the solver, simplifying formulation of contradictions, but
terms describing levels in organizational management task and the concept of
“technical contradiction“ are not identical, and therefore interchangeable.

It is worth noting that when solving organizational and managerial tasks, a
method for quickly resolving technical contradictions is used, which is
especially important at decision organizational and management tasks
characterized by plurality of contradictions. The method was developed in the
System Restriction Theory (CBT) for working with a “thundercloud (the method
is described in detail by Darrell Mann back in 2000 and published in The TRIZ
Journal [12]), however, since it is used in TRIZ a slightly different form of
graphic representation of the contradiction, the author spent adaptation of
this tool, as a result, the method of express analysis of contradictions
became much simpler and now requires much less time than in the original
version [12], which makes this tool extremely practical [11]. However, the
author accepted the decision not to include a description of the modified
contradiction analysis tool in the dissertation, as the author’s innovation
related to the transfer of TOC approaches for TP permissions in TRIZ do not
have sufficient novelty in one form or another used by many TRIZ specialists
[12], [40]. If desired, more details familiarize yourself with the application
of this approach by the example of solving practical problems, \textbf{see
  appendix 3}.

\subsection{The practical relevance of the study}
\begin{itemize}
\item[1.] The proposed methodology of schematization in order to formalize organizational
management tasks allows you to:
\begin{itemize}
\item Define the system contours without missing important details, and on the
  other hand, exclude “Extra“ elements of the system, taking into account the
  objectives of the task at the expense of system visualization and
  highlighting the position of the solver. System Operator the author’s
  opinion is not an alternative to schematization, since this tool has no
  means of describing the relationship between elements of the system, he sets
  only its composition;
\item Put a system of tasks for further solution using TRIZ means, not missing
  from consideration of important aspects of the organizational and managerial
  task;
\item Several times reduce the time for team communication during analysis
inventive situation.
\end{itemize}
\item [2.] The proposed methodology for determining resources as factors
  affecting elements of the operational zone allows you to:
\begin{itemize}
\item Reduce communication time when allocating an operational zone, earlier
  very lengthy discussions had to be carried out in order to single out
  organizational and managerial task in case the solver conceived apply ARIZ
  to resolve the contradiction;
\item Define the resources of the operational area as significant factors
  affecting tool and product in the operational area without searching for
  objects in the business system, which dramatically increases the speed of
  analysis and the quality of inventive solutions.
\end{itemize}
\end{itemize}
All this makes the proposed methods suitable for practical use.
in consulting projects. Detailed application examples
proposed methods in consulting projects (see Appendix, as well as sources
[6], [11]) proves their instrumental nature.

\subsection{Key Points for the defense}
\begin{itemize}
\item[1.] Application of schematization for the invention situations in
  organizational and management tasks and communications schematization with
  tools adopted in TRIZ.
\begin{itemize}
\item The goals of applying schematization in solving business problems.
\item The method of setting a business task through the use of schematization.
\item Terminological apparatus of schematization.
\item The scope of schematization and its application in conjunction with
  others TRIZ tools.
\item Conclusions on the use of schematization.
\end{itemize}
\item[2.] Method for identifying the operational zone in organizational
  management tasks from model of technical contradiction.
\begin{itemize}
\item The objectives of the allocation of the operational zone in business
  tasks.
\item In what cases is it necessary to resort to the allocation of the
  operational zone in business task.
\item Difficulties with determining the operational area in business tasks.
\item The method of isolating the operational zone from the technical
  contradiction model.
\item The definition of tools and products and the allocation of resources as
  factors affecting the tool and product in the operational area.
\item Conclusions on the application of the method for identifying the
  operative zone from the model technical contradiction.
\end{itemize}
\end{itemize}

\subsection{Personal contribution of the applicant}
\begin{itemize}
\item[1.] The use of schematization according to GP Schedrovitsky for
  pretreatment poorly inventive situation in organizational and managerial
  tasks with the purpose of obtaining a system of private tasks, which are
  then processed with using the TRIZ arsenal. Connection of schematization
  with TRIZ tools.
\item[2.] Application of the categories “Generalized object” and “Filling” in
  order to obtain scalable solutions when using TRIZ tools to solve
  organizational and management tasks. The terms “Generalized Object“ and The
  “generalized object“ is explained in detail below. Briefly: “Generalized
  object” and “Filling”-subsystems of a system element, these concepts are
  refined the concept of “system element“ and are of great practical
  importance in the analysis organizational and management tasks from the
  perspective of scaling received decisions;
\item[3.] Development and testing of a method for determining the resources of
  the operational zone in organizational and management tasks.
\end{itemize}
\subsection{Work approbation}
\begin{itemize}
\item[1.] Scientific conference “TRIZ. The practice of applying methodological
  tools.“ Moscow, 2016;
\item[2.] Training under the TRIZ program in full-time and distance format,
  trained more than 300 specialists. During the training, students solved
  problems from their practice under the author’s guidance and used these
  tools in their projects;
\item[3.] At the time of writing the dissertation, the author has completed
  more than 50 consulting projects using these tools;
\item[4.] The book TRIZ. Solution of business problems / A. Kozhemyako. -- M.:
  Synergy University, 2017 .-- 288 pp., Ill. -- answers to questions from
  readers who applied the recommendations in their projects. In 2019 comes the
  2nd edition, revised and supplemented.
\item[5.] Scientific conference “TRIZ-Summit“, Minsk, 2019
\end{itemize}
\subsection{Publications on the topic of the dissertation}
\begin{itemize}
\item[1.] TRIZ. Solution of business problems / A. Kozhemyako. -- M .: Synergy
  University, 2017. -- 288 p.: Ill .;
\item[2.] A. Kozhemyako. Non-technical TRIZ: experience in solving
  organizational and managerial tasks, limitations and tools. Materials for
  the VIII anniversary conference “TRIZ. The practice of using methodological
  tools and their development.”
\item[3.] A. Kozhemyako. Ideas for the joint use of TRIZ and SMD for solving
  problems business. Publication on the site \url{http://www.bmtriz.ru}.
\item[4.] A. Kozhemyako. Ideas for the joint application of TRIZ, SMD and TOS
  for solving problems business. Part 2. Publication on the site
  \url{http://www.bmtriz.ru}.
\item[5.] A. Kozhemyako. A little about the systems thinking of the head of
  the sales department.  We apply system analysis. Sales Management Magazine,
  03 (98), 2018.
\item[6.] A. Kozhemyako. Schematization of the inventive situation in the
  organizational management tasks. Materials for the conference “TRIZ-Summit“.
\item[7.] A. Kozhemyako. Features of the application of TRIZ in organizational
  and managerial tasks. Materials for the conference “TRIZ-Summit“.
\item[8.] Morphological analysis to solve business problems -- publication in
  a journal “Management today.“
\end{itemize}
\subsection{Structure and scope of the work}
The work consists of introduction, three main sections, conclusion, and six
applications, including examples of practical application of the proposed
methods, set out on 83 pages; includes 28 figures, 10 tables, a list of
references from 46 titles, including books and publications by the author on
the topic of the dissertation.
