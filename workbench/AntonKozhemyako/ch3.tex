\chapter{Method for identifying the operative zone in organizational and
  management tasks from a pair of TP} 

\section{Goals and objectives of the study}

It is known that the operational area is the space within which
there is a conflict specified in the model of the problem [47]. The author emphasizes that with
solving organizational and managerial tasks most often, operational zones -
several, that is, usually we are not talking about one conflict, which is the cause
occurrence of an inventive situation, but about their combination. And it’s not at all a fact that
After conducting a causal analysis, the solver is guaranteed to establish one-
the only reason for their appearance, if, of course, it does not move on a causal basis
Inward-outward investigation chain.
The definition of the operational zone is required, first of all, for localization
places for choosing a resource, since decisions obtained using resources taken from
areas of conflict are closest to ideal [22].
In technical tasks, the operational zone is much easier to determine, there
it is always localized in space, and its localization is defined by the boundary
tasks. For example, if during a drilling operation, the cutting edge of the drill
blunts, then the operational zone is in the zone of cutting the metal of the workpiece, to
transition to the micro level, of course. When moving to the micro level, operational
the zone will be in the layers of the material of the cutting edge drill, possibly
will go to the level of the grain boundaries of the metal, etc. For specialists in
material science operational area is quite obvious.
When solving organizational and managerial tasks, such certainty is not
observed. For example, if you assume that you are a scientist and spent a long
analysis of the problem of employee motivation using tools
pre-processing tasks, as a result of which they were able to reach the level
neurobiology and localize the biochemical processes of the brain in the operative zone, then
in this case, the operational area will be determined by the same principles as
and in technical tasks, that is, it will be a piece of space in which there is
conflict leading to an inventive situation. What blocks
sufficient dopamine release [29]? Why not happening
positive reinforcement for which serotonin is associated with other
neurotransmitters [29]? The task becomes material, belonging
physical objects ...
However, the vast majority of managers are not neurophysiological scientists.
Even psychologists primarily operate with abstract concepts and
prefer not to touch the physical levels [23], [29]. Therefore, when deciding
organizational and physical tasks physical layer usually not available
solver. And this is not an exception, but a rule. For example, what is motivation? By
Page 37
37
by and large, motivation is the person’s inner ability to overcome
resistance to rest to achieve the goal [29] Motivation is always
concerns the inner world of man, in contrast to stimulation (external
impact) . Analyzing the phenomenon of motivation, we are forced to analyze such
categories like human abilities, peace, goal, human values ​​and
etc. -- even a cursory analysis of these categories gives an understanding that the physical level
solving such problems is usually not available to the solver. At least in
present. Continue: components of motivation: goal, willpower,
self control. Goals are set by the context, environment and human value system
[24]. It’s very difficult for a solver to “grab” “hard” resources, that is, resources for
physical level. Talking about motivation, we can say a lot about conflicts,
without mentioning a single point in space.
Therefore, if the operational area is the location of the conflict,
which contains a tool (an object that performs a negative impact),
product (an object that perceives this effect) and the environment surrounding
conflicting pair [3], and the operational zone in organizational and managerial
tasks often can not be described at the physical level, then operational
a zone in such a class of tasks is nothing more than a conflicting pair isolated from
TP pairs (TP1 or TP2).
Why is there a point in solving organizational and managerial problems
first highlight the contradictions, and then move on to the allocation of the operational zone?
This is due to the fact that in such problems the contradictions are primary, usually they
manifest either in the conflict of interest of key stakeholders ( author
repeatedly used the following move in projects: after schematization
inventive situation and highlighting key stakeholders carried out
MPV analysis [43] , which allows to identify stakeholder requirements included in
contradictions, in fact -- a conflict of interest. These contradictions are analyzed and
fixed in the form of a pair of TP).
Or contradictions appear when you try to make some
changes, for example, after highlighting tasks in the wake of schematization solver
comes to the conclusion that certain changes are necessary, but with mental
the projection of these changes on your system sees a secondary undesirable effect,
giving rise to a contradiction. At the junction, it is easy to formulate a new pair of TPs, which
usually done. For example, in Sberbank a similar thought experiment with
recently entrenched as an organizational and managerial norm (from
conversation of the author with TRIZ corporate training participants for employees
Sberbank TT Group) .
Understanding that operational and organizational tasks
the zone is determined by the conflicting pair in the working TP with the addition of a description
environment of negative interaction between the tool and the product (note : tool -
the subject of the negative “processing” of the object, that is, the product) , gives the key to
the source of resources closest to the source of the conflict in such tasks and
allows full application of ARIZ mechanisms to organizational
management tasks (the tool will be the state of the system in
previously selected working TP , and the product -- a deteriorating consumer property
systems). The author has already indicated that to address organizational and managerial
tasks it makes sense to use only shortened, “combat” forms of ARIZ in several
steps.
Page 38
38
The main feature of the method of describing health care for management systems,
which the author describes is the use of a new factor approach when
description of the operational area. This method is well suited to the description of OZ in complex
social systems with objects distributed in time, in space and
according to other characteristics by parameters. For example, if in the purchasing department of a company
there are changes in significant areas of the business process, then such changes
entail both positive actions and undesirable effects in other departments, then
there are immediately many operational areas. The question arises: how are they
track, especially in large companies? And whether they need to be tracked, perhaps
is it easier to switch to another method of description -- factorial? In addition, solutions may
have a delayed effect, and in different areas of the business -- at different times.
Therefore, when solving organizational and managerial tasks, it is much more effective
consider what factors affect the state of the system and the properties specified in
identified contradictions, and use them already as resources [11].
That is, we are dealing with a situation where the system contains many
elements, each of which is determined by many factors.
It is very difficult to describe the relationships between elements in a business system, especially with
given that the relationships between the elements are constantly being rebuilt depending
from the influence of external and internal system factors. That is, highlight
elements in operational areas, and even more so, it is extremely difficult to catch the connection between them
it is difficult, therefore, to identify the operational areas associated with the studied
undesirable effect -- almost impossible.
However, if we are dealing with contradictions arising in business
systems (Fig. 14, Appendix 5), in practice it is much easier to determine which
factors determine states and properties in contradiction than to look for
conflicting pairs of elements in the business system associated with the studied
a contradiction, and describe their operational areas, and later -- explore the parameters
of these elements. As noted above , significant factors of X-elements
business systems that determine the state of the system and the properties recorded in
contradictions, it is much easier to detect than the elements themselves, in addition,
discovered parameters are easy to use as resources for solving
set task. That is why the factor approach in working with operational
zone when solving organizational and managerial problems should be considered as
perspective.
The author believes that this conclusion, which was obtained and confirmed in
the course of participation in more than 50 projects allows universalizing ARIZ approaches
and apply them equally successfully to both technical and organizational
management tasks.
The analysis of the ARIZ task remains the same for both organizational and
managerial and technical tasks, with the only difference being operational
the zone of the technical problem is the physical area of ​​space, and in the organizational
management tasks the operational zone is formed by a conflicting pair
“The state of the system is a consumer property” along the negative branch, in fact -
conflicting pair expressed in abstract terms (not physical
area of ​​space).
Page 39
39
Otherwise, the procedure for applying ARIZ to any artificial systems
stays the same, which allows the use of highly recommended
TRIZ tools for solving organizational and managerial tasks.
An example of the allocation of the operational zone in organizational
management task
We apply the found principle to the resolution of the contradiction depicted in
fig. 14. Recall the essence of the problem (in the form of a pair of TP), which will be the starting point in
our further considerations:
If the number of transactions is simultaneously worked out by a department employee
15 sales, then the managers go to the targets faster, however, for
fulfillment of sales plans with existing indicators of transaction conversion
increase the number of managers in the sales department, which is unacceptable (TP1).
On the other hand, if the number of transactions is at the same time
sales person 25, then to fulfill sales plans when
existing conversion rates need to hire fewer employees in
sales department, however, sales staff slowly reach the planned
indicators, which is unacceptable (TP2).
Further, according to the ARIZ logic, the working
technical contradiction, for which we carry out the following reasoning:
A sales department is created for personalized work with a customer on
creating a high subjective value of the proposal, overlapping the value
payment. Personalized work is the core of definition. If possible
the same value is created by influencing immediately on a group of consumers, the sales department does not
needed, it should be curtailed as an extra link in the business [16]. If the sales team
is present in the company and is not a consequence of psychological inertia
managers (sellers should be, because they have always been here), then
personalization of customer impact is the only way for the company
convey the high value of your proposal.
Of course, in this regard, sellers should serve the greatest
the number of transactions at the same time without loss of conversion and without
Reducing the average weight of a transaction within a specific client category . From here
it is clear that one manager should not have deals in simultaneous processing
15, 25. Therefore, the working TP is as follows (Fig. 14):
Page 40
40
Fig. 14. The choice of working TP from a pair of TP
We need to increase the workload of managers, so we select TP2 : if in
At the same time, the manager is working on 25 projects, the number of
managers in the sales department will decrease, but the manager will go on
planned targets are longer, which is unacceptable.
Then the conflicting pair will be as follows :
25 projects were given at the same time being worked out by one manager and 6
months of reaching planned sales figures.
There is clearly a conflict here: 25 projects are being worked out at the same time
one manager (tool), negatively processes the exit to the planned
sales figures (product):
Fig. 15. The operational area, including the tool, product and their environment
interactions. 25 projects -- a tool, planned indicators -- a product.
Next, factors that significantly affect
the tool and product included in the model of the operative zone (Fig. 15). Afterwards
identified factors can be used as resources, substituting them in the rule
ideal end result (RBI).
Page 41
41
Here is an example of identifying factors [11]:
Operational Zone Element
The role of the element Resource as a subsystem of each element
25 projects at the same time
at one manager
Tool
Decision making scheme in a category A project
Decision making scheme in category B project
Stage of the transaction (sales funnel)
Sales channels
Project related work
Errors in recruiting a customer base
Going to targets 6 months
Product
The number of leads (responses to marketing
activity)
Lead quality
Sales funnel conversion
The average frequency of a transaction per year
Employee Competencies
Client base recruitment regulations
Collaboration with colleagues
As a result, we received an impressive list of resources for solving
tasks. Some can be further decomposed, for example: sales channels,
related work of the manager in the project, etc.
If there are many resources received, they can be subjected to the procedure
prioritization, for example, according to the following logic (priority falls on the left
to the right) [25]:
1. The element of the operational zone: Product → Environment → Tool
2. Quantity: Unlimited → Sufficient → Limited
3. Quality: Harmful → Neutral → Useful
4. Value: Free → Penny → Dear
Prioritization of resources: the higher the final score, the higher the priority [11]:
Operational Zone (OZ) Resources
Element oz
amount
Quality
Value
ITO
G
Ed
e
l
and
e
Wed
e
Yes
Ying
page
mind
e
nt
N
e
about
gr
ani
what
nn
th
D
about
stato
h
th
ABOUT
gr
ani
what
nn
th
Vre
dny
(about
tho
dy)
N
e
th
tr
alny
P
about
l
e
know
B
e
spl
atomic
TO
about
ne
e
h
th
D
about
R
about
go
th
Decision making scheme in a category A project
one
3
one
one
6
Decision making scheme in category B project
one
3
one
one
6
Stage of the transaction (sales funnel)
one
3
2
one
7
Sales channels
one
2
2
2
7
Project related work
one
3
2
2
8
Manager errors when recruiting a customer base
one
2
3
3
9
Number of leads
one
one
one
one
four
Lead quality
3
one
one
one
6
Sales funnel conversion
3
one
one
one
6
The average frequency of a transaction per year
3
2
one
2
8
Employee Competencies
3
one
2
one
7
Set-up regulations
2
3
one
one
7
Collaboration with colleagues
2
2
one
one
6
Page 42
42
In our example, three resources allocated in
table in gray. Therefore, they should be used first.
For this problem, more than 10 solutions were obtained using dedicated
resources, and this is for one pair of TP! For example, I would like to show how it worked
harmful resource -- manager errors that get the maximum score.
In the ARIZ logic, the ideal final result (RBI) rule is assigned and then
instead of the X-element, the selected resources are substituted. (RBI -- decision
inventive task, allowing to obtain the desired result with
zero compensation factors. As follows from the laws of physics, such
a solution can never be reached and therefore the concept of perfect
the final result serves to reduce the degree of psychological inertia in
the process of solving the problem by orienting the problem solver to search
solutions with the highest degree of ideality [42] ).
We demonstrate these steps:
1. Rule of RBI : <X-element> itself provides access to planned indicators
manager for 3 months ( condition of the task manager ), provided
performance index of 25 projects at the same time exploring ( with
increasing the load on managers to achieve planned targets in
companies occurred on average for 6 months ).
2. Manager mistakes when recruiting a database of projects themselves provide access to
manager's planned targets for 3 months, subject to fulfillment
indicator of 25 projects at the same time.
3. Since it was not possible to directly obtain a solution from RBI, we proceed to
the formation of physical contradiction (FP) around the selected resource
( FP -- a situation that occurs when a certain attribute
the object of interest to us must have two different meanings
at the same time to ensure the desired result [42]): errors in
recruitment of the base should lead to the correction of technology sales manager,
in order to reach the target indicators for 3 months, and errors in the selection of the base are not
lead to the correction of the manager’s sales technology, since the manager doesn’t
has sufficient skills to reflect errors that occur during recruitment
customer base .
Since the company that set this task has implemented an adaptation system
sales staff, it was easy to take control of the mentor process
recruitment of the base of projects to newly arrived managers and to carry out its reflection
first, 2 times a week, then -- 1 time per week, then 1 time in 2 weeks,
thereby making his mistakes a resource for correcting further work. Similar
reflection is carried out according to performance models developed for mentors
[16].
Page 43
43
After the formation of the FP, the solution involving this resource turned out to be
the obvious is managing the employee’s reflection process in the process
initial set of customer base.
Roadmap for working with the operational zone in the organizational
management tasks
• Formulate a pair of TP;
• Determine the operating TP (TP1 or TP2);
• Select the conflicting pair in the working TP;
• Highlight the operational area, additionally defining the interaction environment
tools and products (the operational area consists of a “tool”,
carrying out harmful effects and “products” perceiving
harmful effects [3]);
• Allocate the resources of the operational area as factors, in a significant way
affecting the tool and product and determine their priority if resources
a lot of.
• Formulate a RBI rule.
• Substitute resources in the RBI rule instead of the X-element. If the decision is not
obtained at this stage, then form around the selected resource
physical contradiction (we act in the logic of ARIZ).
Page 44
44
Conclusion: conclusions and recommendations
The effectiveness of the proposed methods
The effectiveness of the proposed methods is practically confirmed:
1. Schematization -- the tool is used in more than 30 projects;
2. Formulation of the operational area in management tasks -- in more than 30
projects.
These tools are included in the training program implemented by the author in full-time
format and format of the online workshop. 200 online training programs trained
people, according to the full-time program -- about 150 people. During the training, students
(students are company specialists) carry out projects in the field of their
activities and protect projects based on learning outcomes.
Scope and limitations of the proposed methods
The author assumes the use of these techniques to solve organizational
management tasks set in any organized social systems.
These tools may only be used provided that
the solver owns the subject of research, or interacts closely with
specialists with the required subject competencies in the field of
strategic and regular management, marketing, sales, financial
planning, psychology, etc. [2], [13], [23].
The author’s practice shows that the greatest efficiency in application
tools can be achieved in team mode, if the work
teams are effectively supported by flexible project management tools,
First of all, Scrum technology [26].
The author's recommendation for use in organizational and managerial tasks:
1. Always apply schematization to the full clarification of inventive
situations subject to the use of TRIZ in organizational and managerial
tasks;
2. Apply the method of formulating the operational area and working with resources
of the operational zone proposed by the author only if
formulating a TP pair, the solution is not obvious and the solver expressed a desire
continue to move in the logic of ARIZ.
Page 45
45
The possibility of further development of techniques
The author believes that TRIZ specialists should take a closer look at
G.P. Shchedrovitsky [10], [41] and study the application of categories of systems,
proposed by the author, for a more accurate and quick description of inventive
situation.
A special methodological study requires the category of "layer", as well as
“Generalized object” and “filling”, practical recommendations for more
the conscious use of these concepts in solving organizational
management tasks. The author believes that in this direction it is worth continuing
research.
The author believes that the proposed description of the operational area is not
final. The author believes that it is necessary to develop a special
methodological language for the description of the tool, product and, in particular, their environment
interactions . As a result of this description, the allocation of operational resources
zones can be much more accurate, therefore, will give even more interesting
practical results.
The development of criteria according to
with which the solver can make a detailed analysis of the resources of the operational zone.
The use of TRIZ for organizational and managerial tasks today
a day far from established discipline, there is a significant
research work.
Page 46
46
ANNEX 1
An example of using schematization for analyzing inventive
situations in conjunction with S-curve analysis.
Objective: to increase staff productivity and reduce time costs
the first person of the company through a change in the employee motivation system. Note: in
This example does not show the final solution to the problem, it is demonstrated
exclusively an analysis of the inventive situation.
Tasks: pyrotechnic company "Fast and the Furious", St. Petersburg.
From fig. 13 shows that during the existence of the company, the author changed three
motivational models with which he inspired his close-knit team.
As the MPV (main parameter of value) adopted "employee productivity",
expressed in the number of operations per shift with the required level
quality . Since the concept of “operation” is predetermined, and the operations themselves
reflected in the technological maps that are compiled for each event,
perform a performance calculation and determine the level of quality of performance
work is easy.
Fig. 16. Change of various motivation systems in the company.
We describe the motivation system shown in Fig. 16:
Curve No. 1 -- at the beginning of the existence of the company, in the late 90s -- for the promotion
It was considered to be arranged in an organization where there is a normal social package,
stable salary and healthy relationships in the team. At the beginning of this approach
perfectly stimulated employees to work, compared with others not quite
"White" companies. But over time, the "white package" began to be accepted as the norm, and
stability was no longer a motivating factor, but perceived as
due.
Performance
employees
Page 47
47
Curve No. 2 -- the company introduced a system of cash bonuses. She gave
tangible
growth
performance
and
responsibility
but
then
performance declined. Previous bonuses were no longer available for motivation
to work. Numerous studies have been conducted on the effect of money on motivation,
of which it is known that the award is perceived by the employee as a motivator approximately
three months, after which he begins to take it for granted. According to the findings
S. Covey [30], money for business is like air, without them the company cannot work
can, and the employees quit. However, for life, a person needs not only
breathe. Therefore, such a system has very limited resources for its
application.
Curve No. 3 -- a decision was made to develop an intangible system
encouragement, which would be based on their own moral and ethical values
employees who must match the values ​​of the founder of the company, given
requirements of the pyramid of needs A. Maslow [31]. Today the system is in
the beginning of this curve, and according to the director’s forecasts, it will give a smoother, but
steady and continuous growth of labor productivity and personal
responsibility of employees, will spread its influence both on the selection of employees,
and on their retention. Naturally, the value motivation system is by no means
It does not cancel the system of monetary incentives, it supplements it. Observations give
reason to believe that a value approach combined with a powerful system
training can give about a twofold increase in
selected MPV.
From fig. 16 shows that the studied system in this company is located in
the very beginning of the third S-curve, and therefore, all the main efforts of the leader
must be spent on tuning the system -- you need to create such conditions that
the system began to work steadily. There are no other priorities at this stage.
Now you need to set tasks, for which it is proposed to parse the system in
the form in which it exists now, and then determine the desired parameters
system, its future configuration ( Fig. 16 shows that the transition to the third
the curve in the company has just occurred while substantial
value shift in the minds of employees takes time and does not occur
instantly ).
To describe the current inventive situation, we apply
schematization:
Page 48
48
Fig. 17. The use of schematization for the analysis of an inventive situation.
Since we resorted to schematization after applying the analysis on S-
curve, we got some new knowledge. From fig. 16 we see that introduced
the bonus system (curve 2) was implemented quite successfully, which resulted in
MPV growth, although the system reached its saturation quite quickly.
Naturally, in the diagram (Fig. 17) we fixed the structure of this system in a section
"It was".
Then there was a transition to curve 3 (Fig. 16) as a more promising
naturally, with the preservation of the bonus, that is, between the two curves occurred
continuity . If the company didn’t do this, then the transition occurred
there would be a significant "drawdown" of MPV (Fig. 16). Yes and no cash reward
Motivation systems have no prospects, any normal leader knows this.
In addition, the company determined the position of the new motivation system at S-
curve -- this is stage I (Fig. 16). In accordance with the objectives of the first stage, we say not
as much about efficiency, how much about the potential of the system and its minimum
health. Therefore, in the diagram in Fig. 17 we depict the structure based on
assigned tasks: to ensure the minimum performance of the new system
motivation, but taking into account its configuration (at the heart of the system proposed
task manager, the pyramid of needs A. Maslow). The author does not consider the pyramid
A. Maslow is an exceptionally correct model, but to describe the current situation, she
perfect -- over the years of work in this company, employees have "grown" from the point
view of values ​​and the pyramid of A. Maslow it simply and reliably demonstrates.
Carrying out the analysis of the circuit in Fig. 17, we found unwanted effects
(NE) and recorded them in the table:
No. Condition of elements
existing
system, "It was"
Item Status
new system
Tasks
No.
NJJ Description
one
Prize
distributed
directively
the director
Premium distributed
collective according
contribution
SJS 1
Grievances and their hidden
conflicts
SJ 2
Manipulation of employees in relation
to colleagues
Page 49
49
2
Staging
detailed
strictly defined
SMART tasks
Outlining frames
statement of general tasks
and setting constraints
employees themselves
SJS 3
Recently arrived employees cannot
work in this mode, since they don’t
lack of knowledge
SJS 4
Distracting experienced staff for
control tasks less
experienced
Transition to a single environment
planning
for example to flexible
design system
management for small
teams -- SCRUM
SJ 5
Regular weekly planning
takes extra time, usually -
2 ... 3 hours a week.
SJ 6
Irritation from repetitive operations
planning, team briefings, etc. →
decreased attention, attitude to the system
planning as an unnecessary load
four
Employee values
at level 1-2 in A.
Maslow
Mature employees
having values ​​3-4
A. Maslow level
SJF 7
Often, such employees want to open their
business, so they leave the company
SJ 8
Such employees have their own
opinion on working matters with them
need to agree. Manage such
people -- it’s the same as grazing cats.
SJ 9
Need to maintain interest in
all areas of motivation -- money,
emotions, intelligence, meaning and contribution to
a society that requires significant
efforts from the director / owner
SJS 10
Such an employee has versatile
interests, not the fact that manufacturing
tasks will be paramount for him
5
Primary control
carried out by the director
Director carries out
general control
indicators, control
quality operations
do it yourself
employees
SJ 11
With the loss of workplace value for
employee risk of deterioration
quality of operations, which
may go unnoticed
SJ 12
Even when an employee sincerely tries,
he is subject to the factor
eyes ", that is, simply does not see
own flaws that are easy
see from the side. However external
detailed control abolished.
The selection of NE in the analysis of the circuit in Fig. 17.
Thus, 12 tasks were set, the solution of which can provide
working capacity
selected
the system
motivation
staff.
Analysis
inventive situation allowed to quickly identify and formulate 12
specific tasks, which would be difficult without using a system
approach, given the fact that the selected system is at stage I of development according to S-
figurative curve and did not pass approbation.
It was decided to introduce a motivation system taking into account the implementation of the found
solutions to the tasks given in the table.
Page 50
fifty
APPENDIX 2
The task was set by the director of one company as follows: how
register business processes independently, without complicated terminology and unnecessary
paperwork? It was required to give a minimal template that would allow
perform work on the description of the company's business processes so as not to produce
unnecessary information. It would not happen that the developed business processes
would slow down the company, deprived of its required dynamics. At the same time, work on
the intuition, as before, is no longer possible, the young company was faced with the first
growth disease. The company faced a serious controversy.
A lot of literature is available on how to prescribe business processes. But almost
it is not indicated anywhere how recommendations for describing business processes in
depending on what stage the company is at. And even if such
There are recommendations; they are quite heavy and bulky. We set the task to give
capacious and accurate recommendations, differentiated for different stages of development
business, which will answer exactly the question: what model
Need to prescribe business processes for this particular company ?
We apply the system operator in order to better understand the system by
Description of the company's business processes. The structure of the system operator is shown in
fig. 3 in the main part of the dissertation.
System Operator:
1) PRESENT.
1.1. The investigated system: business processes.
The system of business processes, special attention to cross-cutting business processes (affecting
work of 2 or more departments or groups). Process flexibility.
Quote: “Most companies are organized according to a functional principle, but they
should work in conditions of interfunctional interaction. ... processes
break the hierarchical structure. "
1.2. Supersystem
Strategic management, balanced scorecard. Horizontal
employee interaction. Quality management system -- as a methodology. Market,
competitors. The dynamics of the environment. Changes to the law.
Quote: “From the point of view of the process approach, the organization appears as a set
processes. The management of such an organization is based on process management.
Each process has its own goal, which is its criterion.
effectiveness. The goals of all processes are lower level goals, through
the implementation of which top-level goals are achieved -- the goals of the company. ”
Page 51
51
1.3. Subsystems:
Business process system (model), responsibility management, management
personnel, process regulation, personnel reporting, process automation,
process performance management.
2) PAST 30s 20th century.
1.1.
System:
A person in the workplace, instructions of managers (namely “instructions”).
A.K. Gastev focused on the human factor. He believed that the main thing
the role in the work of the enterprise is played by man. Quote: “organizational effectiveness
begins with the personal effectiveness of each person in the workplace, in particular
with the efficient use of time ”(development of a description technique
production processes at this time is primarily associated with the name of this
wonderful person).
The most important problem, according to A.K. Gastev, there was an inability of a working man
obey, work in a team and strictly follow the instructions of the leaders.
1.2.
Supersystem
The survivals of the agricultural system, industrialization, leadership, production pace.
Rigid hierarchical structure in the enterprise.
A.K. Gastev noted that “workers do not know how to keep a single production pace
and work as well as their European counterparts do. Way of life
peasant Russia without rich European working traditions. ”
1.3.
Subsystems:
Workplace; Personal qualities: a sense of time, personal efficiency at work
place, the ability to obey.
A.K. Gastev emphasized that “Russian workers lack a sense of time.
Russia, in which the workers are former serfs who went to the free
bread, this way of life did not initially contribute to the acquisition of the European "installation on
time".
3) THE PAST 70-80s. 20th century.
3.1. System:
SADT standard (Structured Analysis and Design Technique ), functional methodology
Simulation IDEF0 (Integration Definition For Function Modeling) .
One of the best-known methodologies for describing organizations as organizational-
technical systems, has become the methodology of structural analysis and design
SADT systems (Structured Analysis and Design Technique ). It was developed
Page 52
52
American Douglas Ross (D. Ross) in 1973. Particularly widespread use
received one of the SADT subsets -- functional modeling methodology
IDEF0 (Integration Definition For Function Modeling ). The initiator of its development
and further standardization was the US Department of Defense. Methodology
IDEF0 was successfully used in military, commercial organizations to solve
a wide range of tasks (from software development for defense
systems prior to the development of logistics and management systems
finance). Availability and experience of using IDEF0 in various subject areas
areas, along with growing computer support, made it even more affordable
in use. This, in turn, also led to the widespread use of IDEF0
as a methodology for describing the business processes of organizations. In many ways, the popularity
functional modeling methodology IDEF0 due to the ease of notation,
the main elements of which are the function block and arrow.
Also in the USSR at the beginning of the 70s, an Integrated Management System was introduced in the USSR
product quality (CC UKP) . Management was based on the logic of mass
production, economies of scale, centralized control, and also resulting
low rate of change and a rapid loss of relevance.
The control system inherited from the USSR is based on the concept of mass
production, which dominated the entire national economy. The main purpose of this
systems -- get the economic effect of the growth of production. Than
the larger the volume of production, the lower the cost per unit of output. At
it’s easier to standardize and unify processes, and also easier
carry out centralized control. Such a system allowed to produce
a huge amount of TRU (goods, works, services), but in order to change something
had to spend a huge amount of resources due to lack of flexibility
in management and processes. As a result, it turned out that in the international arena, our
enterprises were uncompetitive due to lack of flexibility and
the inability to quickly adapt to the needs of the market.
3.2 Supersystem:
Strategic management, balanced scorecard. System
quality management. Competitors, market ... Relative stability, gradual,
smooth change of scenery (a significant difference from the NS "Real" ). Acting
legislation.
3.3 Subsystems:
IDEF0 Principles, Process Diagrams, Process Performance Management, System
business processes (model), responsibility management, personnel management,
process regulation, staff reporting.
4) FUTURE.
1.1. System:
Flexible business process cards integrated into CRM systems and more
high level (ERP).
Page 53
53
1.2. Supersystem
Self-developing business (company), further development of LEAN, CRM-system, ERP-
systems with integration of machine learning algorithms, BigData, distributed
registries.
1.3. Subsystem:
Instant access to self-updating information. Flexible business process system
(model), responsibility management, personnel management, regulation
processes, automatic reporting by indicators, flexible management
process efficiency. Automation, robotization, competency development system,
knowledge management.
From the analysis of the system using the system operator, we can distinguish
following:
1. Subsystems: Quick access to information. Business Process System (model),
responsibility management, personnel management, process regulation,
personnel reporting, process automation, performance management
processes. We see that business processes must be able to quickly
be extracted from the information environment, have high flexibility, have
reference points that show what will change in the supersystem when changing
business process at the level of a specific position.
2. Strategic management, balanced scorecard. Lean
production,
system
management
quality.
Market,
competition...
Relative stability, gradual, smooth change of scenery
( significant difference from the NS "Real" ). Current legislature.
When designing business processes, a system should be developed
indicators: KPI (key performance indicators) and managerial
indicators by which we track the effectiveness of achieving KPI.
The future shows us that business processes should be included with
knowledge management system, that is, a system should be developed
indicators tied to a competency model. Provide here
no gap! Automatic collection of statistics on indicators
special attention should be paid, to develop a culture of working with numbers,
gradually preparing the control system for the application of methods
machine learning in the future.
3. The human factor significantly affects performance and efficiency
processes. Therefore, when the responsibility matrix is ​​prescribed, the functional
defined, it is necessary to select people in a team with psychological and
competency portrait suitable for the position. Otherwise,
no one guarantees that the processes will work correctly and be fully implemented
volume.
It follows that business processes should not only be tied to
knowledge management system, but also with the profile of the position, which, in general,
is logical.
4. Take into account that the sense of time in humans has evolved since AK. Gastev, but
still far from ideal, so business processes should be
Automated in a CRM system with automatic notification, but in
in any case, before preparing the ToR for CRM, which ultimately will get
BP description, a paper document is created.
It’s important to consider that trying to regulate everything in a row is silly, but in
small companies, such attention to administration is fraught with loss
Page 54
54
business. Therefore, before the regulation of processes should be determined
company position on the S-curve (most conveniently according to I. Adizes) and based on this
set the “scale” of regulation, that is, determine the extent
detailing the process. It is also important to determine the degree of freedom of acceptance.
employee decisions in changing business processes in order to increase them
effectiveness. As stated above, allow for
operational process change, but with setting markers, which of
Related processes will be involuntarily affected. Should provide
differentiation of access rights to change processes.
5. Studying the success of IDEF0 shows us that for presenting business processes
you should try to get as far away as possible from text instructions in favor of
charts -- infographics, drawings with short explanations. If more is needed
detailed explanation, it can be given as a note to
corresponding paragraph of the infographic. Such instructions are perceived and
memorized much better, but there are pitfalls. Good
infographics -- the best option in terms of perception of instructions
by the user, but she has a huge minus in that drawing circuits is very
expensive and long in time. Not all employees can do this.
Today, this problem is resolved. In 2016 -- 2017, the present
boom in integrating graphical display of business processes in CRM-
systems according to IDEF0 recommendations. Hence it’s clear that it’s worth
pay attention to CRM-systems that have just such
opportunities and use them. It is important to consider monitoring for
indicators indicated above, differentiation of access rights, signaling by
reference points when making changes.
6. An integrated product quality management system (CS UKP) of the USSR may be
interesting only in the case of large-scale reengineering of business processes in
large corporations. In other cases, you should not contact her.
You should pay attention to the company’s standard for designations and
corpus of concepts. The corpus of concepts should be the same for everyone in the company and
as much as possible unified with the practice accepted in the world. "Translations" of terms
inside the company is too expensive. Therefore, together with the development
business processes should deal with the standard adopted by the company.
It’s better to immediately lay down standardized concepts and notation than later
spend a lot of time and effort to fix it.
7. When choosing a CRM system and a method for preparing a description of business processes
should take into account the rapid change in the environment, the system should be able to
make changes quickly, better -- without the involvement of IT-specialists. Otherwise
case, the dynamics can be lost, and the PSU will turn into empty trash and
will stop working.
You should not engage in self-written programs, but use ready-made ones
expandable systems to provide the above
functions.
8. When describing the BP should take into account the interaction between units. Exactly
at the junction of departments there is the greatest defect in communications, distortion
information and various kinds of failures.
Recommendations -- see above. Optional: when determining a personality profile
a specific unit should not be considered in isolation, but viewed in
together with departments and process owners with which business processes
intertwined most closely. The problem is solved at the level of generalized objects,
do not go into filling properties (see recommendations for schematization)!
9. In the future, the impact of IT technology will increase, so the final product will be
CRM-system with embedded PSU, giving hints in real time.
In the form of a list of documents BP will exist only at the time of implementation, in
quality of project documentation. Next is only the electronic format.
Pay attention to software manufacturers focusing on
issue of tips and statistics increased attention.
Page 55
55
As a result of the analysis of data concentrated in the system operator,
a matrix for the implementation of a system for describing business processes depending on
the stage of development of the company, the author has not met analogues of this matrix in any
specialized literature, nor in their practical activities (stages
development are given in lines according to I. Adizes [13]). The columns are informational
blocks required in the description of business processes for each stage:
Designations and abbreviations:
+ ... +++ -- the degree of detail of the documentation;
Business Processes (BP)
Organizational Structure (Organ.)
Job Description (CI)
Reporting (Report)
Regulations (P)
Standards Management (Ex. Art.)
Compliance Monitoring Standards (CIS)
Page 56
56
Performers in the description of business processes:
Designations and abbreviations:
"-" absent;
C -- himself;
K -- team;
Co -- consultant (expert).
Case shows that using the system operator provides interesting
results in solving organizational and management problems, but more suitable
to find strategic decisions .
When solving situational organizational and managerial tasks, application
schematization makes it possible to more accurately analyze inventive
situation and set particular tasks, including by applying categories of layers and
generalized filling object.
It seems that this state of affairs outlines the scope
system operator and schematization to solve this class of problems.
Page 57
57
APPENDIX 3
A task:
A modern, fast-growing Chelyabinsk IT company developing
platform solutions for working with digital content using
artificial intelligence (number of staff: approx. 200 people), repeatedly changed
their structure, trying to find a balance between the structure, which are based
departments (marketing, finance and accounting, several development departments, broken
competencies) and the structure based on project teams,
including a variety of specialists, depending on the objectives of the project.
The company faced several inconsistencies, one of which
consisted of the following:
Fig. 18. A pair of TAs related to the organizational structure of an IT company.
Contradiction analysis:
TP1. In order to optimize the loading of specialists, the basis of the structure
companies should have departments (unit units), as the head
unit determines how much time one employee needs to
one or another project and, since he owns information about all projects, he
can optimize the work of the employees of his department. That is, loading
unit specialists are optimized as there is a centralized
work planning by an experienced specialist (department head), who
sets tasks to department specialists and monitors their implementation .
TP2. In order for specialists to better understand the nuances of the project, they must be
completely involved in it, that is, belong to one or another project team,
since discussions of all the nuances of the project take place in the presence of the project
teams, and at all stages of the project -- from the inception and appearance of MVP (minimum
viable product -- the minimum viable product) [32] to obtain a full-fledged
market product, that is, they take part in the discussion of all the nuances
project in all its stages .
Page 58
58
Combining new entities with opposing states of the system on
fig. eighteen:
1. Is it possible to make the project team at the heart of the structure?
the company had a single coordinating center for competencies, which would put
tasks for subject specialists belonging to different teams, and
would control the execution of tasks?
Here came the idea of ​​applying flexible principles of project management at the level of
the whole company, that is, with a built-in system for recording time and project indicators.
Agencies supporting flexible planning principles exist (Agile and SCRUM
[26]), while allowing you to shoot high-quality analytics (such as Asana [33],
eg). At the same time, the top management of the company gets a special,
a guiding and inspiring role; much attention needs to be paid
continuous training of staff, especially project managers, to which
special requirements.
Conclusion: a flexible planning system with integrated performance monitoring
accept, but completely reorient the company to "confederation"
project teams -- a risky strategy.
2. Is it possible to make departments remain at the core of the structure, but their
experts would take part in the discussion of projects not occasionally, but
constantly, at all stages of the project with a wide range of
specialists involved?
Here came the idea of ​​creating a “managerial club” within the company, members
which meets at least 1 time per week and discusses projects in
development in the company. In addition, “design circles” have been created that are not found
less than 1 time per week, in which all specialists involved in
this particular project.
Management Club and Project Clubs are supported by flexible methods
planning and related tools.
This solution is partially implemented; full implementation in the company's practice
planned for 2018.
Transformations in the company for 2018:
• We create a unified system for assessing the success and contribution of the project and
resource allocation.
• We introduce a flexible methodology in which company employees participate in
project level.
• The previously adopted structure of the company, based on which unit-
units from which resources are drawn into projects.
• A set of measures is being developed, according to which, employees will
involved in the product life cycle.
• Developing practices of the “Management Club”.
• “Project circles” are being created.
Page 59
59
APPENDIX 4
A task.
Find a market way to get the company out of the crisis (company “X” takes
about 10\% of the Russian market, while being the second producer in the Russian Federation by market share).
The crisis is generated by the actions of a more powerful competitor, occupying a leading
position in the Russian market and owning a share of more than 50\% of the market.
The main objective of the project (after conducting a preliminary analysis): how
increase the number of branches and managers without resorting to attracting
a significant amount of credit?
The main results.
Found an organizational solution that will allow the company to successfully
compete with a more powerful target competitor without borrowing
cash, as well as without changing the main product line, that is
exclusively in a market way, which was the main requirement of the customer.
Introduction
Company "X" is a manufacturer of concrete additives, takes 2nd place in
market share of the Russian Federation, occupying a little more than 10\% of the Russian market. Main competitor
company X occupies more than 50\% of the Russian market share, and its share continues to grow,
and the market share of company “X”, on the contrary, shows a steady decline.
Market analysis of concrete products and ready-mixed concrete in Russia, as well as volume analysis
cement production in the Russian Federation (indirect indicator) shows that sales decline
company "X" in the last two years can not be caused by the fall of the target market
customers -- manufacturers of ready-mixed concrete and reinforced concrete products (see market analysis). As
source data of the analysis used data obtained by the analytical department
company "X" (see. Fig. 19).
An analysis of the products of company “X” and its main competitor revealed a complete
similarities in the composition and quality of products of these two firms, and the survey data
target customer groups show even slight excess
product quality of the company "X" in comparison with the product of the main competitor.
It should be noted that there is a promising direction in the market of concrete additives
- polycarboxylates. These are more advanced supplements that allow for much
lower concentration give more stable properties of concrete. Manufacturers
such additives are foreign companies. But neither company "X" nor its main competitor
does not produce such additives, moreover, their market is still small, despite
undoubted perspective.
Product comparison data provided by the analytical department
company "X", a survey of target customer groups conducted by us.
Product cost analysis also showed full compliance with the proposal.
company "X" and its main competitor.
Page 60
60
Thus, by exclusion, we can conclude that the main
the difference is in customer service , which means we are dealing not with technical, but with
organizational and management task.
Since we are talking about the B2B market, then studying market promotion, first
the turn should be paid to sales managers [27]. Study
in the form of a customer survey showed that the average qualifications of managers of both
companies are at the same level. At the same time, customer focus
managers “X” marked by customers as much higher compared to
the main competitor, while the share of company “X” continues to fall amid growth
market share of a key competitor.
In quantitative terms, there is a huge difference -- the number
managers in company X is about 6 times lower than the number of managers in
company "Y". The situation is similar with the number of branches -- the number of branches
Company X is 5 times lower than its main competitor. Needless to say
that the management of company "X" is well aware of this quantitative
dissonance, however, only after this comparative analysis became
it is clear that the size of the company, and therefore a multiple difference in financial
opportunities -- the only reason for the stable loss of the market by the company "X" on
the background of the growing market at the time of solving this problem .
Maintain a larger staff of managers and branches does not seem
possible -- not enough money. Financial Opportunities for X and Its
main competitor is almost an order of magnitude different.
Hence the main objective of the project: how to increase the number of
branches and managers, without resorting to attract a significant amount
loan funds ?
Subtask: it is desirable to make sure that the competitor could not copy
this decision .
Immersion in the task. MARKET ANALYSIS 2007 -- 2012.
Charts are based on data provided by company analysts.
"X."
Page 61
61
Fig. 19. Analysis of the Russian market of cement and ready-mixed concrete.
As can be seen from the above diagrams, the market for ready-mixed concrete and cement after
overcoming the crisis of 2008 -- 2009 growing steadily.
Analysis of the sales of company “X” and its most dangerous competitor 2007 -
2012.
According to the end of 2012, in the Russian market of additives for production
ready-mixed concrete the following situation has developed: the main market share is
the main competitor of the company “X” (it owns more than 50\% of this market),
the second place is occupied by company “X” (about 10\%) of the market, followed by other players.
The diagrams are removed from the example for obvious reasons. The diagrams show that
Y sales are growing amid a growing market for precast concrete and ready-mixed concrete.
Sales of the company “X” are falling against the backdrop of a growing market for precast concrete and ready-mixed concrete, and
also amid growing sales of the main competitor. Gradually picture
develops. But while there is no clarity -- is it the sales or the product itself?
Analysis of the product offer of company "X"
A comparative analysis of the product is not given here in view of its significant
volume.
Comparative analysis performed by specialists of the company “X” and a survey
customers, showed full compliance with the products of the company "X" and its main
competitor, and the survey revealed that the quality of the product of company "X" is negligible
superior to the quality of the product of its main competitor in the parameter "dosage" -
that is, a number of products of the company "X" makes it possible to obtain the desired effect
at lower dosages.
The analysis shows that:
product quality indicators of both companies are on the same level
(the conclusion was made on the basis of a study by specialists of the company “X” and a survey
customers).
Page 62
62
the range of products is completely identical, moreover
promising additives -- polycarboxylates -- none of the studied
companies.
Key market indicators -- production of concrete products, ready-mixed concrete and cement
As of the end of 2012, they are showing steady growth.
The market share of the main competitor of the company is growing steadily.
The market share of company “X” is steadily declining.
As stated above, the only significant difference is
market activity of companies: the number of managers and branches of company “X”
less than 6 times than its main competitor.
The problem is that you need to align the number of branches and
managers between competitors, for which you need to increase staff and number
affiliates about 5 times. However, this is impossible due to lack of funds, and
gap in financial capabilities of company “X” and its main competitor
constantly growing.
A key contradiction arises between the need to increase the number of
branches and managers and the inability to do this due to lack of funds:
Fig. 20. A pair of TP in this task.
Working TP isolated from fig. 18: affiliates and managers should be
much to provide the required consumer coverage, but it will require the company
investing a significant amount of cash, which is unacceptable.
Comment: The branch in this task is not just representation
company. The peculiarity of this business is that most
consumers prefers to receive the supplement in liquid form, however, the properties
liquid additives are quickly lost, so each branch is a
a small production site with an additive dilution unit. One such
the node is able to reliably supply consumers in a radius of about 250 km, not
Page 63
63
more. Of course, one node is not enough -- you need to have staff
managers and technologists involved in the sale and subsequent implementation
additives in the enterprise customers. Therefore, each branch should be endowed
functions of production and product promotion. It turns out to work with
consumers, we need branches with production functions and managers,
carrying out the function of sale. But to keep up with a competitor, you need
significant cash .
Operational Zone (OZ). The conflict area in this task is between
a large number of branches and managers of the company "X" and a small number
branches and managers of the company "Y".
In fact, a conflict arises between great financial opportunities
company “Y” and small opportunities of company “X”, processing the market
(of the same consumers).
Fig. 21. The operational area (OZ).
OZ Resources :
Tool: market supply of companies (preliminary analysis
revealed the identity of the proposals of both companies)
Product: final consumers (precast concrete plants and manufacturers of commodity
concrete), traders (resellers).
Let's try to transfer the branch function to other elements of the system, first
the queue -- to the health resources.
RBI-1 to OZ resources: The final consumer carries the functions
branch.
The customer has already tried this strategy -- but enterprises with
a competitor’s branch, they are not going to bear the costs for the performance of this function -
Largely due to this market strategy, business X collapses, although products
Company Y
Company X
Final
Consumers
Traders
Page 64
64
identical and even somewhat superior in stability to the properties of a competitor, about which
said above. This is a dead end.
RBI-1: The trader himself carries the functions of a branch. The customer objects to
of this formulation of RBIs: a trader is only a reseller, a reseller. He doesn't care
what to make money on. Nevertheless, we recorded that a trader is a possible
source of funding, however, like the plant. But there is a difference between them -
the trader needs to do business in the region with numerous customers, and the plant -
have the most convenient supplier.
RBI-2: Traders, working on large logistics sites throughout
Territories of the Russian Federation themselves ensure the impossibility of further development of a competitor
company "X".
Then the problem arose in a new formulation: how to make it so that
more Y will open branches and the more hiring managers, the company X
will be better? -- that is, at this stage the acceptance of TP authorization “surfaced”
No. 22 -- to turn harm in favor.
In general, such a statement of the problem removes the fear that the competitor will repeat
the decision of the company "X". After all, the more it repeats, the more it will "stoke"
by myself.
From the course of strategic marketing it is known that such a task
is interpreted by the expression "you need to find a competitor’s weakness in his strength", i.e. what
need to do something that Y can't repeat? (in other words, cannot
abandon his own strategy and therefore will "drown" himself).
And here the RBI “works”: the trader himself carries the functions of a branch .
Solution Idea:
You need to turn the trader (reseller) into a full-fledged distributor,
for whom technological customer support is part of his business, and
creation of a logistics platform -- an object of investment of a partner of company “X” with
predicted payback period.
To implement this solution, company X will need to guarantee
exclusive to one or two partners in the specified territory for the billing period
time (the question now comes down to calculating the payback period plus -- the time when
distributor makes a profit; this period is, after all, a subject
negotiations of the parties);
Provide training to distributor technologists, transfer to distributor
technology to work with end customers, to provide an opportunity to represent interests
enterprises in a given territory (franchise).
Verification of the solution found.
Is the original contradiction allowed? Yes. Even a superficial analysis
shows that funds are required several times less than for opening
own branches, while the number of managers at the traders is enough for
Page 65
65
elaboration of category B and part C clients. Category A clients may
worked out by the company "X" with the subsequent transfer to the trader.
Checking the decision on the impossibility of repetition by the main competitor:
The competitor has already opened more than 30 branches -- in all significant regions. For
repeating the strategy he will either have to close the branches in which they are invested
funds, or compete with your own distributors in
regions! Distributors will not accept such conditions. Therefore, the implementation of this
a strategy for a competitor of company "X" will be difficult -- this is the weakness
competitor, concluded in his strength. The decision was received completely on organizational-
managerial level, as required by the customer at the stage of setting the task.
(!) It is interesting that before solving the problem using TRIZ company
periodically raised the task of developing traders in order to increase
sales. However, the company's specialists have not previously seen a tool in traders
resolving this contradiction and did not try to develop a strategy
development of a full distribution network. Just earlier management
the company did not set a task like that .
Page 66
66
APPENDIX 5
Cross-cutting case demonstrating joint use
schematization and work with contradictions on the proposed
authored circuits .
A system consisting of:
sales department of an industrial enterprise,
manufacturing tooling from heat-resistant steels, presented
head of sales, sales staff and the current system
sales (Fig. 12).
The essence of the problem: the head of sales (ROP) implements a new sales system,
having advantages over the previous one in terms of depth of study
customers and, as a result, allowing to increase the average amount of the contract and
conversion, however, managers resist and are in no hurry to leave the “beaten”
rails. "
Required: to make managers use only the tools of the new system
sales in their activities (as the task sounded in the original formulation, that in
the process of analyzing the circuit turned out to be not quite the correct goal setting -- see table).
Below, we compose a model of a functioning system, presented in the form of a diagram.
The process of constructing a circuit for this task is described above (see explanations for Fig. 6):
Fig. 12 (repetition). Scheme of an inventive situation in the problem of changing the system
sales.
Page 67
67
The tasks set according to the scheme (Fig. 12) using the categories of schematization:
No.
Object of analysis in
IFS
Tasks
No.
A task
one
interaction
the system
(dotted line) with
elements
supersystems
1.1.
CRM system. The conflict arose largely due to the fact that the existing CRM system
not adapted to the requirements of the new sales system, which creates significant
inconvenience → how to make the CRM system meet the requirements
new sales system and supported it?
1.2
end-to-end business processes. The new sales system is changing end-to-end business processes,
this is especially true in collaboration with the design department and
production → you need to configure end-to-end processes so that
The requirements of the new sales system were provided .
1.3.
customers. The new sales system increases the time to contact
by the customer → how to make the depth of the customers' work increase without
increase time managers?
2
Layers
2.1
The implemented sales system manages the actions of managers, imposing on them
specific requirements → How to make sales system requirements
performed, but did the managers spend as little effort as possible?
2.2
Managers are faced with the fact that for a number of clients the requirements of the new system
redundant, which does not increase, but rather reduces efficiency ( from this point of view
managers "manage" the reaction of customers, hence the distribution of layers on
scheme ) → Differentiate customers and introduce a new sales system
only in relation to such client groups in which increase is expected
conversion and average transaction weight when using this system .
3
Communications
Partially analyzed in paragraphs 1 and 2, additionally:
3.1
Logical conflict between two systems, for example, the approach is completely changing
to identify needs, the stages of the transaction are radically different →
Compare the requirements of the existing and new systems, identify areas
similarities and cardinal discrepancies, disassemble into elementary steps of the area
cardinal differences, thereby simplifying the implementation (such a statement of the problem
allows the solver to rely on existing resources).
3.2
Communication defects ROP managers → Define metrics and reference points in
the new sales system, which should be feedback from
manager to leader. Simplify data retrieval by managers
reference points.
3.3
Establish a relationship CRM-system -- ROP → Having solved the tasks 3.2, bring the CRM-system in
in accordance with the decisions received, make appropriate changes to
order of meetings, strengthening communication on reference points and
reducing communication on irrelevant moments.
four
Processes and
the functions
4.1
The task appeared after setting the task 1.2: to conduct a detailed analysis of business
processes between the sales department and the design department, as well as between
sales department and production department (pre-mapping
processes using BPMN notation) . Highlight bottlenecks and set targets for
to overcome them.
4.2.
After solving task 1.1, set the task to simplify the introduction of the required
data into the CRM system by entering patterns and rules.
5
Groups
5.1
Negative phenomena within a group of managers -- the effect of the adoption of new
technologies according to the model of J. Moore → how to use innovators and early
followers as a resource to introduce a new sales system? how
to identify and neutralize the influence of “farther”?
5.2.
Customer groups, which follows from the analysis of the task 2.2. Carry out customer separation
on categories A, B and C. Define customer categories and target customer groups,
for which the new sales system is redundant. Set a task for synchronization
the work of the department, which should apply both sales systems, if a hypothesis
will be confirmed and the existing sales system will be advisable to maintain
for certain groups of customers amid the introduction of a new one.
6
Generalized
object and
filling
6.1.
Provide training for the “good middle peasants” of the new sales system after the decision
tasks from pp 1-5 and determine whether they have reached the level of "stars" after a given time.
If not, conduct a comparative analysis of the work of those and others and conduct further training
“Good middle peasants” according to the performance model (performance model
explains exactly what competencies make stars stars by comparing them
competencies with the competencies of "good average" in the team and identifying
discrepancies) .
Page 68
68
Next, we divide the subtasks into elementary actions that are necessary
implement on the project. If necessary, we indicate the tools that should be used in
further process this task. Also create unwanted effects
and contradictions, if any, arise during the course of the project.
No.
ass
aci
The task
No.
ass
en
and
I am
Task for execution /
further analysis
Secondary NE /
TP (draft)
1.1
How to make CRM-
the system matched
new system requirements
sales and supported her?
1.1.1
Carry out ABC analysis and describe portraits
customers by category in each channel
1.1.2. Organize a multi-funnel in CRM (process
and resultant) in money and quantities
NE: Manager mistakes during
funnel selection
1.1.3
Provide the ability to assign to one
deal several counterparties indicating
status and affiliation of the transaction in the card
the client
1.1.4
Provide the ability to assign
categories to deal and counterparty
1.1.5
Customize process and result reports
funnel according to TK
1.2
End-to-end configuration required
processes so that
new system requirements
sales were secured
1.2.1
Describe existing business processes
between designated departments in
BPMN notations and highlight points
inconsistencies of the current process with
requirements of the new sales system (after
solutions to problem 1.2.3) *
1.3
How to make depth
customer research
increased without increasing
time-consuming managers?
1.3.1
Develop standard decision making schemes
for priority sales channels indicating
customer entry points and tactics
categories A and B
NE: With 5 priority
channels -- this is at least 10
circuits that still need
identify correctly.
1.3.2
Develop channel matrices
TP: many hypotheses
needs -- more chance
create UTP but difficult
keep in mind
(need to reproduce
quickly in the conversation)
1.3.3
Develop rules for providing bonuses
to clients
TP: many bonuses –large
probability of getting into
need but choose
complicated.
1.3.4
Develop a base of typical questions at work
at different stages of the funnel
TP: a lot of questions -- more
chance to create UTP but quickly
we will tire the client
2.1
How to make
sales system requirements
performed but managers
spent as little as possible
effort?
2.1.1
Provide CRM system with tips
2.1.2 CRM system itself pulls data from
customer portraits by average
2.1.3
Provide a description of the business process
manager for each stage
process and result funnel, highlight
areas of greatest time loss and
set tasks to eliminate them.
2.2
Differentiate
customers and introduce a new
sales system only
relation to such client
groups expected
increase conversion and
average deal weight at
application of this system
2.2.1
Retain existing sales system for
category C customers. Introduce a new system
sales only for customers of categories A and B.
TP: a contradiction in the choice
sales department work schemes
- both systems are implemented
all employees, or
differentiation is carried out?
Page 69
69
3.1
Compare requirements
existing and new systems,
identify areas of similarity and
cardinal discrepancy,
disassemble on
elementary area steps
cardinal differences
simplifying implementation
Partially solved in solving problems 1.3.2 -
1.3.4
3.1.1
The difference in actions at the stage of evaluating options:
create a list of typical selection criteria in
channels for target centers
decision-making (supplement the scheme with
solving problem 1.3.1)
NE: Managers forget
perform actions at the stage
evaluation options let
gravity process
3.1.2
Difference of actions: a new item was added -
economic justification. Give examples
(cases) payback calculations that
manager can use in preparation
commercial offers.
3.1.3
Stage work with objections replaced by stage
"Resolution of doubt", which causes
difficulties. Describe the background
doubt, manifestation
doubt, train to work with doubt.
NE: Managers often
miss occurrence
doubt centers acceptance
decisions even if
have been trained.
3.2
Define metrics and
reference points in the new
sales system by which
reverse should be carried out
communication from manager to
to the leader. Simplify
receiving data
reference managers
points.
3.2.1
Develop quantitative indicators in
funnel in the process and resultant funnels
3.2.2
Develop quality indicators in
sales funnel
NE: Difficulty of control
quality indicators in
sales funnel
3.2.3. Suggest application control options
data to the CRM system by managers
daily ( e-commerce facilities in
no company )
3.2.4
To provide for the formation in the CRM system
Lead from email manager and site
companies provide automatic accounting
leads.
NE: The risk of appearing in the database
data of the same
client under different
names
3.3
Make appropriate
changes in order
meetings reinforcing
reference communication
points and reducing
communication on
irrelevant moments
3.3.1
Create report in CRM-system “movement in
funnel for 1 week "for each
manager
4.1
Conduct a detailed analysis
business processes between
sales department and
design department, and
also between sales and
production department
Included in task 1.2.1
4.2
Set a task to simplify
entering the required data into
CRM system by entering patterns and
regulations
Tasks 1.1.1 -- 1.1.5 and a number of other tasks
4.2.1
After completing tasks 1.1.1; 1.3.1-1.3.4
integrate these documents with
CRM system
5.1
How to use innovators and
early followers in
as a resource for implementation
new sales system? how
identify and
neutralize influence
“Bumpier”?
5.1.1
Create a report in the CRM system for use
its capabilities (which entities are actively
are used).
5.1.2
Provide a correlation report
Entity Use -- Conversion -
average transaction weight -- gross margin per month
(use data when conducting
meetings)
5.2
Set a task for
department work synchronization,
which should apply both
sales systems if hypothesis
confirmed and existing
the sales system will be
advisable to save for
Task 2.2.1
Page 70
70
specific customer groups
amid the introduction of a new
6.1
Provide training for “good
middle peasants "new system
sales after solving problems from
pp 1-5 and determine if achieved
whether they are the level of "stars" through
set time. If not,
conduct a comparative analysis
the work of both and spend
retraining “good
middle peasants "according to the model
performance
6.1.1
Create a performance model indicating
weighting factors on
positive deal outcome
TP: When using
no theoretical data
takes into account the specifics
enterprises but accounted for
new system requirements
sales, and if you take the data
by the enterprise then they
match the specifics, but
accumulated by
the requirements of the previous
sales systems
6.1.2
Observe the work of the "stars" and
refine the performance model
TP: If you watch for a long time, then
get an adequate model
performance but learn by
it doesn’t work right away. BUT
need to be taught immediately, otherwise
the system will not work.
6.1.3
Create a training program /
mentoring using data
obtained in the course of solving tasks 6.1.1 -- 6.1.2
TP: mentoring program
must implement
best practice staff
but at the same time they come off
from core business
* in bold the table indicates tasks that require further analytical
work.
And there are 28 tasks for execution, of which 26 can be put to execution,
if you remove the selected unwanted effects (6) and resolve the contradictions,
formed at this stage in draft form (7), after which you can
finally formulate SMART execution tasks [14] and implement
planning, for example, using SCRUM technology [26]. That is, proceed to
"Managerial" part of the project.
2 more tasks require further analysis, for example, using notations
descriptions of business processes [34].
We highlight the contradictions in graphical form that we encountered (8 contradictions
arose at the stage of formulating tasks for execution, another 7 are secondary
unwanted effects that occur when trying to fulfill the set
tasks (transferring the remedy to the system). Execution tasks not containing
contradictions, are not included in the table, they will be fixed on the task board [26] and go
to work.
If the resolution of the contradiction is obvious, write it to the right in the table. If not, then
We expose the selected TPs to further analysis.
Tasks 1.2.1 and 2.1.3 will be further analyzed, as indicated above,
after which TRIZ tools can also be applied to them.
Page 71
71
No.
TP in graphical form
Authorization / Next Steps
1.1.2
-
one
+
When setting the trade category marker “C”
in the CRM system it is automatically offered
process funnel if marker is installed
“A” and “B” are the result.
Requirements taken into account
two systems
of sales
Funnel in CRM
Errors in
choosing
funnels
+
2
-
1.3.1
-
universal
+
It is solved when performing task 2.1.2, so
how typical schemes are provided
manager in the form of a hint CRM system
when setting a deal marker and channel
sales. Solution: make the channel marker
sales "active field.
Accuracy of description
the situation
Decision making schemes
Complexity
identification
scheme
+
Broken by channels and
categories
-
1.3.2
+
A lot of
-
Break down hypotheses by sales channels and
decision centers. In preparation
to the meeting use only the target
matrix, pre-selecting ~ 5
key hypotheses (empirical figure).
The target matrix is ​​output by analogy with
solved problem 1.3.1
Brighter and
reasoned
USP (unique
trade
sentence)
Hypotheses of needs
the client
Play
from memory during
dialogue with
by customer
-
Few
+
1.3.3
+
Many options
-
Bonus classification matrix: row by row
- classification by groups (financial,
logistics, consulting ...), by
columns -- classification by authority:
Manager, Head of Sales
CEO.
Probability
to get in
need
Bonuses to customers
Complexity
of choice
_
Few options
+
1.3.4
+
A lot of
-
The contradiction is resolved by mentoring -
you need not to remember questions, but to learn
design them quickly with
using typical hypotheses
needs and criteria.
Brighter and
reasoned
USP (unique
trade
sentence)
Customer questions
Play
from memory during
dialogue with
by customer
-
Few
+
2.2.1
+
Differentiated by
sales patterns
-
We try to solve through analysis
TOC contradictions [12]
Optimal
using
qualifications
Department staff
of sales
amount
supported by
standards
-
All employees
apply both schemes
+
3.1.1
-
Single stage of work with
needs
+
Set marker in CRM system,
indicating the duration of stay on
each stage of the funnel depending on
correlation of the stage with the category of the transaction. At
exceeding the limit of being at the stage
"Need recognition" marker
signals manager about high
the probability of a change in stages.
Localization
work with
issues and
criteria
of choice
Sales funnel
Ease
identification
stages
+
Work with needs
beaten in 2 stages
-
3.1.3
+
Stage "permissions
doubt "
-
Let's try to solve using RBI and
resources [11] (previously acted on a hunch,
need to find reliable and easy
decision to identify doubts
Decision Centers)
Conformity
project realities
of sales
Sales funnel
Complexity
identification
-
Stage "work with
objections
+
3.2.2
-
Only quantitative
+
Bring quality indicators to
clear digital form (result -
identified at least 3 needs
leading to benefits not highlighted
less than 3 criteria leading to
benefits, etc.). Provide for
CRM-system markers, allowing to conduct
counting such indicators and display them in
special field in the transaction passport.
Adjustment
transaction movements
→ conversion
Funnel indicators
of sales
Complexity
extract
of information
+
Quantitative and
quality
-
Page 72
72
3.2.4
-
Manually
+
Lida approves in the CRM system only
marketing assistant. Also
assistant distributes leads for
further elaboration → required to enter
in a CRM system, a marker for the appearance of a lead,
notifying employee to whom
attached lead assistant department
marketing.
Lead Preservation
(guarantee that lead
not lost)
Leading in CRM-
the system
Possibility
appearance in the database
defective
records
+
Using
automatic services
-
6.1.1
+
Prepared from
using
experience
-
We try to solve through analysis
TOC contradictions [12]
Using
available data
on application
the system
Model
performance
Conformity
programs
learning
requirements
new system
-
Prepared from scratch
+
6.1.2
+
Based on
current understanding
-
We try to solve through analysis
TOC contradictions [12]
Preparation time
models (launch in
work)
Model
performance
Efficiency
learning
(impact
strictly to the point
growth)
-
Based on
form matching
the behavior of the middle peasants
with the stars
+
6.1.3
+
"Serednyachki" with experience
from 1 year
-
We try to solve through analysis
TOC contradictions [12]
Time of the best
employees on
performance
main
the activities
Who is conducting
training / mentoring
"Concentrate"
practical
skills
-
Best practics
+
So, straight away we managed to find solutions to 8 contradictions out of 13. The remaining 5 contradictions
We will allow by other means.
Resolution of the contradiction 2.2.1 (solved through analysis of TOC):
Fig. 22. Analysis of a pair of TP 2.2.1
Differentiated by
sales patterns
Number of supported
standards
Department staff
of sales
Apply both schemes
(no differentiation)
Optimal use
employee qualifications
+
-
+
-
Qualified transition does not occur
less skilled employees
work and vice versa
You need to have only one circuit
training new employees
Page 73
73
The branch “how to make employees differentiated by sales technology,
but at the same time we had only one scheme of employee training ”can be seen as
promising.
Solution : for the duration of the internship, the employee must be trained according to the program of work with
category C customers, and then, upon detection of potential, it can be transferred
to work with category A and B clients after retraining. In doing so, we
we get a single training system for sales staff, but differentiation by
sales patterns saved.
Super effect : gradual manifestation of employee potential, additional
motivation by complicating tasks with increasing income, reducing the number of unsuccessful
personnel decisions.
Resolution 3.1.3 (use resources and RBIs):
Fig. 23. The choice of working TP from a pair of TP 3.1.3.
Resources:
Stage of “resolving doubts”
(tool)
The process of identifying an “item”
processing (doubt) "- product
Approach to the contract
Personal communication with the adoption center
solutions
New large customer
Information from other stakeholders
Characteristics of the personality of the adoption center
solutions
Experience in similar transactions
Top Management Attention
Behavior Center Analysis
solutions
RBI rule:
The X-element itself provides an unmistakable identification of customer doubts (item
processing), significantly affecting the outcome of the transaction.
As a result of the substitution of resources in the RBI formula, we obtain the following:
1. When approaching the conclusion of a contract in the case of work with large transactions
the head of sales (ROP) takes the transaction under personal control;
Includes Stage
“Resolution of doubt”
Identification difficulty
“Subject of processing” (doubts
hard to detect)
Sales funnel
Includes the “work with
objections
Compliance with realities
project sales
+
-
+
-
Page 74
74
2. If the client has not worked with us before, a major transaction is planned,
we enter information about the personal qualities of the counterparty in the contact passport -- type
individuals by DISC topology (or use the enneatype model by topology
Adizes-Madanes), interests, features of behavior. For briefing
managers detailed cases, examples are shown.
3. If the transaction is large, then at the stage of recognition of needs we study in detail
requirements of decision makers on this transaction. Special attention
we turn to this point if the transaction is significantly larger than those that were made with
this client earlier or the client switched from a competitor. Further, when
promoting the transaction, we fix deviations from the above requirements identified
deviations and will be reliable indicators of doubt.
4. If personal communication with the decision center is interrupted for a while
for objective reasons (the transaction goes to the stage where work
carried out with other decision centers (DPC), periodically
conduct personal communication with the CPR under various pretexts. Periodicity
contacts -- at least 1 time per month.
5. Maintain information with persons influencing the decision and record
deviations from the current trajectory. In case of detection
significant deviations come in contact with the DPC.
6. Use the experience of similar transactions -- on an extended monthly basis.
retrospective analysis of sales
to study the issue of identifying doubts arising in the most
material transactions.
Resolution of the contradiction 6.1.1.
A cursory analysis of the contradiction led to the understanding that express analysis by TOC is unlikely
will give a quality result. Therefore, we will resolve this contradiction with
allocation of resources of the operational area and the application of RBI.
Fig. 24. The choice of working TP from a pair of TP 6.1.1
Prepared from
using
experience gained
Program compliance
learning the requirements of the new
the system
Model
performance
Prepared “from a clean
sheet "
Using accumulated
statistics
+
-
+
-
Page 75
75
Resources:
Performance Model Preparation
“From scratch” -- a tool
Using accumulated statistics
- product
Specification of behaviors
overlapping with existing
sales system
Accumulated Cases
Unique Form Specification
conduct
Cross-channel conversion data
of sales
Manager Reports
Documented Results
meetings
RBI rule:
X-element itself ensures the use of accumulated statistics in the model
performance designed for the new sales system.
As a result of the substitution of resources in the RBI formula, we obtain the following:
1. Describe behaviors in the performance model in relation to the new
sales system and highlight those forms of behavior that are similar to the previous
system. Use existing cases and accumulated experience when working out
“Intersecting” forms of behavior;
2. Analyze in channels where the conversion was above the average for transactions
categories A and B. Use the experience in these transactions as a reference
to verify the adequacy of the performance model created for the new
sales systems.
3. To analyze in channels where the conversion was below the average for category transactions
A and B. Put a thought experiment in the application of the performance model for
a new sales system for such transactions. Mark those forms
Behaviors experienced by experienced staff could correct the situation.
When training employees, pay special attention to the highlighted forms
behavior.
4. To check the adequacy of the conclusions in paragraphs 2 and 3, use the reports of managers and
meeting minutes created during the development of analyzed transactions.
5. Use data from manager reports and meeting minutes to
case studies when teaching selected forms of behavior (if the form
new behavior, then as a case, you can bring a situation where the presence
this form of behavior could have a qualitative impact on successful
transaction outcome).
Page 76
76
Resolution of the contradiction 6.1.2.
Fig. 25. Analysis of TP 6.1.2 pair and application of RBIs.
The application of RBI suggests itself:
RBI 6.1.2-1: the X-element itself ensures the selection of forms that are “not enough” by the middle peasants
behavior without a comparative analysis of the work of the "average" and "stars".
Solution : analysis of the situation by the "stars" themselves: if the "stars" hold a series of meetings
together with the "middle peasants", having on hand a common list of forms of behavior, then spending
subsequent reflection with the "stars" is easy to catch the difference. This operation is not
will require more than 1 week.
Based on
current understanding
Learning Effectiveness
(we act strictly on
growth points)
Model
performance
Based on
job comparisons
"Average" and "stars"
Model preparation time
+
-
+
-
Do not spend time on comparative
analysis of the work of the "average" and "stars" by
behaviors
We are not wasting development efforts
behaviors that
"Average" and so good
reproduce
RBI
Page 77
77
Resolution of the contradiction 6.1.3.
Fig. 26. Analysis of TP pair 6.1.3.
Both branches are recognized as dead ends, no solution found.
We try through resources and RBI:
Fig. 27. The choice of working TP from a pair of TP 6.1.3
Serednyachki with experience from
1 year
Practical concentrate
skills
Who is conducting
training
Employees with experience from 5
years having high
results
Time best employees on
basic tasks
+
-
+
-
Top employees excluded from the process
mentoring
Already shallow all that does not give
steady result
"Serednyachki" with experience from
1 year
Practical concentrate
skills
Who is conducting
training
Employees with experience from 5
years having high
results
Time best employees on
basic tasks
+
-
+
-
Page 78
78
Resources:
Experienced staff → mentors -
tool
Time
experienced
employees
on
basic tasks -- product
Project Sales Experience
Meeting Planning Time
Company product knowledge
Time to analyze transaction information
Accumulated customer base
Time to contact customer
Fame in the professional environment,
reputation
Time to prepare reports in CRM-
the system
Expertise in the client’s business
Time for lunch, rest during working hours
of the day
Psychological competence
RBI rule:
The X-element itself ensures that the best employees do not spend time on
training / mentoring.
As a result of substituting resources into the RBI formula, we obtain the following (we see that with
the RBI data better manages the resource of the product):
1. Adaptation of a new employee is divided into 2 parts: introductory course new employee
passes with the employee who conducts the initial introduction to the course of affairs and
answers some of the beginner's questions. The introductory course is conducted by an employee who has
1 year work experience and stable sales results. After passing
Introductory course, the new employee falls into the second stage of adaptation.
2. The second stage of adaptation is the “shadow”. That's what technology is called when new
the employee is trying to copy the actions of the wizard, in the course of playing it
action he is faced with a lot of obscure moments that
should fix in the form of questions. Then prepared questions new
the employee asks the mentor. The same technology can be used during
contacting an experienced employee with a client.
3. This is where the practice entrenched in a number of companies comes from -- a progress diary
new employee. That is, the new employee does not record anywhere, but in
paper or electronic workbook with pre-prepared fields. So
information is better structured and amenable to subsequent analysis.
4. After successfully completing the second part of the adaptation, an experienced employee
allocates time in his schedule for “polishing” the skills of a new employee in
the process of monitoring its preparation for responsible meetings and contacts with
by the customer.
Those. to completely remove experienced employees from the mentoring process is irrational,
but it is possible to reduce their time in the process of mentoring by 2-3 times without reducing
process efficiency. The contradiction is partially resolved.
Based on the obtained solutions, the task manager developed an implementation program
new sales system, including a step-by-step action plan drawn up in the environment
www.trello.com using SCRUM technology [26]. Currently going
the process of implementing this program of activities .
Page 79
79
APPENDIX 6
General algorithm of work with organizational and managerial
tasks indicating the most commonly used tools.
When solving organizational and managerial tasks, we applied the most
various tools from the TRIZ arsenal [11]. Some tools from the arsenal
TRIZ are used in organizational and management tasks without any
methodological completion, and some had to be converted to requirements
“Soft” systems, primarily, for the requirements of business systems [2, 11].
In fig. 28 presents a simplified scheme of using TRIZ for solving business
tasks used by the author [46]. To describe the groups of tools, we introduce two important
concepts: product and result [10]. These terms are widely used in modern
management. A product is something that can be transferred to the next stage of work with a task,
that is, the "exit" of the instrument. The result is a refined understanding of the system or
process (analysis result).
Most often, such a scheme is used not so much to solve situational problems,
how much when performing projects in business systems:
Fig. 28. General scheme for the use of TRIZ tools in the implementation of projects in
organizational systems.
Page 80
80
So, the work on the task consists of five blocks (Fig. 28):
• Formulation and formalization of the task;
• Initial processing of the task;
• Highlighting a list of key contradictions;
• Decisive mechanisms;
• Verification of the decisions received.
We will reveal these mechanisms in more detail.
Formulation and formalization of the problem (steps 1–5, Fig. 28).
Process: communication in a team of professionals looking for a solution, designed
to understand the conditions of the problem.
The result: an understanding of the system’s design — what are the key elements
the composition of the system and how the most significant relationships between them are organized, as well as
which trajectories (step 5) should further transform the system.
Product: a set of unwanted effects that make up the problematic
situation (step 4) and the trajectories of further work with the task (step 5).
Initial processing of the task (steps 6–10, Fig. 28).
In fig. 1 not all the tools that we use in
TRIZ, but only the most commonly used.
Process: we analyze the cause-effect relationships, parameters and
structure of the system, determine the principles of its work.
Result: a deeper understanding of the connections in the system, a description is made
elements and their key parameters.
Product: key unwanted effects, secondary tasks, solution ideas.
Some ideas might appear at the previous stage, but when using
there are much more tools from this block.
Highlighting a list of key contradictions (step 11, Fig. 28).
Process: we determine the parameters that conflict with each other.
We distinguish secondary unwanted effects associated with the opposition of the system.
We are building contradictions.
Result: sharpening the task to the limit, which allows you to see the nodal points
tasks, to cut off all unnecessary -- “white noise”, which is always in large quantities
accompanies organizational and managerial tasks.
Product: list of contradictions.
Decisive mechanisms (steps 12–13, Fig. 28).
There are several such mechanisms, most often we use two of them.
(steps 12 and 13).
Page 81
81
Note that the process of selecting an operational zone and determining resources in
within the operational area is part of step 12.
Process: using conflict resolution algorithms, we find ideas for
solutions to the problem.
The result and the product here coincide: ideas for solving the problem and tasks of the third
order. Such problems arise if the solutions found require further analysis.
(then return to steps 3, 7-10) or encounter significant resistance
system (then go to step 11 and resolve new contradictions).
Verification of the obtained solutions (step 15, Fig. 28).
Process: we analyze the received decisions, we compare them with the goals of customers
and key stakeholders. We pass a verdict: are solutions suitable for
introductions? If the answer is yes, then we proceed to the verification of the ratio of benefits
to the costs. If the benefits are incomparably greater than the cost of solving this problem (which
it happens, of course, not always), then we proceed to the planning of implementation, if not -
we return to the beginning of the algorithm and highlight additional tasks.
Result: understanding the next steps: we move on to planning implementation
or set new tasks.
Product: solutions and secondary tasks.
Now a little about the "parking" (step 14). "Parking" we call the place where
“Stored” ideas found in the process of working on the task (of course, parking is
not our invention, G.S. Altshuller suggested found solutions to mark as
"GI" is a brilliant idea). Ideas can come up at various stages of processing.
tasks from step 3 to step 15 inclusive. It’s important to park in time and
divided into categories so as not to lose and subsequently translate them into an action plan
on the implementation of ideas received during the implementation of the project.
In fig. 28 shows a general scheme for using TRIZ tools in projects,
implemented in organizational systems. Of course, in practice it’s not necessary
All tools depicted in this diagram apply. Instruments
used selectively. So, in the example from Appendix 5, primary tools
task processing was not used at all, but decomposition was applied
secondary tasks obtained by analyzing the circuit in Fig. 12. The decision process
tasks remains largely a creative process and, unfortunately, is still not enough
formalized and largely dependent on experience, competencies and psychological
solver features.
Page 82
82
SOURCES
1. Sosnin E.A., Poyzner B.N. Social Engineering Workbook
(interdisciplinary project) -- Tomsk: Tomsk University Press,
2001.
2. V. Soushkov. TRIZ and Systematic Business Model Innovation, 2010. Bergamo
University Press. ISBN: 978-88-9633359-4
3. Technology of creative thinking / Mark Meerovich, Larisa Shragina. -- M .:
Alpina Business Books, 2008.
4. Korolev V.A. Fundamentals of the system-process theory of convenience and
the life of organizations. Manuscript deposited in TRIZ fund
CHOUNB.
5. Korolev V.A. Modeling of social objects, 2004. Manuscript
deposited in the TRIZ CHOUNB fund.
6. Examples of the application of TRIZ to organizational and managerial tasks:
http://bmtriz.ru/articles/categories/1/
7. Fundamentals of the general theory of strong thinking, Khomenko NN, 1997. Manuscript
deposited in the TRIZ CHOUNB fund.
8. Shmakov BV, Schepetov EG Compliance of technical and
social systems, Chelyabinsk State University.
9. D. Mann. Hands-on Systematic Innovation for Business and Management ", IFR
Press, 2004. ISBN 1898546738
10. Reus A.G., Zinchenko A.P. Guide to Organization Methodology,
Leadership and Management / Readings on the work of G.P. Shchedrovitsky
M .: Alpina Publisher, 2012.
11. TRIZ: Solving business problems / A. Kozhemyako. -- M .: Synergy University,
2017.
12. D. Mann. Physical Contradictions and Evaporating Clouds (Case Study
Applications of TRIZ and the Theory of Constraints), 2000
13. Adizes I.K. How to overcome management crises. Diagnostics and solution
managerial problems. -- M .: Mann, Ivanov and Ferber, 2014.
14. Use SMART goals to launch management by objectives plan:
https://www.techrepublic.com/article/use-smart-goals-to-launch-management-by-
objectives-plan /
15. TRIZ. Analysis of technical information and the generation of new ideas: educational
allowance / N.A. Shpakovsky. -- M.: Forum, 2010.
16. A. Kozhemyako. The era of smart sales in the B2B market. How to audit
commercial service on your own and break away from competitors, 2016
(electronic edition:
http://bmtriz.ru/anton_kozhemyako_era_umnyh_prodazh_kak_provesti_audit_ko
mmercheskoy_sluzhby_svoimi_silami_i_otorvatsya_ot_konkurentov /)
17. N. Feigenson. Advanced functional approach in TRIZ.
Report at the Scientific Conference “TRIZ. Application practice
methodological tools. " Moscow, 2017
18. Kudryavtsev A.V. TRIZ -- tools for creating innovation for development
enterprises. Textbook, 2013.
19. G.S. Altshuller, B.L. Zlotin et al. Search for new ideas: from insight to
technologies. -- Chisinau: Cartya Moldavanske, 1989.
Page 83
83
20. O. Cowan, E. Fedurko. Fundamentals of the theory of constraints. Library
strategic decisions of CBT, 2012. ISBN 978-9949-9148-1-4
21. Discovering the organization of the future / F. Lalu. -- M .: Mann, Ivanov and Ferber,
2016.
22. Yu.P. Salamatov, I.M. Kondrakov. Evolution model of technical systems,
1986. The manuscript was deposited in the TRIZ CHOUNB fund.
23. Management / Drucker Peter F., Macjarello Joseph A. -- LLC
"I.D. Williams", 2011.
24. Willpower. How to develop and strengthen / Kelly McGonigal; trans. from English Ksenia
Chistopol. -- 2nd ed. -- M .: Mann, Ivanov and Ferber, 2013.
25. Find an idea: Introduction to TRIZ -- the theory of solving inventive problems /
Heinrich Altshuller. -- 3rd ed. -- M .: Alpina Publishers, 2010.
26. Scrum. Revolutionary Project Management Method / Jeff Sutherland;
trans. from English M. Geskina -- M .: Mann, Ivanov and Ferber, 2016.
27. The era of smart sales in the b2b market / A.P. Kozhemyako. -- M.: Moscow
Synergy Financial and Industrial University, 2013.
28. The era of smart sales. Strategy and management / A.P. Kozhemyako. -- M .:
Moscow Financial and Industrial University "Synergy", 2013.
29. Psychological effects in management and marketing. 100+ directions
increasing efficiency in management / A.P. Kozhemyako. -- M.: Moscow
Synergy Financial and Industrial University, 2015.
30. Eighth skill: From efficiency to greatness / Stephen R. Covey; Per. from English
- 4th ed. -- M .: Alpina Publishers, 2009.
31. Abraham H. Maslow. Motivation and Personality (2nd ed.) NY: Harper & Row,
1970; St. Petersburg: Eurasia, 1999. Translation by A. M. Tatlybaeva.
32. Think like a designer. Design thinking for managers / J. Lidtka, T.
Ogilvy. -- M .: Mann, Ivanov and Ferber, 2015.
33. Asana: The easiest way to manage team projects and tasks:
https://asana.com/product.
34. Business process modeling notations:
http://www.businessstudio.ru/products/business_studio/notations/
35. Scribing: description and tools:http://nitforyou.com/scribe/
36. New appointment. Kuryshev V.A., 1994
37. Methodological materials for the 45th author’s seminar Khomenko N.N.
“Modern intellectual technologies based on TRIZ”, Minsk
1996.
38. Anti-fragility. How to capitalize on chaos / N. N. Taleb. -- M .:
Publishing Group "Alphabet-Atticus", 2014.
39. V. Sushkov. Ideality in business, 2015
40. V.G. Siberians. Invention in business or “development” through
Contradictions, 1999
41. G.P. Shchedrovitsky. Organizational thinking: ideology, methodology,
technology (lecture course). -- M.: Publishing house of the studio Artemy Lebedev,
2015 year
42. GLOSSARY OF TRIZ AND TRIZ-RELATED TERMS, VERSION 1.2. Valeri
Souchkov, The International TRIZ Association -- MATRIZ, 2018.
43. A.V. Efimov. Methodology of MPV analysis, 2008
44. M.S. Ruby. TRIZ in small business -- competitive handicap, 2004
Page 84
84
45. Yu.G. Chernikov. System analysis and operations research. -- M .:
Publishing House of Moscow State Mining University, 2006.
46. ​​A. Kozhemyako. A little bit about the systems thinking of the department head
sales. We apply system analysis. Sales Management, 03 (98),
2018 year
47. Altshuller G.S., ARIZ -- means victory -- Sat: Rules of the game without rules,
Petrozavodsk, Karelia, 1989, p. 17.
