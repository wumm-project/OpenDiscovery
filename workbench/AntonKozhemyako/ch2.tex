\chapter{Schematization of inventive situation}

\section{Objectives of the study}

Before a solver using TRIZ to solve organizational management tasks, the
question of a detailed clarification of inventive situation arizes. Practice
shows that usually the solver receives such tasks from task manager in an
insufficiently formalized form [11]. Simply put, we have dealing with “slogan”
formulations of the problem [40], albeit accompanied by “Digitized”
indicators:
\begin{itemize}
\item it is necessary to reduce the cost of our products by 10\% for 3 months;
\item reduce labor costs by 15\% for 6 months;
\item increase the influx of target leads [16] to 300 units. per week until
  the end of 2018
\end{itemize}
Etc.

It should be noted that with the problem of formalization of organizational
and managerial not only TRIZ experts faced the tasks. Actively dealing with
this problem The members of the Moscow Methodological Circle (MMK) were
engaged under the guidance of G.P.  Schedrovitsky [10], as a result of their
activities, a schematization technique appeared similar problem situations
[41], which is able to qualitatively cope with this problem, but does not have
system tools for further solution organizational and managerial tasks, which,
in turn, contains TRIZ.

\textbf{The objective of our research: to show the benefits of applying
  schematization to preliminary analysis of the inventive situation and
  develop a method application of schematization in conjunction with TRIZ
  tools.}  To consider existing TRIZ approaches to formalize the inventive
situation and show areas of similarity and areas of difference. Show the
benefits of schematization.  in relation to other inventive preliminary
analysis tools situation. Link schematization with TRIZ tools when deciding
organizational management tasks.

\section{Minimum Viable Business System}

Since the dissertation is devoted to tools for working with organizational
management tasks, which, as indicated, can be set task managers when trying to
improve an organized social system, it is necessary to better understand the
functioning of such systems. Consider typical elements of an organized social
system on the example of a business system.

We begin by defining the concept of the viability of a business system
developed by Valery Sushkov [2]:

\HGG{Fig. 1 Minimum viable business system.}

As can be seen from the diagram in Fig. 1, so that the business system is
minimally viable, it must contain a delivery processor, a generation processor
values, processor supplying value to the market. In addition, the viability of
the system supported by business processes and management functions (goal
setting and control of their implementation), carriers of these functions is
the personnel of the company, in features -- managers. \textbf{In business
  systems, people are the most significant subsystems and supersystems.} The
complexity is that a person is a system with a high degree of uncertainty (low
probability of predictability behavior), and therefore, the same parameter
value at the input to the system in depending on the specific state of the
system element can give a large spread value of output, it is often possible
to observe a result that is very far from the forecast.

Since sustainable business systems are supported first queue, the functions of
setting tasks and monitoring their implementation, and setting tasks requires
a clear vision from the manager [13], in business systems arises many
inventive situations, the resolution of which can clarify the vision manager.
It’s known that with many inventive tasks, a manager is not coping due to the
lack of a standard solution. In modern management it is estimated that more
than 90\% of managerial decisions are standardized [10], [23]. However
remaining unsolved tasks may reduce the effectiveness of business systems,
remaining unresolved years, as practice shows. The author faced similar tasks.
in the field of document management, staff recruitment and training, marketing
and sales, in areas of CRM systems ... The situation is aggravated by the fact
that it is not always easy to find and adapt decisions from generally accepted
practices -- organizational and managerial the task often needs to be closed
pretty quickly.

Until these “white spots” can be overcome and clarity appears design in the
mind of the manager, the function of setting tasks for execution and
controlling them Executions cannot be fully implemented. The business system
is located in constant movement, therefore such "white spots" in the
activities of modern managers arise with enviable frequency, and time to
overcome them less and less is given away from year to year. Similar tasks
called in this paper organizational and management, have good potential for
application TRIZ tools and approaches.

Since the purpose of this work is to study the features application of TRIZ to
the class of tasks set in business systems should be given attention to the
definition of "organizational and managerial tasks."  The author believes that
in solving this class of problems it is worth distinguishing between three
forms of activity that form the basis of management [41]:
\begin{itemize}
\item organization;
\item leadership;
\item management.
\end{itemize}

Organization is the process of forming supersystems and / or subsystems of
various level in business systems: association, organization, department,
workplace finally.  Organization is precisely the formation of a structure,
that is, elements and their interconnections.

Leadership (from the words to lead, lead by hands) -- this is the setting of
tasks performers and monitoring their implementation.

Management is a change in the activities of performers. That is, when the
structure organized, all tasks are set (including tasks for feedback), but we
are not satisfied with the performance of performers, we are trying to change
them activity in the direction of improvement, that is, we begin to manage to
manage them activity.

All three of these activities form what can ultimately be called "Activities
in business systems." Therefore, I propose to call such a class of tasks
"Organizational and managerial". \emph{Of course, from the standpoint of using
  TRIZ, we only interested in inventive situations that arise when solving
  organizational and management tasks.}  Further, at used term "Organizational
and managerial tasks", the requirement of inventive situations will be
implied.

However, before using TRIZ tools, an inventive situation should be as
previously analyzed. Similar analysis in organizational management tasks has
differences from pre-processing tasks, delivered in technical systems,
primarily because an element of the system is a man -- \emph{a complexly
  organized element that has its own goals}. Besides Moreover, the processes
going on in such systems are often quite confused, contain many difficult to
capture and dynamically changing relationships, therefore require a special
approach to identification and description.

Below we consider the approaches used both in TRIZ and in management
consulting that can be applied (and applied) to describe organizational
management tasks and their comparison with the method proposed by the author
descriptions of the inventive situation in organizational and managerial
tasks.

\section{Application of the SMART method to formalize organizational
  management tasks} 

SMART [14] (the extended formula “SMARTER” is also known) is a mnemonic
abbreviation for the principle of setting goals. This model is today the main
standard when setting tasks to subordinates, we can say that this The main
international standard in this field. According to SMART, the task should be
specific, measurable, attainable, meaningful (relevant), correlated with a
specific period (time-bounded). Perhaps this is the most world-famous way of
formalizing organizational and managerial tasks.

Examples of tasks set in accordance with the SMART model:
\begin{itemize}\it
\item[1)] 10\% of the employees are consistently late for work for 5-10
  minutes, another 3\% of the team are late for more than 15 minutes. “Stable
  Delays” are considered delays more than 5 times a month. Required to lower
  maximum stable lateness of employees up to 3, lateness of 15 minutes and
  above completely exclude.
\item[2)] It is required to increase sales of the company's product (dietary
  supplements from natural raw materials) 20\% until December 31, 2017. Growth
  through expansion of territories unacceptable, an increase should occur in
  existing territories.  Allowable increase in marketing budget -- 10\%.
\end{itemize}

This method of formalizing organizational and managerial tasks is
incommensurably better than the absence of any scheme, however, it is easy to
notice that such the statement of the problem is good for execution, in case
the performers have everything necessary resources to complete the task and if
the performers own ways to accomplish it (they know how to apply these
resources to achieve set goal). However, such a model is categorically
insufficient for formalization of the task in order to find the most effective
conceptual solutions if there is no standard solution to the problem. Hence,
the SMART model does not open up the possibility of applying TRIZ methods;
therefore, it does not can be considered as a method of preliminary
formalization of inventive situation. In fact, the use of the SMART model does
not allow the solver to switch from inventive situation to the task (more
precisely, to the system of tasks) suitable for further processing using TRIZ
tools.  \textbf{According to the author, the use of TRIZ for the solution of
  organizational and managerial tasks in the first place makes it difficult it
  is the lack of reliable tools to move from inventive situation to the system
  of tasks in the field of business systems.}

The SMART model, though, makes important refinements to the understanding of
inventive situation, but does not solve the problem of preparing an inventive
situation in organizational and managerial tasks for its further processing by
TRIZ methods. AT Currently, the SMART method is a generally accepted method of
setting goals in all over the world, mastery of the SMART method is considered
one of the most important competencies modern manager, almost mandatory
elements of management. However for formalizing the inventive situation in
organizational and managerial tasks this the generally accepted method does
not give the desired result -- it does not help to single out inventive
situation system of tasks that can later be processed TRIZ tools.

There is another opinion. For example, a number of TRIZ specialists are
convinced that if not standard methods or not enough resources to solve the
problem, then any TRIZ specialist will formulate a contradiction using
standard formulas.  The author strongly disagrees with this argument, since
before according to standard a contradiction will be formed to the formulas
for a successful solution of the organizational management task is required to
describe a model of a functioning business system (IFS) [15] and put the
system of particular tasks to its elements (subsystems and
supersystems). Otherwise, we get a contradiction that is extremely general
nature, and therefore, there is a high risk of getting output commonplace
decisions. The author recommends formulating contradictions to each of
particular tasks posed after the analysis of the IFS. Otherwise, the solver
risks too "narrow" the obtained solutions, as a result, a situation is
possible when the goal the solver will not be finally reached [10], [11].

It is worth noting that the tree of contradictions also cannot be used on this
stage, since at the stage of formulating the organizational and managerial
tasks many contradictions are not yet visible -- see the example at the bottom
of the section (implementation task new sales system in the company).

\textbf{Conclusion: the SMART method helps to concretize the task for
  execution, but in as a tool for primary processing of an inventive situation
  cannot be used.}

\section{Description of business processes}

Description of business processes [34] is another method that is actively used
in consulting environment for the preliminary processing of an inventive
situation. TO Description of business processes is usually resorted to if the
solver believes that an organized business system does not function
rationally, which means, unlike the previous method of setting tasks has a
narrower scope. how as a rule, this tool is used to analyze the inventive
situation in in order to increase the efficiency of standard, well-established
processes in organized social systems.

The technology for describing business processes in world practice is
standardized and described by generally accepted \emph{notations}, for
example, IDEF0, BPMN, and some others [34].  In TRIZ, a similar method is also
actively used in the form of stream analysis. And although descriptions of
business processes and flow analysis are not quite the same thing [11] (flow
analysis describes the transfer of matter, energy or information, contains
sources, consumers and the flow path, and the description of the business
process focuses on the list of operations performed by the owner of the
process and participants in the process), yet these methods have a lot in
common.

An example of a business process description:

\HGG{Fig. 2. An example of a description of a business process.}

In fig. 2 shows an example of a description of a business process
decision-making by the bank on a client’s loan application using BPMN
notation. As can be seen from fig.  2, a similar method is great for
preliminary analytics of only one class of organizational and managerial
tasks: regularly recurring optimization operations in various parts of the
organized social system. And if the task connected not only with established
processes, what to do in this case? After all, business processes -- this is
only one aspect of the development of organized social systems, there is also
other levels -- making strategic decisions, developing new sites activities,
relationships of process participants and other tasks.

\textbf{Conclusion: the tool is very useful in solving a particular class.
  organizational and management tasks but not possesses required versatility
  and is suitable exclusively for optimization regularly ongoing processes,
  therefore, as a single tool for preliminary processing inventive the
  situation at decision organizational management tasks cannot be
  recommended.}

\section{System operator}

Consider the "classic" TRIZ tools that claim to be funds primary analysis of
the inventive situation in organizational and managerial tasks.

There are cases when TRIZ practitioners use a system operator for this purpose
[25]. The author also uses the system operator to solve the problems posed in
business systems. One example of using a system operator to solve
organizational and management tasks are shown in Appendix 2. Using a system
operator can see the development of the system in dynamics and predict it
structure in the future, and also consider the structure of a functioning
system -- describe subsystems and subsystems of the system under study
(... sub-subsystems, subsystems, system, supersystems, suprasubsystems ...).

True, the system operator has a significant drawback -- it scans system
development through the composition of its elements, but does not take into
account the layers of the system and does not show the connection between the
elements of the system, which, of course, sharply limits the capabilities of
this tool in terms of describing organized social systems. In addition, the
structure of a functioning system is taken only element-wise, in Unlike
schematization, which looks through layers, groups, relationships, processes
and functions. BUT if the supersystems are on completely different management
layers, what to do in this case? How to set specific objectives for management
effectiveness elements of the system? The system operator does not reflect the
phenomenon of control (Fig. 3).

The advantages of the system operator include its versatility and adjustable
level of detail of system elements, as well as the ability to trace evolution
system, its subsystems and supersystems, which determines the prognostic value
of a given tool.

Thus, the minus of the system operator is the impossibility of adjustable
decomposition of the elements of the supersystem, the impossibility of drawing
relationships between elements of the system and supersystem (the structure is
taken as if in isolation, without indicating connections between system
elements), as well as the impossibility of demonstration in the model control
schemes, which is a critical point in the study inventive situation in order
to solve organizational and managerial problems.

\HGG{Fig. 3. The structure of the system operator.}

Nevertheless, the author considers the system operator a perfectly applicable
tool to solve organizational and managerial tasks, but not for the purpose of
preliminary analysis of the inventive situation, and \textbf{in order to study
  the system in the context of its evolution} [11], which is important for a
number of \textbf{strategic} tasks, where situational the interaction of
system elements should not be taken into account.

\textbf{Unlike the specific inventive situation posed by the customer in an
  organized social system, tasks worth exploring with using a system operator,
  usually wear a common or strategic nature (Appendix 2).}

\section{Structural Analysis and Functional Modeling}

These types of analysis used in the PSA [19], most accurately describe
inventive situation when solving organizational and managerial problems, since
show the composition of the system, elements of the supersystem, the
relationship between the elements of the system, functions, and modern
varieties of PSA allow you to select groups of elements and even take into
account the processes occurring in the system [17]. Functional Example shown
in fig. 4:

\HGG{Fig. 4. Simplified functional diagram of the construction helmet (in red
  harmful functions are shown).}

It is this approach that makes it possible to conduct a preliminary analysis
inventive situation with the aim of setting particular tasks, which
subsequently can be solved using TRIZ tools. However functional modeling there
is still the same key flaw characteristic of any methods developed for the
analysis of technical systems: the method does not take into account the
context control between the elements of the system, in addition, part of the
elements of the functioning systems are well-modifiable objects of the
material world, and part -- subjects, that is, people performing certain
functions, but at the same time having their own goals, dynamically changing
emotions, patterns of behavior, etc., that is, objects that are difficult to
predict behavior. For description such objects require a special language in
order to develop solutions that subsequently can be replicated.

When solving organizational and managerial tasks, ignore this impossible,
since people in organized social systems are essential (and often the most
important) elements of the system.

\textbf{Conclusion: the author believes that functional analysis has great
  potential for solving organizational and managerial problems, in the
  author’s book [11] in detail an example of solving a similar problem is
  analyzed, but this tool is more likely applicable for optimizing business
  systems related to improving the structure social systems and their
  subsystems [39] and is not quite convenient for preliminary analysis of
  inventive situations arising in such systems, since not takes into account
  the features of the description of people as subsystems of the social
  system, also it does not take into account the dynamically changing control
  action of elements systems on top of each other in the context of the
  considered inventive situation ( is, does not take into account the features
  of "soft" systems).}

\section{Schematization developed at the Moscow Methodological a circle (MMK)
  under the leadership of G.P. Shchedrovitsky and her application for
  preliminary inventive analysis the situation}

This method was formed in the Moscow Methodological Circle under the
leadership of G.P. Schedrovitsky [10] in order to organize the group’s thought
activity professionals who discuss problematic situations in the field of
organization and management. The meaning of this tool was to gradually apply
to “Map” system elements that are significant in terms of the task being
solved and communication between them, that is, get a tool for the manager,
similar to the main tools of military strategists -- a map, the detail of
which is manifested gradually, in the course of the thought activity of the
officers planning the military operation. The author considers these
developments to be an excellent alternative to the method described.  above
methods in terms of formalizing the inventive situation, however in in its
pure form, getting synthesis schematization + TRIZ was not so easy.

In essence, schematization is a visualization of an inventive situation, made
according to \textbf{system categories} and understanding functioning of the
system [10]. It is this tool that the author considers optimal to conduct an
initial analysis of the inventive situation when solving organizational and
management tasks. It should be noted that the author had to restore the
categories of schematization below yourself on the basis of a detailed study
of the work of G.P. Schedrovitsky, as at present schematization is greatly
simplified and many SMD methodologists build schemes without taking into
account of the categories of systems below, some generally simplified
schematization to scribing [35].

Therefore, the author had to restore principles on his own schematization of
the work of G.P. Schedrovitsky, and then independently develop methods for
setting goals based on the results of schematization of inventive situations
in organizational and managerial tasks.

Based on the analysis of the works of G.P. Schedrovitsky can distinguish the
following categories schematization:
\begin{itemize}\itemsep0pt
\item[1.] IFS (frame);
\item[2.] Layers;
\item[3.] Groups;
\item[4.] Communication;
\item[5.] Functions;
\item[6.] Processes;
\item[7.] Generalized objects;
\item[8.] Filling.
\end{itemize}

Disclaimer required: category \textbf{“system frame”} in terminological
apparatus G.P. Shchedrovitsky may be replaced by the category \textbf{“model
  of functioning systems”} developed by Nikolai Shpakovsky [15].

Since the author spent a significant part of his work in departments marketing
and sales of international corporations and is the author of a three-volume
“The era of smart sales ...”, where he summarized the experience gained and
described his methodological add-ons in this area, and currently leads a
significant part of projects in the field of marketing and sales in the B2B
market [27], then many examples of processing organizational and managerial
tasks the author gives precisely from these areas. Of course, this does not
mean that the circle of organizational managerial tasks are limited to these
areas. Rather, tasks from marketing and sales are a kind of organizational
management tasks.

\subsection*{The model of a functioning system (MFS) is determined in two
  stages:}

\paragraph{1. Isolation of the kernel.}
The inventive situation presented by the task manager indicates that there is
resistance of managers who do not accept the new sales system, implemented at
the enterprise. It’s known that before that, managers didn’t work randomly,
their work was determined by a different sales technology supported by the
enterprise, to which they got used to. After conducting these simple
arguments, we got structural core, which can be represented as a section of
the circuit (Fig. 5):

\HGG{Fig. 5. The core of the problem, presented in the form of a diagram.}

In fig. 5 we see the main defendants in the task -- the “resisting” themselves
managers, what they resist is the complex sales system being implemented
solutions and the existing sales system, which they managed to get used to in
time work. Sales systems conflict with each other (this is a logical
conflict), in this case -- differ in content and requirements, as shown by the
dashed arrow, in addition, managers also conflict with the newly introduced
sales system, as it requires them to restructure their work, which causes
dissatisfaction of sellers.  The new system is promoted by the head of the
sales department, so between the head sales department and managers, we also
observe a conflict of interest. Further on the diagram show this connection.
\textbf{The core of the circuit always represents the minimum circuit.
  conflict given by inventive situation.}

\paragraph{2. Definition of connections from the core of the task to the
  supersystem and the final definition of IFS (framework).}

To whom the managers express their dissatisfaction, who perceives them
resistance? In the course of communication with the task manager, it turns out
that first of all -- to the head of the sales department (in the diagram in
Fig. 5 and 6, he is indicated as ROP). Or maybe and customers, which is
unacceptable to the company. In any case, customers should be plotted, since
they ultimately reflect the effect of internal changes in the sales
department.

Why is a sales system created? To increase conversion rates, efficiency of
work with the client, and as a result -- to conclude more transactions on
large amounts without hiring additional sales staff. What else “Hurt” the
sales system? CRM system implemented in the company (Customer Relationship
Management -- customer relationship management [16]). New sales system will
require major changes to work with the main software software in the sales
department -- CRM system. Therefore, the diagram depicts a conflict a new
sales system with an existing CRM system, to which specialists of the
department sales managed to get used to. The problem is not that the installed
system does not have certain options, this is a secondary task, smaller than
the one we trying to formalize. The problem is that the "relationship" of
sellers with new requirements of the CRM system.

In addition, the work of the sales department relates to the activities of
other company services -- production, warehouse, logistics, accounting
... This is important, but at the stage of setting the task it is too early to
engage in deep detailing of these processes; therefore, we denote “points
touch "sales department with other company services as" end-to-end business
processes ", which can also be transformed under the action of the new sales
models. So we got a formed model of a functioning system consisting of: sales
staff, head of sales, the existing sales system, the new sales system that is
replacing it, target client groups, CRM systems and end-to-end business
processes:

\HGG{Fig. 6. A model of a functioning system, presented in the form of a
  diagram. Abbr .: ROP -- Head of Sales Department.}

\subsection*{The concept of a layer in a circuit.}

The layer in the diagram symbolizes that a system element or group of
elements, placed on a parent layer, controls an element or group of elements,
located on the lower layers in the context of the task (i.e. the layer is
graphic representation of the fact of managing the activity of one element by
relation to another). And although the system in most cases is defined as a
set of interconnected elements, it is obvious that the elements can be in
different relationships with each other in the studied system, that is,
between the elements the management hierarchy is built in terms of a certain
activity, studied by us in the context of the problem being solved. The
concept of a layer allows us to consider elements of the system based on their
dynamically changing hierarchy, where the hierarchy elements is determined
based on the understanding of which element is the control, and which element
carries out “orders” of more “senior” elements of the system, moreover not at
all, but exclusively in the context of specific activities related to set
organizational and managerial task.

It is worth noting that the concept of a layer does not replace the concepts
of a sub-sub-system, a subsystem, a system, a supersystem, etc., so layers are
not at all the same as we depict in system operator. \emph{If subsystems,
  supersystems, etc. pretty tough fixed, layers are constantly changing in the
  context of the activity of elements, depicted in the diagram.}

That is, as an important concept for the analysis of business systems, the
author suggests use schematization with the allocation of "layers", explaining
this with the basic properties organizational and management systems built on
a \emph{dynamically changing hierarchy of management in the context of the
  studied activity of system elements.}

It is important to note that subsystems are subsets of the system, the system
itself is a subset of the supersystem. As you know, subsets have the property
of similarity the sets of which they are a part. In other words, a subsystem
is a multitude elements that make up the system, broken down by some criteria
into subsets [45].

For example, a subset of the decision center [16] in an organization is seller
of the company -- a potential supplier of the product, since the seller is his
a subset from a decision-making perspective (he makes a decision based on
information transmitted by the seller, but not only. In the process of
deciding on the decision center also has other subsets). But what element what
activities does it manage? This is a big question. The answer is only in the
context of the task.  Suppose, in the process of implementing his functions,
the manager conducts research customer needs, on the basis of which will
subsequently prepare a commercial sentence. What element of activity controls
what? Of course, the client controls further actions of the seller based on
the information provided about your needs. Moreover, if we consider a
different situation, where the needs customer is a product, then the layers
will change again, that is, the seller through questions will manage the
activities of the decision center. At the same time, the PS and NS remained at
their places, only layers change. Another example. The seller forms a picture
of the world customer, creating value in his offer. Which element is the
activity of which is driving now? Now the seller manages the activities of the
client from a position of acceptance solutions.  PS also governs activities NA
at this context.  Please note: PS and NS did not change places, and the layers
changed.

This is the most important understanding from the point of view of finding a
solution to a problem in organizational managerial context, therefore, the
concept of a layer is not a substitute for the concepts of NS and PS.

\textbf{Rule image layers in the diagram:} if any elements of the system are
depicted in the diagram is higher than the others, then the solver ranks them
in the upper layer, i.e.  in the context of this inventive situation considers
them as managers elements. The main thing is to always remember the rule:
layers may vary depending on contemplated inventive situation.

In fig. 6 layers are clearly visible in the context of the task: business
processes largely determined the configuration of the existing CRM system;
CRM-system, which was configured based on the requirements of the existing
system sales, at the moment is a deterrent to the introduction of a new sales
systems, as it is reliably connected with end-to-end business processes in the
company, which implies the use of its reports by a number of related units.
It does not support the functions required for the new sales system, but the
transition to another CRM-system, although it will provide the necessary
functions for the introduction of a new sales systems are highly likely to
create similar conflicts at the junction sales department with other divisions
of the company. From here it’s easy to conclude that sales staff are on the
next layer -- they are “managed” established business processes in the company
and the CRM system that has “taken root” in it, supporting the current sales
system, which, in turn, is not satisfied with the head of sales and the best
employees, as it does not provide effective work with customers in a modern
highly competitive market.

It is easy to notice that the schematic representation of the MFS taking into
account the layers makes a lot of important refinements in understanding the
inventive situation. An analysis of the circuit shows that at least a few
contradictions are hidden in the problem (of course, such a representation an
inventive situation requires the solver to develop sketching skills, so in the
process of constructing a scheme is a bidirectional thought process: the
scheme is built in the process of analyzing layers, but on the other hand,
layers “appear” in the process of constructing the circuit. Therefore, in
practice, the IFS scheme is usually redrawn several times, until full clarity
is established in the description inventive situation between the task manager
and the solver ).

To this point, the author received the observation that since a circuit is
rarely created in one passage, then this approach strongly resembles MP\&O.
The author does not consider MP\&O and iterative approach with identical
concepts, as during several iterations the scheme is refined, and not at all
its new image obtained by random way. Does the formulation of TP or RBI solver
does not clarify definitions often iterative? It’s far from always possible to
get the final wording the first time. The same thing happens when applying
schematization.

\subsection*{Groups of IFS elements (groups).}
A group is a combination of elements of a system to fulfill a specific
functions. If the group is taken as a separate system, then we can talk about
the GPF (main useful function) of the group The author believes that instead
of the term “group”, introduced G.P. Shchedrovitsky, it’s completely
appropriate to use the terms TRIZ -- a system, subsystem, sub-subsystem
... Usually, if solving a problem requires more general consideration of the
elements of the system, then more detailed (in the course of work on the task
such transitions are assumed that becomes apparent in the communication
process solver and problem-taker), it is convenient to give on the diagram the
required detailing of the elements and their relationships for deeper
analysis, but at the same time when more surface analysis, combine such
elements (sub-subsystems) into groups (subsystems) with allocation of the
general function of the group. In practical application schematization for the
analysis of an inventive situation, such a division may have very important,
for example, in the task of increasing the efficiency of a department (in this
case, the sales department) the following scheme was developed for analysis
inventive situation:

\HGG{Fig. 7. Scheme of the sales department with the allocation of layers and
  groups. Abbr .: KAM -- key account manager (key account manager), lead --
  potential a client who in one way or another responded to marketing
  communication.}

The groups in the diagram are:
\begin{itemize}\itemsep0pt
\item marketing department;
\item sales department;
\item customers.
\end{itemize}

For example, why did the company have a marketing department? To prepare and
conduct marketing promotions? To study the market? For advertising campaigns?
For marketing analytics? To carry out public relations (PR)? Marketing should
carry out all these functions, but they are not the main ones. They -
auxiliary. All these functions are needed in order to promote your product on
the market. So the GPF of the marketing group (department) is the promotion of
the company's product on market by influencing target client groups (again
groups, only now they combined by functional features of consumption). GPF
group (department) sales the same, but with some differences -- promoting a
company's product on the market by personal impact on the client. In essence,
the differences between them are in the way impact on the target audience.

Groups can be not only departments and departments. It can be design groups
or, for example, groups of employees united by some social sign or, say, the
time of work in the company. It all depends on the condition set task. Only
one important rule should be remembered. Any group has its own GPF, essential
from the point of view of the problem being solved. Arbitrary division of
elements into groups within the framework of a functioning system is
unacceptable, so how it complicates the task, burdens it with unnecessary
information. There is a question: is it worth whether to introduce this
concept or dwell on the concept of a subsystem, traditionally used in TRIZ?
Perhaps, instead of the concept of “group”, the concept of “aggregation”
elements ”, which is also known in TRIZ? At this stage, the author proposes
this leave open the discussion question.

\subsection*{Function.}
Functional language is perfectly developed in TRIZ and, perhaps, is the main
in the analysis and description of the system: first of all, with the use of
FSA. Therefore in Within the framework of this work, a detailed description of
the concept of “function” is not required.

\subsection*{Processes and communications.}
These are the most important categories of schematization.

A process is a course, the course of a phenomenon in time. In this definition
the key word is "phenomenon."

Relationships -- a designation on the diagram showing that in the context of
the task it is important for us that element A affects element B, but what
processes are going on is not important for us.

Since the circuit in Fig. 7 is functional, processes are not indicated on it,
but the arrows show the functions. But if when creating a functional diagram
is not managed to get a working model, that is, when analyzing the scheme,
breaks and promising the trajectories of the solution to the problem were not
visible, then you should start to plunge into processes.

Interestingly, in TRIZ this problem of the process level has already been
discussed, for example, in relation to advanced functional analysis developed
by Naum Feigenson and Oleg Feigenson [17], when as a result of the FSA not
one, but several functional models are built, each for a specific the state of
the system due to the processes occurring in the system.

Here we can clearly see the fact that in most problems the solver has enough
remain at the functional level (then only functions and communication), but
there are times when the system has to be considered in various states
depending on the processes occurring in it. Similarly, when conducting
schematization of an inventive situation, there are times when the scheme
requires indicate also the processes going between the elements of the
system. Understanding the difference processes and relationships are very
important for quality sketching inventive situation and subsequent task
setting by analysis scheme inventive situation.

You can see an example of such a scheme in the diagram below:

\HGG{Fig. 8. The scheme of analysis of the inventive situation to the problem
  of increasing the effectiveness of the mentoring process.}

In the diagram shown in fig. 8, you can see the simulation of the mentoring
system, organized in the sales department. The processes indicated by arrows
that go when planning a strategy for concluding a deal, the student’s work in
field conditions ”and the processes that occur during the organization of
feedback from the student to the mentor. In addition, the task involves
immersion in the processes taking place in the period of supervision by the
mentor of the development of the student's skill in using the tool, which the
latter should develop on the instructions of the mentor.

\textbf{To date, the author does not have data on the existence of methodology
  allowing at the stage of schematization of an inventive situation clearly
  determine whether to go to the process level or enough limited to a scheme
  containing the necessary elements, their functions and relationships.}  The
practical recommendation is as follows: first we build a diagram that
describes inventive situation at the level of connections and functions, and
then, if the data is clearly not enough for further processing with TRIZ
tools, we begin to delve into consideration of processes taking place between
elements of the system, for which purpose stream analysis or business process
analysis using standardized notations [34].

The application of the process scheme leads to inevitable complications of the
scheme, describing an inventive situation, therefore, when conducting
schematization should be guided by the principle of expediency and designate
processes only where there is a need for their detailed study (see the case on
the introduction of a new sales systems at the bottom of the section: Appendix
5).

\paragraph{Generalized object and content.}
Let's move on to the concept of “generalized object” and “filling”. In the
opinion of the author, from categories developed by G.P. Shchedrovitsky, these
concepts are one of the key points view of applying system analysis to
inventive situations arising in areas of organization and management. These
categories apply to schematization, responsible for scaling the resulting
solutions. \textbf{Scaling is critical characteristics of the decisions in the
  organizational and managerial sphere.}

When conducting schematization from the perspective of the categories of
“generalized object” and "Filling", not only sharply increases the likelihood
of getting scalable solutions, but also increases its stability, that is, the
ability to withstand disturbances of the environment. In fact, at the stage of
schematizing the inventive situation laying the foundation for the quality of
a future solution.

So, a generalized object is a vacant unit, like a “shell” of an element, which
sets the requirements for this element.

Generic objects have properties that define the requirements for their
filling.  For example, a generalized object "leader", a generalized object
"Teacher", etc.

Generic objects are related to other generic objects in the structure,
however, for the system to function, generalized objects must be filled.
Someone must be a milling machine operator, manager, driver -- for example,
computer program or person (see Fig. 1 -- it is indicated that the person is
subsystem or supersystem of an organized social system (for example, business
systems), although if we look in more detail, then the subsystem or
super-system will be not a person as such, but a generalized object, which
should have an appropriate filling). \textbf{Filling} is a system that fills a
generalized object in according to his requirements, somehow a person who has
the appropriate competencies or, for example, a computer program with certain
characteristics corresponding to the requirements of generalized objects.

In technology G.P. Shchedrovitsky requirements of generalized objects are
called properties-functions , and filling properties -- attributive properties
(in texts G.P. Shchedrovitsky generalized objects are called "places", but the
author believes that such a name is more likely to cause confusion). So in
organizational management tasks, a special class of problems arises -- tasks
for matching the requirements of generalized objects (properties-functions)
and filling properties (attributive properties).

When solving organizational and managerial tasks, the solver must understand
at what level is it worth solving the problem. If the problem is solved in the
space of the organization or its unit (department, department, site), it is
important to try to find solution at the level of generalized objects, which
will determine further scaling received decision. If the task is set at the
level of a specific content, then the solver must warn the customer that the
solution most likely will be special for this case and the scaling of the
resulting solution will be is fraught with certain difficulties (if at all
possible).

Generalized objects can be represented as subjects (this is a generalized
object, the filling of which is a person, and objects (computer
program, any document, robot, etc., see fig. 9).

Some notation used in the diagrams :

\HGG{Fig. 9. Some designations used in the scheme of the inventive situation.}

\section{Setting objectives based on the results of schematization}

\textbf{Two options for further analysis after schematization:}

\paragraph{Option 1. Work with existing gaps.}
We find gaps, that is, discrepancies between the “how it should be” and the
what is currently shown in the diagram. On breaks private can be put tasks
(see Appendix 1).

Next, we highlight a list of unwanted effects (NE). Work with undesirable
effects in TRIZ well developed, most often for this purpose a causal analysis
is applied. Therefore, according to the results of schematization Inventive
situation, we can advance in a variety of ways:

We get a list of NEs and conduct a causal analysis. This way applicable if all
NEs in the diagram are obvious; We identify the gaps in the diagram and set
the task to eliminate them. To received tasks, you can apply the primary
processing mechanisms of the task, long and successfully used in TRIZ --
stream analysis, functional analysis, causal investigative analysis,
benchmarking ... [18]. There are many variations, we must move from
context. No one forbids applying schematization to clarify in detail gap
structure, if the obtained task is a new inventive situation with many
unknowns [11];

Highlight technical inconsistencies and continue to work with them (technical
contradiction: a situation that arises when trying to solve an inventive
problem by improving a specific feature (parameter) of the system, which leads
to unacceptable degradation of another attribute (parameter) of the same
system [42]). In that in the event decisive TRIZ mechanisms come into
operation.

\paragraph{Option 2. Analysis of the current inventive situation and statement
  of private tasks.}

This is the path we took, solving the problem of a multiple increase in sales
construction equipment in the channel "road construction" (Fig. 10). The
problem was that over the past three years, the company's product margin has
been halved, and sales are steadily falling. Marginality is known to add up to
structure costs and market value. We decided to start from the second and
analyze the structure construction equipment market (Fig. 10 gives an example
of a rough analysis of the situation, but in order in order to understand that
the current business strategy of the company has been chosen incorrectly,
sketching without fine detail turned out to be quite enough):

\HGG{Fig. 10. Preliminary “rough” design scheme of the construction machinery
  market in the southern Urals.}

To come to the choice of a promising road map, we plotted budget allocation
for road construction and sorted out the hierarchy of this distribution -- so
the layers in this diagram were grouped by region (groups). Further,
understanding the price segmentation of construction equipment, it turned out
it is easy to compare these two parameters with the practice of the tasker,
and then note two points on the diagram -- layers the company is working with
now and as much as possible a layer close to the consumer, capable of
acquiring the technology of this segment based on the data plotted on the
scheme and its typical needs. So decided promising roadmap, which was
significantly different from the existing company’s market strategy. To
implement the resulting roadmap it was required to pose a number of tasks and
resolve the contradictions that arose.

At first glance, it may seem that this scenario is similar to the construction
functional model during the PSA, as the scheme and functional model describes
the structure of the system. However, the scheme also clarifies the layers,
showing dynamic control hierarchy, and also allows you to simulate depth
interactions between system elements, by introducing a chain of concepts
communication-function-process, reveals the composition of subsystems, if
necessary with positions of the problem being solved (group or aggregation of
elements), including the ability to select groups passing through the layers
"diagonally", for example, if To the problem condition, it becomes necessary
to analyze the work of project teams with taking into account the employee
performing two or more roles (for example, a member of the project team -- an
employee of the financial service), when conducting schematization, the solver
has the ability to distinguish between filling properties and requirements of
generalized objects, which is critical from the point of view of scalability
of the resulting solution (more one example of an inventive situation is
depicted in the diagram in fig. 11).

\HGG{Fig. 11. The scheme of selection of employees according to the competency
  model.}

In fig. 11 shows a diagram compiled by the author to describe inventive
situation that arose during the selection of employees for certain positions
taking into account changes in the requirements of generalized objects in the
process of evolution organization according to the model of I. Adizes (Fig. 11
shows the so-called PAEI code (P -- production of results, A --
administration, E -- entrepreneurial function and I is integration. PAE refers
more to generalized objects, I to content). The code PAEI is shown in fig. 11
as a function of the requirements of generalized objects and properties
filling. The diagram also shows the significant roles of the process
participants, located on three layers and elements of the personnel selection
system related to generalized objects (competency model) and content
(assessment of competencies and type personality and ability assessment).

Therefore, despite a certain similarity with the functional model, the circuit
a slightly different tool. Its main purpose is to isolate the system of tasks
from primary inventive situation . The author suggests using schematization to
formalize the inventive situation when solving organizational management
tasks. Schematization should be applied immediately after clarification.
problems and goals of the solver before using the usual TRIZ tools.

\section{The fundamental difference between schematization and functional
  model}

\paragraph{Select layers.}
The functional model used in the FSA does not imply building a hierarchical
scheme with the allocation of layers, where the subject of management is
parent layer in comparison with the control object. Similar hierarchy of
elements in the scheme is very important for the analysis of the inventive
situation in the organizational management tasks (the very concept of
organizational and management tasks requires representations of system
elements depending on the control hierarchy in terms of set task).

Representation of a system element as a Generalized object and Filling, a
clear understanding of whether we are solving the problem at the level of a
generalized object or at the filling level . The author pointed out above that
business systems are soft systems, since they include humans as the main
subsystems.

The author also emphasizes that it is impossible to simply transfer tools from
technical sphere in a business system. To handle the inventive situation in
business systems require specific tools that can prepare task to use the
"standard" TRIZ tools.

\section{The algorithm for working with the circuit}
The algorithm for working with the circuit is as follows:
\begin{itemize}
\item based on the situational analysis that we carry out on the scheme, the
  most acceptable way of conducting further transformations, which is defined
  as a priority (road map of future traffic).
\item as soon as the roadmap is selected, we immediately attach it to the
  existing one system, we see the secondary tasks in the form of directions or
  NE, which will appear in the system when implementing the selected roadmap.
\item If necessary, analyze selected tasks with tools primary processing tasks
  adopted in TRIZ.
\item If necessary, we form technical contradictions. Next for permissions
  selected contradictions apply famous TRIZ tools.
\end{itemize}

\section{An example of the use of schematization for setting objectives for
organizational and managerial task (the case is described in detail in
Appendix 5):}

A system consisting of: sales department of an industrial enterprise,
manufacturing tooling from heat-resistant steels, presented head of sales,
sales staff and the current system sales (Fig. 12).

The essence of the problem: the head of sales (ROP) implements a new sales
system, having advantages over the previous one in terms of depth of study
customers and, as a result, allowing to increase the average amount of the
contract and conversion, however, managers resist and are in no hurry to leave
the “beaten” rails. 

Required: to make managers use only the tools of the new system sales in their
activities (as the task sounded in the original formulation, that in the
process of analyzing the circuit turned out to be not quite the correct goal
setting -- see table).

Below, we compose a model of a functioning system, presented in the form of a
diagram.  The process of constructing a circuit for this task is described
above (see explanations for Fig. 6):

\HGG{Fig. 12. The scheme of the inventive situation in the problem of changing
  the sales system.}

The tasks set according to the scheme (Fig. 12) using the categories of
schematization:

1 one interaction the system (dotted line) with elements supersystems

1.1.  CRM system. The conflict arose largely due to the fact that the existing
CRM-system is not adapted to the requirements of the new sales system, which
creates significant inconvenience $\to$ make the CRM system meet the
requirements of the new sales system and supported her.

1.2 end-to-end business processes. The new sales system is changing end-to-end
business processes, this is especially true in collaboration with the design
department and production $\to$ you need to configure end-to-end processes so that
the requirements of the new sales systems were provided.

1.3.  customers. The new sales system increases the time to contact by the
customer $\to$ how to make the depth of the customers' work increase without
increase time managers?

2 Layers

2.1 The implemented sales system manages the actions of managers, imposing on
them specific requirements $\to$ How to make sales system requirements performed,
but did the managers spend as little effort as possible?

2.2 Managers are faced with the fact that for a number of clients the
requirements of the new system redundant, which does not increase, but rather
reduces efficiency (from this point of view managers "manage" the reaction of
customers, hence the distribution of layers on scheme) $\to$ Differentiate
customers and introduce a new sales system only in relation to such client
groups in which conversion increase is expected and the average weight of the
transaction when applying this system.

3 Communications.  Partially analyzed in paragraphs 1 and 2, additionally:

3.1 A logical conflict between two systems, for example, the approach to
identification of needs, the stages of the transaction are radically different
$\to$ Compare the requirements of the existing and new systems, identify areas
similarities and cardinal discrepancies, disassemble into elementary steps of
the area cardinal differences, thereby simplifying the implementation (such a
statement of the problem allows the solver to rely on existing resources).

3.2 Communication defects ROP managers $\to$ Define metrics and reference points
in the new sales system, which should be feedback from manager to to the
leader. Simplify data retrieval by reference point managers.

3.3 Establish a relationship CRM-system -- ROP $\to$ Having solved the tasks 3.2,
bring the CRM-system in in accordance with the decisions received, amend the
procedure accordingly meetings, strengthening communication on reference
points and reducing communication on non-essential items.

4 Processes and the functions

4.1 The task appeared after setting the task 1.2: to conduct a detailed
analysis of business processes between the sales department and the design
department, as well as between the department sales and production department
(pre-mapping processes with using BPMN notation). Highlight bottlenecks and
set tasks for them overcoming.

4.2.  After solving task 1.1, set the task to simplify the entry of the
required data into CRM system by entering patterns and rules.

5 Groups

5.1 Negative phenomena within a group of managers -- the effect of the
adoption of new technology modeled by J. Moore $\to$ how to use innovators and
early adopters as a resource for introducing a new sales system? How to
identify and to neutralize the influence of "farther"?

5.2.  Customer groups, which follows from the analysis of the task 2.2.
Dividing customers into categories A, B and C. Define customer categories and
target customer groups, for which the new sales system is redundant. Set a
task to synchronize work department, which should apply both sales systems, if
the hypothesis is confirmed and the existing sales system would be appropriate
to maintain for certain customer groups amid the introduction of a new one.

6 Generalized object and filling

6.1.  To conduct training of “good middles” in the new sales system after
solving problems from paragraphs 1-5 and determine whether they have reached
the level of "stars" after a given time. If not, to conduct a comparative
analysis of the work of both and conduct additional training of “good average
"according to the performance model (the performance model explains what
specific competencies make stars stars by comparing their competencies with
competencies of “good middle peasants” in the team and identification of
discrepancies).

And in addition, according to the results of applying schematization and
analysis of the scheme, taking 13 categories describing the success of
introducing a new sales systems in the practice of the sales department of
this production company. Author pays special attention to the tasks set in
paragraphs 2 and 6 of the table. If not enter categories of layers and
categories generalized object-filling, data tasks might not be set. And, as
mentioned above, the tasks for compliance requirements of generalized objects
and filling properties are most important from the point of view of solving
organizational and managerial tasks set in organized social systems. Some
tasks from the table do not require application TRIZ tools -- they can be put
to execution using the SMART method [14].  Some tasks require the use of
“primary task processing” tools: streaming analysis, causal analysis,
comparative analysis. Attempt to solve parts of the tasks will lead to the
formulation of technical contradictions [11].

So, schematization allows a primary analysis of inventive situations when
solving organizational and managerial problems and get a set of private tasks,
including taking into account layers and compliance of filling properties with
requirements generalized objects for which standard TRIZ tools.

The second option is to extract unwanted effects from the resulting scheme and
subsequently we work with them -- see appendix 1.

\section{The algorithm of work with the organizational and managerial task with
subsequent use of TRIZ tools:}

The algorithm for solving organizational and management problems using TRIZ
schematization and methods according to the results of project implementation
are presented as follows:

\HGG{Fig. 13. The algorithm of work with organizational and managerial
  tasks. Scheme.}
