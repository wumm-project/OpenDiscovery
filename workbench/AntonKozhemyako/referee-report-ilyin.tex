\section{Referee Report by Ilya Ilyin}

\subsection*{Overview}
The dissertation consists of two sections. The first section is devoted to
schematization which is applied at the stage of analyzing an inventive
situation in the process of solving organizational and management problems.
The second section is devoted to the choice of the operational zone and
resource allocation of the operational zone for resolving organizational
contradictions. The dissertation contains case studies explaining application
of these tools and provides recommendations when these tools can be applied.

\subsection*{Significance}
There is a significant interest to apply TRIZ tools, originally developed to
solve problems of improving technical systems, to solving organizational and
managerial problems.  Indeed, there is a notion that recommendations for
reformulating and solving technical problems in the engineering systems are
defined in such a way that their application to organizational problems can be
useful. For example, coordination, evolution, completeness of system, and any
elements of system analysis. This is a natural assumption considering that one
can argue that some of the tools were originated in the business analysis and
were adopted by TRIZ community for engineering analysis.  Examples would
include S-curve analysis, which usually describe company’s lifecycles in terms
of revenue or cash flow, structural analysis of the industry, SWAT analysis,
etc.  Therefore, the task of combining system analysis and TRIZ solving tools
for solving business problems is significant.

Analytical instruments developed to transition weakly defined business problem
in to the set of specific technical requirements and contradictions became
standard elements of the technical consulting practices. The similar
structural approach would be beneficial in the analysis of business
organization.

In addition, when trying to apply the most powerful TRIZ concept of
contradiction to business situation there is a need for guiding principles to
define contradictory requirements, operational zone and time, as well as
available resources. It is more difficult in the relatively abstract and
qualitative organizational situation than in the course of analysis of the
engineering system. The set of such guiding principles would be very
beneficial for practical application of TRIZ tools in the solving of
organizational contradictions.

\subsection*{Scientific Novelty}
In general, the work contains elements of novelty, since it considers
principles of practical application of tools developed by the author to the
new set of organizational problems which loosely defined in the beginning.

However, the enthusiasm is diminished since there is a plurality of tools and
methodologies developed for analysis of business situation in the management
consulting. Some of them were borrowed by the TRIZ methodologists and adapted
for solving technical problems. The question would be why not to take a step
back and to apply those original tools to the business situation in a way that
would lead to better definition of possible organizational contradictions.
Such combination of known business analysis tools and suggested TRIZ
analytical instruments could derive a truly universal project roadmap.

The particular novelty lays in the author’s definition of layers and places
for the elements of the organizational system and their use in the
schematization methodology.  Schematization was enhanced by connection with
MPVs for various stakeholders. Also novel is a broader definition of the
operational zone including a tool, a product, and their environment as a part
of the operational zone. This way all components of the operational zone can
be later explored to extract specific resources for further problem solving.

\subsection*{Approach}
The author presented the thorough analysis of existing methods and
developments in the modern TRIZ practice. The description of the developed
tools contains necessary details for their practical application. The
recommendations are supported by real-life case studies. These case studies
significantly contribute to the credibility of the work and bring it to the
standards of research in the area of business analysis.

The approach would benefit from review of methods of the modern organizational
strategy particularly in the view of some of the presented examples. The
potential sources might include Michael Porter’s Competitive Strategy or
corresponding publications from Harvard Business Essentials. The reviewer’s
specific request would be to compile and present the typical roadmap of the
analytical project, where developed tools would be allocated to the
corresponding stages of the analysis.

\subsection*{Conclusions}
In conclusion, the presented work of Anton Kozhemyako demonstrates the
sufficient level of candidate’s qualification, is based on the significant
amount of theoretical research and practical implementation, satisfies main
requirements for MATRIZ TRIZ Master and is recommended for certification.
