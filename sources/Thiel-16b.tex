\documentclass[12pt,a4paper]{article}
\usepackage{od}
\usepackage[utf8]{inputenc}

\title{Altschuller, TRIZ und ProHEAL}
\author{Rainer Thiel, Storkow}
\date{24.11.2016} 

\begin{document}
\maketitle
\begin{quote}
  Beitrag zur 21. Leibniz-Konferenz „Systematisches Erfinden“,
  24.--25. November 2016 in Lichtenwalde.
\end{quote}

Alle folgenden Thesen werden von mir nummeriert. Weil ich auch Kritisches zu
TRIZ anmerken möchte, lassen Sie mich bitte mit drei Reminiszensen beginnen:

1. Im Jahre 1973 war ich von Altschuller hellauf begeistert, da war ich
aufmerksam geworden auf Altschullers Buch „Erfinden – (k)ein Problem“,
herausgegeben von Dr. Kurt Willimczik (Berlin-Pankow) im Verlag der
Gewerkschaft FDGB. Ich publizierte eine Streitschrift mit Seitenhieben auf die
Systematische Heuristik von Johannes Müller, in der es mir an Dialektik
fehlte. Doch ich veranstaltete ein großes Kolloquium, Titel „Methodologie und
Schöpfertum“. Ich nutzte meine Position als kleiner Forschungsstellenleiter im
Zentral-Institut für Hochschulbildung an der Humboldt-Universität zu großem
Lob für Altschuller.  Und Altschullers TRIZ-Buch brachte ich 1984 in deutscher
Übersetzung auf den Markt, insgesamt 3 Auflagen. Hauptlast der Übersetzung
trug meine Frau, an Sonntagen, nach 60-Stunden Arbeitswoche als Chefin einer
Zeitschrift. Mein heftiger Einsatz wurde vom Chef des größten und von
Professoren gesteuerten Technik-Verlags anerkannt, er sagte: „Genosse Thiel,
Sie haben hervorragend gekämpft, wir machen das Buch“. Weil sich das lange
hinzog, ließ ich als Provisorium ein Altschuller-Lehrmaterial von Tschus und
Dantschenko aus Dnjepropetrowsk übersetzen und breit streuen.

2. Was mich umgetrieben hatte, verschmolz sich mit praktischer Arbeit im
Ingenieurverband KDT. Vor allem gelangte ich ins Netz des Verdienten Erfinders
Dipl. Ing Michael Herrlich (Leipzig). Stimmungsmacher war auch der Direktor
des Instituts für Schweißtechnik in Halle, der Physiker Dr. Werner Gilde. Und
Micha Herrlich, der energiestrotzende, eloquente, gesellige, auch zu Witzen
aufgelegte Revolutionär hatte Mitstreiter gesammelt, um einen Schneeball zur
Entstehung von Erfinderschulen zu wälzen. Einzelne Amtsträger des Ingenieur-
verbandes KDT wie Dipl. Ing. Rudi Höntzsch – Mitglied des Präsidiums – waren
mitgerissen. Andere Amtsträger waren skeptisch oder gleichgültig. Im Bezirks-
verband Berlin der KDT fand ich Anerkennung vom ehrenamtlichen Vorsitzenden
Dr. Dr. Georg Pohler, dem Generaldirektor des VEB Kabel-Kombinats Ober-
spree. Leider wurden wir innerhalb der KDT bald dem zögerlichen Bereich
„Weiterbildung“ zugeordnet. Doch schon 1981 war die erste bezirks-basierte
Erfinderschule der DDR praktiziert, angesiedelt im VEB Berliner Werkzeug-
maschinenfabrik Marzahn unterm Schirm des F/E-Direktors Dr. Ing. Werner
Bahmann. Trainer und Moderator der meisten Erfinderschulen in Berlin war der
Verdiente Erfinder Dr. Ing. Hans-Jochen Rindfleisch vom VEB Transformatoren-
werk Oberschöneweide. Die ersten Verdienten Erfinder, mit denen ich zusam-
menwirkte, waren Micha Herrlich, Hans-Jochen Rindfleisch sowie Ing. Karl
Speicher vom Turbinenhersteller VEB Bergmann-Borsig in Berlin-Pankow. Karl
Speicher versuchte, das Ministerium für Hoch- und Fachschulwesen zu aktivie-
ren. Doch das Ministerium schlief bis zur Auflösung 1990, und die Dresdner
Professoren waren erleichtert, als sie bemerkten, dass das Kolloquium „Metho-
dologie und Schöpfertum“ 1977 nicht vom Minister angewiesen, sondern von mir
einberufen worden war. Bald sammelten sich im Haus des KDT-Bezirksvorstands
Berlin angeregte Ingenieure und erwarteten, dass etwas geschah.  Doch keiner
wollte den Anfang machen. Auch ich sträubte mich, weil ich kein Ingenieur
bin. Doch immer wieder fiel die Wahl auf einen Philosophen. Das war
ich. Abgesetzt wurde ich nur drei Jahre später, im Hause war ein neuer Partei-
sekretär berufen worden. Ersatz wurde nicht geschaffen, ich setzte meine
Arbeit fort.

Anfang der achtziger Jahre entwickelte sich auch Freundschaft und Kooperation
mit Verdienten Erfindern aus anderen Bezirken der DDR: Mit Hansjürgen Linde
aus einem VEB in Gotha (Thüringen) und mit Dr. rer. nat. habil. Dietmar Zobel
aus dem VEB Stickstoffwerke Piesteritz bei Wittenberg. Linde aus Gotha hatte
selber schon mit Erfinderschulen begonnen. Mir imponierte, wie er über
Ingenieur-Sein und Erfinden sprechen konnte. Da animierte ich ihn zu einer
außerplanmäßigen Aspirantur. Seine Dissertation 1988 an der TU Dresden
benannte er „Widerspruchsorientierte Innovations-Strategie WOIS“. In der
Verteidigung an der TU Dresden begegnete er souverän allen Anrempelungen durch
Professoren. Drei Jahre später wurde er Professor in Bayern. Und der Chef
unsrer gemeinsamen Widersacher in Dresden bekannte 1992 in internationaler
Fachpresse: Linde hat recht: Anspruchsvolle Ingenieure müssen sich
Widersprüchen stellen! 1993 sagte mir die Geschäftsführerin der Nürnberger
Erfindermesse im Telefongespräch: „Und dann, Herr Thiel, haben wir noch etwas
Besonderes. Bei uns spricht ein Professor aus Coburg über Widerspruchs-
orientierte Innovationsstrategie“. Da konnte ich erwidern: „Die kenne ich, sie
ist in meiner Wohnung konzipiert worden.“ Ich verriet nur nicht, wo ich
wohnte: in Berlin Ost.

3. In den achtziger Jahren wurden von Hans-Jochen Rindfleisch und Rainer Thiel
rund 25 Erfinderschulen initiiert und praktiziert. Hans-Jochen war Moderator,
Trainer und Wortführer, zielstrebig und geistreich. Er begann mit elf Notizen
auf einem Blatt Papier. Seine Eigenwilligkeit sollte bald deutlich werden,
rasch wuchs sein Spickzettel. Wir beide begannen, über Altschullers Buch von
1973 hinaus zu denken. Gleichzeitig suchte ich die Übersetzung und Publikation
von Altschullers TRIZ-Buch von 1979 zu bewerkstelligen. Ich kroch gleichzeitig
auf zwei Pfaden. Die Übersetzung musste dem Wortlaut Altschullers folgen, doch
1984 im Titel des neuen Altschuller-Buches vermied ich das Label „TRIZ“, also
„Theorie des Lösens von Erfindungsaufgaben“. Ich meinte: Von einer Theorie des
Erfindens kann noch keine Rede sein, ich gab dem übersetzten Buch den
neutralen Titel „Erfinden – Wege zur Lösung technischer Probleme“. Anno 1998
hatte Professor Möhrle, aus dem Saarland kommend, die 3. Auflage zuwege
gebracht. Kurz zuvor hatte ich mich an die VDI-Zentrale gewandt. Von dorten
kam die Antwort: Erfinderschulen sind in der Bundesrepublik nicht denkbar.
Zuvor aber, im Juli 1990, formell existierte noch die DDR, da besuchte uns in
Ostberlin der Chef der Deutschen Aktionsgemeinschaft „Bildung, Erfindung,
Innovation“, Dr. Matthias Heister aus Bonn. Er sagte: „Sie haben Erfinder-
schulen gemacht. Das ist doch Silbernes, das die DDR einbringt in die Einheit.
Schreiben Sie ihre Erfahrungen auf.“ Das geschah. Bald erschien ein dickes
Buch, finanziert durch Erlöse eines Benefiz-Konzerts des Amateur-Orchesters
der Patent-Behörden in München. Anno 93 war das Buch ausgedruckt, da lief ich
zur Filiale des Bundesministeriums für Bildung und Wissenschaft in Berlin.
Dort sagte man: „Machen Sie doch auch ein Buch für uns, wir bezahlen das.“
Dieses Buch erschien 1994.

In Berlin-Ost hatten sich also zwei Entwicklungslinien miteinander
verschränkt: Popularisierung von Altschuller und Entwicklung über Altschuller
hinaus. So entstand vor allem dank Jochen Rindfleisch unsre gemeinsame Arbeit
„ProHEAL“, also „Programm zum Herausarbeiten von Erfindungsaufgaben und
Lösungsansätzen“. Dieses Programm beruht auf den revolutionären Anstößen von
Altschuller (Baku und Moskau). Dieser Genrich Saulowitsch Altschuller hatte
gewirkt wie ein Kolumbus, der einen unbekannten Kontinent entdeckte.
Altschuller wirkte auch wie ein Alexander von Humboldt, Natur- und Kunst-
schätze beschreibend, die noch nicht beleuchtet worden waren. Altschuller
definierte auch Ideale wie zum Beispiel „Ideale Maschine“ und schrieb gewandt
wie ein Krimi-Autor. Davon bin ich bis heute aufgeladen. Also las ich
Altschuller von 1973 und 1984 immer wieder und notierte meine
Beobachtungen. So entstand allmählich eine Liste mit kritischen Notizen. Jeder
einzelne Satz von Altschuller erscheint einleuchtend, doch manche Sätze
widersprechen sich, auch 1984. Das merkt man, wenn man nachfolgende Sätze
liest. Manche Sätze enthalten Versprechungen, diese werden wiederholt, jedoch
nicht immer eingelöst. Auch das bemerkt man, wenn man
weiterliest. Altschullers Texte hätten viel kürzer ausfallen können. Doch das
tut meiner Hochachtung vor Altschuller keinen Abbruch. Sein Projekt war
neuartig, revolutionär, in sich hoch komplex, schwer überschaubar. Da wird er
wohl unterschwellig gehofft haben: „Vielleicht fällt mir beim Schreiben noch
etwas Besseres ein.“ So erkläre ich mir den großen Umfang seiner Äußerung. Wir
in Berlin begannen viele Jahre später als Altschuller und hatten es viel
einfacher, uns viel kürzer auszudrücken.

Vor allem beim Verdienten Erfinder Dr. Ing. Rindfleisch ging das sehr
schnell. Er begann gleich mit der Frage: Wie entdeckt man eine treffende
Problemstellung?  Und noch etwas: Er war ja nicht nur Erfinder, er war auch
durch die strenge Schule der theoretischen Elektro-Technik gegangen. Und so
vollzog sich Dialektik: Hans-Jochen entwickelte sehr schnell das ProHEAL,
und ich half mit. Dieses Programm ruht nicht nur auf mancherlei Kritik an
Publikationen Altschullers, vor allem setzt es neue Akzente. Das möchte ich
nunmehr andeuten, bevor ich das ProHEAL als solches skizziere:

4.1 ProHEAL wendet sich nicht nur an erfindungswillige Ingenieure. Vielmehr
setzt es an den Anfang die intellektuelle und moralische Auflassung gegenüber
allen Technikern, ihre Tätigkeit an den Bedürfnissen der Gesellschaft zu
orientieren und nicht primär an den Aufträgen des Chefs. Dazu sind alle
Techniker verpflichtet und nicht nur die Liebhaber von Erfindungen. Deshalb
beginnt ProHEAL kurz und bündig mit den Worten: \emph{„Worin besteht das
  gesellschaftliche Bedürfnis?“} Das soziale, ökologische, wirtschaftliche
Bedürfnis? Man wird ja nicht gleich gefeuert, wenn man den Chef daraufhin
anspricht. Deshalb werden in ProHEAL anstelle des Begriffs „Auftrag“ die
Begriffe „Problem“ und „Problemlösung“ verwendet.

4.2 ProHEAL vertraut darauf, dass Techniker gut ausgebildet sind und sich in
Physik, in Chemie und Bionik zurechtfinden können. ProHEAL vertraut auch
darauf, dass Techniker sich Einblick verschaffen können in soziale,
ökonomische und ökologische Zusammenhänge: Entweder sie misstrauen gängigen
Medien, oder sie verschaffen sich Einblick mittels \emph{kritischer} Medien.

4.3 ProHEAL setzt darauf, dass Techniker rationale Anforderungen und Erwar-
tungen respektieren und dass sie gegebene Bedingungen und Restriktionen nicht
fürchten. Schon 1973 Seite 89 hatte Altschuller geschrieben, ich zitiere: „In
der Praxis der Erfinder besteht die Hauptsache oft darin, den technischen
Widerspruch aufzudecken. Ist er erst aufgedeckt, ist es nicht mehr schwer, ihn
zu überwinden. Oft ist auch die präzise Formulierung des Widerspruchs ent-
scheidend. Der Erfinder muß genau bestimmen, was \emph{das nicht zu
  Vereinbarende} und was \emph{das zu Vereinbarende} ist. Hier ist es wichtig,
psychische Trägheit zu überwinden.“ Ende des Zitats. Gerade darauf zielt das
ProHEAL, ohne viele Worte.

4.4 Problemlösungen sind Innovationen. Doch Innovationen, die nur an Markt-
gesetzen und Patentrecht orientiert sind, sind nicht generell Problemlösungen.
Was patentiert wird, muss international eine Neuheit sein. So war es auch in
der DDR. Doch die Patent-Amtler in der DDR wollten auch Problemlösungen sehen.
Deshalb war es in der DDR zunehmend schwerer geworden als in der BRD, ein
Patent zu erlangen. Patent-Einreicher sagten oft: Wenn wir aus Berlin kein
Patent bekommen, melden wir in München an. Altschuller aber hatte seine Prin-
zipe zur Problemlösung überwiegend aus Patentrecherchen extrahiert. ProHEAL
dagegen respektiert den Unterschied zwischen Innovation und Problemlösung von
vornherein. Altschullers WEPOL-Darstellung ist bedeutsamer als die Liste der
40 Prinzipe, weil sie zur Ausnutzung verfügbarer Wechselwirkungen anregt.
Leider ist 1984 bei Altschuller die Verwendung dieser beiden Paradigmen – die
vierzig Prinzipe einerseits und WEPOL andererseits – nicht aufeinander abge-
stimmt. Da geht es streckenweise durcheinander. Vorteil: Da müssen die Leser
grübeln.

4.5 Die meisten der 40 Prinzipe scheinen mir nicht mehr zu sein als Erinnerun-
gen an Kniffs, die jedem Facharbeiter auch ohne Literatur in den Sinn kommen.
Hoch interessant ist, wie Dietmar Zobel in seinem Buch „Systematisches
Erfinden“ (5. Auflage 2009) den meisten der 40 Prinzipe eine Funktion zuweist:
Weniger zum Erfinden geeignet, sondern zur fachgerechten Ausführung techni-
scher Leistungen. Von großem erfindungsrelevanten Kaliber scheint mir nur das
Prinzip „Keil durch Keil“ (1973 S. 304), besser noch „Kompensation“ (S. 28,
29, 32, 50) und 1984 „Umwandlung von Schädlichem in Nützliches“
(S.139). Symbol ist das Pendel-Problem nach Carl Duncker (1935). Doch bei
Altschuller fehlt das Symbol. Ich habe getestet: Von 250 Ingenieuren fand nur
ein einziger die Lösung von selber. Doch von Altschuller wird der Psychologe
Carl Duncker gescholten. Schade.

4.6 Altschullers Riesentabelle mit den vierzig Lösungsprinzipen ist gewiss ein
Mutmacher. Doch Lösungen aus der Tabelle ablesen? Ich betone „ablesen“. Wer
nur abliest, hat noch nicht die \emph{Entstehung} von Widersprüchen im Blick:
\emph{Warum} sind Widersprüche entstanden? Danach zu fragen empfiehlt auch
Dietmar Zobel. Wer nur abliest, ist noch nicht im Bilde. Wer nur abliest, ist
auch noch nicht in der Therapie, wo Geist und Seele massiert werden, um
kreativ zu werden.

4.7 WEPOL – also Stoff-Feld-Analyse – regt schon eher zum Denken an. Die
WEPOL-Graphik ist ein Medium zum Denken wie unsre Wortsprache. Doch WEPOL
schränkt auch das Blickfeld ein. Es gibt ja Beziehungen auch zwischen Körpern
und Beziehungen zwischen Feldern. Darüber täuscht WEPOL hinweg.

4.8 ProHEAL bietet ein matrix-förmiges Format mit 16 Tabellenfeldern, in das
sich Parameter gemäß 4.2 und 4.3 leicht eintragen lassen. ProHEAL fordert dazu
auf, die Werte dieser Parameter (bzw. deren Kehrwerte) in dieser Matrix bis
zur Entstehung von Widersprüchen in die Höhe zu treiben, also Widersprüche
auf Basis 4.1, 4.2 und 4.3 herauszufordern, zu sollizitieren.

4.9 So hat es auch Hansjürgen Linde gesehen. 1990 begann er, in renommierten
Industrie-Unternehmen ganz Deutschlands seine Workshops zu praktizieren unterm
Label „WOIS“.

4.10 Eine Frage: Ist jetzt nicht auch der Widerspruch entstanden zwischen der
Reife literarisch explorierter Methodologie des Erfindens und dem Umfang der
Literatur, der von den Technikern bewältigt werden müsste? Den Technikern im
Beruf und im Studium?

5.1 Nun werfen wir einen Blick auf das Kernstück von ProHEAL, ein
matrix-förmiges Gebilde mit 16 Feldern. (In den Matrix-Feldern waren 1988
beispielhaft Eintragungen für Kraftfahrzeuge vorgenommen worden).

\begin{figure}[h]
\begin{center}
  \providecommand{\abox}[1]{\parbox{2.5cm}{\footnotesize\raggedright #1}}
  \begin{tabular}{|c|c|c|c|c|}\hline
    & \abox{Zweckmäßigkeit} & \abox{Wirtschaftlichkeit}
    & \abox{Beherrschbarkeit} & \abox{Brauchbarkeit}\\\hline
    \textbf{A}nforderungen
    & \abox{Leistungsfähigkeit und Fahrtüchtigkeit bis Fahrgeschwindigkeit von
      $x$ km/h}
    & \abox{1. Kraftstoffsparend\par 2. Abgaswärme nutzend}
    & \abox{1. Leicht bedienbar, Verschließteile leicht zugänglich\par
      2. Ersatzteile an Bord verfügbar (mitführbar)}
    & \abox{\vspace*{1em} 1. Anpassbar an örtlich gegeben
      Verkerkehrsbedingungen\par 2. Verwendbar als Zugmaschine, Lieferwagen und
      Reisewagen\vspace*{1em} } \\\hline
    \textbf{B}edingungen
    & \abox{1. Verkehrstauglich\par 2. Zugbetriebstauglich}
    & \abox{1. Servicefreundlich\par 2. Lastentransportdienlich}
    & \abox{\vspace*{1em} 1. Kurzzeitig auf $x$-fache Normallast überlastbar\par
      2. Fahrverhalten (unverzögert), Lenkung folgend\vspace*{1em} }
    & \abox{1. Steinschlagabhaltend\par 2. Hitzeabwendend\par
      3. Temperaturhaltend\par 4. Feuchteausgleichend} \\\hline
    \textbf{E}rwartungen
    & \abox{1. Hohes Beschleunigungsvermögen\par 2. Verzögerungsfreie
      Beschleunigung}
    & \abox{1. Transportergiebigkeit\par 2. Preisgünstig}
    & \abox{\vspace*{1em} 1. Schleuderbewegungen selbsttätig ausgleichend\par
      2. Auf rasch veränderliche Fahrbahnbedingungen selbst einstellend\par
      3. Selbst überwachend\vspace*{1em} }
    & \abox{1. Unabhängig von Tankstellen\par 2. Unempfindlich gegen tiefe
      Temperaturen (z.B. beim Starten)} \\\hline
    \textbf{R}estriktionen
    & \abox{1. Antriebs- und Brems-System spurgetreu\par 2. Verkehrsregelgemäße
      Licht- und Signalanlage}
    & \abox{1. Anspruchslos in bezug auf Instandhaltung\par2. Genügsam in bezug
      auf Kraftstoffqualität }
    & \abox{1. Verkehrsicher\par 2. Rüttelfest\par 3. Stoß- und schlagfest\par
      4. diebstahlsicher}
    & \abox{\vspace*{1em} 1. Verträglich mit Abgasnorm\par
      2. Korrosionsbeständig bei Tausalz-Einwirkung\par 3. Unbedenklich für
      innerstädtischen Verkehr\vspace*{1em} }\\\hline
  \end{tabular}\par \vskip1em 
  Zielgrößen und Komponenten. Die ABER-Matrix von Hans-Jochen Rindfleisch und
  Rainer Thiel.  Matrix-Felder sind hier ausgefüllt mit Beispielen aus 1988.
\end{center}
\end{figure}

Dabei kann man einen Spaß und zwei Beinahe-Redundanzen bemerken. Begonnen
hatten wir, dem \emph{gesellschaftlichen Bedürfnis} entsprechend die Anforde-
rungen, die Bedingungen und die Restriktionen zu notieren, also \emph{die A,
  die B und die R}. Da meinte Hans-Jochen: Nehmen wir doch gleich noch die
\emph{Erwartungen} mit dem Anfangsbuchstaben E hinzu, das duftet zwar nach
Redundanz, denn Anforderungen, Bedingungen und Restriktionen haben wir schon
im Blick, doch wir haben statt ABR jetzt ABER. Das ist nicht nur ein schönes
Logo aus vier Buchstaben, das ist auch ein Alarm-Signal beim Brainstorming,
dem ein inverses Brainstorming folgen muss. Da spielt in deutscher Sprache die
Kopula „aber“, also der kritische Einwand, eine produktive Rolle. Eine weitere
Beinahe-Redundanz erlaubten wir uns, indem wir außer den ABER auch noch vier
Zielgrößen-Komponenten ins Blickfeld zogen. So entstand aus der eindimen-
sionalen ABER Liste eine zweidimensionale Matrix mit insgesamt 4 hoch 2 gleich
16 Feldern. Und Redundanz wegen der Begriffsverwandtschaft der Zeilen- und der
Spalten-Eingänge? Das ist ein Glücksfall. Denn jetzt entsteht für den
Techniker ein (4-hoch-2)-Generator, das Denken anzukurbeln. Das ist Wind, um
Widersprüche herauszuarbeiten. Wir brauchten nur noch zu sagen: „Und jetzt,
liebe Kolleginnen und Kollegen, treiben wir die üblichen Parameterwerte in
die Höhe.“ In unsren Erfinderschulen hat das funktioniert. Die Techniker
waren aufs höchste angeregt, bald auch aufgeregt, und bald kamen laute
Zwischenrufe: Verdutzte Techniker riefen: „Da kommen wir doch in Wider-
sprüche!“ Ja, genau das wollen wir, das hatte auch Altschuller gewollt, ich
hatte es schon zitiert. Und ich konnte den Technikern sagen: „Da sind Sie von
ihren Professoren getäuscht worden, von wegen in einer Ingenieur-Aufgabe dürfe
nie ein Widerspruch auftreten.“ Wir aber haben gelernt von Hegel und von Marx.
Dort lernten wir auch, dass zur Dialektik gehört: Bäume wachsen nicht bis zum
Himmel. Wer darauf hinwirkt, provoziert dialektische Widersprüche. Auch diesen
Gedanken von Hegel und Marx fand ich ein Mal, aber nur ein Mal bei
Altschuller.

Die ABER-Matrix ist (mit redaktionellen Wandlungen) auch von Hansjürgen Linde
in seiner Dissertation TUD 1988 und 1993 in der Druckausgabe der WOIS vielfach
verwendet worden, ebenso Schreibweisen in Textfassungen, 1980 von mir
eingeführt, um Texte Altschullers komprimieren zu können. Linde und Thiel
waren in herzlicher Freundschaft verbunden. Eines Tages rief er mich an:
„Morgen muss ich ins Krankenhaus.“ Drei Wochen später empfing ich eine
Nachricht aus seinem Institut: Sein reiches Leben hatte sich vollendet.

Nun kommen nur noch drei Positionen mit wenigen Worten, also:

5.2 ProHEAL vertraut darauf, dass Techniker willens und fähig sind, ihr Fach-
wissen gründlich zu nutzen, notfalls auch zu ergänzen, um vor dialektischen
Widersprüchen nicht zurückzuschrecken, also kreative Wege einzuschlagen.

5.3 ProHEAL bietet auf wenigen Druckseiten einen Algorithmus-analogen Leit-
faden, zunächst leichtlösbare Widersprüche zu identifizieren und schnell zu
lösen oder im Bedarfsfall tiefergehende technisch-technologische und noch
tiefer gehende technisch-naturgesetzliche Widersprüche genau und somit an-
greifbar zu formulieren. Leicht lösbare Widersprüche fanden wir schnell und
sehr oft.

5.4 ProHEAL ist auf wenigen Druckseiten nachlesbar, an denen sich Techniker
hinreichend orientieren können. Zusätzlich werden Erläuterungen angeboten, die
sich wahlweise wahrnehmen lassen, sie vergrößern die Lust zum Problem- Lösen.

\section*{TRIZ, ProHEAL und unsere Zukunft}

Anmerkungen von Rainer Thiel in der Abschlusssitzung der Konferenz.

Gestatten Sie bitte zwei ganz kurze Anmerkungen zu den täglich sichtbaren
Leuchtschriften „Innovation“ und „Wachstum“:

\paragraph{1. Innovationen, unser Zeitalter.}
Da ist ja etwas dran. Aber nur etwas. Ich zitiere aus einem Buch, das gerade
in Frankfurt und New York erschienen ist und sich auf umfangreiche Literatur
stützt, u.a. auf eine Fraunhofer-Studie. In diesem Buch heißt es – ich
zitiere: „... dass ein immer größerer Teil der Patentanmeldungen nicht mehr
dadurch motiviert ist, eigene Innovation vor Imitation zu
schützen. ... Stattdessen dominiere das Ziel, die Anwendung bestimmter
Technologien durch Konkurrenten zu blockieren ... Oder es werden Verfahren
patentiert, denen überhaupt keine Innovation zugrunde liege. Immer öfter
würden Patente nicht <deshalb> angemeldet, um sie zu nutzen, sondern um die
Nutzung einer den eigenen Produkten gefährlichen Innovation zu verhindern.“
Und noch ein Wort zum Worte „Innovation“: Einer der kreativsten Menschen aller
Zeiten, Albert Einstein, ein Humanist, den Kommunisten zugetan, forderte den
amerikanischen Präsidenten auf, die Entwicklung der Atombombe administrativ
einzuleiten, um damit Hitler zuvorzukommen. Doch die cleveren US-
Geheimdienste hatten gar nicht bemerkt, dass Hitler noch vor seiner Atom-
bombe besiegt werden konnte. Also begann in Los Alamos die Entwicklung von
Atombomben. Als der Krieg schon entschieden war, wurde aus machtpolitischen
Gründen auf Hiroshima und Nagasaki je eine Bombe geworfen.

2. Meine 2. Anmerkung, nun zu den Leuchtbuchstaben „Wachstum“: Forciert wird
wirtschaftliches Wachstum, das die Bewohnbarkeit unsrer kosmischen Heimat,
unsrer Erde, untergräbt. In nördlichen Industrieländern wird Menge und
Vielfalt von Konsumgütern und Waffen hemmungslos vergrößert. Schon im
19. Jahrhundert begannen Philosophen und Dichter davor zu warnen: Rousseau,
Jean Paul, Karl Marx. Der Dichter Jean Paul erzählt, wie er sich an einen
Freund wandte: Kannst Du denn nicht sehen, „dass die Menschen toll sind und
schon Kaffee, Tee und Schokolade aus besonderen Tassen, Früchte, Salate und
Heringe aus eigenen Tellern, und Hasen, Früchte und Vögel aus eigenen
Schüsseln verspeisen. – Sie werden aber künftig, sag‘ ich Dir, noch toller
werden und in den Fabriken so viele Fruchtschalen herstellen, als in den
Gärten Obstarten abfallen..., und wär‘ ich nur Kronprinz oder Hochmeister, ich
müsste Lerchenschüsseln und Lerchenmesser, Schnepfenschüsseln und
Schnepfenmesser haben, ja eine Hirschkeule von einem Sechsender würd‘ ich auf
keinem Teller anschneiden, auf dem ich einen Achtender gehabt hätte.“ Ende des
Zitats. Ich füge hinzu: So leben wir. Die Schränke voll und voller. Dicht und
dichter gedrängt verdecken Sachen die Sicht auf Sachen, die schon da sind:
Verdeckt, vermisst und abermals gekauft. Man tröstet sich, das Neue sei
moderner... Bis schließlich nur noch Röcheln ist: Wir können nicht
anders. Fahren wir zum Kaufhaus.“ (Auszüge aus R. Thiel: „Marx und
Moritz. Unbekannter Marx. Quer zum Ismus“. Trafo Verlag Berlin 1998.) Dort
auch Marx-Zitat aus MEW 25, Seite 784: „Selbst eine ganze Gesellschaft, eine
Nation, ja alle gleichzeitigen Gesellschaften zusammengenommen, sind nicht
Eigentümer der Erde. Sie sind nur ihre Besitzer, ihre Nutznießer, und haben
sie als boni patres familias den nachfolgenden Generationen verbessert zu
hinterlassen.“

Ist das nicht unsre Wirtschaft seit Jahrzehnten? Nichts gegen Märkte, wir
brauchen sie. Sie werden durch mittelständische, genossenschaftliche,
gemeinnützige Unternehmen belebt. [Sahra Wagenknecht 2016, Hans Küng 2010.]
Doch das Gerüst unbegrenzter Marktwirtschaft strebt Richtung Himmel, und das
hat längst neue Widersprüche hochgepuscht. Beunruhigt sind Mitbürger
christlichen Glaubens, Naturfreunde, Nichtregierungsorganisationen NGO und
einige Linke. Bei ATTAC gibt es eine Arbeitsgruppe „Transformation statt
Wachstum“. Ich war Mitbegründer. Doch Techniker sind kaum dabei.

Was machen wir nun mit den extensiven Texten zu TRIZ? Altschuller hatte in
einem Land gewirkt, in dem noch vieles fehlte, was uns im Westen längst
Gewohnheit war. Auch in Asien und Afrika fehlt es an vielem. Muss aber in
Entwicklungsländern alles wie in nördlichen Industrie-Ländern geraten?
Deshalb ist „Transformation statt Wachstum“ eine Kiste mit vielen Problemen,
mit Widersprüchen, vor denen wir alle stehen. Wir müssen sie erkunden. Mit
ProHEAL und seiner ABER-Matrix sind die Probleme ganz direkt ansprechbar.

Wenn wir Freunde von TRIZ sein wollen, müssen wir auch diese Widersprüche
erkunden. TRIZ darf nicht missbraucht werden. Wir wollen keine Sklaven des
großen Kapitals sein. Wir sollten überlegen: Wie muss TRIZ genutzt werden, um
unsre kosmischen Heimat zu sichern? TRIZ im Gepäck, und ProHEAL wird sein.  Es
lebe das Brot, und es lebe der Wein.

\ccnotice
\end{document}
